%*******************************************************************************
%*********************************** First Chapter *****************************
%*******************************************************************************

\chapter{Introduction}
\label{chap:introduction}

Vision is one of the most important senses that humans possess. It is an amazing process of how the eyes and brain translate light waves into interpreted images of our surroundings. Reflected light waves from objects are bent, or refracted, when passing through the cornea, then bent again passing through the lens and eventually hits the retina. The image is translated into impulses that travels to the brain, more specifically the occipital lobe, through the optic nerve such that the image can be interpreted. The shape of eyes affect how we things in the world are seen and kept in focus. For people with normal vision, the light waves hits the retina at the focal point where the light waves coincide. However, the eyes are longer for nearsighted eyes which moves the focal point closer to the lens such that the light waves are more spread out across the retina. For farsighted people, the eyes are shorter which moves the focal point behind the retina. Luckily, the position of the focal point can be corrected for both near- and farsighted eyes by placing a concave and convex lens respectively in front of the eyes, e.g., by using eyeglasses. However, there exist many severe cases of low vision where not even eyeglasses can help where other kinds of help is necessary for assisting the people in situations where vision is a necessity. 





\MK{ 
\begin{itemize}
	\item the connection between all papers is image classification
	\item talk about how complex the visual system is and how computer vision is used to replicate this
	\item talk about the challenges that can arise without having visual capabilities so that using computer vision would help, for example finding lost items at home
\end{itemize}
}


\section{Vision Impairments}
In 2020, there were almost 600 million people with mild to severe visual impairments around the world~\cite{burton2021lancet}.


\section{Object Recognition for Assistive Vision}


% Thesis contributions

\section{Thesis Contributions}
\label{sec:contributions}

% Thesis outline

\section{Thesis Outline}
\label{sec:outline}


\begin{comment}
Here is an example for referencing figure \ref{EnergySources}. Example of citing \cite{BP2019} and \cite{Chen2016}.
%%%%%%%%%%%%%% Figure: Energy sources
\begin{figure}[h]
	\centering
	\includegraphics[scale=0.6]{Figs/Ch1_EnergySources.png}
	\caption{The world's energy consumption by fuel in 2017. }
	\label{EnergySources}
\end{figure}

%%%%%%%%%%%%%%%%%%%%%%%%%%%%%%%%%%%%%%%%
\section{Section}
Example of a table
%%%%%%%%%%%%%%%%%%%%%%%%%%%%%%%% Table: Tokamaks
\begin{table}[t!]
	\centering
	\caption{List of experimental tokamaks worldwide. Note: ITER is currently under construction and the first plasma is predicted for 2025-2028.}
	\begin{tabular}{ccccc}
		\hline
		\textbf{Name} & \textbf{Location} & \textbf{B-field} & \textbf{Major/minor radius}  \\
		\hline
		JET     & England      & 4.0 T & 3.0 m / 1.3 m \\
		ITER    & France       & 5.3 T & 6.2 m / 2.0 m \\
		AUG		& Germany      & 3.1 T & 1.7 m / 0.7 m \\
		WEST	& France	   & 3.7 T & 2.5 m / 0.5 m \\
		TCV     & Switzerland  & 1.5 T & 0.9 m / 0.3 m \\
		DIII-D  & USA          & 2.2 T & 1.7 m / 0.7 m \\
		TFTR 	& USA          & 6.0 T & 2.5 m / 0.9 m \\
		JT-60   & Japan        & 4.0 T & 3.4 m / 1.0 m \\
		K-STAR  & South Korea  & 3.5 T & 1.8 m / 0.5 m \\
		EAST    & China        & 3.5 T & 1.9 m / 0.5 m \\
		\hline
	\end{tabular}
	\label{TokamakTable}
\end{table}
\end{comment}
