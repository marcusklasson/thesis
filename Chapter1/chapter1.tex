%*******************************************************************************
%*********************************** First Chapter *****************************
%*******************************************************************************

\chapter{Introduction}
\label{chap:introduction}

Vision is one of the most important senses that humans possess. It is an amazing process of how the eyes and brain translate light waves into interpreted images of our surroundings. Reflected light waves from objects are bent, or refracted, when passing through the cornea, then bent again passing through the lens and eventually hits the retina. The image is translated into impulses that travels to the brain, more specifically the occipital lobe, through the optic nerve such that the image can be interpreted. The shape of eyes affect how we things in the world are seen and kept in focus. For people with normal vision, the light waves hits the retina at the focal point where the light waves coincide. However, the eyes are longer for nearsighted eyes which moves the focal point closer to the lens such that the light waves are more spread out across the retina. For farsighted people, the eyes are shorter which moves the focal point behind the retina. Luckily, the position of the focal point can be corrected for both near- and farsighted eyes by placing a concave and convex lens respectively in front of the eyes, e.g., by using eyeglasses. However, there exist many severe cases of low vision where not even eyeglasses can help where other kinds of help is necessary for assisting the people in situations where vision is a necessity. 

There exist many different types of aids and assistive tools for helping visually impaired people (VIP) in their daily lives. A very common aid is the white cane used for enhancing the mobility of VIPs by extending their touch to prepare the users for what is ahead of them. There also exists many electronic devices, e.g., screen readers and Braille typewriter machines, that have enabled VIPs to have near-equal opportunities for office work. More recently, several computer vision-based assistive vision tools have emerged in the form of wearable devices and mobile applications for helping with, e.g., reading printed documents and bar codes for item recognition. However, computer vision-based systems can face several challenges when deployed in the real world which makes their recognition performance suffer. Therefore, there is a necessity for developing computer vision methods that perform various kinds of recognition tasks in a robust and time-efficient manner. 

Despite the immense successes in computer vision in recent years, why is computer vision difficult to perform in real-world settings? One part of the reason is that it is very challenging to specify a model of the visual world that has been injected with knowledge about the rich complexity that can exist in images~\cite{szeliski2010computer}. Interestingly, tasks that are seemingly simple for humans, such as the differences between apples and pears, can be difficult for computers. A popular approach for enabling computers to learn various concepts is machine learning where the computer learns a model of some phenomena from large sets of examples. The approach of learning from data and experiences has been shown across various tasks~\cite{akkaya2019solving, brown2020language, silver2016mastering} other than computer vision to be more efficient than relying on hard-coded knowledge for solving decision-making problems. However, collecting large sets of data is often time-consuming and labor-intensive which makes it challenging to deploy machine learning systems in the real-world where data is even more expensive to collect. Without overcoming the challenge of data collection, we need to develop data-efficient methods that can learn from few examples and generalize to new scenarios for enabling computer vision-based assistive vision devices to be useful for VIPs.   



\MK{ 
\begin{itemize}
	\item the connection between all papers is image classification
	\item talk about how complex the visual system is and how computer vision is used to replicate this
	\item talk about the challenges that can arise without having visual capabilities so that using computer vision would help, for example finding lost items at home
\end{itemize}
}


\section{Vision Impairments}
There are many different degrees of vision impairments ranging from problems with seeing from near or farther distances to blindness. Vision impairments are generally assessed by measuring the visual acuity from a fixed distance~\cite{who2019world}. In 2020, the World Health Organization (WHO) estimated that at least 2.2 billion people live with a near or distance vision impairment, wherein at least 1 billion cases the impairments could have been prevented or yet has to be addressed~\cite{who2019world}. The untreated cases are projected to grow to 1.7 billion people by 2050 mainly due to population growth and aging~\cite{bourne2021trends}. The leading causes for vision loss are uncorrected refractive errors (161 million people with distance vision loss and 510 million people with near vision loss), untreated cataracts (100 million people), age-related macular degenerationm (8.1 million), glaucoma (7.8 million), diabetic retinopathy (4.4 million) where 90\% of vision losses are preventable and treatable~\cite{steinmetz2021causes}. Furthermore, the prevalence of distance vision impairment are estimated to be four times higher in low- and middle-income regions than in high-income regions~\cite{steinmetz2021causes}.  

Several kinds of aids and tools have been developed for VIs to facilitate their capabilities of performing everyday tasks. The so-called \textit{white cane} is by far the most common mobility tool aid which is used for navigation and helping the VI to anticipate what is in front of them while walking. Dog guides are also used for enhancing mobility by helping the VI to maintain a direct route, avoid obstacles, and stops at curbs and stairways until they are told to proceed~\cite{manduchi2012computer}. Technical devices such as screen readers and Braille keyboards have enabled VIs to have nearly equal opportunities when it comes to office-related tasks. Furthermore, eye-service clinics offer counseling and home-skills training for helping VIs with modifying their home to ensure a safe and accessible home. 

\MK{I should mention difference between low-vision and blindness somewhere here too.}


%In Sweden, there are over 100.000 people with 




%In 2020, there were almost 600 million people with mild to severe visual impairments around the world~\cite{bourne2021trends}. 

%Describe common causes for visual impairment. What aids are there for helping them out, which will lead to assistive vision devices in the next section. 

%In a recent study made in Sweden with VIs, the challenge that most VIs are concerned about in general is mobility~\cite{stahl2018levnadsundersokning}. 

%VIPs that had to change their careers due to inaccessible apps and systems at their workplace~\cite{gotesson2019challenges, gotesson2019utmaningar}. Conclusion was that VIPs should be involved in the development and design process of such products, and that this should be included when educating User Experience (UX) designers. 


\section{Computer Vision-Based Assistive Technologies} %\section{Object Recognition for Assistive Vision}

It has been shown that VIs want help with object recognition and identifying personal items~\cite{brady2013visual}. 
The recent advances in computer vision the past two decades have led to assistive vision technologies emerging on the market. There exist support from previous studies that blind people want to record experiences with photographs just like sighted people~\cite{jayant2011supporting}. This can further assist VIs with identifying objects and receive descriptions of their environment through computer vision. To this end, mobile applications (such as SeeingAI~\cite{microsoft2017seeing}, TapTapSee, and iDentifi)  exists for helping visually impaired with visual tasks, such as reading documents and hand-written texts, recognizing various objects, and face recognition, as well as wearable devices (such as Orcam MyEye) with similar capabilities but at a higher price. An alternative to these computer-vision based apps there are other mobile applications called Be My Eyes and Aira~\cite{aira2017aira} where VIPs can call sighted volunteers to help them describe their surroundings from the camera of the user. Furthermore, there exists other smart devices for helping VIs with enhancing their mobility, for example, smart canes (see \cite{manduchi2012computer} for highlights the current state of affairs, challenges, and potential outcomes of electronic devices for assistive vision). 

Even if the computer vision-based assistive devices have potential in enhancing mobility for VIs, there are several challenges these devices have to tackle when used in the real-world. The major challenge is to provide good training data for teaching the model to recognize items in various environments. Image datasets used for pre-training image classification models, such as Imagenet~\cite{deng2009imagenet}, mainly constitutes of web images which can lack of training images from user-centric views. Web images might be enough for recognizing common objects, such as cups and bananas, and well-known commercial products, such as coca-cola cans. However, it might be very challenging for the device to recognize personal items if it is excluded from the training data, which have made researchers focus on developing personal object recognizers in assisitive vision apps~\cite{ahmetovic2020recog, lee2019revisiting}. Furthermore, VIs use and hold the camera of mobile phones differently than sighted users which can create a gap between the training and test data during deployment~\cite{kacorri2017people, vazquez2012helping}. These insghts have been employed when collecting the ORBIT dataset~\cite{massiceti2021orbit} where the focus was to collect an object recognition dataset for training teachable object recognizers from a disability-first procedure where blind/low-vision participants collected all images\cite{theodorou2021disability}. However, the performance of the device also relies a lot on the implemented recognition model, which we will discuss next.

Computer vision models in assistive devices should be required to be data-efficient when learning about objects to recognize, be robust in new and noisy environments, and be capable of performing in real-time. Models deployed in the real-world often has to learn from scarce data and adapt fast to new environments, for example via \textit{transfer learning}~\cite{sharif2014cnn, zhuang2020comprehensive} or \textit{few-shot learning}~\cite{wang2020generalizing}. Another approach can be to use combinations of different data types and modalities, for example, text or audio, with ideas from \textit{multimodal machine learning}~\cite{baltruvsaitis2018multimodal} and \textit{multi-view learning}~\cite{xu2013survey} for learning more discriminative image representations that enhances the predictive performance of the model. Furthermore, the model should be capable of learning how to recognize new objects and concepts about the world which is a field coined \textit{continual learning} (CL)~\cite{delange2021continual, parisi2019continual}. The main challenge to tackle in CL is \textit{catastrophic forgetting}~\cite{mccloskey1989catastrophic} where already existing knowledge in the model gets overwritten with the information about the new task to learn. However, such approaches opens up for having computer vision models that can learn to act in new environments during its lifetime. Finally, the model has to be deployed on edge devices, such as mobile phones and applications, that should act in the real-world and perform fast in real-time. The main focus to achieve real-time inference on phones has been put on reducing computational cost by compressing pre-trained vision models~\cite{han2015deep}, designing lightweight network architectures~\cite{howard2017mobilenets, zhang2018shufflenet}, and hardware accelerations~\cite{huynh2017deepmon}. Recently, there have been several attempts to apply computer vision models on edge devices in the CL setting\cite{ li2019rilod, pellegrini2019latent}.



\section{Scope of Thesis}

We define the scope of this thesis where we focus on certain avenues for developing robust computer vision models for assisting VIs. 

%Discuss how thesis is narrowed down and focused on image classification and like data collection, how to use additional labels, and then how to learn new labels continuously. 


\cite{lanigan2006trinetra} short paper on various assistive devices for recognizing groceries, so this has been an important task for quite some time. 


% Thesis contributions

\section{Thesis Contributions}
\label{sec:contributions}

% Thesis outline

\section{Thesis Outline}
\label{sec:outline}


\begin{comment}
Here is an example for referencing figure \ref{EnergySources}. Example of citing \cite{BP2019} and \cite{Chen2016}.
%%%%%%%%%%%%%% Figure: Energy sources
\begin{figure}[h]
	\centering
	\includegraphics[scale=0.6]{Figs/Ch1_EnergySources.png}
	\caption{The world's energy consumption by fuel in 2017. }
	\label{EnergySources}
\end{figure}

%%%%%%%%%%%%%%%%%%%%%%%%%%%%%%%%%%%%%%%%
\section{Section}
Example of a table
%%%%%%%%%%%%%%%%%%%%%%%%%%%%%%%% Table: Tokamaks
\begin{table}[t!]
	\centering
	\caption{List of experimental tokamaks worldwide. Note: ITER is currently under construction and the first plasma is predicted for 2025-2028.}
	\begin{tabular}{ccccc}
		\hline
		\textbf{Name} & \textbf{Location} & \textbf{B-field} & \textbf{Major/minor radius}  \\
		\hline
		JET     & England      & 4.0 T & 3.0 m / 1.3 m \\
		ITER    & France       & 5.3 T & 6.2 m / 2.0 m \\
		AUG		& Germany      & 3.1 T & 1.7 m / 0.7 m \\
		WEST	& France	   & 3.7 T & 2.5 m / 0.5 m \\
		TCV     & Switzerland  & 1.5 T & 0.9 m / 0.3 m \\
		DIII-D  & USA          & 2.2 T & 1.7 m / 0.7 m \\
		TFTR 	& USA          & 6.0 T & 2.5 m / 0.9 m \\
		JT-60   & Japan        & 4.0 T & 3.4 m / 1.0 m \\
		K-STAR  & South Korea  & 3.5 T & 1.8 m / 0.5 m \\
		EAST    & China        & 3.5 T & 1.9 m / 0.5 m \\
		\hline
	\end{tabular}
	\label{TokamakTable}
\end{table}
\end{comment}
