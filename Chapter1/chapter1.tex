%*******************************************************************************
%*********************************** First Chapter *****************************
%*******************************************************************************

\chapter{Introduction}
\label{chap:introduction}

Vision is one of the most important senses that humans possess. It is an amazing process of how the eyes and brain translate light waves into interpreted images of our surroundings. Reflected light waves from objects are bent, or refracted, when passing through the cornea, then bent again passing through the lens and eventually hits the retina. The image is translated into impulses that travels to the brain, more specifically the occipital lobe, through the optic nerve such that the image can be interpreted. The shape of eyes affect how we things in the world are seen and kept in focus. For people with normal vision, the light waves hits the retina at the focal point where the light waves coincide. However, the eyes are longer for nearsighted eyes which moves the focal point closer to the lens such that the light waves are more spread out across the retina. For farsighted people, the eyes are shorter which moves the focal point behind the retina. Luckily, the position of the focal point can be corrected for both near- and farsighted eyes by placing a concave and convex lens respectively in front of the eyes, e.g., by using eyeglasses. However, there exist many severe cases of low vision where not even eyeglasses can help where other kinds of help is necessary for assisting the people in situations where vision is a necessity. 

There exist many different types of aids and assistive tools for helping visually impaired people (VIP) in their daily lives. A very common aid is the white cane used for enhancing the mobility of VIPs by extending their touch to prepare the users for what is ahead of them. There also exists many electronic devices, e.g., screen readers and Braille typewriter machines, that have enabled VIPs to have near-equal opportunities for office work. More recently, several computer vision-based assistive vision tools have emerged in the form of wearable devices and mobile applications for helping with, e.g., reading printed documents and bar codes for item recognition. However, computer vision-based systems can face several challenges when deployed in the real world which makes their recognition performance suffer. Therefore, there is a necessity for developing computer vision methods that perform various kinds of recognition tasks in a robust and time-efficient manner. 

Despite the immense successes in computer vision in recent years, why is computer vision difficult to perform in real-world settings? One part of the reason is that it is very challenging to specify a model of the visual world that has been injected with knowledge about the rich complexity that can exist in images~\cite{szeliski2010computer}. Interestingly, tasks that are seemingly simple for humans, such as the differences between apples and pears, can be difficult for computers. A popular approach for enabling computers to learn various concepts is machine learning where the computer learns a model of some phenomena from large sets of examples. The approach of learning from data and experiences has been shown across various tasks~\cite{akkaya2019solving, brown2020language, silver2016mastering} other than computer vision to be more efficient than relying on hard-coded knowledge for solving decision-making problems. However, collecting large sets of data is often time-consuming and labor-intensive which makes it challenging to deploy machine learning systems in the real-world where data is even more expensive to collect. Without overcoming the challenge of data collection, we need to develop data-efficient methods that can learn from few examples and generalize to new scenarios for enabling computer vision-based assistive vision devices to be useful for VIPs.   



\MK{ 
\begin{itemize}
	\item the connection between all papers is image classification
	\item talk about how complex the visual system is and how computer vision is used to replicate this
	\item talk about the challenges that can arise without having visual capabilities so that using computer vision would help, for example finding lost items at home
\end{itemize}
}


\section{Vision Impairments}
There are many different degrees of vision impairments ranging from problems with seeing from near or farther distances to blindness. Vision impairments are generally assessed by measuring the visual acuity from a fixed distance~\cite{who2019world}. In 2020, the World Health Organization (WHO) estimated that at least 2.2 billion people live with a near or distance vision impairment, wherein at least 1 billion cases the impairments could have been prevented or yet has to be addressed~\cite{who2019world}. The untreated cases are projected to grow to 1.7 billion people by 2050 mainly due to population growth and aging~\cite{bourne2021trends}. The leading causes for vision loss are uncorrected refractive errors (161 million people with distance vision loss and 510 million people with near vision loss), untreated cataracts (100 million people), age-related macular degenerationm (8.1 million), glaucoma (7.8 million), diabetic retinopathy (4.4 million) where 90\% of vision losses are preventable and treatable~\cite{steinmetz2021causes}. Furthermore, the prevalence of distance vision impairment are estimated to be four times higher in low- and middle-income regions than in high-income regions~\cite{steinmetz2021causes}.  

Several kinds of aids and tools have been developed for VIs to facilitate their capabilities of performing everyday tasks.   


%In Sweden, there are over 100.000 people with 




In 2020, there were almost 600 million people with mild to severe visual impairments around the world~\cite{bourne2021trends}. 

Describe common causes for visual impairment. What aids are there for helping them out, which will lead to assistive vision devices in the next section. 

Challenges: Mobility related challenges are what the VIs are most concerned about in general~\cite{stahl2018levnadsundersokning}.

VIPs that had to change their careers due to inaccessible apps and systems at their workplace~\cite{gotesson2019challenges, gotesson2019utmaningar}. Conclusion was that VIPs should be involved in the development and design process of such products, and that this should be included when educating User Experience (UX) designers. 


\section{Assistive Technologies Based on Computer Vision} %\section{Object Recognition for Assistive Vision}

Describe research and commercial products that uses computer vision-based assistive devices. Discuss challenges with those devices, e.g. getting real data, how to train them more data-efficiently, and how to learn new classes, which will lead to my contributions later. 

\cite{manduchi2012computer} highlights the current state of affairs, challenges, and potential outcomes of electronic devices for assistive vision. 

\section{Scope of Thesis}
Discuss how thesis is narrowed down and focused on image classification and like data collection, how to use additional labels, and then how to learn new labels continuously. 


% Thesis contributions

\section{Thesis Contributions}
\label{sec:contributions}

% Thesis outline

\section{Thesis Outline}
\label{sec:outline}


\begin{comment}
Here is an example for referencing figure \ref{EnergySources}. Example of citing \cite{BP2019} and \cite{Chen2016}.
%%%%%%%%%%%%%% Figure: Energy sources
\begin{figure}[h]
	\centering
	\includegraphics[scale=0.6]{Figs/Ch1_EnergySources.png}
	\caption{The world's energy consumption by fuel in 2017. }
	\label{EnergySources}
\end{figure}

%%%%%%%%%%%%%%%%%%%%%%%%%%%%%%%%%%%%%%%%
\section{Section}
Example of a table
%%%%%%%%%%%%%%%%%%%%%%%%%%%%%%%% Table: Tokamaks
\begin{table}[t!]
	\centering
	\caption{List of experimental tokamaks worldwide. Note: ITER is currently under construction and the first plasma is predicted for 2025-2028.}
	\begin{tabular}{ccccc}
		\hline
		\textbf{Name} & \textbf{Location} & \textbf{B-field} & \textbf{Major/minor radius}  \\
		\hline
		JET     & England      & 4.0 T & 3.0 m / 1.3 m \\
		ITER    & France       & 5.3 T & 6.2 m / 2.0 m \\
		AUG		& Germany      & 3.1 T & 1.7 m / 0.7 m \\
		WEST	& France	   & 3.7 T & 2.5 m / 0.5 m \\
		TCV     & Switzerland  & 1.5 T & 0.9 m / 0.3 m \\
		DIII-D  & USA          & 2.2 T & 1.7 m / 0.7 m \\
		TFTR 	& USA          & 6.0 T & 2.5 m / 0.9 m \\
		JT-60   & Japan        & 4.0 T & 3.4 m / 1.0 m \\
		K-STAR  & South Korea  & 3.5 T & 1.8 m / 0.5 m \\
		EAST    & China        & 3.5 T & 1.9 m / 0.5 m \\
		\hline
	\end{tabular}
	\label{TokamakTable}
\end{table}
\end{comment}
