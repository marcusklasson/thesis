%*******************************************************************************
%*********************************** Third Chapter *****************************
%*******************************************************************************

\chapter{Fine-grained Classification}
\label{chap:finegrained_classification}

%% Write about Paper A and B in a coherent way

This chapter presents an approach for enhancing fine-grained classification performance of grocery items by using web-scraped information. We focus on classification of grocery items due to applicability in assistive vision and its potential to enhance the independence of visually impaired (VI) people [ADD groceries/shopping/object recognition for VI REFs]. Initially, we were interested in learning classifiers with natural images taken in the grocery stores combined with web-scraped information about the grocery items, such as iconic images and text descriptions from supermarket websites. Using iconic images have been used in grocery image classification earlier [ADD grocery paper REFs], however, utilizing text descriptions was as far we know absent for this application even if it has been successfully applied in other image classification problems [ADD REFs, Bujwid and Sullivan, AwA dataset, or other Attribute datasets]. Thus, we collected our own dataset of grocery items images using a mobile phone camera as well as web-scraped images and text descriptions to study whether this multi-view approaches would benefit training the classifiers (Section 2). We then select a multi-view learning framework based on the Variational Autoender (VAE) for investigating how the different data views affect the fine-grained classification performance (Section 3). 

%\section{Introduction}
\section{Related Work}
\section{Dataset Collection}
\section{Representation Learning}
\section{Experiments}
\section{Discussion}


\noindent 
%In this chapter, we provide a summaries of the included paper for this thesis. Paper \ref{sec:paperA} and \ref{sec:paperB} are connected through the Grocery Store dataset where we present the work and then perform an ablation study over which modalities in the dataset that are useful for training classifiers. In Paper \ref{sec:paperC} and \ref{sec:paperD}, we focus on continual learning (CL) and present a new setting that aims to fill the gap between CL research and real-world problems as well as a method for doing so. 

%\begingroup
%\renewcommand\thesection{\Alph{section}} % for changing section numbering to alphabetic
%
\section{A Hierarchical Grocery Store Image Dataset with Visual and Semantic Labels}
\label{sec:paperA}



\subsection{Author Contributions}

%
\section{Using Variational Multi-View Learning for Classification of Grocery Items}
\label{sec:paperB}

\textbf{Authors:} Marcus Klasson, Cheng Zhang, Hedvig Kjellström. 

\paragraph{Summary.} 
We investigate whether training image classifiers can benefit from learning joint representations of grocery items using multi-view learning over the natural images and web-scraped information of the grocery items in the Grocery Store dataset (see Paper \ref{sec:paperA}). We employ a deep multi-view model based on VAEs called Variational Canonical Correlation Analysis (VCCA)~\cite{wang2016deep} for learning joint representations of the different data types, i.e., natural images, iconic images, and text descriptions. We performed a thorough ablation study over all data types to demonstrate how they contribute individually to enhancing the classification performance. Furthermore, we apply two classification approaches where we (i) train the classifier on the joint latent representations, and (ii) using a generative classifier by incorporating a class decoder to the VCCA model that can be used for classifying images. 

We performed a thorough ablation study over all data types to demonstrate how they contribute individually to enhancing the classification performance. To gain further insights into our results, we visualized the learned representations of the grocery items from VCCA and discussed how the iconic images and text descriptions help the model to better distinguish between the groceries. Our results show that the iconic images help to group the items based on their color and shape while text descriptions separate the items based on differences in ingredients and flavor. Finally, we concluded that utilizing the iconic images and text descriptions yielded better classification results than only using natural images. 


\paragraph{Author Contributions.} 
CZ and HK presented the idea and all authors contributed to formalizing the methodology. 
MK performed all the experiments and created the visualizations. 
All authors took part in discussing the results.
All authors contributed to writing the manuscript. 
%
\section{Learn the Time to Learn: Replay Scheduling for Continual Learning}
\label{sec:paperC}

\textbf{Authors:} Marcus Klasson, Hedvig Kjellström, Cheng Zhang. 

\paragraph{Summary.}
In this paper, we show that learning the time to replay different tasks can be critical for continual learning (CL) performance in replay-based methods. As the main assumption in replay-based CL is that only a small set of historical data can be re-visited for mitigating catastrophic forgetting, most works have focused on improving the sample quality of the replay memory. However, in many real-world applications, historical data is accessible at all times, e.g., by storing it on the cloud. But although all historical data could be stored, retraining machine learning systems on a daily basis is prohibitive due to processing times and operational costs. Therefore, small replay memories are still needed in CL to mitigate catastrophic forgetting when learning new tasks. To this end, we propose to learn the time to learn for a CL system, in which we learn schedules over which tasks to replay at different times. Inspired by human learning, we demonstrate that scheduling over the time to replay is critical to the final CL performance with finite memory resources. We then illustrate our idea with scheduling over which tasks to replay by learning such policy with Monte Carlo tree search. We perform extensive evaluation showing that learning replay schedules can significantly improve the performance compared to baselines without learned scheduling. We also show that our method can be combined with any replay-based method and memory selection technique. Finally, our results indicate that the learned schedules are also consistent with human learning insights.



\paragraph{Author Contributions.} 
CZ presented the idea and MK and CZ contributed to formalizing the methodology. 
MK performed all the experiments. 
All authors took part in discussing the results and contributed to writing the manuscript. 
%
\section{Meta Policy Learning for Replay Scheduling in Continual Learning}
\label{sec:paperD}

\textbf{Authors:} Marcus Klasson, Hedvig Kjellström, Cheng Zhang. 

\paragraph{Summary.} 



\paragraph{Author Contributions.} CZ presented the idea. 


%\endgroup
