%*******************************************************************************
%*********************************** Third Chapter *****************************
%*******************************************************************************

\chapter{Fine-grained Classification}
\label{chap:finegrained_classification}

%% Write about Paper A and B in a coherent way

This chapter presents an approach for enhancing fine-grained classification performance of grocery items by using web-scraped information. We focus on classification of grocery items due to applicability in assistive vision and its potential to enhance the independence of visually impaired (VI) people [ADD groceries/shopping/object recognition for VI REFs]. Initially, we were interested in learning classifiers with natural images taken in the grocery stores combined with web-scraped information about the grocery items, such as iconic images and text descriptions from supermarket websites. Using iconic images have been used in grocery image classification earlier [ADD grocery paper REFs], however, utilizing text descriptions was as far we know absent for this application even if it has been successfully applied in other image classification problems [ADD REFs, Bujwid and Sullivan, AwA dataset, or other Attribute datasets]. Thus, we collected our own dataset of grocery items images using a mobile phone camera as well as web-scraped images and text descriptions to study whether this multi-view approaches would benefit training the classifiers (Section 2). We then select a multi-view learning framework based on the Variational Autoender (VAE) for investigating how the different data views affect the fine-grained classification performance (Section 3). 

%\section{Introduction}
\section{Related Work}
\section{Dataset Collection}
\section{Representation Learning}
\section{Experiments}
\section{Discussion}


\noindent 
%In this chapter, we provide a summaries of the included paper for this thesis. Paper \ref{sec:paperA} and \ref{sec:paperB} are connected through the Grocery Store dataset where we present the work and then perform an ablation study over which modalities in the dataset that are useful for training classifiers. In Paper \ref{sec:paperC} and \ref{sec:paperD}, we focus on continual learning (CL) and present a new setting that aims to fill the gap between CL research and real-world problems as well as a method for doing so. 

%\begingroup
%\renewcommand\thesection{\Alph{section}} % for changing section numbering to alphabetic
%
\section{A Hierarchical Grocery Store Image Dataset with Visual and Semantic Labels}
\label{sec:paperA}

\textbf{Authors:} Marcus Klasson, Cheng Zhang, Hedvig Kjellström. 

\paragraph{Summary.} 
We collect a dataset with natural images of raw and refrigerated grocery items taken in grocery stores in Stockholm, Sweden, for evaluating image classification models on a challenging real-world scenario. The data collection was performed by taking photos of groceries with a mobile phone to simulate a scenario of grocery shopping using an assistive vision app. Furthermore, we downloaded iconic images and text descriptions of each grocery item by web-scraping a grocery store website to enhance the dataset with information describing the semantics of each individual item. the items are grouped based on their type, e.g., apple, juice, etc., to provide the dataset with a hierarchical labeling structure. 

We provide benchmark results evaluated using pre-trained and fine-tuned CNNs for image classification. Moreover, we take an initial step towards utilizing the rich product information in the dataset by training the classifiers with representations where both natural and iconic images have been combined through a multi-view VAE. 



\paragraph{Author Contributions.} CZ and HK presented the idea and the data collection procedure for the natural images and web-scraped information. MK performed the data collection including visiting the grocery stores for taking the natural images and the web-scraping of the grocery store website for iconic images and text descriptions. MK performed all the experiments. All authors contributed to discussing the results and contributed to writing the manuscript. 


%
\section{Using Variational Multi-view Learning for Classification of Grocery Items}
\label{sec:paperB}

\textbf{Authors:} Marcus Klasson, Cheng Zhang, Hedvig Kjellström. 
%
\section{Learn the Time to Learn: Replay Scheduling for Continual Learning}
\label{sec:paperC}
%
\section{Meta Policy Learning for Replay Scheduling in Continual Learning}
\label{sec:paperD}

\textbf{Authors:} Marcus Klasson, Hedvig Kjellström, Cheng Zhang. 

\paragraph{Summary.} 



\paragraph{Author Contributions.} CZ presented the idea. 


%\endgroup
