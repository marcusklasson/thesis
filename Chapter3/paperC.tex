
\section{Learn the Time to Learn: Replay Scheduling for Continual Learning}
\label{sec:paperC}

\textbf{Authors:} Marcus Klasson, Hedvig Kjellström, Cheng Zhang. 

\paragraph{Summary.}
In this paper, we show that learning the time to replay different tasks can be critical for continual learning (CL) performance in replay-based methods. As the main assumption in replay-based CL is that only a small set of historical data can be re-visited for mitigating catastrophic forgetting, most works have focused on improving the sample quality of the replay memory. However, in many real-world applications, historical data is accessible at all times, e.g., by storing it on the cloud. But although all historical data could be stored, retraining machine learning systems on a daily basis is prohibitive due to processing times and operational costs. Therefore, small replay memories are still needed in CL to mitigate catastrophic forgetting when learning new tasks. To this end, we propose to learn the time to learn for a CL system, in which we learn schedules over which tasks to replay at different times. Inspired by human learning, we demonstrate that scheduling over the time to replay is critical to the final CL performance with finite memory resources. We then illustrate our idea with scheduling over which tasks to replay by learning such policy with Monte Carlo tree search. We perform extensive evaluation showing that learning replay schedules can significantly improve the performance compared to baselines without learned scheduling. We also show that our method can be combined with any replay-based method and memory selection technique. Finally, our results indicate that the learned schedules are also consistent with human learning insights.



\paragraph{Author Contributions.} 
CZ presented the idea and MK and CZ contributed to formalizing the methodology. 
MK performed all the experiments. 
All authors took part in discussing the results and contributed to writing the manuscript. 