
\section{A Hierarchical Grocery Store Image Dataset with Visual and Semantic Labels}
\label{sec:paperA}

\textbf{Authors:} Marcus Klasson, Cheng Zhang, Hedvig Kjellström. 

\paragraph{Summary.} 
We collect a dataset with natural images of raw and refrigerated grocery items taken in grocery stores in Stockholm, Sweden, for evaluating image classification models on a challenging real-world scenario. The data collection was performed by taking photos of groceries with a smartphone camera to simulate a scenario of grocery shopping using an assistive vision app. Furthermore, we downloaded iconic images and text descriptions of each grocery item by web-scraping a grocery store website to enhance the dataset with information describing the semantics of each individual item. the items are grouped based on their type, e.g., apple, juice, etc., to provide the dataset with a hierarchical labeling structure. 

We provide benchmark results evaluated using pre-trained and fine-tuned CNNs for image classification. Moreover, we take an initial step towards utilizing the rich product information in the dataset by training the classifiers with representations where both natural and iconic images have been combined through a multi-view VAE. 



\paragraph{Author Contributions.} CZ and HK presented the idea and the data collection procedure for the natural images and web-scraped information. MK performed the data collection including visiting the grocery stores for taking the natural images and the web-scraping of the grocery store website for iconic images and text descriptions. MK performed all the experiments. All authors contributed to discussing the results and contributed to writing the manuscript. 

