
\chapter{Conclusions and Future Directions}\label{chap5}

In this chapter, we summarize and provide conclusions to each included paper in the thesis as well as discuss future work (Section \ref{chap5:sec:conclusions}). Finally, we discuss potential long-term future directions for this project that could be interesting for future researchers to dive into for developing computer vision methods for assisting the visually impaired (Section \ref{chap5:sec:future_directions}).

\section{Conclusions}\label{chap5:sec:conclusions}

\begin{itemize}
	\item Paper A: Presented a challenging fine-grained classification dataset of grocery items
	\item Paper B: Learning with web-scraped information can reduce the need for real-world images
	\item Paper C: Replay scheduling is important in our new CL setting that aligns well with real-world needs
	\item Paper D: Our RL-based method for replay scheduling can be applied to new CL scenarios for mitigating catastrophic forgetting
\end{itemize}



%\subsection{Limitations}


\section{Future Directions}\label{chap5:sec:future_directions}

\begin{itemize}
	\item Video data for object recognition instead of images for making systems easier to use. And use a disability-first approach when collecting the data
	\item Federated Learning for decentralizing model updates 
	%\item Uncertainty Quantification - How to make the classifiers trustworthy?
\end{itemize}


\subsection{Disability-first Approaches}


\subsection{Federated Learning}