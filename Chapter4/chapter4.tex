%*******************************************************************************
%*********************************** Fourth Chapter ****************************
%*******************************************************************************

\chapter{Continual Learning}
%\chapter{Summary of Included Papers}
%\label{chap:summary_of_included_papers}

This chapter introduces the idea of replay scheduling for mitigating catastrophic forgetting in continual learning (CL). The problem setting of CL is on learning tasks of recognizing a new set of classes with a dataset given at the current time step. In the standard setting, one main assumption is that the data from past tasks can never be fully revisited by the model. However, in the real-world, many organizations record data from incoming streams for storage rather than deleting it~\cite{bailis2017macrobase, mitchell1999machinelearning} [Add at least 1 more REF]. In contrast to the assumption on data storage in standard CL, we suggest a new setting where we assume that all seen data is accessible at any time for the model to revisit. The challenge then becomes how to select which tasks that needs to be remembered via replay as the data is still incoming from a stream. We propose to learn the time when replaying a certain task is necessary when the model is updating its knowledge with new incoming tasks. In Paper C, we propose the new CL setting where historical data is accessible and introduce the idea of replay scheduling and how it can be used in CL. In Paper D, we propose a framework based on reinforcement learning~\cite{sutton2018reinforcement} (RL) for learning replay scheduling policies that can be applied in new CL scenarios. 
%A simple, yet effcient, approach to mitigate catastrophically forgetting past tasks is to add some 
%Replay scheduling involves selecting which tasks to fetch examples from and add to a replay memory at different times. The replay memory is then mixed with a batch of examples from the current task to learn to help the network to remember the previously learned tasks selected by the scheduler. 

%\section{Introduction}
\section{Related Work}
% General CL papers, add new papers that you have seen. Then Replay/memory-based CL approaches, add new and contrastive approaches. Perhaps something on meta-policy learning approaches, would be useful to tell how we use reinforcement learning here

\section{Replay Scheduling in Continual Learning} 

In this section, we introduce a slightly new CL setting considering the real-world needs where all historical data can be available since data storage is cheap. However, the amount of compute is limited when the model is updated on new data due to operational costs. Hence, it is impossible for the model to leverage from all the available historical data to mitigate catastrophic forgetting. The goal then becomes to learn how we can select subsets of historical data for replay to efficiently reduce forgetting of the old tasks. We will refer to these subsets of historical data as the \textit{replay memory} throughout this chapter. The size of the replay memory affects the processing time when learning new tasks as well as the allowed time for the training phase. When composing the replay memory, we focus on determining the number of samples to draw from the seen tasks in the historical data rather than selecting single stored instances. Next, we introduce the problem setting in more detail as well as the notation of the new CL setting. 


\subsection{Problem Setting}

We introduce the notation of our problem setting which resembles the traditional CL setting for image classification. We let a neural network $f_{\vtheta}$, parameterized by $\vtheta$, learn $T$ tasks from the datasets $\gD_1, \dots, \gD_T$ arriving sequentially one at a time. The $t$-th dataset $\gD_t = \{(\vx_{t}^{(i)}, \vy_{t}^{(i)})\}_{i=1}^{N_{t}}$ consists of $N_t$ samples where $\vx_{t}^{(i)}$ and $\vy_{t}^{(i)}$ are the $i$-th data point and class label respectively. The training objective at task $t$ is given by 
\vspace{-2mm}
\begin{align}
	\underset{\vtheta}{\text{min}} \sum_{i=1}^{N_t} \gL(f_{\vtheta}(\vx_t^{(i)}), \vy_{t}^{(i)}),
\end{align}
where $\gL(\cdot)$ is the loss function, which in our case is the cross-entropy loss.  
When learning task $t$, the network $f_{\vtheta}$ is at risk of catastrophically forgetting the previous $t-1$ tasks. The forgetting effect shows as the decrease in task accuracy between time steps, for example, $A_{t, i} < A_{t, i-1}$ where $A_{t, i}$ is the accuracy for task $t$ at time step $i$. Replay-based methods mitigate catastrophic forgetting by storing a few number of examples from historical tasks in an external memory. The network $f_{\vtheta}$ is then allowed to fetch old examples from the memory and mix these with the current task dataset to remind itself about the previous tasks during training. 

In the new problem setting, we assume that historical data from old tasks are accessible at any time step. However, we can only fill a small replay memory $\gM$ with $M$ historical samples for replaying old tasks due to processing time constraints prohibiting re-using all historical data at the same time. The challenge then becomes how to know which tasks to include in the replay memory that efficiently retain the previous knowledge when learning new tasks. We decide to fill the replay memory with $M$ historical samples using sequence of task proportions $(p_1, \dots, p_{t-1})$ where $\sum_{i=1}^{t-1} p_i = 1$ and $p_{i} \geq 0$. The number of samples from task $i$ to place in $\gM$ is given by $p_i \cdot M$. In the next section, we introduce a method for selecting the task proportions of which old tasks to replay. 


\paragraph{Comparison to Traditional CL.} The new setting has several similarities to the traditional CL setting. Both settings share the fundamental setting that the data arrive in streams and re-training on all historical data is prohibited. Also, the goal that the model should perform well both historical tasks and tasks associated with new data remains the same. In replay-based CL, we also share the same constraints that the memory size is limited. However, we argue that this limitation is mainly associated with compute rather than of storage. Our assumption aligns with the real-world where data storage is cheap and easy to maintain, but retraining large machine learning models is computationally expensive. The only difference is that we allow filling the limited replay memory from historical data or some other external memory. Here, we argue that historical data is stored rather than deleted in many real-world settings~\cite{bailis2017macrobase}. Thus, we should keep the limited memory assumption for training but allow access to historical data to fill the replay memory to make CL align with real-world needs.
\MK{this paragraph can be written in a more humble way, more like "this is something worth to investigate"}



\subsection{Replay Scheduling for Mitigating Catastrophic Forgetting}

In this section, we describe our replay scheduling method for selecting the replay memory at different time steps. A replay schedule is defined as a sequence $S = (\vp_1, \dots, \vp_{T-1})$, where $\vp_i = (p_1, \dots, p_{T-1})$ for $1 \leq i \leq T-1$ is the sequence of task proportions for determining how many samples per task to fill the replay memory with at time step $i$. To make the selection of task proportions tractable, we construct an action space with a discrete number of choices for the task proportions from historical tasks. 

\paragraph{Discrete Action Space Construction.} We show the procudre for creating the discrete action space in Algorithm \ref{alg:action_space_discretization}. At task $i$, we have $i-1$ historical tasks that we can choose from. We then generate all possible bin vectors $\vb_i = [b_1, \dots, b_{i}] \in \gB_i$ of size $i$ where each element are a task index $1, ..., i$. We sort all bin vectors by the order of task indices and only keep the unique bin vectors. For example, at $i=2$, the unique choices of vectors are $[1,1], [1,2], [2,2]$, where $[1,1]$ indicates that all samples in the replay memory should be from task 1, $[1,2]$ indicates that half memory is from task 1 and the other half are from task etc. The task proportions are then computed by counting the number of occurrences of each task index in $\vb_i$ and dividing by $i$, such that $\vp_i = \texttt{bincount}(\vb_i) / (i)$. From this specification, we have built a tree $\gT$ with different task proportions that can be selected at different time steps. We construct a replay schedule $S$ by traversing through $\gT$ and select a task proportion on every level to append to $S$. We can then evaluate the replay schedule $S$ by training a network on the CL task sequence and use $S$ to compose the replay memory to use for mitigating catastrophic fogretting at every task. 


\begin{algorithm}[t]
	\caption{Discretization of action space with task proportions}
	\label{alg:action_space_discretization}
	\begin{algorithmic}[1]
		\Require Number of tasks $T$
		\State $\gT = ()$ \Comment{Initialize sequence for storing actions}
		\For{$i = 1, \dots, T-1$}
		\State $\gP_i = \{\}$ \Comment{Set for storing task proportions at $i$}
		\State $\gB = \texttt{combinations}([1:i], i)$ \Comment{Get bin vectors of size $i$ with bins $1, ..., i$}
		\State $\bar{\gB} = \texttt{unique}(\texttt{sort}(\gB))$ \Comment{Only keep unique bin vectors}
		\For{$\vb_i \in \hat{\gB}$}
		\State $\vp_i = \texttt{bincount}(\vb_i) / i$ \Comment{Calculate task proportion}
		\State $\gP_i = \gP_i \cup \{ \vp_i \}$ \Comment{Add task proportion to set}
		\EndFor
		\State $\gT[i] = \gP_i$ \Comment{Add set of task proportions to action sequence}
		\EndFor
		\State \Return $\gT$ \Comment{Return action sequence as discrete action space}
	\end{algorithmic}
\end{algorithm}






%% Have sectiobn for MCTS. include a limitations paragraph there that can be referred to later in Discussion. It kind of bridges over well to the meta-policy learning too. Say that we will use MCTS for studying if scheduling is important in this new setting.


\section{Meta-Policy Learning for Replay Scheduling}
\section{Experiments}
\section{Discussion}



