%*******************************************************************************
%*********************************** Fourth Chapter ****************************
%*******************************************************************************

\chapter{Continual Learning}
%\chapter{Summary of Included Papers}
%\label{chap:summary_of_included_papers}

This chapter introduces the idea of replay scheduling for mitigating catastrophic forgetting in continual learning (CL). The problem setting of CL is on learning tasks of recognizing a new set of classes with a dataset given at the current time step. In the standard setting, one main assumption is that the data from past tasks can never be fully revisited by the model. However, in the real-world, many organizations record data from incoming streams for storage rather than deleting it~\cite{bailis2017macrobase, mitchell1999machinelearning} [Add at least 1 more REF]. In contrast to the assumption on data storage in standard CL, we suggest a new setting where we assume that all seen data is accessible at any time for the model to revisit. The challenge then becomes how to select which tasks that needs to be remembered via replay as the data is still incoming from a stream. We propose to learn the time when replaying a certain task is necessary when the model is updating its knowledge with new incoming tasks. In Paper C, we propose the new CL setting where historical data is accessible and introduce the idea of replay scheduling and how it can be used in CL. In Paper D, we propose a framework based on reinforcement learning~\cite{sutton2018reinforcement} (RL) for learning replay scheduling policies that can be applied in new CL scenarios. 
%A simple, yet effcient, approach to mitigate catastrophically forgetting past tasks is to add some 
%Replay scheduling involves selecting which tasks to fetch examples from and add to a replay memory at different times. The replay memory is then mixed with a batch of examples from the current task to learn to help the network to remember the previously learned tasks selected by the scheduler. 

\MK{TO-DO: Motivate replay scheduling from mobile phone perspective, perhaps from perspective that phones can store lots of data but how to select which classes to replay. Maybe it should also be from the computational perspective, that we want such scheduling policy to work in many scenarios without additional compute cost for the policy learning. }

%\section{Introduction}
\section{Related Work}\label{ch4/sec:related_work}

In this section, we give an overview of previous works related to Paper C and D. We begin by describing different approaches in CL, especially replay-based approaches, and then discuss meta-learning policies and generalization in RL.

\paragraph{Continual Learning.} There exist many different approaches in CL for mitigating catastrophic forgetting in neural networks. In general, these approaches can be divided into three main cores, namely, \textit{regularization-based}, \textit{architecture-based}, and \textit{replay-based} methods. Regularization-based methods are mainly focused on applying regularization techniques on parameters important for recognizing old tasks and fit the remaining parameters to new tasks~\cite{kirkpatrick2017overcoming, zenke2017continual, nguyen2017variational}. Knowledge distillation methods~\cite{hinton2015distilling} also belong to these approaches where classification logits are used for regularizing the output units for previous tasks in the network~\cite{li2017learning, schwarz2018progress}. More recently, there are some works that uses projection-based approaches for constraining the parameter updates to subspaces which avoid interference with previous tasks~\cite{saha2021gradient, kao2021natural}. Architecture-based approaches focuses on adding task-specific network modules for every seen task~\cite{rusu2016progressive, yoon2017lifelong, yoon2019scalable, ebrahimi2020adversarial}, or isolating parameters for predicting specific task in fixed-size networks~\cite{mallya2018packnet, serra2018overcoming, schwarz2021powerpropagation}. Replay-based methods re-trains on samples of old tasks when learning new tasks. The old samples are either stored in an external memory~\cite{chaudhry2019tiny, hayes2020remind, rolnick2018experience}, or synthesized with a generative model~\cite{shi2019variational, van2018generative, van2020brain, wu2018memory}. Both regularization- and architecture-based methods can be combined with replay for improving the models capability of remembering tasks [Add REFs]. Our replay scheduling idea is originated from replay-based methods which we will cover more in detail next. 


\paragraph{Replay-based Continual Learning.} In this thesis, we focus on replay-based methods with external memories for storing historical data. The most common selection strategy for filling the memory is random sampling from the used datasets. There exist several works focusing on selecting high quality samples for storing in the memory [Add REFs]. However, in image classification problems, random sampling has been shown to often perform on par with more elaborate selection strategies~\cite{chaudhry2018riemannian, hayes2020remind}. In contrast to using various memory selection methods, there has been proposals of retrieval policies over which samples to select for replay from the memory, for instance, selecting the samples that will mostly interfere with the parameter update with batches of new data~\cite{aljundi2019online}. Our replay scheduling approach differs from this method as we focus on selecting which tasks to select for replay rather than the individual samples to retrieve from the memory.  
More recent works have focused on evolving the memory samples through data augmentation to avoid overfitting to the memory [Add REFs], and also by using contrastive learning to improve discriminating between tasks. Another direction has been to increasing the storage capacity to store more samples by compressing raw data into features that are more memory-cheap [Add REFs]. The above mentioned methods assume that the memory is small and allocates equal storage amount for all tasks. Our new problem setting for memory-based CL is different from this assumption as we argue that data storage is cheap in many real-world applications. Hence, we compose a replay memory with data from historical tasks before learning new tasks because the amount of compute is limited. However, replay scheduling can be combined with of the mentioned methods as it only differs with the standard memory-based CL setting in that the replay memory has to be selected at every new task. 

\paragraph{Meta-Policy Learning?}
%In contrast to our proposed problem setting Replay scheduling can be combined with any of the above mentioned methods as it .  
%Replay scheduling can however be combined with any selection strategy and feature compression method.

%there has been work on compressing raw images to feature representations to increase the number of memory examples for replay~\citep{hayes2020remind, iscen2020memory, pellegrini2019latent}. 


%A commonly used memory selection strategy of replay samples is random selection. 
%The simplest selection strategy is random selection of examples to store in the memory for replay. 
%Much research effort has focused on selecting higher quality samples to store in memory~\citep{aljundi2019gradient, borsos2020coresets, chaudhry2019tiny, chrysakis2020online, hayes2019memory, isele2018selective, lopez2017gradient, nguyen2017variational, rebuffi2017icarl, yoon2021online}. \citet{chaudhry2019tiny} reviews several selection strategies in scenarios with tiny memory capacity, e.g., reservoir sampling~\citep{vitter1985random}, first-in first-out buffer~\citep{lopez2017gradient}, k-Means, and Mean-of-Features~\citep{rebuffi2017icarl}. However, more elaborate selection strategies have been shown to give little benefit over random selection for image classification problems~\citep{chaudhry2018riemannian, hayes2020remind}. More recently, there has been work on compressing raw images to feature representations to increase the number of memory examples for replay~\citep{hayes2020remind, iscen2020memory, pellegrini2019latent}. 
%Our approach differs from the above mentioned works since we focus on learning to select which tasks to replay at the current task rather than improving memory selection or compression quality of the memory samples. %samples in the memory. 
%Our approach differs from the above mentioned works since we focus on which memory examples to choose for training at the current task rather than which examples to store in the memory. 
%Replay scheduling can however be combined with any selection strategy and feature compression method.


% General CL papers, add new papers that you have seen. Then Replay/memory-based CL approaches, add new and contrastive approaches. Perhaps something on meta-policy learning approaches, would be useful to tell how we use reinforcement learning here

\paragraph{Similarities and Differences between Continual Learning and other fields.} \MK{A short and cozy table fo this, similar to survey by Delange etal 2021}

\section{Replay Scheduling in Continual Learning} 

In this section, we introduce a slightly new CL setting considering the real-world needs where all historical data can be available since data storage is cheap. However, the amount of compute is limited when the model is updated on new data due to operational costs. Hence, it is impossible for the model to leverage from all the available historical data to mitigate catastrophic forgetting. The goal then becomes to learn how we can select subsets of historical data for replay to efficiently reduce forgetting of the old tasks. We will refer to these subsets of historical data as the \textit{replay memory} throughout this chapter. The size of the replay memory affects the processing time when learning new tasks as well as the allowed time for the training phase. When composing the replay memory, we focus on determining the number of samples to draw from the seen tasks in the historical data rather than selecting single stored instances. Next, we introduce the problem setting in more detail as well as the notation of the new CL setting. 


\subsection{Problem Setting}

We introduce the notation of our problem setting which resembles the traditional CL setting for image classification. We let a neural network $f_{\vtheta}$, parameterized by $\vtheta$, learn $T$ tasks from the datasets $\gD_1, \dots, \gD_T$ arriving sequentially one at a time. The $t$-th dataset $\gD_t = \{(\vx_{t}^{(i)}, \vy_{t}^{(i)})\}_{i=1}^{N_{t}}$ consists of $N_t$ samples where $\vx_{t}^{(i)}$ and $\vy_{t}^{(i)}$ are the $i$-th data point and class label respectively. The training objective at task $t$ is given by 
\vspace{-2mm}
\begin{align}
	\underset{\vtheta}{\text{min}} \sum_{i=1}^{N_t} \gL(f_{\vtheta}(\vx_t^{(i)}), \vy_{t}^{(i)}),
\end{align}
where $\gL(\cdot)$ is the loss function, which in our case is the cross-entropy loss.  
When learning task $t$, the network $f_{\vtheta}$ is at risk of catastrophically forgetting the previous $t-1$ tasks. The forgetting effect shows as the decrease in task accuracy between time steps, for example, $A_{t, i} < A_{t, i-1}$ where $A_{t, i}$ is the accuracy for task $t$ at time step $i$. Replay-based methods mitigate catastrophic forgetting by storing a few number of examples from historical tasks in an external memory. The network $f_{\vtheta}$ is then allowed to fetch old examples from the memory and mix these with the current task dataset to remind itself about the previous tasks during training. 

In the new problem setting, we assume that historical data from old tasks are accessible at any time step. However, we can only fill a small replay memory $\gM$ with $M$ historical samples for replaying old tasks due to processing time constraints prohibiting re-using all historical data at the same time. The challenge then becomes how to know which tasks to include in the replay memory that efficiently retain the previous knowledge when learning new tasks. We decide to fill the replay memory with $M$ historical samples using sequence of task proportions $(p_1, \dots, p_{t-1})$ where $\sum_{i=1}^{t-1} p_i = 1$ and $p_{i} \geq 0$. The number of samples from task $i$ to place in $\gM$ is given by $p_i \cdot M$. In the next section, we introduce a method for selecting the task proportions of which old tasks to replay. 


\paragraph{Comparison to Traditional CL.} The new setting has several similarities to the traditional CL setting. Both settings share the fundamental setting that the data arrive in streams and re-training on all historical data is prohibited. Also, the goal that the model should perform well both historical tasks and tasks associated with new data remains the same. In replay-based CL, we also share the same constraints that the memory size is limited. However, we argue that this limitation is mainly associated with compute rather than of storage. Our assumption aligns with the real-world where data storage is cheap and easy to maintain, but retraining large machine learning models is computationally expensive. The only difference is that we allow filling the limited replay memory from historical data or some other external memory. Here, we argue that historical data is stored rather than deleted in many real-world settings~\cite{bailis2017macrobase}. Thus, we should keep the limited memory assumption for training but allow access to historical data to fill the replay memory to make CL align with real-world needs.
\MK{this paragraph can be written in a more humble way, more like "this is something worth to investigate"}



\subsection{Replay Scheduling for Mitigating Catastrophic Forgetting}

In this section, we describe our replay scheduling method for selecting the replay memory at different time steps. A replay schedule is defined as a sequence $S = (\vp_1, \dots, \vp_{T-1})$, where $\vp_i = (p_1, \dots, p_{T-1})$ for $1 \leq i \leq T-1$ is the sequence of task proportions for determining how many samples per task to fill the replay memory with at time step $i$. To make the selection of task proportions tractable, we construct an action space with a discrete number of choices for the task proportions from historical tasks. 

We show the procudre for creating the discrete action space in Algorithm \ref{alg:action_space_discretization}. At task $i$, we have $i-1$ historical tasks that we can choose from. We then generate all possible bin vectors $\vb_i = [b_1, \dots, b_{i}] \in \gB_i$ of size $i$ where each element are a task index $1, ..., i$. We sort all bin vectors by the order of task indices and only keep the unique bin vectors. For example, at $i=2$, the unique choices of vectors are $[1,1], [1,2], [2,2]$, where $[1,1]$ indicates that all samples in the replay memory should be from task 1, $[1,2]$ indicates that half memory is from task 1 and the other half are from task etc. The task proportions are then computed by counting the number of occurrences of each task index in $\vb_i$ and dividing by $i$, such that $\vp_i = \texttt{bincount}(\vb_i) / (i)$. From this specification, we have built a tree $\gT$ with different task proportions that can be selected at different time steps. We construct a replay schedule $S$ by traversing through $\gT$ and select a task proportion on every level to append to $S$. We can then evaluate the replay schedule $S$ by training a network on the CL task sequence and use $S$ to compose the replay memory to use for mitigating catastrophic forgetting at every task. \MK{Question, Wrap algorithm around this paragraph and remove the comments? Algorithm with comments could be placed in Appendix of Paper D.}


\begin{algorithm}[t]
	\caption{Discretization of action space with task proportions}
	\label{alg:action_space_discretization}
	\begin{algorithmic}[1]
		\Require Number of tasks $T$
		\State $\gT = ()$ \Comment{Initialize sequence for storing actions}
		\For{$i = 1, \dots, T-1$}
		\State $\gP_i = \{\}$ \Comment{Set for storing task proportions at $i$}
		\State $\gB = \texttt{combinations}([1:i], i)$ \Comment{Get bin vectors of size $i$ with bins $1, ..., i$}
		\State $\bar{\gB} = \texttt{unique}(\texttt{sort}(\gB))$ \Comment{Only keep unique bin vectors}
		\For{$\vb_i \in \hat{\gB}$}
		\State $\vp_i = \texttt{bincount}(\vb_i) / i$ \Comment{Calculate task proportion}
		\State $\gP_i = \gP_i \cup \{ \vp_i \}$ \Comment{Add task proportion to set}
		\EndFor
		\State $\gT[i] = \gP_i$ \Comment{Add set of task proportions to action sequence}
		\EndFor
		\State \Return $\gT$ \Comment{Return action sequence as discrete action space}
	\end{algorithmic}
\end{algorithm}



We are interested in studying whether the time to replay different tasks is important in the new CL setting. One option is to use a brute-force approach, for example, breadth-first search, and evaluate every possible replay schedule in the tree. However, as the tree grows fast with the number of tasks, we need a scalable method that can perform searches in large action spaces. We suggest using Monte Carlo tree search~\cite{coulom2006efficient} (MCTS) due to its previous successes in applications with similar conditions as ours~\cite{browne2012survey, silver2016mastering, chaudhry2018feature}. The use of MCTS enables performing search for datasets with longer task horizons. Furthermore, MCTS encourages searches in promising paths based on the selected reward function, which we set as the average accuracy of all tasks after learning the final task achieved by the network. We provide the full details on how MCTS is used to search for replay schedules in Paper C. 
\MK{Question: More info on MCTS?}


%% Have section for MCTS. include a limitations paragraph there that can be referred to later in Discussion. It kind of bridges over well to the meta-policy learning too. Say that we will use MCTS for studying if scheduling is important in this new setting.


\section{Meta Policy Learning for Replay Scheduling}
\label{sec:meta_policy_learning_for_replay_scheduling}

In this section, we present an RL-based framework to learn policies for selecting which tasks to replay in CL scenarios. We are interested in learning such policy that can be transferred to new CL scenarios, such as new task orders and new datasets, without any additional computational cost for updating on the new domain. We take a meta-learning approach where the policy learns from episodes of experience collected from training a classifier in CL settings. The experience from the environment is represented as the classification performance on each seen task in the dataset. The policy receives the task performances for basing its action on which task that needs to be replayed at the next time step. Our goal is to obtain a policy that can generalize to be used for replay scheduling in new CL scenarios to mitigate catastrophic forgetting. Next, we describe in more detail how the framework for learning this policy works.

%By learning from gathering experience, the policy learns better for the future on how to select the tasks to replay. However, as 

\MK{Is this how it should be motivated?}
Imagine the scenario that we can collect experiences from many users applying their phone to CL scenarios for learning different objects to recognize sequentially. Assume that we can store the collected data (limitation here is privacy!), the models will suffer from catastrophic forgetting as they are trained in CL scenario. But we can then use the collected data to train a replay scheduling policy. The learned policy can then be transferred to new users using their phone in CL settings and the policy is used for mitigating catastrophic forgetting in their environment without additional computational cost. 

Limitations here are of course that we potentially need lots of data for learning a policy that generalizes. Additionally, we need lots of training time and hyperparameter tuning as we are dealing with RL. Also, we need to store the data somewhere which is cheap, but it must be secure due to privacy concerns. An alternative there could be to store features instead of raw data, which is not completely flawless (I think that it's possible to revert features back to the real data to some extent) but at least it is a safer alternative. Another option for the data needed can be to gather experience from simulated environments and benchmark datasets. As the policy only takes in states with task performances, we can make use mixes of benchmark datasets and data from real contributing users. 

  
\subsection{Problem Setting}

We consider the setting where an agent selects replay schedules to mitigate catastrophic forgetting in a classifier trained in the CL setting. The environment that the agent interacts with contains a network $f_{\vphi}$ and a dataset $\gD_{1:T}$ of $T$ tasks that $f_{\vphi}$ should learn in sequential order. The dataset is split into training, validation, and test sets as $\gD_{1:T} = \{\gD_{1:T}^{(train)}, \gD_{1:T}^{(val)}, \gD_{1:T}^{(test)}\}$ respectively. The training sets $\gD_{1:T}^{(train)}$ are for the network to learn all $T$ tasks sequentially, while the $\gD_{1:T}^{(val)}$ are for evaluating how well the network performs on each task during training. The task performances on the validation sets can be used for dense rewards to the RL agent. The test sets are for final evaluation and are unseen during training as standard practice to avoid overfitting.  

We consider having a distribution $p(E)$ of Markov Decision Process~\cite{bellman1957markovian} (MDP) where the MDP is represented as a tuple $E_i = (\gS_i, \gA, P_i, R_i, \mu_i, \gamma)$ consisting of the state space $\gS_i$, action space $\gA$, state transition probability $P_i(s' | s, a)$, reward function $R_i(s, a)$, initial state distribution $\mu_i(s_0)$, and discount factor $\gamma$. For training, we assume that we have access to a fixed set of training environments $\gE_{\text{train}} = \{E_1, \dots, E_K\}$, where $E_i \sim p(E)$ for $i=1, ..., K$. 





\section{Experiments}


\section{Discussion}



