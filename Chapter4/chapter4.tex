%*******************************************************************************
%*********************************** Fourth Chapter ****************************
%*******************************************************************************

\chapter{Continual Learning}
%\chapter{Summary of Included Papers}
%\label{chap:summary_of_included_papers}

This chapter introduces the idea of replay scheduling for mitigating catastrophic forgetting in continual learning (CL). The problem setting of CL is on learning tasks of recognizing a new set of classes with a dataset given at the current time step. In the standard setting, one main assumption is that the data from past tasks can never be fully revisited by the model. However, in the real-world, many organizations record data from incoming streams for storage rather than deleting it~\cite{bailis2017macrobase, mitchell1999machinelearning} [Add at least 1 more REF]. In contrast to the assumption on data storage in standard CL, we suggest a new setting where we assume that all seen data is accessible at any time for the model to revisit. The challenge then becomes how to select which tasks that needs to be remembered via replay as the data is still incoming from a stream. We propose to learn the time when replaying a certain task is necessary when the model is updating its knowledge with new incoming tasks. In Paper C, we propose the new CL setting where historical data is accessible and introduce the idea of replay scheduling and how it can be used in CL. In Paper D, we propose a framework based on reinforcement learning~\cite{sutton2018reinforcement} (RL) for learning replay scheduling policies that can be applied in new CL scenarios. 
%A simple, yet effcient, approach to mitigate catastrophically forgetting past tasks is to add some 
%Replay scheduling involves selecting which tasks to fetch examples from and add to a replay memory at different times. The replay memory is then mixed with a batch of examples from the current task to learn to help the network to remember the previously learned tasks selected by the scheduler. 

%\section{Introduction}
\section{Related Work}
% General CL papers, add new papers that you have seen. Then Replay/memory-based CL approaches, add new and contrastive approaches. Perhaps something on meta-policy learning approaches, would be useful to tell how we use reinforcement learning here

\section{Replay Scheduling in Continual Learning} 

In this section, we introduce a slightly new CL setting considering the real-world needs where all historical data can be available since data storage is cheap. However, the amount of compute is limited when the model is updated on new data due to operational costs. Hence, it is impossible for the model to leverage from all the available historical data to mitigate catastrophic forgetting. The goal then becomes to learn how we can select subsets of historical data for replay to efficiently reduce forgetting of the old tasks. We will refer to these subsets of historical data as the \textit{replay memory} throughout this chapter. The size of the replay memory affects the processing time when learning new tasks as well as the allowed time for the training phase. When composing the replay memory, we focus on determining the number of samples to draw from the seen tasks in the historical data rather than selecting single stored instances. Next, we introduce the problem setting in more detail as well as the notation of the new CL setting. 


\subsection{Problem Setting}



%CZ comment in reviews: Comparing our setting with the traditional continual learning setting, we share the fundamental setting that the data comes in streams and we cannot re-train on all historical data. Also, we share the common goal that the model should do well both in tasks with historical data and tasks associated with new data. In memory-based continual learning, we also share the same constraints that the memory size is limited where we argue that this limitation is mainly associated with compute instead of storage which aligns with the real-world where data storage is cheap and easy but retraining large ML models is computationally expensive where the time would not allow. The only difference is that we allow filling this limited memory from historical data or another external memory. Here, we argue that historical data are never thrown away in real-world settings. Thus, to make continual learning align with real-world needs, we should keep the limited memory assumption for training but allow access to historical data to fill this memory.




Here, we introduce the notation of our problem setting which resembles the traditional CL setting for image classification. We let a neural network $f_{\vtheta}$, parameterized by $\vtheta$, learn $T$ tasks sequentially from the datasets $\gD_1, \dots, \gD_T$ arriving one at a time. %given their corresponding task datasets $\gD_1, \dots, \gD_T$ arriving in order. 
The $t$-th dataset $\gD_t = \{(\vx_{t}^{(i)}, y_{t}^{(i)})\}_{i=1}^{N_{t}}$ consists of $N_t$ samples where $\vx_{t}^{(i)}$ and $y_{t}^{(i)}$ are the $i$-th data point and class label respectively. The training objective at task $t$ is given by 
%\vspace{-2mm}
\begin{align}
	\underset{\vtheta}{\text{min}} \sum_{i=1}^{N_t} \ell(f_{\vtheta}(\vx_t^{(i)}), y_{t}^{(i)}),
\end{align}
where $\ell(\cdot)$ is the loss function, e.g., cross-entropy loss in our case. 
When learning task $t$, the network $f_{\vtheta}$ is at risk of catastrophically forgetting the previous $t-1$ tasks.  
%Since $\gD_t$ is only accessible at time step $t$, the network $f_{\vtheta}$ is at risk of catastrophically forgetting the previous $t-1$ tasks when learning the current task. 
Replay-based methods mitigate forgetting historical tasks by storing old examples in an external replay memory, that is mixed with the current task dataset during training. 
%Replay-based continual learning methods mitigate the forgetting of old tasks by storing old examples in an external replay memory, that is mixed with the current task dataset during training. 
Next, we describe our method for constructing this replay memory.  

We assume that historical data from old tasks are accessible at any time step $t$. However, since the historical data is prohibitively large, we can only fill a small replay memory $\gM$ with $M$ historical samples for replay due to processing time constraints. 
The challenge is how to fill the replay memory with $M$ samples that efficiently retain the knowledge of old tasks when learning new tasks. We focus on selecting the samples on task-level by deciding on the task proportion $(a_1, \dots, a_{t-1})$ of samples to fetch from each task, where $a_{i} \geq 0$ is the proportion of $M$ examples from task $i$ to place in $\gM$ and $\sum_{i=1}^{t-1} a_i = 1$. Consequently, we need a method for choosing these task proportions of which old tasks to replay. 
To simplify this selection, we construct a discrete set of choices for possible task proportions that can be used for constructing the replay memory $\gM$.
%telling how many samples from each task to use when constructing the replay memory $\gM$.
%To simplify this selection, we construct a discrete set of choices for possible task proportions telling how many samples from each task to use when constructing the replay memory $\gM$. 

\section{Meta-Policy Learning for Replay Scheduling}
\section{Experiments}
\section{Discussion}



