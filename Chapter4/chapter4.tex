%*******************************************************************************
%*********************************** Fourth Chapter ****************************
%*******************************************************************************

\chapter{Continual Learning}\label{chap:continual_learning}


This chapter introduces the idea of replay scheduling for mitigating catastrophic forgetting in continual learning (CL). The problem setting of CL is on learning tasks of recognizing a new set of classes with a dataset given at the current time step. In the standard setting, one main assumption is that the data from past tasks can never be fully revisited by the model. However, in the real-world, many organizations record data from incoming streams for storage rather than deleting it~\cite{bailis2017macrobase, mitchell1999machinelearning} [Add at least 1 more REF]. In contrast to the assumption on data storage in standard CL, we suggest a new setting where we assume that all seen data is accessible at any time for the model to revisit. The challenge then becomes how to select which tasks that needs to be remembered via replay as the data is still incoming from a stream. We propose to learn the time when replaying a certain task is necessary when the model is updating its knowledge with new incoming tasks. In Paper \ref{chap:paperC}, we propose the new CL setting where historical data is accessible and introduce the idea of replay scheduling and how it can be used in CL. In Paper \ref{chap:paperD}, we propose a framework based on reinforcement learning~\cite{sutton2018reinforcement} (RL) for learning replay scheduling policies that can be applied in new CL scenarios. 
%A simple, yet effcient, approach to mitigate catastrophically forgetting past tasks is to add some 
%Replay scheduling involves selecting which tasks to fetch examples from and add to a replay memory at different times. The replay memory is then mixed with a batch of examples from the current task to learn to help the network to remember the previously learned tasks selected by the scheduler. 

%\MK{TO-DO: Motivate replay scheduling from mobile phone perspective, perhaps from perspective that phones can store lots of data but how to select which classes to replay. Maybe it should also be from the computational perspective, that we want such scheduling policy to work in many scenarios without additional compute cost for the policy learning. }


\section{Related Work}\label{ch4/sec:related_work}

In this section, we give an overview of previous works related to Paper C and D. We begin by describing different approaches in CL, especially replay-based approaches, and then discuss some approaches for fostering generalization in RL.

\vspace{-3mm}
\paragraph{Continual Learning.} There exist many different approaches in CL for mitigating catastrophic forgetting in neural networks. In general, these approaches can be divided into three main cores, namely, \textit{regularization-based}, \textit{architecture-based}, and \textit{replay-based} methods. Regularization-based methods are mainly focused on applying regularization techniques on parameters important for recognizing old tasks and fit the remaining parameters to new tasks~\cite{kirkpatrick2017overcoming, zenke2017continual, nguyen2017variational}. Knowledge distillation methods~\cite{hinton2015distilling} also belong to these approaches where classification logits are used for regularizing the output units for previous tasks in the network~\cite{li2017learning, schwarz2018progress}. More recently, there are some works that uses projection-based approaches for constraining the parameter updates to subspaces which avoid interference with previous tasks~\cite{saha2021gradient, kao2021natural}. Architecture-based approaches focuses on adding task-specific network modules for every seen task~\cite{rusu2016progressive, yoon2017lifelong, yoon2019scalable, ebrahimi2020adversarial}, or isolating parameters for predicting specific task in fixed-size networks~\cite{mallya2018packnet, serra2018overcoming, schwarz2021powerpropagation}. Replay-based methods re-trains on samples of old tasks when learning new tasks. The old samples are either stored in an external memory~\cite{chaudhry2019tiny, hayes2020remind, rolnick2018experience}, or synthesized with a generative model~\cite{shi2019variational, van2018generative, van2020brain, wu2018memory}. Both regularization- and architecture-based methods can be combined with replay for improving the models capability of remembering tasks [Add REFs]. Our replay scheduling idea is originated from replay-based methods which we will cover more in detail next. 

\vspace{-3mm}
\paragraph{Replay-based Continual Learning.} In this thesis, we focus on replay-based methods with external memories for storing historical data. The most common selection strategy for filling the memory is random sampling from the used datasets. There exist several works focusing on selecting high quality samples for storing in the memory [Add REFs]. However, in image classification problems, random sampling has been shown to often perform on par with more elaborate selection strategies~\cite{chaudhry2018riemannian, hayes2020remind}. In contrast to using various memory selection methods, there has been proposals of retrieval policies over which samples to select for replay from the memory, for instance, selecting the samples that will mostly interfere with the parameter update with batches of new data~\cite{aljundi2019online}. Our replay scheduling approach differs from this method as we focus on selecting which tasks to select for replay rather than the individual samples to retrieve from the memory.  
More recent works have focused on evolving the memory samples through data augmentation to avoid overfitting to the memory [Add REFs], and also by using contrastive learning to improve discriminating between tasks. Another direction has been to increasing the storage capacity to store more samples by compressing raw data into features that are more memory-cheap [Add REFs]. The above mentioned methods assume that the memory is small and allocates equal storage amount for all tasks. Our new problem setting for memory-based CL is different from this assumption as we argue that data storage is cheap in many real-world applications. Hence, we compose a replay memory with data from historical tasks before learning new tasks because the amount of compute is limited. However, replay scheduling can be combined with of the mentioned methods as it only differs with the standard memory-based CL setting in that the replay memory has to be selected at every new task. 

\vspace{-3mm}
\paragraph{Generalization in Reinforcement Learning.} Generalization in RL is an active research field where much focus has been put on developing proper benchmark datasets for evaluating generalization capabilities of RL agents~\cite{cobbe2019quantifying, cobbe2020leveraging, nichol2018gotta, yu2020meta}. 
Regularization techniques from supervised learning have been used to investigate whether these can enhance generalization in RL~\cite{cobbe2019quantifying, farebrother2018generalization, igl2019generalization}, and also how variations in the environments help to obtain agents that generalize better~\cite{packer2018assessing, zhang2018dissection}. Algorithms for enabling fast adaptation to new environments, such as new tasks~\cite{finn2017model, kessler2021same} and new action spaces~\cite{chandak2019learning, jain2020generalization}, has also been studied. The survey by Kirk \etal~\cite{kirk2021survey} focuses on zero-shot policy transfer where the policy must generalize to unseen dynamics in the test environments as additional queries could be disallowed in real-world scenarios~\cite{yang2019single, ball2021augmented}. 
In Paper \ref{chap:paperD}, we consider using RL for learning policies for scheduling which tasks to replay in CL settings to mitigate catastrophic forgetting in the classifier. The policy is trained using experiences from multiple CL environments and then tested in new CL environments with unseen dynamics such as new task orders and datasets. 
%We evaluate our learned policies on the zero-shot policy transfer setting where the goal is to mitigate catastrophic forgetting in new CL scenarios. The rewards are accessible through accuracies calculated on validation datasets, however, fine-tuning the policy during CL training is prohibited.  


% General CL papers, add new papers that you have seen. Then Replay/memory-based CL approaches, add new and contrastive approaches. Perhaps something on meta-policy learning approaches, would be useful to tell how we use reinforcement learning here

%TO-DO \MK{A short and cozy table fo this, similar to survey by Delange etal 2021}
%\paragraph{Similarities and Differences between Continual Learning and other fields.} \MK{A short and cozy table fo this, similar to survey by Delange etal 2021}

\section{Replay Scheduling in Continual Learning}\label{sec:replay_scheduling_in_cl}

In this section, we introduce a slightly new CL setting considering the real-world needs where all historical data can be available since data storage is cheap. However, the amount of compute is limited when the model is updated on new data due to operational costs. Hence, it is impossible for the model to leverage from all the available historical data to mitigate catastrophic forgetting. The goal then becomes to learn how we can select subsets of historical data for replay to efficiently reduce forgetting of the old tasks. We will refer to these subsets of historical data as the \textit{replay memory} throughout this chapter. The size of the replay memory affects the processing time when learning new tasks as well as the allowed time for the training phase. When composing the replay memory, we focus on determining the number of samples to draw from the seen tasks in the historical data rather than selecting single stored instances. Next, we introduce the problem setting in more detail as well as the notation of the new CL setting. 


\subsection{Problem Setting}

We introduce the notation of our problem setting which resembles the traditional CL setting for image classification. We let a neural network $f_{\vtheta}$, parameterized by $\vtheta$, learn $T$ tasks from the datasets $\gD_1, \dots, \gD_T$ arriving sequentially one at a time. The $t$-th dataset $\gD_t = \{(\vx_{t}^{(i)}, \vy_{t}^{(i)})\}_{i=1}^{N_{t}}$ consists of $N_t$ samples where $\vx_{t}^{(i)}$ and $\vy_{t}^{(i)}$ are the $i$-th data point and class label respectively. The training objective at task $t$ is given by 
\vspace{-2mm}
\begin{align}
	\underset{\vtheta}{\text{min}} \sum_{i=1}^{N_t} \gL(f_{\vtheta}(\vx_t^{(i)}), \vy_{t}^{(i)}),
\end{align}
where $\gL(\cdot)$ is the loss function, which in our case is the cross-entropy loss.  
When learning task $t$, the network $f_{\vtheta}$ is at risk of catastrophically forgetting the previous $t-1$ tasks. The forgetting effect shows as the decrease in task accuracy between time steps, for example, $A_{t, i} < A_{t, i-1}$ where $A_{t, i}$ is the accuracy for task $t$ at time step $i$. Replay-based methods mitigate catastrophic forgetting by storing a few number of examples from historical tasks in an external memory. The network $f_{\vtheta}$ is then allowed to fetch old examples from the memory and mix these with the current task dataset to remind itself about the previous tasks during training. 

In the new problem setting, we assume that historical data from old tasks are accessible at any time step. However, we can only fill a small replay memory $\gM$ with $M$ historical samples for replaying old tasks due to processing time constraints prohibiting re-using all historical data at the same time. The challenge then becomes how to know which tasks to include in the replay memory that efficiently retain the previous knowledge when learning new tasks. We decide to fill the replay memory with $M$ historical samples using sequence of task proportions $(p_1, \dots, p_{t-1})$ where $\sum_{i=1}^{t-1} p_i = 1$ and $p_{i} \geq 0$. The number of samples from task $i$ to place in $\gM$ is given by $p_i \cdot M$. In the next section, we introduce a method for selecting the task proportions of which old tasks to replay. 


\paragraph{Comparison to Traditional CL.} The new setting has several similarities to the traditional CL setting. Both settings share the fundamental setting that the data arrive in streams and re-training on all historical data is prohibited. Also, the goal that the model should perform well both historical tasks and tasks associated with new data remains the same. In replay-based CL, we also share the same constraints that the memory size is limited. However, we argue that this limitation is mainly associated with compute rather than of storage. Our assumption aligns with the real-world where data storage is cheap and easy to maintain, but retraining large machine learning models is computationally expensive. The only difference is that we allow filling the limited replay memory from historical data or some other external memory. Here, we argue that historical data is stored rather than deleted in many real-world settings~\cite{bailis2017macrobase}. Thus, we should keep the limited memory assumption for training but allow access to historical data to fill the replay memory to make CL align with real-world needs.
%\MK{TO-DO: this paragraph can be written in a more humble way, more like "this is something worth to investigate"}



\subsection{Replay Scheduling for Mitigating Catastrophic Forgetting}

In this section, we describe our replay scheduling method for selecting the replay memory at different time steps. A replay schedule is defined as a sequence $S = (\vp_1, \dots, \vp_{T-1})$, where $\vp_i = (p_1, \dots, p_{T-1})$ for $1 \leq i \leq T-1$ is the sequence of task proportions for determining how many samples per task to fill the replay memory with at time step $i$. To make the selection of task proportions tractable, we construct an action space with a discrete number of choices for the task proportions from historical tasks. 

We show the procudre for creating the discrete action space in Algorithm \ref{alg:action_space_discretization}. At task $i$, we have $i-1$ historical tasks that we can choose from. We then generate all possible bin vectors $\vb_i = [b_1, \dots, b_{i}] \in \gB_i$ of size $i$ where each element are a task index $1, ..., i$. We sort all bin vectors by the order of task indices and only keep the unique bin vectors. For example, at $i=2$, the unique choices of vectors are $[1,1], [1,2], [2,2]$, where $[1,1]$ indicates that all samples in the replay memory should be from task 1, $[1,2]$ indicates that half memory is from task 1 and the other half are from task etc. The task proportions are then computed by counting the number of occurrences of each task index in $\vb_i$ and dividing by $i$, such that $\vp_i = \texttt{bincount}(\vb_i) / (i)$. From this specification, we have built a tree $\gT$ with different task proportions that can be selected at different time steps. We construct a replay schedule $S$ by traversing through $\gT$ and select a task proportion on every level to append to $S$. We can then evaluate the replay schedule $S$ by training a network on the CL task sequence and use $S$ to compose the replay memory to use for mitigating catastrophic forgetting at every task. %\MK{Question, Wrap algorithm around this paragraph and remove the comments? Algorithm with comments could be placed in Appendix of Paper D.}


\begin{algorithm}[t]
	\caption{Discretization of action space with task proportions}
	\label{alg:action_space_discretization}
	\begin{algorithmic}[1]
		\Require Number of tasks $T$
		\State $\gT = ()$ \Comment{Initialize sequence for storing actions}
		\For{$i = 1, \dots, T-1$}
		\State $\gP_i = \{\}$ \Comment{Set for storing task proportions at $i$}
		\State $\gB = \texttt{combinations}([1:i], i)$ \Comment{Get bin vectors of size $i$ with bins $1, ..., i$}
		\State $\bar{\gB} = \texttt{unique}(\texttt{sort}(\gB))$ \Comment{Only keep unique bin vectors}
		\For{$\vb_i \in \hat{\gB}$}
		\State $\vp_i = \texttt{bincount}(\vb_i) / i$ \Comment{Calculate task proportion}
		\State $\gP_i = \gP_i \cup \{ \vp_i \}$ \Comment{Add task proportion to set}
		\EndFor
		\State $\gT[i] = \gP_i$ \Comment{Add set of task proportions to action sequence}
		\EndFor
		\State \Return $\gT$ \Comment{Return action sequence as discrete action space}
	\end{algorithmic}
\end{algorithm}



We are interested in studying whether the time to replay different tasks is important in the new CL setting. One option is to use a brute-force approach, for example, breadth-first search, and evaluate every possible replay schedule in the tree. However, as the tree grows fast with the number of tasks, we need a scalable method that can perform searches in large action spaces. We suggest using Monte Carlo tree search~\cite{coulom2006efficient} (MCTS) due to its previous successes in applications with similar conditions as ours~\cite{browne2012survey, silver2016mastering, chaudhry2018feature}. The use of MCTS enables performing search for datasets with longer task horizons. Furthermore, MCTS encourages searches in promising paths based on the selected reward function, which we set as the average accuracy of all tasks after learning the final task achieved by the network. We provide the full details on how MCTS is used to search for replay schedules in Paper \ref{chap:paperC}. 
%\MK{Question: More info on MCTS?}


%% Have section for MCTS. include a limitations paragraph there that can be referred to later in Discussion. It kind of bridges over well to the meta-policy learning too. Say that we will use MCTS for studying if scheduling is important in this new setting.


\section{Meta Policy Learning for Replay Scheduling}
\label{sec:meta_policy_learning_for_replay_scheduling}

In this section, we present an RL-based framework to learn policies for selecting which tasks to replay in CL scenarios. We are interested in learning such policy that can be transferred to new CL scenarios, such as new task orders and new datasets, without any additional computational cost for updating on the new domain. We take a meta-learning approach where the policy learns from episodes of experience collected from training a classifier in CL settings. The experience from the environment is represented as the classification performance on each seen task in the dataset. The policy receives the task performances for basing its action on which task that needs to be replayed at the next time step. Our goal is to obtain a policy that can generalize to be used for replay scheduling in new CL scenarios to mitigate catastrophic forgetting. Next, we describe in more detail how the framework for learning this policy works.

%By learning from gathering experience, the policy learns better for the future on how to select the tasks to replay. However, as 

\MK{Is this how it should be motivated?}
Imagine the scenario that we can collect experiences from many users applying their phone to CL scenarios for learning different objects to recognize sequentially. Assume that we can store the collected data (limitation here is privacy!), the models will suffer from catastrophic forgetting as they are trained in CL scenario. But we can then use the collected data to train a replay scheduling policy. The learned policy can then be transferred to new users using their phone in CL settings and the policy is used for mitigating catastrophic forgetting in their environment without additional computational cost. 

%Limitations here are of course that we potentially need lots of data for learning a policy that generalizes. Additionally, we need lots of training time and hyperparameter tuning as we are dealing with RL. Also, we need to store the data somewhere which is cheap, but it must be secure due to privacy concerns. An alternative there could be to store features instead of raw data, which is not completely flawless (I think that it's possible to revert features back to the real data to some extent) but at least it is a safer alternative. Another option for the data needed can be to gather experience from simulated environments and benchmark datasets. As the policy only takes in states with task performances, we can make use mixes of benchmark datasets and data from real contributing users. 

  
\subsection{Problem Setting}

We consider the setting where an agent selects replay schedules to mitigate catastrophic forgetting in a classifier trained in the CL setting. The environment that the agent interacts with contains a network $f_{\vphi}$ and a dataset $\gD_{1:T}$ of $T$ tasks that $f_{\vphi}$ should learn in sequential order. The dataset is split into training, validation, and test sets as $\gD_{1:T} = \{\gD_{1:T}^{\train}, \gD_{1:T}^{\val}, \gD_{1:T}^{\test}\}$ respectively. The training sets $\gD_{1:T}^{\train}$ are for the network to learn all $T$ tasks sequentially, while the $\gD_{1:T}^{\val}$ are for evaluating how well the network performs on each task during training. The task performances on the validation sets can be used for dense rewards to the RL agent. The test sets are for final evaluation and are unseen during training as standard practice to avoid overfitting.  

We consider having a distribution $p(E)$ of Markov Decision Process~\cite{bellman1957markovian} (MDP) where the MDP is represented as a tuple $E_i = (\gS_i, \gA, P_i, R_i, \mu_i, \gamma)$ consisting of the state space $\gS_i$, action space $\gA$, state transition probability $P_i(s' | s, a)$, reward function $R_i(s, a)$, initial state distribution $\mu_i(s_0)$, and discount factor $\gamma$. For training, we assume that we have access to a fixed set of training environments $\gE_{\train} = \{E_1, \dots, E_K\}$, where $E_i \sim p(E)$ for $i=1, ..., K$. Each environment $E_i$ contains of a network $f_{\vphi}$ and $T$ datasets $\gD_{1:T}$ where the $t$-th dataset is learned at time step $t$. We let the dynamics of the environments depend on the network initialization and the task order of the datasets. The states $s$ are given by the validation performance on each task from $f_{\vphi}$ which can be accuracies. Hence, the state space $\gS_i$ depends on the parameter initialization for $\vphi$. Regarding the datasets, we allow the task orders in the datasets $\gD_{1:T}$ to be shuffled for each sampled environment $E_i \sim p(E)$. Therefore, the state space is also affected by the task order as this can yield different CL performances. We assume that the action space is the same for every environment as the agent will interact with each training environment in $\gE_{\train}$. 

The goal for the agent is to maximize the accumulated returns in each training environment. The reward is given by the CL performance. We assume that we have a dense reward that we can compute from the validation sets at each time step $t$. We are therefore after a policy that works well on all environments. The intention for learning such policy is that it should generalize across environments of new CL settings, such as new task orders and datasets. As we are not allowed to go back in time in CL, we are not allowed to query the testing environments during the transfer. Hence we must apply the policy in a zero-shot setting where tuning the policy on the testing environment is prohibited.


\subsection{Deep Q-Networks for Learning Replay Scheduling Policy}

We employ DQNs for learning the policy for replay scheduling. The architecture takes states as input and outputs action values for the valid actions at the current time step $t$. We define the states $s$ as a vector with the validation performance of every seen task, where we use the validation accuracy to represent the task performance. We use zero-padding on the vector elements in the states that represent the performance for future tasks. The states can also be represented by other performance metrics, for example, the accuracy difference between time steps or forgetting measures such as backward transfer~\cite{lopez2017gradient} (BWT). For the actions, we use the discrete action space built using Algorithm \ref{alg:action_space_discretization} such that the action space $\gA_t$ is time-dependent and grows per seen task. To handle the growing action space, we assume that we know the total number of actions such that the DQN can use a fixed output layer. Therefore, we use action masking on the predicted action values to prevent the DQN from selecting invalid actions. 

\begin{algorithm}[t]
	\caption{Learning replay scheduling policy with DQN}
	\label{alg:learning_replay_scheduling_policy_with_dqn}
	\begin{algorithmic}[1]
		\Require $\gE_{\train}$: Training environments
		\Require $\vtheta$: DQN parameters
		\State $\gB = \{\}$ \Comment{Initialize replay buffer}
		\For{$i = 1, \dots, n_{\text{episodes}}$}
		\State $s_1^{i} \sim \mu_i \, \forall E_i \in \gE_{\train}$ \Comment{Get initial states}
		\For{$t=1, \dots, T-1$}
		\For{$E_i \in \gE_{\train}$}
		\State $a_t^{i} = \argmax_{a' \in \gA_t} Q_{\vtheta}(s_t^{i}, a')$ \Comment{Get actions from states}
		\State $r_t^{i}, s_{t+1}^{i} = \texttt{CLStep}(t, a_t^{i}, E_i)$
		\State $\gB = \gB \cup \{(s_{t}^{i}, a_{t}^{i}, r_{t}^{i}, s_{t+1}^{i})\}$ \Comment{Store transition in buffer}
		\State $(s_j, a_j, r_j, s_{j+1}) \stackrel{B}{\sim} \gB$ \Comment{Get mini-batch of $B$ samples from buffer}					
		%\State $\{(s_t^{(j)}, a_t^{(j)}, r_t^{(j)}, s_{t+1}^{(j)})\}_{j=1}^{B} \stackrel{B}{\sim} \gB$
		\State $\vtheta \leftarrow \vtheta - \eta \nabla_{\vtheta} (r_t + \max_{a'} Q_{\vtheta^{-}}(s_{j+1}, a') - Q_{\vtheta}(s_j, a_j))$ \Comment{Update DQN}
		\EndFor
		\EndFor 
		\EndFor
		\State \Return $\vtheta$ \Comment{Return DQN}
		
		\Statex
		
		\Function{\texttt{CLStep}}{$t$, $a_{t}$, $E$}
		\State $\gM_t \leftarrow \texttt{SampleMemory}(a_t)$ \Comment{Sample historical data from action}
		\State $f_{\vphi}^{E} \leftarrow \texttt{Train}(f_{\vphi}^{E}, \gD_{t+1}^{\train}, \gM_t)$ \Comment{Train network on current task with replay}
		\State $r_t \leftarrow \frac{1}{t+1} \sum_{i=1}^{t+1} A_{i}^{\val}$ \Comment{Compute reward from validation datasets}
		\State $s_{t+1} \leftarrow \texttt{GetState}(f_{\vphi}^{E}, \gD_{1:t+1}^{\val})$ \Comment{Get next state from validation performance}
		\State \Return $r_t$, $s_{t+1}$ \Comment{Return reward and next state}
		\EndFunction
	\end{algorithmic}
\end{algorithm}

We outline the procedure for training the DQN on multiple CL environments in Algorithm \ref{alg:learning_replay_scheduling_policy_with_dqn}. The main idea to obtain a policy that generalize is to train it on several environments with different dynamics for learning from more diverse training data~\cite{zhang2018dissection}. We let the DQN store transitions $(s_{t}^{i}, a_{t}^{i}, r_{t}^{i}, s_{t+1}^{i})$ from all environments in the same experience replay buffer $\gB$ to use for training the network. The first step is to receive the intial state $s_1$ for all environments by training the network $f_{\vphi}$ on the first dataset $\gD_1^{\train}$ in each environment, such that $s_{1}^{i} = [A_{1,1}^{\val}, 0, ..., 0]$. The DQN then selects the action $a_t$ for the current environment from the input state or selects a random action with probability $\epsilon$. The agent retrieves the reward and next state by training the environment-specific network $f_{\vphi}^{i}$ on the current task $t$ in \texttt{CLStep}. Before training starts, the action $a_{t}^{i}$ is used for sampling the replay memory $\gM_t$ using the task proportions $(\vp_1, \dots, \vp_{T-1})$ translated from $a_{t}^{i}$. The replay memory is then used for mitigating catastrophic forgetting when training on the current task. After training, the agent obtains the reward from the reward function 
\begin{align}
	R(s, a) = \frac{1}{t} \sum_{i=1}^{t} A_{t, i}^{\val},
\end{align}
where $A_{t, i}^{\val}$ is the validation accuracy on task $i$ after $f_{\vphi}$ has learned task $t$. The next state is also obtained by utilizing the validation accuracies up to task $t$, such as $s_{t+1}^{i} = [A_{t, 1}^{\val}, ..., A_{t, t}^{\val}, 0, ..., 0] \in [0, 1]^{T-1}$. The DQN is then updated by sampling a mini-batch of $B$ samples from the shared buffer $\gB$ and taking a gradient step to minimize the loss
\begin{align}
	\gL_{\text{DQN}}(\vtheta) = r_j + \max_{a'} Q_{\vtheta^{-}}(s_{j+1}, a') - Q_{\vtheta}(s_j, a_j), 
\end{align}  
where $\vtheta^{-}$ is target network which is copy of the parameters $\vtheta$ from some previous time step. By using the shared replay buffer, the DQN is trained on diverse set of environments that can generalize well to new environments with unseen dynamics from an unseen task order or a new dataset. Next, we will describe the experiments we perform to evaluate the generalization capability of the policy to new CL scenarios. 



\section{Experiments}

In this section, we give an overview of the experimental results in Paper \ref{chap:paperC} and \ref{chap:paperD}. In Paper \ref{chap:paperC}, we perform experiments to show that the importance of replay scheduling in our proposed problem setting for CL described in Section \ref{sec:replay_scheduling_in_cl}. We demonstrate that learning when to replay different tasks by using MCTS as an exemplar method for finding good replay schedules that are efficient in mitigating catastrophic forgetting in the classifier. In Paper \ref{chap:paperD}, we demonstrate on benchmark datasets for CL how our RL-based framework can be used for learning replay scheduling policies. We show that our proposed method can learn policies that generalize to CL scenarios with new task orders and datasets unseen during training.  

\vspace{-3mm}
\paragraph{Datasets and Architectures.} In Paper \ref{chap:paperC} and \ref{chap:paperD}, we perform experiments on common CL benchmark datasets with 5 tasks such as Split MNIST~\cite{zenke2017continual}, Split Fashion-MNIST~\cite{xiao2017fashion}, Split notMNIST~\cite{bulatov2011notMNIST}, and Split CIFAR10~\cite{krizhevsky2009learning}, 10-task Permuted MNIST~\cite{goodfellow2013empirical}, and 20-task Split CIFAR-100~\cite{krizhevsky2009learning, lopez2017gradient, rebuffi2017icarl} and Split miniImagenet~\cite{vinyals2016matching}. We use a 2-layer MLP with 256 hidden units and ReLU activation for Split MNIST, Split FashionMNIST, Split notMNIST, and Permuted MNIST. We use a multi-head output layer for each dataset except Permuted MNIST where the network uses single-head output layer. For Split CIFAR-10 and CIFAR-100, we use a multi-head CNN architecture built according to the CNN in \cite{adel2019continual, schwarz2018progress, vinyals2016matching}. For Split mniImagenet, we use the reduced ResNet-18 from \cite{lopez2017gradient} with multi-head output layer. See each paper for training details and hyperparameter settings. 



\vspace{-3mm}
\paragraph{Baselines.} We compare our methods with several scheduling baselines to verify the importance of learning replay schedules. The baselines we compare against are
\begin{itemize}[itemsep=0em,topsep=1pt]
	\item {\bf Random Schedule}: Randomly select which tasks to replay.
	\item {\bf Equal Task Schedule (ETS)}: Replay all seen tasks equally.
	\item {\bf Heuristic Schedule}: Replay tasks which validation performance is below a tuned threshold found through hyperparameter searches.
\end{itemize} 

\vspace{-3mm}
\paragraph{Evaluation.} The evaluation metric we use for measuring the CL performance is defined as
\begin{align}
	\ACC = \frac{1}{T} \sum_{j=1}^{T} A_{T, j}^{\test},
\end{align}
where $A_{T, j}^{\test}$ is the accuracy of task $j$ after learning the final task $T$.  In Paper \ref{chap:paperD}, we use a ranking method for comparing the ACC performance of each scheduling method within the testing environments, since the performance between environments can differ significantly when the task order changes. We then average all obtained ranking lists over the number of testing environments to obtain the ranking metric. 


\subsection{Replay Scheduling Using Monte Carlo Tree Search}

In this section, we provide an overview of the experimental results from Paper \ref{chap:paperC}. We have named our approach as Replay Scheduling MCTS (RS-MCTS) below. Firstly, we evaluate the CL performance of RS-MCTS and the baselines on experiments with varying memory sizes. Secondly, we add replay scheduling to three recent CL methods to show that scheduling is applicable to any memory-based method. Finally, we demonstrate the efficiency benefits of replay scheduling in settings when the memory size is smaller than the number of classes. See Paper \ref{chap:paperC} for the full experimental results and settings. 
%We show that our method can improve the performance across different choices of memory size.
%In this experiment, we set the replay memory size to ensure the ETS baseline replays an equal number of samples per class at the final task. 
%The memory size is set to $M = n_{cpt} \cdot n_{spc} \cdot (T-1)$, where $n_{cpt}$ is the number of classes per task in the dataset and $n_{spc}$ are the number of samples per class we wish to replay at task $T$ for the ETS baseline. 
%In Figure \ref{fig:acc_over_replay_memory_size}, we observe that RS-MCTS obtains better task accuracies than ETS, especially for small memory sizes. Both RS-MCTS and ETS perform better than Heuristic as $M$ increases showing that Heuristic requires careful tuning of the validation accuracy threshold. 


\vspace{-3mm}
\begin{wrapfigure}{r}{0.5\textwidth}
	\centering
	\setlength{\figwidth}{0.36\textwidth}
	\setlength{\figheight}{.16\textheight}
	\vspace{-3mm}
	\hspace{-11mm}
	\resizebox{0.58\textwidth}{!}{
	


\pgfplotsset{every axis title/.append style={at={(0.5,0.80)}}} 
\pgfplotsset{every tick label/.append style={font=\tiny}}
\pgfplotsset{every major tick/.append style={major tick length=2pt}}
\pgfplotsset{every minor tick/.append style={minor tick length=1pt}}
\pgfplotsset{every axis x label/.append style={at={(0.5,-0.22)}}}
\pgfplotsset{every axis y label/.append style={at={(-0.17,0.5)}}}
´

\begin{tikzpicture}
\tikzstyle{every node}=[font=\scriptsize]
\definecolor{color0}{rgb}{0.12156862745098,0.466666666666667,0.705882352941177}
\definecolor{color1}{rgb}{1,0.498039215686275,0.0549019607843137}
\definecolor{color2}{rgb}{0.172549019607843,0.627450980392157,0.172549019607843}
\definecolor{color3}{rgb}{0.83921568627451,0.152941176470588,0.156862745098039}

\begin{groupplot}[group style={group size= 2 by 2, horizontal sep=0.7cm, vertical sep=0.8cm}]

\nextgroupplot[title=Split MNIST,
height=\figheight,
legend cell align={left},
legend columns=4,
legend style={
  nodes={scale=0.80},
  fill opacity=0.8,
  draw opacity=1,
  text opacity=1,
  at={(0.00,1.30)}, %{(3.70,-0.30)},   %{(3.72,0.13)},
  anchor=south west,
  draw=white!80!black
},
minor xtick={},
minor ytick={},
tick align=outside,
tick pos=left,
width=\figwidth,
x grid style={white!69.0196078431373!black},
xmajorgrids,
%xlabel={Memory size \(\displaystyle M\)},
xmin=0.7, xmax=7.3,
xtick style={color=black},
xtick={1,2,3,4,5,6,7},
xticklabels={8,24,80,120,200,400,800},
y grid style={white!69.0196078431373!black},
ymajorgrids,
ylabel={ACC (\%)},
ymin=0.939, ymax=0.997729043511215,
ytick style={color=black},
ytick={0.92,0.93,0.94,0.95,0.96,0.97,0.98,0.99,1},
yticklabels={92,93,94,95,96,97,98,99,100}
]
\path [fill=color0, fill opacity=0.2, line width=1pt]
(axis cs:1,0.92504620552063)
--(axis cs:1,0.981783270835876)
--(axis cs:2,0.981537520885468)
--(axis cs:3,0.988991677761078)
--(axis cs:4,0.986135482788086)
--(axis cs:5,0.991755723953247)
--(axis cs:6,0.989374697208405)
--(axis cs:7,0.994002878665924)
--(axis cs:7,0.976767957210541)
--(axis cs:7,0.976767957210541)
--(axis cs:6,0.979170620441437)
--(axis cs:5,0.964531421661377)
--(axis cs:4,0.978874921798706)
--(axis cs:3,0.979120314121246)
--(axis cs:2,0.937642157077789)
--(axis cs:1,0.92504620552063)
--cycle;

\path [fill=color1, fill opacity=0.2, line width=1pt]
(axis cs:1,0.946015386753438)
--(axis cs:1,0.970621573246562)
--(axis cs:2,0.984533286206628)
--(axis cs:3,0.988692537329328)
--(axis cs:4,0.990023216819651)
--(axis cs:5,0.990669279851693)
--(axis cs:6,0.992800317892538)
--(axis cs:7,0.994021984494329)
--(axis cs:7,0.992872175505671)
--(axis cs:7,0.992872175505671)
--(axis cs:6,0.990845362107462)
--(axis cs:5,0.986030240148307)
--(axis cs:4,0.986402863180349)
--(axis cs:3,0.980789062670673)
--(axis cs:2,0.972900553793372)
--(axis cs:1,0.946015386753438)
--cycle;

\path [fill=color2, fill opacity=0.2, line width=1pt]
(axis cs:1,0.953295697386423)
--(axis cs:1,0.969893182613577)
--(axis cs:2,0.989900327489411)
--(axis cs:3,0.987615676138068)
--(axis cs:4,0.987468694435405)
--(axis cs:5,0.987904186526891)
--(axis cs:6,0.990390603540462)
--(axis cs:7,0.991611458769248)
--(axis cs:7,0.983504221230752)
--(axis cs:7,0.983504221230752)
--(axis cs:6,0.978347876459538)
--(axis cs:5,0.975166293473109)
--(axis cs:4,0.979932985564595)
--(axis cs:3,0.979642883861932)
--(axis cs:2,0.966454952510589)
--(axis cs:1,0.953295697386423)
--cycle;

\path [fill=color3, fill opacity=0.2, line width=1pt]
(axis cs:1,0.976848125457764)
--(axis cs:1,0.986456036567688)
--(axis cs:2,0.988981068134308)
--(axis cs:3,0.992213249206543)
--(axis cs:4,0.991635799407959)
--(axis cs:5,0.991742670536041)
--(axis cs:6,0.992980241775513)
--(axis cs:7,0.994612276554108)
--(axis cs:7,0.993073880672455)
--(axis cs:7,0.993073880672455)
--(axis cs:6,0.992027640342712)
--(axis cs:5,0.990043103694916)
--(axis cs:4,0.989341855049133)
--(axis cs:3,0.989036798477173)
--(axis cs:2,0.982677280902863)
--(axis cs:1,0.976848125457764)
--cycle;

\addplot [line width=1.0pt, color0, mark=*, mark size=1, mark options={solid}]
table {%
1 0.953414738178253
2 0.959589838981628
3 0.984055995941162
4 0.982505202293396
5 0.978143572807312
6 0.984272658824921
7 0.985385417938232
};
\addlegendentry{Random}
\addplot [line width=1.0pt, color1, mark=*, mark size=1, mark options={solid}]
table {%
1 0.95831848
2 0.97871692
3 0.9847408
4 0.98821304
5 0.98834976
6 0.99182284
7 0.99344708
};\addlegendentry{ETS}
\addplot [line width=1.0pt, color2, mark=*, mark size=1, mark options={solid}]
table {%
1 0.96159444
2 0.97817764
3 0.98362928
4 0.98370084
5 0.98153524
6 0.98436924
7 0.98755784
};
\addlegendentry{Heuristic}
\addplot [line width=1.0pt, color3, mark=*, mark size=1, mark options={solid}]
table {%
1 0.981652081012726
2 0.985829174518585
3 0.990625023841858
4 0.990488827228546
5 0.990892887115479
6 0.992503941059113
7 0.993843078613281
};\addlegendentry{Ours}




%
\nextgroupplot[title=Split FashionMNIST,
height=\figheight,
legend cell align={left},
legend style={
  nodes={scale=0.7},
  fill opacity=0.8,
  draw opacity=1,
  text opacity=1,
  at={(0.48,0.03)},
  anchor=south west,
  draw=white!80!black
},
minor xtick={},
minor ytick={0.89, 0.91, 0.93, 0.95, 0.97, 0.99},
tick align=outside,
tick pos=left,
width=\figwidth,
x grid style={white!69.0196078431373!black},
xmajorgrids,
%xlabel={Memory size \(\displaystyle M\)},
xmin=0.7, xmax=7.3,
xtick style={color=black},
xtick={1,2,3,4,5,6,7},
xticklabels={8,24,80,120,200,400,800},
y grid style={white!69.0196078431373!black},
ymajorgrids,
yminorgrids,
%ylabel={ACC},
ymin=0.899,%0.891766529814355,
ymax=0.996494715797058,
ytick style={color=black},
ytick={0.88,0.9,0.92,0.94,0.96,0.98,1},
yticklabels={88, 90, 92, 94, 96, 98, 100}
]
\path [fill=color0, fill opacity=0.2, line width=1pt]
(axis cs:1,0.96419358253479)
--(axis cs:1,0.976246356964111)
--(axis cs:2,0.984048008918762)
--(axis cs:3,0.98274040222168)
--(axis cs:4,0.982234418392181)
--(axis cs:5,0.985867142677307)
--(axis cs:6,0.985980749130249)
--(axis cs:7,0.988571465015411)
--(axis cs:7,0.980668723583221)
--(axis cs:7,0.980668723583221)
--(axis cs:6,0.972739219665527)
--(axis cs:5,0.975212812423706)
--(axis cs:4,0.954325616359711)
--(axis cs:3,0.976219773292542)
--(axis cs:2,0.962551951408386)
--(axis cs:1,0.96419358253479)
--cycle;

\path [fill=color1, fill opacity=0.2, line width=1pt]
(axis cs:1,0.965724160568608)
--(axis cs:1,0.979275839431391)
--(axis cs:2,0.983843648420532)
--(axis cs:3,0.987767732323772)
--(axis cs:4,0.988072744085454)
--(axis cs:5,0.988506841180757)
--(axis cs:6,0.989178836422143)
--(axis cs:7,0.990307375485654)
--(axis cs:7,0.989052624514346)
--(axis cs:7,0.989052624514346)
--(axis cs:6,0.987461163577857)
--(axis cs:5,0.983413158819243)
--(axis cs:4,0.982567255914546)
--(axis cs:3,0.970792267676228)
--(axis cs:2,0.972876351579468)
--(axis cs:1,0.965724160568608)
--cycle;

\path [fill=color2, fill opacity=0.2, line width=1pt]
(axis cs:1,0.955000050761585)
--(axis cs:1,0.970759949238415)
--(axis cs:2,0.98603178628668)
--(axis cs:3,0.981832320290786)
--(axis cs:4,0.983967730086771)
--(axis cs:5,0.978253058795492)
--(axis cs:6,0.991734343706935)
--(axis cs:7,0.969833098095523)
--(axis cs:7,0.896526901904478)
--(axis cs:7,0.896526901904478)
--(axis cs:6,0.916425656293065)
--(axis cs:5,0.938266941204508)
--(axis cs:4,0.921192269913229)
--(axis cs:3,0.942567679709214)
--(axis cs:2,0.92920821371332)
--(axis cs:1,0.955000050761585)
--cycle;

\path [fill=color3, fill opacity=0.2, line width=1pt]
(axis cs:1,0.982129275798798)
--(axis cs:1,0.984390676021576)
--(axis cs:2,0.987218379974365)
--(axis cs:3,0.988464534282684)
--(axis cs:4,0.989360094070435)
--(axis cs:5,0.989464402198792)
--(axis cs:6,0.989146828651428)
--(axis cs:7,0.990860760211945)
--(axis cs:7,0.988419115543365)
--(axis cs:7,0.988419115543365)
--(axis cs:6,0.98689329624176)
--(axis cs:5,0.987175583839417)
--(axis cs:4,0.985639929771423)
--(axis cs:3,0.983615458011627)
--(axis cs:2,0.982341647148132)
--(axis cs:1,0.982129275798798)
--cycle;
\addplot [line width=1.0pt, color0, mark=*, mark size=1, mark options={solid}]
table {%
1 0.970219969749451
2 0.973299980163574
3 0.979480087757111
4 0.968280017375946
5 0.980539977550507
6 0.979359984397888
7 0.984620094299316
};
%\addlegendentry{Random}
\addplot [line width=1.0pt, color1, mark=*, mark size=1, mark options={solid}]
table {%
1 0.9725
2 0.97836
3 0.97928
4 0.98532
5 0.98596
6 0.98832
7 0.98968
};
%\addlegendentry{ETS}
\addplot [line width=1.0pt, color2, mark=*, mark size=1, mark options={solid}]
table {%
1 0.96288
2 0.95762
3 0.9622
4 0.95258
5 0.95826
6 0.95408
7 0.93318
};
%\addlegendentry{Heuristic}
\addplot [line width=1.0pt, color3, mark=*, mark size=1, mark options={solid}]
table {%
1 0.983259975910187
2 0.984780013561249
3 0.986039996147156
4 0.987500011920929
5 0.988319993019104
6 0.988020062446594
7 0.989639937877655
};
%\addlegendentry{Ours}



%\nextgroupplot[title=Split notMNIST,
height=\figheight,
legend cell align={left},
legend style={
  nodes={scale=0.7},
  fill opacity=0.8,
  draw opacity=1,
  text opacity=1,
  at={(0.48,0.03)},
  anchor=south west,
  draw=white!80!black
},
minor xtick={},
minor ytick={0.91, 0.93, 0.95, 0.97},
tick align=outside,
tick pos=left,
width=\figwidth,
x grid style={white!69.0196078431373!black},
xmajorgrids,
%xlabel={Memory size \(\displaystyle M\)},
xmin=0.7, xmax=7.3,
xtick style={color=black},
xtick={1,2,3,4,5,6,7},
xticklabels={8,24,80,120,200,400,800},
y grid style={white!69.0196078431373!black},
ymajorgrids,
yminorgrids,
%ylabel={ACC},
ymin=0.899, ymax=0.968037298370892,
ytick style={color=black},
ytick={0.9,0.92,0.94,0.96,0.98},
yticklabels={90, 92, 94, 96, 98}
]
\path [fill=color0, fill opacity=0.2, line width=1pt]
(axis cs:1,0.898027122020721)
--(axis cs:1,0.932646334171295)
--(axis cs:2,0.949522912502289)
--(axis cs:3,0.941804111003876)
--(axis cs:4,0.945932507514954)
--(axis cs:5,0.944690823554993)
--(axis cs:6,0.958006739616394)
--(axis cs:7,0.963123440742493)
--(axis cs:7,0.937735080718994)
--(axis cs:7,0.937735080718994)
--(axis cs:6,0.941412806510925)
--(axis cs:5,0.925552010536194)
--(axis cs:4,0.917471051216125)
--(axis cs:3,0.924349844455719)
--(axis cs:2,0.907234847545624)
--(axis cs:1,0.898027122020721)
--cycle;

\path [fill=color1, fill opacity=0.2, line width=1pt]
(axis cs:1,0.905642837956127)
--(axis cs:1,0.936302282043873)
--(axis cs:2,0.945930444061739)
--(axis cs:3,0.952986088925952)
--(axis cs:4,0.959824157530047)
--(axis cs:5,0.952829054017519)
--(axis cs:6,0.959656752670672)
--(axis cs:7,0.962564882794123)
--(axis cs:7,0.947773837205876)
--(axis cs:7,0.947773837205876)
--(axis cs:6,0.948538687329328)
--(axis cs:5,0.942894385982481)
--(axis cs:4,0.946525122469953)
--(axis cs:3,0.939538551074048)
--(axis cs:2,0.917796675938261)
--(axis cs:1,0.905642837956127)
--cycle;

\path [fill=color2, fill opacity=0.2, line width=1pt]
(axis cs:1,0.92575112968193)
--(axis cs:1,0.94785415031807)
--(axis cs:2,0.955205714329457)
--(axis cs:3,0.958007719572493)
--(axis cs:4,0.952085855254064)
--(axis cs:5,0.955086757682649)
--(axis cs:6,0.947191690371997)
--(axis cs:7,0.964993528242304)
--(axis cs:7,0.936188951757696)
--(axis cs:7,0.936188951757696)
--(axis cs:6,0.909628229628003)
--(axis cs:5,0.922580682317351)
--(axis cs:4,0.916396464745937)
--(axis cs:3,0.906849720427507)
--(axis cs:2,0.904118125670543)
--(axis cs:1,0.92575112968193)
--cycle;

\path [fill=color3, fill opacity=0.2, line width=1pt]
(axis cs:1,0.939071178436279)
--(axis cs:1,0.951526403427124)
--(axis cs:2,0.958871364593506)
--(axis cs:3,0.954144775867462)
--(axis cs:4,0.952358484268188)
--(axis cs:5,0.95714259147644)
--(axis cs:6,0.956836819648743)
--(axis cs:7,0.9615877866745)
--(axis cs:7,0.9475919008255)
--(axis cs:7,0.9475919008255)
--(axis cs:6,0.94806444644928)
--(axis cs:5,0.94659161567688)
--(axis cs:4,0.93076229095459)
--(axis cs:3,0.942625343799591)
--(axis cs:2,0.945715665817261)
--(axis cs:1,0.939071178436279)
--cycle;

\addplot [line width=1.0pt, color0, mark=*, mark size=1, mark options={solid}]
table {%
1 0.915336728096008
2 0.928378880023956
3 0.933076977729797
4 0.93170177936554
5 0.935121417045593
6 0.94970977306366
7 0.950429260730743
};
%\addlegendentry{Random}

\addplot [line width=1.0pt, color1, mark=*, mark size=1, mark options={solid}]
table {%
1 0.92097256
2 0.93186356
3 0.94626232
4 0.95317464
5 0.94786172
6 0.95409772
7 0.95516936
};
%\addlegendentry{ETS}
\addplot [line width=1.0pt, color2, mark=*, mark size=1, mark options={solid}]
table {%
1 0.93680264
2 0.92966192
3 0.93242872
4 0.93424116
5 0.93883372
6 0.92840996
7 0.95059124
};
%\addlegendentry{Heuristic}
\addplot [line width=1.0pt, color3, mark=*, mark size=1, mark options={solid}]
table {%
1 0.945298790931702
2 0.952293515205383
3 0.948385059833527
4 0.941560387611389
5 0.95186710357666
6 0.952450633049011
7 0.95458984375
};
%\addlegendentry{RS-MCTS}



\nextgroupplot[title=Permuted MNIST,
height=\figheight,
legend cell align={left},
legend style={
  nodes={scale=0.7},
  fill opacity=0.8,
  draw opacity=1,
  text opacity=1,
  at={(0.48,0.03)},
  anchor=south west,
  draw=white!80!black
},
minor xtick={},
minor ytick={},
tick align=outside,
tick pos=left,
width=\figwidth,
x grid style={white!69.0196078431373!black},
xmajorgrids,
%xlabel={Memory size \(\displaystyle M\)},
xmin=0.8, xmax=5.2,
xtick style={color=black},
xtick={1,2,3,4,5},
xticklabels={90,270,450,900,2250},
y grid style={white!69.0196078431373!black},
ymajorgrids,
ylabel={},%{ACC (\%)},
ymin=0.637838739156723, ymax=0.93909129061632,
ytick style={color=black},
ytick={0.60,0.65,0.7,0.75,0.8,0.85,0.9,0.95},
yticklabels={60,65,70,75,80,85,90,95}
]
\path [fill=color0, fill opacity=0.2, line width=1pt]
(axis cs:1,0.651649475097656)
--(axis cs:1,0.679890394210815)
--(axis cs:2,0.806510269641876)
--(axis cs:3,0.825519323348999)
--(axis cs:4,0.878083646297455)
--(axis cs:5,0.909908771514893)
--(axis cs:5,0.906343221664429)
--(axis cs:5,0.906343221664429)
--(axis cs:4,0.867364227771759)
--(axis cs:3,0.817520618438721)
--(axis cs:2,0.77967768907547)
--(axis cs:1,0.651649475097656)
--cycle;

\path [fill=color1, fill opacity=0.2, line width=1pt]
(axis cs:1,0.703322259916344)
--(axis cs:1,0.726421740083656)
--(axis cs:2,0.813376771868685)
--(axis cs:3,0.855015954553463)
--(axis cs:4,0.884852627163868)
--(axis cs:5,0.912982539743805)
--(axis cs:5,0.908409460256195)
--(axis cs:5,0.908409460256195)
--(axis cs:4,0.875559372836131)
--(axis cs:3,0.832572045446536)
--(axis cs:2,0.800187228131315)
--(axis cs:1,0.703322259916344)
--cycle;

\path [fill=color2, fill opacity=0.2, line width=1pt]
(axis cs:1,0.738538401173349)
--(axis cs:1,0.771401598826651)
--(axis cs:2,0.798919804972782)
--(axis cs:3,0.839665791039941)
--(axis cs:4,0.835268339924516)
--(axis cs:5,0.850651036649128)
--(axis cs:5,0.774932963350872)
--(axis cs:5,0.774932963350872)
--(axis cs:4,0.770147660075484)
--(axis cs:3,0.793590208960059)
--(axis cs:2,0.767396195027218)
--(axis cs:1,0.738538401173349)
--cycle;

\path [fill=color3, fill opacity=0.2, line width=1pt]
(axis cs:1,0.709841430187225)
--(axis cs:1,0.723326504230499)
--(axis cs:2,0.824357867240906)
--(axis cs:3,0.859517455101013)
--(axis cs:4,0.893725514411926)
--(axis cs:5,0.927864193916321)
--(axis cs:5,0.921083688735962)
--(axis cs:5,0.921083688735962)
--(axis cs:4,0.88515043258667)
--(axis cs:3,0.851162552833557)
--(axis cs:2,0.817286252975464)
--(axis cs:1,0.709841430187225)
--cycle;

\addplot [line width=1.0pt, color0, mark=*, mark size=1, mark options={solid}]
table {%
1 0.665769934654236
2 0.793093979358673
3 0.82151997089386
4 0.872723937034607
5 0.908125996589661
};
%\addlegendentry{Random}

\addplot [line width=1.0pt, color1, mark=*, mark size=1, mark options={solid}]
table {%
1 0.714872
2 0.806782
3 0.843794
4 0.880206
5 0.910696
};
%\addlegendentry{ETS}
\addplot [line width=1.0pt, color2, mark=*, mark size=1, mark options={solid}]
table {%
1 0.75497
2 0.783158
3 0.816628
4 0.802708
5 0.812792
};
%\addlegendentry{Heuristic}
\addplot [line width=1.0pt, color3, mark=*, mark size=1, mark options={solid}]
table {%
1 0.716583967208862
2 0.820822060108185
3 0.855340003967285
4 0.889437973499298
5 0.924473941326141
};
%\addlegendentry{Ours}



\nextgroupplot[title=Split CIFAR-100,
height=\figheight,
legend cell align={left},
legend style={
  nodes={scale=0.7},
  fill opacity=0.8,
  draw opacity=1,
  text opacity=1,
  at={(0.48,0.03)},
  anchor=south west,
  draw=white!80!black
},
minor xtick={},
minor ytick={},
tick align=outside,
tick pos=left,
width=\figwidth,
x grid style={white!69.0196078431373!black},
xmajorgrids,
xlabel={Memory size \(\displaystyle M\)},
xmin=0.8, xmax=5.2,
xtick style={color=black},
xtick={1,2,3,4,5},
xticklabels={95,285,475,950,1900},
y grid style={white!69.0196078431373!black},
ymajorgrids,
ylabel={ACC (\%)},
ymin=0.495, ymax=0.810957793848251,
ytick style={color=black},
ytick={0.5, 0.55, 0.6,0.65,0.7,0.75,0.8,0.85},
yticklabels={50, 55, 60, 65, 70, 75, 80, 85}
]
\path [fill=color0, fill opacity=0.2, line width=1pt]
(axis cs:1,0.567187964916229)
--(axis cs:1,0.595052063465118)
--(axis cs:2,0.683687448501587)
--(axis cs:3,0.721232891082764)
--(axis cs:4,0.759949862957001)
--(axis cs:5,0.789501786231995)
--(axis cs:5,0.782858371734619)
--(axis cs:5,0.782858371734619)
--(axis cs:4,0.748370110988617)
--(axis cs:3,0.711007118225098)
--(axis cs:2,0.667912483215332)
--(axis cs:1,0.567187964916229)
--cycle;

\path [fill=color1, fill opacity=0.2, line width=1pt]
(axis cs:1,0.54420843806795)
--(axis cs:1,0.55951156193205)
--(axis cs:2,0.660018780412226)
--(axis cs:3,0.701282341220295)
--(axis cs:4,0.74849294674193)
--(axis cs:5,0.780335209714517)
--(axis cs:5,0.772824790285483)
--(axis cs:5,0.772824790285483)
--(axis cs:4,0.73630705325807)
--(axis cs:3,0.686757658779705)
--(axis cs:2,0.640341219587774)
--(axis cs:1,0.54420843806795)
--cycle;

\path [fill=color2, fill opacity=0.2, line width=1pt]
(axis cs:1,0.499704018706963)
--(axis cs:1,0.577535981293037)
--(axis cs:2,0.631322030526373)
--(axis cs:3,0.68081333911034)
--(axis cs:4,0.714779020310597)
--(axis cs:5,0.698210881385584)
--(axis cs:5,0.575789118614416)
--(axis cs:5,0.575789118614416)
--(axis cs:4,0.604540979689403)
--(axis cs:3,0.61926666088966)
--(axis cs:2,0.573797969473627)
--(axis cs:1,0.499704018706963)
--cycle;

\path [fill=color3, fill opacity=0.2, line width=1pt]
(axis cs:1,0.594093382358551)
--(axis cs:1,0.615146577358246)
--(axis cs:2,0.694118440151215)
--(axis cs:3,0.734759986400604)
--(axis cs:4,0.772708475589752)
--(axis cs:5,0.798255443572998)
--(axis cs:5,0.786504507064819)
--(axis cs:5,0.786504507064819)
--(axis cs:4,0.760211646556854)
--(axis cs:3,0.723879992961884)
--(axis cs:2,0.682881414890289)
--(axis cs:1,0.594093382358551)
--cycle;

\addplot [line width=1.0pt, color0, mark=*, mark size=1, mark options={solid}]
table {%
1 0.581120014190674
2 0.675799965858459
3 0.716120004653931
4 0.754159986972809
5 0.786180078983307
};
%\addlegendentry{Random}

\addplot [line width=1.0pt, color1, mark=*, mark size=1, mark options={solid}]
table {%
1 0.55186
2 0.65018
3 0.69402
4 0.7424
5 0.77658
};

\addplot [line width=1.0pt, color2, mark=*, mark size=1, mark options={solid}]
table {%
1 0.53862
2 0.60256
3 0.65004
4 0.65966
5 0.637
};

\addplot [line width=1.0pt, color3, mark=*, mark size=1, mark options={solid}]
table {%
1 0.604619979858398
2 0.688499927520752
3 0.729319989681244
4 0.766460061073303
5 0.792379975318909
};





\nextgroupplot[title=Split miniImagenet,
height=\figheight,
legend cell align={left},
legend style={
  nodes={scale=0.7},
  fill opacity=0.8,
  draw opacity=1,
  text opacity=1,
  at={(0.48,0.03)},
  anchor=south west,
  draw=white!80!black
},
minor xtick={},
minor ytick={0.525, 0.575, 0.625, 0.675},
tick align=outside,
tick pos=left,
width=\figwidth,
x grid style={white!69.0196078431373!black},
xmajorgrids,
xlabel={Memory size \(\displaystyle M\)},
xmin=0.8, xmax=5.2,
xtick style={color=black},
xtick={1,2,3,4,5},
xticklabels={95,285,475,950,1900},
y grid style={white!69.0196078431373!black},
ymajorgrids, yminorgrids,
%ylabel={ACC},
ymin=0.485468408535527, ymax=0.656408178377217,
ytick style={color=black},
ytick={0.5,0.55,0.6,0.65,0.70},
yticklabels={50,55,60,65,70}
]
\path [fill=color0, fill opacity=0.2, line width=1pt]
(axis cs:1,0.510739743709564)
--(axis cs:1,0.526020228862762)
--(axis cs:2,0.560024738311768)
--(axis cs:3,0.586868524551392)
--(axis cs:4,0.607182204723358)
--(axis cs:5,0.620157301425934)
--(axis cs:5,0.612762629985809)
--(axis cs:5,0.612762629985809)
--(axis cs:4,0.584657728672028)
--(axis cs:3,0.566691517829895)
--(axis cs:2,0.545615077018738)
--(axis cs:1,0.510739743709564)
--cycle;

\path [fill=color1, fill opacity=0.2, line width=1pt]
(axis cs:1,0.493238398073785)
--(axis cs:1,0.533441601926215)
--(axis cs:2,0.57011855059778)
--(axis cs:3,0.605819593487538)
--(axis cs:4,0.614540951326373)
--(axis cs:5,0.630535426185857)
--(axis cs:5,0.616984573814143)
--(axis cs:5,0.616984573814143)
--(axis cs:4,0.595619048673627)
--(axis cs:3,0.586140406512462)
--(axis cs:2,0.55356144940222)
--(axis cs:1,0.493238398073785)
--cycle;

\path [fill=color2, fill opacity=0.2, line width=1pt]
(axis cs:1,0.492947858361007)
--(axis cs:1,0.520172141638993)
--(axis cs:2,0.555463345135852)
--(axis cs:3,0.547130468985637)
--(axis cs:4,0.556312161267811)
--(axis cs:5,0.595653887589556)
--(axis cs:5,0.532706112410444)
--(axis cs:5,0.532706112410444)
--(axis cs:4,0.463447838732189)
--(axis cs:3,0.455229531014363)
--(axis cs:2,0.511216654864149)
--(axis cs:1,0.492947858361007)
--cycle;

\path [fill=color3, fill opacity=0.2, line width=1pt]
(axis cs:1,0.533491373062134)
--(axis cs:1,0.538228631019592)
--(axis cs:2,0.57997077703476)
--(axis cs:3,0.604763567447662)
--(axis cs:4,0.625060558319092)
--(axis cs:5,0.648638188838959)
--(axis cs:5,0.627881824970245)
--(axis cs:5,0.627881824970245)
--(axis cs:4,0.611339330673218)
--(axis cs:3,0.595156371593475)
--(axis cs:2,0.578189194202423)
--(axis cs:1,0.533491373062134)
--cycle;

\addplot [line width=1.0pt, color0, mark=*, mark size=1, mark options={solid}]
table {%
1 0.518379986286163
2 0.552819907665253
3 0.576780021190643
4 0.595919966697693
5 0.616459965705872
};
%\addlegendentry{Random}

\addplot [line width=1.0pt, color1, mark=*, mark size=1, mark options={solid}]
table {%
1 0.51334
2 0.56184
3 0.59598
4 0.60508
5 0.62376
};
%\addlegendentry{ETS}
\addplot [line width=1.0pt, color2, mark=*, mark size=1, mark options={solid}]
table {%
1 0.50656
2 0.53334
3 0.50118
4 0.50988
5 0.56418
};
%\addlegendentry{Heuristic}
\addplot [line width=1.0pt, color3, mark=*, mark size=1, mark options={solid}]
table {%
1 0.535860002040863
2 0.579079985618591
3 0.599959969520569
4 0.618199944496155
5 0.638260006904602
};
%\addlegendentry{RS-MCTS}





\end{groupplot}

\end{tikzpicture}
	}
	\vspace{-7mm}
	\captionsetup{width=.9\linewidth}
	\caption{Performance comparisons with ACC over different replay memory sizes $M$ for RS-MCTS (Ours) and the baselines. All results have been averaged over 5 seeds. %The results show that replay scheduling can outperform replaying with random and ETS schedules, %equal task proportions,
	%especially for small $M$, on both small and large datasets across different backbone choices. %Furthermore, our method requires less careful tuning than the Heuristic baseline as $M$ increases.
}
\vspace{-3mm}
\label{fig:acc_over_replay_memory_size}
\end{wrapfigure}
\paragraph{Varying Memory Size.} We show that the replay schedules found by MCTS improves the CL performance across different memory sizes. In Figure \ref{fig:acc_over_replay_memory_size}, we observe that RS-MCTS obtains better task accuracies than ETS, especially for small memory sizes. Both RS-MCTS and ETS perform better than Heuristic as $M$ increases showing that Heuristic requires careful tuning of the validation accuracy threshold. Finally, these results show that replay scheduling can outperform simpler scheduling methods on both small and large datasets. 

\vspace{-3mm}
\paragraph{Applying Scheduling to Recent Replay Methods.} We show that replay scheduling can be combined with any memory-based CL method to enhance the performance. We combine replay scheduling with three recent memory-based CL methods called Hindsight Anchor Learning (HAL)~\cite{chaudhry2021using}, Meta-Experience Replay (MER)~\cite{riemer2018learning}, Dark Experience Replay (DER)~\cite{buzzega2020dark}. Table \ref{tab:results_applying_scheduling_to_recent_replay_methods} shows the performance comparison between our RS-MCTS against using Random, ETS, and Heuristic schedules for each method. Especially for HAL and DER, the schedule found by RS-MCTS mostly outperforms the other schedules across the different datasets. For MER, the Heuristic baseline performs better than RS-MCTS and Permuted-MNIST and Split miniImagenet which could potentially be adjusted with more hyperparameter searches. Furthermore, the Heuristic baseline occasionally did not converge for some seeds which could indicate that this scheduling method is more sensitive to the settings than the other scheduling methods. Finally, the results in this experiment confirm that replay scheduling is important for the final performance given the same memory constraints and it can benefit any existing CL framework. 


\begin{table}[t]
\footnotesize
\centering
\caption{
    Performance comparison with ACC between scheduling methods RS-MCTS (Ours), Random, ETS, and Heuristic combined with replay-based methods HAL, MER, and DER. We use 'S' and 'P' as short for 'Split' and 'Permuted' for the datasets. 
    Replay memory sizes are $M=10$ and $M=100$ for the 5-task and 10/20-task datasets respectively. We report the mean and standard deviation averaged over 5 seeds. Results on Heuristic where some seed did not converge is denoted by $^{*}$. Applying RS-MCTS to each method can enhance the performance compared to using the baseline schedules. 
}
\vspace{-3mm}
%\setlength{\tabcolsep}{5pt}
\scalebox{0.67}{
\begin{tabular}{l l c c c c c c}
    \toprule
     & & \multicolumn{3}{c}{{\bf 5-task Datasets}} & \multicolumn{3}{c}{{\bf 10- and 20-task Datasets}} \\
    \cmidrule(lr){3-5} \cmidrule(lr){6-8}
    {\bf Method} & {\bf Schedule}  & S-MNIST & S-FashionMNIST & S-notMNIST & P-MNIST & S-CIFAR-100 & S-miniImagenet \\
    \midrule
    \multirow{4}{*}{HAL} & Random & 97.24 {\scriptsize ($\pm$ 0.70)}  & 86.74 {\scriptsize ($\pm$ 6.05)} & 93.61 {\scriptsize($\pm$ 1.31)}  & 88.49 {\scriptsize ($\pm$ 0.99)}  & 36.09 {\scriptsize ($\pm$ 1.77)} & 38.51 {\scriptsize ($\pm$ 2.22)} \\
     & ETS  & 94.02 {\scriptsize ($\pm$ 4.25)} &  95.81 {\scriptsize ($\pm$ 3.53)} & 91.01 {\scriptsize ($\pm$ 1.39)} & 88.46 {\scriptsize ($\pm$ 0.86)}  & 34.90 {\scriptsize ($\pm$ 2.02)} & 38.13 {\scriptsize ($\pm$ 1.18)} \\
    & Heuristic & 97.69 {\scriptsize ($\pm$ 0.19)} & $^{*}$74.16 {\scriptsize ($\pm$ 11.19)} & 93.64 {\scriptsize ($\pm$ 0.93)} & $^{*}$66.63 {\scriptsize ($\pm$ 28.50)} & 35.07 {\scriptsize($\pm$ 1.29)} & 39.51 {\scriptsize($\pm$ 1.49)} \\
     & Ours &  {\bf 97.93} {\scriptsize ($\pm$ 0.56)} &  {\bf 98.27} {\scriptsize ($\pm$ 0.17)} & {\bf 94.64} {\scriptsize ($\pm$ 0.39)} & {\bf 89.14} {\scriptsize ($\pm$ 0.74)}  & {\bf 40.22} {\scriptsize ($\pm$ 1.57)} & {\bf 41.39} {\scriptsize ($\pm$ 1.15)} \\
    \midrule 
    \multirow{4}{*}{MER} & Random & 93.07 {\scriptsize ($\pm$ 0.81)} & 85.53 {\scriptsize ($\pm$ 3.30)} & 91.13 {\scriptsize ($\pm$ 0.86)} & 75.90 {\scriptsize ($\pm$ 1.34)} & 42.96 {\scriptsize ($\pm$ 1.70)} & 31.48 {\scriptsize ($\pm$ 1.65)}  \\ 
     & ETS  & 92.89 {\scriptsize ($\pm$ 3.53)} & 96.47 {\scriptsize ($\pm$ 0.85)} & {\bf 93.80} {\scriptsize ($\pm$ 0.82)} & 73.01 {\scriptsize ($\pm$ 0.96)} & 43.38 {\scriptsize ($\pm$ 1.81)} & 33.58 {\scriptsize ($\pm$ 1.53)} \\
    & Heuristic & 94.30 {\scriptsize ($\pm$ 2.79)} & 96.91 {\scriptsize ($\pm$ 0.62)} & 90.90 {\scriptsize ($\pm$ 1.30)} & {\bf 83.86} {\scriptsize ($\pm$ 3.19)} & 40.90 {\scriptsize ($\pm$ 1.70)}  & {\bf 34.22} {\scriptsize ($\pm$ 1.93)} \\
    & Ours &  {\bf 98.20} {\scriptsize ($\pm$ 0.16)} & {\bf 98.48} {\scriptsize ($\pm$ 0.26)} &  93.61 {\scriptsize ($\pm$ 0.71)} & 79.72 {\scriptsize ($\pm$ 0.71)} & {\bf 44.29} {\scriptsize ($\pm$ 0.69)} & 32.74 {\scriptsize ($\pm$ 1.29)} \\  
    \midrule 
    \multirow{4}{*}{DER} & Random & 98.23 {\scriptsize ($\pm$ 0.53)} & 96.56 {\scriptsize ($\pm$ 1.79)}  & 92.89 {\scriptsize ($\pm$ 0.86)} & 87.51 {\scriptsize ($\pm$ 1.10)} & 56.83 {\scriptsize ($\pm$ 0.76)} & 42.19  {\scriptsize ($\pm$ 0.67)} \\ 
     & ETS  & 98.17 {\scriptsize ($\pm$ 0.35)} & 97.69 {\scriptsize ($\pm$ 0.58)} & 94.74 {\scriptsize ($\pm$ 1.05)} & 85.71 {\scriptsize ($\pm$ 0.75)} & 52.58 {\scriptsize ($\pm$ 1.49)} & 35.50  {\scriptsize ($\pm$ 2.84)} \\
    & Heuristic & 94.57 {\scriptsize ($\pm$ 1.71)} & $^{*}$72.49 {\scriptsize ($\pm$ 19.32)} & $^{*}$77.88 {\scriptsize ($\pm$ 12.58)} & 81.56 {\scriptsize ($\pm$ 2.28)} & 55.75 {\scriptsize ($\pm$ 1.08)} & {\bf 43.62} {\scriptsize ($\pm$ 0.88)} \\
     & Ours & {\bf 99.02} {\scriptsize ($\pm$ 0.10)} & {\bf 98.33} {\scriptsize ($\pm$ 0.51)} & {\bf 95.02} {\scriptsize ($\pm$ 0.33)} & {\bf 90.11} {\scriptsize ($\pm$ 0.18)} & {\bf 58.99} {\scriptsize ($\pm$ 0.98)} & 43.46  {\scriptsize ($\pm$ 0.95)}  \\  
    %\midrule 
    %\multirow{4}{*}{DER++} & Random & 97.90 {\scriptsize ($\pm$ 0.52)} & 97.10 {\scriptsize ($\pm$ 1.03)}  & 93.29 {\scriptsize ($\pm$ 1.43)} & 87.89 {\scriptsize ($\pm$ 1.10)} & 58.49  {\scriptsize ($\pm$ 1.44)} & 48.40  {\scriptsize ($\pm$ 0.69)} \\ 
    % & ETS & 97.98 {\scriptsize ($\pm$ 0.52)} & 98.12 {\scriptsize ($\pm$ 0.40)} & 94.53 {\scriptsize ($\pm$ 1.02)} & 85.25 {\scriptsize ($\pm$ 0.88)} & 52.54  {\scriptsize ($\pm$ 1.06)} & 41.36  {\scriptsize ($\pm$ 2.90)} \\
    %& Heuristic & 92.35 {\scriptsize ($\pm$ 2.42)} & $^{*}$67.31 {\scriptsize ($\pm$ 21.20)} & 93.88 {\scriptsize ($\pm$ 1.33)} & 79.17 {\scriptsize ($\pm$ 2.44)} & 56.70 {\scriptsize ($\pm$ 1.27)} & 45.73 {\scriptsize ($\pm$ 0.84)} \\
    % & Ours & {\bf 98.84} {\scriptsize ($\pm$ 0.21)} & {\bf 98.38} {\scriptsize ($\pm$ 0.43)} & {\bf 94.73} {\scriptsize ($\pm$ 0.20)} & {\bf 89.84} {\scriptsize ($\pm$ 0.22)} & {\bf 59.23}  {\scriptsize ($\pm$ 0.83)} & {\bf 49.45}  {\scriptsize ($\pm$ 0.68)} \\
    \bottomrule
\end{tabular}
}
\vspace{-3mm}
\label{tab:results_applying_scheduling_to_recent_replay_methods}
\end{table}



\vspace{-3mm}
\paragraph{Efficiency of Replay Scheduling.} We illustrate the efficiency of replay scheduling with comparisons to several common replay-based CL baselines in an even more extreme memory setting.
Our goal is to investigate if scheduling over which tasks to replay can be more efficient in situations where the memory size is even smaller than the number of classes. 
To this end, we set the replay memory size for our method
to $M=2$ for the 5-task datasets. We then compare against the most memory efficient CL baselines, namely A-GEM~\cite{chaudhry2018efficient}, ER-Ring~\cite{chaudhry2019tiny} which show promising results with 1 sample per class for replay, as well as with uniform memory selection as reference. 
Additionally, we compare to using random replay schedules (Random) with the same memory setting as for RS-MCTS.
We visualize the memory usage for our method and the baselines when training on a 5-task dataset in Figure \ref{fig:tiny_memory_experiment_memory_usage}. In Paper \ref{chap:paperC}, we provide results for the same experiments with the larger datasets Permuted MNIST, Split CIFAR-100, and Split miniImagenet. 

Table \ref{tab:efficiency_of_replay_scheduling_5task_datasets} shows the ACC for each method across the 5-task datasets. Despite using significantly fewer samples for replay, RS-MCTS performs better or on par with the best baselines on each dataset. These results show that replay scheduling can be even more efficient than the common memory efficient replay-based methods, which indicate that learning the time to learn is an important research direction in CL.

\begin{figure} % necessary to add figure environment for setting up right captions
\begin{minipage}{\textwidth}
	\begin{minipage}[b]{0.34\textwidth}
		\centering
		\setlength{\figwidth}{0.96\textwidth}
		\setlength{\figheight}{.15\textheight}
		

\pgfplotsset{compat=1.11,
    /pgfplots/ybar legend/.style={
    /pgfplots/legend image code/.code={%
       \draw[##1,/tikz/.cd,yshift=-0.25em]
        (0cm,0cm) rectangle (3pt,0.8em);},
   },
}

\begin{tikzpicture}
\tikzstyle{every node}=[font=\footnotesize]
\definecolor{color0}{rgb}{0.12156862745098,0.466666666666667,0.705882352941177}
\definecolor{color1}{rgb}{1,0.498039215686275,0.0549019607843137}
\definecolor{color2}{rgb}{0.172549019607843,0.627450980392157,0.172549019607843}
\definecolor{color3}{rgb}{0.83921568627451,0.152941176470588,0.156862745098039}
\definecolor{color4}{rgb}{0.580392156862745,0.403921568627451,0.741176470588235}

\begin{axis}[title={},
ybar,
bar width = 4pt,
height=\figheight,
legend cell align={left},
legend columns=1,
legend style={
  fill opacity=0.8,
  draw opacity=1,
  text opacity=1,
  at={(1.03,0.20)},
  anchor=south west,
  draw=white!80!black
},
%legend columns=2,
%legend style={
%  fill opacity=0.8,
%  draw opacity=1,
%  text opacity=1,
%  at={(0.03,1.05)},
%  anchor=south west,
%  draw=white!80!black
%},
minor xtick={},
minor ytick={},
tick align=outside,
tick pos=left,
width=\figwidth,
x grid style={white!69.0196078431373!black},
xlabel={Task},
x label style={at={(axis description cs:0.5,-0.36)},anchor=north},
xmin=1.5, xmax=5.5,
xtick style={color=black},
xtick={2,3,4,5},
xticklabels={2,3,4,5},
y grid style={white!69.0196078431373!black},
ymajorgrids,
ylabel={\#Samples},
ymin=0.0, ymax=8.7,
ytick style={color=black},
ytick={0,2, 4, 6, 8, 10},
yticklabels={0, 2, 4, 6, 8, 10}
]

\addplot [draw=black, fill=color0] coordinates {
(2,2)
(3,2)
(4,2)
(5,2)
};\addlegendentry{Ours}

\addplot [draw=black, fill=color1] coordinates {
(2,2)
(3,4)
(4,6)
(5,8)
};\addlegendentry{Baselines}

\end{axis}

\end{tikzpicture}
		\vspace{-3mm}
		\caption{Number of replayed samples per task for the 5-task datasets for RS-MCTS (Ours) and the baselines in the tiny memory setting.}
		\label{fig:tiny_memory_experiment_memory_usage}
	\end{minipage}
	\hfill
	\begin{minipage}[b]{0.64\textwidth}
		\centering
		\footnotesize
		\vspace{-3mm}
		\scalebox{0.87}{
		\begin{tabular}{l c c c}
			\toprule
			& \multicolumn{3}{c}{{\bf 5-task Datasets}} \\ 
			\cmidrule(lr){2-4} 
			{\bf Method} & S-MNIST & S-FashionMNIST & S-notMNIST \\
			\midrule 
			Random & 92.56 ($\pm$ 2.90) & 92.70 ($\pm$ 3.78) & 89.53 ($\pm$ 3.96)  \\
			A-GEM  & 94.97 ($\pm$ 1.50) & 94.81 ($\pm$ 0.86) & 92.27 ($\pm$ 1.16)   \\
			ER-Ring & 94.94 ($\pm$ 1.56) & 95.83 ($\pm$ 2.15) & 91.10 ($\pm$ 1.89)   \\
			Uniform &  95.77 ($\pm$ 1.12) & 97.12 ($\pm$ 1.57) & 92.14 ($\pm$ 1.45)   \\
			\midrule 
			Ours & 96.07 ($\pm$ 1.60) &  97.17 ($\pm$ 0.78) & 93.41 ($\pm$ 1.11)  \\
			\bottomrule
		\end{tabular} }
		\vspace{-3mm}
		\captionsetup{type=table, width=.94\linewidth}
		\caption{Performance comparison with ACC averaged over 5 seeds in the 1 ex/class memory setting. RS-MCTS (Ours) has replay memory size $M=2$, while the baselines replay all available memory samples. }
		\label{tab:efficiency_of_replay_scheduling_5task_datasets}
	\end{minipage}
\end{minipage}
\vspace{-3mm}
\end{figure}



\subsection{Policy Generalization to New Continual Learning Scenarios}

In Paper \ref{chap:paperD}, we perform two experiments to evaluate how well the learned replay scheduling policy can generalize to new CL scenarios. First, we investigate whether the policy can be applied to environments with new dataset task splits which are different from the splits that the policy was trained on. Secondly, we study if the policy can be applied to environments with new datasets than it was originally trained on. Finally, we visualize the learned policy and compare it against one of the heuristic baselines to bring further insights into the advantages of learning the replay scheduling policy.

We run experiments on CL datasets of $T=5$ tasks to have a discrete action space of reasonable size. The replay memory size is set to $M=10$ in all experiments. We use 15\% of the training set for validation for the DQN and the Heuristic baselines. The Random and ETS baselines use the validation set for training since these policies are fixed.  
We use three different heuristic rules for scheduling, namely
\begin{itemize}[itemsep=0em,topsep=1pt]
	\item \textbf{Heuristic Global Drop (Heur-GD)}: Replay task when the validation accuracy has dropped below threshold of the highest achieved validation accuracy across all previous time steps.
	
	\item \textbf{Heuristic Local Drop (Heur-LD)}: Replay task when the validation accuracy has dropped below threshold of the achieved validation accuracy at the previous time step.
	
	\item \textbf{Heuristic Accuracy Threshold (Heur-AT)}: Replay task when the validation accuracy has dropped below threshold of the minimum allowed validation task accuracy. 
\end{itemize}
The thresholds are found from grid searches and we provide the used values in Paper \ref{chap:paperD}. 

\vspace{-3mm}
\begin{wraptable}{r}{0.35\textwidth}
	\footnotesize
	\vspace{-5mm}
	\captionsetup{width=.84\linewidth}
	%\hspace{-3mm}
	\caption{Average rankings for generalizing to Split MNIST environments with new task orders. 
		%Results are averaged over 10 environments and 3 DQN seeds.
	}
	\label{tab:avg_ranking_split_mnist_new_task_orders}
	\vspace{-7mm}
	\begin{tabular}{l c} \\
		\toprule  
		{\bf Method} & {\bf Avg. Rank} \\ 
		\midrule
		Random 		& 3.77 \\
		ETS 		& 3.27 \\
		Heur-GD 	& 4.47 \\
		Heur-LD 	& 3.27 \\
		Heur-AT 	& 3.37 \\
		\midrule
		DQN (Ours)		& {\bf 2.87} \\
		\bottomrule
	\end{tabular}
\end{wraptable} 
\paragraph{Generalization to New Task Orders.} In this experiment, we evaluate the performance of each scheduling method when generalizing to environments of Split MNIST with new task orders unseen during training. The DQN uses 30 environments for training and we average its results over 3 seeds for parameter initialization. Table \ref{tab:avg_ranking_split_mnist_new_task_orders} shows the average ranking over 10 test environments for all methods. We observe that the DQN achieves the best average ranking compared to the baselines. These results indicate that learning the replay scheduling policy is  more beneficial than creating heuristic scheduling rules and fixed policies such as ETS. 



\vspace{-3mm}
\begin{wraptable}{r}{0.35\textwidth}
	\footnotesize
	\vspace{-5mm}
	\captionsetup{width=.84\linewidth}
	%\hspace{-3mm}
	\caption{Average rankings for generalizing to Split notMNIST environments from training on Split MNIST and FashionMNIST. 
	}
	\label{tab:avg_ranking_split_notmnist_new_dataset}
	\vspace{-7mm}
	\begin{tabular}{l c} \\
		\toprule  
		{\bf Method} 	& {\bf Avg. Rank} \\ 
		\midrule
		Random 			& 3.27 \\
		ETS 			& 3.60 \\
		Heur-GD 		& 4.53 \\
		Heur-LD 		& 4.10 \\
		Heur-AT 		& 2.80 \\
		\midrule
		DQN (Ours)		& {\bf 2.70} \\
		\bottomrule
	\end{tabular}
\end{wraptable} 
\paragraph{Generalization to New Datasets.} Here, we evaluate the performance of each scheduling method when generalizing to environments of Split notMNIST datasets while the methods have only been trained on environments of Split MNIST and Split FashionMNIST datasets. The DQN uses 30 environments in total for training (15 Split MNIST + 15 Split FashionMNIST) and we average its results over 3 seeds for parameter initialization. Table \ref{tab:avg_ranking_split_notmnist_new_dataset} shows the average ranking over 10 Split notMNIST environments for all methods. We observe again that the DQN achieves the best average ranking compared to the baselines, where Heur-AT performs almost on par with the DQN. These results further shows that the it is indeed possible to learn general scheduling policies that are capable of performing in new environments without additional computational time and training. 


%TO-DO: Update this part when I have new results for NeurIPS paper!
\vspace{-3mm}
\paragraph{Visualization of Learned Policy.}
We visualize the learned policies for the DQN to bring further insights into the advantages of learning the replay scheduling policy. We analyze the task accuracy progress and the replay schedule used for mitigating catastrophic forgetting by visualizing the schedule as a bubble plot. In the bubble plots shown in Figure \ref{fig:mnist_policy_illustration_dqn_vs_heuristic2}, the x- and y-axis shows the current and replay task respectively, and the bubbles represent the proportion of the replayed task in the replay memory at the current task. Each color of the circles corresponds to a seen task that is being replayed at the current task. The sum of all points in the circles at a current task column is fixed since the replay memory size is constant. 

Figure \ref{fig:mnist_policy_illustration_dqn_vs_heuristic2} shows the task accuracy progression and replay schedules from the DQN and Heur-LD in one test environment with Split MNIST. Using the replay schedule from the DQN gives higher ACC since the DQN manages to mitigate forgetting tasks 1 and 2 better than Heur-LD. The DQN decides to replay task 1 and task 2 as soon as these tasks have been learned. At task 4, the DQN decides to replay both tasks 1 and 2 which helps the classifier to retain its knowledge of task 2. However, as the classifier forgets task 1, the DQN selects to only replay task 1 at task 5 which makes the classifier improve its performance on task 1. 
The heuristic baseline Heur-LD enables replay on tasks 3 and 5 according to its rule since the performance of task 1 and 2 has decreased below its threshold at those time steps, but the classifier forgets these tasks more than when using the learned replay schedule from the DQN. This example shows creating a good heuristic for replay scheduling can be difficult which is why learning such a policy is preferable. 

\begin{figure}[t]
	\centering
	\setlength{\figwidth}{0.26\textwidth}
	\setlength{\figheight}{.15\textheight}
	
\pgfplotsset{every axis title/.append style={at={(0.5,0.82)}}}
\pgfplotsset{every tick label/.append style={font=\tiny}}
\pgfplotsset{every major tick/.append style={major tick length=2pt}}
\pgfplotsset{every minor tick/.append style={minor tick length=1pt}}
\pgfplotsset{every axis x label/.append style={at={(0.5,-0.22)}}}
\pgfplotsset{every axis y label/.append style={at={(-0.22,0.5)}}}
\begin{tikzpicture}
\tikzstyle{every node}=[font=\scriptsize]
\definecolor{color0}{rgb}{0.12156862745098,0.466666666666667,0.705882352941177}
\definecolor{color1}{rgb}{1,0.498039215686275,0.0549019607843137}
\definecolor{color2}{rgb}{0.172549019607843,0.627450980392157,0.172549019607843}
\definecolor{color3}{rgb}{0.83921568627451,0.152941176470588,0.156862745098039}
\definecolor{color4}{rgb}{0.580392156862745,0.403921568627451,0.741176470588235}

\begin{groupplot}[group style={group size= 4 by 1, horizontal sep=1.40cm, vertical sep=1.00cm}]



\nextgroupplot[title={DQN (ACC: 95.50\%)} ,
height=\figheight,
legend cell align={left},
legend columns=-1,
legend style={
  nodes={scale=0.9},
  fill opacity=0.8,
  draw opacity=1,
  text opacity=1,
  at={(1.20,1.40)},
  anchor=south west,
  draw=white!80!black
},
minor xtick={},
minor ytick={0.55,0.65,0.75,0.85,0.95,1.05},
tick align=outside,
tick pos=left,
width=\figwidth,
x grid style={white!69.0196078431373!black},
xmajorgrids,
xlabel={Task},
xmin=0.8, xmax=5.2,
xtick style={color=black},
xtick={1,2,3,4,5},
xticklabel style={font=\tiny, inner sep=1pt},
xticklabels={1,2,3,4,5},
y grid style={white!69.0196078431373!black},
ymajorgrids,
yminorgrids,
ylabel={Accuracy},
ymin=0.491, ymax=1.01,
ytick style={color=black},
ytick={0.5,0.6,0.7,0.8,0.9,1.0},
yticklabel style={font=\tiny, inner sep=0pt},
yticklabels={50, 60, 70, 80, 90, 100}
]
\addplot [color0, mark=*, mark size=1.5, mark options={solid}]
table {%
1 0.994017958641052
2 0.912761688232422
3 0.868394792079926
4 0.750747740268707
5 0.837986052036285
};\addlegendentry{Task 1}
\addplot [ color1, mark=*, mark size=1.5, mark options={solid}]
table {%
2 0.99297297000885
3 0.918378353118896
4 0.967027008533478
5 0.969729721546173
};\addlegendentry{Task 2}
\addplot [ color2, mark=*, mark size=1.5, mark options={solid}]
table {%
3 0.999067604541779
4 0.987412571907043
5 0.981818199157715
};\addlegendentry{Task 3}
\addplot [ color3, mark=*, mark size=1.5, mark options={solid}]
table {%
4 0.996513962745667
5 0.994521915912628
};\addlegendentry{Task 4}
\addplot [color4, mark=*, mark size=1.5, mark options={solid}]
table {%
5 0.990959346294403
};\addlegendentry{Task 5}





\nextgroupplot[title={DQN Schedule},
height=\figheight,
minor xtick={},
minor ytick={},
tick align=outside,
tick pos=left,
width=\figwidth,
x grid style={white!69.0196078431373!black},
grid=major,
xlabel={Current Task},
xmin=1.5, xmax=5.5,
xtick style={color=black},
xtick={2,3,4,5},
xticklabels={2,3,4,5},
xticklabel style={font=\tiny, inner sep=1pt},
y grid style={white!69.0196078431373!black},
ylabel={Replayed Task},
%ylabel style={at={(0.5,0.0)},
ymin=0.45, ymax=4.5,
ytick style={color=black},
ytick={1,2,3,4},
yticklabels={1,2,3,4},
yticklabel style={font=\tiny, inner sep=1pt},
]
\addplot+[
mark=*,
only marks,
scatter,
scatter src=explicit,
scatter/@pre marker code/.code={%
  \expanded{%
    \noexpand\definecolor{thispointdrawcolor}{RGB}{\drawcolor}%
    \noexpand\definecolor{thispointfillcolor}{RGB}{\fillcolor}%
  }%
  \scope[draw=thispointdrawcolor, fill=thispointfillcolor]%
},
visualization depends on={\thisrow{sizedata}/5 \as \perpointmarksize},
visualization depends on={value \thisrow{draw} \as \drawcolor},
visualization depends on={value \thisrow{fill} \as \fillcolor},
%visualization depends on=
%{5*cos(deg(x)) \as \perpointmarksize},
scatter/@pre marker code/.append style=
{/tikz/mark size=\perpointmarksize}
]
table[x=x, y=y, meta=sizedata]{
x  y  draw  fill  sizedata
2 1 31,119,180 31,119,180 17.841241161527712
3 1 31,119,180 31,119,180 0.0
4 1 31,119,180 31,119,180 14.567312407894388
5 1 31,119,180 31,119,180 17.841241161527712

3 2 255,127,14 255,127,14 17.841241161527712
4 2 255,127,14 255,127,14 10.300645387285055
5 2 255,127,14 255,127,14 0.0

};






\nextgroupplot[title={Heur-LD (ACC: 90.77\%)} ,
height=\figheight,
legend cell align={left},
legend style={
  nodes={scale=0.9},
  fill opacity=0.8,
  draw opacity=1,
  text opacity=1,
  at={(3.72,0.28)},
  anchor=south west,
  draw=white!80!black
},
minor xtick={},
minor ytick={0.55,0.65,0.75,0.85,0.95,1.05},
tick align=outside,
tick pos=left,
width=\figwidth,
x grid style={white!69.0196078431373!black},
xmajorgrids,
xlabel={Task},
xmin=0.8, xmax=5.2,
xtick style={color=black},
xtick={1,2,3,4,5},
xticklabels={1,2,3,4,5},
xticklabel style={font=\tiny, inner sep=1pt},
y grid style={white!69.0196078431373!black},
ymajorgrids,
yminorgrids,
ylabel={Accuracy},
ymin=0.491, ymax=1.01,
ytick style={color=black},
ytick={0.5,0.6,0.7,0.8,0.9,1.0},
yticklabels={50, 60, 70, 80, 90, 100},
yticklabel style={font=\tiny, inner sep=0pt}
]
\addplot [color0, mark=*, mark size=1.5, mark options={solid}]
table {%
1 0.994017946161516
2 0.777168494516451
3 0.809571286141575
4 0.567298105682951
5 0.708374875373878
};
%\addlegendentry{Task 1}
\addplot [color1, mark=*, mark size=1.5, mark options={solid}]
table {%
2 0.994054054054054
3 0.970810810810811
4 0.792972972972973
5 0.855675675675676
};
%\addlegendentry{Task 2}
\addplot [color2, mark=*, mark size=1.5, mark options={solid}]
table {%
3 0.998601398601399
4 0.990675990675991
5 0.995804195804196
};
%\addlegendentry{Task 3}
\addplot [color3, mark=*, mark size=1.5, mark options={solid}]
table {%
4 0.99601593625498
5 0.989043824701195
};
%\addlegendentry{Task 4}
\addplot [color4, mark=*, mark size=1.5, mark options={solid}]
table {%
5 0.989452536413862
};
%\addlegendentry{Task 5}
%\addlegendentry{Task 5}








\nextgroupplot[title={Heur-LD Schedule},
height=\figheight,
minor xtick={},
minor ytick={},
tick align=outside,
tick pos=left,
width=\figwidth,
x grid style={white!69.0196078431373!black},
grid=major,
xlabel={Current Task},
xmin=1.5, xmax=5.5,
xtick style={color=black},
xtick={2,3,4,5},
xticklabels={2,3,4,5},
xticklabel style={font=\tiny, inner sep=1pt},
y grid style={white!69.0196078431373!black},
ylabel={Replayed Task},
%ylabel style={at={(0.5,0.0)},
ymin=0.45, ymax=4.5,
ytick style={color=black},
ytick={1,2,3,4},
yticklabels={1,2,3,4},
yticklabel style={font=\tiny, inner sep=1pt},
]
\addplot+[
mark=*,
only marks,
scatter,
scatter src=explicit,
scatter/@pre marker code/.code={%
  \expanded{%
    \noexpand\definecolor{thispointdrawcolor}{RGB}{\drawcolor}%
    \noexpand\definecolor{thispointfillcolor}{RGB}{\fillcolor}%
  }%
  \scope[draw=thispointdrawcolor, fill=thispointfillcolor]%
},
visualization depends on={\thisrow{sizedata}/5 \as \perpointmarksize},
visualization depends on={value \thisrow{draw} \as \drawcolor},
visualization depends on={value \thisrow{fill} \as \fillcolor},
%visualization depends on=
%{5*cos(deg(x)) \as \perpointmarksize},
scatter/@pre marker code/.append style=
{/tikz/mark size=\perpointmarksize}
]
table[x=x, y=y, meta=sizedata]{
x  y  draw  fill  sizedata
2 1 31,119,180 31,119,180 0.0
3 1 31,119,180 31,119,180 17.841241161527712
4 1 31,119,180 31,119,180 0.0
5 1 31,119,180 31,119,180 12.6156626101008

3 2 255,127,14 255,127,14 0.0
4 2 255,127,14 255,127,14 0.0
5 2 255,127,14 255,127,14 12.6156626101008

};

\end{groupplot}

\end{tikzpicture}
	\vspace{-3mm}
	\caption{ Comparison of test classification accuracies for tasks 1-5 on Split MNIST from using the replay schedules from the DQN and the Heuristic Local Drop (Heur-LD) scheduling baseline from 1 seed. We visualize the corresponding replay schedule used by both methods as a bubble plots. The learned policy (DQN) remembers tasks 1 and 2 better than when replaying based on this heuristic. 
	}
	\vspace{-3mm}
	\label{fig:mnist_policy_illustration_dqn_vs_heuristic2}
\end{figure}



\section{Discussion}

In this chapter, we have focused on a new problem setting in CL where historical data is cheap to store but processing time is still limited. In Paper \ref{chap:paperC}, we present this new setting and demonstrate the importance of scheduling which tasks to replay at different times with MCTS. We show that using a good replay schedule can mitigate catastrophic forgetting significantly across several small and large datasets as well as different backbone choices which indicates that learning the time to learn old tasks again will be important in settings where the historical data storage is huge. As MCTS requires multiple trials to find a good scheduling policy, we are disallowed to use MCTS in a real CL setting as this would violate the premise that tasks can never be fully revisited in CL. Therefore, in Paper \ref{chap:paperD}, we propose a framework based on RL for learning a general replay scheduling policy that can be trained on several CL environments and then transferred to new CL scenarios. We used DQN as the RL agent and showed that the learned policies managed to generalize to new task orders and a new dataset without additional computational cost. The next step here is to apply RL-methods that scale to datasets with longer task horizons and large action spaces. 

These results bring the applicability of replay scheduling closer to applying this idea to real-world CL scenarios. For example, the policy could be used for mitigating catastrophic forgetting in assistive vision devices that are personalized to the specific environments of users where new objects need to recognized regularly.  




\MK{Limitations:} Limitations here are of course that we potentially need lots of data for learning a policy that generalizes. Additionally, we need lots of training time and hyperparameter tuning as we are dealing with RL. Also, we need to store the data somewhere which is cheap, but it must be secure due to privacy concerns. An alternative there could be to store features instead of raw data, which is not completely flawless (I think that it's possible to revert features back to the real data to some extent) but at least it is a safer alternative. Another option for the data needed can be to gather experience from simulated environments and benchmark datasets. As the policy only takes in states with task performances, we can make use mixes of benchmark datasets and data from real contributing users. 

