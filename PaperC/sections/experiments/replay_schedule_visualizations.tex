


\subsection{Replay Schedule Visualizations}
\label{paperC:sec:replay_schedule_visualization}

We visualize a learned replay schedule from Split CIFAR-100 with memory size $M=100$ to gain insights into the behavior of the scheduling policy from RS-MCTS. Figure \ref{fig:replay_schedule_vis_cifar100} shows a bubble plot of the task proportions that are used for filling the replay memory at every task. 
Each circle color corresponds to a historical task and the circle size represents its proportion of replay samples at the current task.
%Each color of the circles corresponds to a historical task and its size represents the proportion of examples that are replayed at the current task. 
The sum of all points from all circles at each column is fixed at the different time steps since the memory size $M$ is fixed. The task proportions vary dynamically over time in a sophisticated nonlinear way which would be hard to replace by a heuristic method. Moreover, we can observe spaced repetition-style scheduling on many tasks, e.g., task 1-3 are replayed with similar proportion at the initial tasks but eventually starts varying the time interval between replay. Also, task 4 and 6 need less replay in their early stages, which could potentially be that they are simpler or correlated with other tasks. We provide a similar visualization for Split MNIST in Figure \ref{fig:split_mnist_task_accuracies_and_bubble_plot} in Appendix \ref{paperC:app:replay_schedule_visualization_for_split_mnist} to bring more insights to the benefits of replay scheduling.




