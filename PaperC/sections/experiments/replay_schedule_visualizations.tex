
\begin{comment}
\begin{figure}[t]
  \centering
  \setlength{\figwidth}{0.46\textwidth}
  \setlength{\figheight}{.22\textheight}
  %\vspace{-3mm}
  
\pgfplotsset{every axis title/.append style={at={(0.5,0.92)}}}
\pgfplotsset{every tick label/.append style={font=\tiny}}
\pgfplotsset{every major tick/.append style={major tick length=2pt}}
\pgfplotsset{every axis x label/.append style={at={(0.5,-0.10)}}}
\pgfplotsset{every axis y label/.append style={at={(-0.08,0.5)}}}
\begin{tikzpicture}
\tikzstyle{every node}=[font=\scriptsize]

\begin{axis}[title={},
height=\figheight,
minor xtick={},
minor ytick={},
tick align=outside,
tick pos=left,
width=\figwidth,
x grid style={white!69.0196078431373!black},
grid=major,
xlabel={Current Task},
xmin=1.5, xmax=20.5,
xtick style={color=black},
xtick={2,3,4,5,6,7,8,9,10,11,12,13,14,15,16,17,18,19,20},
xticklabels={2,3,4,5,6,7,8,9,10,11,12,13,14,15,16,17,18,19,20},
xticklabel style = {font=\tiny},
y grid style={white!69.0196078431373!black},
ylabel={Replayed Task},
%ylabel style={at={(0.5,0.0)},
ymin=0.25, ymax=19.5,
ytick style={color=black},
ytick={1,2,3,4,5,6,7,8,9,10,11,12,13,14,15,16,17,18,19},
yticklabels={1,2,3,4,5,6,7,8,9,10,11,12,13,14,15,16,17,18,19},
yticklabel style = {font=\tiny},
]
\addplot+[
mark=*,
only marks,
scatter,
scatter src=explicit,
scatter/@pre marker code/.code={%
  \expanded{%
    \noexpand\definecolor{thispointdrawcolor}{RGB}{\drawcolor}%
    \noexpand\definecolor{thispointfillcolor}{RGB}{\fillcolor}%
  }%
  \scope[draw=thispointdrawcolor, fill=thispointfillcolor]%
},
visualization depends on={\thisrow{sizedata}/5 \as \perpointmarksize},
visualization depends on={value \thisrow{draw} \as \drawcolor},
visualization depends on={value \thisrow{fill} \as \fillcolor},
%visualization depends on=
%{5*cos(deg(x)) \as \perpointmarksize},
scatter/@pre marker code/.append style=
{/tikz/mark size=\perpointmarksize}
]
table[x=x, y=y, meta=sizedata]{PaperC/figures/replay_schedule_visualizations/cifar100/table_seed3.dat};
\end{axis}

\end{tikzpicture}




  \vspace{-3mm}
  \caption{ Replay schedule learned from Split CIFAR-100 visualized as a bubble plot. %Visualization using a bubble plot of a replay schedule learned from Split CIFAR-100. %The color of the circles corresponds to a historical task and its size represents the task proportion that is replayed. The memory size is $M=100$ and the results are from 1 seed. 
  The task proportions vary dynamically over time which would be hard to replace by a heuristic method. 
  } 
  \vspace{-3mm}
  \label{fig:replay_schedule_vis_cifar100}
\end{figure}
\end{comment}



\subsection{Replay Schedule Visualizations}
\label{paperC:sec:replay_schedule_visualization}

We visualize a learned replay schedule from Split CIFAR-100 with memory size $M=100$ to gain insights into the behavior of the scheduling policy from RS-MCTS. Figure \ref{fig:replay_schedule_vis_cifar100} shows a bubble plot of the task proportions that are used for filling the replay memory at every task. 
Each circle color corresponds to a historical task and the circle size represents its proportion of replay samples at the current task.
%Each color of the circles corresponds to a historical task and its size represents the proportion of examples that are replayed at the current task. 
The sum of all points from all circles at each column is fixed at the different time steps since the memory size $M$ is fixed. The task proportions vary dynamically over time in a sophisticated nonlinear way which would be hard to replace by a heuristic method. Moreover, we can observe space repetition style scheduling on many tasks, e.g., task 1-3 are replayed with similar proportion at the initial tasks but eventually starts varying the time interval between replay. Also, task 4 and 6 need less replay in their early stages, which could potentially be that they are simpler or correlated with other tasks. We provide a similar visualization for Split MNIST in Figure \ref{fig:split_mnist_task_accuracies_and_bubble_plot} in Appendix \ref{app:replay_schedule_visualization_for_split_mnist} to bring more insights to the benefits of replay scheduling. %which could potentially be that they are correlated with other tasks or simpler. 




