


\subsection{Varying Memory Size}
\label{paperC:sec:results_with_varying_replay_memory_size}

We show that our method can improve the performance across different choices of memory size.
In this experiment, we set the replay memory size to ensure the ETS baseline replays an equal number of samples per class at the final task. 
The memory size is set to $M = n_{cpt} \cdot n_{spc} \cdot (T-1)$, where $n_{cpt}$ is the number of classes per task in the dataset and $n_{spc}$ are the number of samples per class we wish to replay at task $T$ for the ETS baseline. 
In Figure \ref{fig:acc_over_replay_memory_size}, we observe that RS-MCTS obtains better task accuracies than ETS, especially for small memory sizes. Both RS-MCTS and ETS perform better than Heuristic as $M$ increases showing that Heuristic requires careful tuning of the validation accuracy threshold. These results show that replay scheduling can outperform the baselines, especially for small $M$, on both small and large datasets across different backbone choices.




