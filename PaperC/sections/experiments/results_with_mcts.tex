
%

\begin{figure}[t]
  \centering
  \setlength{\figwidth}{0.23\textwidth}
  \setlength{\figheight}{.13\textheight}
  


\pgfplotsset{every axis title/.append style={at={(0.5,0.79)}}}
\pgfplotsset{every tick label/.append style={font=\tiny}}
\pgfplotsset{every major tick/.append style={major tick length=2pt}}
\pgfplotsset{every minor tick/.append style={minor tick length=1pt}}
\pgfplotsset{every axis x label/.append style={at={(0.5,-0.34)}}}
\pgfplotsset{every axis y label/.append style={at={(-0.32,0.5)}}}
\begin{tikzpicture}
\tikzstyle{every node}=[font=\scriptsize]
\definecolor{color0}{rgb}{0.12156862745098,0.466666666666667,0.705882352941177}
\definecolor{color1}{rgb}{1,0.498039215686275,0.0549019607843137}
\definecolor{color2}{rgb}{0.172549019607843,0.627450980392157,0.172549019607843}
\definecolor{color3}{rgb}{0.83921568627451,0.152941176470588,0.156862745098039}
\definecolor{color4}{rgb}{0.580392156862745,0.403921568627451,0.741176470588235}

\begin{groupplot}[group style={group size= 6 by 1, horizontal sep=0.69cm, vertical sep=0.85cm}]

\nextgroupplot[title=S-MNIST,
height=\figheight,
legend cell align={left},
legend columns=1,
legend style={
  nodes={scale=0.67},%{scale=0.8},
  fill opacity=0.8,
  draw opacity=1,
  text opacity=1,
  at={(7.94,-0.05)},%{(8.15,-0.05)},
  anchor=south west,
  draw=white!80!black
},
minor xtick={25, 75},
minor ytick={0.95, 0.97},
tick align=outside,
tick pos=left,
width=\figwidth,
x grid style={white!69.0196078431373!black},
xlabel={Iteration},
xmajorgrids,
xminorgrids,
xmin=0, xmax=101,
xtick style={color=black},
xtick={-25,0,50,100,125},%xtick={-25,0,25,50,75,100,125},
xticklabels={-25,0,50,100,125},%xticklabels={-25,0,25,50,75,100,125},
y grid style={white!69.0196078431373!black},
ylabel={ACC (\%)},
ymajorgrids,
yminorgrids,
ymin=0.938084134, ymax=0.984917426,
ytick style={color=black},
ytick={0.90,0.92,0.94,0.96,0.98,1.0},
yticklabels={90,92,94,96,98,100}
]
\addplot [line width=1.5pt, color0, mark options={solid}]
table {%
1 0.959084630012512
100 0.959084630012512
};
\addlegendentry{Random}
\addplot [line width=1.5pt, color1, style={dashed}]
table {%
1 0.94021292
100 0.94021292
};
\addlegendentry{ETS}
\addplot [line width=1.5pt, color2, style={dashdotted}]
table {%
1 0.96024116
100 0.96024116
};
\addlegendentry{Heuristic}
\addplot [line width=1.5pt, color3]
table {%
1 0.959084630012512
2 0.96649968624115
3 0.967570662498474
4 0.968766689300537
5 0.968943953514099
6 0.969228386878967
7 0.969228386878967
8 0.972156524658203
9 0.972156524658203
10 0.972156524658203
11 0.972027599811554
12 0.972027599811554
13 0.972027599811554
14 0.971873462200165
15 0.972598731517792
16 0.972422778606415
17 0.972422778606415
18 0.972376048564911
19 0.972171783447266
20 0.976440250873566
21 0.976440250873566
22 0.976578116416931
23 0.976578116416931
24 0.976578116416931
25 0.97786808013916
26 0.97786808013916
27 0.97786808013916
28 0.97786808013916
29 0.97786808013916
30 0.97786808013916
31 0.97786808013916
32 0.97786808013916
33 0.97786808013916
34 0.97786808013916
35 0.97786808013916
36 0.978223025798798
37 0.978223025798798
38 0.978223025798798
39 0.978212535381317
40 0.978212535381317
41 0.978469491004944
42 0.978469491004944
43 0.978469491004944
44 0.978469491004944
45 0.978469491004944
46 0.978469491004944
47 0.978469491004944
48 0.978469491004944
49 0.978469491004944
50 0.978469491004944
51 0.978469491004944
52 0.978469491004944
53 0.978469491004944
54 0.978469491004944
55 0.978469491004944
56 0.978469491004944
57 0.978469491004944
58 0.978469491004944
59 0.978469491004944
60 0.978469491004944
61 0.978469491004944
62 0.978469491004944
63 0.978469491004944
64 0.978469491004944
65 0.978469491004944
66 0.978469491004944
67 0.978469491004944
68 0.978469491004944
69 0.978469491004944
70 0.978469491004944
71 0.978469491004944
72 0.978469491004944
73 0.978469491004944
74 0.978469491004944
75 0.978469491004944
76 0.978469491004944
77 0.978469491004944
78 0.978469491004944
79 0.978469491004944
80 0.978469491004944
81 0.978469491004944
82 0.978469491004944
83 0.978469491004944
84 0.978469491004944
85 0.978469491004944
86 0.978469491004944
87 0.978654503822327
88 0.978654503822327
89 0.978654503822327
90 0.978654503822327
91 0.978654503822327
92 0.978992760181427
93 0.978992760181427
94 0.979289412498474
95 0.979289412498474
96 0.979289412498474
97 0.979289412498474
98 0.979289412498474
99 0.979289412498474
100 0.979289412498474
};
\addlegendentry{Ours}
\addplot [line width=1.5pt, color4]
table {%
1 0.98278864
100 0.98278864
};
\addlegendentry{BFS}




\nextgroupplot[title=S-FashionMNIST,
height=\figheight,
minor xtick={25, 75},
minor ytick={},
tick align=outside,
tick pos=left,
width=\figwidth,
x grid style={white!69.0196078431373!black},
xlabel={Iteration},
xmajorgrids,
xminorgrids,
xmin=0, xmax=101,
xtick style={color=black},
xtick={-25,0,50,100,125},%xtick={-25,0,25,50,75,100,125},
xticklabels={-25,0,50,100,125},%xticklabels={-25,0,25,50,75,100,125},
y grid style={white!69.0196078431373!black},
%ylabel={Best Reward},
ymajorgrids,
ymin=0.955, ymax=0.991,
ytick style={color=black},
ytick={0.92,0.94,0.96,0.97,0.98,0.99,1.0},
yticklabels={92,94,96,97,98,99,100},
]
\addplot [line width=1.5pt, color0]
table {%
1 0.9582200050354
100 0.9582200050354
};
%\addlegendentry{Random}
\addplot [line width=1.5pt, color1, style={dashed}]
table {%
1 0.95814
100 0.95814
};
%\addlegendentry{ETS}
\addplot [line width=1.5pt, color2, style={dashdotted}]
table {%
1 0.97094
100 0.97094
};
%\addlegendentry{Heur}
\addplot [line width=1.5pt, color3]
table {%
1 0.9582200050354
2 0.969580054283142
3 0.970239996910095
4 0.970239996910095
5 0.972180008888245
6 0.972180008888245
7 0.972780048847198
8 0.97382003068924
9 0.97382003068924
10 0.97382003068924
11 0.97382003068924
12 0.97382003068924
13 0.974880039691925
14 0.975259959697723
15 0.975259959697723
16 0.975259959697723
17 0.975259959697723
18 0.977100014686584
19 0.977100014686584
20 0.977800011634827
21 0.977800011634827
22 0.977800011634827
23 0.978839993476868
24 0.978839993476868
25 0.978839993476868
26 0.978960037231445
27 0.978960037231445
28 0.978960037231445
29 0.9792799949646
30 0.9792799949646
31 0.9792799949646
32 0.979800045490265
33 0.980080008506775
34 0.980080008506775
35 0.980080008506775
36 0.98089998960495
37 0.98089998960495
38 0.98089998960495
39 0.98089998960495
40 0.98089998960495
41 0.98089998960495
42 0.98089998960495
43 0.98089998960495
44 0.98089998960495
45 0.98089998960495
46 0.98089998960495
47 0.98089998960495
48 0.98089998960495
49 0.98089998960495
50 0.98089998960495
51 0.98089998960495
52 0.98089998960495
53 0.98089998960495
54 0.98089998960495
55 0.98089998960495
56 0.98089998960495
57 0.98089998960495
58 0.982640087604523
59 0.982640087604523
60 0.982640087604523
61 0.982640087604523
62 0.982640087604523
63 0.982640087604523
64 0.982640087604523
65 0.982640087604523
66 0.982640087604523
67 0.982920050621033
68 0.982920050621033
69 0.982920050621033
70 0.982920050621033
71 0.982920050621033
72 0.982920050621033
73 0.982800006866455
74 0.982800006866455
75 0.982639968395233
76 0.982639968395233
77 0.982639968395233
78 0.982639968395233
79 0.982639968395233
80 0.982639968395233
81 0.982639968395233
82 0.982639968395233
83 0.982639968395233
84 0.982639968395233
85 0.982639968395233
86 0.982639968395233
87 0.982620060443878
88 0.982620060443878
89 0.982620060443878
90 0.982620060443878
91 0.982620060443878
92 0.982620060443878
93 0.982620060443878
94 0.982620060443878
95 0.982620060443878
96 0.982620060443878
97 0.982620060443878
98 0.982740044593811
99 0.982740044593811
100 0.982740044593811
};
%\addlegendentry{RS-MCTS}
\addplot [line width=1.5pt, color4]
table {%
1 0.98508
100 0.98508
};
%\addlegendentry{BFS}


\nextgroupplot[title=S-notMNIST,
height=\figheight,
minor xtick={25,75},
minor ytick={0.92, 0.94, 0.96},
tick align=outside,
tick pos=left,
width=\figwidth,
x grid style={white!69.0196078431373!black},
xlabel={Iteration},
xmajorgrids,
xminorgrids,
xmin=0, xmax=101,
xtick style={color=black},
xtick={-25,0,50,100,125},%xtick={-25,0,25,50,75,100,125},
xticklabels={-25,0,50,100,125},%xticklabels={-25,0,25,50,75,100,125},
y grid style={white!69.0196078431373!black},
%ylabel={Best Reward},
ymajorgrids,
yminorgrids, 
ymin=0.905, ymax=0.968,%0.961,
ytick style={color=black},
ytick={0.9, 0.92, 0.94, 0.96, 0.98 },%{0.9, 0.91, 0.93, 0.95 },
yticklabels={90, 92, 94, 96, 98 }%{90, 91, 93, 95}
]
\addplot [line width=1.5pt, color0]
table {%
1 0.9184
100 0.9184
};
%\addlegendentry{Random}
\addplot [line width=1.5pt, color1, style={dashed}]
table {%
1 0.91013312
100 0.91013312
};
%\addlegendentry{ETS}
\addplot [line width=1.5pt, color2, style={dashdotted}]
table {%
1 0.91260732
100 0.91260732
};
%\addlegendentry{Heur}
\addplot [line width=1.5pt, color3]
table {%
1 0.923885941505432
2 0.924565613269806
3 0.934316277503967
4 0.93102616071701
5 0.929313182830811
6 0.936463952064514
7 0.936463952064514
8 0.938537895679474
9 0.938537895679474
10 0.936725318431854
11 0.936725318431854
12 0.933971762657166
13 0.933971762657166
14 0.939128279685974
15 0.939128279685974
16 0.939128279685974
17 0.939128279685974
18 0.940870583057404
19 0.940870583057404
20 0.940870583057404
21 0.940870583057404
22 0.941275954246521
23 0.941275954246521
24 0.941275954246521
25 0.941275954246521
26 0.9381343126297
27 0.93654453754425
28 0.938980221748352
29 0.93954610824585
30 0.93954610824585
31 0.93954610824585
32 0.93954610824585
33 0.93954610824585
34 0.93954610824585
35 0.93954610824585
36 0.93954610824585
37 0.93954610824585
38 0.93954610824585
39 0.93954610824585
40 0.93954610824585
41 0.93954610824585
42 0.93954610824585
43 0.943730056285858
44 0.943730056285858
45 0.943730056285858
46 0.943730056285858
47 0.946393609046936
48 0.946393609046936
49 0.946393609046936
50 0.946393609046936
51 0.946393609046936
52 0.946393609046936
53 0.946393609046936
54 0.946393609046936
55 0.946393609046936
56 0.946393609046936
57 0.946393609046936
58 0.946393609046936
59 0.946393609046936
60 0.946393609046936
61 0.946393609046936
62 0.946393609046936
63 0.946393609046936
64 0.946393609046936
65 0.946393609046936
66 0.946393609046936
67 0.946393609046936
68 0.946393609046936
69 0.946393609046936
70 0.946393609046936
71 0.946393609046936
72 0.946393609046936
73 0.946393609046936
74 0.946393609046936
75 0.946393609046936
76 0.946393609046936
77 0.946393609046936
78 0.946393609046936
79 0.946393609046936
80 0.946393609046936
81 0.946393609046936
82 0.946393609046936
83 0.946393609046936
84 0.946393609046936
85 0.946393609046936
86 0.946393609046936
87 0.946393609046936
88 0.946393609046936
89 0.946393609046936
90 0.946393609046936
91 0.946393609046936
92 0.946393609046936
93 0.946393609046936
94 0.946393609046936
95 0.946393609046936
96 0.946393609046936
97 0.946393609046936
98 0.946393609046936
99 0.946393609046936
100 0.946393609046936
};
%\addlegendentry{RS-MCTS}
\addplot [line width=1.5pt, color4]
table {%
1 0.95536152
100 0.95536152
};
%\addlegendentry{BFS}
\addplot [line width=1.5pt, color5]
table {%
	1 0.96119168
	100 0.96119168
};
%\addlegendentry{Joint}


\nextgroupplot[title=P-MNIST,
height=\figheight,
minor xtick={25, 75},
minor ytick={0.72, 0.74, 0.76},
tick align=outside,
tick pos=left,
width=\figwidth,
x grid style={white!69.0196078431373!black},
xlabel={Iteration},
xmajorgrids,
xminorgrids,
xmin=0, xmax=101,
xtick style={color=black},
xtick={-25,0,50,100,125},%xtick={-25,0,25,50,75,100,125},
xticklabels={-25,0,50,100,125},%xticklabels={-25,0,25,50,75,100,125},
y grid style={white!69.0196078431373!black},
%ylabel={Best Reward (\%)},
ymajorgrids,
yminorgrids, 
ymin=0.7081526, ymax=0.771,
ytick style={color=black},
ytick={0.7,0.71, 0.73, 0.75, 0.77},
yticklabels={70, 71, 73, 75, 77},
]
\addplot [line width=1.5pt, color0]
table {%
1 0.7259
100 0.7259
};
%\addlegendentry{Random}
\addplot [line width=1.5pt, color1, style={dashed}]
table {%
1 0.710946
100 0.710946
};
\addplot [line width=1.5pt, color2, style={dashdotted}]
table {%
1 0.766814
100 0.766814
};
\addplot [line width=1.5pt, color3]
table {%
1 0.724425971508026
2 0.731194078922272
3 0.741815984249115
4 0.747211992740631
5 0.747211992740631
6 0.747211992740631
7 0.747233986854553
8 0.748315930366516
9 0.748315930366516
10 0.748315930366516
11 0.748315930366516
12 0.748315930366516
13 0.751919984817505
14 0.751919984817505
15 0.751919984817505
16 0.753744006156921
17 0.753744006156921
18 0.753744006156921
19 0.753744006156921
20 0.753744006156921
21 0.753744006156921
22 0.753744006156921
23 0.753744006156921
24 0.753744006156921
25 0.753744006156921
26 0.753744006156921
27 0.753744006156921
28 0.755731999874115
29 0.755731999874115
30 0.755731999874115
31 0.755731999874115
32 0.755731999874115
33 0.755731999874115
34 0.755731999874115
35 0.755731999874115
36 0.755731999874115
37 0.755731999874115
38 0.755731999874115
39 0.757443964481354
40 0.757443964481354
41 0.757443964481354
42 0.757443964481354
43 0.759189963340759
44 0.759189963340759
45 0.759189963340759
46 0.759871959686279
47 0.759871959686279
48 0.759871959686279
49 0.759871959686279
50 0.759871959686279
51 0.759871959686279
52 0.759871959686279
53 0.759871959686279
54 0.759871959686279
55 0.759871959686279
56 0.759871959686279
57 0.759871959686279
58 0.761359989643097
59 0.761359989643097
60 0.761359989643097
61 0.761359989643097
62 0.761359989643097
63 0.761359989643097
64 0.761359989643097
65 0.761359989643097
66 0.761359989643097
67 0.761359989643097
68 0.761359989643097
69 0.761359989643097
70 0.761359989643097
71 0.761359989643097
72 0.761359989643097
73 0.761359989643097
74 0.761359989643097
75 0.761359989643097
76 0.761359989643097
77 0.761359989643097
78 0.761359989643097
79 0.761359989643097
80 0.761359989643097
81 0.761359989643097
82 0.763378024101257
83 0.763378024101257
84 0.763378024101257
85 0.763378024101257
86 0.763378024101257
87 0.763378024101257
88 0.763378024101257
89 0.763378024101257
90 0.763378024101257
91 0.763378024101257
92 0.763378024101257
93 0.763378024101257
94 0.763378024101257
95 0.763378024101257
96 0.763378024101257
97 0.763378024101257
98 0.763378024101257
99 0.763378024101257
100 0.763378024101257
};



\nextgroupplot[title=S-CIFAR-100,
height=\figheight,
minor xtick={25, 75},
minor ytick={0.48, 0.52, 0.56},
tick align=outside,
tick pos=left,
width=\figwidth,
x grid style={white!69.0196078431373!black},
xlabel={Iteration},
xmajorgrids,
xminorgrids,
xmin=0, xmax=101,
xtick style={color=black},
xtick={-25,0,50,100,125},%xtick={-25,0,25,50,75,100,125},
xticklabels={-25,0,50,100,125},%xticklabels={-25,0,25,50,75,100,125},
y grid style={white!69.0196078431373!black},
%ylabel={Best Reward},
ymajorgrids,
yminorgrids,
ymin=0.472152, ymax=0.581,
ytick style={color=black},
ytick={0.46, 0.5, 0.54, 0.58},
yticklabels={0.46, 50, 54, 58}
]
\addplot [line width=1.5pt, color0]
table {%
1 0.5376
100 0.5376
};

\addplot [line width=1.5pt, color1, style={dashed}]
table {%
1 0.47696
100 0.47696
};

\addplot [line width=1.5pt, color2, style={dashdotted}]
table {%
1 0.57312
100 0.57312
};

\addplot [line width=1.5pt, color3]
table {%
1 0.539879977703094
2 0.543919920921326
3 0.555779993534088
4 0.555899977684021
5 0.555899977684021
6 0.555899977684021
7 0.555899977684021
8 0.560000002384186
9 0.563799977302551
10 0.563799977302551
11 0.563799977302551
12 0.563799977302551
13 0.563799977302551
14 0.563799977302551
15 0.563799977302551
16 0.563799977302551
17 0.563799977302551
18 0.563799977302551
19 0.563799977302551
20 0.563459992408752
21 0.563459992408752
22 0.563459992408752
23 0.563459992408752
24 0.563459992408752
25 0.563459992408752
26 0.563459992408752
27 0.563459992408752
28 0.563459992408752
29 0.563459992408752
30 0.563459992408752
31 0.563459992408752
32 0.563459992408752
33 0.563459992408752
34 0.563459992408752
35 0.563459992408752
36 0.563459992408752
37 0.563459992408752
38 0.563459992408752
39 0.563459992408752
40 0.563459992408752
41 0.563459992408752
42 0.56497997045517
43 0.56497997045517
44 0.56497997045517
45 0.565619945526123
46 0.565619945526123
47 0.565619945526123
48 0.565619945526123
49 0.565619945526123
50 0.565619945526123
51 0.567999958992004
52 0.567999958992004
53 0.567999958992004
54 0.567999958992004
55 0.567999958992004
56 0.567999958992004
57 0.567999958992004
58 0.567999958992004
59 0.567999958992004
60 0.567999958992004
61 0.567999958992004
62 0.567999958992004
63 0.567999958992004
64 0.567999958992004
65 0.567999958992004
66 0.567999958992004
67 0.567999958992004
68 0.567999958992004
69 0.567999958992004
70 0.567999958992004
71 0.567999958992004
72 0.566319942474365
73 0.566319942474365
74 0.566319942474365
75 0.565959930419922
76 0.565959930419922
77 0.565959930419922
78 0.565959930419922
79 0.565959930419922
80 0.565959930419922
81 0.565959930419922
82 0.565959930419922
83 0.565959930419922
84 0.565959930419922
85 0.565959930419922
86 0.565959930419922
87 0.565959930419922
88 0.565959930419922
89 0.565959930419922
90 0.565959930419922
91 0.565959930419922
92 0.565959930419922
93 0.565959930419922
94 0.565959930419922
95 0.565959930419922
96 0.565959930419922
97 0.565959930419922
98 0.565959930419922
99 0.565959930419922
100 0.565959930419922
};



\nextgroupplot[title=S-miniImagenet,
height=\figheight,
minor xtick={25,75},
minor ytick={0.465, 0.465, 0.475, 0.485, 0.495},
tick align=outside,
tick pos=left,
width=\figwidth,
x grid style={white!69.0196078431373!black},
xlabel={Iteration},
xmajorgrids,
xminorgrids,
xmin=0, xmax=101,
xtick style={color=black},
xtick={-25,0,50,100,125},%xtick={-25,0,25,50,75,100,125},
xticklabels={-25,0,50,100,125},%xticklabels={-25,0,25,50,75,100,125},
y grid style={white!69.0196078431373!black},
%ylabel={Best Reward},
ymajorgrids,
yminorgrids,
ymin=0.468083999929428, ymax=0.50363600148201,
ytick style={color=black},
ytick={0.46, 0.47, 0.48, 0.49, 0.5, 0.51},
yticklabels={46, 47, 48, 49, 50, 51}
]
\addplot [line width=1.5pt, color0]
table {%
1 0.4989
100 0.4989
};

\addplot [line width=1.5pt, color1, style={dashed}]
table {%
1 0.4697
100 0.4697
};

\addplot [line width=1.5pt, color2, style={dashdotted}]
table {%
1 0.49664
100 0.49664
};

\addplot [line width=1.5pt, color3]
table {%
1 0.480820029973984
2 0.487500041723251
3 0.489940017461777
4 0.492440044879913
5 0.492440044879913
6 0.494700014591217
7 0.496280014514923
8 0.499839961528778
9 0.499839961528778
10 0.499440014362335
11 0.498420000076294
12 0.498420000076294
13 0.498420000076294
14 0.498420000076294
15 0.498420000076294
16 0.498420000076294
17 0.498420000076294
18 0.498420000076294
19 0.498420000076294
20 0.498420000076294
21 0.498420000076294
22 0.498420000076294
23 0.498420000076294
24 0.498420000076294
25 0.498420000076294
26 0.498420000076294
27 0.498420000076294
28 0.498420000076294
29 0.498420000076294
30 0.498420000076294
31 0.498420000076294
32 0.498420000076294
33 0.498420000076294
34 0.498420000076294
35 0.498420000076294
36 0.498420000076294
37 0.498420000076294
38 0.498420000076294
39 0.498420000076294
40 0.498420000076294
41 0.498420000076294
42 0.498420000076294
43 0.498420000076294
44 0.498420000076294
45 0.498420000076294
46 0.498420000076294
47 0.498420000076294
48 0.498420000076294
49 0.498420000076294
50 0.499619960784912
51 0.499619960784912
52 0.499619960784912
53 0.499619960784912
54 0.499619960784912
55 0.499619960784912
56 0.499619960784912
57 0.499619960784912
58 0.499619960784912
59 0.499619960784912
60 0.499619960784912
61 0.499619960784912
62 0.499619960784912
63 0.499619960784912
64 0.499619960784912
65 0.499619960784912
66 0.499619960784912
67 0.499619960784912
68 0.499619960784912
69 0.499619960784912
70 0.499619960784912
71 0.499619960784912
72 0.499619960784912
73 0.499619960784912
74 0.499619960784912
75 0.500059962272644
76 0.500059962272644
77 0.500059962272644
78 0.500059962272644
79 0.501500010490417
80 0.501500010490417
81 0.502020001411438
82 0.502020001411438
83 0.502020001411438
84 0.502020001411438
85 0.502020001411438
86 0.502020001411438
87 0.502020001411438
88 0.502020001411438
89 0.502020001411438
90 0.502020001411438
91 0.502020001411438
92 0.502020001411438
93 0.502020001411438
94 0.502020001411438
95 0.502020001411438
96 0.502020001411438
97 0.502020001411438
98 0.502020001411438
99 0.502020001411438
100 0.502020001411438
};


\end{groupplot}

\end{tikzpicture}
  \vspace{-3mm}
  \caption{ Average test accuracies over tasks after learning the final task (ACC) over the MCTS simulations for all datasets, where 'S' and 'P' are used as short for 'Split' and 'Permuted'. We compare performance for RS-MCTS (Ours) against random replay schedules (Random), Equal Task Schedule (ETS), and Heuristic Scheduling (Heuristic) baselines. 
  For the first three datasets, we show the best ACC found from a breadth-first search (BFS) as an upper bound. All results have been averaged over 5 seeds. These results show that replay scheduling can improve over ETS and outperform or perform on par with Heuristic across different datasets and network architectures. 
  }
  \label{fig:mcts_best_rewards}
  \vspace{-3mm}
\end{figure}




\subsection{Performance Progress of RS-MCTS}\label{paperC:sec:results_with_mcts}

In the first experiments, we show that the replay schedules from RS-MCTS yield better performance than replaying an equal amount of samples per task. 
The replay memory size is fixed to $M=10$ for Split MNIST, FashionMNIST, and notMNIST, and $M=100$ for Permuted MNIST, Split CIFAR-100, and Split miniImagenet. Uniform sampling is used as the memory selection method for all methods in this experiment.
For the 5-task datasets, we provide the optimal replay schedule found from a breadth-first search (BFS) over all 1050 possible replay schedules in our action space (which corresponds to a tree with depth of 4) as an upper bound for RS-MCTS. As the search space grows fast with the number of tasks, BFS becomes computationally infeasible when we have 10 or more tasks.


Figure \ref{fig:mcts_best_rewards} shows the progress of ACC over %the MCTS
iterations by RS-MCTS for all datasets. We also show the best ACC metrics for Random, ETS, Heuristic, and BFS (where appropriate) as straight lines. 
Furthermore, we include the ACC achieved by training on all seen datasets jointly at every task (Joint) for the 5-task datasets.
We observe that RS-MCTS outperforms Random and ETS successively with more iterations. Furthermore, RS-MCTS approaches the upper limit of BFS on the 5-task datasets. For Permuted MNIST and Split CIFAR-100, the Heuristic baseline and RS-MCTS perform on par after 50 iterations. This shows that Heuristic with careful tuning of the validation accuracy threshold can be a strong baseline when comparing replay scheduling methods. Table \ref{tab:results_memory_selection_methods} at the rows for Uniform selection shows the ACC for each method in this experiment. We note that RS-MCTS outperforms ETS significantly on most datasets and performs on par with Heuristic. 



