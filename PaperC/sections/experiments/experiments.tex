


\begin{figure}[t]
  \centering
  \setlength{\figwidth}{0.23\textwidth}
  \setlength{\figheight}{.13\textheight}
  


\pgfplotsset{every axis title/.append style={at={(0.5,0.82)}}}
\pgfplotsset{every tick label/.append style={font=\tiny}}
\pgfplotsset{every major tick/.append style={major tick length=2pt}}
\pgfplotsset{every minor tick/.append style={minor tick length=1pt}}
\pgfplotsset{every axis x label/.append style={at={(0.5,-0.22)}}}
\pgfplotsset{every axis y label/.append style={at={(-0.19,0.5)}}}
\begin{tikzpicture}
\tikzstyle{every node}=[font=\scriptsize]
\definecolor{color0}{rgb}{0.12156862745098,0.466666666666667,0.705882352941177}
\definecolor{color1}{rgb}{1,0.498039215686275,0.0549019607843137}
\definecolor{color2}{rgb}{0.172549019607843,0.627450980392157,0.172549019607843}
\definecolor{color3}{rgb}{0.83921568627451,0.152941176470588,0.156862745098039}
\definecolor{color4}{rgb}{0.580392156862745,0.403921568627451,0.741176470588235}

\begin{groupplot}[group style={group size= 6 by 1, horizontal sep=0.69cm, vertical sep=0.85cm}]

\nextgroupplot[title=S-MNIST,
height=\figheight,
legend cell align={left},
legend columns=-1,
legend style={
  nodes={scale=0.70},%{scale=0.8},
  fill opacity=0.8,
  draw opacity=1,
  text opacity=1,
  at={(0.94,1.55)},%{(8.15,-0.05)},
  anchor=south west,
  draw=white!80!black
},
minor xtick={25, 75},
minor ytick={0.95, 0.97},
tick align=outside,
tick pos=left,
width=\figwidth,
x grid style={white!69.0196078431373!black},
xlabel={Iteration},
xmajorgrids,
xminorgrids,
xmin=0, xmax=101,
xtick style={color=black},
xtick={-25,0,50,100,125},%xtick={-25,0,25,50,75,100,125},
xticklabels={-25,0,50,100,125},%xticklabels={-25,0,25,50,75,100,125},
y grid style={white!69.0196078431373!black},
ylabel={ACC (\%)},
ymajorgrids,
yminorgrids,
ymin=0.938084134, ymax=0.984917426,
ytick style={color=black},
ytick={0.90,0.92,0.94,0.96,0.98,1.0},
yticklabels={90,92,94,96,98,100}
]
\addplot [line width=1.5pt, color0, mark options={solid}]
table {%
1 0.959084630012512
100 0.959084630012512
};
\addlegendentry{Random}
\addplot [line width=1.5pt, color1, style={dashed}]
table {%
1 0.94021292
100 0.94021292
};
\addlegendentry{ETS}
\addplot [line width=1.5pt, color2, style={dashdotted}]
table {%
1 0.96024116
100 0.96024116
};
\addlegendentry{Heuristic}
\addplot [line width=1.5pt, color3]
table {%
1 0.959084630012512
2 0.96649968624115
3 0.967570662498474
4 0.968766689300537
5 0.968943953514099
6 0.969228386878967
7 0.969228386878967
8 0.972156524658203
9 0.972156524658203
10 0.972156524658203
11 0.972027599811554
12 0.972027599811554
13 0.972027599811554
14 0.971873462200165
15 0.972598731517792
16 0.972422778606415
17 0.972422778606415
18 0.972376048564911
19 0.972171783447266
20 0.976440250873566
21 0.976440250873566
22 0.976578116416931
23 0.976578116416931
24 0.976578116416931
25 0.97786808013916
26 0.97786808013916
27 0.97786808013916
28 0.97786808013916
29 0.97786808013916
30 0.97786808013916
31 0.97786808013916
32 0.97786808013916
33 0.97786808013916
34 0.97786808013916
35 0.97786808013916
36 0.978223025798798
37 0.978223025798798
38 0.978223025798798
39 0.978212535381317
40 0.978212535381317
41 0.978469491004944
42 0.978469491004944
43 0.978469491004944
44 0.978469491004944
45 0.978469491004944
46 0.978469491004944
47 0.978469491004944
48 0.978469491004944
49 0.978469491004944
50 0.978469491004944
51 0.978469491004944
52 0.978469491004944
53 0.978469491004944
54 0.978469491004944
55 0.978469491004944
56 0.978469491004944
57 0.978469491004944
58 0.978469491004944
59 0.978469491004944
60 0.978469491004944
61 0.978469491004944
62 0.978469491004944
63 0.978469491004944
64 0.978469491004944
65 0.978469491004944
66 0.978469491004944
67 0.978469491004944
68 0.978469491004944
69 0.978469491004944
70 0.978469491004944
71 0.978469491004944
72 0.978469491004944
73 0.978469491004944
74 0.978469491004944
75 0.978469491004944
76 0.978469491004944
77 0.978469491004944
78 0.978469491004944
79 0.978469491004944
80 0.978469491004944
81 0.978469491004944
82 0.978469491004944
83 0.978469491004944
84 0.978469491004944
85 0.978469491004944
86 0.978469491004944
87 0.978654503822327
88 0.978654503822327
89 0.978654503822327
90 0.978654503822327
91 0.978654503822327
92 0.978992760181427
93 0.978992760181427
94 0.979289412498474
95 0.979289412498474
96 0.979289412498474
97 0.979289412498474
98 0.979289412498474
99 0.979289412498474
100 0.979289412498474
};
\addlegendentry{Ours}
\addplot [line width=1.5pt, color4]
table {%
1 0.98278864
100 0.98278864
};
\addlegendentry{BFS}




\nextgroupplot[title=S-FashionMNIST,
height=\figheight,
minor xtick={25, 75},
minor ytick={},
tick align=outside,
tick pos=left,
width=\figwidth,
x grid style={white!69.0196078431373!black},
xlabel={Iteration},
xmajorgrids,
xminorgrids,
xmin=0, xmax=101,
xtick style={color=black},
xtick={-25,0,50,100,125},%xtick={-25,0,25,50,75,100,125},
xticklabels={-25,0,50,100,125},%xticklabels={-25,0,25,50,75,100,125},
y grid style={white!69.0196078431373!black},
%ylabel={Best Reward},
ymajorgrids,
ymin=0.955, ymax=1.0,%0.991,
ytick style={color=black},
ytick={0.92,0.94,0.96,0.97,0.98,0.99,1.0},
yticklabels={92,94,96,97,98,99,100},
]
\addplot [line width=1.5pt, color0]
table {%
1 0.9589 
100 0.9589 
};
%\addlegendentry{Random}
\addplot [line width=1.5pt, color1, style={dashed}]
table {%
1 0.95814
100 0.95814
};
%\addlegendentry{ETS}
\addplot [line width=1.5pt, color2, style={dashdotted}]
table {%
1 0.97094
100 0.97094
};
%\addlegendentry{Heur}
\addplot [line width=1.5pt, color3]
table {%
1 0.9582200050354
2 0.969580054283142
3 0.970239996910095
4 0.970239996910095
5 0.972180008888245
6 0.972180008888245
7 0.972780048847198
8 0.97382003068924
9 0.97382003068924
10 0.97382003068924
11 0.97382003068924
12 0.97382003068924
13 0.974880039691925
14 0.975259959697723
15 0.975259959697723
16 0.975259959697723
17 0.975259959697723
18 0.977100014686584
19 0.977100014686584
20 0.977800011634827
21 0.977800011634827
22 0.977800011634827
23 0.978839993476868
24 0.978839993476868
25 0.978839993476868
26 0.978960037231445
27 0.978960037231445
28 0.978960037231445
29 0.9792799949646
30 0.9792799949646
31 0.9792799949646
32 0.979800045490265
33 0.980080008506775
34 0.980080008506775
35 0.980080008506775
36 0.98089998960495
37 0.98089998960495
38 0.98089998960495
39 0.98089998960495
40 0.98089998960495
41 0.98089998960495
42 0.98089998960495
43 0.98089998960495
44 0.98089998960495
45 0.98089998960495
46 0.98089998960495
47 0.98089998960495
48 0.98089998960495
49 0.98089998960495
50 0.98089998960495
51 0.98089998960495
52 0.98089998960495
53 0.98089998960495
54 0.98089998960495
55 0.98089998960495
56 0.98089998960495
57 0.98089998960495
58 0.982640087604523
59 0.982640087604523
60 0.982640087604523
61 0.982640087604523
62 0.982640087604523
63 0.982640087604523
64 0.982640087604523
65 0.982640087604523
66 0.982640087604523
67 0.982920050621033
68 0.982920050621033
69 0.982920050621033
70 0.982920050621033
71 0.982920050621033
72 0.982920050621033
73 0.982800006866455
74 0.982800006866455
75 0.982639968395233
76 0.982639968395233
77 0.982639968395233
78 0.982639968395233
79 0.982639968395233
80 0.982639968395233
81 0.982639968395233
82 0.982639968395233
83 0.982639968395233
84 0.982639968395233
85 0.982639968395233
86 0.982639968395233
87 0.982620060443878
88 0.982620060443878
89 0.982620060443878
90 0.982620060443878
91 0.982620060443878
92 0.982620060443878
93 0.982620060443878
94 0.982620060443878
95 0.982620060443878
96 0.982620060443878
97 0.982620060443878
98 0.982740044593811
99 0.982740044593811
100 0.982740044593811
};
%\addlegendentry{RS-MCTS}
\addplot [line width=1.5pt, color4]
table {%
1 0.98508
100 0.98508
};
%\addlegendentry{BFS}
\addplot [line width=1.5pt, color5]
table {%
	1 0.99336
	100 0.99336
};
%\addlegendentry{Joint}

\nextgroupplot[title=S-notMNIST,
height=\figheight,
minor xtick={25,75},
minor ytick={0.92, 0.94, 0.96},
tick align=outside,
tick pos=left,
width=\figwidth,
x grid style={white!69.0196078431373!black},
xlabel={Iteration},
xmajorgrids,
xminorgrids,
xmin=0, xmax=101,
xtick style={color=black},
xtick={-25,0,50,100,125},%xtick={-25,0,25,50,75,100,125},
xticklabels={-25,0,50,100,125},%xticklabels={-25,0,25,50,75,100,125},
y grid style={white!69.0196078431373!black},
%ylabel={Best Reward},
ymajorgrids,
yminorgrids, 
ymin=0.905, ymax=0.961,
ytick style={color=black},
ytick={0.9, 0.91, 0.93, 0.95 },
yticklabels={90, 91, 93, 95}
]
\addplot [line width=1.5pt, color0]
table {%
1 0.923885941505432
100 0.923885941505432
};
%\addlegendentry{Random}
\addplot [line width=1.5pt, color1, style={dashed}]
table {%
1 0.91013312
100 0.91013312
};
%\addlegendentry{ETS}
\addplot [line width=1.5pt, color2, style={dashdotted}]
table {%
1 0.91260732
100 0.91260732
};
%\addlegendentry{Heur}
\addplot [line width=1.5pt, color3]
table {%
1 0.923885941505432
2 0.924565613269806
3 0.934316277503967
4 0.93102616071701
5 0.929313182830811
6 0.936463952064514
7 0.936463952064514
8 0.938537895679474
9 0.938537895679474
10 0.936725318431854
11 0.936725318431854
12 0.933971762657166
13 0.933971762657166
14 0.939128279685974
15 0.939128279685974
16 0.939128279685974
17 0.939128279685974
18 0.940870583057404
19 0.940870583057404
20 0.940870583057404
21 0.940870583057404
22 0.941275954246521
23 0.941275954246521
24 0.941275954246521
25 0.941275954246521
26 0.9381343126297
27 0.93654453754425
28 0.938980221748352
29 0.93954610824585
30 0.93954610824585
31 0.93954610824585
32 0.93954610824585
33 0.93954610824585
34 0.93954610824585
35 0.93954610824585
36 0.93954610824585
37 0.93954610824585
38 0.93954610824585
39 0.93954610824585
40 0.93954610824585
41 0.93954610824585
42 0.93954610824585
43 0.943730056285858
44 0.943730056285858
45 0.943730056285858
46 0.943730056285858
47 0.946393609046936
48 0.946393609046936
49 0.946393609046936
50 0.946393609046936
51 0.946393609046936
52 0.946393609046936
53 0.946393609046936
54 0.946393609046936
55 0.946393609046936
56 0.946393609046936
57 0.946393609046936
58 0.946393609046936
59 0.946393609046936
60 0.946393609046936
61 0.946393609046936
62 0.946393609046936
63 0.946393609046936
64 0.946393609046936
65 0.946393609046936
66 0.946393609046936
67 0.946393609046936
68 0.946393609046936
69 0.946393609046936
70 0.946393609046936
71 0.946393609046936
72 0.946393609046936
73 0.946393609046936
74 0.946393609046936
75 0.946393609046936
76 0.946393609046936
77 0.946393609046936
78 0.946393609046936
79 0.946393609046936
80 0.946393609046936
81 0.946393609046936
82 0.946393609046936
83 0.946393609046936
84 0.946393609046936
85 0.946393609046936
86 0.946393609046936
87 0.946393609046936
88 0.946393609046936
89 0.946393609046936
90 0.946393609046936
91 0.946393609046936
92 0.946393609046936
93 0.946393609046936
94 0.946393609046936
95 0.946393609046936
96 0.946393609046936
97 0.946393609046936
98 0.946393609046936
99 0.946393609046936
100 0.946393609046936
};
%\addlegendentry{RS-MCTS}
\addplot [line width=1.5pt, color4]
table {%
1 0.95536152
100 0.95536152
};
%\addlegendentry{BFS}


\nextgroupplot[title=P-MNIST,
height=\figheight,
minor xtick={25, 75},
minor ytick={0.72, 0.74, 0.76},
tick align=outside,
tick pos=left,
width=\figwidth,
x grid style={white!69.0196078431373!black},
xlabel={Iteration},
xmajorgrids,
xminorgrids,
xmin=0, xmax=101,
xtick style={color=black},
xtick={-25,0,50,100,125},%xtick={-25,0,25,50,75,100,125},
xticklabels={-25,0,50,100,125},%xticklabels={-25,0,25,50,75,100,125},
y grid style={white!69.0196078431373!black},
%ylabel={Best Reward (\%)},
ymajorgrids,
yminorgrids, 
ymin=0.7081526, ymax=0.771,
ytick style={color=black},
ytick={0.7,0.71, 0.73, 0.75, 0.77},
yticklabels={70, 71, 73, 75, 77},
]
\addplot [line width=1.5pt, color0]
table {%
1 0.724425971508026
100 0.724425971508026
};
%\addlegendentry{Random}
\addplot [line width=1.5pt, color1, style={dashed}]
table {%
1 0.710946
100 0.710946
};
\addplot [line width=1.5pt, color2, style={dashdotted}]
table {%
1 0.766814
100 0.766814
};
\addplot [line width=1.5pt, color3]
table {%
1 0.724425971508026
2 0.731194078922272
3 0.741815984249115
4 0.747211992740631
5 0.747211992740631
6 0.747211992740631
7 0.747233986854553
8 0.748315930366516
9 0.748315930366516
10 0.748315930366516
11 0.748315930366516
12 0.748315930366516
13 0.751919984817505
14 0.751919984817505
15 0.751919984817505
16 0.753744006156921
17 0.753744006156921
18 0.753744006156921
19 0.753744006156921
20 0.753744006156921
21 0.753744006156921
22 0.753744006156921
23 0.753744006156921
24 0.753744006156921
25 0.753744006156921
26 0.753744006156921
27 0.753744006156921
28 0.755731999874115
29 0.755731999874115
30 0.755731999874115
31 0.755731999874115
32 0.755731999874115
33 0.755731999874115
34 0.755731999874115
35 0.755731999874115
36 0.755731999874115
37 0.755731999874115
38 0.755731999874115
39 0.757443964481354
40 0.757443964481354
41 0.757443964481354
42 0.757443964481354
43 0.759189963340759
44 0.759189963340759
45 0.759189963340759
46 0.759871959686279
47 0.759871959686279
48 0.759871959686279
49 0.759871959686279
50 0.759871959686279
51 0.759871959686279
52 0.759871959686279
53 0.759871959686279
54 0.759871959686279
55 0.759871959686279
56 0.759871959686279
57 0.759871959686279
58 0.761359989643097
59 0.761359989643097
60 0.761359989643097
61 0.761359989643097
62 0.761359989643097
63 0.761359989643097
64 0.761359989643097
65 0.761359989643097
66 0.761359989643097
67 0.761359989643097
68 0.761359989643097
69 0.761359989643097
70 0.761359989643097
71 0.761359989643097
72 0.761359989643097
73 0.761359989643097
74 0.761359989643097
75 0.761359989643097
76 0.761359989643097
77 0.761359989643097
78 0.761359989643097
79 0.761359989643097
80 0.761359989643097
81 0.761359989643097
82 0.763378024101257
83 0.763378024101257
84 0.763378024101257
85 0.763378024101257
86 0.763378024101257
87 0.763378024101257
88 0.763378024101257
89 0.763378024101257
90 0.763378024101257
91 0.763378024101257
92 0.763378024101257
93 0.763378024101257
94 0.763378024101257
95 0.763378024101257
96 0.763378024101257
97 0.763378024101257
98 0.763378024101257
99 0.763378024101257
100 0.763378024101257
};



\nextgroupplot[title=S-CIFAR-100,
height=\figheight,
minor xtick={25, 75},
minor ytick={0.48, 0.52, 0.56},
tick align=outside,
tick pos=left,
width=\figwidth,
x grid style={white!69.0196078431373!black},
xlabel={Iteration},
xmajorgrids,
xminorgrids,
xmin=0, xmax=101,
xtick style={color=black},
xtick={-25,0,50,100,125},%xtick={-25,0,25,50,75,100,125},
xticklabels={-25,0,50,100,125},%xticklabels={-25,0,25,50,75,100,125},
y grid style={white!69.0196078431373!black},
%ylabel={Best Reward},
ymajorgrids,
yminorgrids,
ymin=0.472152, ymax=0.581,
ytick style={color=black},
ytick={0.46, 0.5, 0.54, 0.58},
yticklabels={0.46, 50, 54, 58}
]
\addplot [line width=1.5pt, color0]
table {%
1 0.5376
100 0.5376
};

\addplot [line width=1.5pt, color1, style={dashed}]
table {%
1 0.47696
100 0.47696
};

\addplot [line width=1.5pt, color2, style={dashdotted}]
table {%
1 0.57312
100 0.57312
};

\addplot [line width=1.5pt, color3]
table {%
1 0.539879977703094
2 0.543919920921326
3 0.555779993534088
4 0.555899977684021
5 0.555899977684021
6 0.555899977684021
7 0.555899977684021
8 0.560000002384186
9 0.563799977302551
10 0.563799977302551
11 0.563799977302551
12 0.563799977302551
13 0.563799977302551
14 0.563799977302551
15 0.563799977302551
16 0.563799977302551
17 0.563799977302551
18 0.563799977302551
19 0.563799977302551
20 0.563459992408752
21 0.563459992408752
22 0.563459992408752
23 0.563459992408752
24 0.563459992408752
25 0.563459992408752
26 0.563459992408752
27 0.563459992408752
28 0.563459992408752
29 0.563459992408752
30 0.563459992408752
31 0.563459992408752
32 0.563459992408752
33 0.563459992408752
34 0.563459992408752
35 0.563459992408752
36 0.563459992408752
37 0.563459992408752
38 0.563459992408752
39 0.563459992408752
40 0.563459992408752
41 0.563459992408752
42 0.56497997045517
43 0.56497997045517
44 0.56497997045517
45 0.565619945526123
46 0.565619945526123
47 0.565619945526123
48 0.565619945526123
49 0.565619945526123
50 0.565619945526123
51 0.567999958992004
52 0.567999958992004
53 0.567999958992004
54 0.567999958992004
55 0.567999958992004
56 0.567999958992004
57 0.567999958992004
58 0.567999958992004
59 0.567999958992004
60 0.567999958992004
61 0.567999958992004
62 0.567999958992004
63 0.567999958992004
64 0.567999958992004
65 0.567999958992004
66 0.567999958992004
67 0.567999958992004
68 0.567999958992004
69 0.567999958992004
70 0.567999958992004
71 0.567999958992004
72 0.566319942474365
73 0.566319942474365
74 0.566319942474365
75 0.565959930419922
76 0.565959930419922
77 0.565959930419922
78 0.565959930419922
79 0.565959930419922
80 0.565959930419922
81 0.565959930419922
82 0.565959930419922
83 0.565959930419922
84 0.565959930419922
85 0.565959930419922
86 0.565959930419922
87 0.565959930419922
88 0.565959930419922
89 0.565959930419922
90 0.565959930419922
91 0.565959930419922
92 0.565959930419922
93 0.565959930419922
94 0.565959930419922
95 0.565959930419922
96 0.565959930419922
97 0.565959930419922
98 0.565959930419922
99 0.565959930419922
100 0.565959930419922
};



\nextgroupplot[title=S-miniImagenet,
height=\figheight,
minor xtick={25,75},
minor ytick={0.465, 0.465, 0.475, 0.485, 0.495},
tick align=outside,
tick pos=left,
width=\figwidth,
x grid style={white!69.0196078431373!black},
xlabel={Iteration},
xmajorgrids,
xminorgrids,
xmin=0, xmax=101,
xtick style={color=black},
xtick={-25,0,50,100,125},%xtick={-25,0,25,50,75,100,125},
xticklabels={-25,0,50,100,125},%xticklabels={-25,0,25,50,75,100,125},
y grid style={white!69.0196078431373!black},
%ylabel={Best Reward},
ymajorgrids,
yminorgrids,
ymin=0.468083999929428, ymax=0.50363600148201,
ytick style={color=black},
ytick={0.46, 0.47, 0.48, 0.49, 0.5, 0.51},
yticklabels={46, 47, 48, 49, 50, 51}
]
\addplot [line width=1.5pt, color0]
table {%
1 0.4989
100 0.4989
};

\addplot [line width=1.5pt, color1, style={dashed}]
table {%
1 0.4697
100 0.4697
};

\addplot [line width=1.5pt, color2, style={dashdotted}]
table {%
1 0.49664
100 0.49664
};

\addplot [line width=1.5pt, color3]
table {%
1 0.480820029973984
2 0.487500041723251
3 0.489940017461777
4 0.492440044879913
5 0.492440044879913
6 0.494700014591217
7 0.496280014514923
8 0.499839961528778
9 0.499839961528778
10 0.499440014362335
11 0.498420000076294
12 0.498420000076294
13 0.498420000076294
14 0.498420000076294
15 0.498420000076294
16 0.498420000076294
17 0.498420000076294
18 0.498420000076294
19 0.498420000076294
20 0.498420000076294
21 0.498420000076294
22 0.498420000076294
23 0.498420000076294
24 0.498420000076294
25 0.498420000076294
26 0.498420000076294
27 0.498420000076294
28 0.498420000076294
29 0.498420000076294
30 0.498420000076294
31 0.498420000076294
32 0.498420000076294
33 0.498420000076294
34 0.498420000076294
35 0.498420000076294
36 0.498420000076294
37 0.498420000076294
38 0.498420000076294
39 0.498420000076294
40 0.498420000076294
41 0.498420000076294
42 0.498420000076294
43 0.498420000076294
44 0.498420000076294
45 0.498420000076294
46 0.498420000076294
47 0.498420000076294
48 0.498420000076294
49 0.498420000076294
50 0.499619960784912
51 0.499619960784912
52 0.499619960784912
53 0.499619960784912
54 0.499619960784912
55 0.499619960784912
56 0.499619960784912
57 0.499619960784912
58 0.499619960784912
59 0.499619960784912
60 0.499619960784912
61 0.499619960784912
62 0.499619960784912
63 0.499619960784912
64 0.499619960784912
65 0.499619960784912
66 0.499619960784912
67 0.499619960784912
68 0.499619960784912
69 0.499619960784912
70 0.499619960784912
71 0.499619960784912
72 0.499619960784912
73 0.499619960784912
74 0.499619960784912
75 0.500059962272644
76 0.500059962272644
77 0.500059962272644
78 0.500059962272644
79 0.501500010490417
80 0.501500010490417
81 0.502020001411438
82 0.502020001411438
83 0.502020001411438
84 0.502020001411438
85 0.502020001411438
86 0.502020001411438
87 0.502020001411438
88 0.502020001411438
89 0.502020001411438
90 0.502020001411438
91 0.502020001411438
92 0.502020001411438
93 0.502020001411438
94 0.502020001411438
95 0.502020001411438
96 0.502020001411438
97 0.502020001411438
98 0.502020001411438
99 0.502020001411438
100 0.502020001411438
};


\end{groupplot}

\end{tikzpicture}
  \vspace{-3mm}
  \caption{ Average test accuracies over tasks after learning the final task (ACC) over the MCTS simulations for all datasets, where 'S' and 'P' are used as short for 'Split' and 'Permuted'. We compare performance for RS-MCTS (Ours) against random replay schedules (Random), Equal Task Schedule (ETS), and Heuristic Scheduling (Heuristic) baselines. 
  For the first three datasets, we show the best ACC found from a breadth-first search (BFS) as an upper bound. All results have been averaged over 5 seeds. These results show that replay scheduling can improve over ETS and outperform or perform on par with Heuristic across different datasets and network architectures. 
  }
  \label{fig:mcts_best_rewards}
  \vspace{-3mm}
\end{figure}




\section{Experiments}\label{paperC:sec:experiments}

We evaluated our replay scheduling method empirically on six common benchmark datasets for CL. 
We denote our method as Replay Scheduling MCTS (RS-MCTS) and select the result on the held-out test sets from the replay schedule that yielded the best reward on the validation set during the search.
Full details on experimental settings are in Appendix \ref{paperC:app:experimental_settings} and additional results are in Appendix \ref{paperC:app:additional_experimental_results}. 
%Code will be available upon acceptance.

\vspace{-3mm}
\paragraph{Datasets.} We conduct experiments on six datasets commonly used as benchmarks in the CL literature: Split MNIST~\citeC{C:lecun1998gradient, C:zenke2017continual}, Fashion-MNIST~\citeC{C:xiao2017fashion}, Split notMNIST~\citeC{C:bulatov2011notMNIST}, Permuted MNIST~\citeC{C:goodfellow2013empirical}, Split CIFAR-100~\citeC{C:krizhevsky2009learning}, and Split miniImagenet~\citeC{C:vinyals2016matching}. We randomly sample 15\% of the training data from each task to use for validation when computing the reward for the MCTS simulations. 

\vspace{-3mm}
\paragraph{Baselines.} We compare RS-MCTS to using 1) random replay schedules (Random), 2) equal task schedules (ETS), and 3) a heuristic scheduling method (Heuristic). The ETS baseline uses equal task proportions, such that $M/(t-1)$ samples per task are replayed during learning of task $t$,and use both training and validations sets for training such that ETS use the same amount of data as RS-MCTS. %such that $M/(t-1)$ samples per task are replayed during learning of task $t$, and use both training and validations sets for training such that they use the same amount of data as RS-MCTS. 
The Heuristic baseline replays the tasks which accuracy on the validation set is below a certain threshold proportional to the best achieved validation accuracy on the task. Here, the replay memory is filled with $M/k$ samples per task where $k$ is the number of selected tasks. If $k=0$, then we skip applying replay at the current task.  
See Appendix \ref{paperC:app:heuristic_scheduling_baseline} for more details on the Heuristic baseline. 

%%% Table 1, put it here for placement

\begin{table}[t]
\footnotesize
\centering
\caption{
Performance comparison with ACC between RS-MCTS (Ours), Random scheduling (Random), Equal Task Schedule (ETS), and Heuristic Scheduling (Heuristic) with various memory selection methods evaluated across all datasets. We provide the metrics for training on all seen task datasets jointly (Joint) as an upper bound, as well as include the results from a breadth-first search (BFS) with Uniform memory selection for the 5-task datasets. 
%We use 'S' and 'P' as short for 'Split' and 'Permuted' for the datasets. 
Replay memory sizes are $M=10$ and $M=100$ for the 5-task and 10/20-task datasets respectively. We report the mean and standard deviation averaged over 5 seeds. Ours performs better or on par with the baselines on most datasets and selection methods, where MoF yields the best results %performance 
in general.
}
\vspace{-3mm}
%\setlength{\tabcolsep}{5pt}
\resizebox{\textwidth}{!}{ %\scalebox{0.87}{
\begin{tabular}{l l c c c c c c}
    \toprule
     & & \multicolumn{3}{c}{{\bf 5-task Datasets}} & \multicolumn{3}{c}{{\bf 10- and 20-task Datasets}} \\
    \cmidrule(lr){3-5} \cmidrule(lr){6-8}
    {\bf Selection} & {\bf Method} & S-MNIST & S-FashionMNIST & S-notMNIST & P-MNIST & S-CIFAR-100 & S-miniImagenet \\
    \midrule 
    \multicolumn{1}{c}{--} & Joint & 99.75 ($\pm$ 0.06) & 99.34 ($\pm$ 0.08) & 96.12 ($\pm$ 0.57) & 95.34 ($\pm$ 0.13) & 84.73 ($\pm$ 0.81) & 74.03 ($\pm$ 0.83) \\
	Uniform & BFS & 98.28 ($\pm$ 0.49)  & 98.51 ($\pm$ 0.23) & 95.54 ($\pm$ 0.67) & -- & -- & -- \\
	\midrule 
    \multirow{4}{*}{Uniform} & Random & 95.91 {\scriptsize ($\pm$ 1.56)}  & 95.82 {\scriptsize ($\pm$ 1.45)} & 92.39 {\scriptsize ($\pm$ 1.29)} & 72.44 {\scriptsize ($\pm$ 1.15)} & 53.99 {\scriptsize ($\pm$ 0.51)} & 48.08 {\scriptsize ($\pm$ 1.36)} \\
     & ETS  & 94.02 {\scriptsize ($\pm$ 4.25)} &  95.81 {\scriptsize ($\pm$ 3.53)} & 91.01 {\scriptsize ($\pm$ 1.39)} & 71.09 {\scriptsize ($\pm$ 2.31)} & 47.70 {\scriptsize ($\pm$ 2.16)} & 46.97 {\scriptsize ($\pm$ 1.24)} \\
     & Heuristic &  96.02 {\scriptsize ($\pm$ 2.32)} &  97.09 {\scriptsize ($\pm$ 0.62)} & 91.26 {\scriptsize ($\pm$ 3.99)} & {\bf 76.68} {\scriptsize ($\pm$ 2.13)} & {\bf 57.31} {\scriptsize ($\pm$ 1.21)} & 49.66 {\scriptsize ($\pm$ 1.10)} \\
     & Ours &  {\bf 97.93} {\scriptsize ($\pm$ 0.56)} &  {\bf 98.27} {\scriptsize ($\pm$ 0.17)} & {\bf 94.64} {\scriptsize ($\pm$ 0.39)} & 76.34 {\scriptsize ($\pm$ 0.98)} & 56.60 {\scriptsize ($\pm$ 1.13)} & {\bf 50.20} {\scriptsize ($\pm$ 0.72)} \\
    \midrule 
    \multirow{4}{*}{$k$-means} & Random & 94.24 {\scriptsize ($\pm$ 3.20)} & 96.30 {\scriptsize ($\pm$ 1.62)} & 91.64 {\scriptsize ($\pm$ 1.39)} & 74.30 {\scriptsize ($\pm$ 1.43)} & 53.18 {\scriptsize ($\pm$ 1.66)} & 49.47 {\scriptsize ($\pm$ 2.70)}  \\ 
     & ETS  & 92.89 {\scriptsize ($\pm$ 3.53)} & 96.47 {\scriptsize ($\pm$ 0.85)} & {\bf 93.80} {\scriptsize ($\pm$ 0.82)} & 69.40 {\scriptsize ($\pm$ 1.32)} & 47.51 {\scriptsize ($\pm$ 1.14)} & 45.82 {\scriptsize ($\pm$ 0.92)} \\
     & Heuristic &  96.28 {\scriptsize ($\pm$ 1.68)} &  95.78 {\scriptsize ($\pm$ 1.50)}  & 91.75 {\scriptsize ($\pm$ 0.94)} & 75.57 {\scriptsize ($\pm$ 1.18)} & 54.31 {\scriptsize ($\pm$  3.94)} & 49.25 {\scriptsize ($\pm$ 1.00)} \\
    & Ours & {\bf 98.20} {\scriptsize ($\pm$ 0.16)} & {\bf 98.48} {\scriptsize ($\pm$ 0.26)} &  93.61 {\scriptsize ($\pm$ 0.71)} & {\bf 77.74} {\scriptsize ($\pm$ 0.80)} & {\bf 56.95} {\scriptsize ($\pm$ 0.92)} & {\bf 50.47} {\scriptsize ($\pm$ 0.85)} \\  
    \midrule 
    \multirow{4}{*}{$k$-center} & Random & 96.40 {\scriptsize ($\pm$ 0.68)} & 95.57 {\scriptsize ($\pm$ 3.16)}  & 92.61 {\scriptsize ($\pm$ 1.70)} & 71.41 {\scriptsize ($\pm$ 2.75)} & 48.46 {\scriptsize ($\pm$ 0.31)} & 44.76 {\scriptsize ($\pm$ 0.96)} \\ 
     & ETS  & 94.84 {\scriptsize ($\pm$ 1.40)} & 97.28 {\scriptsize ($\pm$ 0.50)} & 91.08 {\scriptsize ($\pm$ 2.48)} & 69.11 {\scriptsize ($\pm$ 1.69)} & 44.13 {\scriptsize ($\pm$ 1.06)} & 41.35 {\scriptsize ($\pm$ 1.23)} \\
     & Heuristic & 94.55 {\scriptsize ($\pm$ 2.79)} & 94.08 {\scriptsize ($\pm$ 3.72)} & 92.06 {\scriptsize ($\pm$ 1.20)} & 74.33 {\scriptsize ($\pm$ 2.00)} & 50.32 {\scriptsize ($\pm$ 1.97)} & 44.13 {\scriptsize ($\pm$ 0.95)} \\
     & Ours & {\bf 98.24} {\scriptsize ($\pm$ 0.36)} & {\bf 98.06} {\scriptsize ($\pm$ 0.35)} & {\bf 94.26} {\scriptsize ($\pm$ 0.37)} & {\bf 76.55} {\scriptsize ($\pm$ 1.16)} & {\bf 51.37} {\scriptsize ($\pm$ 1.63)} & {\bf 46.76} {\scriptsize ($\pm$ 0.96)}  \\  
    \midrule 
    \multirow{4}{*}{MoF} & Random & 95.18 {\scriptsize ($\pm$ 3.18)} & 95.76 {\scriptsize ($\pm$ 1.41)}  & 91.33 {\scriptsize ($\pm$ 1.75)} & 77.96 {\scriptsize ($\pm$ 1.84)} & 61.93 {\scriptsize ($\pm$ 1.05)} & 54.50 {\scriptsize ($\pm$ 1.33)} \\ 
     & ETS & 97.04 {\scriptsize ($\pm$ 1.23)} & 96.48 {\scriptsize ($\pm$ 1.33)} & 92.64 {\scriptsize ($\pm$ 0.87)} & 77.62 {\scriptsize ($\pm$ 1.12)} & 60.43 {\scriptsize ($\pm$ 1.17)} & 56.12 {\scriptsize ($\pm$ 1.12)} \\
     & Heuristic & 96.46 {\scriptsize ($\pm$ 2.41)} &  95.84 {\scriptsize ($\pm$ 0.89)} & 93.24 {\scriptsize ($\pm$ 0.77)} & 77.27 {\scriptsize ($\pm$ 1.45)} & 55.60 {\scriptsize ($\pm$ 2.70)} & 52.30 {\scriptsize ($\pm$ 0.59)} \\
     & Ours & {\bf 98.37} {\scriptsize ($\pm$ 0.24)} & {\bf 97.84} {\scriptsize ($\pm$ 0.32)} & {\bf 94.62} {\scriptsize ($\pm$ 0.42)} & {\bf 81.58} {\scriptsize ($\pm$ 0.75)} & {\bf 64.22} {\scriptsize ($\pm$ 0.65)} & {\bf 57.70} {\scriptsize ($\pm$ 0.51)} \\
    \bottomrule
\end{tabular}
}
\vspace{-3mm}
\label{tab:results_memory_selection_methods}
\end{table}

\begin{comment}
\begin{table*}[t]
\footnotesize
\centering
\caption{
Performance comparison between RS-MCTS (Ours), Random scheduling (Random), Equal Task Schedule (ETS), and Heuristic Scheduling (Heuristic) with various memory selection methods evaluated across all datasets. 
%We use 'S' and 'P' as short for 'Split' and 'Permuted' for the datasets. 
The replay memory size is $M=10$ and $M=100$ for the 5-task and 10/20-task datasets respectively. We report the mean and standard deviation averaged over 5 seeds. RS-MCTS performs better or on par than the baselines on most datasets and selection methods, where MoF yields the best results %performance 
in general.
}
%\vspace{-3mm}
%\setlength{\tabcolsep}{5pt}
\scalebox{0.87}{
\begin{tabular}{l l c c c c c c}
    \toprule
     & & \multicolumn{3}{c}{{\bf 5-task Datasets}} & \multicolumn{3}{c}{{\bf 10- and 20-task Datasets}} \\
    \cmidrule(lr){3-5} \cmidrule(lr){6-8}
    {\bf Selection} & {\bf Method} & S-MNIST & S-FashionMNIST & S-notMNIST & P-MNIST & S-CIFAR-100 & S-miniImagenet \\
    \midrule 
    \multirow{4}{*}{Uniform} & Random & 95.91 {\scriptsize ($\pm$ 1.56)}  & 95.82 {\scriptsize ($\pm$ 1.45)} & 92.39 {\scriptsize ($\pm$ 1.29)} & 72.44 {\scriptsize ($\pm$ 1.15)} & 53.99 {\scriptsize ($\pm$ 0.51)} & 48.08 {\scriptsize ($\pm$ 1.36)} \\
     & ETS  & 94.02 {\scriptsize ($\pm$ 4.25)} &  95.81 {\scriptsize ($\pm$ 3.53)} & 91.01 {\scriptsize ($\pm$ 1.39)} & 71.09 {\scriptsize ($\pm$ 2.31)} & 47.70 {\scriptsize ($\pm$ 2.16)} & 46.97 {\scriptsize ($\pm$ 1.24)} \\
     & Heuristic &  96.02 {\scriptsize ($\pm$ 2.32)} &  97.09 {\scriptsize ($\pm$ 0.62)} & 91.26 {\scriptsize ($\pm$ 3.99)} & 76.68 {\scriptsize ($\pm$ 2.13)} & 57.31 {\scriptsize ($\pm$ 1.21)} & 49.66 {\scriptsize ($\pm$ 1.10)} \\
     & Ours &  97.93 {\scriptsize ($\pm$ 0.56)} &  98.27 {\scriptsize ($\pm$ 0.17)} & 94.64 {\scriptsize ($\pm$ 0.39)} & 76.34 {\scriptsize ($\pm$ 0.98)} & 56.60 {\scriptsize ($\pm$ 1.13)} & 50.20 {\scriptsize ($\pm$ 0.72)} \\
    \midrule 
    \multirow{4}{*}{$k$-means} & Random & 94.24 {\scriptsize ($\pm$ 3.20)} & 96.30 {\scriptsize ($\pm$ 1.62)} & 91.64 {\scriptsize ($\pm$ 1.39)} & 74.30 {\scriptsize ($\pm$ 1.43)} & 53.18 {\scriptsize ($\pm$ 1.66)} & 49.47 {\scriptsize ($\pm$ 2.70)}  \\ 
     & ETS  & 92.89 {\scriptsize ($\pm$ 3.53)} & 96.47 {\scriptsize ($\pm$ 0.85)} & 93.80 {\scriptsize ($\pm$ 0.82)} & 69.40 {\scriptsize ($\pm$ 1.32)} & 47.51 {\scriptsize ($\pm$ 1.14)} & 45.82 {\scriptsize ($\pm$ 0.92)} \\
     & Heuristic &  96.28 {\scriptsize ($\pm$ 1.68)} &  95.78 {\scriptsize ($\pm$ 1.50)}  & 91.75 {\scriptsize ($\pm$ 0.94)} & 75.57 {\scriptsize ($\pm$ 1.18)} & 54.31 {\scriptsize ($\pm$  3.94)} & 49.25 {\scriptsize ($\pm$ 1.00)} \\
    & Ours & 98.20 {\scriptsize ($\pm$ 0.16)} & 98.48 {\scriptsize ($\pm$ 0.26)} &  93.61 {\scriptsize ($\pm$ 0.71)} & 77.74 {\scriptsize ($\pm$ 0.80)} & 56.95 {\scriptsize ($\pm$ 0.92)} & 50.47 {\scriptsize ($\pm$ 0.85)} \\  
    \midrule 
    \multirow{4}{*}{$k$-center} & Random & 96.40 {\scriptsize ($\pm$ 0.68)} & 95.57 {\scriptsize ($\pm$ 3.16)}  & 92.61 {\scriptsize ($\pm$ 1.70)} & 71.41 {\scriptsize ($\pm$ 2.75)} & 48.46 {\scriptsize ($\pm$ 0.31)} & 44.76 {\scriptsize ($\pm$ 0.96)} \\ 
     & ETS  & 94.84 {\scriptsize ($\pm$ 1.40)} & 97.28 {\scriptsize ($\pm$ 0.50)} & 91.08 {\scriptsize ($\pm$ 2.48)} & 69.11 {\scriptsize ($\pm$ 1.69)} & 44.13 {\scriptsize ($\pm$ 1.06)} & 41.35 {\scriptsize ($\pm$ 1.23)} \\
     & Heuristic & 94.55 {\scriptsize ($\pm$ 2.79)} & 94.08 {\scriptsize ($\pm$ 3.72)} & 92.06 {\scriptsize ($\pm$ 1.20)} & 74.33 {\scriptsize ($\pm$ 2.00)} & 50.32 {\scriptsize ($\pm$ 1.97)} & 44.13 {\scriptsize ($\pm$ 0.95)} \\
     & Ours & 98.24 {\scriptsize ($\pm$ 0.36)} & 98.06 {\scriptsize ($\pm$ 0.35)} & 94.26 {\scriptsize ($\pm$ 0.37)} & 76.55 {\scriptsize ($\pm$ 1.16)} & 51.37 {\scriptsize ($\pm$ 1.63)} & 46.76 {\scriptsize ($\pm$ 0.96)}  \\  
    \midrule 
    \multirow{4}{*}{MoF} & Random & 95.18 {\scriptsize ($\pm$ 3.18)} & 95.76 {\scriptsize ($\pm$ 1.41)}  & 91.33 {\scriptsize ($\pm$ 1.75)} & 77.96 {\scriptsize ($\pm$ 1.84)} & 61.93 {\scriptsize ($\pm$ 1.05)} & 54.50 {\scriptsize ($\pm$ 1.33)} \\ 
     & ETS & 97.04 {\scriptsize ($\pm$ 1.23)} & 96.48 {\scriptsize ($\pm$ 1.33)} & 92.64 {\scriptsize ($\pm$ 0.87)} & 77.62 {\scriptsize ($\pm$ 1.12)} & 60.43 {\scriptsize ($\pm$ 1.17)} & 56.12 {\scriptsize ($\pm$ 1.12)} \\
     & Heuristic & 96.46 {\scriptsize ($\pm$ 2.41)} &  95.84 {\scriptsize ($\pm$ 0.89)} & 93.24 {\scriptsize ($\pm$ 0.77)} & 77.27 {\scriptsize ($\pm$ 1.45)} & 55.60 {\scriptsize ($\pm$ 2.70)} & 52.30 {\scriptsize ($\pm$ 0.59)} \\
     & Ours & 98.37 {\scriptsize ($\pm$ 0.24)} & 97.84 {\scriptsize ($\pm$ 0.32)} & 94.62 {\scriptsize ($\pm$ 0.42)} & 81.58 {\scriptsize ($\pm$ 0.75)} & 64.22 {\scriptsize ($\pm$ 0.65)} & 57.70 {\scriptsize ($\pm$ 0.51)} \\
    \bottomrule
\end{tabular}
}
\vspace{-3mm}
\label{tab:results_memory_selection_methods}
\end{table*}
\end{comment}


\begin{comment}
\begin{table*}[t]
\footnotesize
\centering
\caption{
Accuracy comparison between RS-MCTS (Ours), Random choice scheduling (Random), Heuristic Scheduling (Heuristic), and Equal Task Schedule (ETS) with various memory selection methods evaluated across all datasets. We use 'S' and 'P' as short for 'Split' and 'Permuted' for the datasets. The replay memory size is $M=10$ and $M=100$ for the 5-task and 10/20-task datasets respectively. We report the mean and standard deviation averaged over 5 seeds. RS-MCTS performs better or on par than the baselines on most datasets and selection methods, where MoF yields the best performance in general.
}
%\vspace{-3mm}
\setlength{\tabcolsep}{5pt}
\begin{tabular}{l l c c c c c c}
    \toprule
     & & \multicolumn{3}{c}{{\bf 5-task Datasets}} & \multicolumn{3}{c}{{\bf 10- and 20-task Datasets}} \\
    \cmidrule(lr){3-5} \cmidrule(lr){6-8}
    {\bf Selection} & {\bf Method} & S-MNIST & S-FashionMNIST & S-notMNIST & P-MNIST & S-CIFAR-100 & S-miniImagenet \\
    \midrule 
    \multirow{4}{*}{Uniform} & Random & 95.91 $\pm$ 1.56  & 95.82 $\pm$ 1.45 & 92.39 $\pm$ 1.29 & 72.44 $\pm$ 1.15 & 53.99 $\pm$ 0.51 & 48.08 $\pm$ 1.36 \\
     & ETS  & 94.02 $\pm$ 4.25 &  95.81 $\pm$ 3.53 & 91.01 $\pm$ 1.39 & 71.09 $\pm$ 2.31 & 47.70 $\pm$ 2.16 & 46.97 $\pm$ 1.24 \\
     & Heuristic &  96.02 $\pm$ 2.32 &  97.09 $\pm$ 0.62 & 91.26 $\pm$ 3.99 & 76.68 $\pm$ 2.13 & 57.31 $\pm$ 1.21 & 49.66 $\pm$ 1.10 \\
     & Ours &  97.93 $\pm$ 0.56 &  98.27 $\pm$ 0.17 & 94.64 $\pm$ 0.39 & 76.34 $\pm$ 0.98 & 56.60 $\pm$ 1.13 & 50.20 $\pm$ 0.72 \\
    \midrule 
    \multirow{4}{*}{$k$-means} & Random & 94.24 $\pm$ 3.20 & 96.30 $\pm$ 1.62 & 91.64 $\pm$ 1.39 & 74.30 $\pm$ 1.43 & 53.18 $\pm$ 1.66 & 49.47 $\pm$ 2.70  \\ 
     & ETS  & 92.89 $\pm$ 3.53 & 96.47 $\pm$ 0.85 & 93.80 $\pm$ 0.82 & 69.40 $\pm$ 1.32 & 47.51 $\pm$ 1.14 & 45.82 $\pm$ 0.92 \\
     & Heuristic &  96.28 $\pm$ 1.68 &  95.78 $\pm$ 1.50  & 91.75 $\pm$ 0.94 & 75.57 $\pm$ 1.18 & 54.31 $\pm$  3.94 & 49.25 $\pm$ 1.00 \\
    & Ours & 98.20 $\pm$ 0.16 & 98.48 $\pm$ 0.26 &  93.61 $\pm$ 0.71 & 77.74 $\pm$ 0.80 & 56.95 $\pm$ 0.92 & 50.47 $\pm$ 0.85 \\  
    \midrule 
    \multirow{4}{*}{$k$-center} & Random & 96.40 $\pm$ 0.68 & 95.57 $\pm$ 3.16  & 92.61 $\pm$ 1.70 & 71.41 $\pm$ 2.75 & 48.46 $\pm$ 0.31 & 44.76 $\pm$ 0.96 \\ 
     & ETS  & 94.84 $\pm$ 1.40 & 97.28 $\pm$ 0.50 & 91.08 $\pm$ 2.48 & 69.11 $\pm$ 1.69 & 44.13 $\pm$ 1.06 & 41.35 $\pm$ 1.23 \\
     & Heuristic & 94.55 $\pm$ 2.79 & 94.08 $\pm$ 3.72 & 92.06 $\pm$ 1.20 & 74.33 $\pm$ 2.00 & 50.32 $\pm$ 1.97 & 44.13 $\pm$ 0.95 \\
     & Ours & {98.24 $\pm$ 0.36} & 98.06 $\pm$ 0.35 & 94.26 $\pm$ 0.37 & 76.55 $\pm$ 1.16 & {51.37 $\pm$ 1.63} & { 46.76 $\pm$ 0.96}  \\  
    \midrule 
    \multirow{4}{*}{MoF} & Random & 95.18 $\pm$ 3.18 & 95.76 $\pm$ 1.41  & 91.33 $\pm$ 1.75 & 77.96 $\pm$ 1.84 & 61.93 $\pm$ 1.05 & 54.50 $\pm$ 1.33 \\ 
     & ETS & 97.04 $\pm$ 1.23 & 96.48 $\pm$ 1.33 & 92.64 $\pm$ 0.87 & 77.62 $\pm$ 1.12 & 60.43 $\pm$ 1.17 & 56.12 $\pm$ 1.12 \\
     & Heuristic & 96.46 $\pm$ 2.41 &  95.84 $\pm$ 0.89 & 93.24 $\pm$ 0.77 & 77.27 $\pm$ 1.45 & 55.60 $\pm$ 2.70 & 52.30 $\pm$ 0.59 \\
     & Ours & { 98.37 $\pm$ 0.24} & { 97.84 $\pm$ 0.32} & {94.62 $\pm$ 0.42} & {81.58 $\pm$ 0.75} & {64.22 $\pm$ 0.65} & { 57.70 $\pm$ 0.51} \\
    \bottomrule
\end{tabular}
\vspace{-3mm}
\label{tab:results_memory_selection_methods}
\end{table*}
\end{comment}




\vspace{-3mm}
\paragraph{Architectures.} We use a multi-head output layer for all datasets except for Permuted MNIST where the network uses single-head output. We use a 2-layer MLP with 256 hidden units for Split MNIST, Split FashionMNIST, Split notMNIST, and Permuted MNIST. For Split CIFAR-100, we use the CNN architecture used in \citeC{C:vinyals2016matching,C:schwarz2018progress}. For Split miniImagenet, we apply the reduced ResNet-18 from \citeC{C:lopez2017gradient}. 

\vspace{-3mm}
\paragraph{Evaluation Metric.} We use the average test accuracy over all tasks after learning the final task, i.e., $\text{ACC} = \frac{1}{T} \sum_{i=1}^{T} A_{T, i}^{(test)}$ where $A_{T, i}^{(test)}$ is the test accuracy of task $i$ after learning task $T$. We report means and standard deviations of ACC using 5 different seeds on all datasets. 





%%% MCTS Results

%

\begin{figure}[t]
  \centering
  \setlength{\figwidth}{0.23\textwidth}
  \setlength{\figheight}{.13\textheight}
  


\pgfplotsset{every axis title/.append style={at={(0.5,0.82)}}}
\pgfplotsset{every tick label/.append style={font=\tiny}}
\pgfplotsset{every major tick/.append style={major tick length=2pt}}
\pgfplotsset{every minor tick/.append style={minor tick length=1pt}}
\pgfplotsset{every axis x label/.append style={at={(0.5,-0.22)}}}
\pgfplotsset{every axis y label/.append style={at={(-0.19,0.5)}}}
\begin{tikzpicture}
\tikzstyle{every node}=[font=\scriptsize]
\definecolor{color0}{rgb}{0.12156862745098,0.466666666666667,0.705882352941177}
\definecolor{color1}{rgb}{1,0.498039215686275,0.0549019607843137}
\definecolor{color2}{rgb}{0.172549019607843,0.627450980392157,0.172549019607843}
\definecolor{color3}{rgb}{0.83921568627451,0.152941176470588,0.156862745098039}
\definecolor{color4}{rgb}{0.580392156862745,0.403921568627451,0.741176470588235}

\begin{groupplot}[group style={group size= 6 by 1, horizontal sep=0.69cm, vertical sep=0.85cm}]

\input{PaperC/figures/mcts_rewards_comparison/MNIST_rewards}

\input{PaperC/figures/mcts_rewards_comparison/FashionMNIST_rewards}

\input{PaperC/figures/mcts_rewards_comparison/notMNIST_rewards}

\input{PaperC/figures/mcts_rewards_comparison/PermutedMNIST_rewards}

\input{PaperC/figures/mcts_rewards_comparison/CIFAR100_rewards}

\input{PaperC/figures/mcts_rewards_comparison/miniImagenet_rewards}

\end{groupplot}

\end{tikzpicture}
  \vspace{-3mm}
  \caption{ Average test accuracies over tasks after learning the final task (ACC) over the MCTS simulations for all datasets, where 'S' and 'P' are used as short for 'Split' and 'Permuted'. We compare performance for RS-MCTS (Ours) against random replay schedules (Random), Equal Task Schedule (ETS), and Heuristic Scheduling (Heuristic) baselines. 
  For the first three datasets, we show the best ACC found from a breadth-first search (BFS) as an upper bound. All results have been averaged over 5 seeds. These results show that replay scheduling can improve over ETS and outperform or perform on par with Heuristic across different datasets and network architectures. 
  }
  \label{fig:mcts_best_rewards}
  \vspace{-3mm}
\end{figure}




\subsection{Performance Progress of RS-MCTS}\label{paperC:sec:results_with_mcts}

In the first experiments, we show that the replay schedules from RS-MCTS yield better performance than replaying an equal amount of samples per task. 
The replay memory size is fixed to $M=10$ for Split MNIST, FashionMNIST, and notMNIST, and $M=100$ for Permuted MNIST, Split CIFAR-100, and Split miniImagenet. Uniform sampling is used as the memory selection method for all methods in this experiment.
For the 5-task datasets, we provide the optimal replay schedule found from a breadth-first search (BFS) over all 1050 possible replay schedules in our action space (which corresponds to a tree with depth of 4) as an upper bound for RS-MCTS. As the search space grows fast with the number of tasks, BFS becomes computationally infeasible when we have 10 or more tasks.
%For the 5-task datasets, we provide the optimal replay schedule found from a breadth-first search (BFS) over all possible replay schedules in our action space as an upper bound for RS-MCTS. 

%As the search space grows fast with the number of tasks, it will even with only 5 continual learning tasks (which corresponds to a tree with depth of 4) yield a tree with 1050 leaf nodes. Thus, BFS becomes computationally infeasible when we have 10 or more tasks.


Figure \ref{fig:mcts_best_rewards} shows the progress of ACC over %the MCTS
iterations by RS-MCTS for all datasets. We also show the best ACC metrics for Random, ETS, Heuristic, and BFS (where appropriate) as straight lines. We observe that RS-MCTS outperforms Random and ETS successively with more iterations. Furthermore, RS-MCTS approaches the upper limit of BFS on the 5-task datasets. For Permuted MNIST and Split CIFAR-100, the Heuristic baseline and RS-MCTS perform on par after 50 iterations. This shows that Heuristic with careful tuning of the validation accuracy threshold can be a strong baseline when comparing replay scheduling methods. The top row of Table \ref{tab:results_memory_selection_methods} shows the ACC for each method for this experiment. We note that RS-MCTS outperforms ETS significantly on most datasets and performs on par with Heuristic. 







%%% Visualizations



\subsection{Replay Schedule Visualizations}
\label{paperC:sec:replay_schedule_visualization}



%
\begin{figure}[t]
  \centering
  \setlength{\figwidth}{0.46\textwidth}
  \setlength{\figheight}{.22\textheight}
  %\vspace{-3mm}
  
\pgfplotsset{every axis title/.append style={at={(0.5,0.92)}}}
\pgfplotsset{every tick label/.append style={font=\tiny}}
\pgfplotsset{every major tick/.append style={major tick length=2pt}}
\pgfplotsset{every axis x label/.append style={at={(0.5,-0.10)}}}
\pgfplotsset{every axis y label/.append style={at={(-0.10,0.5)}}}
\begin{tikzpicture}
\tikzstyle{every node}=[font=\scriptsize]

\begin{axis}[title={},
height=\figheight,
minor xtick={},
minor ytick={},
tick align=outside,
tick pos=left,
width=\figwidth,
x grid style={white!69.0196078431373!black},
grid=major,
xlabel={Current Task},
xmin=1.5, xmax=20.5,
xtick style={color=black},
xtick={2,3,4,5,6,7,8,9,10,11,12,13,14,15,16,17,18,19,20},
xticklabels={2,3,4,5,6,7,8,9,10,11,12,13,14,15,16,17,18,19,20},
xticklabel style = {font=\tiny},
y grid style={white!69.0196078431373!black},
ylabel={Replayed Task},
%ylabel style={at={(0.5,0.0)},
ymin=0.25, ymax=19.5,
ytick style={color=black},
ytick={1,2,3,4,5,6,7,8,9,10,11,12,13,14,15,16,17,18,19},
yticklabels={1,2,3,4,5,6,7,8,9,10,11,12,13,14,15,16,17,18,19},
yticklabel style = {font=\tiny},
]
\addplot+[
mark=*,
only marks,
scatter,
scatter src=explicit,
scatter/@pre marker code/.code={%
  \expanded{%
    \noexpand\definecolor{thispointdrawcolor}{RGB}{\drawcolor}%
    \noexpand\definecolor{thispointfillcolor}{RGB}{\fillcolor}%
  }%
  \scope[draw=thispointdrawcolor, fill=thispointfillcolor]%
},
visualization depends on={\thisrow{sizedata}/5 \as \perpointmarksize},
visualization depends on={value \thisrow{draw} \as \drawcolor},
visualization depends on={value \thisrow{fill} \as \fillcolor},
%visualization depends on=
%{5*cos(deg(x)) \as \perpointmarksize},
scatter/@pre marker code/.append style=
{/tikz/mark size=\perpointmarksize}
]
table[x=x, y=y, meta=sizedata]{figures/replay_schedule_visualizations/cifar100/table_seed3.dat};
\end{axis}

\end{tikzpicture}




  \vspace{-4mm}
  \caption{ Replay schedule learned from Split CIFAR-100 visualized as a bubble plot. %Visualization using a bubble plot of a replay schedule learned from Split CIFAR-100. %The color of the circles corresponds to a historical task and its size represents the task proportion that is replayed. The memory size is $M=100$ and the results are from 1 seed. 
  The task proportions vary dynamically over time which would be hard to replace by a heuristic method. 
  } 
  \vspace{-5mm}
  \label{fig:replay_schedule_vis_cifar100}
\end{figure}
\begin{wrapfigure}{r}{0.52\textwidth}
	%\centering
  	\setlength{\figwidth}{0.56\textwidth}
	\setlength{\figheight}{.42\textwidth}
	\vspace{-3mm}
	
\pgfplotsset{every axis title/.append style={at={(0.5,0.92)}}}
\pgfplotsset{every tick label/.append style={font=\tiny}}
\pgfplotsset{every major tick/.append style={major tick length=2pt}}
\pgfplotsset{every axis x label/.append style={at={(0.5,-0.10)}}}
\pgfplotsset{every axis y label/.append style={at={(-0.10,0.5)}}}
\begin{tikzpicture}
\tikzstyle{every node}=[font=\scriptsize]

\begin{axis}[title={},
height=\figheight,
minor xtick={},
minor ytick={},
tick align=outside,
tick pos=left,
width=\figwidth,
x grid style={white!69.0196078431373!black},
grid=major,
xlabel={Current Task},
xmin=1.5, xmax=20.5,
xtick style={color=black},
xtick={2,3,4,5,6,7,8,9,10,11,12,13,14,15,16,17,18,19,20},
xticklabels={2,3,4,5,6,7,8,9,10,11,12,13,14,15,16,17,18,19,20},
xticklabel style = {font=\tiny},
y grid style={white!69.0196078431373!black},
ylabel={Replayed Task},
%ylabel style={at={(0.5,0.0)},
ymin=0.25, ymax=19.5,
ytick style={color=black},
ytick={1,2,3,4,5,6,7,8,9,10,11,12,13,14,15,16,17,18,19},
yticklabels={1,2,3,4,5,6,7,8,9,10,11,12,13,14,15,16,17,18,19},
yticklabel style = {font=\tiny},
]
\addplot+[
mark=*,
only marks,
scatter,
scatter src=explicit,
scatter/@pre marker code/.code={%
  \expanded{%
    \noexpand\definecolor{thispointdrawcolor}{RGB}{\drawcolor}%
    \noexpand\definecolor{thispointfillcolor}{RGB}{\fillcolor}%
  }%
  \scope[draw=thispointdrawcolor, fill=thispointfillcolor]%
},
visualization depends on={\thisrow{sizedata}/5 \as \perpointmarksize},
visualization depends on={value \thisrow{draw} \as \drawcolor},
visualization depends on={value \thisrow{fill} \as \fillcolor},
%visualization depends on=
%{5*cos(deg(x)) \as \perpointmarksize},
scatter/@pre marker code/.append style=
{/tikz/mark size=\perpointmarksize}
]
table[x=x, y=y, meta=sizedata]{figures/replay_schedule_visualizations/cifar100/table_seed3.dat};
\end{axis}

\end{tikzpicture}




	\vspace{-3mm}
	%\captionsetup{width=.85\linewidth}
	\caption{Replay schedule learned from Split CIFAR-100 visualized as a bubble plot. %Visualization using a bubble plot of a replay schedule learned from Split CIFAR-100. %The color of the circles corresponds to a historical task and its size represents the task proportion that is replayed. The memory size is $M=100$ and the results are from 1 seed. 
		The task proportions vary dynamically over time which would be hard to replace by a heuristic method. 
	}
	%\vspace{-4mm}
	\label{fig:replay_schedule_vis_cifar100}
\end{wrapfigure}
We visualize a learned replay schedule from Split CIFAR-100 with memory size $M=100$ to gain insights into the behavior of the scheduling policy from RS-MCTS. Figure \ref{fig:replay_schedule_vis_cifar100} shows a bubble plot of the task proportions that are used for filling the replay memory at every task. 
Each circle color corresponds to a historical task and the circle size represents its proportion of replay samples at the current task.
%Each color of the circles corresponds to a historical task and its size represents the proportion of examples that are replayed at the current task. 
The sum of all points from all circles at each column is fixed at the different time steps since the memory size $M$ is fixed. The task proportions vary dynamically over time in a sophisticated nonlinear way which would be hard to replace by a heuristic method. Moreover, we can observe spaced repetition-style scheduling on many tasks, e.g., task 1-3 are replayed with similar proportion at the initial tasks but eventually starts varying the time interval between replay. Also, task 4 and 6 need less replay in their early stages, which could potentially be that they are simpler or correlated with other tasks. We provide a similar visualization for Split MNIST in Figure \ref{fig:split_mnist_task_accuracies_and_bubble_plot} in Appendix \ref{paperC:app:replay_schedule_visualization_for_split_mnist} to bring more insights to the benefits of replay scheduling.






%
\begin{figure}[t]
  \centering
  \setlength{\figwidth}{0.46\textwidth}
  \setlength{\figheight}{.22\textheight}
  %\vspace{-3mm}
  
\pgfplotsset{every axis title/.append style={at={(0.5,0.92)}}}
\pgfplotsset{every tick label/.append style={font=\tiny}}
\pgfplotsset{every major tick/.append style={major tick length=2pt}}
\pgfplotsset{every axis x label/.append style={at={(0.5,-0.10)}}}
\pgfplotsset{every axis y label/.append style={at={(-0.10,0.5)}}}
\begin{tikzpicture}
\tikzstyle{every node}=[font=\scriptsize]

\begin{axis}[title={},
height=\figheight,
minor xtick={},
minor ytick={},
tick align=outside,
tick pos=left,
width=\figwidth,
x grid style={white!69.0196078431373!black},
grid=major,
xlabel={Current Task},
xmin=1.5, xmax=20.5,
xtick style={color=black},
xtick={2,3,4,5,6,7,8,9,10,11,12,13,14,15,16,17,18,19,20},
xticklabels={2,3,4,5,6,7,8,9,10,11,12,13,14,15,16,17,18,19,20},
xticklabel style = {font=\tiny},
y grid style={white!69.0196078431373!black},
ylabel={Replayed Task},
%ylabel style={at={(0.5,0.0)},
ymin=0.25, ymax=19.5,
ytick style={color=black},
ytick={1,2,3,4,5,6,7,8,9,10,11,12,13,14,15,16,17,18,19},
yticklabels={1,2,3,4,5,6,7,8,9,10,11,12,13,14,15,16,17,18,19},
yticklabel style = {font=\tiny},
]
\addplot+[
mark=*,
only marks,
scatter,
scatter src=explicit,
scatter/@pre marker code/.code={%
  \expanded{%
    \noexpand\definecolor{thispointdrawcolor}{RGB}{\drawcolor}%
    \noexpand\definecolor{thispointfillcolor}{RGB}{\fillcolor}%
  }%
  \scope[draw=thispointdrawcolor, fill=thispointfillcolor]%
},
visualization depends on={\thisrow{sizedata}/5 \as \perpointmarksize},
visualization depends on={value \thisrow{draw} \as \drawcolor},
visualization depends on={value \thisrow{fill} \as \fillcolor},
%visualization depends on=
%{5*cos(deg(x)) \as \perpointmarksize},
scatter/@pre marker code/.append style=
{/tikz/mark size=\perpointmarksize}
]
table[x=x, y=y, meta=sizedata]{figures/replay_schedule_visualizations/cifar100/table_seed3.dat};
\end{axis}

\end{tikzpicture}




  \vspace{-4mm}
  \caption{ Replay schedule learned from Split CIFAR-100 visualized as a bubble plot. %Visualization using a bubble plot of a replay schedule learned from Split CIFAR-100. %The color of the circles corresponds to a historical task and its size represents the task proportion that is replayed. The memory size is $M=100$ and the results are from 1 seed. 
  The task proportions vary dynamically over time which would be hard to replace by a heuristic method. 
  } 
  \vspace{-5mm}
  \label{fig:replay_schedule_vis_cifar100}
\end{figure}


%%% Results on memory selection methods

\subsection{Alternative Memory Selection Methods}
\label{sec:alternative_memory_selection_methods}

%In this section, we 
We show that our method can be combined with any memory selection method for storing replay samples. In addition to uniform sampling, we apply various memory selection methods commonly used in the CL literature, namely $k$-means clustering, $k$-center clustering~\citep{nguyen2017variational}, and Mean-of-Features (MoF)~\citep{rebuffi2017icarl}. We compare our method and ETS combined with these different selection methods. 
The replay memory sizes are $M=10$ for the 5-task datasets and $M=100$ for the 10- and 20-task datasets. 
%As in Section \ref{sec:results_with_mcts}, we set the replay memory size $M=10$ for the 5-task datasets and $M=100$ for the 10- and 20-task datasets. 
Table \ref{tab:results_memory_selection_methods} shows the results across all datasets. 
We note that using the replay schedule from RS-MCTS outperforms the baselines when using the alternative selection methods, where MoF performs the best on most datasets. 







%%% Figure Acc over memory size

\begin{figure}[t]
  \centering
  \setlength{\figwidth}{0.35\textwidth}
  \setlength{\figheight}{.16\textheight}%{.12\textheight}
  
\pgfplotsset{every axis title/.append style={at={(0.5,0.80)}}} %{at={(0.5,0.84)}}}
\pgfplotsset{every tick label/.append style={font=\tiny}}
\pgfplotsset{every major tick/.append style={major tick length=2pt}}
\pgfplotsset{every minor tick/.append style={minor tick length=1pt}}
\pgfplotsset{every axis x label/.append style={at={(0.5,-0.40)}}}
\pgfplotsset{every axis y label/.append style={at={(-0.20,0.5)}}}
\begin{tikzpicture}
\tikzstyle{every node}=[font=\scriptsize]
\definecolor{color0}{rgb}{0.12156862745098,0.466666666666667,0.705882352941177}
\definecolor{color1}{rgb}{1,0.498039215686275,0.0549019607843137}
\definecolor{color2}{rgb}{0.172549019607843,0.627450980392157,0.172549019607843}
\definecolor{color3}{rgb}{0.83921568627451,0.152941176470588,0.156862745098039}

\begin{groupplot}[group style={group size= 3 by 2, horizontal sep=0.6cm, vertical sep=0.9cm}]

% This file was created with tikzplotlib v0.9.12.

\nextgroupplot[title=Split MNIST,
height=\figheight,
legend cell align={left},
legend columns=1,
legend style={
  nodes={scale=0.95},%=0.95},%{scale=0.85},
  fill opacity=0.8,
  draw opacity=1,
  text opacity=1,
  at={(2.6,-1.65)}, %{(3.70,-0.30)},   %{(3.72,0.13)},
  anchor=south west,
  draw=white!80!black
},
minor xtick={},
minor ytick={},
tick align=outside,
tick pos=left,
width=\figwidth,
x grid style={white!69.0196078431373!black},
xlabel={},%{Memory size \(\displaystyle M\)},
xmajorgrids,
xmin=0.8, xmax=5.2,
xtick style={color=black},
xtick={1,2,3,4,5},
xticklabel style={rotate=90},
xticklabels={10,20,40,100,200},
y grid style={white!69.0196078431373!black},
ymajorgrids,
ylabel={ACC (\%)},
ymin=0.207334596818869, ymax=0.826252910877684,
ytick style={color=black},
ytick={0.2, 0.3, 0.4, 0.5, 0.6, 0.7, 0.8},
yticklabels={20,30,40,50,60,70,80}
]
\path [fill=color0, fill opacity=0.2, line width=1pt]
(axis cs:1,0.239599034372759)
--(axis cs:1,0.270952805627241)
--(axis cs:2,0.361675353904186)
--(axis cs:3,0.490514412785041)
--(axis cs:4,0.65900150682517)
--(axis cs:5,0.73319342111661)
--(axis cs:5,0.58319113888339)
--(axis cs:5,0.58319113888339)
--(axis cs:4,0.53162081317483)
--(axis cs:3,0.416339587214959)
--(axis cs:2,0.320832646095814)
--(axis cs:1,0.239599034372759)
--cycle;

\path [fill=color1, fill opacity=0.2, line width=1pt]
(axis cs:1,0.240663741878652)
--(axis cs:1,0.261369858121347)
--(axis cs:2,0.372254728113571)
--(axis cs:3,0.50465675646231)
--(axis cs:4,0.691481387295536)
--(axis cs:5,0.790365097244895)
--(axis cs:5,0.764244422755105)
--(axis cs:5,0.764244422755105)
--(axis cs:4,0.670328772704464)
--(axis cs:3,0.46668260353769)
--(axis cs:2,0.331055991886429)
--(axis cs:1,0.240663741878652)
--cycle;

\path [fill=color2, fill opacity=0.2, line width=1pt]
(axis cs:1,0.235467247457906)
--(axis cs:1,0.265579712542094)
--(axis cs:2,0.360101980104783)
--(axis cs:3,0.497597244200565)
--(axis cs:4,0.513972718801836)
--(axis cs:5,0.547238328540204)
--(axis cs:5,0.529725671459795)
--(axis cs:5,0.529725671459795)
--(axis cs:4,0.482284241198164)
--(axis cs:3,0.406836115799435)
--(axis cs:2,0.317198179895217)
--(axis cs:1,0.235467247457906)
--cycle;

\path [fill=color3, fill opacity=0.2, line width=1pt]
(axis cs:1,0.260097861289978)
--(axis cs:1,0.303615570068359)
--(axis cs:2,0.40355184674263)
--(axis cs:3,0.532415688037872)
--(axis cs:4,0.690707743167877)
--(axis cs:5,0.798120260238647)
--(axis cs:5,0.764758944511414)
--(axis cs:5,0.764758944511414)
--(axis cs:4,0.646546065807343)
--(axis cs:3,0.483370840549469)
--(axis cs:2,0.388861387968063)
--(axis cs:1,0.260097861289978)
--cycle;

\addplot [line width=1.0pt, color0, mark=*, mark size=1, mark options={solid}]
table {%
1 0.25527592
2 0.341254
3 0.453427
4 0.59531116
5 0.65819228
};
\addlegendentry{Random}
\addplot [line width=1.0pt, color1, mark=*, mark size=1, mark options={solid}]
table {%
1 0.2510168
2 0.35165536
3 0.48566968
4 0.68090508
5 0.77730476
};
\addlegendentry{ETS}
\addplot [line width=1.0pt, color2, mark=*, mark size=1, mark options={solid}]
table {%
1 0.25052348
2 0.33865008
3 0.45221668
4 0.49812848
5 0.538482
};
\addlegendentry{Heur-GD}
\addplot [line width=1.0pt, color3, mark=*, mark size=1, mark options={solid}]
table {%
1 0.281856715679169
2 0.396206617355347
3 0.507893264293671
4 0.66862690448761
5 0.781439602375031
};
\addlegendentry{MCTS}


\nextgroupplot[title=Split FashionMNIST,
height=\figheight,
legend cell align={left},
legend style={
  fill opacity=0.8,
  draw opacity=1,
  text opacity=1,
  at={(0.97,0.03)},
  anchor=south east,
  draw=white!80!black
},
minor xtick={},
minor ytick={},
tick align=outside,
tick pos=left,
width=\figwidth,
x grid style={white!69.0196078431373!black},
xlabel={},%{Memory size \(\displaystyle M\)},
xmajorgrids,
xmin=0.8, xmax=5.2,
xtick style={color=black},
xtick={1,2,3,4,5},
xticklabel style={rotate=90},
xticklabels={10,20,40,100,200},
y grid style={white!69.0196078431373!black},
ymajorgrids,
ylabel={},%{ACC (\%)},
ymin=0.259424595274879, ymax=0.720812421552797,
ytick style={color=black},
ytick={0.2, 0.3, 0.4, 0.5, 0.6, 0.7},
yticklabels={20,30,40,50,60,70}
]
\path [fill=color0, fill opacity=0.2, line width=1pt]
(axis cs:1,0.280396769196603)
--(axis cs:1,0.335123230803398)
--(axis cs:2,0.407187817882043)
--(axis cs:3,0.48739351280264)
--(axis cs:4,0.592706736474621)
--(axis cs:5,0.663078371171996)
--(axis cs:5,0.559201628828004)
--(axis cs:5,0.559201628828004)
--(axis cs:4,0.526493263525379)
--(axis cs:3,0.42820648719736)
--(axis cs:2,0.348852182117957)
--(axis cs:1,0.280396769196603)
--cycle;

\path [fill=color1, fill opacity=0.2, line width=1pt]
(axis cs:1,0.283089035628021)
--(axis cs:1,0.330270964371979)
--(axis cs:2,0.403394694538454)
--(axis cs:3,0.46754458659089)
--(axis cs:4,0.602092782062309)
--(axis cs:5,0.672530100400932)
--(axis cs:5,0.657589899599068)
--(axis cs:5,0.657589899599068)
--(axis cs:4,0.583147217937691)
--(axis cs:3,0.44329541340911)
--(axis cs:2,0.371885305461546)
--(axis cs:1,0.283089035628021)
--cycle;

\path [fill=color2, fill opacity=0.2, line width=1pt]
(axis cs:1,0.286364363477931)
--(axis cs:1,0.349795636522069)
--(axis cs:2,0.444960565648486)
--(axis cs:3,0.532675570275331)
--(axis cs:4,0.610988944594497)
--(axis cs:5,0.550032759051677)
--(axis cs:5,0.543687240948323)
--(axis cs:5,0.543687240948323)
--(axis cs:4,0.551771055405503)
--(axis cs:3,0.50080442972467)
--(axis cs:2,0.405359434351514)
--(axis cs:1,0.286364363477931)
--cycle;

\path [fill=color3, fill opacity=0.2, line width=1pt]
(axis cs:1,0.310655891895294)
--(axis cs:1,0.364784121513367)
--(axis cs:2,0.464605748653412)
--(axis cs:3,0.542859613895416)
--(axis cs:4,0.633972704410553)
--(axis cs:5,0.699840247631073)
--(axis cs:5,0.647799789905548)
--(axis cs:5,0.647799789905548)
--(axis cs:4,0.585347354412079)
--(axis cs:3,0.483660370111465)
--(axis cs:2,0.420714259147644)
--(axis cs:1,0.310655891895294)
--cycle;

\addplot [line width=1.0pt, color0, mark=*, mark size=1, mark options={solid}]
table {%
1 0.30776
2 0.37802
3 0.4578
4 0.5596
5 0.61114
};
%\addlegendentry{Random}
\addplot [line width=1.0pt, color1, mark=*, mark size=1, mark options={solid}]
table {%
1 0.30668
2 0.38764
3 0.45542
4 0.59262
5 0.66506
};
%\addlegendentry{ETS}
\addplot [line width=1.0pt, color2, mark=*, mark size=1, mark options={solid}]
table {%
1 0.31808
2 0.42516
3 0.51674
4 0.58138
5 0.54686
};
%\addlegendentry{Heuristic GD}
\addplot [line width=1.0pt, color3, mark=*, mark size=1, mark options={solid}]
table {%
1 0.33772000670433
2 0.442660003900528
3 0.513260006904602
4 0.609660029411316
5 0.673820018768311
};
%\addlegendentry{RS-MCTS}

\nextgroupplot[title=Split notMNIST,
height=\figheight,
legend cell align={left},
legend style={
  fill opacity=0.8,
  draw opacity=1,
  text opacity=1,
  at={(0.97,0.03)},
  anchor=south east,
  draw=white!80!black
},
minor xtick={},
minor ytick={},
tick align=outside,
tick pos=left,
width=\figwidth,
x grid style={white!69.0196078431373!black},
xlabel={},%{Memory size \(\displaystyle M\)},
xmajorgrids,
xmin=0.8, xmax=5.2,
xtick style={color=black},
xtick={1,2,3,4,5},
xticklabel style={rotate=90},
xticklabels={10,20,40,100,200},
y grid style={white!69.0196078431373!black},
ymajorgrids,
ylabel={},%{ACC (\%)},
ymin=0.225458743774345, ymax=0.835672401123413,
ytick style={color=black},
ytick={0.2, 0.3, 0.4, 0.5, 0.6, 0.7, 0.8},
yticklabels={20,30,40,50,60,70,80}
]
\path [fill=color0, fill opacity=0.2, line width=1pt]
(axis cs:1,0.253195728199302)
--(axis cs:1,0.356370351800698)
--(axis cs:2,0.440226502061888)
--(axis cs:3,0.586615478105434)
--(axis cs:4,0.68543926490655)
--(axis cs:5,0.727179797674989)
--(axis cs:5,0.614241002325012)
--(axis cs:5,0.614241002325012)
--(axis cs:4,0.61583833509345)
--(axis cs:3,0.527461881894566)
--(axis cs:2,0.415446297938112)
--(axis cs:1,0.253195728199302)
--cycle;

\path [fill=color1, fill opacity=0.2, line width=1pt]
(axis cs:1,0.25642352035413)
--(axis cs:1,0.37705271964587)
--(axis cs:2,0.511336372802685)
--(axis cs:3,0.619436120812713)
--(axis cs:4,0.741949565448152)
--(axis cs:5,0.773916340300466)
--(axis cs:5,0.769029259699534)
--(axis cs:5,0.769029259699534)
--(axis cs:4,0.709001874551848)
--(axis cs:3,0.541882599187286)
--(axis cs:2,0.392921947197315)
--(axis cs:1,0.25642352035413)
--cycle;

\path [fill=color2, fill opacity=0.2, line width=1pt]
(axis cs:1,0.263747909513255)
--(axis cs:1,0.333500890486745)
--(axis cs:2,0.488109228342112)
--(axis cs:3,0.60002014707097)
--(axis cs:4,0.602947658076147)
--(axis cs:5,0.603284795039121)
--(axis cs:5,0.537684164960879)
--(axis cs:5,0.537684164960879)
--(axis cs:4,0.562598421923853)
--(axis cs:3,0.47801193292903)
--(axis cs:2,0.414494531657888)
--(axis cs:1,0.263747909513255)
--cycle;

\path [fill=color3, fill opacity=0.2, line width=1pt]
(axis cs:1,0.337080836296082)
--(axis cs:1,0.407582759857178)
--(axis cs:2,0.486506044864655)
--(axis cs:3,0.657773017883301)
--(axis cs:4,0.747565984725952)
--(axis cs:5,0.807935416698456)
--(axis cs:5,0.710279047489166)
--(axis cs:5,0.710279047489166)
--(axis cs:4,0.676063060760498)
--(axis cs:3,0.593173265457153)
--(axis cs:2,0.470134317874908)
--(axis cs:1,0.337080836296082)
--cycle;

\addplot [line width=1.0pt, color0, mark=*, mark size=1, mark options={solid}]
table {%
1 0.30478304
2 0.4278364
3 0.55703868
4 0.6506388
5 0.6707104
};
%\addlegendentry{Random}
\addplot [line width=1.0pt, color1, mark=*, mark size=1, mark options={solid}]
table {%
1 0.31673812
2 0.45212916
3 0.58065936
4 0.72547572
5 0.7714728
};
%\addlegendentry{ETS}
\addplot [line width=1.0pt, color2, mark=*, mark size=1, mark options={solid}]
table {%
1 0.2986244
2 0.45130188
3 0.53901604
4 0.58277304
5 0.57048448
};
%\addlegendentry{Heuristic GD}
\addplot [line width=1.0pt, color3, mark=*, mark size=1, mark options={solid}]
table {%
1 0.37233179807663
2 0.478320181369781
3 0.625473141670227
4 0.711814522743225
5 0.759107232093811
};
%\addlegendentry{RS-MCTS}

\nextgroupplot[title=Split CIFAR-100,
height=\figheight,
legend cell align={left},
legend style={
  fill opacity=0.8,
  draw opacity=1,
  text opacity=1,
  at={(0.97,0.03)},
  anchor=south east,
  draw=white!80!black
},
minor xtick={},
minor ytick={0.025, 0.075, 0.125, 0.175, 0.225},
tick align=outside,
tick pos=left,
width=\figwidth,
x grid style={white!69.0196078431373!black},
xmajorgrids,
xlabel={Memory size \(\displaystyle M\)},
xmin=0.8, xmax=5.2,
xtick style={color=black},
xtick={1,2,3,4,5},
xticklabel style={rotate=90},
xticklabels={100,200,400,800,1600},
y grid style={white!69.0196078431373!black},
ymajorgrids, yminorgrids,
ylabel={ACC (\%)},
ymin=0.0369820315859416, ymax=0.186984212534666,
ytick style={color=black},
ytick={0.0,0.05,0.10,0.15,0.20,0.25},
yticklabels={0,5,10,15,20,25}
]
\path [fill=color0, fill opacity=0.2, line width=1pt]
(axis cs:1,0.0445679784952452)
--(axis cs:1,0.0453120215047548)
--(axis cs:2,0.0536421292893962)
--(axis cs:3,0.073841679435899)
--(axis cs:4,0.111484486208065)
--(axis cs:5,0.162301395207369)
--(axis cs:5,0.148538604792631)
--(axis cs:5,0.148538604792631)
--(axis cs:4,0.101315513791935)
--(axis cs:3,0.068838320564101)
--(axis cs:2,0.0511978707106038)
--(axis cs:1,0.0445679784952452)
--cycle;

\path [fill=color1, fill opacity=0.2, line width=1pt]
(axis cs:1,0.0438003125381563)
--(axis cs:1,0.0463596874618437)
--(axis cs:2,0.051187725463806)
--(axis cs:3,0.0659694979413528)
--(axis cs:4,0.0933042529486208)
--(axis cs:5,0.143101874451005)
--(axis cs:5,0.136698125548995)
--(axis cs:5,0.136698125548995)
--(axis cs:4,0.0892557470513792)
--(axis cs:3,0.0609905020586472)
--(axis cs:2,0.048332274536194)
--(axis cs:1,0.0438003125381563)
--cycle;

\path [fill=color2, fill opacity=0.2, line width=1pt]
(axis cs:1,0.0442478019326001)
--(axis cs:1,0.0467521980673999)
--(axis cs:2,0.053592546295606)
--(axis cs:3,0.0826581724800806)
--(axis cs:4,0.117993399620744)
--(axis cs:5,0.175061735276269)
--(axis cs:5,0.138178264723731)
--(axis cs:5,0.138178264723731)
--(axis cs:4,0.108606600379256)
--(axis cs:3,0.0706218275199194)
--(axis cs:2,0.050487453704394)
--(axis cs:1,0.0442478019326001)
--cycle;

\path [fill=color3, fill opacity=0.2, line width=1pt]
(axis cs:1,0.0451920330524445)
--(axis cs:1,0.0472079664468765)
--(axis cs:2,0.0600229986011982)
--(axis cs:3,0.0851799994707108)
--(axis cs:4,0.122953526675701)
--(axis cs:5,0.180165931582451)
--(axis cs:5,0.171714082360268)
--(axis cs:5,0.171714082360268)
--(axis cs:4,0.118206478655338)
--(axis cs:3,0.080939993262291)
--(axis cs:2,0.0568170063197613)
--(axis cs:1,0.0451920330524445)
--cycle;

\addplot [line width=1.0pt, color0, mark=*, mark size=1, mark options={solid}]
table {%
1 0.04494
2 0.05242
3 0.07134
4 0.1064
5 0.15542
};
%\addlegendentry{Random}
\addplot [line width=1.0pt, color1, mark=*, mark size=1, mark options={solid}]
table {%
1 0.04508
2 0.04976
3 0.06348
4 0.09128
5 0.1399
};
%\addlegendentry{ETS}
\addplot [line width=1.0pt, color2, mark=*, mark size=1, mark options={solid}]
table {%
1 0.0455
2 0.05204
3 0.07664
4 0.1133
5 0.15662
};
%\addlegendentry{Heuristic GD}
\addplot [line width=1.0pt, color3, mark=*, mark size=1, mark options={solid}]
table {%
1 0.0461999997496605
2 0.0584200024604797
3 0.0830599963665009
4 0.12058000266552
5 0.175940006971359
};
%\addlegendentry{RS-MCTS}

\nextgroupplot[title=Split miniImagenet,
height=\figheight,
legend cell align={left},
legend style={
  fill opacity=0.8,
  draw opacity=1,
  text opacity=1,
  at={(0.97,0.03)},
  anchor=south east,
  draw=white!80!black
},
minor xtick={},
minor ytick={0.025, 0.075, 0.125, 0.175},
tick align=outside,
tick pos=left,
width=\figwidth,
x grid style={white!69.0196078431373!black},
xlabel={Memory size \(\displaystyle M\)},
xmajorgrids,
xmin=0.8, xmax=5.2,
xtick style={color=black},
xticklabel style={rotate=90},
xtick={1,2,3,4,5},
xticklabels={100,200,400,800,1600},
y grid style={white!69.0196078431373!black},
ymajorgrids, yminorgrids,
ylabel={},%{ACC},
ymin=0.0250194875900794, ymax=0.163680961167506,
ytick style={color=black},
ytick={0.0,0.05,0.1,0.15,0.2},
yticklabels={0,5,10,15,20}
]
\path [fill=color0, fill opacity=0.2, line width=1pt]
(axis cs:1,0.0313222818435987)
--(axis cs:1,0.0358777181564013)
--(axis cs:2,0.0383538253904577)
--(axis cs:3,0.0520830291444803)
--(axis cs:4,0.094789623249212)
--(axis cs:5,0.150202973465274)
--(axis cs:5,0.124237026534726)
--(axis cs:5,0.124237026534726)
--(axis cs:4,0.073290376750788)
--(axis cs:3,0.0441569708555197)
--(axis cs:2,0.0335661746095423)
--(axis cs:1,0.0313222818435987)
--cycle;

\path [fill=color1, fill opacity=0.2, line width=1pt]
(axis cs:1,0.0313590386836849)
--(axis cs:1,0.0355209613163151)
--(axis cs:2,0.03886513344562)
--(axis cs:3,0.045034926708348)
--(axis cs:4,0.0766174212255784)
--(axis cs:5,0.135689281611291)
--(axis cs:5,0.107710718388709)
--(axis cs:5,0.107710718388709)
--(axis cs:4,0.0663825787744216)
--(axis cs:3,0.041965073291652)
--(axis cs:2,0.03605486655438)
--(axis cs:1,0.0313590386836849)
--cycle;

\path [fill=color2, fill opacity=0.2, line width=1pt]
(axis cs:1,0.0314855663138367)
--(axis cs:1,0.0343544336861633)
--(axis cs:2,0.0430923381698826)
--(axis cs:3,0.0774133476640739)
--(axis cs:4,0.0958946944283004)
--(axis cs:5,0.0974744892773979)
--(axis cs:5,0.0704055107226021)
--(axis cs:5,0.0704055107226021)
--(axis cs:4,0.0729053055716996)
--(axis cs:3,0.0619066523359261)
--(axis cs:2,0.0359476618301174)
--(axis cs:1,0.0314855663138367)
--cycle;

\path [fill=color3, fill opacity=0.2, line width=1pt]
(axis cs:1,0.0332380868494511)
--(axis cs:1,0.0370019115507603)
--(axis cs:2,0.0444378703832626)
--(axis cs:3,0.0833664909005165)
--(axis cs:4,0.128646552562714)
--(axis cs:5,0.157378166913986)
--(axis cs:5,0.150621831417084)
--(axis cs:5,0.150621831417084)
--(axis cs:4,0.102633446455002)
--(axis cs:3,0.0694335028529167)
--(axis cs:2,0.0406821295619011)
--(axis cs:1,0.0332380868494511)
--cycle;

\addplot [line width=1.0pt, color0, mark=*, mark size=1, mark options={solid}]
table {%
1 0.0336
2 0.03596
3 0.04812
4 0.08404
5 0.13722
};
%\addlegendentry{Random}
\addplot [line width=1.0pt, color1, mark=*, mark size=1, mark options={solid}]
table {%
1 0.03344
2 0.03746
3 0.0435
4 0.0715
5 0.1217
};
%\addlegendentry{ETS}
\addplot [line width=1.0pt, color2, mark=*, mark size=1, mark options={solid}]
table {%
1 0.03292
2 0.03952
3 0.06966
4 0.0844
5 0.08394
};
%\addlegendentry{Heuristic GD}
\addplot [line width=1.0pt, color3, mark=*, mark size=1, mark options={solid}]
table {%
1 0.0351199992001057
2 0.0425599999725819
3 0.0763999968767166
4 0.115639999508858
5 0.153999999165535
};
%\addlegendentry{RS-MCTS}




\end{groupplot}

\end{tikzpicture}
  \vspace{-3mm}
  \caption{Average test accuracies over tasks after learning the final task (ACC) over different replay memory sizes $M$ for the RS-MCTS (Ours) and the Random, ETS, and Heuristic baselines on all datasets.
  All results have been averaged over 5 seeds. The results show that replay scheduling can outperform replaying with random and ETS schedules, %equal task proportions,
  especially for small $M$, on both small and large datasets across different backbone choices. Furthermore, our method requires less careful tuning than the Heuristic baseline as $M$ increases.
  }
  \vspace{-3mm}
  \label{fig:acc_over_replay_memory_size}
\end{figure}



%%%% Results of ACC over memory size



\subsection{Results with Varying Replay Memory Size}
\label{sec:results_with_varying_replay_memory_size}
We show that our method can improve the performance across different choices of memory size.
In this experiment, we set the replay memory size to ensure the ETS baseline replays an equal number of samples per class at the final task. 
The memory size is set to $M = n_{cpt} \cdot n_{spc} \cdot (T-1)$, where $n_{cpt}$ is the number of classes per task in the dataset and $n_{spc}$ are the number of samples per class we wish to replay at task $T$ for the ETS baseline. 
In Figure \ref{fig:acc_over_replay_memory_size}, we observe that RS-MCTS obtains better task accuracies than ETS, especially for small memory sizes. Both RS-MCTS and ETS perform better than Heuristic as $M$ increases showing that Heuristic requires careful tuning of the validation accuracy threshold. 






%%% Table 2, put it here for placement

\begin{table}[t]
\footnotesize
\centering
\caption{
    Performance comparison with ACC between scheduling methods RS-MCTS (Ours), Random, ETS, and Heuristic combined with replay-based methods HAL, MER, and DER. We use 'S' and 'P' as short for 'Split' and 'Permuted' for the datasets. 
    Replay memory sizes are $M=10$ and $M=100$ for the 5-task and 10/20-task datasets respectively. We report the mean and standard deviation averaged over 5 seeds. Results on Heuristic where some seed did not converge is denoted by $^{*}$. Applying RS-MCTS to each method can enhance the performance compared to using the baseline schedules. 
}
\vspace{-3mm}
%\setlength{\tabcolsep}{5pt}
\scalebox{0.67}{
\begin{tabular}{l l c c c c c c}
    \toprule
     & & \multicolumn{3}{c}{{\bf 5-task Datasets}} & \multicolumn{3}{c}{{\bf 10- and 20-task Datasets}} \\
    \cmidrule(lr){3-5} \cmidrule(lr){6-8}
    {\bf Method} & {\bf Schedule}  & S-MNIST & S-FashionMNIST & S-notMNIST & P-MNIST & S-CIFAR-100 & S-miniImagenet \\
    \midrule
    \multirow{4}{*}{HAL} & Random & 97.24 {\scriptsize ($\pm$ 0.70)}  & 86.74 {\scriptsize ($\pm$ 6.05)} & 93.61 {\scriptsize($\pm$ 1.31)}  & 88.49 {\scriptsize ($\pm$ 0.99)}  & 36.09 {\scriptsize ($\pm$ 1.77)} & 38.51 {\scriptsize ($\pm$ 2.22)} \\
     & ETS  & 94.02 {\scriptsize ($\pm$ 4.25)} &  95.81 {\scriptsize ($\pm$ 3.53)} & 91.01 {\scriptsize ($\pm$ 1.39)} & 88.46 {\scriptsize ($\pm$ 0.86)}  & 34.90 {\scriptsize ($\pm$ 2.02)} & 38.13 {\scriptsize ($\pm$ 1.18)} \\
    & Heuristic & 97.69 {\scriptsize ($\pm$ 0.19)} & $^{*}$74.16 {\scriptsize ($\pm$ 11.19)} & 93.64 {\scriptsize ($\pm$ 0.93)} & $^{*}$66.63 {\scriptsize ($\pm$ 28.50)} & 35.07 {\scriptsize($\pm$ 1.29)} & 39.51 {\scriptsize($\pm$ 1.49)} \\
     & Ours &  {\bf 97.93} {\scriptsize ($\pm$ 0.56)} &  {\bf 98.27} {\scriptsize ($\pm$ 0.17)} & {\bf 94.64} {\scriptsize ($\pm$ 0.39)} & {\bf 89.14} {\scriptsize ($\pm$ 0.74)}  & {\bf 40.22} {\scriptsize ($\pm$ 1.57)} & {\bf 41.39} {\scriptsize ($\pm$ 1.15)} \\
    \midrule 
    \multirow{4}{*}{MER} & Random & 93.07 {\scriptsize ($\pm$ 0.81)} & 85.53 {\scriptsize ($\pm$ 3.30)} & 91.13 {\scriptsize ($\pm$ 0.86)} & 75.90 {\scriptsize ($\pm$ 1.34)} & 42.96 {\scriptsize ($\pm$ 1.70)} & 31.48 {\scriptsize ($\pm$ 1.65)}  \\ 
     & ETS  & 92.89 {\scriptsize ($\pm$ 3.53)} & 96.47 {\scriptsize ($\pm$ 0.85)} & {\bf 93.80} {\scriptsize ($\pm$ 0.82)} & 73.01 {\scriptsize ($\pm$ 0.96)} & 43.38 {\scriptsize ($\pm$ 1.81)} & 33.58 {\scriptsize ($\pm$ 1.53)} \\
    & Heuristic & 94.30 {\scriptsize ($\pm$ 2.79)} & 96.91 {\scriptsize ($\pm$ 0.62)} & 90.90 {\scriptsize ($\pm$ 1.30)} & {\bf 83.86} {\scriptsize ($\pm$ 3.19)} & 40.90 {\scriptsize ($\pm$ 1.70)}  & {\bf 34.22} {\scriptsize ($\pm$ 1.93)} \\
    & Ours &  {\bf 98.20} {\scriptsize ($\pm$ 0.16)} & {\bf 98.48} {\scriptsize ($\pm$ 0.26)} &  93.61 {\scriptsize ($\pm$ 0.71)} & 79.72 {\scriptsize ($\pm$ 0.71)} & {\bf 44.29} {\scriptsize ($\pm$ 0.69)} & 32.74 {\scriptsize ($\pm$ 1.29)} \\  
    \midrule 
    \multirow{4}{*}{DER} & Random & 98.23 {\scriptsize ($\pm$ 0.53)} & 96.56 {\scriptsize ($\pm$ 1.79)}  & 92.89 {\scriptsize ($\pm$ 0.86)} & 87.51 {\scriptsize ($\pm$ 1.10)} & 56.83 {\scriptsize ($\pm$ 0.76)} & 42.19  {\scriptsize ($\pm$ 0.67)} \\ 
     & ETS  & 98.17 {\scriptsize ($\pm$ 0.35)} & 97.69 {\scriptsize ($\pm$ 0.58)} & 94.74 {\scriptsize ($\pm$ 1.05)} & 85.71 {\scriptsize ($\pm$ 0.75)} & 52.58 {\scriptsize ($\pm$ 1.49)} & 35.50  {\scriptsize ($\pm$ 2.84)} \\
    & Heuristic & 94.57 {\scriptsize ($\pm$ 1.71)} & $^{*}$72.49 {\scriptsize ($\pm$ 19.32)} & $^{*}$77.88 {\scriptsize ($\pm$ 12.58)} & 81.56 {\scriptsize ($\pm$ 2.28)} & 55.75 {\scriptsize ($\pm$ 1.08)} & {\bf 43.62} {\scriptsize ($\pm$ 0.88)} \\
     & Ours & {\bf 99.02} {\scriptsize ($\pm$ 0.10)} & {\bf 98.33} {\scriptsize ($\pm$ 0.51)} & {\bf 95.02} {\scriptsize ($\pm$ 0.33)} & {\bf 90.11} {\scriptsize ($\pm$ 0.18)} & {\bf 58.99} {\scriptsize ($\pm$ 0.98)} & 43.46  {\scriptsize ($\pm$ 0.95)}  \\  
    %\midrule 
    %\multirow{4}{*}{DER++} & Random & 97.90 {\scriptsize ($\pm$ 0.52)} & 97.10 {\scriptsize ($\pm$ 1.03)}  & 93.29 {\scriptsize ($\pm$ 1.43)} & 87.89 {\scriptsize ($\pm$ 1.10)} & 58.49  {\scriptsize ($\pm$ 1.44)} & 48.40  {\scriptsize ($\pm$ 0.69)} \\ 
    % & ETS & 97.98 {\scriptsize ($\pm$ 0.52)} & 98.12 {\scriptsize ($\pm$ 0.40)} & 94.53 {\scriptsize ($\pm$ 1.02)} & 85.25 {\scriptsize ($\pm$ 0.88)} & 52.54  {\scriptsize ($\pm$ 1.06)} & 41.36  {\scriptsize ($\pm$ 2.90)} \\
    %& Heuristic & 92.35 {\scriptsize ($\pm$ 2.42)} & $^{*}$67.31 {\scriptsize ($\pm$ 21.20)} & 93.88 {\scriptsize ($\pm$ 1.33)} & 79.17 {\scriptsize ($\pm$ 2.44)} & 56.70 {\scriptsize ($\pm$ 1.27)} & 45.73 {\scriptsize ($\pm$ 0.84)} \\
    % & Ours & {\bf 98.84} {\scriptsize ($\pm$ 0.21)} & {\bf 98.38} {\scriptsize ($\pm$ 0.43)} & {\bf 94.73} {\scriptsize ($\pm$ 0.20)} & {\bf 89.84} {\scriptsize ($\pm$ 0.22)} & {\bf 59.23}  {\scriptsize ($\pm$ 0.83)} & {\bf 49.45}  {\scriptsize ($\pm$ 0.68)} \\
    \bottomrule
\end{tabular}
}
\vspace{-3mm}
\label{tab:results_applying_scheduling_to_recent_replay_methods}
\end{table}


%%% Results on applying replay scheduling to recent replay methods

\subsection{Applying Scheduling to Recent Replay Methods}\label{paperC:sec:applying_scheduling_to_recent_replay_methods}

In this experiment, we show that replay scheduling can be combined with any replay method to enhance the CL performance. We combine RS-MCTS with Hindsight Anchor Learning (HAL)~\citeC:{C:chaudhry2021using}, Meta-Experience Replay (MER)~\citeC:{riemer2018learning}, Dark Experience Replay (DER)~\citeC{C:buzzega2020dark}, and DER++. We provide the hyperparameter settings in Appendix \ref{paperC:app:apply_scheduling_to_recent_replay_methods}. Table \ref{tab:results_applying_scheduling_to_recent_replay_methods} shows the performance comparison between our proposed replay scheduling against using Random, ETS, and Heuristic schedules for each method. The results confirm that replay scheduling %learning the time to learn 
is important for the final performance given the same memory constraints and it can benefit any existing CL framework. 




%%% Efficiency of Replay Scheduling
%\newpage



%%% FIGURE ON MEMORY USAGE 5-TASK DATASETS

\begin{figure}[t]
  \centering
  \setlength{\figwidth}{0.3\textwidth}
  \setlength{\figheight}{.12\textheight}
  \vspace{-2mm}
  

\pgfplotsset{compat=1.11,
    /pgfplots/ybar legend/.style={
    /pgfplots/legend image code/.code={%
       \draw[##1,/tikz/.cd,yshift=-0.25em]
        (0cm,0cm) rectangle (3pt,0.8em);},
   },
}

\pgfplotsset{every tick label/.append style={font=\tiny}}
\pgfplotsset{every major tick/.append style={major tick length=2pt}}

\begin{tikzpicture}
\tikzstyle{every node}=[font=\scriptsize]
\definecolor{color0}{rgb}{0.12156862745098,0.466666666666667,0.705882352941177}
\definecolor{color1}{rgb}{1,0.498039215686275,0.0549019607843137}
\definecolor{color2}{rgb}{0.172549019607843,0.627450980392157,0.172549019607843}
\definecolor{color3}{rgb}{0.83921568627451,0.152941176470588,0.156862745098039}
\definecolor{color4}{rgb}{0.580392156862745,0.403921568627451,0.741176470588235}

\begin{axis}[title={},
ybar,
bar width = 4pt,
height=\figheight,
legend cell align={left},
legend columns=-1,
legend style={
  nodes={scale=0.85},
  fill opacity=0.8,
  draw opacity=1,
  text opacity=1,
  at={(0.0,1.05)},
  anchor=south west,
  draw=white!80!black
},
%legend columns=2,
%legend style={
%  fill opacity=0.8,
%  draw opacity=1,
%  text opacity=1,
%  at={(0.03,1.05)},
%  anchor=south west,
%  draw=white!80!black
%},
minor xtick={},
minor ytick={},
tick align=outside,
tick pos=left,
width=\figwidth,
x grid style={white!69.0196078431373!black},
xlabel={Task},
x label style={at={(axis description cs:0.5,-0.36)},anchor=north},
xmin=1.5, xmax=5.5,
xtick style={color=black},
xtick={2,3,4,5},
xticklabels={2,3,4,5},
y grid style={white!69.0196078431373!black},
ymajorgrids,
ylabel={\#Samples},
ymin=0.0, ymax=8.7,
ytick style={color=black},
ytick={0,2, 4, 6, 8, 10},
yticklabels={0, 2, 4, 6, 8, 10}
]

\addplot [draw=black, fill=color0] coordinates {
(2,2)
(3,2)
(4,2)
(5,2)
};\addlegendentry{Ours}

\addplot [draw=black, fill=color1] coordinates {
(2,2)
(3,4)
(4,6)
(5,8)
};\addlegendentry{Baselines}

\end{axis}

\end{tikzpicture}
  \vspace{-4mm}
  \caption{
  Number of replayed samples per task for the 5-task datasets in the tiny memory setting. Ours use $M=2$ samples for replay, while the baselines increment their memory per task. %Our method uses a fixed $M=2$ samples for replay, while the baselines increment their memory per task. 
  }
  \vspace{-4mm}
  \label{fig:tiny_memory_experiment_memory_usage}
\end{figure}

\subsection{Efficiency of Replay Scheduling}
\label{sec:efficiency_of_replay_scheduling}



%Here, we 
We illustrate the efficiency of replay scheduling with comparisons to several common replay-based CL baselines in an even more extreme memory setting.
Our goal is to investigate if scheduling over which tasks to replay can be more efficient in situations where the memory size is even smaller than the number of classes. % than replaying all available samples at every task.
To this end, we set the replay memory size for our method
to $M=2$ for the 5-task datasets, such that only 2 samples can be selected for replay at all times. For the 10- and 20-task datasets which have 100 classes, we set $M=50$. We then compare against the most memory efficient CL baselines, namely A-GEM~\citep{chaudhry2018efficient}, ER-Ring~\citep{chaudhry2019tiny} which show promising results with 1 sample per class for replay, % for replay after learning each task, 
and with uniform memory selection as reference. 
Additionally, we compare to using random replay schedules (Random) with the same memory setting as for RS-MCTS.
We visualize the memory usage for our method and the baselines when training on a 5-task dataset in Figure \ref{fig:tiny_memory_experiment_memory_usage}. 
%We visualize the memory usage when training on one of the 5-task datasets for our method and the baselines in Figure \ref{fig:tiny_memory_experiment_memory_usage}. 
Figure \ref{fig:memory_usage_10_and_20task_datasets} in Appendix \ref{app:additional_figures} shows the memory usage for the other datasets.
%We also visualize the memory usages for the other benchmarks in the Appendix \ref{app:additional_figures}.   

Table \ref{tab:efficiency_of_replay_scheduling_all_datasets} shows the ACC for each method across all datasets. Despite using significantly fewer samples for replay, RS-MCTS performs better or on par with the best baselines on all datasets except Split CIFAR-100. These results show that replay scheduling can be even more efficient than the state-of-the-art memory efficient replay-based methods, which indicate that learning the time to learn is an important research direction in CL.

%%% TABLE


%\begin{table}[t]
\footnotesize %\scriptsize
%\centering
%\caption{
Performance comparison with ACC averaged over 5 seeds in the 1 ex/class %example per class 
memory setting evaluated across all datasets. %We use 'S' and 'P' as short for 'Split' and 'Permuted' for the datasets. 
RS-MCTS (Ours) has replay memory size $M=2$, while the baselines replay all available memory samples. 
%RS-MCTS (Ours) has replay memory size $M=2$ and $M=50$ for the 5-task and 10/20-task datasets respectively. The baselines replay all available memory samples. 
%We report the mean and standard deviation averaged over 5 seeds. 
%RS-MCTS performs on par with the best baselines on all datasets except S-CIFAR-100.
%}
%\vspace{-3mm}
\scalebox{0.85}{
\begin{tabular}{l c c c}
    \toprule
     & \multicolumn{3}{c}{{\bf 5-task Datasets}} \\ 
    \cmidrule(lr){2-4} 
    {\bf Method} & S-MNIST & S-FashionMNIST & S-notMNIST \\
    \midrule 
    Random & 92.56 ($\pm$ 2.90) & 92.70 ($\pm$ 3.78) & 89.53 ($\pm$ 3.96)  \\
    A-GEM  & 94.97 ($\pm$ 1.50) & 94.81 ($\pm$ 0.86) & 92.27 ($\pm$ 1.16)   \\
    ER-Ring & 94.94 ($\pm$ 1.56) & 95.83 ($\pm$ 2.15) & 91.10 ($\pm$ 1.89)   \\
    Uniform &  95.77 ($\pm$ 1.12) & 97.12 ($\pm$ 1.57) & 92.14 ($\pm$ 1.45)   \\
    \midrule 
    Ours & 96.07 ($\pm$ 1.60) &  97.17 ($\pm$ 0.78) & 93.41 ($\pm$ 1.11)  \\
    %\midrule \midrule 
    %& \multicolumn{3}{c}{{\bf 10- and 20-task Datasets}} \\
    %\cmidrule(lr){2-4}
    %{\bf Method} & P-MNIST & S-CIFAR-100 & S-miniImagenet \\
    %\midrule 
    %Random & 70.02 ($\pm$ 1.76) & 48.62 ($\pm$ 1.02) & 48.85 ($\pm$ 1.38) \\
    %A-GEM & 64.71 ($\pm$ 1.78)  & 42.22 ($\pm$ 2.13) &  32.06 ($\pm$ 1.83) \\ 
    %ER-Ring & 69.73 ($\pm$ 1.13) & 53.93 ($\pm$ 1.13) & 49.82 ($\pm$ 1.69) \\
    %Uniform & 69.85 ($\pm$ 1.01) & 52.63 ($\pm$ 1.62)  & 50.56 ($\pm$ 1.07) \\
    %\midrule
    %Ours & 72.52 ($\pm$ 0.54) & 51.50 ($\pm$ 1.19) & 50.70 ($\pm$ 0.54) \\
    \bottomrule
\end{tabular}
}
%\vspace{-6mm}
\label{tab:efficiency_of_replay_scheduling_all_datasets}
%\end{table}




