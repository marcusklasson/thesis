
\subsection{Monte Carlo Tree Search for Replay Schedules}
\label{paperC:sec:mcts_for_replay_scheduling}

In this section, we describe how we enable using MCTS for studying the benefits of replay scheduling in CL scenarios.
The tree-shaped action space of task proportions described in Section \ref{paperC:sec:replay_scheduling_in_continual_learning} grows fast with the number of tasks, which complicates studying replay scheduling in datasets with longer task-horizons. Since the search space is too big for using exhaustive searches, 
we need a scalable method that enables tree searches in large action spaces.
To this end, we propose to use MCTS since it has been successful in applications with large action spaces~\citeC{C:browne2012survey, C:chaudhry2018feature, C:gelly2006modification, C:silver2016mastering}. In our case, MCTS concentrates the search for replay schedules in directions with promising CL performance in the environment. We use MCTS in an ideal CL setting, wherein multiple episodes are allowed, for demonstration purposes to show that replay scheduling can be critical for the CL performance. 

Each memory composition in the action space corresponds to a node that can be visited by MCTS. For example, in Figure \ref{fig:replay_scheduling_mcts_tree_example}, the nodes correspond to the possible memory examples which can be visited during the MCTS rollouts.
At task $t$, the node $v_t$ is related to a task proportions $\va_{t}$ used for retrieving a replay memory from the historical data. 
We store the related task proportion $\va_t$ from every visited node $v_t$ in the replay schedule $S$. 
The final replay schedule is then used for constructing the replay memories at each task during the CL training.  
Next, we briefly outline the MCTS steps for performing the replay schedule search (more details in Appendix \ref{app:rs_mcts_algorithm}):

\vspace{-1mm}
\begin{itemize}[leftmargin=*, topsep=0pt, noitemsep, label={}]
	\item {\bf Selection.} During a rollout, the current node $v_t$ either moves randomly to unvisited children, or selects the next node by evaluating the Upper Confidence Tree (UCT)~\citeC{C:kocsis2006bandit} if all children has been visited earlier.
	The child $v_{t+1}$ with the highest UCT score is selected using the function from \cite{chaudhry2018feature}:
	\begin{align}\label{eq:uct}
		UCT(v_t, v_{t+1}) = \text{max}(q(v_{t+1})) + C \sqrt{\frac{2 \log(n(v_{t}))}{n(v_{t+1})}},
	\end{align}
	where $q(\cdot)$ is the reward function, $C$ the exploration constant, and $n(\cdot)$ the number of node visits. 
	
	\item {\bf Expansion.} Whenever the current node $v_t$ has unvisited child nodes, the search tree is expanded with one of the unvisited child nodes $v_{t+1}$ selected with uniform sampling. 
	
	\item {\bf Simulation and Reward.} After expansion, the succeeding nodes are selected randomly until reaching a terminal node $v_T$. The task proportions from the visited nodes in the rollout constitutes the replay schedule $S$. After training the network using $S$ for replay, we calculate the reward for the rollout is given by $r = \frac{1}{T} \sum_{i=1}^T A_{T, i}^{(val)}$, where $A_{T, i}^{(val)}$ is the validation accuracy of task $i$ at task $T$.  
	
	\item {\bf Backpropagation.} Reward $r$ is backpropagated from the expanded node $v_t$ to the root $v_1$, where the reward function $q(\cdot)$ and number of visits $n(\cdot)$ are updated at each node. 
\end{itemize}


