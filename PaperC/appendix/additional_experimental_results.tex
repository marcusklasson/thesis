
\section{Additional Experimental Results}
\label{paperC:app:additional_experimental_results}

In this section, we bring more insights to the benefits of replay scheduling in Section \ref{paperC:app:replay_schedule_visualization_for_split_mnist} as well as provide metrics for catastrophic forgetting in Section \ref{paperC:app:analysis_of_catastrophic_forgetting}.

\subsection{Replay Schedule Visualization for Split MNIST}
\label{paperC:app:replay_schedule_visualization_for_split_mnist}

In Figure \ref{fig:split_mnist_task_accuracies_and_bubble_plot}, we show the progress in test classification performance for each task when using ETS and RS-MCTS with memory size $M=10$ on Split MNIST. For comparison, we also show the performance from a network that is fine-tuning on the current task without using replay. Both ETS and RS-MCTS overcome catastrophic forgetting to a large degree compared to the fine-tuning network. Our method RS-MCTS further improves the performance compared to ETS with the same memory, which indicates that learning the time to learn can be more efficient against catastrophic forgetting. Especially, Task 1 and 2 seems to be the most difficult task to remember since it has the lowest final performance using the fine-tuning network. Both ETS and RS-MCTS manage to retain their performance on Task 1 using replay, however, RS-MCTS remembers Task 2 better than ETS by around 5\%. 

To bring more insights to this behavior, we have visualized the task proportions of the replay examples using a bubble plot showing the corresponding replay schedule from RS-MCTS in Figure \ref{fig:split_mnist_task_accuracies_and_bubble_plot}(right). At Task 3 and 4, we see that the schedule fills the memory with data from Task 2 and discards replaying Task 1. This helps the network to retain knowledge about Task 2 better than ETS at the cost of forgetting Task 3 slightly when learning Task 4. This shows that the learned policy has considered the difficulty level of different tasks. At the next task, the RS-MCTS schedule has decided to rehearse Task 3 and reduces replaying Task 2 when learning Task 5. This behavior is similar to spaced repetition, where increasing the time interval between rehearsals helps memory retention.
We emphasize that even on datasets with few tasks, using learned replay schedules can overcome catastrophic forgetting better than standard ETS approaches.



\begin{figure}[t]
  \centering
  \setlength{\figwidth}{0.27\textwidth}
  \setlength{\figheight}{.15\textheight}
  



\pgfplotsset{every axis title/.append style={at={(0.5,0.80)}}}
\pgfplotsset{every tick label/.append style={font=\tiny}}
\pgfplotsset{every major tick/.append style={major tick length=2pt}}
\pgfplotsset{every minor tick/.append style={minor tick length=1pt}}
\pgfplotsset{every axis x label/.append style={at={(0.5,-0.25)}}}
\pgfplotsset{every axis y label/.append style={at={(-0.22,0.5)}}}
\begin{tikzpicture}
\tikzstyle{every node}=[font=\tiny]
\definecolor{color0}{rgb}{0.12156862745098,0.466666666666667,0.705882352941177}
\definecolor{color1}{rgb}{1,0.498039215686275,0.0549019607843137}
\definecolor{color2}{rgb}{0.172549019607843,0.627450980392157,0.172549019607843}
\definecolor{color3}{rgb}{0.83921568627451,0.152941176470588,0.156862745098039}
\definecolor{color4}{rgb}{0.580392156862745,0.403921568627451,0.741176470588235}

\begin{groupplot}[group style={group size= 4 by 1, horizontal sep=1.05cm, vertical sep=0.9cm}]


\nextgroupplot[title=Finetune (ACC: 89.84\%),
height=\figheight,
legend cell align={left},
legend style={
  nodes={scale=0.8},
  fill opacity=0.8,
  draw opacity=1,
  text opacity=1,
  at={(5.27,0.05)},
  anchor=south west,
  draw=white!80!black
},
minor xtick={},
minor ytick={0.55,0.65,0.75,0.85,0.95,1.05},
tick align=outside,
tick pos=left,
width=\figwidth,
x grid style={white!69.0196078431373!black},
xmajorgrids,
xlabel={Task},
xmin=0.8, xmax=5.2,
xtick style={color=black},
xtick={1,2,3,4,5},
xticklabels={1,2,3,4,5},
y grid style={white!69.0196078431373!black},
ymajorgrids,
yminorgrids,
ylabel={Accuracy},
ymin=0.491, ymax=1.01,
ytick style={color=black},
ytick={0.5,0.6,0.7,0.8,0.9,1.0},
yticklabels={50, 60, 70, 80, 90, 100}
]
\addplot [line width=1.5pt, color0, mark=*, mark size=1, mark options={solid}]
table {%
1 0.999527215957642
2 0.994799077510834
3 0.652009427547455
4 0.552718698978424
5 0.712529540061951
};\addlegendentry{Task 1}
\addplot [line width=1.5pt, color1, mark=*, mark size=1, mark options={solid}]
table {%
2 0.996572017669678
3 0.974045038223267
4 0.952987253665924
5 0.81292849779129
};\addlegendentry{Task 2}
\addplot [line width=1.5pt, color2, mark=*, mark size=1, mark options={solid}]
table {%
3 0.997865557670593
4 0.954642474651337
5 0.97438633441925
};\addlegendentry{Task 3}
\addplot [line width=1.5pt, color3, mark=*, mark size=1, mark options={solid}]
table {%
4 0.998489439487457
5 0.996475338935852
};\addlegendentry{Task 4}
\addplot [line width=1.5pt, color4, mark=*, mark size=1, mark options={solid}]
table {%
5 0.995461404323578
};\addlegendentry{Task 5}





\nextgroupplot[title=ETS (ACC: 96.13\%),
height=\figheight,
legend cell align={left},
legend style={
  nodes={scale=0.9},
  fill opacity=0.8,
  draw opacity=1,
  text opacity=1,
  at={(3.72,0.28)},
  anchor=south west,
  draw=white!80!black
},
minor xtick={},
minor ytick={0.55,0.65,0.75,0.85,0.95,1.05},
tick align=outside,
tick pos=left,
width=\figwidth,
x grid style={white!69.0196078431373!black},
xmajorgrids,
xlabel={Task},
xmin=0.8, xmax=5.2,
xtick style={color=black},
xtick={1,2,3,4,5},
xticklabels={1,2,3,4,5},
y grid style={white!69.0196078431373!black},
ymajorgrids,
yminorgrids,
ylabel={Accuracy},
ymin=0.491, ymax=1.01,
ytick style={color=black},
ytick={0.5,0.6,0.7,0.8,0.9,1.0},
yticklabels={50, 60, 70, 80, 90, 100}
]
\addplot [line width=1.5pt, color0, mark=*, mark size=1, mark options={solid}]
table {%
1 1
2 0.998108744621277
3 0.999054372310638
4 0.994326233863831
5 0.949881792068481
};
\addplot [line width=1.5pt, color1, mark=*, mark size=1, mark options={solid}]
table {%
2 0.996082246303558
3 0.957884430885315
4 0.84965717792511
5 0.884427011013031
};
\addplot [line width=1.5pt, color2, mark=*, mark size=1, mark options={solid}]
table {%
3 0.998932778835297
4 0.975453555583954
5 0.982390582561493
};
\addplot [line width=1.5pt, color3, mark=*, mark size=1, mark options={solid}]
table {%
4 0.999496459960938
5 0.993957698345184
};
\addplot [line width=1.5pt, color4, mark=*, mark size=1, mark options={solid}]
table {%
5 0.995965719223022
};


\nextgroupplot[title=Ours (ACC: 98.24\%),
height=\figheight,
legend cell align={left},
legend style={
  nodes={scale=0.9},
  fill opacity=0.8,
  draw opacity=1,
  text opacity=1,
  at={(3.72,0.28)},
  anchor=south west,
  draw=white!80!black
},
minor xtick={},
minor ytick={0.55,0.65,0.75,0.85,0.95,1.05},
tick align=outside,
tick pos=left,
width=\figwidth,
x grid style={white!69.0196078431373!black},
xmajorgrids,
xlabel={Task},
xmin=0.8, xmax=5.2,
xtick style={color=black},
xtick={1,2,3,4,5},
xticklabels={1,2,3,4,5},
y grid style={white!69.0196078431373!black},
ymajorgrids,
yminorgrids,
ylabel={Accuracy},
ymin=0.491, ymax=1.01,
ytick style={color=black},
ytick={0.5,0.6,0.7,0.8,0.9,1.0},
yticklabels={50, 60, 70, 80, 90, 100}
]
\addplot [line width=1.5pt, color0, mark=*, mark size=1, mark options={solid}]
table {%
1 0.999527215957642
2 0.99858158826828
3 0.999527215957642
4 0.996690332889557
5 0.998108744621277
};
\addplot [line width=1.5pt, color1, mark=*, mark size=1, mark options={solid}]
table {%
2 0.994613111019135
3 0.961312413215637
4 0.946131229400635
5 0.953476965427399
};
\addplot [line width=1.5pt, color2, mark=*, mark size=1, mark options={solid}]
table {%
3 0.998399138450623
4 0.961579501628876
5 0.978655278682709
};
\addplot [line width=1.5pt, color3, mark=*, mark size=1, mark options={solid}]
table {%
4 0.99899297952652
5 0.994461238384247
};
\addplot [line width=1.5pt, color4, mark=*, mark size=1, mark options={solid}]
table {%
5 0.992435693740845
};


\nextgroupplot[title={Replay Schedule},
height=\figheight,
minor xtick={},
minor ytick={},
tick align=outside,
tick pos=left,
width=\figwidth,
x grid style={white!69.0196078431373!black},
grid=major,
xlabel={Current Task},
xmin=1.5, xmax=5.5,
xtick style={color=black},
xtick={2,3,4,5},
xticklabels={2,3,4,5},
xticklabel style = {font=\tiny},
y grid style={white!69.0196078431373!black},
ylabel={Replayed Task},
%ylabel style={at={(0.5,0.0)},
ymin=0.45, ymax=4.5,
ytick style={color=black},
ytick={1,2,3,4},
yticklabels={1,2,3,4},
yticklabel style = {font=\tiny},
]
\addplot+[
mark=*,
only marks,
scatter,
scatter src=explicit,
scatter/@pre marker code/.code={%
  \expanded{%
    \noexpand\definecolor{thispointdrawcolor}{RGB}{\drawcolor}%
    \noexpand\definecolor{thispointfillcolor}{RGB}{\fillcolor}%
  }%
  \scope[draw=thispointdrawcolor, fill=thispointfillcolor]%
},
visualization depends on={\thisrow{sizedata}/5 \as \perpointmarksize},
visualization depends on={value \thisrow{draw} \as \drawcolor},
visualization depends on={value \thisrow{fill} \as \fillcolor},
%visualization depends on=
%{5*cos(deg(x)) \as \perpointmarksize},
scatter/@pre marker code/.append style=
{/tikz/mark size=\perpointmarksize}
]
table[x=x, y=y, meta=sizedata]{
x  y  draw  fill  sizedata
2 1 31,119,180 31,119,180 17.841241161527712
3 1 31,119,180 31,119,180 0.0
4 1 31,119,180 31,119,180 0.0
5 1 31,119,180 31,119,180 0.0

3 2 255,127,14 255,127,14 17.841241161527712
4 2 255,127,14 255,127,14 17.841241161527712
5 2 255,127,14 255,127,14 12.6156626101008

4 3 44,160,44 44,160,44 0.0
5 3 44,160,44 44,160,44 12.6156626101008

};

\end{groupplot}

\end{tikzpicture}
  \vspace{-3mm}
  \caption{ Comparison of test classification accuracies for Task 1-5 on Split MNIST from a network trained without replay (Finetune), ETS, and RS-MCTS (Ours). The ACC metric for each method is shown on top of each figure. We also visualize the replay schedule found by RS-MCTS as a bubble plot to the right. The memory size is set to $M=10$ with uniform memory selection for ETS and RS-MCTS. Results are shown for 1 seed. 
  }
  \vspace{-3mm}
  \label{fig:split_mnist_task_accuracies_and_bubble_plot}
\end{figure}



\subsection{Statistical Significance Tests}\label{paperC:app:statistical_significance_tests}

We provide p-values from Welch's t-tests to show the statistical significance between the methods in the experiments on different memory selections methods (Section \ref{paperC:sec:alternative_memory_selection_methods}), applying scheduling to recent replay methods (Section \ref{paperC:sec:applying_scheduling_to_recent_replay_methods}), and efficiency of replay scheduling (Section \ref{paperC:sec:efficiency_of_replay_scheduling}).

\vspace{-3mm}
\paragraph{Memory Selection Methods.} Table \ref{tab:t_test_memory_selection_appendix} presents the statistical significance tests between MCTS and the baselines that corresponds to the performance results in Table \ref{tab:results_memory_selection_methods}. We note that our method in general achieves significantly higher ACC comparing to the baselines showing that learning the time to learn is important.



\vspace{-3mm}
\paragraph{Recent Replay Methods.} Table \ref{tab:t_test_alternative_replay_methods} presents the statistical significance tests between MCTS and the baselines that corresponds to the performance results in Table \ref{tab:results_memory_selection_methods}. We note that our method (MCTS) in general achieves significantly higher ACC compared to the scheduling baselines, which shows that replay scheduling can improve the CL performance when combined with any replay method. 


\vspace{-3mm}
\paragraph{Efficiency of Scheduling.} Table \ref{tab:t_test_efficiency_replay_scheduling} presents the statistical significance tests between MCTS and the baselines that corresponds to the performance results in Table \ref{tab:efficiency_of_replay_scheduling_all_datasets}. Our method (MCTS) mostly performs on par with the baselines, but is significantly better than the baselines on Permuted MNIST, despite that MCTS replays fewer samples than the baselines.




\begin{table}[t]
	\centering
	\caption{Two-tailed Welch's $t$-test results for the various memory selection methods presented in Table \ref{tab:results_memory_selection_methods}. 
	}
	\vspace{-3mm}
	\resizebox{0.98\textwidth}{!}{
		\begin{tabular}{llcccccc}
\toprule
                          &                   & \multicolumn{2}{c}{\textbf{Split MNIST}} & \multicolumn{2}{c}{\textbf{Split FashionMNIST}} & \multicolumn{2}{c}{\textbf{Split notMNIST}} \\
\cmidrule(lr){3-4} \cmidrule(lr){5-6} \cmidrule(lr){7-8}
\textbf{Memory}                    & \textbf{Schedule} & $t$                 & $p$                    & $t$                    & $p$                        & $t$                   & $p$                     \\ \midrule
\multirow{3}{*}{Uniform}  & MCTS vs Random    & 2.34            & 0.074           & 2.34               & 0.078               & 3.66              & \textbf{0.017}            \\
                          & MCTS vs ETS       & 1.82            & 0.140           & 1.39               & 0.236               & 5.04              & \textbf{0.005}            \\
                          & MCTS vs Heuristic   & 1.60            & 0.178           & 3.64               & \textbf{0.017}               & 1.69              & 0.166            \\ \midrule
\multirow{3}{*}{k-means}  & MCTS vs Random    & 7.97            & \textbf{0.001}           & 3.90               & \textbf{0.017}               & 0.81              & 0.445            \\
                          & MCTS vs ETS       & 3.00            & \textbf{0.040}           & 4.54               & \textbf{0.007}               & -0.36             & 0.732            \\
                          & MCTS vs Heuristic   & 2.27            & 0.085           & 3.55               & \textbf{0.022}               & 3.15              & \textbf{0.015}            \\ \midrule
\multirow{3}{*}{k-center} & MCTS vs Random    & 6.15            & \textbf{0.001}           & 3.37               & \textbf{0.027}               & 2.59              & 0.057            \\
                          & MCTS vs ETS       & 4.71            & \textbf{0.007}           & 2.56               & \textbf{0.037}               & 2.53              & 0.062            \\
                          & MCTS vs Heuristic   & 2.62            & 0.057           & 2.13               & 0.099               & 3.51              & \textbf{0.019}            \\ \midrule
\multirow{3}{*}{MoF}      & MCTS vs Random    & 2.07            & 0.103           & 1.68               & 0.163               & 2.10              & 0.093            \\
                          & MCTS vs ETS       & 2.13            & 0.095           & 1.99               & 0.110               & 4.11              & \textbf{0.007}            \\
                          & MCTS vs Heuristic   & 1.58            & 0.188           & 4.21               & \textbf{0.008}               & 3.15              & \textbf{0.019}      \\
\bottomrule \toprule
                          &                   & \multicolumn{2}{c}{\textbf{Permuted MNIST}} & \multicolumn{2}{c}{\textbf{Split CIFAR-100}} & \multicolumn{2}{c}{\textbf{Split miniImagenet}} \\ 
\cmidrule(lr){3-4} \cmidrule(lr){5-6} \cmidrule(lr){7-8}
\textbf{Memory}           & \textbf{Schedule} & $t$                   & $p$                     & $t$                   & $p$                      & $t$                     & $p$                       \\ \midrule
\multirow{3}{*}{Uniform}  & MCTS vs Random    & 4.14              & \textbf{0.005}            & 2.67              & \textbf{0.033}             & 0.49                & 0.636              \\
                          & MCTS vs ETS       & 4.18              & \textbf{0.007}            & 7.30              & \textbf{0.000}             & 4.52                & \textbf{0.003}              \\ 
                          & MCTS vs Heuristic   & -0.29             & 0.780            & -0.87             & 0.412             & 0.82                & 0.441              \\ \midrule
\multirow{3}{*}{k-means}  & MCTS vs Random    & 7.91              & \textbf{0.000}            & 4.39              & \textbf{0.003}             & 0.61                & 0.565              \\
                          & MCTS vs ETS       & 10.83             & \textbf{0.000}            & 12.93             & \textbf{0.000}             & 7.42                & \textbf{0.000}              \\
                          & MCTS vs Heuristic   & 3.05              & \textbf{0.019}            & 1.31              & 0.255             & 1.87                & 0.099              \\ \midrule
\multirow{3}{*}{k-center} & MCTS vs Random    & 4.70              & \textbf{0.003}            & 2.32              & 0.051             & 2.85                & \textbf{0.024}              \\
                          & MCTS vs ETS       & 7.25              & \textbf{0.000}            & 7.46              & \textbf{0.000}             & 6.94                & \textbf{0.000}              \\
                          & MCTS vs Heuristic   & 1.92              & 0.100            & 0.82              & 0.437             & 3.89                & \textbf{0.005}              \\ \midrule
\multirow{3}{*}{MoF}      & MCTS vs Random    & 4.26              & \textbf{0.004}            & 2.67              & \textbf{0.037}             & 2.75                & \textbf{0.036}              \\
                          & MCTS vs ETS       & 5.86              & \textbf{0.001}            & 5.69              & \textbf{0.001}             & 2.57                & \textbf{0.045}              \\
                          & MCTS vs Heuristic   & 5.26              & \textbf{0.002}            & 6.22              & \textbf{0.002}             & 13.90               & \textbf{0.000}   \\
\bottomrule
\end{tabular}
	}
	\vspace{-3mm}
	\label{tab:t_test_memory_selection_appendix}
\end{table}


\begin{table}[t]
	\centering
	\caption{Two-tailed Welch's $t$-test results for the alternative replay methods results presented in Table \ref{tab:results_applying_scheduling_to_recent_replay_methods}. 
	}
	\vspace{-3mm}
	\resizebox{0.98\textwidth}{!}{
		\begin{tabular}{llcccccc}
\toprule
                       &                   & \multicolumn{2}{c}{\textbf{Split MNIST}} & \multicolumn{2}{c}{\textbf{Split FashionMNIST}} & \multicolumn{2}{c}{\textbf{Split notMNIST}} \\ 
\cmidrule(lr){3-4} \cmidrule(lr){5-6} \cmidrule(lr){7-8}
\textbf{Memory}        & \textbf{Schedule} & $t$      & $p$          & $t$        & $p$          & $t$      & $p$          \\ \midrule
\multirow{3}{*}{HAL}   & MCTS vs Random    & 1.84 & 0.139 & 3.33   & \textbf{0.028} & 1.41 & 0.198 \\
                       & MCTS vs ETS       & 1.20 & 0.295 & 2.27   & 0.053 & 2.31 & 0.064 \\
                       & MCTS vs Heuristic   & 2.26 & 0.056 & 4.18   & \textbf{0.014} & 1.34 & 0.218 \\ \midrule
\multirow{3}{*}{MER}   & MCTS vs Random    & 2.08 & 0.099 & -4.15  & \textbf{0.006} & 2.48 & 0.059 \\
                       & MCTS vs ETS       & 3.71 & \textbf{0.012} & 0.64   & 0.542 & 3.32 & \textbf{0.011} \\
                       & MCTS vs Heuristic   & 1.48 & 0.204 & -4.96  & \textbf{0.007} & 2.04 & 0.084 \\ \midrule
\multirow{3}{*}{DER}   & MCTS vs Random    & 2.85 & \textbf{0.046} & 190.21 & \textbf{0.000} & 2.56 & 0.063 \\
                       & MCTS vs ETS       & 4.64 & \textbf{0.007} & 1.64   & 0.139 & 0.51 & 0.635 \\
                       & MCTS vs Heuristic   & 5.21 & \textbf{0.006} & 2.67   & 0.055 & 2.73 & 0.053 \\ \midrule
\multirow{3}{*}{DER++} & MCTS vs Random    & 1.75 & 0.156 & 227.62 & \textbf{0.000} & 6.68 & \textbf{0.003} \\
                       & MCTS vs ETS       & 3.09 & \textbf{0.026} & 0.89   & 0.397 & 0.39 & 0.712 \\
                       & MCTS vs Heuristic   & 5.36 & \textbf{0.006} & 2.93   & \textbf{0.043} & 1.26 & 0.272 \\
\bottomrule \toprule
                       &                   & \multicolumn{2}{c}{\textbf{Permuted MNIST}} & \multicolumn{2}{c}{\textbf{Split CIFAR-100}} & \multicolumn{2}{c}{\textbf{Split miniImagenet}} \\
\cmidrule(lr){3-4} \cmidrule(lr){5-6} \cmidrule(lr){7-8}
\textbf{Memory}        & \textbf{Schedule} & $t$        & $p$          & $t$       & $p$          & $t$       & $p$          \\ \midrule
\multirow{3}{*}{HAL}   & MCTS vs Random    & 0.46   & 0.657 & 2.96  & 0.022 & 0.49  & 0.640 \\
                       & MCTS vs ETS       & 1.20   & 0.266 & 4.16  & \textbf{0.004} & 3.97  & \textbf{0.004} \\
                       & MCTS vs Heuristic   & 1.58   & 0.189 & 5.07  & \textbf{0.001} & 2.01  & 0.082 \\ \midrule
\multirow{3}{*}{MER}   & MCTS vs Random    & -17.61 & \textbf{0.000} & 2.92  & \textbf{0.020} & -0.14 & 0.889 \\
                       & MCTS vs ETS       & 11.24  & \textbf{0.000} & 0.95  & 0.386 & -0.84 & 0.425 \\
                       & MCTS vs Heuristic   & -2.54  & 0.059 & 3.70  & \textbf{0.013} & -1.27 & 0.244 \\ \midrule
\multirow{3}{*}{DER}   & MCTS vs Random    & -3.12  & \textbf{0.019} & 3.46  & \textbf{0.010} & 3.96  & \textbf{0.014} \\
                       & MCTS vs ETS       & 11.36  & \textbf{0.000} & 7.18  & \textbf{0.000} & 5.31  & \textbf{0.003} \\
                       & MCTS vs Heuristic   & 7.47   & \textbf{0.002} & 4.44  & \textbf{0.002} & -0.25 & 0.809 \\ \midrule
\multirow{3}{*}{DER++} & MCTS vs Random    & 0.03   & 0.981 & -2.75 & \textbf{0.025} & 2.57  & 0.051 \\
                       & MCTS vs ETS       & 10.17  & \textbf{0.000} & 9.94  & \textbf{0.000} & 5.43  & \textbf{0.004} \\
                       & MCTS vs Heuristic   & 8.72   & \textbf{0.001} & 3.34  & \textbf{0.013} & 6.89  & \textbf{0.000} \\
\bottomrule
\end{tabular}
	}
	\vspace{-3mm}
	\label{tab:t_test_alternative_replay_methods}
\end{table}


\begin{table}[t]
	\centering
	\caption{Two-tailed Welch's $t$-test results for efficiency of replay scheduling presented in Table \ref{tab:efficiency_of_replay_scheduling_all_datasets}. 
	}
	\vspace{-3mm}
	\resizebox{0.95\textwidth}{!}{
		\begin{tabular}{lcccccc}
\toprule
                   & \multicolumn{2}{c}{\textbf{Split MNIST}} & \multicolumn{2}{c}{\textbf{Split FashionMNIST}} & \multicolumn{2}{c}{\textbf{Split notMNIST}} \\
\cmidrule(lr){2-3} \cmidrule(lr){4-5} \cmidrule(lr){6-7}
\textbf{Methods}   & $t$                 & $p$                    & $t$                    & $p$                        & $t$                  & $p$                      \\ \midrule
MCTS vs Random     & 2.12            & 0.077           & 2.32               & 0.076               & 1.89             & 0.122             \\
MCTS vs A-GEM      & 1.00            & 0.347           & 4.08               & \textbf{0.004}               & 1.41             & 0.195             \\
MCTS vs ER-Ring    & 1.01            & 0.342           & 1.18               & 0.292               & 2.11             & 0.076             \\
MCTS vs Uniform & 0.30            & 0.770           & 0.07               & 0.950               & 1.39             & 0.203            \\
\bottomrule \toprule
                   & \multicolumn{2}{c}{\textbf{Permuted MNIST}} & \multicolumn{2}{c}{\textbf{Split CIFAR-100}} & \multicolumn{2}{c}{\textbf{Split miniImagenet}} \\
\cmidrule(lr){2-3} \cmidrule(lr){4-5} \cmidrule(lr){6-7}
\textbf{Methods}   & $t$                  & $p$                      & $t$                   & $p$                      & $t$                     & $p$                       \\ \midrule
MCTS vs Random     & 2.72             & \textbf{0.044}             & 3.67              & \textbf{0.007}             & 2.48                & 0.054              \\
MCTS vs A-GEM      & 8.40             & \textbf{0.001}             & 7.60              & \textbf{0.000}             & 19.52               & \textbf{0.000}              \\
MCTS vs ER-Ring    & 4.46             & \textbf{0.005}             & -2.96             & \textbf{0.018}             & 0.99                & 0.369              \\
MCTS vs Uniform & 4.68             & \textbf{0.003}             & -1.13             & \textbf{0.296}             & 0.23                & 0.824             \\
\bottomrule
\end{tabular}
	}
	\vspace{-3mm}
	\label{tab:t_test_efficiency_replay_scheduling}
\end{table}


\clearpage


\subsection{Varying Memory Size in Different Continual Learning Setting}
\label{paperC:app:varying_memory_size}

Here, we provide the ACC and BWT metrics from the varying memory size experiments from figure \ref{fig:acc_over_replay_memory_size} in Section \ref{paperC:sec:results_with_varying_replay_memory_size}. We also provide the p-values from Welch's t-tests to show the statistical significance between the methods. 
\begin{itemize}[topsep=0pt, noitemsep]
	\item \textbf{Domain-IL, 10-task Permuted MNIST:} Metrics in Table \ref{tab:varying_memory_size_domain_il_permutedmnist}, t-tests in Table \ref{tab:t_test_varying_memory_size_domain_il_permutedmnist}.
	
	\item \textbf{Task-IL, 5-task Split MNIST/FashionMNIST/notMNIST:} Metrics in Table \ref{tab:varying_memory_size_task_il_5task_datasets}, t-tests in Table \ref{tab:t_test_varying_memory_size_task_il_5task_datasets}. 
	
	\item \textbf{Task-IL, 20-task Split CIFAR-100/miniImagenet:} Metrics in Table \ref{tab:varying_memory_size_task_il_20task_datasets}, t-tests in Table \ref{tab:t_test_varying_memory_size_task_il_20task_datasets}. 
	
	\item \textbf{Class-IL, 5-task Split MNIST/FashionMNIST/notMNIST:} Metrics in Table \ref{tab:varying_memory_size_class_il_5task_datasets}, t-tests in Table \ref{tab:t_test_varying_memory_size_class_il_5task_datasets}. 
	
	\item \textbf{Class-IL, 20-task Split CIFAR-100/miniImagenet:} Metrics in Table \ref{tab:varying_memory_size_class_il_20task_datasets}, t-tests in Table \ref{tab:t_test_varying_memory_size_class_il_20task_datasets}. 
	%\vspace{-1mm}
\end{itemize}
We note that MCTS mostly performs significantly better than the baselines on the 10-task Permuted MNIST and, interestingly, on the 20-task Split CIFAR-100 and Split miniImagenet in both Task-IL and Clas-IL settings, which shows the importance of replay scheduling for CL datasets with long task horizons.    

\begin{table}[h]
	%\caption{Global caption}
	
	\begin{minipage}{.5\textwidth}
	\centering
	\captionsetup{width=.9\linewidth}
	\caption{Performance comparison in the Domain Incremental Learning setting over various memory sizes for the methods on Permuted MNIST. 
	}
	\vspace{-3mm}
	\resizebox{0.95\textwidth}{!}{
	\begin{tabular}{llcc}
\toprule
                        &                 & \multicolumn{2}{c}{\textbf{Permuted MNIST}} \\
\cmidrule(lr){3-4}
\textbf{Memory Size}    & \textbf{Method} & ACC (\%)             & BWT (\%)             \\
\midrule
\multirow{4}{*}{M=90}   & Random          & 72.63 $\pm$ 1.06       & -25.62 $\pm$ 1.20      \\
                        & ETS             & 71.49 $\pm$ 1.15       & -26.92 $\pm$ 1.26      \\
                        & Heuristic         & 75.50 $\pm$ 1.64       & -22.14 $\pm$ 1.93      \\
                        & MCTS            & 71.66 $\pm$ 0.67       & -28.70 $\pm$ 0.81      \\
\midrule
\multirow{4}{*}{M=270}  & Random          & 82.01 $\pm$ 0.79       & -15.24 $\pm$ 0.86      \\
                        & ETS             & 80.68 $\pm$ 0.66       & -16.72 $\pm$ 0.77      \\
                        & Heuristic         & 78.32 $\pm$ 1.58       & -19.01 $\pm$ 1.83      \\
                        & MCTS            & 82.08 $\pm$ 0.35       & -17.18 $\pm$ 0.38      \\
\midrule
\multirow{4}{*}{M=450}  & Random          & 84.72 $\pm$ 1.39       & -12.16 $\pm$ 1.53      \\
                        & ETS             & 84.38 $\pm$ 1.12       & -12.54 $\pm$ 1.20      \\
                        & Heuristic         & 81.66 $\pm$ 2.30       & -15.27 $\pm$ 2.48      \\
                        & MCTS            & 85.53 $\pm$ 0.42       & -13.34 $\pm$ 0.49      \\
\midrule
\multirow{4}{*}{M=900}  & Random          & 88.02 $\pm$ 1.03       & -8.51 $\pm$ 1.09       \\
                        & ETS             & 88.02 $\pm$ 0.46       & -8.52 $\pm$ 0.47       \\
                        & Heuristic         & 80.27 $\pm$ 3.26       & -16.77 $\pm$ 3.76      \\
                        & MCTS            & 88.94 $\pm$ 0.43       & -9.55 $\pm$ 0.47       \\
\midrule
\multirow{4}{*}{M=2250} & Random          & 90.94 $\pm$ 0.46       & -5.26 $\pm$ 0.43       \\
                        & ETS             & 91.07 $\pm$ 0.23       & -5.11 $\pm$ 0.21       \\
                        & Heuristic         & 81.28 $\pm$ 3.79       & -15.68 $\pm$ 4.32      \\
                        & MCTS            & 92.45 $\pm$ 0.34       & -5.63 $\pm$ 0.41 \\
\bottomrule
\end{tabular}
	}
	%\vspace{-5mm}
	\label{tab:varying_memory_size_domain_il_permutedmnist}
	\end{minipage}% \quad
	\begin{minipage}{.5\textwidth}
	\centering
	\captionsetup{width=.9\linewidth}
	\caption{Two-tailed Welch's $t$-test results for the varying memory size experiments presented in Table \ref{tab:varying_memory_size_domain_il_permutedmnist} in the Domain Incremental Learning setting on Permuted MNIST. }
	\vspace{-3mm}
	\resizebox{0.95\textwidth}{!}{
	\begin{tabular}{llcc}
\toprule
                        &                  & \multicolumn{2}{c}{\textbf{Permuted MNIST}} \\
\cmidrule(lr){3-4}
\textbf{Memory Size}    & \textbf{Methods} & $t$                  & $p$                    \\
\midrule
\multirow{3}{*}{M=90}   & MCTS vs Random   & -1.55             & 0.166            \\
                        & MCTS vs ETS      & 0.26              & 0.806            \\
                        & MCTS vs Heuristic  & -4.32             & \textbf{0.007}            \\
\midrule
\multirow{3}{*}{M=270}  & MCTS vs Random   & 0.17              & 0.872            \\
                        & MCTS vs ETS      & 3.75              & \textbf{0.009}            \\
                        & MCTS vs Heuristic  & 4.66              & \textbf{0.008}            \\
\midrule
\multirow{3}{*}{M=450}  & MCTS vs Random   & 1.13              & 0.313            \\
                        & MCTS vs ETS      & 1.93              & 0.111            \\
                        & MCTS vs Heuristic  & 3.31              & \textbf{0.027}            \\
\midrule
\multirow{3}{*}{M=900}  & MCTS vs Random   & 1.66              & 0.153            \\
                        & MCTS vs ETS      & 2.92              & \textbf{0.019}            \\
                        & MCTS vs Heuristic  & 5.28              & \textbf{0.006}            \\
\midrule
\multirow{3}{*}{M=2250} & MCTS vs Random   & 5.31              & \textbf{0.001}            \\
                        & MCTS vs ETS      & 6.74              & \textbf{0.000}            \\
                        & MCTS vs Heuristic  & 5.88              & \textbf{0.004}    \\
\bottomrule
\end{tabular}
	}
	\label{tab:t_test_varying_memory_size_domain_il_permutedmnist}
	\end{minipage} 
\end{table}

\begin{table}[h]

\end{table}

\begin{comment}

\begin{table}[h]
	\centering
	\caption{Two-tailed Welch's $t$-test results for the varying memory size experiments presented in Table \ref{tab:varying_memory_size_task_il_5task_datasets} in the Domain Incremental Learning setting on Permuted MNIST. 
	}
	\vspace{-3mm}
	\resizebox{0.5\textwidth}{!}{
		\begin{tabular}{llcc}
\toprule
                        &                  & \multicolumn{2}{c}{\textbf{Permuted MNIST}} \\
\cmidrule(lr){3-4}
\textbf{Memory Size}    & \textbf{Methods} & $t$                  & $p$                    \\
\midrule
\multirow{3}{*}{M=90}   & MCTS vs Random   & -1.55             & 0.166            \\
                        & MCTS vs ETS      & 0.26              & 0.806            \\
                        & MCTS vs Heuristic  & -4.32             & \textbf{0.007}            \\
\midrule
\multirow{3}{*}{M=270}  & MCTS vs Random   & 0.17              & 0.872            \\
                        & MCTS vs ETS      & 3.75              & \textbf{0.009}            \\
                        & MCTS vs Heuristic  & 4.66              & \textbf{0.008}            \\
\midrule
\multirow{3}{*}{M=450}  & MCTS vs Random   & 1.13              & 0.313            \\
                        & MCTS vs ETS      & 1.93              & 0.111            \\
                        & MCTS vs Heuristic  & 3.31              & \textbf{0.027}            \\
\midrule
\multirow{3}{*}{M=900}  & MCTS vs Random   & 1.66              & 0.153            \\
                        & MCTS vs ETS      & 2.92              & \textbf{0.019}            \\
                        & MCTS vs Heuristic  & 5.28              & \textbf{0.006}            \\
\midrule
\multirow{3}{*}{M=2250} & MCTS vs Random   & 5.31              & \textbf{0.001}            \\
                        & MCTS vs ETS      & 6.74              & \textbf{0.000}            \\
                        & MCTS vs Heuristic  & 5.88              & \textbf{0.004}    \\
\bottomrule
\end{tabular}
	}
	\label{tab:t_test_varying_memory_size_domain_il_permutedmnist}
\end{table}
	content...
\end{comment}


\begin{table}[t]
	\centering
	\caption{Performance comparison in the Task Incremental Learning setting over various memory sizes for the methods on the 5-task datasets. 
	}
	\vspace{-3mm}
	\resizebox{0.9\textwidth}{!}{
		
\begin{tabular}{llcccccc}
\toprule
                       &                 & \multicolumn{2}{c}{\textbf{Split MNIST}} & \multicolumn{2}{c}{\textbf{Split FashionMNIST}} & \multicolumn{2}{c}{\textbf{Split notMNIST}} \\
\cmidrule(lr){3-4} \cmidrule(lr){5-6} \cmidrule(lr){7-8}
\textbf{Memory Size}   & \textbf{Method} & ACC (\%)            & BWT (\%)           & ACC (\%)               & BWT (\%)               & ACC (\%)             & BWT (\%)             \\
\midrule
\multirow{4}{*}{M=8}   & Random          & 95.50 $\pm$ 1.92      & -5.38 $\pm$ 2.38     & 95.04 $\pm$ 2.43         & -5.42 $\pm$ 3.06         & 91.36 $\pm$ 1.76       & -6.15 $\pm$ 1.98       \\
                       & ETS             & 95.83 $\pm$ 1.23      & -4.96 $\pm$ 1.53     & 97.25 $\pm$ 0.68         & -2.66 $\pm$ 0.82         & 92.10 $\pm$ 1.53       & -5.10 $\pm$ 1.97       \\
                       & Heuristic         & 96.16 $\pm$ 0.83      & -4.45 $\pm$ 1.05     & 96.29 $\pm$ 0.79         & -3.83 $\pm$ 1.05         & 93.68 $\pm$ 1.11       & -3.06 $\pm$ 1.54       \\
                       & MCTS            & 98.17 $\pm$ 0.48      & -1.99 $\pm$ 0.61     & 98.33 $\pm$ 0.11         & -1.27 $\pm$ 0.11         & 94.53 $\pm$ 0.62       & -1.67 $\pm$ 1.03       \\
\midrule
\multirow{4}{*}{M=24}  & Random          & 97.21 $\pm$ 1.70      & -3.24 $\pm$ 2.14     & 97.20 $\pm$ 0.71         & -2.72 $\pm$ 0.92         & 92.31 $\pm$ 1.50       & -4.88 $\pm$ 1.75       \\
                       & ETS             & 97.87 $\pm$ 0.58      & -2.41 $\pm$ 0.74     & 97.84 $\pm$ 0.55         & -1.91 $\pm$ 0.64         & 93.19 $\pm$ 1.41       & -3.84 $\pm$ 1.48       \\
                       & Heuristic         & 97.82 $\pm$ 1.17      & -2.40 $\pm$ 1.46     & 95.76 $\pm$ 2.84         & -4.48 $\pm$ 3.52         & 92.97 $\pm$ 2.55       & -3.54 $\pm$ 3.07       \\
                       & MCTS            & 98.58 $\pm$ 0.32      & -1.47 $\pm$ 0.41     & 98.48 $\pm$ 0.24         & -1.06 $\pm$ 0.32         & 95.23 $\pm$ 0.66       & -1.08 $\pm$ 1.72       \\
\midrule
\multirow{4}{*}{M=80}  & Random          & 97.57 $\pm$ 2.03      & -2.78 $\pm$ 2.53     & 96.85 $\pm$ 1.58         & -3.10 $\pm$ 2.05         & 94.14 $\pm$ 0.95       & -2.50 $\pm$ 1.59       \\
                       & ETS             & 98.47 $\pm$ 0.40      & -1.65 $\pm$ 0.49     & 97.93 $\pm$ 0.85         & -1.76 $\pm$ 0.98         & 94.63 $\pm$ 0.67       & -1.82 $\pm$ 0.74       \\
                       & Heuristic         & 98.36 $\pm$ 0.40      & -1.72 $\pm$ 0.50     & 96.22 $\pm$ 1.96         & -3.92 $\pm$ 2.45         & 93.24 $\pm$ 2.56       & -3.73 $\pm$ 2.71       \\
                       & MCTS            & 99.06 $\pm$ 0.16      & -0.89 $\pm$ 0.21     & 98.60 $\pm$ 0.24         & -0.87 $\pm$ 0.34         & 94.84 $\pm$ 0.58       & -0.97 $\pm$ 1.28       \\
\midrule
\multirow{4}{*}{M=120} & Random          & 97.82 $\pm$ 2.29      & -2.47 $\pm$ 2.88     & 98.29 $\pm$ 0.44         & -1.26 $\pm$ 0.56         & 94.91 $\pm$ 1.03       & -1.65 $\pm$ 1.04       \\
                       & ETS             & 98.82 $\pm$ 0.18      & -1.22 $\pm$ 0.26     & 98.53 $\pm$ 0.28         & -0.96 $\pm$ 0.36         & 95.32 $\pm$ 0.66       & -1.21 $\pm$ 1.30       \\
                       & Heuristic         & 98.37 $\pm$ 0.38      & -1.71 $\pm$ 0.48     & 95.26 $\pm$ 3.14         & -5.12 $\pm$ 3.98         & 93.42 $\pm$ 1.78       & -3.47 $\pm$ 2.22       \\
                       & MCTS            & 99.05 $\pm$ 0.11      & -0.88 $\pm$ 0.15     & 98.75 $\pm$ 0.19         & -0.77 $\pm$ 0.25         & 94.16 $\pm$ 1.08       & -1.99 $\pm$ 1.69       \\
\midrule
\multirow{4}{*}{M=200} & Random          & 97.99 $\pm$ 1.59      & -2.25 $\pm$ 2.00     & 96.68 $\pm$ 3.33         & -3.38 $\pm$ 4.19         & 93.94 $\pm$ 1.40       & -2.12 $\pm$ 1.70       \\
                       & ETS             & 98.83 $\pm$ 0.23      & -1.19 $\pm$ 0.28     & 98.60 $\pm$ 0.25         & -0.99 $\pm$ 0.29         & 94.79 $\pm$ 0.50       & -1.58 $\pm$ 1.03       \\
                       & Heuristic         & 98.15 $\pm$ 0.64      & -1.97 $\pm$ 0.81     & 95.83 $\pm$ 2.00         & -4.40 $\pm$ 2.52         & 93.88 $\pm$ 1.63       & -2.71 $\pm$ 1.69       \\
                       & MCTS            & 99.09 $\pm$ 0.08      & -0.83 $\pm$ 0.11     & 98.83 $\pm$ 0.11         & -0.65 $\pm$ 0.15         & 95.19 $\pm$ 0.53       & -0.49 $\pm$ 0.47       \\
\midrule
\multirow{4}{*}{M=400} & Random          & 97.98 $\pm$ 2.23      & -2.28 $\pm$ 2.80     & 98.00 $\pm$ 1.59         & -1.74 $\pm$ 2.00         & 94.68 $\pm$ 1.11       & -1.77 $\pm$ 1.09       \\
                       & ETS             & 99.18 $\pm$ 0.10      & -0.78 $\pm$ 0.13     & 98.83 $\pm$ 0.09         & -0.71 $\pm$ 0.12         & 95.41 $\pm$ 0.56       & -0.20 $\pm$ 1.07       \\
                       & Heuristic         & 98.44 $\pm$ 0.60      & -1.64 $\pm$ 0.77     & 95.41 $\pm$ 3.77         & -4.93 $\pm$ 4.69         & 92.84 $\pm$ 1.88       & -3.95 $\pm$ 2.05       \\
                       & MCTS            & 99.25 $\pm$ 0.05      & -0.63 $\pm$ 0.09     & 98.80 $\pm$ 0.11         & -0.62 $\pm$ 0.17         & 95.25 $\pm$ 0.44       & -0.97 $\pm$ 1.20       \\
\midrule
\multirow{4}{*}{M=800} & Random          & 98.60 $\pm$ 1.42      & -1.51 $\pm$ 1.77     & 97.00 $\pm$ 3.93         & -2.98 $\pm$ 4.94         & 94.97 $\pm$ 0.73       & -1.86 $\pm$ 0.77       \\
                       & ETS             & 99.34 $\pm$ 0.06      & -0.57 $\pm$ 0.04     & 98.97 $\pm$ 0.06         & -0.51 $\pm$ 0.10         & 95.52 $\pm$ 0.74       & -0.61 $\pm$ 1.21       \\
                       & Heuristic         & 98.76 $\pm$ 0.41      & -1.23 $\pm$ 0.55     & 93.32 $\pm$ 3.67         & -7.54 $\pm$ 4.65         & 95.06 $\pm$ 1.44       & -1.17 $\pm$ 1.30       \\
                       & MCTS            & 99.38 $\pm$ 0.08      & -0.45 $\pm$ 0.10     & 98.96 $\pm$ 0.12         & -0.45 $\pm$ 0.18         & 95.46 $\pm$ 0.70       & -0.48 $\pm$ 0.81      \\
\bottomrule
\end{tabular}

	}
	\label{tab:varying_memory_size_task_il_5task_datasets}
\end{table}



\begin{table}[t]
	\centering
	\caption{Two-tailed Welch's $t$-test results for the varying memory size experiments presented in Table \ref{tab:varying_memory_size_task_il_5task_datasets} in the Task Incremental Learning setting on the 5-task datasets. 
	}
	\vspace{-3mm}
	\resizebox{0.9\textwidth}{!}{
		\begin{tabular}{llcccccc}
\toprule
                       &                  & \multicolumn{2}{c}{\textbf{Split MNIST}} & \multicolumn{2}{c}{\textbf{Split FashionMNIST}} & \multicolumn{2}{c}{\textbf{Split notMNIST}} \\
\cmidrule(lr){3-4} \cmidrule(lr){5-6} \cmidrule(lr){7-8}
\textbf{Memory size}   & \textbf{Methods} & $t$                & $p$                   & $t$                    & $p$                      & $t$                  & $p$                    \\
\midrule
\multirow{3}{*}{M=8}   & MCTS vs Random  & 2.70   & \textbf{0.048}  & 2.70       & 0.054              & 3.41     & \textbf{0.019}   \\
                       & MCTS vs ETS     & 3.53   & \textbf{0.016}  & 3.13       & \textbf{0.033}     & 2.94     & \textbf{0.030}   \\
                       & MCTS vs Heuristic & 4.18   & \textbf{0.005}  & 5.12       & \textbf{0.006}     & 1.34     & 0.227            \\
\midrule
\multirow{3}{*}{M=24}  & MCTS vs Random  & 1.59   & 0.183           & 3.39       & \textbf{0.020}     & 3.55     & \textbf{0.014}   \\
                       & MCTS vs ETS     & 2.15   & 0.074           & 2.14       & \textbf{0.080}     & 2.63     & \textbf{0.041}   \\
                       & MCTS vs Heuristic & 1.26   & 0.268           & 1.90       & 0.128              & 1.72     & 0.153            \\
\midrule
\multirow{3}{*}{M=80}  & MCTS vs Random  & 1.47   & 0.215           & 2.18       & 0.091              & 1.26     & 0.251            \\
                       & MCTS vs ETS     & 2.76   & \textbf{0.038}  & 1.53       & 0.191              & 0.48     & 0.645            \\
                       & MCTS vs Heuristic & 3.26   & \textbf{0.021}  & 2.41       & 0.072              & 1.22     & 0.285            \\
\midrule
\multirow{3}{*}{M=120} & MCTS vs Random  & 1.07   & 0.344           & 1.95       & 0.105              & -1.01    & 0.343            \\
                       & MCTS vs ETS     & 2.12   & 0.073           & 1.31       & 0.231              & -1.83    & 0.112            \\
                       & MCTS vs Heuristic & 3.45   & \textbf{0.020}  & 2.22       & 0.090              & 0.70     & 0.507            \\
\midrule
\multirow{3}{*}{M=200} & MCTS vs Random  & 1.38   & 0.240           & 1.29       & 0.266              & 1.67     & 0.154            \\
                       & MCTS vs ETS     & 2.06   & 0.094           & 1.69       & 0.146              & 1.11     & 0.301            \\
                       & MCTS vs Heuristic & 2.91   & \textbf{0.042}  & 3.00       & \textbf{0.040}     & 1.53     & 0.190            \\
\midrule
\multirow{3}{*}{M=400} & MCTS vs Random  & 1.14   & 0.318           & 1.00       & 0.373              & 0.94     & 0.389            \\
                       & MCTS vs ETS     & 1.25   & 0.258           & -0.42      & 0.684              & -0.47    & 0.655            \\
                       & MCTS vs Heuristic & 2.69   & 0.054           & 1.80       & 0.146              & 2.49     & 0.061            \\
\midrule
\multirow{3}{*}{M=800} & MCTS vs Random  & 1.10   & 0.332           & 1.00       & 0.373              & 0.96     & 0.364            \\
                       & MCTS vs ETS     & 0.82   & 0.435           & -0.06      & 0.955              & -0.11    & 0.912            \\
                       & MCTS vs Heuristic & 3.05   & \textbf{0.035}  & 3.08       & \textbf{0.037}     & 0.50     & 0.636           \\
\bottomrule
\end{tabular}
	}
	\label{tab:t_test_varying_memory_size_task_il_5task_datasets}
\end{table}

\begin{table}[t]
	\centering
	\caption{Performance comparison in the Task Incremental Learning setting over various memory sizes for the methods on the 20-task datasets. 
	}
	\vspace{-3mm}
	\resizebox{0.9\textwidth}{!}{
		\begin{tabular}{llcccc}
\toprule
                        &                 & \multicolumn{2}{c}{\textbf{Split CIFAR-100}} & \multicolumn{2}{c}{\textbf{Split miniImagenet}} \\
\cmidrule(lr){3-4} \cmidrule(lr){5-6}
\textbf{Memory Size}    & \textbf{Method} & ACC (\%)             & BWT (\%)              & ACC (\%)               & BWT (\%)               \\
\midrule
\multirow{4}{*}{M=95}   & Random          & 56.92 $\pm$ 0.72       & -31.98 $\pm$ 0.69       & 52.06 $\pm$ 0.94         & -12.49 $\pm$ 1.41        \\
                        & ETS             & 55.19 $\pm$ 0.77       & -33.70 $\pm$ 0.82       & 51.33 $\pm$ 2.01         & -13.32 $\pm$ 2.15        \\
                        & Heuristic         & 53.86 $\pm$ 3.89       & -34.51 $\pm$ 4.25       & 50.66 $\pm$ 1.36         & -10.99 $\pm$ 1.39        \\
                        & MCTS            & 60.46 $\pm$ 1.05       & -27.67 $\pm$ 1.23       & 53.59 $\pm$ 0.24         & -9.49 $\pm$ 0.51         \\
\midrule
\multirow{4}{*}{M=285}  & Random          & 67.79 $\pm$ 1.19       & -20.31 $\pm$ 1.12       & 57.85 $\pm$ 1.09         & -6.33 $\pm$ 1.30         \\
                        & ETS             & 65.02 $\pm$ 0.98       & -23.36 $\pm$ 1.10       & 56.18 $\pm$ 0.83         & -8.43 $\pm$ 1.37         \\
                        & Heuristic         & 60.26 $\pm$ 2.88       & -27.44 $\pm$ 2.87       & 53.33 $\pm$ 2.21         & -7.96 $\pm$ 2.18         \\
                        & MCTS            & 68.85 $\pm$ 0.56       & -18.41 $\pm$ 0.63       & 57.91 $\pm$ 0.09         & -5.18 $\pm$ 0.54         \\
\midrule
\multirow{4}{*}{M=475}  & Random          & 70.86 $\pm$ 1.24       & -17.19 $\pm$ 1.26       & 59.54 $\pm$ 1.25         & -4.22 $\pm$ 1.96         \\
                        & ETS             & 69.40 $\pm$ 0.73       & -18.68 $\pm$ 0.82       & 59.60 $\pm$ 0.98         & -4.75 $\pm$ 1.31         \\
                        & Heuristic         & 65.00 $\pm$ 3.08       & -22.58 $\pm$ 3.20       & 50.12 $\pm$ 4.60         & -11.79 $\pm$ 5.08        \\
                        & MCTS            & 72.93 $\pm$ 0.54       & -14.15 $\pm$ 0.78       & 60.00 $\pm$ 0.48         & -2.85 $\pm$ 0.34         \\
\midrule
\multirow{4}{*}{M=950}  & Random          & 75.39 $\pm$ 0.46       & -12.24 $\pm$ 0.46       & 61.87 $\pm$ 0.85         & -2.32 $\pm$ 1.06         \\
                        & ETS             & 74.24 $\pm$ 0.61       & -13.49 $\pm$ 0.44       & 60.51 $\pm$ 0.95         & -4.01 $\pm$ 1.28         \\
                        & Heuristic         & 65.97 $\pm$ 5.51       & -21.40 $\pm$ 5.58       & 50.99 $\pm$ 4.64         & -11.21 $\pm$ 4.62        \\
                        & MCTS            & 76.65 $\pm$ 0.62       & -10.31 $\pm$ 0.84       & 61.82 $\pm$ 0.69         & -1.38 $\pm$ 0.55         \\
\midrule
\multirow{4}{*}{M=1900} & Random          & 78.86 $\pm$ 0.38       & -8.64 $\pm$ 0.37        & 62.95 $\pm$ 0.69         & -1.56 $\pm$ 0.99         \\
                        & ETS             & 77.66 $\pm$ 0.38       & -9.72 $\pm$ 0.36        & 62.38 $\pm$ 0.68         & -1.55 $\pm$ 0.90         \\
                        & Heuristic         & 63.70 $\pm$ 6.12       & -24.07 $\pm$ 6.60       & 56.42 $\pm$ 3.15         & -5.43 $\pm$ 3.63         \\
                        & MCTS            & 79.24 $\pm$ 0.59       & -7.43 $\pm$ 0.64        & 63.83 $\pm$ 1.04         & 0.41 $\pm$ 1.10   \\
\bottomrule
\end{tabular}
	}
	\label{tab:varying_memory_size_task_il_20task_datasets}
\end{table}


\begin{table}[t]
	\centering
	\caption{Two-tailed Welch's $t$-test results for the varying memory size experiments presented in Table \ref{tab:varying_memory_size_task_il_20task_datasets} in the Task Incremental Learning setting on the 20-task datasets. 
	}
	\vspace{-3mm}
	\resizebox{0.9\textwidth}{!}{
		\begin{tabular}{llcccc}
\toprule
\textbf{}               &                  & \multicolumn{2}{c}{\textbf{Split CIFAR-100}} & \multicolumn{2}{c}{\textbf{Split miniImagenet}} \\
\cmidrule(lr){3-4} \cmidrule(lr){5-6}
\textbf{Memory size}    & \textbf{Methods} & $t$                  & $p$                     & $t$                    & $p$                      \\
\midrule
\multirow{3}{*}{M=95}   & MCTS vs Random  & 5.56     & \textbf{0.001}    & 3.15       & \textbf{0.029}     \\
                        & MCTS vs ETS     & 8.11     & \textbf{0.000}    & 2.23       & 0.088              \\
                        & MCTS vs Heuristic & 3.27     & \textbf{0.025}    & 4.24       & \textbf{0.012}     \\
\midrule
\multirow{3}{*}{M=285}  & MCTS vs Random  & 1.62     & 0.159             & 0.11       & 0.918              \\
                        & MCTS vs ETS     & 6.76     & \textbf{0.000}    & 4.14       & \textbf{0.014}     \\
                        & MCTS vs Heuristic & 5.87     & \textbf{0.003}    & 4.13       & \textbf{0.014}     \\
\midrule
\multirow{3}{*}{M=475}  & MCTS vs Random  & 3.05     & \textbf{0.025}    & 0.68       & 0.525              \\
                        & MCTS vs ETS     & 7.78     & \textbf{0.000}    & 0.73       & 0.496              \\
                        & MCTS vs Heuristic & 5.07     & \textbf{0.006}    & 4.28       & \textbf{0.012}     \\
\midrule
\multirow{3}{*}{M=950}  & MCTS vs Random  & 3.25     & \textbf{0.013}    & -0.09      & 0.929              \\
                        & MCTS vs ETS     & 5.51     & \textbf{0.001}    & 2.25       & 0.058              \\
                        & MCTS vs Heuristic & 3.85     & \textbf{0.017}    & 4.62       & \textbf{0.009}     \\
\midrule
\multirow{3}{*}{M=1900} & MCTS vs Random  & 1.08     & 0.317             & 1.41       & 0.203              \\
                        & MCTS vs ETS     & 4.53     & \textbf{0.003}    & 2.34       & 0.052              \\
                        & MCTS vs Heuristic & 5.05     & \textbf{0.007}    & 4.47       & \textbf{0.007}    \\
\bottomrule
\end{tabular}
	}
	\label{tab:t_test_varying_memory_size_task_il_20task_datasets}
\end{table}

\begin{table}[t]
	\centering
	\caption{Performance comparison in the Class Incremental Learning setting over various memory sizes for the methods on the 5-task datasets. 
	}
	\vspace{-3mm}
	\resizebox{0.95\textwidth}{!}{
		\begin{tabular}{llcccccc}
 \toprule
                       &                 & \multicolumn{2}{c}{\textbf{Split MNIST}} & \multicolumn{2}{c}{\textbf{Split FashionMNIST}} & \multicolumn{2}{c}{\textbf{Split notMNIST}} \\
\cmidrule(lr){3-4} \cmidrule(lr){5-6}
\textbf{Memory Size}   & \textbf{Method} & ACC (\%)           & BWT (\%)            & ACC (\%)               & BWT (\%)               & ACC (\%)             & BWT (\%)             \\
\midrule
\multirow{4}{*}{M=10}  & Random          & 25.53 $\pm$ 1.57     & -92.74 $\pm$ 1.99     & 30.78 $\pm$ 2.74         & -85.70 $\pm$ 3.47        & 30.48 $\pm$ 5.16       & -81.80 $\pm$ 6.18      \\
                       & ETS             & 25.10 $\pm$ 1.04     & -93.27 $\pm$ 1.32     & 30.67 $\pm$ 2.36         & -85.83 $\pm$ 3.03        & 31.67 $\pm$ 6.03       & -80.22 $\pm$ 7.36      \\
                       & Heuristic         & 25.05 $\pm$ 1.51     & -93.27 $\pm$ 1.85     & 31.81 $\pm$ 3.17         & -84.33 $\pm$ 3.96        & 29.86 $\pm$ 3.49       & -82.07 $\pm$ 4.26      \\
                       & MCTS            & 28.19 $\pm$ 2.18     & -89.34 $\pm$ 2.71     & 33.77 $\pm$ 2.71         & -81.91 $\pm$ 3.45        & 37.23 $\pm$ 3.53       & -73.93 $\pm$ 3.89      \\
\midrule
\multirow{4}{*}{M=20}  & Random          & 34.13 $\pm$ 2.04     & -81.98 $\pm$ 2.57     & 37.80 $\pm$ 2.92         & -76.83 $\pm$ 3.62        & 42.78 $\pm$ 1.24       & -65.75 $\pm$ 1.92      \\
                       & ETS             & 35.17 $\pm$ 2.06     & -80.65 $\pm$ 2.54     & 38.76 $\pm$ 1.58         & -75.63 $\pm$ 1.93        & 45.21 $\pm$ 5.92       & -63.10 $\pm$ 7.33      \\
                       & Heuristic         & 33.87 $\pm$ 2.15     & -82.20 $\pm$ 2.65     & 42.52 $\pm$ 1.98         & -70.94 $\pm$ 2.47        & 45.13 $\pm$ 3.68       & -63.32 $\pm$ 4.31      \\
                       & MCTS            & 39.62 $\pm$ 0.73     & -75.05 $\pm$ 0.92     & 44.27 $\pm$ 2.19         & -68.71 $\pm$ 2.74        & 47.83 $\pm$ 0.82       & -59.91 $\pm$ 1.11      \\
\midrule
\multirow{4}{*}{M=40}  & Random          & 45.34 $\pm$ 3.71     & -67.96 $\pm$ 4.64     & 45.78 $\pm$ 2.96         & -66.84 $\pm$ 3.72        & 55.70 $\pm$ 2.96       & -49.58 $\pm$ 3.36      \\
                       & ETS             & 48.57 $\pm$ 1.90     & -63.93 $\pm$ 2.37     & 45.54 $\pm$ 1.21         & -67.15 $\pm$ 1.48        & 58.07 $\pm$ 3.88       & -46.85 $\pm$ 4.92      \\
                       & Heuristic         & 45.22 $\pm$ 4.54     & -68.05 $\pm$ 5.65     & 51.67 $\pm$ 1.59         & -59.43 $\pm$ 2.02        & 53.90 $\pm$ 6.10       & -52.09 $\pm$ 7.73      \\
                       & MCTS            & 50.79 $\pm$ 2.45     & -61.07 $\pm$ 3.08     & 51.33 $\pm$ 2.96         & -59.87 $\pm$ 3.70        & 62.55 $\pm$ 3.23       & -41.20 $\pm$ 4.81      \\
\midrule
\multirow{4}{*}{M=100} & Random          & 59.53 $\pm$ 6.37     & -50.12 $\pm$ 7.96     & 55.96 $\pm$ 3.31         & -53.98 $\pm$ 4.17        & 65.06 $\pm$ 3.48       & -37.86 $\pm$ 4.30      \\
                       & ETS             & 68.09 $\pm$ 1.06     & -39.49 $\pm$ 1.33     & 59.26 $\pm$ 0.95         & -49.92 $\pm$ 1.14        & 72.55 $\pm$ 1.65       & -28.88 $\pm$ 2.31      \\
                       & Heuristic         & 49.81 $\pm$ 1.58     & -62.17 $\pm$ 1.95     & 58.14 $\pm$ 2.96         & -51.24 $\pm$ 3.71        & 58.28 $\pm$ 2.02       & -46.70 $\pm$ 2.63      \\
                       & MCTS            & 66.86 $\pm$ 2.21     & -40.96 $\pm$ 2.75     & 60.97 $\pm$ 2.43         & -47.63 $\pm$ 2.99        & 71.18 $\pm$ 3.58       & -30.27 $\pm$ 4.15      \\
\midrule
\multirow{4}{*}{M=200} & Random          & 65.82 $\pm$ 7.50     & -42.27 $\pm$ 9.36     & 61.11 $\pm$ 5.19         & -47.26 $\pm$ 6.48        & 67.07 $\pm$ 5.65       & -35.36 $\pm$ 6.96      \\
                       & ETS             & 77.73 $\pm$ 1.31     & -27.42 $\pm$ 1.63     & 66.51 $\pm$ 0.75         & -40.60 $\pm$ 0.99        & 77.15 $\pm$ 0.24       & -22.61 $\pm$ 0.89      \\
                       & Heuristic         & 53.85 $\pm$ 0.88     & -57.05 $\pm$ 1.11     & 54.69 $\pm$ 0.32         & -55.42 $\pm$ 0.43        & 57.05 $\pm$ 3.28       & -47.66 $\pm$ 4.09      \\
                       & MCTS            & 78.14 $\pm$ 1.67     & -26.81 $\pm$ 2.00     & 67.38 $\pm$ 2.60         & -39.41 $\pm$ 3.26        & 75.91 $\pm$ 4.88       & -24.15 $\pm$ 5.98     \\
\bottomrule
\end{tabular}
	}
	\label{tab:varying_memory_size_class_il_5task_datasets}
\end{table}


\begin{table}[t]
	\centering
	\caption{Two-tailed Welch's $t$-test results for the varying memory size experiments presented in Table \ref{tab:varying_memory_size_class_il_5task_datasets} in the Class Incremental Learning setting on the 5-task datasets. 
	}
	\vspace{-3mm}
	\resizebox{0.95\textwidth}{!}{
		\begin{tabular}{llcccccc}
\toprule
                       &                  & \multicolumn{2}{c}{\textbf{Split MNIST}} & \multicolumn{2}{c}{\textbf{Split FashionMNIST}} & \multicolumn{2}{c}{\textbf{Split notMNIST}} \\
\cmidrule(lr){3-4} \cmidrule(lr){5-6} \cmidrule(lr){7-8}
\textbf{Memory size}   & \textbf{Methods} & $t$                & $p$                   & $t$                    & $p$                      & $t$                  & $p$                    \\
\midrule
\multirow{3}{*}{M=10}  & MCTS vs Random  & 1.98   & 0.086           & 1.56       & 0.158              & 2.16     & 0.067            \\
                       & MCTS vs ETS     & 2.56   & \textbf{0.045}  & 1.73       & 0.123              & 1.59     & 0.159            \\
                       & MCTS vs Heuristic & 2.37   & \textbf{0.049}  & 0.94       & 0.374              & 2.97     & \textbf{0.018}   \\
\midrule
\multirow{3}{*}{M=20}  & MCTS vs Random  & 5.06   & \textbf{0.004}  & 3.54       & \textbf{0.009}     & 6.80     & \textbf{0.000}   \\
                       & MCTS vs ETS     & 4.07   & \textbf{0.010}  & 4.07       & \textbf{0.004}     & 0.88     & 0.429            \\
                       & MCTS vs Heuristic & 5.08   & \textbf{0.004}  & 1.18       & 0.271              & 1.43     & 0.219            \\
\midrule
\multirow{3}{*}{M=40}  & MCTS vs Random  & 2.45   & \textbf{0.044}  & 2.65       & \textbf{0.029}     & 3.13     & \textbf{0.014}   \\
                       & MCTS vs ETS     & 1.43   & 0.192           & 3.62       & \textbf{0.014}     & 1.78     & 0.115            \\
                       & MCTS vs Heuristic & 2.16   & 0.073           & -0.21      & 0.843              & 2.51     & \textbf{0.046}   \\
\midrule
\multirow{3}{*}{M=100} & MCTS vs Random  & 2.18   & 0.082           & 2.44       & \textbf{0.043}     & 2.45     & \textbf{0.040}   \\
                       & MCTS vs ETS     & -1.00  & 0.356           & 1.31       & 0.246              & -0.69    & 0.515            \\
                       & MCTS vs Heuristic & 12.55  & \textbf{0.000}  & 1.48       & 0.180              & 6.29     & \textbf{0.001}   \\
\midrule
\multirow{3}{*}{M=200} & MCTS vs Random  & 3.21   & \textbf{0.029}  & 2.16       & 0.075              & 2.37     & \textbf{0.046}   \\
                       & MCTS vs ETS     & 0.39   & 0.707           & 0.65       & 0.548              & -0.51    & 0.639            \\
                       & MCTS vs Heuristic & 25.79  & \textbf{0.000}  & 9.69       & \textbf{0.001}     & 6.41     & \textbf{0.000}  \\
\bottomrule
\end{tabular}
	}
	\label{tab:t_test_varying_memory_size_class_il_5task_datasets}
\end{table}

\begin{table}[t]
	\centering
	\caption{Performance comparison in the Class Incremental Learning setting over various memory sizes for the methods on the 20-task datasets. 
	}
	\vspace{-3mm}
	\resizebox{0.9\textwidth}{!}{
		\begin{tabular}{llcccc}
\toprule
                        &                 & \multicolumn{2}{c}{\textbf{Split CIFAR-100}} & \multicolumn{2}{c}{\textbf{Split miniImagenet}} \\
\cmidrule(lr){3-4} \cmidrule(lr){5-6}
\textbf{Memory Size}    & \textbf{Method} & ACC (\%)             & BWT (\%)              & ACC (\%)               & BWT (\%)               \\
\midrule
\multirow{4}{*}{M=100}  & Random          & 4.49 $\pm$ 0.04        & -85.73 $\pm$ 0.34       & 3.36 $\pm$ 0.23          & -60.11 $\pm$ 0.83        \\
                        & ETS             & 4.51 $\pm$ 0.13        & -85.62 $\pm$ 0.34       & 3.34 $\pm$ 0.21          & -60.25 $\pm$ 0.47        \\
                        & Heuristic         & 4.55 $\pm$ 0.13        & -84.62 $\pm$ 0.44       & 3.29 $\pm$ 0.14          & -57.01 $\pm$ 0.45        \\
                        & MCTS            & 4.62 $\pm$ 0.10        & -84.73 $\pm$ 0.23       & 3.51 $\pm$ 0.19          & -57.57 $\pm$ 1.13        \\
\midrule
\multirow{4}{*}{M=200}  & Random          & 5.24 $\pm$ 0.12        & -84.53 $\pm$ 0.28       & 3.60 $\pm$ 0.24          & -58.83 $\pm$ 0.81        \\
                        & ETS             & 4.98 $\pm$ 0.14        & -85.05 $\pm$ 0.19       & 3.75 $\pm$ 0.14          & -58.92 $\pm$ 1.03        \\
                        & Heuristic         & 5.20 $\pm$ 0.16        & -83.33 $\pm$ 0.29       & 3.95 $\pm$ 0.36          & -54.81 $\pm$ 0.58        \\
                        & MCTS            & 5.84 $\pm$ 0.16        & -82.89 $\pm$ 0.28       & 4.26 $\pm$ 0.19          & -55.19 $\pm$ 0.73        \\
\midrule
\multirow{4}{*}{M=400}  & Random          & 7.13 $\pm$ 0.25        & -81.93 $\pm$ 0.29       & 4.81 $\pm$ 0.40          & -56.36 $\pm$ 0.44        \\
                        & ETS             & 6.35 $\pm$ 0.25        & -82.92 $\pm$ 0.23       & 4.35 $\pm$ 0.15          & -57.28 $\pm$ 0.83        \\
                        & Heuristic         & 7.66 $\pm$ 0.60        & -79.82 $\pm$ 0.60       & 6.97 $\pm$ 0.78          & -49.44 $\pm$ 1.04        \\
                        & MCTS            & 8.31 $\pm$ 0.21        & -79.58 $\pm$ 0.51       & 7.64 $\pm$ 0.70          & -48.73 $\pm$ 1.71        \\
\midrule
\multirow{4}{*}{M=800}  & Random          & 10.64 $\pm$ 0.51       & -77.27 $\pm$ 0.62       & 8.40 $\pm$ 1.07          & -50.95 $\pm$ 1.85        \\
                        & ETS             & 9.13 $\pm$ 0.20        & -79.54 $\pm$ 0.43       & 7.15 $\pm$ 0.51          & -53.95 $\pm$ 0.74        \\
                        & Heuristic         & 11.33 $\pm$ 0.47       & -74.75 $\pm$ 0.73       & 8.44 $\pm$ 1.15          & -48.89 $\pm$ 1.43        \\
                        & MCTS            & 12.06 $\pm$ 0.24       & -74.70 $\pm$ 0.29       & 11.56 $\pm$ 1.30         & -44.30 $\pm$ 1.63        \\
\midrule
\multirow{4}{*}{M=1600} & Random          & 15.54 $\pm$ 0.69       & -70.42 $\pm$ 0.55       & 13.72 $\pm$ 1.30         & -44.19 $\pm$ 1.73        \\
                        & ETS             & 13.99 $\pm$ 0.32       & -72.42 $\pm$ 0.24       & 12.17 $\pm$ 1.40         & -47.55 $\pm$ 1.39        \\
                        & Heuristic         & 15.66 $\pm$ 1.84       & -67.24 $\pm$ 1.75       & 8.39 $\pm$ 1.35          & -49.35 $\pm$ 1.77        \\
                        & MCTS            & 17.59 $\pm$ 0.42       & -66.59 $\pm$ 0.66       & 15.40 $\pm$ 0.34         & -42.27 $\pm$ 0.73       \\
\bottomrule
\end{tabular}
	}
	\label{tab:varying_memory_size_class_il_20task_datasets}
\end{table}


\begin{table}[t]
	\centering
	\caption{Two-tailed Welch's $t$-test results for the varying memory size experiments presented in Table \ref{tab:varying_memory_size_class_il_20task_datasets} in the Class Incremental Learning setting on the 20-task datasets. 
	}
	\vspace{-3mm}
	\resizebox{0.9\textwidth}{!}{
		\begin{tabular}{llcccc}
\toprule
                        &                  & \multicolumn{2}{c}{\textbf{Split CIFAR-100}} & \multicolumn{2}{c}{\textbf{Split miniImagenet}} \\
\cmidrule(lr){3-4} \cmidrule(lr){5-6}
\textbf{Memory size}    & \textbf{Methods} & $t$                  & $p$                     & $t$                    & $p$                      \\
\midrule
\multirow{3}{*}{M=100}  & MCTS vs Random  & 2.35     & 0.065             & 1.03       & 0.335              \\
                        & MCTS vs ETS     & 1.38     & 0.208             & 1.20       & 0.266              \\
                        & MCTS vs Heuristic & 0.87     & 0.410             & 1.86       & 0.103              \\
\midrule
\multirow{3}{*}{M=200}  & MCTS vs Random  & 5.95     & \textbf{0.000}    & 4.34       & \textbf{0.003}     \\
                        & MCTS vs ETS     & 8.07     & \textbf{0.000}    & 4.35       & \textbf{0.003}     \\
                        & MCTS vs Heuristic & 5.72     & \textbf{0.000}    & 1.51       & 0.182              \\
\midrule
\multirow{3}{*}{M=400}  & MCTS vs Random  & 7.15     & \textbf{0.000}    & 7.06       & \textbf{0.000}     \\
                        & MCTS vs ETS     & 11.98    & \textbf{0.000}    & 9.22       & \textbf{0.000}     \\
                        & MCTS vs Heuristic & 2.01     & 0.101             & 1.29       & 0.232              \\
\midrule
\multirow{3}{*}{M=800}  & MCTS vs Random  & 5.05     & \textbf{0.003}    & 3.75       & \textbf{0.006}     \\
                        & MCTS vs ETS     & 18.79    & \textbf{0.000}    & 6.32       & \textbf{0.001}     \\
                        & MCTS vs Heuristic & 2.77     & \textbf{0.033}    & 3.60       & \textbf{0.007}     \\
\midrule
\multirow{3}{*}{M=1600} & MCTS vs Random  & 5.08     & \textbf{0.002}    & 2.50       & 0.059              \\
                        & MCTS vs ETS     & 13.60    & \textbf{0.000}    & 4.49       & \textbf{0.008}     \\
                        & MCTS vs Heuristic & 2.04     & 0.104             & 10.04      & \textbf{0.000}     \\
\bottomrule
\end{tabular}
	}
	\label{tab:t_test_varying_memory_size_class_il_20task_datasets}
\end{table}





\begin{comment}
\subsection{Analysis of Catastrophic Forgetting}\label{paperC:app:analysis_of_catastrophic_forgetting}

We have compared the degree of catastrophic forgetting for our method against the baselines by measuring the backward transfer (BWT) metric from \citeC{C:lopez2017gradient}, which is given by
\begin{align}
	\textnormal{BWT} = \frac{1}{T-1} \sum_{i=1}^{T-1} A_{T, i} - A_{i, i},
\end{align}
where $A_{t, i}$ is the test accuracy for task $t$ after learning task $i$. Table \ref{tab:bwt_alternative_memory_selection} shows the ACC and BWT metrics for the experiments in Section \ref{paperC:sec:alternative_memory_selection_methods}. In general, the BWT metric is consistently better when the corresponding ACC is better. We find an exception in Table \ref{tab:bwt_alternative_memory_selection} on Split CIFAR-100 and Split miniImagenet between Ours and Heuristic with uniform selection method, where Heuristic has better BWT while its mean of ACC is slightly lower than ACC for Ours. Table \ref{tab:bwt_efficiency_of_replay_scheduling} shows the ACC and BWT metrics for the experiments in Section \ref{paperC:sec:efficiency_of_replay_scheduling}, where we see a similar pattern that better ACC yields better BWT. The BWT of RS-MCTS is on par with the other baselines except on Split CIFAR-100 where the ACC on our method was a bit lower than the best baselines.



\subsection{Applying Scheduling to Recent Replay Methods}
\label{paperC:app:apply_scheduling_to_recent_replay_methods}

In Section \ref{paperC:sec:applying_scheduling_to_recent_replay_methods}, we showed that RS-MCTS can be applied to any replay method. We combined RS-MCTS together with four recent replay methods, namely Hindsight Anchor Learning (HAL)~\citeC{C:chaudhry2021using}, Meta Experience Replay (MER)~\citeC{C:riemer2018learning}, and Dark Experience Replay (DER)~\citeC{C:buzzega2020dark}. Table \ref{tab:bwt_sota_models_applied_to_rsmcts} shows the ACC and BWT for all methods combined with the scheduling from Random, ETS, Heuristic, and RS-MCTS. We observe that RS-MCTS can further improve the performance for each of the replay methods across the different datasets.  

We present the hyperparameters used for each method in Table \ref{tab:hyperparameters_applying_scheduling_to_recent_replay_methods}. The hyperparameters for each method are denoted as
\begin{itemize}[topsep=0pt, leftmargin=*, noitemsep]
	\item {\bf HAL.} $\eta$: learning rate, $\lambda$: regularization, $\gamma$: mean embedding strength, $\beta$: decay rate, $k$: gradient steps on anchors   
	\item {\bf MER.} $\gamma$: across batch meta-learning rate, $\beta$: within batch meta-learning rate 
	\item {\bf DER.} $\alpha$: loss coefficient for memory logits 
	\item {\bf DER++.} $\alpha$: loss coefficient for memory logits, $\beta$: loss coefficient for memory labels
\end{itemize}
We %For the experiments, we 
used the same architectures and hyperparameters as described in Appendix \ref{paperC:app:experimental_settings} for all datasets, except for the optimizer in HAL where we used the SGD optimizer since using Adam made the model diverge in our experiments. %if not mentioned otherwise. 
We used the Adam optimizer with learning rate $\eta=0.001$ for MER, DER, and DER++. 
%For HAL, we used the SGD optimizer since using Adam made the model diverge in our experiments. 

	content...



\section{Additional Figures}\label{paperC:app:additional_figures}

\paragraph{Memory usage.} We visualize the memory usage for Permuted MNIST and the 20-task datasets Split CIFAR-100 and Split miniImagenet in Figure \ref{fig:memory_usage_10_and_20task_datasets} for our method and the baselines used in the experiment in Section \ref{paperC:sec:efficiency_of_replay_scheduling}. Our method uses a fixed memory size of $M=50$ samples for replay on all three datasets. The memory size capacity for our method is reached after learning task 6 and task 11 on the Permuted MNIST and the 20-task datasets respectively, while the baselines continue incrementing their replay memory size.

\begin{figure}[h]
	\centering
	\vspace{-1mm}
	\setlength{\figwidth}{0.75\textwidth}
	\setlength{\figheight}{.20\textheight}
	



\pgfplotsset{compat=1.11,
    /pgfplots/ybar legend/.style={
    /pgfplots/legend image code/.code={%
       \draw[##1,/tikz/.cd,yshift=-0.25em]
        (0cm,0cm) rectangle (3pt,0.8em);},
   },
}
\pgfplotsset{every axis title/.append style={at={(0.5,0.90)}}}
\pgfplotsset{every tick label/.append style={font=\tiny}}
\pgfplotsset{every major tick/.append style={major tick length=2pt}}
\pgfplotsset{every minor tick/.append style={minor tick length=1pt}}
\pgfplotsset{every axis x label/.append style={at={(0.5,-0.14)}}}
\pgfplotsset{every axis y label/.append style={at={(-0.10,0.5)}}}

\begin{tikzpicture}
\tikzstyle{every node}=[font=\footnotesize]
\definecolor{color0}{rgb}{0.12156862745098,0.466666666666667,0.705882352941177}
\definecolor{color1}{rgb}{1,0.498039215686275,0.0549019607843137}
%\definecolor{color2}{rgb}{0.172549019607843,0.627450980392157,0.172549019607843}

\begin{groupplot}[group style={group size= 1 by 2, horizontal sep=0.9cm, vertical sep=1.2cm}]

\nextgroupplot[title={Memory Usage for 10-task Permuted MNIST },
ybar,
bar width = 4pt,
height=\figheight,
legend cell align={left},
legend columns=2,
legend style={
  fill opacity=0.8,
  draw opacity=1,
  text opacity=1,
  at={(0.03,0.82)},
  anchor=south west,
  draw=white!80!black
},
minor xtick={},
minor ytick={},
tick align=outside,
tick pos=left,
width=\figwidth,
x grid style={white!69.0196078431373!black},
xlabel={Task},
%x label style={at={(axis description cs:0.5,-0.3)},anchor=north},
xmin=1.5, xmax=10.5,
xtick style={color=black},
xtick={2,3,4,5,6,7,8,9,10},
xticklabels={2,3,4,5,6,7,8,9,10},
y grid style={white!69.0196078431373!black},
ymajorgrids,
ylabel={\#Replayed Samples},
ymin=0.0, ymax=102,
ytick style={color=black},
ytick={0,10,20,30,40,50,60,70,80,90,100},
yticklabels={0,10,20,30,40,50,60,70,80,90,100}
]

\addplot [draw=black, fill=color0] coordinates {
(2,10)
(3,20)
(4,30)
(5,40)
(6,50)
(7,50)
(8,50)
(9,50)
(10,50)
};\addlegendentry{Ours}

\addplot [draw=black, fill=color1] coordinates {
(2,10)
(3,20)
(4,30)
(5,40)
(6,50)
(7,60)
(8,70)
(9,80)
(10,90)
};\addlegendentry{Baselines}

\nextgroupplot[title={Memory Usage for the 20-task Datasets},
ybar,
bar width = 4pt,
height=\figheight,
legend cell align={left},
legend columns=2,
legend style={
  fill opacity=0.8,
  draw opacity=1,
  text opacity=1,
  at={(0.03,0.75)},
  anchor=south west,
  draw=white!80!black
},
minor xtick={},
minor ytick={},
tick align=outside,
tick pos=left,
width=\figwidth,
x grid style={white!69.0196078431373!black},
xlabel={Task},
%x label style={at={(axis description cs:0.5,-0.3)},anchor=north},
xmin=1.5, xmax=20.5,
xtick style={color=black},
xtick={2,3,4,5,6,7,8,9,10,11,12,13,14,15,16,17,18,19,20},
xticklabels={2,3,4,5,6,7,8,9,10,11,12,13,14,15,16,17,18,19,20},
y grid style={white!69.0196078431373!black},
ymajorgrids,
ylabel={\#Replayed Samples},
ymin=0.0, ymax=102,
ytick style={color=black},
ytick={0,10,20,30,40,50,60,70,80,90,100},
yticklabels={0,10,20,30,40,50,60,70,80,90,100}
]

\addplot [draw=black, fill=color0] coordinates {
(2,5)
(3,10)
(4,15)
(5,20)
(6,25)
(7,30)
(8,35)
(9,40)
(10,45)
(11,50)
(12,50)
(13,50)
(14,50)
(15,50)
(16,50)
(17,50)
(18,50)
(19,50)
(20,50)
};\addlegendentry{Ours}

\addplot [draw=black, fill=color1] coordinates {
(2,5)
(3,10)
(4,15)
(5,20)
(6,25)
(7,30)
(8,35)
(9,40)
(10,45)
(11,50)
(12,55)
(13,60)
(14,65)
(15,70)
(16,75)
(17,80)
(18,85)
(19,90)
(20,95)
};\addlegendentry{Baselines}

\end{groupplot}

\end{tikzpicture}
	\vspace{-3mm}
	\caption{Number of replayed samples per task for 10-task Permuted MNIST (top) and the 20-task datasets in the experiment in Section \ref{paperC:sec:efficiency_of_replay_scheduling}. The fixed memory size $M=50$ for our method is reached after learning task 6 and task 11 on the Permuted MNIST and the 20-task datasets respectively, while the baselines continue incrementing their number of replay samples per task.}
	\label{fig:memory_usage_10_and_20task_datasets}
\end{figure}

%%% TABLE


\begin{table}[t]
    %\footnotesize
    \centering
    \caption{
    Performance comparison with ACC and BWT metrics for all datasets between MCTS (Ours) and the baselines with various memory selection methods. We provide the metrics for training on all seen task datasets jointly (Joint) as an upper bound. Furthermore, we include the results from a breadth-first search (BFS) with Uniform memory selection for the 5-task datasets. 
	The memory size is set to $M=10$ and $M=100$ for the 5-task and 10/20-task datasets respectively. We report the mean and standard deviation of ACC and BWT, where all results have been averaged over 5 seeds. MCTS performs better or on par than the baselines on most datasets and selection methods, where MoF yields the best performance in general.
    }
    \vspace{-3mm}
    \resizebox{0.95\textwidth}{!}{
    \begin{tabular}{l l c c c c c c}
        \toprule
         & & \multicolumn{2}{c}{{\bf Split MNIST} } & \multicolumn{2}{c}{{\bf Split FashionMNIST} } & \multicolumn{2}{c}{{\bf Split notMNIST} } \\
         \cmidrule(lr){3-4} \cmidrule(lr){5-6} \cmidrule(lr){7-8} %\cline{3-8}
         {\bf Selection} & {\bf Method} & ACC(\%)$\uparrow$ & BWT(\%)$\uparrow$ & ACC(\%)$\uparrow$ & BWT(\%)$\uparrow$ & ACC(\%)$\uparrow$ & BWT(\%)$\uparrow$ \\
        \midrule 
        \multicolumn{1}{c}{--} & Joint & 99.75 ($\pm$ 0.06) & 0.01 ($\pm$ 0.06) & 99.34 ($\pm$ 0.08) & -0.01 ($\pm$ 0.14) & 96.12 ($\pm$ 0.57) & -0.21 ($\pm$ 0.71) \\
		Uniform & BFS & 98.28 ($\pm$ 0.49) & -1.84 ($\pm$ 0.63) & 98.51 ($\pm$ 0.23) & -1.03 ($\pm$ 0.28) & 95.54 ($\pm$ 0.67) & -1.04 ($\pm$ 0.87) \\
		\midrule 
        \multirow{4}{*}{Uniform} & Random & 95.91 ($\pm$ 1.56) & -4.79 ($\pm$ 1.95) & 95.82 ($\pm$ 1.45) & -4.35 ($\pm$ 1.79 ) & 92.39 ($\pm$ 1.29) & -4.56 ($\pm$ 1.29) \\
         & ETS & 94.02 ($\pm$ 4.25) & -7.22 ($\pm$ 5.33) & 95.81 ($\pm$ 3.53) & -4.45 ($\pm$ 4.34) & 91.01 ($\pm$ 1.39) & -6.16 ($\pm$ 1.82) \\
         & Heuristic & 96.02 ($\pm$ 2.32) & -4.64 ($\pm$ 2.90) & 97.09 ($\pm$ 0.62) & -2.82 ($\pm$ 0.84) & 91.26 ($\pm$ 3.99) & -6.06 ($\pm$ 4.70) \\
         & Ours & 97.93 ($\pm$ 0.56) & -2.27 ($\pm$ 0.71) & 98.27 ($\pm$ 0.17) & -1.29 ($\pm$ 0.20) & 94.64 ($\pm$ 0.39) & -1.47 ($\pm$ 0.79) \\
        \midrule 
        \multirow{4}{*}{$k$-means} & Random & 94.24 ($\pm$ 3.20) & -6.88 ($\pm$ 4.00) & 96.30 ($\pm$ 1.62) & -3.77 ($\pm$ 2.05) & 91.64 ($\pm$ 1.39) & -5.64 ($\pm$ 1.77) \\
         & ETS & 92.89 ($\pm$ 3.53) & -8.66 ($\pm$ 4.42) & 96.47 ($\pm$ 0.85) & -3.55 ($\pm$ 1.07) & 93.80 ($\pm$ 0.82) & -2.84 ($\pm$ 0.81) \\
         & Heuristic & 96.28 ($\pm$ 1.68) & -4.32 ($\pm$ 2.11) & 95.78 ($\pm$ 1.50) & -4.46 ($\pm$ 1.87) & 91.75 ($\pm$ 0.94) & -5.60 ($\pm$ 2.07) \\
         & Ours & 98.20 ($\pm$ 0.16) & -1.94 ($\pm$ 0.22) & 98.48 ($\pm$ 0.26) & -1.04 ($\pm$ 0.31) & 93.61 ($\pm$ 0.71) & -3.11 ($\pm$ 0.55) \\
        \midrule 
        \multirow{4}{*}{$k$-center} & Random & 96.40 ($\pm$ 0.68) & -4.21 ($\pm$ 0.84) & 95.57 ($\pm$ 3.16) & -7.20 ($\pm$ 3.93) & 92.61 ($\pm$ 1.70) & -4.14 ($\pm$ 2.37) \\
         & ETS & 94.84 ($\pm$ 1.40) & -6.20 ($\pm$ 1.77) & 97.28 ($\pm$ 0.50) & -2.58 ($\pm$ 0.66) & 91.08 ($\pm$ 2.48) & -6.39 ($\pm$ 3.46) \\
         & Heuristic & 94.55 ($\pm$ 2.79) & -6.47 ($\pm$ 3.50) & 94.08 ($\pm$ 3.72) & -6.59 ($\pm$ 4.57) & 92.06 ($\pm$ 1.20) & -4.70 ($\pm$ 2.09) \\
         & Ours & 98.24 ($\pm$ 0.36) & -1.93 ($\pm$ 0.44) & 98.06 ($\pm$ 0.35) & -1.59 ($\pm$ 0.45) & 94.26 ($\pm$ 0.37) & -1.97 ($\pm$ 1.02) \\
        \midrule
        \multirow{4}{*}{MoF} & Random & 95.18 ($\pm$ 3.18) & -5.73 ($\pm$ 3.95) & 95.76 ($\pm$ 1.41) & -4.41 ($\pm$ 1.75) & 91.33 ($\pm$ 1.75) & -5.94 ($\pm$ 1.92) \\
         & ETS & 97.04 ($\pm$ 1.23) & -3.46 ($\pm$ 1.50) & 96.48 ($\pm$ 1.33) & -3.55 ($\pm$ 1.73) & 92.64 ($\pm$ 0.87) & -4.57 ($\pm$ 1.59) \\
         & Heuristic & 96.46 ($\pm$ 2.41) & -4.09 ($\pm$ 3.01) & 95.84 ($\pm$ 0.89) & -4.39 ($\pm$ 1.15) & 93.24 ($\pm$ 0.77) & -3.48 ($\pm$ 1.37) \\
         & Ours & 98.37 ($\pm$ 0.24) & -1.70 ($\pm$ 0.28) & 97.84 ($\pm$ 0.32) & -1.81 ($\pm$ 0.39) & 94.62 ($\pm$ 0.42) & -1.80 ($\pm$ 0.56) \\
        \bottomrule
        \toprule
         & & \multicolumn{2}{c}{{\bf Permuted MNIST} } & \multicolumn{2}{c}{{\bf Split CIFAR-100} } & \multicolumn{2}{c}{{\bf Split miniImagenet} } \\
         \cmidrule(lr){3-4} \cmidrule(lr){5-6} \cmidrule(lr){7-8} %\cline{3-8}
         {\bf Selection} & {\bf Method} & ACC(\%)$\uparrow$ & BWT(\%)$\uparrow$ & ACC(\%)$\uparrow$ & BWT(\%)$\uparrow$ & ACC(\%)$\uparrow$ & BWT(\%)$\uparrow$ \\
        \midrule 
        \multicolumn{1}{c}{--} & Joint & 95.34 ($\pm$ 0.13) & 0.17 ($\pm$ 0.18) & 84.73 ($\pm$ 0.81) & -1.06 ($\pm$ 0.81) & 74.03 ($\pm$ 0.83) & 9.70 ($\pm$ 0.68) \\
		\midrule
        \multirow{4}{*}{Uniform} & Random & 72.44 ($\pm$ 1.15) & -25.56 ($\pm$ 1.39) & 53.99 ($\pm$ 0.51) & -34.19 ($\pm$ 0.66) & 48.08 ($\pm$ 1.36) & -15.98 ($\pm$ 1.08) \\
         & ETS & 71.09 ($\pm$ 2.31) & -27.39 ($\pm$ 2.59) & 47.70 ($\pm$ 2.16) & -41.68 ($\pm$ 2.37) & 46.97 ($\pm$ 1.24) & -18.32 ($\pm$ 1.34) \\
         & Heuristic & 76.68 ($\pm$ 2.13) & -20.82 ($\pm$ 2.41) & 57.31 ($\pm$ 1.21) & -30.76 ($\pm$ 1.45) & 49.66 ($\pm$ 1.10) & -12.04 ($\pm$ 0.59) \\
         & Ours & 76.34 ($\pm$ 0.98) & -21.21 ($\pm$ 1.16) & 56.60 ($\pm$ 1.13) & -31.39 ($\pm$ 1.11) & 50.20 ($\pm$ 0.72) & -13.46 ($\pm$ 1.22) \\
        \midrule 
        \multirow{4}{*}{$k$-means} & Random & 74.30 ($\pm$ 1.43) & -23.50 ($\pm$ 1.64) & 53.18 ($\pm$ 1.66) & -35.15 ($\pm$ 1.61) & 49.47 ($\pm$ 2.70) & -14.10 ($\pm$ 2.71) \\
         & ETS & 69.40 ($\pm$ 1.32) & -29.23 ($\pm$ 1.47) & 47.51 ($\pm$ 1.14) & -41.77 ($\pm$ 1.30) & 45.82 ($\pm$ 0.92) & -19.53 ($\pm$ 1.10) \\
         & Heuristic & 75.57 ($\pm$ 1.18) & -22.11 ($\pm$ 1.22) & 54.31 ($\pm$ 3.94) & -33.80 ($\pm$ 4.24) & 49.25 ($\pm$ 1.00) & -12.92 ($\pm$ 1.22) \\
         & Ours & 77.74 ($\pm$ 0.80) & -19.66 ($\pm$ 0.95) & 56.95 ($\pm$ 0.92) & -30.92 ($\pm$ 0.83) & 50.47 ($\pm$ 0.85) & -13.31 ($\pm$ 1.24) \\
        \midrule 
        \multirow{4}{*}{$k$-center} & Random & 71.41 ($\pm$ 2.75) & -26.73 ($\pm$ 3.11) & 48.46 ($\pm$ 0.31) & -39.89 ($\pm$ 0.27) & 44.76 ($\pm$ 0.96) & -18.72 ($\pm$ 1.17) \\
         & ETS & 69.11 ($\pm$ 1.69) & -29.58 ($\pm$ 1.81) & 44.13 ($\pm$ 1.06) & -45.28 ($\pm$ 1.04) & 41.35 ($\pm$ 0.96) & -23.71 ($\pm$ 1.45) \\
         & Heuristic & 74.33 ($\pm$ 2.00) & -23.45 ($\pm$ 2.27) & 50.32 ($\pm$ 1.97) & -37.99 ($\pm$ 2.14) & 44.13 ($\pm$ 0.95) & -18.26 ($\pm$ 1.05) \\
         & Ours & 76.55 ($\pm$ 1.16) & -21.06 ($\pm$ 1.32) & 51.37 ($\pm$ 1.63) & -37.01 ($\pm$ 1.62) & 46.76 ($\pm$ 0.96) & -16.56 ($\pm$ 0.90) \\
        \midrule
        \multirow{4}{*}{MoF} & Random & 77.96 ($\pm$ 1.84) & -19.44 ($\pm$ 2.13) & 61.93 ($\pm$ 1.05) & -25.89 ($\pm$ 1.07) & 54.50 ($\pm$ 1.33) & -8.64 ($\pm$ 1.26) \\
         & ETS & 77.62 ($\pm$ 1.12) & -20.10 ($\pm$ 1.26) & 60.43 ($\pm$ 1.17) & -28.22 ($\pm$ 1.26) & 56.12 ($\pm$ 1.12) & -8.93 ($\pm$ 0.83) \\
         & Heuristic & 77.27 ($\pm$ 1.45) & -20.15 ($\pm$ 1.63) & 55.60 ($\pm$ 2.70) & -32.57 ($\pm$ 2.77) & 52.30 ($\pm$ 0.59) & -9.61 ($\pm$ 0.67) \\
         & Ours & 81.58 ($\pm$ 0.75) & -15.41 ($\pm$ 0.86) & 64.22 ($\pm$ 0.65) & -23.48 ($\pm$ 1.02) & 57.70 ($\pm$ 0.51) & -5.31 ($\pm$ 0.55) \\
        \bottomrule
    \end{tabular}
    }
    \vspace{-2mm}
    \label{tab:bwt_alternative_memory_selection}
\end{table}


\begin{comment}
\begin{table}[h]
    \footnotesize
    \centering
    \caption{
    Performance comparison with ACC and BWT metrics for all datasets between RS-MCTS and the baselines with various memory selection methods. The memory size is set to $M=10$ and $M=100$ for the 5-task and 10/20-task datasets respectively. We report the mean and standard deviation of ACC and BWT, where all results have been averaged over 5 seeds. RS-MCTS performs better or on par than the baselines on most datasets and selection methods, where MoF yields the best performance in general.
    }
    %\vspace{-3mm}
    \begin{tabular}{l l c c c c c c}
        \toprule
         & & \multicolumn{2}{c}{{\bf Split MNIST} } & \multicolumn{2}{c}{{\bf Split FashionMNIST} } & \multicolumn{2}{c}{{\bf Split notMNIST} } \\
         \cmidrule(lr){3-4} \cmidrule(lr){5-6} \cmidrule(lr){7-8} %\cline{3-8}
         {\bf Selection} & {\bf Method} & ACC(\%)$\uparrow$ & BWT(\%)$\uparrow$ & ACC(\%)$\uparrow$ & BWT(\%)$\uparrow$ & ACC(\%)$\uparrow$ & BWT(\%)$\uparrow$ \\
        \midrule 
        \multirow{4}{*}{Uniform} & Random & 95.91 $\pm$ 1.56 & -4.79 $\pm$ 1.95 & 95.82 $\pm$ 1.45 & -4.35 $\pm$ 1.79  & 92.39 $\pm$ 1.29 & -4.56 $\pm$ 1.29 \\
         & ETS & 94.02 $\pm$ 4.25 & -7.22 $\pm$ 5.33 & 95.81 $\pm$ 3.53 & -4.45 $\pm$ 4.34  & 91.01 $\pm$ 1.39 & -6.16 $\pm$ 1.82 \\
         & Heuristic & 96.02 $\pm$ 2.32 & -4.64 $\pm$ 2.90 & 97.09 $\pm$ 0.62 & -2.82 $\pm$ 0.84  & 91.26 $\pm$ 3.99 & -6.06 $\pm$ 4.70 \\
         & Ours & 97.93 $\pm$ 0.56 & -2.27 $\pm$ 0.71 & 98.27 $\pm$ 0.17 & -1.29 $\pm$ 0.20  & 94.64 $\pm$ 0.39 & -1.47 $\pm$ 0.79 \\
        \midrule 
        \multirow{4}{*}{$k$-means} & Random & 94.24 $\pm$ 3.20 & -6.88 $\pm$ 4.00 & 96.30 $\pm$ 1.62 & -3.77 $\pm$ 2.05  & 91.64 $\pm$ 1.39 & -5.64 $\pm$ 1.77 \\
         & ETS & 92.89 $\pm$ 3.53 & -8.66 $\pm$ 4.42 & 96.47 $\pm$ 0.85 & -3.55 $\pm$ 1.07  & 93.80 $\pm$ 0.82 & -2.84 $\pm$ 0.81 \\
         & Heuristic & 96.28 $\pm$ 1.68 & -4.32 $\pm$ 2.11 & 95.78 $\pm$ 1.50 & -4.46 $\pm$ 1.87  & 91.75 $\pm$ 0.94 & -5.60 $\pm$ 2.07 \\
         & Ours & 98.20 $\pm$ 0.16 & -1.94 $\pm$ 0.22 & 98.48 $\pm$ 0.26 & -1.04 $\pm$ 0.31  & 93.61 $\pm$ 0.71 & -3.11 $\pm$ 0.55 \\
        \midrule 
        \multirow{4}{*}{$k$-center} & Random & 96.40 $\pm$ 0.68 & -4.21 $\pm$ 0.84 & 95.57 $\pm$ 3.16 & -7.20 $\pm$ 3.93  & 92.61 $\pm$ 1.70 & -4.14 $\pm$ 2.37 \\
         & ETS & 94.84 $\pm$ 1.40 & -6.20 $\pm$ 1.77 & 97.28 $\pm$ 0.50 & -2.58 $\pm$ 0.66  & 91.08 $\pm$ 2.48 & -6.39 $\pm$ 3.46 \\
         & Heuristic & 94.55 $\pm$ 2.79 & -6.47 $\pm$ 3.50 & 94.08 $\pm$ 3.72 & -6.59 $\pm$ 4.57  & 92.06 $\pm$ 1.20 & -4.70 $\pm$ 2.09 \\
         & Ours & 98.24 $\pm$ 0.36 & -1.93 $\pm$ 0.44 & 98.06 $\pm$ 0.35 & -1.59 $\pm$ 0.45  & 94.26 $\pm$ 0.37 & -1.97 $\pm$ 1.02 \\
        \midrule
        \multirow{4}{*}{MoF} & Random & 95.18 $\pm$ 3.18 & -5.73 $\pm$ 3.95 & 95.76 $\pm$ 1.41 & -4.41 $\pm$ 1.75  & 91.33 $\pm$ 1.75 & -5.94 $\pm$ 1.92 \\
         & ETS & 97.04 $\pm$ 1.23 & -3.46 $\pm$ 1.50 & 96.48 $\pm$ 1.33 & -3.55 $\pm$ 1.73  & 92.64 $\pm$ 0.87 & -4.57 $\pm$ 1.59 \\
         & Heuristic & 96.46 $\pm$ 2.41 & -4.09 $\pm$ 3.01 & 95.84 $\pm$ 0.89 & -4.39 $\pm$ 1.15  & 93.24 $\pm$ 0.77 & -3.48 $\pm$ 1.37 \\
         & Ours & 98.37 $\pm$ 0.24 & -1.70 $\pm$ 0.28 & 97.84 $\pm$ 0.32 & -1.81 $\pm$ 0.39  & 94.62 $\pm$ 0.42 & -1.80 $\pm$ 0.56 \\
        \bottomrule
        \toprule
         & & \multicolumn{2}{c}{{\bf Permuted MNIST} } & \multicolumn{2}{c}{{\bf Split CIFAR-100} } & \multicolumn{2}{c}{{\bf Split miniImagenet} } \\
         \cmidrule(lr){3-4} \cmidrule(lr){5-6} \cmidrule(lr){7-8} %\cline{3-8}
         {\bf Selection} & {\bf Method} & ACC(\%)$\uparrow$ & BWT(\%)$\uparrow$ & ACC(\%)$\uparrow$ & BWT(\%)$\uparrow$ & ACC(\%)$\uparrow$ & BWT(\%)$\uparrow$ \\
        \midrule 
        \multirow{4}{*}{Uniform} & Random & 72.44 $\pm$ 1.15 & -25.56 $\pm$ 1.39 & 53.99 $\pm$ 0.51 & -34.19 $\pm$ 0.66  & 48.08 $\pm$ 1.36 & -15.98 $\pm$ 1.08 \\
         & ETS & 71.09 $\pm$ 2.31 & -27.39 $\pm$ 2.59 & 47.70 $\pm$ 2.16 & -41.68 $\pm$ 2.37  & 46.97 $\pm$ 1.24 & -18.32 $\pm$ 1.34 \\
         & Heuristic & 76.68 $\pm$ 2.13 & -20.82 $\pm$ 2.41 & 57.31 $\pm$ 1.21 & -30.76 $\pm$ 1.45  & 49.66 $\pm$ 1.10 & -12.04 $\pm$ 0.59 \\
         & Ours & 76.34 $\pm$ 0.98 & -21.21 $\pm$ 1.16 & 56.60 $\pm$ 1.13 & -31.39 $\pm$ 1.11  & 50.20 $\pm$ 0.72 & -13.46 $\pm$ 1.22 \\
        \midrule 
        \multirow{4}{*}{$k$-means} & Random & 74.30 $\pm$ 1.43 & -23.50 $\pm$ 1.64 & 53.18 $\pm$ 1.66 & -35.15 $\pm$ 1.61  & 49.47 $\pm$ 2.70 & -14.10 $\pm$ 2.71 \\
         & ETS & 69.40 $\pm$ 1.32 & -29.23 $\pm$ 1.47 & 47.51 $\pm$ 1.14 & -41.77 $\pm$ 1.30  & 45.82 $\pm$ 0.92 & -19.53 $\pm$ 1.10 \\
         & Heuristic & 75.57 $\pm$ 1.18 & -22.11 $\pm$ 1.22 & 54.31 $\pm$ 3.94 & -33.80 $\pm$ 4.24  & 49.25 $\pm$ 1.00 & -12.92 $\pm$ 1.22 \\
         & Ours & 77.74 $\pm$ 0.80 & -19.66 $\pm$ 0.95 & 56.95 $\pm$ 0.92 & -30.92 $\pm$ 0.83  & 50.47 $\pm$ 0.85 & -13.31 $\pm$ 1.24 \\
        \midrule 
        \multirow{4}{*}{$k$-center} & Random & 71.41 $\pm$ 2.75 & -26.73 $\pm$ 3.11 & 48.46 $\pm$ 0.31 & -39.89 $\pm$ 0.27  & 44.76 $\pm$ 0.96 & -18.72 $\pm$ 1.17  \\
         & ETS & 69.11 $\pm$ 1.69 & -29.58 $\pm$ 1.81 & 44.13 $\pm$ 1.06 & -45.28 $\pm$ 1.04  & 41.35 $\pm$ 0.96 & -23.71 $\pm$ 1.45 \\
         & Heuristic & 74.33 $\pm$ 2.00 & -23.45 $\pm$ 2.27 & 50.32 $\pm$ 1.97 & -37.99 $\pm$ 2.14  & 44.13 $\pm$ 0.95 & -18.26 $\pm$ 1.05 \\
         & Ours & 76.55 $\pm$ 1.16 & -21.06 $\pm$ 1.32 & 51.37 $\pm$ 1.63 & -37.01 $\pm$ 1.62  & 46.76 $\pm$ 0.96 & -16.56 $\pm$ 0.90 \\
        \midrule
        \multirow{4}{*}{MoF} & Random & 77.96 $\pm$ 1.84 & -19.44 $\pm$ 2.13 & 61.93 $\pm$ 1.05 & -25.89 $\pm$ 1.07  & 54.50 $\pm$ 1.33 & -8.64 $\pm$ 1.26 \\
         & ETS & 77.62 $\pm$ 1.12 & -20.10 $\pm$ 1.26 & 60.43 $\pm$ 1.17 & -28.22 $\pm$ 1.26  & 56.12 $\pm$ 1.12 & -8.93 $\pm$ 0.83 \\
         & Heuristic & 77.27 $\pm$ 1.45 & -20.15 $\pm$ 1.63 & 55.60 $\pm$ 2.70 & -32.57 $\pm$ 2.77  & 52.30 $\pm$ 0.59 & -9.61 $\pm$ 0.67 \\
         & Ours & 81.58 $\pm$ 0.75 & -15.41 $\pm$ 0.86 & 64.22 $\pm$ 0.65 & -23.48 $\pm$ 1.02  & 57.70 $\pm$ 0.51 & -5.31 $\pm$ 0.55 \\
        \bottomrule
    \end{tabular}
    \vspace{-3mm}
    \label{tab:bwt_alternative_memory_selection}
\end{table}
\end{comment}

\begin{table}[t]
	%\scriptsize
	\centering
	\caption{
		Hyperparameters for replay-based methods HAL, MER, DER and DER++ used in experiments on applying RS-MCTS to recent replay-based methods in Section \ref{paperC:sec:applying_scheduling_to_recent_replay_methods}.
	}
	\vspace{-3mm}
	\resizebox{0.95\textwidth}{!}{
		\begin{tabular}{l c c c c c c c}
			\toprule
			& & \multicolumn{3}{c}{{\bf 5-task Datasets}} & \multicolumn{3}{c}{{\bf 10- and 20-task Datasets}} \\
			\cmidrule(lr){3-5} \cmidrule(lr){6-8}
			{\bf Method} & {\bf Hyperparam.}  & S-MNIST & S-FashionMNIST & S-notMNIST & P-MNIST & S-CIFAR-100 & S-miniImagenet \\
			\midrule
			\multirow{5}{*}{HAL} & $\eta$ & 0.1 & 0.1 & 0.1 & 0.1 & 0.03 & 0.03 \\
			& $\lambda$  & 0.1 & 0.1 & 0.1 & 0.1 & 1.0 & 0.03 \\
			& $\gamma$ & 0.5 & 0.1 & 0.1 & 0.1 & 0.1 & 0.1 \\
			& $\beta$ & 0.7 & 0.5 & 0.5 & 0.5 & 0.5 & 0.5 \\
			& $k$ & 100 & 100 & 100 & 100 & 100 & 100 \\
			\midrule 
			\multirow{2}{*}{MER} & $\gamma$ & 1.0 & 1.0 & 1.0 & 1.0 & 1.0 & 1.0  \\ 
			& $\beta$  & 1.0 & 0.01 & 1.0 & 1.0 & 0.1 & 0.1 \\
			\midrule 
			\multirow{1}{*}{DER} & $\alpha$ & 0.2 & 0.2 & 0.1 & 1.0 & 1.0 & 0.1 \\ 
			\midrule 
			\multirow{2}{*}{DER++} & $\alpha$ & 0.2 & 0.2 & 0.1 & 1.0 & 1.0 & 0.1 \\ 
			& $\beta$ & 1.0 & 1.0 & 1.0 & 1.0 & 1.0 & 1.0 \\
			\bottomrule
		\end{tabular}
	}
	\vspace{-2mm}
	\label{tab:hyperparameters_applying_scheduling_to_recent_replay_methods}
\end{table}





\begin{table}[t]
    \footnotesize
    \centering
    \caption{
    Performance comparison with ACC and BWT metrics between scheduling methods RS-MCTS (Ours), Random, ETS, and Heuristic when combining them with replay-based methods Hindsight Anchor Learning (HAL), Meta Experience Replay (MER), Dark Experience Replay (DER), and DER++. 
    Replay memory sizes are $M=10$ and $M=100$ for the 5-task and 10/20-task datasets respectively. We report the mean and standard deviation averaged over 5 seeds. Results on Heuristic where some seed did not converge is denoted by $^{*}$. Applying RS-MCTS to each method can enhance the performance compared to using the baseline schedules. 
    }
    %\vspace{-3mm}
    \resizebox{\textwidth}{!}{ 
    \begin{tabular}{l l c c c c c c}
        \toprule
         & & \multicolumn{2}{c}{{\bf Split MNIST} } & \multicolumn{2}{c}{{\bf Split FashionMNIST} } & \multicolumn{2}{c}{{\bf Split notMNIST} } \\
         \cmidrule{3-4} \cmidrule{5-6} \cmidrule{7-8} %\cline{3-8}
         {\bf Method} & {\bf Schedule} & ACC(\%)$\uparrow$ & BWT(\%)$\uparrow$ & ACC(\%)$\uparrow$ & BWT(\%)$\uparrow$ & ACC(\%)$\uparrow$ & BWT(\%)$\uparrow$ \\
        \midrule 

        \multirow{4}{*}{HAL} & Random & 97.24 ($\pm$ 0.70) & -2.77 ($\pm$ 0.90) & 86.74 ($\pm$ 6.05) & -15.54 ($\pm$ 7.58) & 93.61 ($\pm$ 1.31) & -2.73 ($\pm$ 1.33) \\
         & ETS & 97.21 ($\pm$ 1.25) & -2.80 ($\pm$ 1.59) & 96.75 ($\pm$ 0.50) & -2.84 ($\pm$ 0.75) & 92.16 ($\pm$ 1.82) &  -5.04 ($\pm$ 2.24) \\
         & Heuristic & 97.69 ($\pm$ 0.19) & -2.22 ($\pm$ 0.24) & $^{*}$74.16 ($\pm$ 11.19) & -31.26 ($\pm$ 14.00) & 93.64 ($\pm$ 0.93) & -2.80 ($\pm$ 1.20) \\
         & Ours & 97.96 ($\pm$ 0.15) & -1.85 ($\pm$ 0.18) & 97.56 ($\pm$ 0.51) & -2.02 ($\pm$ 0.63) & 94.47 ($\pm$ 0.82) &  -1.67 ($\pm$ 0.64) \\
        \midrule 
        \multirow{4}{*}{MER} & Random & 93.07 ($\pm$ 0.81)  & -8.36 ($\pm$ 0.99) & 85.53 ($\pm$ 3.30) & -0.56 ($\pm$ 3.36) & 91.13 ($\pm$ 0.86) & -5.32 ($\pm$ 0.95) \\
         & ETS & 92.97 ($\pm$ 1.73) &  -8.52 ($\pm$ 2.15) & 84.88 ($\pm$ 3.85) & -3.34 ($\pm$ 5.59) & 90.56 ($\pm$ 0.83) & -6.11 ($\pm$ 1.06) \\
         & Heuristic & 94.30 ($\pm$ 2.79) & -6.46 ($\pm$ 3.50) & 96.91 ($\pm$ 0.62) & -1.34 ($\pm$ 0.76) & 90.90 ($\pm$ 1.30) & -6.24 ($\pm$ 1.96) \\
         & Ours & 96.44 ($\pm$ 0.72)  & -4.14 ($\pm$ 0.94)  & 86.67 ($\pm$ 4.09) & 0.85 ($\pm$ 3.85)  & 92.44 ($\pm$ 0.77) & -3.63 ($\pm$ 1.06) \\
        \midrule 
        \multirow{4}{*}{DER} & Random & 98.23 ($\pm$ 0.53) & -1.89 ($\pm$ 0.65) & 96.56 ($\pm$ 1.79) & -3.48 ($\pm$ 2.22) & 92.89 ($\pm$ 0.86) & -3.75 ($\pm$ 1.16) \\
         & ETS & 98.17 ($\pm$ 0.35) & -2.00 ($\pm$ 0.42) & 97.69 ($\pm$ 0.58) & -2.05 ($\pm$ 0.71) & 94.74 ($\pm$ 1.05) & -1.94 ($\pm$ 1.17) \\
         & Heuristic & 94.57 ($\pm$ 1.71) & -6.08 ($\pm$ 2.09) & $^{*}$72.49 ($\pm$ 19.32) & -20.88 ($\pm$ 11.46) & $^{*}$77.88 ($\pm$ 12.58) & -12.66 ($\pm$ 4.17) \\
         & Ours & 99.02 ($\pm$ 0.10) & -0.91 ($\pm$ 0.13) & 98.33 ($\pm$ 0.51) & -1.26 ($\pm$ 0.63) & 95.02 ($\pm$ 0.33) & -0.97 ($\pm$ 0.81) \\
        \midrule
        \multirow{4}{*}{DER++} & Random & 97.90 ($\pm$ 0.52) & -2.32 ($\pm$ 0.67) & 97.10 ($\pm$ 1.03) & -2.77 ($\pm$ 1.29) & 93.29 ($\pm$ 1.43) & -3.11 ($\pm$ 1.60 ) \\
         & ETS & 97.98 ($\pm$ 0.52) & -2.24 ($\pm$ 0.66) & 98.12 ($\pm$ 0.40) & -1.59 ($\pm$ 0.52) & 94.53 ($\pm$ 1.02) & -1.82 ($\pm$ 1.02) \\
         & Heuristic & 92.35 ($\pm$ 2.42) & -8.83 ($\pm$ 2.99) & $^{*}$67.31 ($\pm$ 21.20) & -24.86 ($\pm$ 16.34) & 93.88 ($\pm$ 1.33) & -2.86 ($\pm$ 1.49) \\
         & Ours & 98.84 ($\pm$ 0.21)  & -1.14 ($\pm$ 0.26)  & 98.38 ($\pm$ 0.43) & -1.17 ($\pm$ 0.51) & 94.73 ($\pm$ 0.20) & -1.21 ($\pm$ 1.12) \\
        \bottomrule
        \toprule
         & & \multicolumn{2}{c}{{\bf Permuted MNIST} } & \multicolumn{2}{c}{{\bf Split CIFAR-100} } & \multicolumn{2}{c}{{\bf Split miniImagenet} } \\
         \cmidrule{3-4} \cmidrule{5-6} \cmidrule{7-8} %\cline{3-8}
         {\bf Method} & {\bf Schedule} & ACC(\%)$\uparrow$ & BWT(\%)$\uparrow$ & ACC(\%)$\uparrow$ & BWT(\%)$\uparrow$ & ACC(\%)$\uparrow$ & BWT(\%)$\uparrow$ \\
        \midrule 
        \multirow{4}{*}{HAL} & Random & 88.49 ($\pm$ 0.99) & -7.03 ($\pm$ 1.05) & 36.09 ($\pm$ 1.77) & -17.49 ($\pm$ 1.78 ) & 38.51 ($\pm$ 2.22) & -6.65 ($\pm$ 1.43) \\
         & ETS & 88.46 ($\pm$ 0.86) & -7.26 ($\pm$ 0.90) & 34.90 ($\pm$ 2.02) & -18.92 ($\pm$ 0.91) & 38.13 ($\pm$ 1.18) & -8.19 ($\pm$ 1.73) \\
         & Heuristic & $^{*}$66.63 ($\pm$ 28.50) & -29.68 ($\pm$ 27.90) & 35.07 ($\pm$ 1.29) & -24.76 ($\pm$ 2.41) & 39.51 ($\pm$ 1.49) & -5.65 ($\pm$ 0.77) \\
         & Ours & 89.14 ($\pm$ 0.74) & -6.29 ($\pm$ 0.74) & 40.22 ($\pm$ 1.57) & -12.77 ($\pm$ 1.30) & 41.39 ($\pm$ 1.15) & -3.69 ($\pm$ 1.86) \\
        \midrule 
        \multirow{4}{*}{MER} & Random & 75.90 ($\pm$ 1.34) & -21.69 ($\pm$ 1.47) & 42.96 ($\pm$ 1.70) & -34.01 ($\pm$ 2.07) & 31.48 ($\pm$ 1.65) & -6.99 ($\pm$ 1.27) \\
         & ETS & 73.01 ($\pm$ 0.96) & -25.19 ($\pm$ 1.10) & 43.38 ($\pm$ 1.81) & -34.84 ($\pm$ 1.98) & 33.58 ($\pm$ 1.53) & -6.80 ($\pm$ 1.46) \\
         & Heuristic & 83.86 ($\pm$ 3.19) & -12.48 ($\pm$ 3.60) & 40.90 ($\pm$ 1.70) & -44.10 ($\pm$ 2.03) & 34.22 ($\pm$ 1.93) & -7.57 ($\pm$ 1.63) \\
         & Ours & 79.72 ($\pm$ 0.71) & -17.42 ($\pm$ 0.78) & 44.29 ($\pm$ 0.69) & -32.73 ($\pm$ 0.88) & 32.74 ($\pm$ 1.29) & -5.77 ($\pm$ 1.04) \\
        \midrule 
        \multirow{4}{*}{DER} & Random & 87.51 ($\pm$ 1.10) & -8.81 ($\pm$ 1.28) & 56.83 ($\pm$ 0.76) & -27.34 ($\pm$ 0.63) & 42.19  ($\pm$ 0.67) & -10.60  ($\pm$ 1.28) \\
         & ETS & 85.71 ($\pm$ 0.75) & -11.15 ($\pm$ 0.87) & 52.58 ($\pm$ 1.49) & -32.93 ($\pm$ 2.04) & 35.50  ($\pm$ 2.84) & -10.94  ($\pm$ 2.21) \\
         & Heuristic & 81.56 ($\pm$ 2.28) & -15.06 ($\pm$ 2.51) & 55.75 ($\pm$ 1.08) & -31.27 ($\pm$ 1.02) & 43.62 ($\pm$ 0.88) & -8.18 ($\pm$ 1.16) \\
         & Ours & 90.11 ($\pm$ 0.18) & -5.89 ($\pm$ 0.23) & 58.99 ($\pm$ 0.98) & -24.95 ($\pm$ 0.64) & 43.46  ($\pm$ 0.95) & -9.32  ($\pm$ 1.37) \\
        \midrule
        \multirow{4}{*}{DER++} & Random & 87.89 ($\pm$ 1.10& -8.35 ($\pm$ 1.33) & 58.49  ($\pm$ 1.44) & -25.48  ($\pm$ 1.55) & 48.40  ($\pm$ 0.69) & -4.20  ($\pm$ 0.86) \\
         & ETS & 85.25 ($\pm$ 0.88) & -11.60 ($\pm$ 1.03) & 52.54  ($\pm$ 1.06) & -33.22  ($\pm$ 1.51) & 41.36  ($\pm$ 2.90) & -4.07  ($\pm$ 2.28) \\
         & Heuristic & 79.17 ($\pm$ 2.44) & -17.68 ($\pm$ 2.68) & 56.70 ($\pm$ 1.27) & -30.33 ($\pm$ 1.41) & 45.73 ($\pm$ 0.84) & -6.09 ($\pm$ 1.24) \\
         & Ours & 89.84 ($\pm$ 0.22) & -6.13 ($\pm$ 0.29) & 59.23  ($\pm$ 0.83) & -24.61  ($\pm$ 0.91) & 49.45  ($\pm$ 0.68) & -3.12  ($\pm$ 0.89) \\
        \bottomrule
    \end{tabular}
    }
    \vspace{-3mm}
    \label{tab:bwt_sota_models_applied_to_rsmcts}
\end{table}


\begin{comment}
\begin{table}[h]
    \footnotesize
    \centering
    \caption{
    Performance comparison between scheduling methods RS-MCTS (Ours), Random, ETS, and Heuristic when combining them with replay-based methods Hindsight Anchor Learning (HAL), Meta Experience Replay (MER), Dark Experience Replay (DER), and DER++. 
    Replay memory sizes are $M=10$ and $M=100$ for the 5-task and 10/20-task datasets respectively. We report the mean and standard deviation averaged over 5 seeds. Results on Heuristic where some seed did not converge is denoted by $^{*}$. Applying RS-MCTS to each method can enhance the performance compared to using the baseline schedules. 
    }
    %\vspace{-3mm}
    \scalebox{0.9}{
    \begin{tabular}{l l c c c c c c}
        \toprule
         & & \multicolumn{2}{c}{{\bf Split MNIST} } & \multicolumn{2}{c}{{\bf Split FashionMNIST} } & \multicolumn{2}{c}{{\bf Split notMNIST} } \\
         \cmidrule{3-4} \cmidrule{5-6} \cmidrule{7-8} %\cline{3-8}
         {\bf Method} & {\bf Schedule} & ACC(\%)$\uparrow$ & BWT(\%)$\uparrow$ & ACC(\%)$\uparrow$ & BWT(\%)$\uparrow$ & ACC(\%)$\uparrow$ & BWT(\%)$\uparrow$ \\
        \midrule 

        \multirow{4}{*}{HAL} & Random & 97.24 $\pm$ 0.70 & -2.77 $\pm$ 0.90 & 86.74 $\pm$ 6.05 & -15.54 $\pm$ 7.58 & 93.61 $\pm$ 1.31 & -2.73 $\pm$ 1.33 \\
         & ETS & 97.21 $\pm$ 1.25 & -2.80 $\pm$ 1.59 & 96.75 $\pm$ 0.50 & -2.84 $\pm$ 0.75 & 92.16 $\pm$ 1.82 &  -5.04 $\pm$ 2.24 \\
         & Heuristic & 97.69 $\pm$ 0.19 & -2.22 $\pm$ 0.24 & $^{*}$74.16 $\pm$ 11.19 & -31.26 $\pm$ 14.00 & 93.64 $\pm$ 0.93 & -2.80 $\pm$ 1.20 \\
         & Ours & 97.96 $\pm$ 0.15 & -1.85 $\pm$ 0.18 & 97.56 $\pm$ 0.51 & -2.02 $\pm$ 0.63 & 94.47 $\pm$ 0.82 &  -1.67 $\pm$ 0.64 \\
        \midrule 
        \multirow{4}{*}{MER} & Random & 93.07 $\pm$ 0.81  & -8.36 $\pm$ 0.99 & 85.53 $\pm$ 3.30 & -0.56 $\pm$ 3.36 & 91.13 $\pm$ 0.86 & -5.32 $\pm$ 0.95 \\
         & ETS & 92.97 $\pm$ 1.73 &  -8.52 $\pm$ 2.15 & 84.88 $\pm$ 3.85 & -3.34 $\pm$ 5.59 & 90.56 $\pm$ 0.83 & -6.11 $\pm$ 1.06 \\
         & Heuristic & 94.30 $\pm$ 2.79 & -6.46 $\pm$ 3.50 & 96.91 $\pm$ 0.62 & -1.34 $\pm$ 0.76 & 90.90 $\pm$ 1.30 & -6.24 $\pm$ 1.96 \\
         & Ours & {96.44 $\pm$ 0.72}  & {-4.14 $\pm$ 0.94}  & 86.67 $\pm$ 4.09 & 0.85 $\pm$ 3.85  & 92.44 $\pm$ 0.77 & -3.63 $\pm$ 1.06 \\
        \midrule 
        \multirow{4}{*}{DER} & Random & 98.23 $\pm$ 0.53 & -1.89 $\pm$ 0.65 & 96.56 $\pm$ 1.79 & -3.48 $\pm$ 2.22 & 92.89 $\pm$ 0.86 & -3.75 $\pm$ 1.16 \\
         & ETS & 98.17 $\pm$ 0.35 & -2.00 $\pm$ 0.42 & 97.69 $\pm$ 0.58 & -2.05 $\pm$ 0.71 & 94.74 $\pm$ 1.05 & -1.94 $\pm$ 1.17 \\
         & Heuristic & 94.57 $\pm$ 1.71 & -6.08 $\pm$ 2.09 & $^{*}$72.49 $\pm$ 19.32 & -20.88 $\pm$ 11.46 & $^{*}$77.88 $\pm$ 12.58 & -12.66 $\pm$ 4.17 \\
         & Ours & {99.02 $\pm$ 0.10} & { -0.91 $\pm$ 0.13} & 98.33 $\pm$ 0.51 & -1.26 $\pm$ 0.63 & 95.02 $\pm$ 0.33 & -0.97 $\pm$ 0.81 \\
        \midrule
        \multirow{4}{*}{DER++} & Random & 97.90 $\pm$ 0.52 & -2.32 $\pm$ 0.67 & 97.10 $\pm$ 1.03 & -2.77 $\pm$ 1.29 & 93.29 $\pm$ 1.43 & -3.11 $\pm$ 1.60  \\
         & ETS & 97.98 $\pm$ 0.52 & -2.24 $\pm$ 0.66 & 98.12 $\pm$ 0.40 & -1.59 $\pm$ 0.52 & 94.53 $\pm$ 1.02 & -1.82 $\pm$ 1.02 \\
         & Heuristic & 92.35 $\pm$ 2.42 & -8.83 $\pm$ 2.99 & $^{*}$67.31 $\pm$ 21.20 & -24.86 $\pm$ 16.34 & 93.88 $\pm$ 1.33 & -2.86 $\pm$ 1.49 \\
         & Ours & {98.84 $\pm$ 0.21}  & {-1.14 $\pm$ 0.26}  & 98.38 $\pm$ 0.43 & -1.17 $\pm$ 0.51 & 94.73 $\pm$ 0.20 & -1.21 $\pm$ 1.12 \\
        \bottomrule
        \toprule
         & & \multicolumn{2}{c}{{\bf Permuted MNIST} } & \multicolumn{2}{c}{{\bf Split CIFAR-100} } & \multicolumn{2}{c}{{\bf Split miniImagenet} } \\
         \cmidrule{3-4} \cmidrule{5-6} \cmidrule{7-8} %\cline{3-8}
         {\bf Method} & {\bf Schedule} & ACC(\%)$\uparrow$ & BWT(\%)$\uparrow$ & ACC(\%)$\uparrow$ & BWT(\%)$\uparrow$ & ACC(\%)$\uparrow$ & BWT(\%)$\uparrow$ \\
        \midrule 
        \multirow{4}{*}{HAL} & Random & 88.49 $\pm$ 0.99 & -7.03 $\pm$ 1.05 & 36.09 $\pm$ 1.77 & -17.49 $\pm$ 1.78  & 38.51 $\pm$ 2.22 & -6.65 $\pm$ 1.43 \\
         & ETS & 88.46 $\pm$ 0.86 & -7.26 $\pm$ 0.90 & 34.90 $\pm$ 2.02 & -18.92 $\pm$ 0.91  & 38.13 $\pm$ 1.18 & -8.19 $\pm$ 1.73 \\
         & Heuristic & $^{*}$66.63 $\pm$ 28.50 & -29.68 $\pm$ 27.90 & 35.07 $\pm$ 1.29 & -24.76 $\pm$ 2.41 & 39.51 $\pm$ 1.49 & -5.65 $\pm$ 0.77 \\
         & Ours & 89.14 $\pm$ 0.74 & -6.29 $\pm$ 0.74 & 40.22 $\pm$ 1.57 & -12.77 $\pm$ 1.30  & 41.39 $\pm$ 1.15 & -3.69 $\pm$ 1.86\\
        \midrule 
        \multirow{4}{*}{MER} & Random & 75.90 $\pm$ 1.34  & -21.69 $\pm$ 1.47 & 42.96 $\pm$ 1.70 & -34.01 $\pm$ 2.07  & 31.48 $\pm$ 1.65 & -6.99 $\pm$ 1.27 \\
         & ETS & 73.01 $\pm$ 0.96 & -25.19 $\pm$ 1.10 & 43.38 $\pm$ 1.81 & -34.84 $\pm$ 1.98  & 33.58 $\pm$ 1.53 & -6.80 $\pm$ 1.46 \\
         & Heuristic & 83.86 $\pm$ 3.19 & -12.48 $\pm$ 3.60 & 40.90 $\pm$ 1.70 & -44.10 $\pm$ 2.03 & 34.22 $\pm$ 1.93 & -7.57 $\pm$ 1.63 \\
         & Ours & 79.72 $\pm$ 0.71 & -17.42 $\pm$ 0.78 & 44.29 $\pm$ 0.69 & -32.73 $\pm$ 0.88  & 32.74 $\pm$ 1.29 & -5.77 $\pm$ 1.04 \\
        \midrule 
        \multirow{4}{*}{DER} & Random & 87.51 $\pm$ 1.10 & -8.81 $\pm$ 1.28 & 56.83 $\pm$ 0.76 & -27.34 $\pm$ 0.63 & 42.19  $\pm$ 0.67 & -10.60  $\pm$ 1.28 \\
         & ETS & 85.71 $\pm$ 0.75 & -11.15 $\pm$ 0.87 & 52.58 $\pm$ 1.49 & -32.93 $\pm$ 2.04 & 35.50  $\pm$ 2.84 & -10.94  $\pm$ 2.21 \\
         & Heuristic & 81.56 $\pm$ 2.28 & -15.06 $\pm$ 2.51 & 55.75 $\pm$ 1.08 & -31.27 $\pm$ 1.02 & 43.62 $\pm$ 0.88 & -8.18 $\pm$ 1.16 \\
         & Ours & 90.11 $\pm$ 0.18 & -5.89 $\pm$ 0.23 & 58.99 $\pm$ 0.98 & -24.95 $\pm$ 0.64 & 43.46  $\pm$ 0.95 & -9.32  $\pm$ 1.37 \\
        \midrule
        \multirow{4}{*}{DER++} & Random & 87.89 $\pm$ 1.10& -8.35 $\pm$ 1.33  & 58.49  $\pm$ 1.44 & -25.48  $\pm$ 1.55  & 48.40  $\pm$ 0.69 & -4.20  $\pm$ 0.86 \\
         & ETS & 85.25 $\pm$ 0.88  & -11.60 $\pm$ 1.03  & 52.54  $\pm$ 1.06 & -33.22  $\pm$ 1.51 & 41.36  $\pm$ 2.90 & -4.07  $\pm$ 2.28 \\
         & Heuristic & 79.17 $\pm$ 2.44 & -17.68 $\pm$ 2.68 & 56.70 $\pm$ 1.27 & -30.33 $\pm$ 1.41 & 45.73 $\pm$ 0.84 & -6.09 $\pm$ 1.24 \\
         & Ours & 89.84 $\pm$ 0.22 & -6.13 $\pm$ 0.29 & 59.23  $\pm$ 0.83 & -24.61  $\pm$ 0.91  & 49.45  $\pm$ 0.68 & -3.12  $\pm$ 0.89 \\
        \bottomrule
    \end{tabular}
    }
    \vspace{-3mm}
    \label{tab:bwt_sota_models_applied_to_rsmcts}
\end{table}
\end{comment}



\begin{table}[t]
    %\footnotesize
    \centering
    \caption{
    Performance comparison with ACC and BWT metrics for all datasets between RS-MCTS and the baselines in the setting where only 1 sample per class can be replayed. The memory sizes are set to $M=10$ and $M=100$ for the 5-task and 10/20-task datasets respectively. We report the mean and standard deviation of ACC and BWT, where all results have been averaged over 5 seeds. RS-MCTS performs on par with the best baselines for both metrics on all datasets except S-CIFAR-100.
    }
    \vspace{-3mm}
    \resizebox{0.95\textwidth}{!}{
    \begin{tabular}{l c c c c c c}
        \toprule
         & \multicolumn{2}{c}{{\bf Split MNIST} } & \multicolumn{2}{c}{{\bf Split FashionMNIST} } & \multicolumn{2}{c}{{\bf Split notMNIST} } \\
         \cmidrule(lr){2-3} \cmidrule(lr){4-5} \cmidrule(lr){6-7} %\cline{2-7}
         {\bf Method} & ACC(\%)$\uparrow$ & BWT(\%)$\uparrow$ & ACC(\%)$\uparrow$ & BWT(\%)$\uparrow$ & ACC(\%)$\uparrow$ & BWT(\%)$\uparrow$ \\
        \midrule 
        Random & 92.56 ($\pm$ 2.90) & -8.97 ($\pm$ 3.62) & 92.70 ($\pm$ 3.78) & -8.24 ($\pm$ 4.75) & 89.53 ($\pm$ 3.96) & -8.13 ($\pm$ 5.02) \\
        A-GEM & 94.97 ($\pm$ 1.50) & -6.03 ($\pm$ 1.87) & 94.81 ($\pm$ 0.86) & -5.65 ($\pm$ 1.06) & 92.27 ($\pm$ 1.16) & -4.17 ($\pm$ 1.39) \\
        ER-Ring & 94.94 ($\pm$ 1.56) & -6.07 ($\pm$ 1.92) & 95.83 ($\pm$ 2.15) & -4.38 ($\pm$ 2.59) & 91.10 ($\pm$ 1.89) & -6.27 ($\pm$ 2.35) \\ 
        Uniform & 95.77 ($\pm$ 1.12) & -5.02 ($\pm$ 1.39) & 97.12 ($\pm$ 1.57) & -2.79 ($\pm$ 1.98) & 92.14 ($\pm$ 1.45) & -4.90 ($\pm$ 1.41) \\ 
        \midrule 
        RS-MCTS (Ours) & 96.07 ($\pm$ 1.60) & -4.59 ($\pm$ 2.01) & 97.17 ($\pm$ 0.78) & -2.64 ($\pm$ 0.99) & 93.41 ($\pm$ 1.11) & -3.36 ($\pm$ 1.56)  \\ 
        \midrule 
        \midrule 
        %\bottomrule
        %\toprule
         & \multicolumn{2}{c}{{\bf Permuted MNIST} } & \multicolumn{2}{c}{{\bf Split CIFAR-100} } & \multicolumn{2}{c}{{\bf Split miniImagenet} } \\
         \cmidrule(lr){2-3} \cmidrule(lr){4-5} \cmidrule(lr){6-7} %\cline{2-7}
         {\bf Method} & ACC(\%)$\uparrow$ & BWT(\%)$\uparrow$ & ACC(\%)$\uparrow$ & BWT(\%)$\uparrow$ & ACC(\%)$\uparrow$ & BWT(\%)$\uparrow$ \\
        \midrule 
        Random & 70.02 ($\pm$ 1.76) & -28.22 ($\pm$ 1.92) & 48.62 ($\pm$ 1.02) & -39.95 ($\pm$ 1.10) & 48.85 ($\pm$ 1.38) & -14.55 ($\pm$ 1.86) \\
        A-GEM & 64.71 ($\pm$ 1.78) & -34.41 ($\pm$ 2.05) & 42.22 ($\pm$ 2.13) & -46.90 ($\pm$ 2.21) & 32.06 ($\pm$ 1.83) & -30.81 ($\pm$ 1.79) \\
        ER-Ring & 69.73 ($\pm$ 1.13) & -28.87 ($\pm$ 1.29) & 53.93 ($\pm$ 1.13) & -34.91 ($\pm$ 1.18) & 49.82 ($\pm$ 1.69) & -14.38 ($\pm$ 1.57) \\ 
        Uniform & 69.85 ($\pm$ 1.01) & -28.74 ($\pm$ 1.17) & 52.63 ($\pm$ 1.62) & -36.43 ($\pm$ 1.81) &  50.56 ($\pm$ 1.07) & -13.52 ($\pm$ 1.34) \\ 
        \midrule 
        RS-MCTS (Ours) & 72.52 ($\pm$ 0.54) & -25.43 ($\pm$ 0.65) & 51.50 ($\pm$ 1.19) & -37.01 ($\pm$ 1.08) & 50.70 ($\pm$ 0.54) & -12.60 ($\pm$ 1.13)  \\ 
        \bottomrule
    \end{tabular}
    }
    \vspace{-2mm}
    \label{tab:bwt_efficiency_of_replay_scheduling}
\end{table}

\begin{comment}

\begin{table}[h]
    \footnotesize
    \centering
    \caption{
    Performance comparison with ACC and BWT metrics for all datasets between RS-MCTS and the baselines in the setting where only 1 sample per class can be replayed. The memory sizes are set to $M=10$ and $M=100$ for the 5-task and 10/20-task datasets respectively. We report the mean and standard deviation of ACC and BWT, where all results have been averaged over 5 seeds. RS-MCTS performs on par with the best baselines for both metrics on all datasets except S-CIFAR-100.
    }
    %\vspace{-3mm}
    \begin{tabular}{l c c c c c c}
        \toprule
         & \multicolumn{2}{c}{{\bf Split MNIST} } & \multicolumn{2}{c}{{\bf Split FashionMNIST} } & \multicolumn{2}{c}{{\bf Split notMNIST} } \\
         \cmidrule(lr){2-3} \cmidrule(lr){4-5} \cmidrule(lr){6-7} %\cline{2-7}
         {\bf Method} & ACC(\%)$\uparrow$ & BWT(\%)$\uparrow$ & ACC(\%)$\uparrow$ & BWT(\%)$\uparrow$ & ACC(\%)$\uparrow$ & BWT(\%)$\uparrow$ \\
        \midrule 
        Random & 92.56 $\pm$ 2.90 & -8.97 $\pm$ 3.62 & 92.70 $\pm$ 3.78 & -8.24 $\pm$ 4.75 & 89.53 $\pm$ 3.96 & -8.13 $\pm$ 5.02  \\
        A-GEM & 94.97 $\pm$ 1.50 & -6.03 $\pm$ 1.87 & 94.81 $\pm$ 0.86 & -5.65 $\pm$ 1.06 & 92.27 $\pm$ 1.16 & -4.17 $\pm$ 1.39  \\
        ER-Ring & 94.94 $\pm$ 1.56 & -6.07 $\pm$ 1.92 & 95.83 $\pm$ 2.15 & -4.38 $\pm$ 2.59 & 91.10 $\pm$ 1.89 & -6.27 $\pm$ 2.35 \\ 
        Uniform & 95.77 $\pm$ 1.12 & -5.02 $\pm$ 1.39 & 97.12 $\pm$ 1.57 & -2.79 $\pm$ 1.98 & 92.14 $\pm$ 1.45 & -4.90 $\pm$ 1.41 \\ 
        \midrule 
        RS-MCTS (Ours) & 96.07 $\pm$ 1.60 & -4.59 $\pm$ 2.01  & 97.17 $\pm$ 0.78 & -2.64 $\pm$ 0.99 & 93.41 $\pm$ 1.11 & -3.36 $\pm$ 1.56  \\ 
        \bottomrule
        \toprule
         & \multicolumn{2}{c}{{\bf Permuted MNIST} } & \multicolumn{2}{c}{{\bf Split CIFAR-100} } & \multicolumn{2}{c}{{\bf Split miniImagenet} } \\
         \cmidrule(lr){2-3} \cmidrule(lr){4-5} \cmidrule(lr){6-7} %\cline{2-7}
         {\bf Method} & ACC(\%)$\uparrow$ & BWT(\%)$\uparrow$ & ACC(\%)$\uparrow$ & BWT(\%)$\uparrow$ & ACC(\%)$\uparrow$ & BWT(\%)$\uparrow$ \\
        \midrule 
        Random & 70.02 $\pm$ 1.76 & -28.22 $\pm$ 1.92 & 48.62 $\pm$ 1.02 & -39.95 $\pm$ 1.10 & 48.85 $\pm$ 1.38 & -14.55 $\pm$ 1.86  \\
        A-GEM & 64.71 $\pm$ 1.78 & -34.41 $\pm$ 2.05 & 42.22 $\pm$ 2.13 & -46.90 $\pm$ 2.21 & 32.06 $\pm$ 1.83 & -30.81 $\pm$ 1.79  \\
        ER-Ring & 69.73 $\pm$ 1.13 & -28.87 $\pm$ 1.29 & 53.93 $\pm$ 1.13 & -34.91 $\pm$ 1.18 & 49.82 $\pm$ 1.69 & -14.38 $\pm$ 1.57 \\ 
        Uniform & 69.85 $\pm$ 1.01 & -28.74 $\pm$ 1.17 & 52.63 $\pm$ 1.62 & -36.43 $\pm$ 1.81 &  50.56 $\pm$ 1.07 & -13.52 $\pm$ 1.34 \\ 
        \midrule 
        RS-MCTS (Ours) & 72.52 $\pm$ 0.54 & -25.43 $\pm$ 0.65 & 51.50 $\pm$ 1.19 & -37.01 $\pm$ 1.08 & 50.70 $\pm$ 0.54 & -12.60 $\pm$ 1.13  \\ 
        \bottomrule
    \end{tabular}
    \vspace{-3mm}
    \label{tab:bwt_efficiency_of_replay_scheduling}
\end{table}
\end{comment}





%\clearpage
%\clearpage
\end{comment}
