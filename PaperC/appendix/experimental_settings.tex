
\section{Experimental Settings}\label{paperC:app:experimental_settings}

In this section, we describe the full details of the experimental settings used in this paper. 

\vspace{-3mm}
\paragraph{Datasets.}
We conduct experiments on six datasets commonly used in the continual learning literature. Split MNIST~\citeC{C:zenke2017continual} is a variant of the MNIST~\citeC{C:lecun1998gradient} dataset where the classes have been divided into 5 tasks incoming in the order 0/1, 2/3, 4/5, 6/7, and 8/9. Split Fashion-MNIST~\citeC{C:xiao2017fashion} is of similar size to MNIST and consists of grayscale images of different clothes, where the classes have been divided into the 5 tasks T-shirt/Trouser, Pullover/Dress, Coat/Sandals, Shirt/Sneaker, and Bag/Ankle boots. Similar to MNIST, Split notMNIST~\citeC{C:bulatov2011notMNIST} consists of 10 classes of the letters A-J with various fonts, where the classes are divided into the 5 tasks A/B, C/D, E/F, G/H, and I/J. We use training/test split provided by \citeC{C:ebrahimi2020adversarial} for Split notMNIST. Permuted MNIST~\citeC{C:goodfellow2013empirical} dataset consists of applying a unique random permutation of the pixels of the images in original MNIST to create each task, except for the first task that is to learn the original MNIST dataset. We reduce the original MNIST dataset to 10k samples and create 9 unique random permutations to get a 10-task version of Permuted MNIST. In Split CIFAR-100~\citeC{C:krizhevsky2009learning}, the 100 classes are divided into 20 tasks with 5 classes for each task~\citeC{C:lopez2017gradient, C:rebuffi2017icarl}. Similarly, Split miniImagenet~\citeC{C:vinyals2016matching} consists of 100 classes randomly chosen from the original Imagenet dataset where the 100 classes are divided into 20 tasks with 5 classes per task.

\vspace{-3mm}
\paragraph{Network Architectures.} We use a 2-layer MLP with 256 hidden units and ReLU activation for Split MNIST, Split FashionMNIST, Split notMNIST, and Permuted MNIST. We use a multi-head output layer for each dataset except Permuted MNIST where the network uses single-head output layer. For Split CIFAR-100, we use a multi-head CNN architecture built according to the CNN in \citeC{C:adel2019continual, C:schwarz2018progress, C:vinyals2016matching}, which consists of four 3x3 convolutional blocks, i.e. convolutional layer followed by batch normalization~\citeC{C:ioffe2015batch}, with 64 filters, ReLU activations, and 2x2 Max-pooling. For Split mniImagenet, we use the reduced ResNet-18 from \citeC{C:lopez2017gradient} with multi-head output layer. 

\vspace{-3mm}
\paragraph{Hyperparameters.} We train all networks with the Adam optimizer~\citeC{C:kingma2014adam} with learning rate $\eta = 0.001$ and hyperparameters $\beta_1 = 0.9$ and $\beta_2 = 0.999$. Note that the learning rate for Adam is not reset before training on a new task. Next, we give details on number of training epochs and batch sizes specific for each dataset:
\begin{itemize}[topsep=1pt,noitemsep]
    \item Split MNIST: 10 epochs/task, batch size 128.
    \item Split FashionMNIST: 30 epochs/task, batch size 128.
    \item Split notMNIST: 50 epochs/task, batch size 128.
    \item Permuted MNIST: 20 epochs/task, batch size 128.
    \item Split CIFAR-100: 25 epochs/task, batch size 256.
    \item Split miniImagenet: 1 epoch/task (task 1 trained for 5 epochs as warm up), batch size 32.
\end{itemize}

\vspace{-3mm}
\paragraph{Monte Carlo Tree Search.} We run MCTS for 100 iterations in all experiments. The replay schedules used in the reported results on the held-out test sets are from the replay schedule that gave the highest reward on the validation sets. The exploration constant for UCT in Equation \ref{eq:uct} is set to $C=0.1$ in all experiments~\citeC{C:chaudhry2018feature}.

\vspace{-3mm}
\paragraph{Computational Cost.} All experiments were performed on one NVIDIA GeForce RTW 2080Ti. The wall clock time for ETS on Split MNIST was around 1.5 minutes, and MCTS and BFS takes 40 seconds on average to run one iteration, where BFS runs 1050 iterations in total for Split MNIST. 


\vspace{-3mm}
\paragraph{Implementations.} We adapted the implementation released by Borsos \etal~\citeC{C:borsos2020coresets} for the memory selection strategies Uniform sampling, $k$-means clustering, $k$-center clustering~\citeC{C:nguyen2017variational}, and Mean-of-Features~\citeC{C:rebuffi2017icarl}. For HAL~\citeC{C:chaudhry2021using}, MER~\citeC{C:riemer2018learning}, DER~\citeC{C:buzzega2020dark}, and DER++, we follow the implementations released by \citeC{C:buzzega2020dark} for each method to apply them to our replay scheduling methods. Furthermore, we follow the implementations released in \citeC{C:chaudhry2019tiny} and \citeC{C:mirzadeh2021cl-gym} for A-GEM~\citeC{C:chaudhry2018efficient} and ER-Ring~\citeC{C:chaudhry2019tiny}. For MCTS, we adapted the implementation from {\footnotesize \url{https://github.com/int8/monte-carlo-tree-search}} to search for replay schedules.

\vspace{-3mm}
\paragraph{Experimental Settings for Single Task Replay Memory Experiment.} We motivated the need for replay scheduling in continual learning with Figure \ref{fig:single_task_replay_with_M10} in Section \ref{paperC:sec:introduction}. This simple experiment was performed on Split MNIST where the replay memory only contains samples from the first task, i.e., learning the classes 0/1. Furthermore, the memory can only be replayed at one point in time and we show the performance on each task when the memory is replayed at different time steps. We set the memory size to $M=10$ samples such that the memory holds 5 samples from both classes. We use the same network architecture and hyperparameters as described above for Split MNIST. The ACC metric above each subfigure corresponds to the ACC for training a network with the single task memory replay at different tasks. We observe that choosing different time points to replay the same memory leads to noticeably different results in the final performance, and in this example, the best final performance is achieved when the memory is used when learning task 5. Therefore, we argue that finding the proper schedule of what tasks to replay at what time in the fixed memory situation can be critical for continual learning. 

\subsection{Heuristic Scheduling Baseline}\label{paperC:app:heuristic_scheduling_baseline}

We implemented a heuristic scheduling baseline to compare against MCTS. The baseline keeps a validation set for the old tasks and replays the tasks which validation accuracy is below a certain threshold. We set the threshold in the following way: Let $A_{t, i}$ be the validation accuracy for task $t$ evaluated at time step $i$. The best evaluated validation accuracy for task $t$ at time $i$ is given by $A_{t, i}^{(best)} = \max(\{A_{t, 1}, ..., A_{t, i} \})$. The condition for replaying task $t$ on the next time step is then $A_{t, i} < \tau A_{t, i}^{(best)}$, where $\tau \in [0, 1]$ is a ratio controlling how much the current accuracy on task $t$ is allowed to decrease w.r.t. the best accuracy. The replay memory is filled with $M/k$, where $k$ is the number of tasks that need to be replayed according to their decrease in validation accuracy. This heuristic scheduling corresponds to the intuition of re-learning when a task has been forgotten. Training on the current task is performed without replay if the accuracy on all old tasks is above their corresponding threshold.       

\vspace{-3mm}
\paragraph{Grid search for $\tau$.} We performed a coarse-to-fine grid search for the ratio $\tau$ on each dataset. The best value for $\tau$ is selected according to the highest mean accuracy on the validation set averaged over 5 seeds. The validation set consists of 15\% of the training data and is the same for MCTS. We use the same experimental settings as described in Appendix \ref{paperC:app:experimental_settings}. The memory sizes are set to $M=10$ and $M=100$ for the 5-task datasets and the 10/20-task datasets respectively, and we apply uniform sampling as the memory selection method. We provide the ranges for $\tau$ that was used on each dataset and put the best value in \textbf{bold}:
\begin{itemize}[topsep=1pt,noitemsep]
	\item Split MNIST: $\tau =$ [0.9, 0.93, 0.95, \textbf{0.96}, 0.97, 0.98, 0.99]
	\item Split FashionMNIST: $\tau =$ [0.9, 0.93, 0.95, 0.96, \textbf{0.97}, 0.98, 0.99]
	\item Split notMNIST: $\tau =$ [0.9, 0.93, 0.95, 0.96, 0.97, \textbf{0.98}, 0.99]
	\item Permuted MNIST: $\tau =$ [0.5, 0.55, 0.6, 0.65, 0.7, \textbf{0.75}, 0.8, 0.9, 0.95, 0.97, 0.99]
	\item Split CIFAR-100: $\tau =$ [0.3, 0.4, 0.45, \textbf{0.5}, 0.55, 0.6, 0.65, 0.7, 0.8, 0.9, 0.95, 0.97, 0.99]
	\item  Split miniImagenet: $\tau =$ [0.5, 0.6, 0.65, 0.7, \textbf{0.75}, 0.8, 0.85, 0.9, 0.95, 0.97, 0.99]
\end{itemize}
Note that we use these values for $\tau$ on all experiments with Heuristic for the corresponding datasets. The performance for this heuristic highly depends on careful tuning for the ratio $\tau$ when the memory size or memory selection method changes, as can be seen in in Figure \ref{fig:acc_over_replay_memory_size} and Table \ref{tab:results_memory_selection_methods}. We also provide the ranges for $\tau$ that was used on each dataset in the Class Incremental Learning setting and put the best value in \textbf{bold}:
\begin{itemize}[topsep=1pt,noitemsep]
	\item Split MNIST: $\tau =$ \{0.2, 0.3, 0.5, \textbf{0.75}, 0.9\}
	\item Split FashionMNIST: $\tau =$ \{0.2, 0.3, \textbf{0.5},  0.75, 0.9\}
	\item Split notMNIST:  $\tau =$ \{0.2, 0.3, \textbf{0.5},  0.75, 0.9\}
	\item Split CIFAR-100: $\tau =$ \{\textbf{0.01}, 0.025, 0.05, 0.1, 0.25, 0.5\}
	\item  Split miniImagenet: $\tau =$ \{\textbf{0.01}, 0.025, 0.05, 0.1, 0.25, 0.5\}
\end{itemize}



\subsection{Hyperparameters for Recent Replay Methods}
\label{paperC:app:apply_scheduling_to_recent_replay_methods}

In Section \ref{paperC:sec:applying_scheduling_to_recent_replay_methods}, we showed that MCTS can be applied to any replay method. We combined MCTS together with four recent replay methods, namely Hindsight Anchor Learning (HAL)~\citeC{C:chaudhry2021using}, Meta Experience Replay (MER)~\citeC{C:riemer2018learning}, and Dark Experience Replay (DER)~\citeC{C:buzzega2020dark}. Table \ref{tab:results_applying_scheduling_to_recent_replay_methods} shows the ACC and BWT for all methods combined with the scheduling from Random, ETS, Heuristic, and MCTS. We observe that MCTS can further improve the performance for each of the replay methods across the different datasets.  

We present the hyperparameters used for each method in Table \ref{tab:hyperparameters_applying_scheduling_to_recent_replay_methods}. The hyperparameters for each method are denoted as
\begin{itemize}[topsep=0pt, noitemsep]
	\item {\bf HAL.} $\eta$: learning rate, $\lambda$: regularization, $\gamma$: mean embedding strength, $\beta$: decay rate, $k$: gradient steps on anchors   
	\item {\bf MER.} $\gamma$: across batch meta-learning rate, $\beta$: within batch meta-learning rate 
	\item {\bf DER.} $\alpha$: loss coefficient for memory logits 
	\item {\bf DER++.} $\alpha$: loss coefficient for memory logits, $\beta$: loss coefficient for memory labels
\end{itemize}
We %For the experiments, we 
used the same architectures and hyperparameters as described in Appendix \ref{paperC:app:experimental_settings} for all datasets, except for the optimizer in HAL where we used the SGD optimizer since using Adam made the model diverge in our experiments. We used the Adam optimizer with learning rate $\eta=0.001$ for MER, DER, and DER++. 


\begin{table}[t]
	%\scriptsize
	\centering
	\caption{
		Hyperparameters for replay-based methods HAL, MER, DER and DER++ used in experiments on applying MCTS to recent replay-based methods in Section \ref{paperC:sec:applying_scheduling_to_recent_replay_methods}.
	}
	\vspace{-3mm}
	\resizebox{0.95\textwidth}{!}{
		
\begin{tabular}{l c c c c c c c}
	\toprule
	& & \multicolumn{3}{c}{{\bf 5-task Datasets}} & \multicolumn{3}{c}{{\bf 10- and 20-task Datasets}} \\
	\cmidrule(lr){3-5} \cmidrule(lr){6-8}
	{\bf Method} & {\bf Hyperparam.}  & S-MNIST & S-FashionMNIST & S-notMNIST & P-MNIST & S-CIFAR-100 & S-miniImagenet \\
	\midrule
	\multirow{5}{*}{HAL} & $\eta$ & 0.1 & 0.1 & 0.1 & 0.1 & 0.03 & 0.03 \\
	& $\lambda$  & 0.1 & 0.1 & 0.1 & 0.1 & 1.0 & 0.03 \\
	& $\gamma$ & 0.5 & 0.1 & 0.1 & 0.1 & 0.1 & 0.1 \\
	& $\beta$ & 0.7 & 0.5 & 0.5 & 0.5 & 0.5 & 0.5 \\
	& $k$ & 100 & 100 & 100 & 100 & 100 & 100 \\
	\midrule 
	\multirow{2}{*}{MER} & $\gamma$ & 1.0 & 1.0 & 1.0 & 1.0 & 1.0 & 1.0  \\ 
	& $\beta$  & 1.0 & 0.01 & 1.0 & 1.0 & 0.1 & 0.1 \\
	\midrule 
	\multirow{1}{*}{DER} & $\alpha$ & 0.2 & 0.2 & 0.1 & 1.0 & 1.0 & 0.1 \\ 
	\midrule 
	\multirow{2}{*}{DER++} & $\alpha$ & 0.2 & 0.2 & 0.1 & 1.0 & 1.0 & 0.1 \\ 
	& $\beta$ & 1.0 & 1.0 & 1.0 & 1.0 & 1.0 & 1.0 \\
	\bottomrule
\end{tabular}
	}
	\vspace{-2mm}
	\label{tab:hyperparameters_applying_scheduling_to_recent_replay_methods}
\end{table}



\subsection{Memory Usage in Efficiency of Replay Scheduling Experiment}\label{paperC:app:additional_figures}

We visualize the memory usage for Permuted MNIST and the 20-task datasets Split CIFAR-100 and Split miniImagenet in Figure \ref{fig:memory_usage_10_and_20task_datasets} for our method and the baselines used in the experiment in Section \ref{paperC:sec:efficiency_of_replay_scheduling}. Our method uses a fixed memory size of $M=50$ samples for replay on all three datasets. The memory size capacity for our method is reached after learning task 6 and task 11 on the Permuted MNIST and the 20-task datasets respectively, while the baselines continue incrementing their replay memory size.

\begin{figure}[h]
	\centering
	\vspace{-1mm}
	\setlength{\figwidth}{0.75\textwidth}
	\setlength{\figheight}{.20\textheight}
	



\pgfplotsset{compat=1.11,
    /pgfplots/ybar legend/.style={
    /pgfplots/legend image code/.code={%
       \draw[##1,/tikz/.cd,yshift=-0.25em]
        (0cm,0cm) rectangle (3pt,0.8em);},
   },
}
\pgfplotsset{every axis title/.append style={at={(0.5,0.90)}}}
\pgfplotsset{every tick label/.append style={font=\tiny}}
\pgfplotsset{every major tick/.append style={major tick length=2pt}}
\pgfplotsset{every minor tick/.append style={minor tick length=1pt}}
\pgfplotsset{every axis x label/.append style={at={(0.5,-0.14)}}}
\pgfplotsset{every axis y label/.append style={at={(-0.10,0.5)}}}

\begin{tikzpicture}
\tikzstyle{every node}=[font=\footnotesize]
\definecolor{color0}{rgb}{0.12156862745098,0.466666666666667,0.705882352941177}
\definecolor{color1}{rgb}{1,0.498039215686275,0.0549019607843137}
%\definecolor{color2}{rgb}{0.172549019607843,0.627450980392157,0.172549019607843}

\begin{groupplot}[group style={group size= 1 by 2, horizontal sep=0.9cm, vertical sep=1.2cm}]

\nextgroupplot[title={Memory Usage for 10-task Permuted MNIST },
ybar,
bar width = 4pt,
height=\figheight,
legend cell align={left},
legend columns=2,
legend style={
  fill opacity=0.8,
  draw opacity=1,
  text opacity=1,
  at={(0.03,0.82)},
  anchor=south west,
  draw=white!80!black
},
minor xtick={},
minor ytick={},
tick align=outside,
tick pos=left,
width=\figwidth,
x grid style={white!69.0196078431373!black},
xlabel={Task},
%x label style={at={(axis description cs:0.5,-0.3)},anchor=north},
xmin=1.5, xmax=10.5,
xtick style={color=black},
xtick={2,3,4,5,6,7,8,9,10},
xticklabels={2,3,4,5,6,7,8,9,10},
y grid style={white!69.0196078431373!black},
ymajorgrids,
ylabel={\#Replayed Samples},
ymin=0.0, ymax=102,
ytick style={color=black},
ytick={0,10,20,30,40,50,60,70,80,90,100},
yticklabels={0,10,20,30,40,50,60,70,80,90,100}
]

\addplot [draw=black, fill=color0] coordinates {
(2,10)
(3,20)
(4,30)
(5,40)
(6,50)
(7,50)
(8,50)
(9,50)
(10,50)
};\addlegendentry{Ours}

\addplot [draw=black, fill=color1] coordinates {
(2,10)
(3,20)
(4,30)
(5,40)
(6,50)
(7,60)
(8,70)
(9,80)
(10,90)
};\addlegendentry{Baselines}

\nextgroupplot[title={Memory Usage for the 20-task Datasets},
ybar,
bar width = 4pt,
height=\figheight,
legend cell align={left},
legend columns=2,
legend style={
  fill opacity=0.8,
  draw opacity=1,
  text opacity=1,
  at={(0.03,0.75)},
  anchor=south west,
  draw=white!80!black
},
minor xtick={},
minor ytick={},
tick align=outside,
tick pos=left,
width=\figwidth,
x grid style={white!69.0196078431373!black},
xlabel={Task},
%x label style={at={(axis description cs:0.5,-0.3)},anchor=north},
xmin=1.5, xmax=20.5,
xtick style={color=black},
xtick={2,3,4,5,6,7,8,9,10,11,12,13,14,15,16,17,18,19,20},
xticklabels={2,3,4,5,6,7,8,9,10,11,12,13,14,15,16,17,18,19,20},
y grid style={white!69.0196078431373!black},
ymajorgrids,
ylabel={\#Replayed Samples},
ymin=0.0, ymax=102,
ytick style={color=black},
ytick={0,10,20,30,40,50,60,70,80,90,100},
yticklabels={0,10,20,30,40,50,60,70,80,90,100}
]

\addplot [draw=black, fill=color0] coordinates {
(2,5)
(3,10)
(4,15)
(5,20)
(6,25)
(7,30)
(8,35)
(9,40)
(10,45)
(11,50)
(12,50)
(13,50)
(14,50)
(15,50)
(16,50)
(17,50)
(18,50)
(19,50)
(20,50)
};\addlegendentry{Ours}

\addplot [draw=black, fill=color1] coordinates {
(2,5)
(3,10)
(4,15)
(5,20)
(6,25)
(7,30)
(8,35)
(9,40)
(10,45)
(11,50)
(12,55)
(13,60)
(14,65)
(15,70)
(16,75)
(17,80)
(18,85)
(19,90)
(20,95)
};\addlegendentry{Baselines}

\end{groupplot}

\end{tikzpicture}
	\vspace{-3mm}
	\caption{Number of replayed samples per task for 10-task Permuted MNIST (top) and the 20-task datasets in the experiment in Section \ref{paperC:sec:efficiency_of_replay_scheduling}. The fixed memory size $M=50$ for our method is reached after learning task 6 and task 11 on the Permuted MNIST and the 20-task datasets respectively, while the baselines continue incrementing their number of replay samples per task.}
	\label{fig:memory_usage_10_and_20task_datasets}
\end{figure}
