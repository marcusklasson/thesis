
\begin{figure*}[t]
  \centering
  \setlength{\figwidth}{0.32\textwidth}
  \setlength{\figheight}{0.25\textwidth}
  \input{appendix/figures/mnist_alternative_selection_methods/best_rewards_rs_mcts_selection_methods_M24_groupplot}
  \caption{Best rewards measured in ACC for ETS and RS-MCTS using memory size $M=24$ on Split MNIST for different memory selection strategies Random Selection, Mean-of-Features, and K-center Coreset. Results are averaged over 5 seeds. 
  }%\vspace{-12pt}
  \label{fig:MNIST_mcts_best_rewards_M24_selection_methods_appendix}
\end{figure*}

\begin{figure*}[t]
  \centering
  \setlength{\figwidth}{0.6\textwidth}
  \setlength{\figheight}{0.4\textwidth}
  \input{appendix/figures/mnist_alternative_selection_methods/MNIST_accuracy_bwt_curveplot}
  \caption{ACC over memory size $M$ for ETS and RS-MCTS on Split MNIST using different memory selection methods, namely Random Selection (Random), Mean-of-Features (MoF), and K-center Coreset. All results have been averaged over 5 seeds. 
  }\vspace{-12pt}
  \label{fig:MNIST_ACC_over_M_selection_methods_appendix}
\end{figure*}

\section{Additional Experiments}\label{app:additional_experiments}

This section includes additional experiments with RS-MCTS using alternative memory selection strategies for selecting historical data as well as applying RS-MCTS in the class incremental learning (Class-IL) scenario where task information is inaccessible. 

\subsection{Alternative Memory Selection Strategies}
%\CZ{This is more important than class-IL.}
We evaluate RS-MCTS on two alternative selection strategies, namely Mean-of-Features (MoF)~\cite{rebuffi2017icarl} and K-center Coreset~\cite{nguyen2017variational}, and compare these to using random selection of examples to store as historical data. We begin by giving a brief description of both strategies: \vspace{-12pt}

\paragraph{Mean-of-Features:} We extract feature representations before the classification layer of all images for every class. Then the examples to store in memory are selected based on the similarity between the feature representations and their moving mean value. This selection strategy was used in iCaRL~\cite{rebuffi2017icarl} where they grabbed such examples to create an exemplar set for performing classification in feature space. Similar to Chaudhry \etal~\cite{chaudhry2019tiny}, we use this strategy for selecting memory examples that should be close to the mode of the class in feature space.\vspace{-12pt}

\paragraph{K-center Coreset:} We employ the greedy K-center algorithm~\cite{gonzalez1985clustering} that was used for memory replay in Variational Continual Learning~\cite{nguyen2017variational}. In contrast to MoF, this strategy operates in input space and returns examples that are spread throughout the input space.\vspace{12pt}

We perform experiments on Split MNIST to compare MoF and K-center Coreset to random selection as selection strategies. Figure \ref{fig:MNIST_mcts_best_rewards_M24_selection_methods_appendix} shows the progress in ACC when searching for replay schedules with RS-MCTS using the three strategies and memory size $M=24$. We also show the performance from using ETS with each strategy as baselines. All three strategies find better replay schedules than ETS using less than 50 simulations. Both MoF and K-center Coreset strategies perform on par with random selection over the 500 simulations with RS-MCTS. 

In Figure \ref{fig:MNIST_ACC_over_M_selection_methods_appendix}, we compare the performance using the selection strategies for various memory sizes. The schedules from RS-MCTS for the selection strategies perform on par as the memory size $M$ increases where K-center Coreset strategy gains slightly superior performance at $M=120$ and $M=200$. A potential reason for this slight performance increase could be that K-center Coreset selects memory examples to store that are spread out in the input space for each class rather than selecting examples that are near the mode for all examples as in MoF.   


\subsection{Class Incremental Learning}

In the first experiment, we apply replay scheduling to the class incremental learning (Class-IL) setting where task information is unavailable at test time. This scenario requires the model to both infer the task and class of the test input. This continual learning scenario is considered to be more challenging than the task incremental learning (Task-IL) setting where task labels are available at test time that we address in the main paper. We assess RS-MCTS for Class-IL on the Split MNIST dataset. The experimental settings are the same as in the main paper, except that the network architecture uses a single-head classification layer in this scenario. 

We need to adjust the way to construct the action space of memory combinations to apply RS-MCTS to the Class-IL setting. As earlier, we create $t-1$ bins $[b_1, b_2, ....b_{t-1}]$ and choose a historical task to sample for each bin $b_i \in {1,.., t-1}$ before learning task $t$. In the Class-IL setting, the model will catastrophically forget tasks that are excluded from the selected memory combination. Therefore, we ensure that the memory combinations includes some proportion of samples from all previous tasks by extending the memory with $t-1$ additional bins $[c_1, c_2, ..., c_{t-1}]$ where each $c_i$ includes samples from its corresponding task $i$. For example, at Task 3, we have Task 1 and 2 in history; so the unique choices of bins are $[1,1], [1,2], [2,2]$. We then extend each of these unique choices with $[1, 2]$, such that the unique choices become $[1,1,1,2], [1,2,1,2], [2,2,1,2]$, where $[1,1,1,2]$ indicates that 75\% and 25\% of the memory is from Task 1 and Task 2 respectively, and $[1,2,1,2]$ indicates that half memory is from Task 1 and the other half are from Task 2 etc.

In the first experiment, we inspect the performance progress measured in ACC over MCTS simulations when $M=24$ for Split MNIST. Figure \ref{fig:MNIST_mcts_best_rewards_M24_class_il_appendix} shows that RS-MCTS quickly finds a significantly better ACC than ETS using less than 50 iterations. Furthermore, RS-MCTS can reach similar performance as from the breadth-first search (BFS) with significantly less computational budget. These results confirm that scheduling of which memory examples to replay is important in the Class-IL scenario. 

In Figure \ref{fig:MNIST_ACC_BWT_over_M_class_il_appendix}, we show that the performance of RS-MCTS improves on both ACC and BWT across  different  memory sizes $M$. We observe that RS-MCTS replay schedules yield better final classification performance compared to ETS, especially for small $M$ in the Class-IL scenario. The performance for both RS-MCTS and ETS schedules increases rapidly as the memory size becomes larger which confirms the importance of the memory size in Class-IL. The results from using the ETS schedules approach the ACC and BWT metrics achieved by RS-MCTS as the memory size grows. This indicates that replay scheduling is necessary for situations with limited memory budgets in the Class-IL scenario. 

\begin{figure}[h]
  \centering
  \setlength{\figwidth}{0.9\columnwidth}
  \setlength{\figheight}{0.3\textwidth}
  \input{appendix/figures/mnist_class_incremental_learning/best_rewards_rs_mcts_with_upper_bound_M24}
  \caption{MCTS simulation performance on Split MNIST in the Class Incremental Learning setting where the average test classification accuracies over tasks after training on the final task (ACC) is used as the reward. The ETS performance is shown in blue as a baseline and the green line show the ACC from the optimal schedules found from a breadth-first search (BFS) as an upper bound. We use $M= 24$ and all results have been averaged over 5 seeds.
  }\vspace{-12pt}
  \label{fig:MNIST_mcts_best_rewards_M24_class_il_appendix}
\end{figure}

\begin{figure}[h]
  \centering
  \setlength{\figwidth}{0.85\columnwidth}
  \setlength{\figheight}{0.3\textwidth}
  \input{appendix/figures/mnist_class_incremental_learning/MNIST_accuracy_bwt_curveplot}
  \caption{Average test classification accuracies over tasks after training on the final task (ACC) and backward transfer (BWT) over different memory sizes $M$ for Split MNIST in the Class Incremental Learning setting. We show accuracies obtained by using an equal proportion of examples from previous tasks (ETS) as well as the result from the best replay schedules found with MCTS (RS-MCTS). All results have been averaged over 5 seeds.
  }\vspace{-12pt}
  \label{fig:MNIST_ACC_BWT_over_M_class_il_appendix}
\end{figure}


\begin{figure}[t]
\centering
\setlength{\figwidth}{0.95\columnwidth}
\setlength{\figheight}{.3\textheight}

\begin{tikzpicture}
\tikzstyle{every node}=[font=\small]
\definecolor{color0}{rgb}{0.12156862745098,0.466666666666667,0.705882352941177}
\definecolor{color1}{rgb}{1,0.498039215686275,0.0549019607843137}
\definecolor{color2}{rgb}{0.172549019607843,0.627450980392157,0.172549019607843}
\definecolor{color3}{rgb}{0.83921568627451,0.152941176470588,0.156862745098039}
\definecolor{color4}{rgb}{0.580392156862745,0.403921568627451,0.741176470588235}

\begin{axis}[title={ETS for 5-Task Dataset},
ybar,
bar width = 4pt,
height=\figheight,
legend cell align={left},
legend columns=2,
legend style={
  fill opacity=0.99,
  draw opacity=1,
  text opacity=1,
  at={(0.67,1.13)},
  anchor=north west,
  draw=white!80!black
},
minor xtick={},
minor ytick={},
tick align=outside,
tick pos=left,
width=\figwidth,
x grid style={white!69.0196078431373!black},
xlabel={Task Number},
xmin=1.5, xmax=5.5,
xtick style={color=black},
xtick={2,3,4,5},
y grid style={white!69.0196078431373!black},
ymajorgrids,
ylabel={Proportion (\%)},
ymin=0.0, ymax=1.1,
ytick style={color=black},
ytick={0,0.25, 0.5, 0.75, 1, 1.2},
yticklabels={0, 25, 50, 75, 100, 120}
]
\addplot [draw=black, fill=color0] coordinates {
(2,1.0)
(3,0.5)
(4,0.333)
(5,0.25)
};\addlegendentry{Task 1}

\addplot [draw=black, fill=color1] coordinates {
(3,0.5)
(4,0.333)
(5,0.25)
};\addlegendentry{Task 2}

\addplot [draw=black, fill=color2] coordinates {
(4,0.333)
(5,0.25)
};\addlegendentry{Task 3}

\addplot [draw=black, fill=color3] coordinates {
(5,0.25)
};\addlegendentry{Task 4}
\end{axis}

\end{tikzpicture}

\caption{testing proportion barplot in tikz }
\label{fig:testing_proportion_barplot}
\end{figure}