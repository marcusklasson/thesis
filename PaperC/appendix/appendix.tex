
%%%%%%%%% BODY TEXT

\section*{Appendix}
This supplementary material is structured as follows: 
\begin{itemize}
    \item Appendix \ref{app:experimental_settings}: Full details of the experimental settings.
    \item Appendix \ref{app:rs_mcts_algorithm}: Pseudocode of RS-MCTS in Algorithm \ref{alg:replay_scheduling_mcts} to provide more details about our method. 
    \item Appendix \ref{app:heuristic_scheduling_baseline} describes the Heuristic scheduling baseline.
    \item Appendix \ref{app:additional_experimental_results} shows additional experimental results. 
    \item Appendix \ref{app:additional_figures} includes additional figures. 
    \item Our code is provided in the zip-file {\footnotesize \texttt{code\_replay\_scheduling\_mcts\_icml22.zip}} 
    as part of the supplementary material and will be made publicly available upon acceptance. 
\end{itemize}


%------------------------------------------------------------------------

\section{Experimental Settings}\label{paperD:app:experimental_settings}

In this section, we describe the full details of the experimental settings used in this paper. We first provide details on hyperparameter settings for the CL and RL parts, including hyperparameters for DQN~\citeD{D:mnih2013playing, D:mnih2015human} and A2C~\citeD{D:mnih2016asynchronous}, followed by description of the heuristic baselines, and the ranking method for assesing the generalization capability of the learned policy.  



\vspace{-2mm}
\paragraph{Datasets.}
We generate CL environments with four datasets commonly used in the CL literature. 
Split MNIST~\citeD{D:zenke2017continual} is a variant of the MNIST~\citeD{D:lecun1998gradient} dataset where the classes have been divided into 5 tasks incoming in the order 0/1, 2/3, 4/5, 6/7, and 8/9. Split FashionMNIST~\citeD{D:xiao2017fashion} is of similar size to MNIST and consists of grayscale images of different clothes, where the classes have been divided into the 5 tasks T-shirt/Trouser, Pullover/Dress, Coat/Sandals, Shirt/Sneaker, and Bag/Ankle boots. Similar to MNIST, Split notMNIST~\citeD{D:bulatov2011notMNIST} consists of 10 classes of the letters A-J with various fonts, where the classes are divided into the 5 tasks A/B, C/D, E/F, G/H, and I/J. We use training/test split provided by \citeD{D:ebrahimi2020adversarial} for Split notMNIST. 
In Split CIFAR-10~\citeD{D:krizhevsky2009learning}, the 10 classes are divided into 5 tasks with 2 classes for each task. 

\vspace{-2mm}
\paragraph{CL Network Architectures.} We use a 2-layer MLP with 256 hidden units and ReLU activation for Split MNIST, Split FashionMNIST, and Split notMNIST. For Split CIFAR-10, we use the ConvNet architecture used in \citeD{D:adel2019continual, D:schwarz2018progress, D:vinyals2016matching}, which consists of four 3x3 convolutional blocks, i.e. convolutional layer followed by batch normalization~\citeD{D:ioffe2015batch}, with 64 filters, ReLU activations, and 2x2 Max-pooling. We use a multi-head output layer for each dataset and assume task labels are available at test time for selecting the correct output head related to the task. 

\vspace{-2mm}
\paragraph{CL Hyperparameters.} We train all networks with the Adam optimizer~\citeD{D:kingma2014adam} with learning rate $\eta = 0.001$ and hyperparameters $\beta_1 = 0.9$ and $\beta_2 = 0.999$. Note that the learning rate for Adam is not reset before training on a new task. Next, we give details on number of training epochs and batch sizes specific for each dataset:
\begin{itemize}[topsep=1pt]
    \item Split MNIST: 10 epochs/task, batch size 128.
    \item Split FashionMNIST: 10 epochs/task, batch size 128.
    \item Split notMNIST: 20 epochs/task, batch size 128.
    \item Split CIFAR-100: 20 epochs/task, batch size 256.
\end{itemize}

\vspace{-2mm}
\paragraph{Generating CL Environments.} We generate multiple CL environments with pre-set random seeds for initializing the network parameters $\vphi$ and shuffling the task order. The pre-set random seeds are in the range $0-49$, such that we have 50 environments for each dataset. We shuffle the task order by permuting the class order and then split the classes into 5 pairs (tasks) with 2 classes/pair. For environments with seed $0$, we keep the original task order in the dataset. 
Taking a step at task $t$ in the CL environments involves training the CL network on the $t$-th dataset with a replay memory $\gM_t$ from the discrete action space described in Section \ref{paperD:app:action_space}. Therefore, to speed up the experiments with the RL algorithms, we run a breadth-first search (BFS) through the discrete action space and save the classification results for re-use during policy learning. Note that the action space has 1050 possible paths of replay schedules for the datasets with $T=5$ tasks, which makes the environment generation time-consuming. Hence, we only generate environments where the replay memory size $M=10$ have been used, and leave analysis of different memory sizes as future work. The CL environments that we used will be provided in the public code. 

\vspace{-2mm}
\paragraph{Computational Cost.} 
All experiments were performed on one NVIDIA GeForce RTW 2080Ti on an internal GPU cluster. Generating a CL environment for one seed with Split MNIST took on around 9.5 hours averaged over 10 runs of BFS. Similarly for Split CIFAR-10, generating one CL environment took on average 16.1 hours. Note that BFS runs 1050 iterations in total for all 5-task datasets. The wall clock time for ETS on Split MNIST was around 1.5 minutes. 


\vspace{-2mm}
\paragraph{DQN and A2C Architectures.}
The input layer has size $T-1$ where each unit is inputting the task performances since the states are represented by the validation accuracies $s_t = [A_{t, 1}^{(val)}, ..., A_{t, t}^{(val)}, 0, ..., 0]$. The current task can therefore be determined by the number of non-zero state inputs. The output layer has 35 units representing the possible actions at $T=5$ with the discrete action space we have constructed in Section \ref{paperD:sec:replay_scheduling_in_continual_learning}. We use action masking on the output units to prevent the network from selection invalid actions for constructing the replay memory at the current task. The DQN is a 2-layer MLP with 512 hidden units and ReLU activations. For A2C, we use separate networks for parameterizing the policy and the value function, where both networks are 2-layer MLPs with 64 hidden units of Tanh activations. 

\vspace{-2mm}
\paragraph{DQN and A2C Hyperparameters.}
We provide the hyperparameters for the both DQN and A2C in Table \ref{tab:dqn_hyperparameters_new_task_orders}-\ref{tab:a2c_hyperparameters_new_dataset}. Table \ref{tab:dqn_hyperparameters_new_task_orders} and \ref{tab:a2c_hyperparameters_new_task_orders} includes the hyperparameters on the New Task Order experiment for DQN and A2C respectively, while Table \ref{tab:dqn_hyperparameters_new_dataset} and \ref{tab:a2c_hyperparameters_new_dataset} includes the hyperparameters on the New Dataset experiment for DQN and A2C respectively. Regarding the training environments in Table \ref{tab:dqn_hyperparameters_new_dataset} and \ref{tab:a2c_hyperparameters_new_dataset}, we use two different datasets in the training environments to increase the diversity. When Spit notMNIST is for testing, half the amount of training environments are using Split MNIST and the other half uses Split FashionMNIST. For example, in Table \ref{tab:a2c_hyperparameters_new_dataset}, A2C uses 10 training environments which means that there are 5 Split MNIST environments and 5 Split FashionMNIST environments. Similarly, half the amount of training environments are using Split MNIST and the other half uses Split notMNIST when the testing environments uses Split FashionMNIST.   

\vspace{-2mm}
\paragraph{Implementations.} 
The code for DQN was adapted from OpenAI baselines~\citeD{D:dhariwal2017baselines} and the PyTorch~\citeD{D:paszke2019pytorch} tutorial on DQN {\small \url{https://pytorch.org/tutorials/intermediate/reinforcement_q_learning.html}}. For A2C, we followed the implementations released by Kostrikov~\citeD{D:kostrikov2018pytorchrl} and Igl \etal~\citeD{D:igl2020transient}. We make the code publicly available upon acceptance. 

%\clearpage

\begin{table}[h]
\small
\centering
\caption{DQN hyperparameters for the experiments on {\bf New Task Orders} in Section \ref{paperD:sec:results_on_policy_generalization}.  
}
\label{tab:dqn_hyperparameters_new_task_orders}
\resizebox{0.99\textwidth}{!}{
\begin{tabular}{l c c c}
\toprule
 {\bf Hyperparameters} & {\bf Split MNIST} & {\bf Split FashionMNIST} & {\bf Split CIFAR-10} \\
 \midrule
 Training Environments & 30 & 20 & 10 \\
 Learning Rate  & 0.0001 & 0.0003 & 0.0003 \\
 Optimizer      & Adam & Adam & Adam \\
 Buffer Size    & 10k & 10k & 10k  \\
 Target Update per step  & 500 & 500 & 500 \\
 Batch Size     & 32 & 32 & 32 \\
 Discount Factor $\gamma$ & 1.0 & 1.0 & 1.0 \\
 Exploration Start $\epsilon_{start}$ & 1.0 & 1.0 & 1.0 \\
 Exploration Final $\epsilon_{final}$ & 0.02 & 0.02 & 0.02 \\
 Exploration Annealing (episodes) & 2.5k & 2.5k & 2.5k \\
 Training Episodes & 10k & 10k & 10k \\
\bottomrule
\end{tabular}
}
\end{table}

\begin{table}[h]
\small
\centering
\caption{A2C hyperparameters for the experiments on {\bf New Task Orders} in Section \ref{paperD:sec:results_on_policy_generalization}.  
}
\label{tab:a2c_hyperparameters_new_task_orders}
\begin{tabular}{l c c c}
\toprule
 {\bf Hyperparameters} & {\bf Split MNIST} & {\bf Split FashionMNIST} & {\bf Split CIFAR-10} \\
 \midrule
 Training Environments & 10 & 10 & 10 \\
 Learning Rate  & 0.0001 & 0.0003 & 0.00003 \\
 Optimizer      & RMSProp & RMSProp & RMSProp \\
 Gradient Clipping & 0.5 & 0.5 & 0.5 \\
 GAE parameter $\lambda$ & 0.95 & 0.95 & 0.95 \\
 VF coefficient & 0.5 & 0.5 & 0.5 \\
 Entropy coefficient & 0.01 & 0.01 & 0.01 \\
 Number of steps $n_{steps}$ & 5 & 5 & 5 \\
 Discount Factor $\gamma$ & 1.0 & 1.0 & 1.0 \\
 Training Episodes & 100k & 100k & 100k \\
\bottomrule
\end{tabular}
\end{table}

\begin{table}[h]
\small
\centering
\caption{DQN hyperparameters for the experiments on {\bf New Dataset} in Section \ref{paperD:sec:results_on_policy_generalization}. Split notMNIST and Split FashionMNIST indicate the dataset used in the test environments. 
}
\label{tab:dqn_hyperparameters_new_dataset}
\begin{tabular}{l c c}
\toprule
 {\bf Hyperparameters} & {\bf Split notMNIST} & {\bf Split FashionMNIST}  \\
 \midrule
 Training Environments & 30 & 30  \\
 Learning Rate  & 0.0001 & 0.0001  \\
 Optimizer      & Adam & Adam  \\
 Buffer Size    & 10k & 10k   \\
 Target Update per step  & 500 & 500  \\
 Batch Size     & 32 & 32  \\
 Discount Factor $\gamma$ & 1.0 & 1.0  \\
 Exploration Start $\epsilon_{start}$ & 1.0 & 1.0 \\
 Exploration Final $\epsilon_{final}$ & 0.02 & 0.02  \\
 Exploration Annealing (episodes) & 2.5k & 2.5k  \\
 Training Episodes & 10k & 10k  \\
\bottomrule
\end{tabular}
\end{table}

\begin{table}[h]
\small
\centering
\caption{A2C hyperparameters for the experiments on {\bf New Dataset} in Section \ref{paperD:sec:results_on_policy_generalization}. Split notMNIST and Split FashionMNIST indicate the dataset used in the test environments. 
}
\label{tab:a2c_hyperparameters_new_dataset}
\begin{tabular}{l c c c}
\toprule
 {\bf Hyperparameters} & {\bf Split notMNIST} & {\bf Split FashionMNIST}  \\
 \midrule
 Training Environments & 10 & 10  \\
 Learning Rate  & 0.0001 & 0.0003  \\
 Optimizer      & RMSProp & RMSProp  \\
 Gradient Clipping & 0.5 & 0.5  \\
 GAE parameter $\lambda$ & 0.95 & 0.95  \\
 VF coefficient & 0.5 & 0.5 \\
 Entropy coefficient & 0.01 & 0.01  \\
 Number of steps $n_{steps}$ & 5 & 5  \\
 Discount Factor $\gamma$ & 1.0 & 1.0  \\
 Training Episodes & 100k & 100k  \\
\bottomrule
\end{tabular}
\end{table}


%\clearpage

\subsection*{Heuristic Scheduling Baselines}\label{paperD:app:heuristic_scheduling_baselines}

We implemented three heuristic scheduling baselines to compare against our proposed methods. These heuristics are based on the intuition of re-learning tasks when they have been forgotten. We keep a validation set for each task to determine whether any task should be replayed by comparing the validation accuracy against a hand-tuned threshold. If the validation accuracy is below the threshold, then the corresponding task is replayed. Let $A_{t, i}^{(val)}$ be the validation accuracy for task $t$ evaluated at time step $i$. The threshold is set differently in each of the baselines:
\begin{itemize}[leftmargin=*, topsep=0pt]
    \item {\bf Heuristic Global Drop (Heur-GD).} Heuristic policy that replays tasks  
    with validation accuracy below a certain threshold proportional to the best achieved validation accuracy on the task. The best achieved validation accuracy for task $i$ is given by $A_{t, i}^{(best)} = \max\{(A_{1, i}^{(val)}, \dots, A_{t, i}^{(val)})\}$. Task $i$ is replayed if $A_{t, i}^{(val)} < \tau A_{t, i}^{(best)}$ where $\tau \in [0, 1]$ is a ratio representing the degree of how much the validation accuracy of a task is allowed to drop. 

    \item {\bf Heuristic Local Drop (Heur-LD).} Heuristic policy that replays tasks with validation accuracy below a threshold proportional to the previous achieved validation accuracy on the task. Task $i$ is replayed if $A_{t, i}^{(val)} < \tau A_{t-1, i}^{(val)}$ where $tau$ again represents the degree of how much the validation accuracy of a task is allowed to drop. 

    \item {\bf Heuristic Accuracy Threshold (Heur-AT).} Heuristic policy that replays tasks with validation accuracy below a fixed threshold. Task $i$ is replayed if if $A_{t, i}^{(val)} < \tau$ where $\tau \in [0, 1]$ represents the least tolerated accuracy before we need to replay the task. 
    %The third heuristic has a fixed threshold $\tau \in [0, 1]$ on the accuracy such that task $i$ is replayed after learning task $t$ if $A_{t, i} < \tau$.
\end{itemize}
The replay memory is filled with $M/k$ samples from each selected task, where $k$ is the number of tasks that need to be replayed according to their decrease in validation accuracy. We skip replaying any tasks if no tasks are selected for replay, i.e., $k=0$. 


\vspace{-1mm}
\paragraph{Grid search for $\tau$.}
We performed a grid search for the parameter $\tau$ for the three heuristic scheduling baselines for each experiment to compare against the learned replay scheduling policies. The validation set consists of 15\% of the training data and is identical to the validation set used for learning the RL policies. 
We select the parameter based on ACC scores achieved in the same number of training environments used by either DQN or A2C. The search range we use is $\tau \in \{0.90, 0.95, 0.999\}$. In Table \ref{tab:grid_search_heuristics_multiple_cl_environments}, we show the selected parameter value of $\tau$ and the number of environments used for selecting the value for each method and experiment in Section \ref{sec:results_on_policy_generalization}. The same parameters are used to generate the results on the heuristics in Table \ref{tab:average_ranking_rl_experiment}. 

\begin{table}[t]
%\footnotesize%\small
\centering
\caption{The threshold parameter $\tau$ used in the heuristic scheudling baselines Heuristic Global Drop (Heur-GD), Heuristic Local Drop (Heur-LD), and Heuristic Accuracy Threshold (Heur-AT). The search range is $\tau \in \{0.90, 0.95, 0.999\}$ for all methods and we display the number of environments used for selecting the parameter used at test time.  
}
\label{tab:grid_search_heuristics_multiple_cl_environments}
\resizebox{\textwidth}{!}{
\begin{tabular}{l c c c c c c c c c c}
\toprule %\toprule
 & \multicolumn{6}{c}{ {\bf New Task Order}} & \multicolumn{4}{c}{ {\bf New Dataset}}\\
 \cmidrule(lr){2-7}  \cmidrule(lr){8-11}
 %{\bf Method} & S-MNIST & S-FashionMNIST & S-CIFAR-10 & S-notMNIST & S-FashionMNIST  \\
  & \multicolumn{2}{c}{ {\bf S-MNIST}} & \multicolumn{2}{c}{ {\bf S-FashionMNIST}} & \multicolumn{2}{c}{ {\bf S-CIFAR-10}} & \multicolumn{2}{c}{ {\bf S-notMNIST}} & \multicolumn{2}{c}{ {\bf S-FashionMNIST}}   \\
 \cmidrule(lr){2-3} \cmidrule(lr){4-5} \cmidrule(lr){6-7} \cmidrule(lr){8-9} \cmidrule(lr){10-11} 
 {\bf Method} & $\tau$ & \#Envs & $\tau$ & \#Envs & $\tau$ & \#Envs & $\tau$ & \#Envs & $\tau$ &  \#Envs \\
 \midrule
    Heur-GD & 0.9 & 10 & 0.95  & 20 & 0.9   & 10 & 0.9  & 10 & 0.9   & 10 \\ 
    Heur-LD & 0.9 & 10 & 0.999 & 20 & 0.999 & 10 & 0.95 & 10 & 0.999 & 10 \\ 
    Heur-AT & 0.9 & 10 & 0.999 & 20 & 0.9   & 10 & 0.9 & 10 & 0.95  & 10 \\
    %Heur-GD        & \{0.90, 0.95, 0.999\} & \{0.90, 0.95, 0.999\} & \{0.90, 0.95, 0.999\} & \{0.90, 0.95, 0.999\} & \{0.90, 0.95, 0.999\} \\
    %Heur-LD        & \{0.90, 0.95, 0.999\} & \{0.90, 0.95, 0.999\} & \{0.90, 0.95, 0.999\} & \{0.90, 0.95, 0.999\} & \{0.90, 0.95, 0.999\} \\
    %Heur-AT        & \{0.90, 0.95, 0.999\} & \{0.90, 0.95, 0.999\} & \{0.90, 0.95, 0.999\} & \{0.90, 0.95, 0.999\} & \{0.90, 0.95, 0.999\} \\
\bottomrule %\bottomrule
\end{tabular}
} 
%\vspace{-3mm}
\end{table}





\subsection*{Assessing Generalization with Ranking Method}\label{paperD:app:ranking_method}

We use a ranking method based on the CL performance in every test environment for performance comparison between the methods in Section \ref{paperD:sec:results_on_policy_generalization}. We use rankings because the performances can vary greatly between environments with different task orders and datasets. To measure the CL performance in the environments, we use the average test accuracy over all tasks after learning the final task, i.e.,
\begin{align*}
    \ACC = \frac{1}{T} \sum_{i=1}^{T} A_{T, i}^{(test)},
\end{align*}
where $A_{t, i}^{(test)}$ is the test accuracy of task $i$ after learning task $t$. Each method are ranked in descending order based on the ACC achieved in an environment. For example, assume that we want to compare the CL performance from using learned replay scheduling policies with DQN and A2C against a Random scheduling policy in one environment. The CL performances achieved for each method are given by
\vspace{2mm}
\begin{align*}
    [\ACC_{\Random}, \ACC_{\DQN}, \ACC_{\text{A2C}}] = [90\%, 99\%, 95\%].
\end{align*}

We get the following ranking order between the methods based on their corresponding ACC:
\vspace{2mm}
\begin{align*}
    \texttt{ranking}([\ACC_{\Random}, \ACC_{\DQN}, \ACC_{\text{A2C}}]) = [3, 1, 2],
\end{align*} 

where DQN is ranked in 1st place, A2C in 2nd, and Random in 3rd. When there are multiple environments for evaluation, we compute the average ranking across the ranking positions in every environment for each method to compare.

The average ranking for DQN and A2C are computed over the seed for initializing the network parameters as well as the seed of the environment. Similarly, the Random baseline is affected by the seed setting the random selection of actions and the environment seed. However, the performance of the ETS and Heuristic baselines are affected by the seed of the environment as these policies are fixed. We use copied values of the performance in environments for the ETS and Heuristic baselines when we need to compare across different random seeds for Random, DQN, and A2C. We show an example of such ranking calculation for ETS, a Heuristic baseline, DQN, and A2C. Consider the following performances for one environment:
\vspace{2mm}
\begin{align*}
\begin{bmatrix}
\ACC_{\ETS}^{1} & \ACC_{\Heur}^{1} & \ACC_{\DQN}^{1} & \ACC_{\text{A2C}}^{1} \\[3pt] \ACC_{\ETS}^{2}  & \ACC_{\Heur}^{2} & \ACC_{\DQN}^{2} & \ACC_{\text{A2C}}^{2} 
\end{bmatrix}  
=
\begin{bmatrix}
90\% & 95\% & 95\% & 99\% \\[3pt]
 * & * & 97\% & 98\%  
\end{bmatrix}  ,
\end{align*}

where $*$ denotes a copy of the ACC  value in the first row. The subscript on ACC denotes the method and the superscript the seed used for initializing the policy network $\vtheta$. Therefore, we copy the values for ETS and Heur such that the $\ACC_{\DQN}^2$ for seed 2 can be compared against ETS and Heur. Note that there is a tie between $\ACC_{\Heur}^{1}$ and $\ACC_{\DQN}^{1}$ as they have ACC $95\%$. We handle ties by assigning tied methods the average of their ranks, such that the ranks for both seeds will be
\vspace{2mm}
\begin{align*}
&
\texttt{ranking}
\left(
\begin{bmatrix}
\ACC_{\ETS}^{1} & \ACC_{\Heur}^{1} & \ACC_{\DQN}^{1} & \ACC_{\text{A2C}}^{1} \\[3pt] \ACC_{\ETS}^{2}  & \ACC_{\Heur}^{2} & \ACC_{\DQN}^{2} & \ACC_{\text{A2C}}^{2} 
\end{bmatrix}, \texttt{axis=-1, keepdim=True}  \right) \\[3pt]
= & 
\texttt{ranking}
\left(
\begin{bmatrix}
90\% & 95\% & 95\% & 99\% \\[3pt]
90\% & 95\% & 97\% & 98\%  
\end{bmatrix}, \texttt{axis=-1, keepdim=True}  \right) \\[3pt]
= & 
\begin{bmatrix}
4 & 2.5 & 2.5 & 1 \\[3pt] 
4 & 3 & 2 & 1 
\end{bmatrix},
\end{align*}

where we inserted the copied values, such that $\ACC_{\ETS}^{1} = \ACC_{\ETS}^{2} = 90\%$ and $\ACC_{\Heur}^{1} = \ACC_{\Heur}^{2} = 95\%$. The mean ranking across the seeds thus becomes
\vspace{2mm}
\begin{align*}
\texttt{mean}
\left(
\begin{bmatrix}
4 & 2.5 & 2.5 & 1 \\[3pt] 
4 & 3 & 2 & 1 
\end{bmatrix}, \texttt{axis=0} \right)
= 
\begin{bmatrix} 4 & 2.75 & 2.25 & 1  \end{bmatrix} 
\end{align*}

where A2C comes in 1st place, DQN in 2nd, Heur. in 3rd, and ETS on 4th place. We average across seeds and environments to obtain the final ranking score for each method for comparison. 



%-------------------------------------------------------------------------


\section[Replay Scheduling MCTS Algorithm]{Replay Scheduling Monte Carlo Tree Search Algorithm}\label{paperC:app:rs_mcts_algorithm}

%%% MCTS ALGORITHMS


In this section, we provide more details on the methodology for replay scheduling with MCTS. Algorithm \ref{paperC:alg:action_space_discretization} outlines the steps for how we discretized the action space of task proportions to enable searching for replay schedules (Section \ref{paperC:sec:replay_scheduling_in_continual_learning}). Furthermore, we provide pseudo-code in Algorithm \ref{alg:replay_scheduling_mcts} outlining the steps for our method Replay Scheduling Monte Carlo tree search (RS-MCTS) described Section \ref{paperC:sec:mcts_for_replay_scheduling}. %in the main paper (Section \ref{paperC:sec:mcts_for_replay_scheduling}). 

In Algorithm \ref{alg:replay_scheduling_mcts}, the MCTS procedure selects actions over which task proportions to fill the replay memory with at every task, where the selected task proportions are stored in the replay schedule $S$. The schedule is then passed to 
the function \textsc{EvaluateReplaySchedule$(\cdot)$} 
where the continual learning part executes the training with replay memories filled according to the schedule. The reward for the schedule $S$ is the average validation accuracy over all tasks after learning task $T$, i.e., ACC, which is backpropagated through the tree to update the statistics of the selected nodes. The schedule $S_{best}$ yielding the best ACC score is returned to be used for evaluation on the held-out test sets. The function $\textsc{GetReplayMemory}(\cdot)$ is the policy for retrieving the replay memory $\gM$ from the historical data given the task proportion $\va$. The number of samples per task determined by the task proportions are rounded up or down accordingly to fill $\gM$ with $M$ replay samples in total. 
The function $\textsc{GetTaskProportion}(\cdot)$ simply returns the task proportion that is related to given node.



\begin{algorithm}[h!]
	\small
	\caption{Discretization of action space with task proportions}
	\label{paperC:alg:action_space_discretization}
	\begin{algorithmic}[1]
		\Require Number of tasks $T$
		\State $\gT = ()$ \Comment{Initialize sequence for storing actions}
		\For{$i = 1, \dots, T-1$}
		\State $\gP_i = \{\}$ \Comment{Set for storing task proportions at $i$}
		\State $\gB = \texttt{combinations}([1:i], i)$ \Comment{Get bin vectors of size $i$ with bins $1, ..., i$}
		\State $\bar{\gB} = \texttt{unique}(\texttt{sort}(\gB))$ \Comment{Only keep unique bin vectors}
		\For{$\vb_i \in \hat{\gB}$}
		\State $\va_i = \texttt{bincount}(\vb_i) / i$ \Comment{Calculate task proportion}
		\State $\gP_i = \gP_i \cup \{ \va_i \}$ \Comment{Add task proportion to set}
		\EndFor
		\State $\gT[i] = \gP_i$ \Comment{Add set of task proportions to action sequence}
		\EndFor
		\State \Return $\gT$ \Comment{Return action sequence as discrete action space}
	\end{algorithmic}
\end{algorithm}



\begin{algorithm}[h!]
\caption{Replay Scheduling Monte Carlo tree search}
\label{alg:replay_scheduling_mcts}
\algrenewcommand\alglinenumber[1]{\tiny #1:}
\scriptsize
\begin{algorithmic}[1]
\Require Tree nodes $v_{1:T}$, Datasets $\gD_{1:T}$, Learning rate $\eta$
\Require Replay memory size $M$ %, Historical data sample size $H$
%\State Initialize historical data $\gH$ using $\gD_{1:T}$
%\State Initialize model $\vtheta_0$
\State $\textnormal{ACC}_{best} \leftarrow 0$, $S_{best} \leftarrow ()$
%\State $\vtheta_1 \leftarrow \textsc{TrainModel}(\gD_1, \vtheta_0)$
\While{within computational budget}
    \State $S \leftarrow ()$
    \State $v_{t}, S \leftarrow \textsc{TreePolicy}(v_1, S)$%\State $v_{t}, \vtheta_t \leftarrow \textsc{TreePolicy}(v_1, \vtheta_1, \gD_{1:T}, \gH, M)$
    \State $v_{T}, S \leftarrow \textsc{DefaultPolicy}(v_{t}, S)$%\State $v_{t}, \vtheta_T, \textnormal{acc} \leftarrow \textsc{DefaultPolicy}(v_{t}, \vtheta_t, \gD_{1:T}, \gH, M)$
    \State ACC $\leftarrow \textsc{EvaluateReplaySchedule}(\gD_{1:T}, S, M)$
    \State $\textsc{Backpropagate}(v_{t}, \textnormal{ACC})$
    \If{$\textnormal{ACC} > \textnormal{ACC}_{best}$}
        %\State $v_T^{best} \leftarrow v_T$
        %\State $\vtheta_T^{best} \leftarrow \vtheta_{T}$
        \State $\textnormal{ACC}_{best} \leftarrow \textnormal{ACC}$
        \State $S_{best} \leftarrow S$
    \EndIf
\EndWhile
\State \Return $\textnormal{ACC}_{best}, S_{best}$

\Statex 
\vspace{-3pt}
\Function{TreePolicy}{$v_t, S$}
\While{$v_t$ is non-terminal}  
\If{$v_t$ not fully expanded}
    \State \Return $\textsc{Expansion}(v_t, S)$
\Else
    \State $v_{t} \leftarrow \textsc{BestChild}(v_t)$
    %\State $\va_{t} \leftarrow \textsc{GetTaskProportion}(v_t)$
    \State $S$.append$(\va_{t})$, where $\va_{t} \leftarrow \textsc{GetTaskProportion}(v_t)$
\EndIf
\EndWhile
\State \Return $v_t, S$
\EndFunction

\Statex

\Function{Expansion}{$v_t, S$}
\State Sample $v_{t+1}$ uniformly among unvisited children of $v_t$
%\State $\gM \stackrel{M}{\sim} \gH$ using memory combination in $v_{t+1}$
%\State $\vtheta_{t+1} \leftarrow \textsc{TrainModel}(\gD_{t+1} \cup \gM, \vtheta_{t})$
%\State $\va_{t+1} \leftarrow \textsc{GetTaskProportion}(v_{t+1})$
\State $S$.append$(\va_{t+1})$, where $\va_{t+1} \leftarrow \textsc{GetTaskProportion}(v_{t+1})$
\State Add new child $v_{t+1}$ to node $v_{t}$%\State Add new child $v_{t+1}$ with model $\vtheta_{t+1}$ to node $v_{t}$
\State \Return $v_{t+1}, S$
\EndFunction

\Statex 

\Function{BestChild}{$v_t$}
    \State $v_{t+1} =   \argmax\limits_{v_{t+1} \in \text{ children of } v} \text{max}(Q(v_{t+1})) + C \sqrt{\frac{2 \log(N(v_{t}))}{N(v_{t+1})}}$
    %\State Get model $\vtheta_{t+1}$ from node $v_{t+1}$
    \State \Return $v_{t+1}$
\EndFunction

\Statex 

\Function{DefaultPolicy}{$v_t, S$}
\While{$v_t$ is non-terminal}  
    \State Sample $v_{t+1}$ uniformly among children of $v_t$
    %\State $\va_{t+1} \leftarrow \textsc{GetTaskProportion}(v_{t+1})$
    \State $S$.append$(\va_{t+1})$, where $\va_{t+1} \leftarrow \textsc{GetTaskProportion}(v_{t+1})$
    %\State $\gM \stackrel{M}{\sim} \gH$ using memory combination in $v_{t+1}$
    %\State $\vtheta_{t+1} \leftarrow \textsc{TrainModel}(\gD_{t+1} \cup \gM, \vtheta_{t})$
    \State Update $v_{t} \leftarrow v_{t+1}$ %, $\vtheta_{t} \leftarrow \vtheta_{t+1}$
\EndWhile
%\State $\textnormal{acc} \leftarrow \textsc{TestModel}(\gD_{1:T}, \vtheta_{t})$
\State \Return $v_t, S$
\EndFunction

\Statex

\Function{EvaluateReplaySchedule}{$\gD_{1:T}, S, M$}
\State Initialize neural network $f_{\vtheta}$
%\State $\gH \leftarrow \{\}$
\For{$t = 1, \dots, T$}  
    \State $\va \leftarrow S[t-1]$
    \State $\gM \leftarrow \textsc{GetReplayMemory}(M, \va)$
    \For{$ \gB \sim \gD_t^{(train)}$}  
        \State $\vtheta \leftarrow SGD(\gB \cup \gM, \vtheta, \eta)$ 
        %\State $Q(v_t) \leftarrow R$
        %\State $v_t \leftarrow$ parent of $v_t$
    \EndFor
    %\State $\gH \leftarrow \textsc{UpdateHistoricalData}(\gD_t^{(train)}, H)$
\EndFor
\State $A_{1:T}^{(val)} \leftarrow \textsc{EvaluateAccuracy}(f_{\vtheta}, \gD_{1:T}^{(val)})$
\State $\textnormal{ACC} \leftarrow \frac{1}{T} \sum_{i=1}^{T} A_{T,i}^{(val)}$
\State \Return ACC
\EndFunction

\Statex

\Function{Backpropagate}{$v_t, R$}
\While{$v_t$ is not root}  
    \State $N(v_t) \leftarrow N(v_t)+1$
    \State $Q(v_t) \leftarrow R$
    \State $v_t \leftarrow$ parent of $v_t$
\EndWhile
\EndFunction
\end{algorithmic}
\end{algorithm}







%-------------------------------------------------------------------------
%
\section{Additional Experimental Results}\label{app:additional_experimental_results}

\begin{table}[t]
    \scriptsize
    \centering
    \caption{The ACC metrics and ranks from all environments and DQN seeds for test on FashionMNIST experiment. We show the ACC (\%) and the rank in parenthesis ($\cdot$) for each method and seed. Note that we have copied the ACCs for ETS, Heuristic1, and Heuristic2 since they do not have a seed for initializing the DQN or setting the random replay schedule as in Random.  \MK{Use longtable package to split table across pages}}
    \begin{tabular}{l l c c c c c}
        \toprule
        & & \multicolumn{5}{c}{{\bf Methods}} \\
        \cmidrule{3-7}
        {\bf Environment} & {\bf DQN Seed} & Random & ETS & Heuristic1 & Heuristic2 & DQN \\
        \midrule 
         \multirow{5}{*}{Env. 1} & Seed 1 & 96.66 (4) & 94.17 (5) & 98.52 (2) & 98.69 (1) & 97.60 (3)  \\
         \cmidrule{2-7}
         & Seed 2 & 90.16 (5) & 94.17 (4) & 98.52 (2) & 98.69 (1) & 96.83 (3)  \\
         \cmidrule{2-7}
         & Seed 3 & 97.89 (3) & 94.17 (5) & 98.52 (2) & 98.69 (1) & 96.34 (4)  \\
         \cmidrule{2-7}
         & Seed 4 & 97.20 (4) & 94.17 (5) & 98.52 (2) & 98.69 (1) & 97.37 (3) \\
         \cmidrule{2-7}
         & Seed 5 & 97.58 (3) & 94.17 (5) & 98.52 (2) & 98.69 (1) & 96.37 (4)  \\
        \midrule
         \multirow{5}{*}{Env. 2} & Seed 1 & 92.84 (4) & 94.10 (2) & 94.55 (1) & 90.03 (5) & 93.92 (3)  \\
         \cmidrule{2-7}
         & Seed 2 & 95.06 (2) & 94.10 (4) & 94.55 (3) & 90.03 (5) & 95.34 (1)   \\
         \cmidrule{2-7}
         & Seed 3 & 93.72 (4) & 94.10 (3) & 94.55 (2) & 90.03 (5) & 95.34 (1)  \\
         \cmidrule{2-7}
         & Seed 4 & 94.93 (2) & 94.10 (4) & 94.55 (3) & 90.03 (5) & 95.17 (1) \\
         \cmidrule{2-7}
         & Seed 5 & 84.76 (5) & 94.10 (3) & 94.55 (2) & 90.03 (4) & 95.34 (1)  \\
        \midrule
         \multirow{5}{*}{Env. 3} & Seed 1 & 96.07 (2) & 95.95 (4) & 96.06 (3) & 84.75 (5) & 97.12 (1)  \\
         \cmidrule{2-7}
         & Seed 2 & 95.55 (4) & 95.95 (3) & 96.06 (2) & 84.75 (5) & 97.12 (1)  \\
         \cmidrule{2-7}
         & Seed 3 & 95.57 (4) & 95.95 (3) & 96.06 (2) & 84.75 (5) & 96.94 (1)  \\
         \cmidrule{2-7}
         & Seed 4 & 97.04 (1) & 95.95 (3) & 96.06 (2) & 84.75 (5) & 86.20 (4)  \\
         \cmidrule{2-7}
         & Seed 5 & 95.13 (4) & 95.95 (3) & 96.06 (2) & 84.75 (5) & 96.69 (1)  \\
         \midrule 
         \multirow{5}{*}{Env. 4} & Seed 1 & 94.40 (5) & 96.71 (3) & 97.31 (2) & 96.28 (4) & 98.00 (1)   \\
         \cmidrule{2-7}
         & Seed 2 & 98.94 (1) & 96.71 (4) & 97.31 (3) & 96.28 (5) & 97.87 (2)   \\
         \cmidrule{2-7}
         & Seed 3 & 97.05 (3) & 96.71 (4) & 97.31 (2) & 96.28 (5) & 97.96 (1)   \\
         \cmidrule{2-7}
         & Seed 4 & 95.35 (5) & 96.71 (3) & 97.31 (1) & 96.28 (4) & 96.99 (2)  \\
         \cmidrule{2-7}
         & Seed 5 & 95.64 (4) & 96.71 (2) & 97.31 (1) & 96.28 (3) & 94.87 (5)  \\
         \midrule 
         \multirow{5}{*}{Env. 5} & Seed 1 & 83.91 (4) & 81.24 (5) & 94.25 (2) & 93.58 (3) & 94.39 (1) \\
         \cmidrule{2-7}
         & Seed 2 & 88.75 (4) & 81.24 (5) & 94.25 (1) & 93.58 (3) & 94.03 (2)   \\
         \cmidrule{2-7}
         & Seed 3 & 83.54 (4) & 81.24 (5) & 94.25 (1) & 93.58 (3) & 93.60 (2)   \\
         \cmidrule{2-7}
         & Seed 4 & 84.34 (4) & 81.24 (5) & 94.25 (1) & 93.58 (2) & 90.82 (3)  \\
         \cmidrule{2-7}
         & Seed 5 & 88.29 (3) & 81.24 (5) & 94.25 (1) & 93.58 (2) & 87.85 (4)  \\
         \midrule 
         \multirow{5}{*}{Env. 6} & Seed 1 & 82.93 (5) & 87.94 (4) & 89.35 (2) & 89.55 (1) & 89.22 (3)  \\
         \cmidrule{2-7}
         & Seed 2 & 91.19 (1) & 87.94 (5) & 89.35 (4) & 89.55 (3) & 91.05 (2)  \\
         \cmidrule{2-7}
         & Seed 3 & 90.30 (1) & 87.94 (5) & 89.35 (3) & 89.55 (2) & 88.46 (4)   \\
         \cmidrule{2-7}
         & Seed 4 & 89.81 (1) & 87.94 (5) & 89.35 (3) & 89.55 (2) & 89.23 (4)   \\
         \cmidrule{2-7}
         & Seed 5 & 88.02 (4) & 87.94 (5) & 89.35 (3) & 89.55 (2) & 91.48 (1)  \\
         \midrule 
         \multirow{5}{*}{Env. 7} & Seed 1 & 86.13 (5) & 91.10 (3) & 95.16 (2) & 95.56 (1) & 89.18 (4)  \\
         \cmidrule{2-7}
         & Seed 2 & 94.04 (3) & 91.10 (4) & 95.16 (2) & 95.56 (1) & 77.08 (5)\\
         \cmidrule{2-7}
         & Seed 3 & 94.82 (3) & 91.10 (4) & 95.16 (2) & 95.56 (1) & 80.92 (5)   \\
         \cmidrule{2-7}
         & Seed 4 & 94.67 (3) & 91.10 (4) & 95.16 (2) & 95.56 (1) & 84.96 (5)   \\
         \cmidrule{2-7}
         & Seed 5 & 91.70 (4) & 91.10 (5) & 95.16 (2) & 95.56 (1) & 93.36 (3)  \\
         \midrule
         \multirow{5}{*}{Env. 8} & Seed 1 & 90.10 (4) & 90.70 (3) & 94.89 (1) & 91.25 (2) & 89.27 (5)  \\
         \cmidrule{2-7}
         & Seed 2 & 87.13 (5) & 90.70 (4) & 94.89 (1) & 91.25 (3) & 93.11 (2)  \\
         \cmidrule{2-7}
         & Seed 3 & 94.44 (2) & 90.70 (5) & 94.89 (1) & 91.25 (4) & 94.00 (3)    \\
         \cmidrule{2-7}
         & Seed 4 & 94.89 (2) & 90.70 (5) & 94.89 (1) & 91.25 (4) & 91.39 (3)   \\
         \cmidrule{2-7}
         & Seed 5 & 93.74 (2) & 90.70 (4) & 94.89 (1) & 91.25 (3) & 89.48 (5)  \\
         \midrule
         \multirow{5}{*}{Env. 9} & Seed 1 & 98.10 (1) & 97.67 (2) & 97.51 (3) & 96.41 (4) & 88.08 (5) \\
         \cmidrule{2-7}
         & Seed 2 & 94.00 (5) & 97.67 (1) & 97.51 (2) & 96.41 (3) & 94.01 (4)  \\
         \cmidrule{2-7}
         & Seed 3 & 97.93 (1) & 97.67 (2) & 97.51 (3) & 96.41 (4) & 87.87 (5)    \\
         \cmidrule{2-7}
         & Seed 4 & 96.37 (4) & 97.67 (1) & 97.51 (2) & 96.41 (3) & 89.83 (5)   \\
         \cmidrule{2-7}
         & Seed 5 & 97.38 (3) & 97.67 (1) & 97.51 (2) & 96.41 (5) & 97.29 (4)  \\
         \midrule
         \multirow{5}{*}{Env. 10 } & Seed 1 & 96.84 (2) & 73.10 (5) & 91.10 (4) & 93.58 (3) & 97.25 (1) \\
         \cmidrule{2-7}
         & Seed 2 & 88.82 (4) & 73.10 (5) & 91.10 (2) & 93.58 (1) & 89.83 (3)   \\
         \cmidrule{2-7}
         & Seed 3 & 86.83 (4) & 73.10 (5) & 91.10 (3) & 93.58 (2) & 96.00 (1)    \\
         \cmidrule{2-7}
         & Seed 4 & 84.38 (4) & 73.10 (5) & 91.10 (3) & 93.58 (2) & 95.48 (1)   \\
         \cmidrule{2-7}
         & Seed 5 & 90.72 (4) & 73.10 (5) & 91.10 (3) & 93.58 (2) & 96.04 (1)   \\
        \bottomrule
    \end{tabular}
    \label{tab:accuracy_per_environment_fashionmnist}
\end{table}















 
























 











%

\section{Experimental Analysis}\label{app:experimental_analysis}

In this section, we present experimental analysis on learning replay schedules with MCTS. First, we show that our method has greater advantage over the standard replay strategy without scheduling when the memory size is small in Section \ref{app:results_with_varying_memory_size}. In Section \ref{app:task_accuracies_and_replay_schedule_visualization}, we analyze the progress of task accuracies during learning by inspecting the replay schedules and deduce that the learned replay schedules are consistent with human learning processes, such as spaced repetition. Furthermore, we illustrate that RS-MCTS can be used with various sample selection strategies than random selection in Appendix \ref{app:alternative_memory_selection_strategies}, and also be applied to the class incremental learning setting in Appendix \ref{app:class_incremental_learning}.

\subsection{Results with Varying Memory Size $M$}\label{app:results_with_varying_memory_size}

In these experiments, we show that RS-MCTS improves on both ACC and BWT across different memory sizes $M$ for all four datasets with Figure \ref{fig:test_classification_accuracy_curves}.
We observe that RS-MCTS replay schedules yield better final classification performance compared to ETS, especially for small $M$. Furthermore, the replay schedules from RS-MCTS yields slightly better BWT than ETS. 
%For the MNIST datasets, the BWT metric is fairly stable as the memory size increases for RS-MCTS, while BWT increases more gradually for Split CIFAR-100 for larger $M$. 
As the memory size increases, the results from using ETS approaches the ACC and BWT achieved by RS-MCTS. This indicates that replay scheduling is extremely needed when the memory budget is limited as in many real-world cases. 



\subsection{Task Accuracies and Replay Schedule Visualizations}\label{app:task_accuracies_and_replay_schedule_visualization}

We bring further insights about the performance boost from good replay schedules by inspecting how the task accuracies progress during the learning phase. 


\vspace{-2mm}
\paragraph{Split MNIST:} In Figure \ref{fig:splitMNIST_task_accuracies_M24_seed2}, we show the progress in classification performance for each task when using ETS and RS-MCTS with memory size $M=24$ on Split MNIST. 
%In Figure \ref{fig:splitMNIST_task_accuracies_M24_seed2}, we show how the each classification performance varies when training on each task for ETS and RS-MCTS with memory size $M=24$. 
For comparison, we also show the performance from a network that is fine-tuning on the current task without using replay. Both ETS and RS-MCTS overcome catastrophic forgetting to a large degree compared to the fine-tuning network. Our method RS-MCTS further improves the performance compared to ETS with the same memory, which indicates that learning the time to learn can be more efficient against catastrophic forgetting. 
Especially, Task 2 in Split MNIST seems to be the most difficult task to remember since it has the lowest final performance using the fine-tuning network.
%Especially, Task 2 in Split MNIST seems to be the most difficult task to remember since it has the lowest final performance in the baseline on the left. 
Our RS-MCTS schedule manages to retain its performance on Task 2 better than ETS. To bring more insights to this behavior, we look into the proportions of the memory examples per task in the corresponding replay schedule from RS-MCTS in Figure \ref{fig:replay_schedules_proportion_for_mnist_fashionmnist_notmnist} (top). At Task 3, we see that the schedule fills the memory with data from Task 2 and discards replaying Task 1. This helps the network to retain knowledge about Task 2 better than ETS at the cost of forgetting Task 1 slightly. This shows that the learned policy has considered the difficulty level of different tasks. At the next task, the RS-MCTS schedule has decided to rehearse on Task 1 and discards rehearse on Task 2 until training on Task 5. This behavior is similar to spaced repetition, where increasing the time interval between rehearsal helps memory retention. %to retain memory longer. 
We emphasize that even on datasets with few tasks, using learned replay schedules can overcome catastrophic forgetting better than standard ETS approaches.
%\vspace{-10pt}

\begin{figure}[t]
\centering
\setlength{\figwidth}{0.85\columnwidth}
\setlength{\figheight}{.15\textheight}
\input{appendix/figures/replay_schedules/mnist_M24_replay_schedule_barplot}
\input{appendix/figures/replay_schedules/fashionmnist_M24_replay_schedule_barplot}
\input{appendix/figures/replay_schedules/notmnist_M24_replay_schedule_barplot}
\caption{The proportion of memory examples per task from replay schedule found by RS-MCTS on Split MNIST (top), Split FashionMNIST (middle), and Split notMNIST (bottom) with memory size $M=24$. The x-axis displays the current task and the y-axis shows the proportion of examples from the historical tasks. These replay schedules correspond to the learning progress in task accuracy performance for RS-MCTS in Figures \ref{fig:splitMNIST_task_accuracies_M24_seed2}, \ref{fig:splitFashionMNIST_task_accuracies_M24_seed1_appendix}, and \ref{fig:splitNotMNIST_task_accuracies_M24_seed2_appendix} for Split MNIST, Split FashionMNIST, and Split notMNIST respectively. }\vspace{-3mm}
\label{fig:replay_schedules_proportion_for_mnist_fashionmnist_notmnist}
\end{figure}

\begin{figure*}[t]
\centering
\setlength{\figwidth}{0.65\textwidth}
\setlength{\figheight}{0.6\columnwidth}

\pgfplotsset{compat=1.11,
    /pgfplots/ybar legend/.style={
    /pgfplots/legend image code/.code={%
       \draw[##1,/tikz/.cd,yshift=-0.25em]
        (0cm,0cm) rectangle (3pt,0.8em);},
   },
}
\pgfplotsset{every axis title/.append style={at={(0.5,0.92)}}}
\begin{tikzpicture}

\definecolor{color0}{rgb}{0.12156862745098,0.466666666666667,0.705882352941177}
\definecolor{color1}{rgb}{1,0.498039215686275,0.0549019607843137}


\begin{axis}[title={Split CIFAR-100},
ybar,
bar width = 3pt,
height=\figheight,
legend cell align={left},
legend style={
  fill opacity=0.8,
  draw opacity=1,
  text opacity=1,
  at={(0.03,0.97)},
  anchor=north west,
  draw=white!80!black
},
minor xtick={},
minor ytick={},
tick align=outside,
tick pos=left,
width=\figwidth,
x grid style={white!69.0196078431373!black},
xlabel={Task Number},
xmin=-0.335, xmax=21.335,
xtick style={color=black},
xtick={1,2,3,4,5,6,7,8,9,10,11,12,13,14,15,16,17,18,19,20},
y grid style={white!69.0196078431373!black},
ylabel={Accuracy},
ymin=0.49, ymax=0.95,
ytick style={color=black},
ytick={0,0.2,0.4,0.5, 0.6,0.7, 0.8,0.9, 1,1.2},
yticklabels={0.0,0.2,0.4,0.5, 0.6,0.7, 0.8,0.9,1.0,1.2}
]
\addplot [draw=black, fill=color0] coordinates {
(1,0.546000003814697)
(2,0.583999991416931)
(3,0.674000024795532)
(4,0.611999988555908)
(5,0.666000008583069)
(6,0.589999973773956)
(7,0.65200001001358)
(8,0.653999984264374)
(9,0.670000016689301)
(10,0.772000014781952)
(11,0.777999997138977)
(12,0.695999979972839)
(13,0.794000029563904)
(14,0.689999997615814)
(15,0.716000020503998)
(16,0.720000028610229)
(17,0.741999983787537)
(18,0.73199999332428)
(19,0.829999983310699)
(20,0.882000029087067)
};\addlegendentry{ETS}

\addplot [draw=black, fill=color1] coordinates {
(1, 0.576)
(2, 0.646)
(3 ,0.71)
(4, 0.654)
(5, 0.714)
(6, 0.588)
(7, 0.736)
(8, 0.696)
(9, 0.718)
(10, 0.802)
(11, 0.694)
(12, 0.748)
(13, 0.758)
(14, 0.71)
(15, 0.738)
(16, 0.694)
(17, 0.718)
(18, 0.8)
(19, 0.852)
(20, 0.908)
};\addlegendentry{RS-MCTS}
\end{axis}

\end{tikzpicture}
\vspace{-10pt}
\caption{Comparison of test classification accuracies after learning the final task on Split CIFAR-100 from a network trained with ETS and RS-MCTS with memory size $M=285$. RS-MCTS yields higher accuracies on all tasks except Task 6, 11, 13, 16, and 17. The corresponding replay schedule for RS-MCTS are found in Figure \ref{fig:cifar100_replay_schedules_appendix}. Results are shown for a single seed.}
\label{fig:splitCifar100_task_accuracy_seed5_barplot_appendix}
\end{figure*}

\paragraph{Split Fashion-MNIST:} Figure \ref{fig:splitFashionMNIST_task_accuracies_M24_seed1_appendix} shows the progress in classification performance for a network using fine-tuning and replay schedules from ETS and RS-MCTS with memory size $M=24$ when training on each task in Split FashionMNIST. Both ETS and RS-MCTS overcome catastrophic forgetting significantly compared to the fine-tuning network. We observe that our RS-MCTS schedule manages to retain its performance on Task 1 on the final time step even if the Task 1 performance had dropped to almost 85\% at the previous time step. Figure \ref{fig:replay_schedules_proportion_for_mnist_fashionmnist_notmnist} (middle) shows however that this performance boost on Task 1 was the outcome of replaying Task 3 and 4 rather than Task 1 at the final time step. The performance increase of Task 1 is observed for the ETS schedule as well, which could mean that Task 1 benefits from learning Task 5. Still, RS-MCTS manages to receive better overall performance after learning Task 5 than ETS which shows the advantages of using a flexible replay schedule. 
\vspace{-10pt}



\paragraph{Split notMNIST:} Figure \ref{fig:splitNotMNIST_task_accuracies_M24_seed2_appendix} shows a similar visualization of the task classification performance for Split notMNIST. The fine-tuning network forgets Task 1 and 3, which both ETS and RS-MCTS manage to remember. However, RS-MCTS maintains its performance on Task 1 and 2 better than ETS which only yields good performance on the succeeding tasks. Figure \ref{fig:replay_schedules_proportion_for_mnist_fashionmnist_notmnist} (bottom) shows the proportion of memory examples for RS-MCTS. In particular, the schedule fills the memory with data from Task 2 at Task 3 but still retains its performance on the future time steps even if the schedule replays other tasks than Task 2.
\vspace{-10pt}



\paragraph{Split CIFAR-100:}
Due to the large number of tasks, instead of showing the whole learning process, we show the results at the final step for Split CIFAR-100 in Figure \ref{fig:splitCifar100_task_accuracy_seed5_barplot_appendix}. We observe that our method performs better than ETS on 15 out of 20 tasks with the same memory size $M=285$. We visualize the corresponding replay schedule to gain insights into the behavior of the RS-MCTS policy.
Figure \ref{fig:cifar100_replay_schedules_appendix} shows a bubble plot of the proportion of memory examples that are used for replay at each task. Each color of the circles corresponds to a historical task and its size represents the proportion of examples that are replayed at the current task. Thus, each column of the circles corresponds to the memory composition at the current task time. As the memory is limited, the sum of areas of circles in each column is fixed across different time steps. 
The memory examples per task vary dynamically over time in a sophisticated nonlinear way which would be difficult to replace by any heuristics.
This motivates the importance of learning the time to learn in continual learning with fixed size memory.
Furthermore, we observe that the schedule replays the early tasks with a similar proportion in the beginning of learning but gradually increases the time interval between rehearsal.
%Furthermore, we observe that the schedule replays Task 1-3 with a similar proportion at the initial tasks but gradually increases the time interval between rehearsal. 
This shows that our learned schedule establishment stems well with the idea of spaced repetition that early tasks require less repetition in the future for retaining knowledge. 
Moreover, our method is potentially more optimal than native spaced repetition as it considers the relationship among tasks. For example, Task 5-8 need less rehearsal which could potentially be that they are correlated with other tasks or they are simpler.  



\clearpage

%%%% Split MNIST Task Accuracies
\begin{figure*}[t]
\centering
\setlength{\figwidth}{0.31\textwidth}
\setlength{\figheight}{.17\textheight}
\input{figures/splitMnist_task_accuracy_seed2_groupplot}
\caption{Comparison of test classification accuracies for Task 1-5 on Split MNIST from a network trained without replay (Fine-tuning), ETS, and RS-MCTS. The memory size is set to $M=24$ for ETS and RS-MCTS. The replay schedule found by RS-MCTS is shown in Figure \ref{fig:replay_schedules_proportion_for_mnist_fashionmnist_notmnist} (top). Results are shown for a single seed. }
\label{fig:splitMNIST_task_accuracies_M24_seed2}
\end{figure*}

%%%% Split Fashion-MNIST Task Accuracies
\begin{figure*}[t]
\centering
\setlength{\figwidth}{0.31\textwidth}
\setlength{\figheight}{.17\textheight}


%\renewcommand\thesubfigure{(\alph{subfigure})}
\pgfplotsset{every axis title/.append style={at={(0.5,0.90)}}}
\begin{tikzpicture}

\definecolor{color0}{rgb}{0.12156862745098,0.466666666666667,0.705882352941177}
\definecolor{color1}{rgb}{1,0.498039215686275,0.0549019607843137}
\definecolor{color2}{rgb}{0.172549019607843,0.627450980392157,0.172549019607843}
\definecolor{color3}{rgb}{0.83921568627451,0.152941176470588,0.156862745098039}
\definecolor{color4}{rgb}{0.580392156862745,0.403921568627451,0.741176470588235}

\begin{groupplot}[group style={group size= 3 by 1, horizontal sep=2cm, vertical sep=2cm, group name=Task_group}, ]

\nextgroupplot[title={Fine-tuning ($M=0$)},
height=\figheight,
    legend cell align={left},
    legend style={
      fill opacity=0.8,
      draw opacity=1,
      text opacity=1,
      at={(0.03,0.03)},
      anchor=south west,
      draw=white!80!black
    },
minor xtick={},
minor ytick={},
tick align=outside,
tick pos=left,
width=\figwidth,
x grid style={white!69.0196078431373!black},
xlabel={Task Number},
xmajorgrids,
xmin=0.8, xmax=5.2,
xtick style={color=black},
xtick={1,2,3,4,5},
y grid style={white!69.0196078431373!black},
ylabel={Accuracy},
ymajorgrids,
ymin=0.48, ymax=1.01,
ytick style={color=black},
ytick={0.5, 0.6, 0.7, 0.8, 0.9, 1, 1.05},
yticklabels={0.5, 0.6, 0.7, 0.8, 0.9, 1.0,1.05}
]
\addplot [line width=1.5pt, color0, mark=*, mark size=1.5, mark options={solid}]
table {%
1 0.994000017642975
2 0.902999997138977
3 0.508499979972839
4 0.500500023365021
5 0.5
};\label{plots:Task1}
\addplot [line width=1.5pt, color1, mark=*, mark size=1.5, mark options={solid}]
table {%
2 0.973999977111816
3 0.610000014305115
4 0.501999974250793
5 0.505999982357025
};\label{plots:Task2}
\addplot [line width=1.5pt, color2, mark=*, mark size=1.5, mark options={solid}]
table {%
3 1
4 0.998000025749207
5 0.967499971389771
};\label{plots:Task3}
\addplot [line width=1.5pt, color3, mark=*, mark size=1.5, mark options={solid}]
table {%
4 1
5 0.976000010967255
};\label{plots:Task4}
\addplot [line width=1.5pt, color4, mark=*, mark size=1.5, mark options={solid}]
table {%
5 0.998000025749207
};\label{plots:Task5}

\nextgroupplot[title={ETS ($M = 24$)},
height=\figheight,
legend cell align={left},
legend style={
  fill opacity=0.8,
  draw opacity=1,
  text opacity=1,
  at={(0.03,0.03)},
  anchor=south west,
  draw=white!80!black
},
minor xtick={},
minor ytick={0.65, 0.75, 0.85, 0.95, 1.05},
tick align=outside,
tick pos=left,
width=\figwidth,
x grid style={white!69.0196078431373!black},
xlabel={Task Number},
xmajorgrids,
xmin=0.8, xmax=5.2,
xtick style={color=black},
xtick={1,2,3,4,5},
y grid style={white!69.0196078431373!black},
ylabel={Accuracy},
ymajorgrids,
yminorgrids,
ymin=0.7, ymax=1.01,
ytick style={color=black},
ytick={0.7, 0.8, 0.9, 1.0, 1.1},
yticklabels={0.7, 0.8, 0.9, 1.0, 1.1}
]
\addplot [line width=1.5pt, color0, mark=*, mark size=1.5, mark options={solid}]
table {%
1 0.994000017642975
2 0.935000002384186
3 0.921500027179718
4 0.92849999666214
5 0.948000013828278
};
\addplot [line width=1.5pt, color1, mark=*, mark size=1.5, mark options={solid}]
table {%
2 0.975000023841858
3 0.932500004768372
4 0.930499970912933
5 0.888999998569489
};
\addplot [line width=1.5pt, color2, mark=*, mark size=1.5, mark options={solid}]
table {%
3 0.999499976634979
4 0.998000025749207
5 0.986999988555908
};
\addplot [line width=1.5pt, color3, mark=*, mark size=1.5, mark options={solid}]
table {%
4 1
5 0.996500015258789
};
\addplot [line width=1.5pt, color4, mark=*, mark size=1.5, mark options={solid}]
table {%
5 0.998000025749207
};
%\addlegendentry{RS-MCTS}

\nextgroupplot[title={RS-MCTS ($M = 24$)},
height=\figheight,
legend cell align={left},
legend style={
  fill opacity=0.8,
  draw opacity=1,
  text opacity=1,
  at={(0.03,0.03)},
  anchor=south west,
  draw=white!80!black
},
minor xtick={},
minor ytick={0.65, 0.75, 0.85, 0.95, 1.05},
tick align=outside,
tick pos=left,
width=\figwidth,
x grid style={white!69.0196078431373!black},
xlabel={Task Number},
xmajorgrids,
xmin=0.8, xmax=5.2,
xtick style={color=black},
xtick={1,2,3,4,5},
y grid style={white!69.0196078431373!black},
ylabel={Accuracy},
ymajorgrids,
yminorgrids,
ymin=0.7, ymax=1.01,
ytick style={color=black},
ytick={0.7, 0.8, 0.9, 1.0, 1.1},
yticklabels={0.7, 0.8, 0.9, 1.0, 1.1}
]
\addplot [line width=1.5pt, color0, mark=*, mark size=1.5, mark options={solid}]
table {%
1 0.994000017642975
2 0.972500026226044
3 0.970000028610229
4 0.860499978065491
5 0.973999977111816
};
\addplot [line width=1.5pt, color1, mark=*, mark size=1.5, mark options={solid}]
table {%
2 0.978500008583069
3 0.976499974727631
4 0.949000000953674
5 0.970000028610229
};
\addplot [line width=1.5pt, color2, mark=*, mark size=1.5, mark options={solid}]
table {%
3 1
4 0.999000012874603
5 0.996999979019165
};
\addplot [line width=1.5pt, color3, mark=*, mark size=1.5, mark options={solid}]
table {%
4 1
5 0.997500002384186
};
\addplot [line width=1.5pt, color4, mark=*, mark size=1.5, mark options={solid}]
table {%
5 0.998000025749207
};
%\addlegendentry{RS-MCTS}

\end{groupplot}

\path (Task_group c1r1.north west|-current bounding box.north)--
      coordinate(legendpos)
      (Task_group c3r1.north east|-current bounding box.north);
\matrix[
    matrix of nodes,
    anchor=south,
    draw,
    inner sep=0.2em,
    draw
  ]at([yshift=1ex]legendpos)
  {
    \ref{plots:Task1}& Task 1&[5pt]
    \ref{plots:Task2}& Task 2&[5pt]
    \ref{plots:Task3}& Task 3&[5pt]
    \ref{plots:Task4}& Task 4&[5pt]
    \ref{plots:Task5}& Task 5&[5pt]\\};

\end{tikzpicture} 

\caption{Comparison of test classification accuracies for Task 1-5 on Split FashionMNIST from a network trained without replay (Fine-tuning), ETS, and RS-MCTS. The memory size is set to $M=24$ for ETS and RS-MCTS. The replay schedule found by RS-MCTS is shown in Figure \ref{fig:replay_schedules_proportion_for_mnist_fashionmnist_notmnist} (middle). Results are shown for a single seed. }
\label{fig:splitFashionMNIST_task_accuracies_M24_seed1_appendix}
\end{figure*}

%%%% Split notMNIST Task Accuracies
\begin{figure*}[t]
\centering
\setlength{\figwidth}{0.31\textwidth}
\setlength{\figheight}{.17\textheight}


%\renewcommand\thesubfigure{(\alph{subfigure})}
\pgfplotsset{every axis title/.append style={at={(0.5,0.90)}}}
\begin{tikzpicture}

\definecolor{color0}{rgb}{0.12156862745098,0.466666666666667,0.705882352941177}
\definecolor{color1}{rgb}{1,0.498039215686275,0.0549019607843137}
\definecolor{color2}{rgb}{0.172549019607843,0.627450980392157,0.172549019607843}
\definecolor{color3}{rgb}{0.83921568627451,0.152941176470588,0.156862745098039}
\definecolor{color4}{rgb}{0.580392156862745,0.403921568627451,0.741176470588235}

\begin{groupplot}[group style={group size= 3 by 1, horizontal sep=2cm, vertical sep=2cm, group name=Task_group}, ]

\nextgroupplot[title={Fine-tuning ($M=0$)},
height=\figheight,
    legend cell align={left},
    legend style={
      fill opacity=0.8,
      draw opacity=1,
      text opacity=1,
      at={(0.03,0.03)},
      anchor=south west,
      draw=white!80!black
    },
minor xtick={},
minor ytick={0.65, 0.75, 0.85, 0.95, 1.05},
tick align=outside,
tick pos=left,
width=\figwidth,
x grid style={white!69.0196078431373!black},
xlabel={Task Number},
xmajorgrids,
xmin=0.8, xmax=5.2,
xtick style={color=black},
xtick={1,2,3,4,5},
y grid style={white!69.0196078431373!black},
ylabel={Accuracy},
ymajorgrids,
yminorgrids,
ymin=0.7, ymax=1.01,
ytick style={color=black},
ytick={0.7, 0.8, 0.9, 1.0, 1.1},
yticklabels={0.7, 0.8, 0.9, 1.0, 1.1}
]
\addplot [line width=1.5pt, color0, mark=*, mark size=1.5, mark options={solid}]
table {%
1 0.962406039237976
2 0.962406039237976
3 0.74436092376709
4 0.827067673206329
5 0.774436116218567
};\label{plots:Task1}
\addplot [line width=1.5pt, color1, mark=*, mark size=1.5, mark options={solid}]
table {%
2 0.990196049213409
3 0.892156839370728
4 0.872548997402191
5 0.921568632125854
};\label{plots:Task2}
\addplot [line width=1.5pt, color2, mark=*, mark size=1.5, mark options={solid}]
table {%
3 0.977777779102325
4 0.855555534362793
5 0.766666650772095
};\label{plots:Task3}
\addplot [line width=1.5pt, color3, mark=*, mark size=1.5, mark options={solid}]
table {%
4 0.985294103622437
5 0.941176474094391
};\label{plots:Task4}
\addplot [line width=1.5pt, color4, mark=*, mark size=1.5, mark options={solid}]
table {%
5 0.954545438289642
};\label{plots:Task5}

\nextgroupplot[title={ETS ($M = 24$)},
height=\figheight,
legend cell align={left},
legend style={
  fill opacity=0.8,
  draw opacity=1,
  text opacity=1,
  at={(0.03,0.03)},
  anchor=south west,
  draw=white!80!black
},
minor xtick={},
minor ytick={0.65, 0.75, 0.85, 0.95, 1.05},
tick align=outside,
tick pos=left,
width=\figwidth,
x grid style={white!69.0196078431373!black},
xlabel={Task Number},
xmajorgrids,
xmin=0.8, xmax=5.2,
xtick style={color=black},
xtick={1,2,3,4,5},
y grid style={white!69.0196078431373!black},
ylabel={Accuracy},
ymajorgrids,
yminorgrids,
ymin=0.7, ymax=1.01,
ytick style={color=black},
ytick={0.7, 0.8, 0.9, 1.0, 1.1},
yticklabels={0.7, 0.8, 0.9, 1.0, 1.1}
]
\addplot [line width=1.5pt, color0, mark=*, mark size=1.5, mark options={solid}]
table {%
1 0.962406039237976
2 0.947368443012238
3 0.924812018871307
4 0.947368443012238
5 0.887218058109283
};
\addplot [line width=1.5pt, color1, mark=*, mark size=1.5, mark options={solid}]
table {%
2 0.990196049213409
3 0.911764681339264
4 0.921568632125854
5 0.921568632125854
};
\addplot [line width=1.5pt, color2, mark=*, mark size=1.5, mark options={solid}]
table {%
3 0.98888885974884
4 0.977777779102325
5 0.977777779102325
};
\addplot [line width=1.5pt, color3, mark=*, mark size=1.5, mark options={solid}]
table {%
4 0.970588207244873
5 0.970588207244873
};
\addplot [line width=1.5pt, color4, mark=*, mark size=1.5, mark options={solid}]
table {%
5 0.954545438289642
};
%\addlegendentry{RS-MCTS}

\nextgroupplot[title={RS-MCTS ($M = 24$)},
height=\figheight,
legend cell align={left},
legend style={
  fill opacity=0.8,
  draw opacity=1,
  text opacity=1,
  at={(0.03,0.03)},
  anchor=south west,
  draw=white!80!black
},
minor xtick={},
minor ytick={0.65, 0.75, 0.85, 0.95, 1.05},
tick align=outside,
tick pos=left,
width=\figwidth,
x grid style={white!69.0196078431373!black},
xlabel={Task Number},
xmajorgrids,
xmin=0.8, xmax=5.2,
xtick style={color=black},
xtick={1,2,3,4,5},
y grid style={white!69.0196078431373!black},
ylabel={Accuracy},
ymajorgrids,
yminorgrids,
ymin=0.7, ymax=1.01,
ytick style={color=black},
ytick={0.7, 0.8, 0.9, 1.0, 1.1},
yticklabels={0.7, 0.8, 0.9, 1.0, 1.1}
]
\addplot [line width=1.5pt, color0, mark=*, mark size=1.5, mark options={solid}]
table {%
1 0.954887211322784
2 0.969924807548523
3 0.954887211322784
4 0.969924807548523
5 0.954887211322784
};
\addplot [line width=1.5pt, color1, mark=*, mark size=1.5, mark options={solid}]
table {%
2 0.990196049213409
3 0.970588207244873
4 0.980392158031464
5 0.980392158031464
};
\addplot [line width=1.5pt, color2, mark=*, mark size=1.5, mark options={solid}]
table {%
3 0.98888885974884
4 0.944444417953491
5 0.966666638851166
};
\addplot [line width=1.5pt, color3, mark=*, mark size=1.5, mark options={solid}]
table {%
4 0.941176474094391
5 0.985294103622437
};
\addplot [line width=1.5pt, color4, mark=*, mark size=1.5, mark options={solid}]
table {%
5 0.939393937587738
};
%\addlegendentry{RS-MCTS}

\end{groupplot}

\path (Task_group c1r1.north west|-current bounding box.north)--
      coordinate(legendpos)
      (Task_group c3r1.north east|-current bounding box.north);
\matrix[
    matrix of nodes,
    anchor=south,
    draw,
    inner sep=0.2em,
    draw
  ]at([yshift=1ex]legendpos)
  {
    \ref{plots:Task1}& Task 1&[5pt]
    \ref{plots:Task2}& Task 2&[5pt]
    \ref{plots:Task3}& Task 3&[5pt]
    \ref{plots:Task4}& Task 4&[5pt]
    \ref{plots:Task5}& Task 5&[5pt]\\};

\end{tikzpicture} 

\caption{Comparison of test classification accuracies for Task 1-5 on Split notMNIST from a network trained without replay (Fine-tuning), ETS, and RS-MCTS. The memory size is set to $M=24$ for ETS and RS-MCTS. The replay schedule found by RS-MCTS is shown in Figure \ref{fig:replay_schedules_proportion_for_mnist_fashionmnist_notmnist} (bottom). Results are shown for a single seed. }\vspace{-10pt}
\label{fig:splitNotMNIST_task_accuracies_M24_seed2_appendix}
\end{figure*}

\clearpage

%\begin{figure*}[t]
%\centering
%\begin{minipage}{0.95\columnwidth}
%  \centering
%  \setlength{\figwidth}{\textwidth}
%  \setlength{\figheight}{0.7\columnwidth}
%  \input{appendix/figures/replay_schedules/cifar100_ets_M285_seed5_scatter}
%\end{minipage}
%\begin{minipage}{0.95\columnwidth}
%  \centering
%  \setlength{\figwidth}{\textwidth}
%  \setlength{\figheight}{0.7\columnwidth}
%  \input{appendix/figures/replay_schedules/cifar100_rs_mcts_M285_seed5_scatter}
%\end{minipage}
%\caption{The proportion of memory examples that are used for replay at each task from replay schedules using ETS (left) and RS-MCTS (right) when training on Split CIFAR-100 with memory size $M=285$. The color of the circles corresponds to a historical task and its size represents the proportion of examples that are replayed at the current task. The sum of areas of circles in each column is fixed across different time. We show the ACC yielded by the replay schedules above each figure.}
%\vspace{-10pt}
%\label{fig:cifar100_replay_schedules_appendix}
%\end{figure*}

%\clearpage

\begin{figure*}[h!]
  \centering
  \setlength{\figwidth}{0.32\textwidth}
  \setlength{\figheight}{0.25\textwidth}
  \input{appendix/figures/mnist_alternative_selection_methods/best_rewards_rs_mcts_selection_methods_M24_groupplot}
  \vspace{-6pt}
  \caption{Best rewards measured in ACC for ETS and RS-MCTS using memory size $M=24$ on Split MNIST for different memory selection strategies Random Selection, Mean-of-Features, and K-center Coreset. Results are averaged over 5 seeds. 
  }
  \label{fig:MNIST_mcts_best_rewards_M24_selection_methods_appendix}
\end{figure*}

\subsection{Alternative Memory Selection Strategies}\label{app:alternative_memory_selection_strategies}

In this section, we evaluate RS-MCTS on two alternative selection strategies, namely Mean-of-Features (MoF)~\citep{rebuffi2017icarl} and K-center Coreset~\citep{nguyen2017variational}, and compare these to using random selection of examples to store as historical data. We begin by giving a brief description of both strategies: %\vspace{-12pt}

\paragraph{Mean-of-Features:} We extract feature representations before the classification layer of all images for every class. Then the examples to store in memory are selected based on the similarity between the feature representations and their moving mean value. This selection strategy was used in iCaRL~\citep{rebuffi2017icarl} where they grabbed such examples to create an exemplar set for performing classification in feature space. Similar to \citet{chaudhry2019tiny}, we use this strategy for selecting memory examples that should be close to the mode of the class in feature space.%\vspace{-pt}

\paragraph{K-center Coreset:} We employ the greedy K-center algorithm~\citep{gonzalez1985clustering} that was used for memory replay in Variational Continual Learning~\citep{nguyen2017variational}. In contrast to MoF, this strategy operates in input space and returns examples that are spread throughout the input space.%\vspace{12pt}

We perform experiments on Split MNIST to compare MoF and K-center Coreset to random selection as selection strategies. Figure \ref{fig:MNIST_mcts_best_rewards_M24_selection_methods_appendix} shows the progress in ACC when searching for replay schedules with RS-MCTS using the three strategies and memory size $M=24$. We also show the performance from using ETS with each strategy as baselines. All three strategies find better replay schedules than ETS using less than 50 simulations. Both MoF and K-center Coreset strategies perform on par with random selection over the 500 simulations with RS-MCTS. 

In Figure \ref{fig:MNIST_ACC_over_M_selection_methods_appendix}, we compare the performance using the selection strategies for various memory sizes. The schedules from RS-MCTS for the selection strategies perform on par as the memory size $M$ increases where K-center Coreset strategy gains slightly superior performance at $M=120$ and $M=200$. A potential reason for this slight performance increase could be that K-center Coreset selects memory examples to store that are spread out in the input space for each class rather than selecting examples that are near the mode for all examples as in MoF.   



\begin{figure}[t]
  \centering
  \setlength{\figwidth}{0.6\textwidth}
  \setlength{\figheight}{0.4\textwidth}
  %\vspace{-8pt}
  \input{appendix/figures/mnist_alternative_selection_methods/MNIST_accuracy_bwt_curveplot}
  \caption{ACC over memory size $M$ for ETS and RS-MCTS on Split MNIST using different memory selection methods, namely Random Selection (Random), Mean-of-Features (MoF), and K-center Coreset. All results have been averaged over 5 seeds. 
  }\vspace{-6mm}
  \label{fig:MNIST_ACC_over_M_selection_methods_appendix}
\end{figure}

\subsection{Class Incremental Learning}\label{app:class_incremental_learning}

In this section, we apply replay scheduling to the class incremental learning (Class-IL) setting where task information is unavailable at test time. This scenario requires the model to both infer the task and class of the test input. This continual learning scenario is considered to be more challenging than the task incremental learning (Task-IL) setting where task labels are available at test time that we address in the main paper. We assess RS-MCTS for Class-IL on the Split MNIST dataset. The experimental settings are the same as in the main paper, except that the network architecture uses a single-head classification layer in this scenario. 

We need to adjust the way to construct the action space of memory combinations to apply RS-MCTS to the Class-IL setting. As earlier, we create $t-1$ bins $[b_1, b_2, ....b_{t-1}]$ and choose a historical task to sample for each bin $b_i \in {1,.., t-1}$ before learning task $t$. In the Class-IL setting, the model will catastrophically forget tasks that are excluded from the selected memory combination. Therefore, we ensure that the memory combinations includes some proportion of samples from all previous tasks by extending the memory with $t-1$ additional bins $[c_1, c_2, ..., c_{t-1}]$ where each $c_i$ includes samples from its corresponding task $i$. For example, at Task 3, we have Task 1 and 2 in history; so the unique choices of bins are $[1,1], [1,2], [2,2]$. We then extend each of these unique choices with $[1, 2]$, such that the unique choices become $[1,1,1,2], [1,2,1,2], [2,2,1,2]$, where $[1,1,1,2]$ indicates that 75\% and 25\% of the memory is from Task 1 and Task 2 respectively, and $[1,2,1,2]$ indicates that half memory is from Task 1 and the other half are from Task 2 etc.

In the first experiment, we inspect the performance progress measured in ACC over MCTS simulations when $M=24$ for Split MNIST. Figure \ref{fig:MNIST_mcts_best_rewards_M24_class_il_appendix} shows that RS-MCTS quickly finds a significantly better ACC than ETS using less than 50 iterations. Furthermore, RS-MCTS can reach similar performance as from the breadth-first search (BFS) with significantly less computational budget. These results confirm that scheduling of which memory examples to replay is important in the Class-IL scenario. 

In Figure \ref{fig:MNIST_ACC_BWT_over_M_class_il_appendix}, we show that the performance of RS-MCTS improves on both ACC and BWT across  different  memory sizes $M$. We observe that RS-MCTS replay schedules yield better final classification performance compared to ETS, especially for small $M$ in the Class-IL scenario. The performance for both RS-MCTS and ETS schedules increases rapidly as the memory size becomes larger which confirms the importance of the memory size in Class-IL. The results from using the ETS schedules approach the ACC and BWT metrics achieved by RS-MCTS as the memory size grows. This indicates that replay scheduling is necessary for situations with limited memory budgets in the Class-IL scenario. 

\begin{figure}[h]
  \centering
  \setlength{\figwidth}{0.5\columnwidth}
  \setlength{\figheight}{0.3\textwidth}
  \input{appendix/figures/mnist_class_incremental_learning/best_rewards_rs_mcts_with_upper_bound_M24}
  \vspace{-6pt}
  \caption{MCTS simulation performance on Split MNIST in the Class-IL setting where the average test classification accuracies over tasks after training on the final task (ACC) is used as the reward. The ETS performance is shown in blue as a baseline and the green line show the ACC from the optimal schedules found from a breadth-first search (BFS) as an upper bound. We use $M= 24$ and all results have been averaged over 5 seeds.
  }%\vspace{-12pt}
  \label{fig:MNIST_mcts_best_rewards_M24_class_il_appendix}
\end{figure}

\begin{figure}[h]
  \centering
  \setlength{\figwidth}{0.5\columnwidth}
  \setlength{\figheight}{0.3\textwidth}
  \input{appendix/figures/mnist_class_incremental_learning/MNIST_accuracy_bwt_curveplot}
  \vspace{-4mm}
  \caption{Average test classification accuracies over tasks after training on the final task (ACC) and backward transfer (BWT) over different memory sizes $M$ for Split MNIST in the Class Incremental Learning setting. We show accuracies obtained by using an equal proportion of examples from previous tasks (ETS) as well as the result from the best replay schedules found with MCTS (RS-MCTS). All results have been averaged over 5 seeds.
  }%\vspace{-12pt}
  \label{fig:MNIST_ACC_BWT_over_M_class_il_appendix}
\end{figure}








%------------------------------------------------------------------------
%
\begin{figure*}[t]
  \centering
  \setlength{\figwidth}{0.32\textwidth}
  \setlength{\figheight}{0.25\textwidth}
  \input{appendix/figures/mnist_alternative_selection_methods/best_rewards_rs_mcts_selection_methods_M24_groupplot}
  \caption{Best rewards measured in ACC for ETS and RS-MCTS using memory size $M=24$ on Split MNIST for different memory selection strategies Random Selection, Mean-of-Features, and K-center Coreset. Results are averaged over 5 seeds. 
  }%\vspace{-12pt}
  \label{fig:MNIST_mcts_best_rewards_M24_selection_methods_appendix}
\end{figure*}

\begin{figure*}[t]
  \centering
  \setlength{\figwidth}{0.6\textwidth}
  \setlength{\figheight}{0.4\textwidth}
  \input{appendix/figures/mnist_alternative_selection_methods/MNIST_accuracy_bwt_curveplot}
  \caption{ACC over memory size $M$ for ETS and RS-MCTS on Split MNIST using different memory selection methods, namely Random Selection (Random), Mean-of-Features (MoF), and K-center Coreset. All results have been averaged over 5 seeds. 
  }\vspace{-12pt}
  \label{fig:MNIST_ACC_over_M_selection_methods_appendix}
\end{figure*}

\section{Additional Experiments}\label{app:additional_experiments}

This section includes additional experiments with RS-MCTS using alternative memory selection strategies for selecting historical data as well as applying RS-MCTS in the class incremental learning (Class-IL) scenario where task information is inaccessible. 

\subsection{Alternative Memory Selection Strategies}
%\CZ{This is more important than class-IL.}
We evaluate RS-MCTS on two alternative selection strategies, namely Mean-of-Features (MoF)~\cite{rebuffi2017icarl} and K-center Coreset~\cite{nguyen2017variational}, and compare these to using random selection of examples to store as historical data. We begin by giving a brief description of both strategies: \vspace{-12pt}

\paragraph{Mean-of-Features:} We extract feature representations before the classification layer of all images for every class. Then the examples to store in memory are selected based on the similarity between the feature representations and their moving mean value. This selection strategy was used in iCaRL~\cite{rebuffi2017icarl} where they grabbed such examples to create an exemplar set for performing classification in feature space. Similar to Chaudhry \etal~\cite{chaudhry2019tiny}, we use this strategy for selecting memory examples that should be close to the mode of the class in feature space.\vspace{-12pt}

\paragraph{K-center Coreset:} We employ the greedy K-center algorithm~\cite{gonzalez1985clustering} that was used for memory replay in Variational Continual Learning~\cite{nguyen2017variational}. In contrast to MoF, this strategy operates in input space and returns examples that are spread throughout the input space.\vspace{12pt}

We perform experiments on Split MNIST to compare MoF and K-center Coreset to random selection as selection strategies. Figure \ref{fig:MNIST_mcts_best_rewards_M24_selection_methods_appendix} shows the progress in ACC when searching for replay schedules with RS-MCTS using the three strategies and memory size $M=24$. We also show the performance from using ETS with each strategy as baselines. All three strategies find better replay schedules than ETS using less than 50 simulations. Both MoF and K-center Coreset strategies perform on par with random selection over the 500 simulations with RS-MCTS. 

In Figure \ref{fig:MNIST_ACC_over_M_selection_methods_appendix}, we compare the performance using the selection strategies for various memory sizes. The schedules from RS-MCTS for the selection strategies perform on par as the memory size $M$ increases where K-center Coreset strategy gains slightly superior performance at $M=120$ and $M=200$. A potential reason for this slight performance increase could be that K-center Coreset selects memory examples to store that are spread out in the input space for each class rather than selecting examples that are near the mode for all examples as in MoF.   


\subsection{Class Incremental Learning}

In the first experiment, we apply replay scheduling to the class incremental learning (Class-IL) setting where task information is unavailable at test time. This scenario requires the model to both infer the task and class of the test input. This continual learning scenario is considered to be more challenging than the task incremental learning (Task-IL) setting where task labels are available at test time that we address in the main paper. We assess RS-MCTS for Class-IL on the Split MNIST dataset. The experimental settings are the same as in the main paper, except that the network architecture uses a single-head classification layer in this scenario. 

We need to adjust the way to construct the action space of memory combinations to apply RS-MCTS to the Class-IL setting. As earlier, we create $t-1$ bins $[b_1, b_2, ....b_{t-1}]$ and choose a historical task to sample for each bin $b_i \in {1,.., t-1}$ before learning task $t$. In the Class-IL setting, the model will catastrophically forget tasks that are excluded from the selected memory combination. Therefore, we ensure that the memory combinations includes some proportion of samples from all previous tasks by extending the memory with $t-1$ additional bins $[c_1, c_2, ..., c_{t-1}]$ where each $c_i$ includes samples from its corresponding task $i$. For example, at Task 3, we have Task 1 and 2 in history; so the unique choices of bins are $[1,1], [1,2], [2,2]$. We then extend each of these unique choices with $[1, 2]$, such that the unique choices become $[1,1,1,2], [1,2,1,2], [2,2,1,2]$, where $[1,1,1,2]$ indicates that 75\% and 25\% of the memory is from Task 1 and Task 2 respectively, and $[1,2,1,2]$ indicates that half memory is from Task 1 and the other half are from Task 2 etc.

In the first experiment, we inspect the performance progress measured in ACC over MCTS simulations when $M=24$ for Split MNIST. Figure \ref{fig:MNIST_mcts_best_rewards_M24_class_il_appendix} shows that RS-MCTS quickly finds a significantly better ACC than ETS using less than 50 iterations. Furthermore, RS-MCTS can reach similar performance as from the breadth-first search (BFS) with significantly less computational budget. These results confirm that scheduling of which memory examples to replay is important in the Class-IL scenario. 

In Figure \ref{fig:MNIST_ACC_BWT_over_M_class_il_appendix}, we show that the performance of RS-MCTS improves on both ACC and BWT across  different  memory sizes $M$. We observe that RS-MCTS replay schedules yield better final classification performance compared to ETS, especially for small $M$ in the Class-IL scenario. The performance for both RS-MCTS and ETS schedules increases rapidly as the memory size becomes larger which confirms the importance of the memory size in Class-IL. The results from using the ETS schedules approach the ACC and BWT metrics achieved by RS-MCTS as the memory size grows. This indicates that replay scheduling is necessary for situations with limited memory budgets in the Class-IL scenario. 

\begin{figure}[h]
  \centering
  \setlength{\figwidth}{0.9\columnwidth}
  \setlength{\figheight}{0.3\textwidth}
  \input{appendix/figures/mnist_class_incremental_learning/best_rewards_rs_mcts_with_upper_bound_M24}
  \caption{MCTS simulation performance on Split MNIST in the Class Incremental Learning setting where the average test classification accuracies over tasks after training on the final task (ACC) is used as the reward. The ETS performance is shown in blue as a baseline and the green line show the ACC from the optimal schedules found from a breadth-first search (BFS) as an upper bound. We use $M= 24$ and all results have been averaged over 5 seeds.
  }\vspace{-12pt}
  \label{fig:MNIST_mcts_best_rewards_M24_class_il_appendix}
\end{figure}

\begin{figure}[h]
  \centering
  \setlength{\figwidth}{0.85\columnwidth}
  \setlength{\figheight}{0.3\textwidth}
  \input{appendix/figures/mnist_class_incremental_learning/MNIST_accuracy_bwt_curveplot}
  \caption{Average test classification accuracies over tasks after training on the final task (ACC) and backward transfer (BWT) over different memory sizes $M$ for Split MNIST in the Class Incremental Learning setting. We show accuracies obtained by using an equal proportion of examples from previous tasks (ETS) as well as the result from the best replay schedules found with MCTS (RS-MCTS). All results have been averaged over 5 seeds.
  }\vspace{-12pt}
  \label{fig:MNIST_ACC_BWT_over_M_class_il_appendix}
\end{figure}


\begin{figure}[t]
\centering
\setlength{\figwidth}{0.95\columnwidth}
\setlength{\figheight}{.3\textheight}

\begin{tikzpicture}
\tikzstyle{every node}=[font=\small]
\definecolor{color0}{rgb}{0.12156862745098,0.466666666666667,0.705882352941177}
\definecolor{color1}{rgb}{1,0.498039215686275,0.0549019607843137}
\definecolor{color2}{rgb}{0.172549019607843,0.627450980392157,0.172549019607843}
\definecolor{color3}{rgb}{0.83921568627451,0.152941176470588,0.156862745098039}
\definecolor{color4}{rgb}{0.580392156862745,0.403921568627451,0.741176470588235}

\begin{axis}[title={ETS for 5-Task Dataset},
ybar,
bar width = 4pt,
height=\figheight,
legend cell align={left},
legend columns=2,
legend style={
  fill opacity=0.99,
  draw opacity=1,
  text opacity=1,
  at={(0.67,1.13)},
  anchor=north west,
  draw=white!80!black
},
minor xtick={},
minor ytick={},
tick align=outside,
tick pos=left,
width=\figwidth,
x grid style={white!69.0196078431373!black},
xlabel={Task Number},
xmin=1.5, xmax=5.5,
xtick style={color=black},
xtick={2,3,4,5},
y grid style={white!69.0196078431373!black},
ymajorgrids,
ylabel={Proportion (\%)},
ymin=0.0, ymax=1.1,
ytick style={color=black},
ytick={0,0.25, 0.5, 0.75, 1, 1.2},
yticklabels={0, 25, 50, 75, 100, 120}
]
\addplot [draw=black, fill=color0] coordinates {
(2,1.0)
(3,0.5)
(4,0.333)
(5,0.25)
};\addlegendentry{Task 1}

\addplot [draw=black, fill=color1] coordinates {
(3,0.5)
(4,0.333)
(5,0.25)
};\addlegendentry{Task 2}

\addplot [draw=black, fill=color2] coordinates {
(4,0.333)
(5,0.25)
};\addlegendentry{Task 3}

\addplot [draw=black, fill=color3] coordinates {
(5,0.25)
};\addlegendentry{Task 4}
\end{axis}

\end{tikzpicture}

\caption{testing proportion barplot in tikz }
\label{fig:testing_proportion_barplot}
\end{figure}

\clearpage 


\section{Heuristic Scheduling Baseline}\label{app:heuristic_scheduling_baseline}

We implemented a heuristic scheduling baseline to compare against RS-MCTS. The baseline keeps a validation set for the old tasks and replays the tasks which validation accuracy is below a certain threshold. We set the threshold in the following way: Let $A_{t, i}$ be the validation accuracy for task $t$ evaluated at time step $i$. The best evaluated validation accuracy for task $t$ at time $i$ is given by $A_{t, i}^{(best)} = \max(\{A_{t, 1}, ..., A_{t, i} \})$. The condition for replaying task $t$ on the next time step is then $A_{t, i} < \tau A_{t, i}^{(best)}$, where $\tau \in [0, 1]$ is a ratio controlling how much the current accuracy on task $t$ is allowed to decrease w.r.t. the best accuracy. The replay memory is filled with $M/k$, where $k$ is the number of tasks that need to be replayed according to their decrease in validation accuracy. This heuristic scheduling corresponds to the intuition of re-learning when a task has been forgotten. Training on the current task is performed without replay if the accuracy on all old tasks is above their corresponding threshold.       

\paragraph{Grid search for $\tau$.} We performed a coarse-to-fine grid search for the ratio $\tau$ on each dataset. The best value for $\tau$ is selected according to the highest mean accuracy on the validation set averaged over 5 seeds. The validation set consists of 15\% of the training data and is the same for RS-MCTS. We use the same experimental settings as described in Appendix \ref{app:experimental_settings}. The memory sizes are set to $M=10$ and $M=100$ for the 5-task datasets and the 10/20-task datasets respectively, and we apply uniform sampling as the memory selection method. We provide the ranges for $\tau$ that was used on each dataset and put the best value in \textbf{bold}:
\begin{itemize}[topsep=1pt]
    \item Split MNIST: $\tau =$ [0.9, 0.93, 0.95, \textbf{0.96}, 0.97, 0.98, 0.99]
    \item Split FashionMNIST: $\tau =$ [0.9, 0.93, 0.95, 0.96, \textbf{0.97}, 0.98, 0.99]
    \item Split notMNIST: $\tau =$ [0.9, 0.93, 0.95, 0.96, 0.97, \textbf{0.98}, 0.99]
    \item Permuted MNIST: $\tau =$ [0.5, 0.55, 0.6, 0.65, 0.7, \textbf{0.75}, 0.8, 0.9, 0.95, 0.97, 0.99]
    \item Split CIFAR-100: $\tau =$ [0.3, 0.4, 0.45, \textbf{0.5}, 0.55, 0.6, 0.65, 0.7, 0.8, 0.9, 0.95, 0.97, 0.99]
    \item  Split miniImagenet: $\tau =$ [0.5, 0.6, 0.65, 0.7, \textbf{0.75}, 0.8, 0.85, 0.9, 0.95, 0.97, 0.99]
\end{itemize}
Note that we use these values for $\tau$ on all experiments with Heuristic for the corresponding datasets. The performance for this heuristic highly depends on careful tuning for the ratio $\tau$ when the memory size or memory selection method changes, as can be seen in in Figure \ref{fig:acc_over_replay_memory_size} and Table \ref{tab:results_memory_selection_methods}. 



\clearpage 



\section{Additional Experimental Results}\label{app:additional_experimental_results}

\begin{table}[t]
    \scriptsize
    \centering
    \caption{The ACC metrics and ranks from all environments and DQN seeds for test on FashionMNIST experiment. We show the ACC (\%) and the rank in parenthesis ($\cdot$) for each method and seed. Note that we have copied the ACCs for ETS, Heuristic1, and Heuristic2 since they do not have a seed for initializing the DQN or setting the random replay schedule as in Random.  \MK{Use longtable package to split table across pages}}
    \begin{tabular}{l l c c c c c}
        \toprule
        & & \multicolumn{5}{c}{{\bf Methods}} \\
        \cmidrule{3-7}
        {\bf Environment} & {\bf DQN Seed} & Random & ETS & Heuristic1 & Heuristic2 & DQN \\
        \midrule 
         \multirow{5}{*}{Env. 1} & Seed 1 & 96.66 (4) & 94.17 (5) & 98.52 (2) & 98.69 (1) & 97.60 (3)  \\
         \cmidrule{2-7}
         & Seed 2 & 90.16 (5) & 94.17 (4) & 98.52 (2) & 98.69 (1) & 96.83 (3)  \\
         \cmidrule{2-7}
         & Seed 3 & 97.89 (3) & 94.17 (5) & 98.52 (2) & 98.69 (1) & 96.34 (4)  \\
         \cmidrule{2-7}
         & Seed 4 & 97.20 (4) & 94.17 (5) & 98.52 (2) & 98.69 (1) & 97.37 (3) \\
         \cmidrule{2-7}
         & Seed 5 & 97.58 (3) & 94.17 (5) & 98.52 (2) & 98.69 (1) & 96.37 (4)  \\
        \midrule
         \multirow{5}{*}{Env. 2} & Seed 1 & 92.84 (4) & 94.10 (2) & 94.55 (1) & 90.03 (5) & 93.92 (3)  \\
         \cmidrule{2-7}
         & Seed 2 & 95.06 (2) & 94.10 (4) & 94.55 (3) & 90.03 (5) & 95.34 (1)   \\
         \cmidrule{2-7}
         & Seed 3 & 93.72 (4) & 94.10 (3) & 94.55 (2) & 90.03 (5) & 95.34 (1)  \\
         \cmidrule{2-7}
         & Seed 4 & 94.93 (2) & 94.10 (4) & 94.55 (3) & 90.03 (5) & 95.17 (1) \\
         \cmidrule{2-7}
         & Seed 5 & 84.76 (5) & 94.10 (3) & 94.55 (2) & 90.03 (4) & 95.34 (1)  \\
        \midrule
         \multirow{5}{*}{Env. 3} & Seed 1 & 96.07 (2) & 95.95 (4) & 96.06 (3) & 84.75 (5) & 97.12 (1)  \\
         \cmidrule{2-7}
         & Seed 2 & 95.55 (4) & 95.95 (3) & 96.06 (2) & 84.75 (5) & 97.12 (1)  \\
         \cmidrule{2-7}
         & Seed 3 & 95.57 (4) & 95.95 (3) & 96.06 (2) & 84.75 (5) & 96.94 (1)  \\
         \cmidrule{2-7}
         & Seed 4 & 97.04 (1) & 95.95 (3) & 96.06 (2) & 84.75 (5) & 86.20 (4)  \\
         \cmidrule{2-7}
         & Seed 5 & 95.13 (4) & 95.95 (3) & 96.06 (2) & 84.75 (5) & 96.69 (1)  \\
         \midrule 
         \multirow{5}{*}{Env. 4} & Seed 1 & 94.40 (5) & 96.71 (3) & 97.31 (2) & 96.28 (4) & 98.00 (1)   \\
         \cmidrule{2-7}
         & Seed 2 & 98.94 (1) & 96.71 (4) & 97.31 (3) & 96.28 (5) & 97.87 (2)   \\
         \cmidrule{2-7}
         & Seed 3 & 97.05 (3) & 96.71 (4) & 97.31 (2) & 96.28 (5) & 97.96 (1)   \\
         \cmidrule{2-7}
         & Seed 4 & 95.35 (5) & 96.71 (3) & 97.31 (1) & 96.28 (4) & 96.99 (2)  \\
         \cmidrule{2-7}
         & Seed 5 & 95.64 (4) & 96.71 (2) & 97.31 (1) & 96.28 (3) & 94.87 (5)  \\
         \midrule 
         \multirow{5}{*}{Env. 5} & Seed 1 & 83.91 (4) & 81.24 (5) & 94.25 (2) & 93.58 (3) & 94.39 (1) \\
         \cmidrule{2-7}
         & Seed 2 & 88.75 (4) & 81.24 (5) & 94.25 (1) & 93.58 (3) & 94.03 (2)   \\
         \cmidrule{2-7}
         & Seed 3 & 83.54 (4) & 81.24 (5) & 94.25 (1) & 93.58 (3) & 93.60 (2)   \\
         \cmidrule{2-7}
         & Seed 4 & 84.34 (4) & 81.24 (5) & 94.25 (1) & 93.58 (2) & 90.82 (3)  \\
         \cmidrule{2-7}
         & Seed 5 & 88.29 (3) & 81.24 (5) & 94.25 (1) & 93.58 (2) & 87.85 (4)  \\
         \midrule 
         \multirow{5}{*}{Env. 6} & Seed 1 & 82.93 (5) & 87.94 (4) & 89.35 (2) & 89.55 (1) & 89.22 (3)  \\
         \cmidrule{2-7}
         & Seed 2 & 91.19 (1) & 87.94 (5) & 89.35 (4) & 89.55 (3) & 91.05 (2)  \\
         \cmidrule{2-7}
         & Seed 3 & 90.30 (1) & 87.94 (5) & 89.35 (3) & 89.55 (2) & 88.46 (4)   \\
         \cmidrule{2-7}
         & Seed 4 & 89.81 (1) & 87.94 (5) & 89.35 (3) & 89.55 (2) & 89.23 (4)   \\
         \cmidrule{2-7}
         & Seed 5 & 88.02 (4) & 87.94 (5) & 89.35 (3) & 89.55 (2) & 91.48 (1)  \\
         \midrule 
         \multirow{5}{*}{Env. 7} & Seed 1 & 86.13 (5) & 91.10 (3) & 95.16 (2) & 95.56 (1) & 89.18 (4)  \\
         \cmidrule{2-7}
         & Seed 2 & 94.04 (3) & 91.10 (4) & 95.16 (2) & 95.56 (1) & 77.08 (5)\\
         \cmidrule{2-7}
         & Seed 3 & 94.82 (3) & 91.10 (4) & 95.16 (2) & 95.56 (1) & 80.92 (5)   \\
         \cmidrule{2-7}
         & Seed 4 & 94.67 (3) & 91.10 (4) & 95.16 (2) & 95.56 (1) & 84.96 (5)   \\
         \cmidrule{2-7}
         & Seed 5 & 91.70 (4) & 91.10 (5) & 95.16 (2) & 95.56 (1) & 93.36 (3)  \\
         \midrule
         \multirow{5}{*}{Env. 8} & Seed 1 & 90.10 (4) & 90.70 (3) & 94.89 (1) & 91.25 (2) & 89.27 (5)  \\
         \cmidrule{2-7}
         & Seed 2 & 87.13 (5) & 90.70 (4) & 94.89 (1) & 91.25 (3) & 93.11 (2)  \\
         \cmidrule{2-7}
         & Seed 3 & 94.44 (2) & 90.70 (5) & 94.89 (1) & 91.25 (4) & 94.00 (3)    \\
         \cmidrule{2-7}
         & Seed 4 & 94.89 (2) & 90.70 (5) & 94.89 (1) & 91.25 (4) & 91.39 (3)   \\
         \cmidrule{2-7}
         & Seed 5 & 93.74 (2) & 90.70 (4) & 94.89 (1) & 91.25 (3) & 89.48 (5)  \\
         \midrule
         \multirow{5}{*}{Env. 9} & Seed 1 & 98.10 (1) & 97.67 (2) & 97.51 (3) & 96.41 (4) & 88.08 (5) \\
         \cmidrule{2-7}
         & Seed 2 & 94.00 (5) & 97.67 (1) & 97.51 (2) & 96.41 (3) & 94.01 (4)  \\
         \cmidrule{2-7}
         & Seed 3 & 97.93 (1) & 97.67 (2) & 97.51 (3) & 96.41 (4) & 87.87 (5)    \\
         \cmidrule{2-7}
         & Seed 4 & 96.37 (4) & 97.67 (1) & 97.51 (2) & 96.41 (3) & 89.83 (5)   \\
         \cmidrule{2-7}
         & Seed 5 & 97.38 (3) & 97.67 (1) & 97.51 (2) & 96.41 (5) & 97.29 (4)  \\
         \midrule
         \multirow{5}{*}{Env. 10 } & Seed 1 & 96.84 (2) & 73.10 (5) & 91.10 (4) & 93.58 (3) & 97.25 (1) \\
         \cmidrule{2-7}
         & Seed 2 & 88.82 (4) & 73.10 (5) & 91.10 (2) & 93.58 (1) & 89.83 (3)   \\
         \cmidrule{2-7}
         & Seed 3 & 86.83 (4) & 73.10 (5) & 91.10 (3) & 93.58 (2) & 96.00 (1)    \\
         \cmidrule{2-7}
         & Seed 4 & 84.38 (4) & 73.10 (5) & 91.10 (3) & 93.58 (2) & 95.48 (1)   \\
         \cmidrule{2-7}
         & Seed 5 & 90.72 (4) & 73.10 (5) & 91.10 (3) & 93.58 (2) & 96.04 (1)   \\
        \bottomrule
    \end{tabular}
    \label{tab:accuracy_per_environment_fashionmnist}
\end{table}















 
























 












\clearpage

\section{Additional Figures}\label{app:additional_figures}

\paragraph{Memory usage.} We visualize the memory usage for Permuted MNIST and the 20-task datasets Split CIFAR-100 and Split miniImagenet in Figure \ref{fig:memory_usage_10_and_20task_datasets} for our method and the baselines used in the experiment in Section \ref{sec:efficiency_of_replay_scheduling}. Our method uses a fixed memory size of $M=50$ samples for replay on all three datasets. The memory size capacity for our method is reached after learning task 6 and task 11 on the Permuted MNIST and the 20-task datasets respectively, while the baselines continue incrementing their replay memory size.

\begin{figure}[h]
  \centering
  \setlength{\figwidth}{0.78\textwidth}
  \setlength{\figheight}{.22\textheight}
  



\pgfplotsset{compat=1.11,
    /pgfplots/ybar legend/.style={
    /pgfplots/legend image code/.code={%
       \draw[##1,/tikz/.cd,yshift=-0.25em]
        (0cm,0cm) rectangle (3pt,0.8em);},
   },
}
\pgfplotsset{every axis title/.append style={at={(0.5,0.90)}}}
\pgfplotsset{every tick label/.append style={font=\tiny}}
\pgfplotsset{every major tick/.append style={major tick length=2pt}}
\pgfplotsset{every minor tick/.append style={minor tick length=1pt}}
\pgfplotsset{every axis x label/.append style={at={(0.5,-0.14)}}}
\pgfplotsset{every axis y label/.append style={at={(-0.10,0.5)}}}

\begin{tikzpicture}
\tikzstyle{every node}=[font=\footnotesize]
\definecolor{color0}{rgb}{0.12156862745098,0.466666666666667,0.705882352941177}
\definecolor{color1}{rgb}{1,0.498039215686275,0.0549019607843137}
%\definecolor{color2}{rgb}{0.172549019607843,0.627450980392157,0.172549019607843}

\begin{groupplot}[group style={group size= 1 by 2, horizontal sep=0.9cm, vertical sep=1.2cm}]

\nextgroupplot[title={Memory Usage for 10-task Permuted MNIST },
ybar,
bar width = 4pt,
height=\figheight,
legend cell align={left},
legend columns=2,
legend style={
  fill opacity=0.8,
  draw opacity=1,
  text opacity=1,
  at={(0.03,0.82)},
  anchor=south west,
  draw=white!80!black
},
minor xtick={},
minor ytick={},
tick align=outside,
tick pos=left,
width=\figwidth,
x grid style={white!69.0196078431373!black},
xlabel={Task},
%x label style={at={(axis description cs:0.5,-0.3)},anchor=north},
xmin=1.5, xmax=10.5,
xtick style={color=black},
xtick={2,3,4,5,6,7,8,9,10},
xticklabels={2,3,4,5,6,7,8,9,10},
y grid style={white!69.0196078431373!black},
ymajorgrids,
ylabel={\#Replayed Samples},
ymin=0.0, ymax=102,
ytick style={color=black},
ytick={0,10,20,30,40,50,60,70,80,90,100},
yticklabels={0,10,20,30,40,50,60,70,80,90,100}
]

\addplot [draw=black, fill=color0] coordinates {
(2,10)
(3,20)
(4,30)
(5,40)
(6,50)
(7,50)
(8,50)
(9,50)
(10,50)
};\addlegendentry{Ours}

\addplot [draw=black, fill=color1] coordinates {
(2,10)
(3,20)
(4,30)
(5,40)
(6,50)
(7,60)
(8,70)
(9,80)
(10,90)
};\addlegendentry{Baselines}

\nextgroupplot[title={Memory Usage for the 20-task Datasets},
ybar,
bar width = 4pt,
height=\figheight,
legend cell align={left},
legend columns=2,
legend style={
  fill opacity=0.8,
  draw opacity=1,
  text opacity=1,
  at={(0.03,0.75)},
  anchor=south west,
  draw=white!80!black
},
minor xtick={},
minor ytick={},
tick align=outside,
tick pos=left,
width=\figwidth,
x grid style={white!69.0196078431373!black},
xlabel={Task},
%x label style={at={(axis description cs:0.5,-0.3)},anchor=north},
xmin=1.5, xmax=20.5,
xtick style={color=black},
xtick={2,3,4,5,6,7,8,9,10,11,12,13,14,15,16,17,18,19,20},
xticklabels={2,3,4,5,6,7,8,9,10,11,12,13,14,15,16,17,18,19,20},
y grid style={white!69.0196078431373!black},
ymajorgrids,
ylabel={\#Replayed Samples},
ymin=0.0, ymax=102,
ytick style={color=black},
ytick={0,10,20,30,40,50,60,70,80,90,100},
yticklabels={0,10,20,30,40,50,60,70,80,90,100}
]

\addplot [draw=black, fill=color0] coordinates {
(2,5)
(3,10)
(4,15)
(5,20)
(6,25)
(7,30)
(8,35)
(9,40)
(10,45)
(11,50)
(12,50)
(13,50)
(14,50)
(15,50)
(16,50)
(17,50)
(18,50)
(19,50)
(20,50)
};\addlegendentry{Ours}

\addplot [draw=black, fill=color1] coordinates {
(2,5)
(3,10)
(4,15)
(5,20)
(6,25)
(7,30)
(8,35)
(9,40)
(10,45)
(11,50)
(12,55)
(13,60)
(14,65)
(15,70)
(16,75)
(17,80)
(18,85)
(19,90)
(20,95)
};\addlegendentry{Baselines}

\end{groupplot}

\end{tikzpicture}
  %\vspace{-3mm}
  \caption{Number of replayed samples per task for 10-task Permuted MNIST (top) and the 20-task datasets in the experiment in Section \ref{sec:efficiency_of_replay_scheduling}. The fixed memory size $M=50$ for our method is reached after learning task 6 and task 11 on the Permuted MNIST and the 20-task datasets respectively, while the baselines continue incrementing their number of replay samples per task.}
  \label{fig:memory_usage_10_and_20task_datasets}
\end{figure}





