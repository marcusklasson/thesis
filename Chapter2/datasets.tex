
\section{Datasets for Object Recognition} % Visual Recognition?
\label{sec:datasets_for_object_recognition}

The increasing accessibility of large-scale image datasets have enhanced the possibility for applying machine learning in various applications for visual recognition. One of the most well-known datasets in computer vision the is Imagenet database~\cite{deng2009imagenet} introduced in 2009 which is still used for benchmarking models on large-scale image classification. Imagenet was collected through image search results from the Internet and then further assessed by crowdsourcers from Amazon Mechanical Turk to achieve higher quality of the collected images as well as labeling them. There has been several efforts to create more large-scale vision datasets, e.g., Pascal VOC~\cite{everingham2015pascal}, Microsoft Coco~\cite{lin2014microsoft}, Visual genome~\cite{krishna2017visual}, which in addition to object classes include object attributes, bounding boxes for detecting objects, and pixel-wise segmentation masks. Additionally, there exists datasets where images have been combined with text descriptions of things that are present and where they are located in relation to each other for image captioning and visual question answering tasks, e.g., Flickr30k~\cite{young2014image}, VQA~\cite{antol2015vqa}, GQA~\cite{hudson2019gqa}, as well as the already mentioned Microsoft Coco and Visual Genome. These publicly accessible datasets are one of the major reasons for enabling machine learning and computer vision research at a large scale which opens up for developing products that can be deployed on everyday products, such as mobile phones. 


In order to deploy machine learning models in real-world scenarios, we first need training data that are close to the occurring events specific for the application where the model will be used. 
But even if we can train a model on huge amounts of images, it is extremely challenging to provide examples that cover all possible scenario that can happen or every different shape and color an object can have. Furthermore, other circumstances such as lighting conditions, occlusions, and other objects in the surroundings of objects of interest which can be hard to control can also make the recognition performance less accurate. This is critical when training models where the user is strongly relying on the recognition performance, such as in assistive vision systems for blind or low-vision people. As many of the popular image datasets for computer vision contains images from the Internet, there can be a large gap between the training images and the images that will be seen after deployment as Internet images often have good conditions and the objects are fully visible and centered. However, when the model would be used in the real-world, the objects could be occluded or not fully visible in the image and be disturbed by objects in the background. Therefore, it can be critical for the performance and robustness of the model to train it on images that are tailored for the scenarios where it will be applied. 






