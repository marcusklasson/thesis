
\section{Datasets for Object Recognition} % Visual Recognition?
\label{sec:datasets_for_object_recognition}

The increasing accessibility of large-scale image datasets have enhanced the possibility for applying machine learning in various applications for visual recognition. One of the most well-known datasets in computer vision the is Imagenet database~\cite{deng2009imagenet} introduced in 2009 which is still used for benchmarking models on large-scale image classification. Imagenet was collected through image search results from the Internet and then further assessed by crowdsourcers from Amazon Mechanical Turk to achieve higher quality of the collected images as well as labeling them. There has been several efforts to create more large-scale vision datasets, e.g., Pascal VOC~\cite{everingham2015pascal}, Microsoft coco~\cite{lin2014microsoft}, Visual genome~\cite{krishna2017visual}, which in addition to object classes include object attributes, bounding boxes for detecting objects, and pixel-wise segmentation masks. Additionally, there exists datasets where images have been combined with text descriptions of things that are present and where they are located in relation to each other for image captioning and visual question answering tasks, e.g., Flickr30k~\cite{young2014image}, VQA~\cite{antol2015vqa}, GQA~\cite{hudson2019gqa}, as well as the already mentioned Microsoft coco and Visual Genome. These publicly accessible datasets are one of the major reasons for enabling machine learning and computer vision research at a large scale which opens up for developing products that can be deployed on everyday products, such as mobile phones.  

Approximating the real-world with benchmark datasets. 