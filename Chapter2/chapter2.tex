%*******************************************************************************
%*********************************** Second Chapter ****************************
%*******************************************************************************

\chapter{Background}
\label{chap:background}

The goal with this thesis is to provide machine learning methods for recognizing objects from images. Machine Learning (ML) is a field within Artificial Intelligence where computer programs learns from experiences how to make predictions in new situations. There are three essential parts to enable the computer to make predictions with ML. Firstly, we need data representing the scenarios where we have objects that we wish to predict what they are. Secondly, we need a model that learns how to make the decisions based on the provided data. Thirdly, we need a learning algorithm for fitting the model to the data we have such that good and sensible decisions can be made on future data. ML has proven to be successful on various types of data, including, images and video, text, and audio, and there exists many different kinds of models and algorithms for learning decision-making from data. 

One of the main challenges with ML is to obtain datasets representing the events that the model should make predictions for. 

%In this thesis, 
%we will focus on the first and second requirements where we need to collect datasets that are useful for object recognition in real-world scenarios as well as designing models that can make accurate predictions in new scenarios. 



%Data -> Modeling -> Evaluation/Generalization

\section{Object Recognition Datasets}

\section{Modeling}

\subsection{Deep Learning}

\subsection{Multi-view Learning}

\subsection{Continual Learning}


