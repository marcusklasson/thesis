%*******************************************************************************
%*********************************** Second Chapter ****************************
%*******************************************************************************

\chapter{Background}
\label{chap:background}

The goal with this thesis is to provide machine learning methods for recognizing objects from images. Machine learning is a field within Artificial Intelligence where computer programs learns from experiences how to make predictions in new situations. There are three essential parts to enable the computer to make predictions with machine learning. Firstly, we need data representing the scenarios where we have objects that we wish to predict what they are. Secondly, we need a model that learns how to make the decisions based on the provided data. Thirdly, we need a learning algorithm for fitting the model to the data we have such that good and sensible decisions can be made on future data. Machine learning has proven to be successful on various types of data, including, images and video, text, and audio, and there exists many different kinds of models and algorithms for learning decision-making from data. 

One of the main goals with machine learning is to have models that generalize to unseen data and events. However, there are several challenges that has to be tackled to achieve this goal. The first challenge is to obtain datasets that represent the events that the model should make predictions for. Machine learning models often require vast amounts of examples to learn from, and also, the examples should be annotated with some information describing each example in order to ease the learning. But even if we have large datasets, we must have models that have the capacity of preserving the knowledge gained from the dataset. Furthermore, we must have algorithms that can train the model from the huge amount of data in computationally efficient both time- and processing-wise. Especially for visual data, it has become much cheaper to obtain vast amounts of images and videos from the internet. Occasionally, these can be annotated through search words or, alternatively, from crowdsourcing. Moreover, computational power has also become cheaper through smaller and more efficient micro-processors, semi-conductors, and cloud computing. Deep learning~\cite{goodfellow2016deep} is a class of machine learning models based on neural networks that are capable of learning from large and high-dimensional datasets due to their capacity. They are trained using an optimization algorithm called Stochastic Gradient Descent (SGD)~\cite{bottou2010large} which works well for large-scale data and can be applied on graphical processing units (GPUs) with recent machine learning programming libraries, such as TensorFlow, PyTorch, and Jax. However, deep learning still faces lots of challenges in generalization, especially when they are applied in environments that were not present in the training data, and it is still an open research problem on how to make them generalize better. 

There are several approaches for enabling better generalization for deep learning models. A good start is to collect datasets that are similar and represent the events in the environment where the model will be deployed. Related to this, one can also collect different data types from various modalities, such as visual signals in the form of images and video as well as natural language which can be written or spoken, if these are available in the data collection process and are sensible for the task to be solved. Multimodal machine learning opens up the possibility of learning correspondences between the different data types to gain better understanding of the phenomenon of interest~\cite{baltruvsaitis2018multimodal, xu2013survey}, which can help the model to be more accurate and robust. However, in order to enhance the utility of machine learning models, they should be capable of continuously updating their knowledge as many environments where object recognition is useful are ever-changing~\cite{delange2021continual, parisi2019continual}. We should build models that can add new objects of interest to recognize as well as delete concepts that are obsolete or non-relevant. It would also be useful if we could update models with personalized objects to recognize to narrow down the scope of items to recognize for object recognizers to make the tasks easier.  

In this chapter, we cover related works on datasets on object recognition both with image and text data in Section \ref{sec:datasets_for_object_recognition}. Next, we provide a description of machine learning, especially deep learning, models that were used in the included papers in Section \ref{sec:machine_learning} and \ref{sec:deep_learning}. In Section \ref{sec:continual_learning}, we discuss the setting of continual learning for updating the knowledge of machine learning models that is aiming to make the models capable of handling ever-changing environments as a step towards to enabling life-long learning. 

\section{Machine Learning Basics}

\section{Deep Learning}

\subsection{Autoencoders}

\section{Reinforcement Learning}

\subsection{Monte Carlo Tree Search}

%Data -> Modeling -> Evaluation/Generalization

% Datasets for object recognition
%\section{Datasets for Object Recognition} % Visual Recognition?
\label{sec:datasets_for_object_recognition}

The increasing accessibility of large-scale image datasets have enhanced the possibility for applying machine learning in various applications for visual recognition. One of the most well-known datasets in computer vision the is Imagenet database~\cite{deng2009imagenet} introduced in 2009 which is still used for benchmarking models on large-scale image classification. Imagenet was collected through image search results from the Internet and then further assessed by crowdsourcers from Amazon Mechanical Turk to achieve higher quality of the collected images as well as labeling them. There has been several efforts to create more large-scale vision datasets, e.g., Pascal VOC~\cite{everingham2015pascal}, Microsoft COCO~\cite{lin2014microsoft}, Visual genome~\cite{krishna2017visual}, which in addition to object classes include object attributes, bounding boxes for detecting objects, and pixel-wise segmentation masks. Additionally, there exists datasets where images have been combined with text descriptions of things that are present and where they are located in relation to each other for image captioning and visual question answering tasks, e.g., Flickr30k~\cite{young2014image}, VQA~\cite{antol2015vqa}, GQA~\cite{hudson2019gqa}, and Microsoft COCO Captions~\cite{chen2015microsoft}. These publicly accessible datasets are one of the major reasons for enabling machine learning and computer vision research at a large scale which opens up for developing products that can be deployed on everyday products, such as mobile phones. 


In order to deploy machine learning models in real-world scenarios, we first need training data that are close to the occurring events specific for the application where the model will be used. 
But even if we can train a model on huge amounts of images, it is extremely challenging to provide examples that cover all possible scenario that can happen or every different shape and color an object can have. Furthermore, other circumstances such as lighting conditions, occlusions, and other objects in the surroundings of objects of interest which can be hard to control can also make the recognition performance less accurate. This is critical when training models where the user is strongly relying on the recognition performance, such as in assistive vision systems for blind or low-vision people. As many of the popular image datasets for computer vision contains images from the Internet, there can be a large gap between the training images and the images that will be seen after deployment as Internet images often have good conditions and the objects are fully visible and centered. However, when the model would be used in the real-world, the objects could be occluded or not fully visible in the image and be disturbed by objects in the background. Therefore, it can be critical for the performance and robustness of the model to train it on images that are tailored for the scenarios where it will be applied. 

There have been several attempts to build datasets with images collected by visually impaired to obtain more realistic datasets with potential scenarios. VizWiz~\cite{gurari2018vizwiz} is one of the first large-scale image datasets with mobile phone images taken by blind people where the user have asked a question about each image that are answered by crowdworkers. This dataset is very challenging as questions asked by the collectors can be unanswerable since the objects can be occluded or even out of frame. The ORBIT dataset~\cite{massiceti2021orbit} is a more recent dataset that have built a dataset of videos from blind or low-vision people of their personal items, e.g., keys, wallets, remote controls, to enable few-shot learning of personalizable object recognizers. This dataset is the first to contain video recordings which potentially can be more user friendly as this allows the user to rotate the items that could yield more accurate performance. 




% Machine Learning introduction
%\section{Machine Learning} 
\label{sec:machine_learning}

In this section, we give a brief introduction to basics in machine learning and some mathematical notation that will be used in this chapter. The overall goal is to learn some phenomena from a set of $N$ data points $X = \{\vx_1, \dots, \vx_N\}$ where $\vx_i \in \R^{D}$ has $D$ dimensions for $1 \leq i \leq N$. The learning part is referred to as the training phase where we the goal is to fit the machine learning model to the data points, or observations, $X$. The objective of the training phase is determined by what the target task is tasks that we wish to solve using the model. Next, we give a brief description of three paradigms in machine learning with different end goals depending on what we wish to explore.

\paragraph{Supervised Learning.} Many types of data have a corresponding target $\vy$ that explains the content of each data point. Cases where the target comes from a discrete number of classes and the goal is to determine which class that the data point corresponds to are called classification problem. An classic example of a classification problem is distinguishing whether there is a cat or dog in images. If the target $\vy$ is a continuous valued variable and the goal is to predict this value, then we have a regression problem, where an example can be predicting the outdoor temperature tomorrow given the current temperature. The goal for both of these problems is to learn a function $f(\vx)$ that can classify or predict the given target data as accurately as possible. 

\paragraph{Unsupervised Learning.} For some problems, we may only have the data points for our disposal where we are interested in finding some patterns in the given data. The goal in unsupervised learning problems can be to discover similar groups with clustering techniques, or estimating the distribution of the data known as density estimation. The practitioner typically needs to define some assumptions about the data, e.g., how the similarity between two data points should be measured, before we can execute the algorithm for discovering the patterns. 

\paragraph{Reinforcement Learning.} In this paradigm, we are concerned with decision-making where the goal is to take actions that maximize some reward. The decision-making is modeled by a policy which bases the action selection on observations that are collected by interacting with the task environment through the selected actions. One main challenge is how to handle the trade-off between exploration of different actions in new situations and exploitation by selecting actions where the agent already has experienced good reward signals. Furthermore, the reward signal can be received either in dense or sparse forms, where sparse rewards are typically more challenging to learn from and are less sample-efficient. 


%%%% THINKING ABOUT MOVIND DATASETS TO HERE AND STARTING WITH ML BACKGROUND

% Deep Learning introduction
%\section{Deep Learning} 
\label{sec:deep_learning}

Neural networks is a class of machine learning models which popularity have grown immensely due to their ability to learn from large and high-dimensional datasets. Moreover, neural networks have been successfully applied in various number of fields in computer vision~\cite{he2016deep, krizhevsky2012imagenet}, natural language processing~\cite{devlin2018bert}, and reinforcement learning~\cite{mnih2015human, silver2016mastering}. These models are constructed by stacking layers of parameters that extract representations of the input data until reaching the final layer that outputs the target answer from the queried input. For example, the operation for passing the input data $\vx$ through the first layer can be denoted as the matrix multiplication $\vh = \mW_1\vx$ where $\mW_1$ are the weights of the first layer. An essential part for enabling neural networks to learn more interesting non-linear functions is to add an activation function right after the matrix multiplication of each layer, otherwise the neural network would only be capable of learning linear functions. A common activation function is the $a(\vx) = \max(0, \vx)$, or the so called Rectified Linear Unit (ReLU) activation, which outputs $\vx$ when $\vx > 0$ or otherwise zero. Stacking two layers together in a neural network with such activation between then becomes $\vh = \mW_2\max(0,  \mW_1\vx)$. In classification problems, the output layer outputs the score for which class the input $\vx$ could belong to. The weight parameters $\mW_1, \mW_2$ are learned with SGD, and their gradients are derived using the chain rule and computed using backpropagation (see \cite{goodfellow2016deep} for an introduction to backpropagation).  

\MK{Feel like I would like to include some brief info on CNNs and RNNs, but just a paragraph long for each. Maybe also describe the Autoencoders a bit. }

\subsection{Variational Autoencoders}
\label{sec:variational_autoencoders}

Generative models are used for approximating a data distribution $p_{data}$ from a given set of samples $\gD$. In most cases, we parameterize the distribution with $\vtheta$ and learn the parameters from the given data by minimizing some distance metric between the estimated distribution $p_{\vtheta}$ and $p_{data}$. In deep generative models, the distribution $p_{\theta}$ is parameterized by a neural network where the parameters $\vtheta$ represent the weights and biases in the network. These models can be broadly divided into three classes of models, namely Variational Autoencoders (VAEs)~\cite{kingma2013auto}, Generative Adversarial Networks (GANs)~\cite{goodfellow2014generative}, and Normalizing Flows~\cite{rezende2015variational}. Their commonality is that they are based on latent variable models where it is assumed that the observed data is generated from some hidden process from a simpler distribution than $p_{data}$. However, which class of models to select depends on the application and the goals with the task. In this thesis, we focus on VAEs because of their capability of learning lower-dimensional representations.

A key ingredient in learning generative models is to introduce a latent variable $\vz$ where we assume that $\vz$ is related to the observed variable $\vx$ through the data generation process. We can still estimate the parameterized distribution when introducing the latent variables since
\begin{align}\label{eq:marginal_density}
	p_{\vtheta}(\vx) = \int p_{\vtheta}(\vx, \vz) \,d\vz = \int p_{\vtheta}(\vx | \vz) p(\vz) \,d\vz,
\end{align}
where we incorporate $\vz$ to obtain the joint distribution $p_{\vtheta}(\vx, \vz)$ and use the chain rule for probabilities to obtain the likelihood $p_{\vtheta}(\vx | \vz)$ and the prior distribution $p(\vz)$. The prior $p(\vz)$ is where we can define our assumption of how the underlying hidden reasons are distributed by selecting a simple and well-known distribution for this space, e.g., a Gaussian distribution. The goal with latent variable models is often to compute the posterior $p_{\vtheta}(\vz | \vx)$ over the latent variables given the data $\vx$ which can be done using Bayes' rule 
\begin{align}
	p_{\vtheta}(\vz | \vx) = \frac{p_{\vtheta}(\vx | \vz) p(\vz)}{p_{\vtheta}(\vx)}. 
\end{align}
Unfortunately, the evidence $p_{\vtheta}(\vx)$ is very hard to compute as it requires calculating the integral $\int p_{\vtheta}(\vx, \vz) \,d\vz$ over all dimensions of the latent variable space. To overcome this issue, we will propose a distribution $q_{\vz}$ that should approximate the true posterior $p_{\vtheta}(\vz | \vx)$. Variational Inference (VI)~\cite{blei2017variational, zhang2018advances} is a method for approximating probability densities through optimization which is used in VAEs for approximating the posterior. Next, we give a description of how the posterior is approximated.
%The approximation can either be done using Markov Chain Monte Carlo (MCMC)~\cite{andrieu2003introduction} or Variational Inference (VI)~\cite{blei2017variational, zhang2018advances}. In MCMC, the idea is to use an easy-to-sample proposal distribution to draw samples that are used approximating the target distribution, while VI is a method for approximating probabilty densities through optimization. 

The learning objective in VAEs is based on the derivation of the marginal density, or evidence, in Equation \ref{eq:marginal_density}. As the evidence is intractable to compute exactly, we will instead maximize a lower bound on the evidence that is called the Evidence Lower BOund (ELBO). The ELBO is derived in log-space such that we can apply Jensen's inequality to obtain the lower bound when we have introduced the approximate posterior $q_{\vphi}(\vz | \vx)$ parameterized by $\vphi$. We begin be deriving a general expression for the ELBO: 
\begin{align}\label{eq:elbo}
	\begin{split}
		\log p_{\vtheta}(\vx) & = \log \int p_{\vtheta}(\vx, \vz) \,d\vz \\ 
		& = \log \int \frac{q_{\vphi}(\vz | \vx) p_{\vtheta}(\vx , \vz)}{q_{\vphi}(\vz | \vx)} \,d\vz \\
		& = \log \E_{q_{\vphi}(\vz | \vx)}\left[ \frac{p_{\vtheta}(\vx , \vz)}{q_{\vphi}(\vz | \vx)}\right] \\
		& \geq \E_{q_{\vphi}(\vz | \vx)}\left[ \log \frac{p_{\vtheta}(\vx , \vz)}{q_{\vphi}(\vz | \vx)}\right] ,
	\end{split}
\end{align}
where we applied Jensen's inequality between the third and fourth line to obtain the ELBO. This expression can be derived further to obtain the common VAE objective
\begin{align}\label{eq:vae}
	\begin{split}
		\log p_{\vtheta}(\vx) & \geq \E_{q_{\vphi}(\vz | \vx)}\left[ \log \frac{p_{\vtheta}(\vx , \vz)}{q_{\vphi}(\vz | \vx)}\right] \\
		& = \E_{q_{\vphi}(\vz | \vx)}\left[ \log p_{\vtheta}(\vx | \vz) \right] - \KL[q_{\vphi}(\vz | \vx) \,||\, p(\vz)], 
	\end{split}
\end{align}
where the second term is the Kullback-Leibler (KL) divergence between the approximate posterior and the prior. The KL divergence can be computed exactly when selecting both $q_{\vphi}(\vz | \vx)$ and $p(\vz)$ to be Gaussian densities, while the expectation of the log-likelihood can be estimated with Monte Carlo approximation. Furthermore, the reparameterization trick~\cite{kingma2013auto, rezende2014stochastic} is used to enable computing the gradients through the sampling step when estimating the log-likelihood that can be used for backpropagation. 


\subsection{Multi-View Learning}

Humans make use of many different sensory signals to recognize objects in the world. Visual signals are rich in the sense they convey the identity of objects. But even without vision capabilities, we can use touch, smell, and sound to ease the recognition problem of a near item, or even text for describing the features of objects. Combining each of these signals should help us obtain better recognition performance as we are adding information. This is a popular approach in machine learning to improve the recognition performance by utilizing various \textit{modalities} and \textit{views}. The difference between a modality and a view is that modalities refers to the way in which something happens or is experienced~\cite{baltruvsaitis2018multimodal}, e.g., natural language that can be written or spoken and visual signals in the form of images and videos, while views can be broadly defined as any sensor stream of a scene or event~\cite{salzmann2010factorized}. Example of views can be images taken from different camera view points or different environments, as long as the camera sensors are different. Another example are written or spoken languages where each language is considered its own view. In the rest of this section, even if they are very similar, we will discuss multi-view learning rather than multimodal learning as we utilize images from two different views in Paper \ref{sec:paperB} when training classifiers. 
\MK{Add figure with examples of what a view is, I can take inspiration from Fig 1 in C. Xu Survey on MVL. }

Multi-view learning 







% Continual Learning introduction
%\section{Continual Learning in Neural Networks}
\label{sec:continual_learning}

A common assumption in machine learning is that the data comes from stationary distributions such that the properties of the data are the same over time. However, this assumption may complicate deployments in real-world applications where variations in the data is common and the data seen during deployment can be very different from the data used for training. Such models has to be capable of learning from experiences online and update its knowledge about new concepts that are encountered during the lifespan. Furthermore, as the model is operating in an online manner, the model has to adapt fast when learning new knowledge to minimize the time that the system has to be down during training. These requirements has lead to research in Continual Learning (CL)~\cite{delange2021continual, parisi2019continual} where the goal is to push neural networks closer to how humans and animals are learning various tasks during their lifespan.

Next, we introduce some notation of the problem setting in CL for image classification. We let a neural network $f_{\vtheta}$, parameterized by $\vtheta$, learn $T$ tasks sequentially given their corresponding task datasets $\gD_1, \dots, \gD_T$ arriving in order. The $t$-th dataset $\gD_t = \{(\vx_{t}^{(i)}, y_{t}^{(i)})\}_{i=1}^{N_{t}}$ consists of $N_t$ samples where $\vx_{t}^{(i)}$ and $y_{t}^{(i)}$ are the $i$-th data point and class label respectively. The training objective at task $t$ is given by 
\begin{align}
	\underset{\vtheta}{\text{min}} \sum_{i=1}^{N_t} \ell(f_{\vtheta}(\vx_t^{(i)}), y_{t}^{(i)}),
\end{align}
where $\ell(\cdot)$ is the loss function, e.g., cross-entropy loss in image classification problems. Since $\gD_t$ is only accessible at time step $t$, the network $f_{\vtheta}$ is at risk of \textit{catastrophically forgetting} the previous $t-1$ tasks when learning the current task. 
%Replay-based continual learning methods mitigate the forgetting of old tasks by storing old examples in an external replay memory, that is mixed with the current task dataset during training. Next, we describe our method for constructing this replay memory.  

There are several desired properties of CL systems mentioned in~\cite{schwarz2018progress} to enhance their utility. Firstly, they should be capable of retaining already existing knowledge to mitigate the catastrophic forgetting effect that neural networks are susceptible to when trained in CL settings. Secondly, we hope that the model can benefit from the previously learned tasks to ease the learning of new tasks, which is referred to as \textit{forward transfer}. Moreover, we wish that learning new tasks can have positive influence on the performance of old tasks which is called \textit{backward transfer}. It also needs to be \textit{scalable} to a large number of tasks that could be encountered during its life time. Finally, it should ideally be capable of learning without the need for task labels as there might be unclear task boundaries when the model is deployed. 

Research in CL can be divided into three main areas, namely regularization-based, architecture-based, and replay-based approaches. Regularization-based methods aim to mitigate catastrophic forgetting by protecting parameters influencing the predictive performance from wide changes and use the rest of the parameters for learning the new tasks~\cite{adel2019continual, chaudhry2018riemannian, kirkpatrick2017overcoming, li2017learning, nguyen2017variational, rannen2017encoder, schwarz2018progress, zenke2017continual}. Architecture-based methods isolate task-specific parameters by either increasing network capacity~\cite{rusu2016progressive, yoon2019scalable, yoon2017lifelong} or freezing parts of the network~\cite{mallya2018packnet, serra2018overcoming} to maintain good performance on previous tasks. 
Replay-based methods mix samples from old tasks with the current dataset to mitigate catastrophic forgetting, where the replay samples are either stored in an external memory~\cite{chaudhry2019tiny, hayes2020remind, isele2018selective, lopez2017gradient} or generated using a generative model~\cite{shin2017continual, van2018generative}. More recently, there has been work on compressing raw images to feature representations to increase the number of memory examples for replay~\cite{hayes2020remind, iscen2020memory, pellegrini2019latent} as well as applying ideas from contrastive learning to improve retaining past knowledge~\cite{cha2021co2l, mai2021supervised}. 
Regularization-based approaches and dynamic architectures have been combined with replay-based approaches to methods to overcome their limitations~\cite{chaudhry2018riemannian, chaudhry2018efficient, douillard2020podnet, ebrahimi2020adversarial, joseph2020meta, mirzadeh2020linear, nguyen2017variational, pan2020continual, pellegrini2019latent, rolnick2018experience, von2019continual}. In general, these methods share the goal of mitigating catastrophic forgetting in various ways and then they can have additional goals related to the other desired properties on forward- and backward transfer, task scalability, and avoiding the need for task labels. 






