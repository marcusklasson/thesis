%*******************************************************************************
%*********************************** Second Chapter ****************************
%*******************************************************************************

\chapter{Background}
\label{chap:background}

The goal with this thesis is to provide machine learning methods for recognizing objects from images. Machine Learning (ML) is a field within Artificial Intelligence where computer programs learns from experiences how to make predictions in new situations. There are three essential parts to enable the computer to make predictions with ML. Firstly, we need data representing the scenarios where we have objects that we wish to predict what they are. Secondly, we need a model that learns how to make the decisions based on the provided data. Thirdly, we need a learning algorithm for fitting the model to the data we have such that good and sensible decisions can be made on future data. ML has proven to be successful on various types of data, including, images and video, text, and audio, and there exists many different kinds of models and algorithms for learning decision-making from data. 

One of the main goals with ML is to have models that generalize to unseen data and events. However, there are several challenges that has to be tackled to achieve this goal. The first challenge is to obtain datasets that represent the events that the model should make predictions for. ML models often require vast amounts of examples to learn from, and also, the examples should be annotated with some information describing each example in order to ease the learning. But even if we have large datasets, we must have models that have the capacity of preserving the knowledge gained from the dataset. Furthermore, we must have algorithms that can train the model from the huge amount of data in computationally efficient both time- and processing-wise. Especially for visual data, it has become much cheaper to obtain vast amounts of images and videos from the internet. Occasionally, these can be annotated through search words or, alternatively, from crowdsourcing. Moreover, computational power has also become cheaper through smaller and more efficient micro-processors, semi-conductors, and cloud computing. Deep learning~\cite{goodfellow2016deep} is a class of ML models based on neural networks that are capable of learning from large and high-dimensional datasets due to their capacity. They are trained using an optimization algorithm called Stochastic Gradient Descent (SGD)~\cite{bottou2010large} which works well for large-scale data and can be applied on graphical processing units (GPUs) with recent ML programming libraries, such as TensorFlow, PyTorch, and Jax. However, deep learning still faces lots of challenges in generalization, especially when they are applied in environments that were not present in the training data, and it is still an open research problem on how to make them generalize better. 

Learn with multiple data types to learn more accurately and faster. Evaluated on more natural scenarios that are closer to how they would have been applied in the real-world. 

%In this thesis, 
%we will focus on the first and second requirements where we need to collect datasets that are useful for object recognition in real-world scenarios as well as designing models that can make accurate predictions in new scenarios. 



%Data -> Modeling -> Evaluation/Generalization

\section{Object Recognition Datasets}

\section{Modeling}

\subsection{Deep Learning}

\subsection{Multi-view Learning}

\subsection{Continual Learning}


