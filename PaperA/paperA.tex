


%\renewcommand{\thechapter}{A}  % use A, B, C for chapter numbers
%\renewcommand{\chaptername}{Paper} % A chapter is now called Paper
%\renewcommand\thesection{\arabic{section}}
%\renewcommand*{\thepage}{A\arabic{page}}
%\setcounter{page}{1}
%\setcounter{section}{0}

%%% Paper content

\chapter{
	\centering{A Hierarchical Grocery Store Image Dataset with Visual and Semantic Labels}
}\label{chap:paperA}
\chaptermark{Hierarchical Grocery Store Image Dataset}
\begin{comment}
\vspace{-1cm}
\large{\textbf{Marcus Klasson}} \\
\large{\textit{Robotics, Perception, and Learning, EECS}} \\
\large{\textit{KTH Royal Institute of Technology, Stockholm, Sweden}} \\

\noindent\large{\textbf{Cheng Zhang}} \\
\large{\textit{Microsoft Research}} \\
\large{\textit{Cambridge, United Kingdom}} \\

\noindent\large{\textbf{Hedvig Kjellstr\"{o}m}} \\
\large{\textit{Robotics, Perception, and Learning, EECS}} \\
\large{\textit{KTH Royal Institute of Technology, Stockholm, Sweden}} 
\end{comment}
\vspace{-5mm}
\begin{center}
\large{\textbf{Marcus Klasson$^{*}$, Cheng Zhang$^{\dagger}$, Hedvig Kjellström$^{*}$}} \\[2mm]
\small{$^{*}$KTH Royal Institute of Technology, Stockholm, Sweden} \\
\small{$^{\dagger}$Microsoft Research, Cambridge, United Kingdom} \\
\end{center}
%\vspace{5pt}


%%%%%%%%%%%%%%%%%%%%%%%%%%%%%%%%%%%%%%%%%%%%%%%%%%%%%%%%%%%%%%%%%%%%%%%%%%%%%%%%
%%%%%%%%%%%%%%%%%%%%%%%%%%%%%%%%%%%%%%%%%%%%%%%%%%%%%%%%%%%%%%%%%%%%%%%%%%%%%%%%
\begin{abstract}
	\noindent Image classification models built into visual support systems and other assistive devices need to provide accurate predictions about their environment. We focus on an application of assistive technology for people with visual impairments, for daily activities such as shopping or cooking. In this paper, we provide a new benchmark dataset for a challenging task in this application -- classification of fruits, vegetables, and refrigerated products, e.g. milk packages and juice cartons, in grocery stores. To enable the learning process to utilize multiple sources of structured information, this dataset not only contains a large volume of natural images but also includes the corresponding information of the product from an online shopping website. Such information encompasses the hierarchical structure of the object classes, as well as an iconic image of each type of object. This dataset can be used to train and evaluate image classification models for helping visually impaired people in natural environments. Additionally, we provide benchmark results evaluated on pretrained convolutional neural networks often used for image understanding purposes, and also a multi-view variational autoencoder, which is capable of utilizing the rich product information in the dataset.
\end{abstract}

%%% Contents

\section{Introduction}\label{paperD:sec:introduction}

Scheduling over which historical tasks to replay has been shown to mitigate catastrophic forgetting efficiently in continual learning (CL)~\citeD{D:aljundi2019online, D:klasson2021learn}. 
Such replay strategies are important in real-world applications with machine learning systems that needs to learn continuously from streaming data, wherein the incoming data is stored rather than deleted.
In such scenarios, re-training on all stored data is often prohibited due to limitations on computational resources and data transmission times. Therefore, we need a policy that schedules from immense amounts of recorded data which previously learned abilities to replay when learning new tasks. 

Although memory scheduling is essential for real-world CL, an efficient policy that schedules the time to replay which tasks is currently missing from the literature. For example, the policy in~\citeD{D:klasson2021learn} is searched after using multiple CL episodes which is prohibited in the CL setting. Moreover, the method in~\citeD{D:aljundi2019online} performs forward passes on all stored data for selecting the tasks to replay, which becomes challenging in scenarios where the amounts of historical data is huge. 
If we could learn the scheduling policy, we would want the possibility to transfer the policy to new CL scenarios without the need for additional training in the target domain. Ideally, the policy should be domain-agnostic such that it is capable of generalizing to any CL dataset unseen during training. Such policy could potentially enable replay methods to efficiently mitigate catastrophic forgetting in real-world CL applications where the number of seen classes exceeds the replay memory budget. 

In this paper, we propose a framework based on reinforcement learning (RL)~\citeD{D:sutton2018reinforcement} for learning a general replay scheduling policy that selects which tasks to replay at different times. We focus on the CL setting introduced in~\citeD{D:klasson2021learn} which simulates a scenario where a replay memory is sampled from a large historical data storage for mitigating catastrophic forgetting. The policy is learned from multiple episodes in a CL environment by selecting which tasks to replay to efficiently mitigate catastrophic forgetting in the CL network. 
Our goal is to learn a replay scheduling policy that generalizes to new CL scenarios without added computational cost. 
We enable the possibility for generalizing to new domains by using the intuition that the policy should replay tasks that are to be forgotten by the classifier. Hence, we provide the policy with the evaluated task performances of the classifier in the CL environment, such that the policy can select the tasks to efficiently retain overall CL performance.  
In summary, our contributions are:
\begin{itemize}[topsep=1pt,] %noitemsep]
	\item We take the first steps to enable replay scheduling in real-world CL scenarios by proposing an RL-based framework for learning policies that generalize across different CL environments (Section \ref{paperD:sec:methodology}). 
	\item We show that the learned policies can efficiently mitigate catastrophic forgetting in CL scenarios with new task orders and datasets unseen during training without added computational cost (Section \ref{paperD:sec:results_on_policy_generalization}).
\end{itemize}

\begin{comment}


\begin{figure*}[t] 
\centering
\begin{minipage}[t]{0.47\textwidth}
\centering
\subfigure[Royal Gala]{\label{subfig:real-image-a}\includegraphics[width=0.30\columnwidth]{PaperA/dataset-figure/Royal-Gala-Apple_84_crop.jpg}}~
\subfigure[Golden Delicious]{\label{subfig:real-image-c}\includegraphics[width=0.30\columnwidth]{PaperA/dataset-figure/Golden-Delicious-Apple_7.jpg}}~
\subfigure[Orange]{\label{subfig:real-image-f}\includegraphics[width=0.30\columnwidth]{PaperA/dataset-figure/Orange_025.jpg}}~ \\ %\vspace{0.2cm}
\subfigure[Aubergine]{\label{subfig:real-image-h}\includegraphics[width=0.30\columnwidth]{PaperA/dataset-figure/Aubergine_016.jpg}}~
\subfigure[Onion]{\label{subfig:real-image-j}\includegraphics[width=0.30\columnwidth]{PaperA/dataset-figure/Yellow-Onion_25.jpg}}~
\subfigure[Zucchini]{\label{subfig:real-image-l}\includegraphics[width=0.30\columnwidth]{PaperA/dataset-figure/Zucchini_015.jpg}}~ \\ %\vspace{0.2cm}
\subfigure[Apple Juice]{\label{subfig:real-image-n}\includegraphics[width=0.30\columnwidth]{PaperA/dataset-figure/Tropicana-Apple-Juice_16.jpg}}~
\subfigure[Milk Medium Fat]{\label{subfig:real-image-q}\includegraphics[width=0.30\columnwidth]{PaperA/dataset-figure/Arla-Milk-Medium-Fat_17.jpg}}~
\subfigure[Yogurt Natural]{\label{subfig:real-image-r}\includegraphics[width=0.30\columnwidth]{PaperA/dataset-figure/Arla-Natural-Yoghurt_031.jpg}}~
   \caption{Examples of natural images in our dataset, where each image have been taken inside a grocery store. Image examples of fruits, vegetables and refrigerated products are presented in each row respectively.
   }
\label{fig:PaperA/dataset-figure}
\end{minipage}
\hspace{10pt}
\begin{minipage}[t]{0.47\textwidth}
\vspace{0pt}
\centering
\subfigure[Royal Gala]{\label{subfig:clean-img-a}\includegraphics[width=0.30\columnwidth]{PaperA/clean-image-figure/Royal-Gala-Apple_Clean.jpg}}~
\subfigure[Golden Delicious]{\label{subfig:clean-img-c}\includegraphics[width=0.30\columnwidth]{PaperA/clean-image-figure/Golden-Delicious-Apple_Clean.jpg}}~
\subfigure[Orange]{\label{subfig:clean-img-f}\includegraphics[width=0.30\columnwidth]{PaperA/clean-image-figure/Orange_Clean.jpg}}~ \\ %\vspace{-0.2cm}
\subfigure[Aubergine]{\label{subfig:clean-img-h}\includegraphics[width=0.30\columnwidth]{PaperA/clean-image-figure/Aubergine_Clean.jpg}}~
\subfigure[Onion]{\label{subfig:clean-img-j}\includegraphics[width=0.30\columnwidth]{PaperA/clean-image-figure/Yellow-Onion_Clean.jpg}}~
\subfigure[Zucchini]{\label{subfig:clean-img-l}\includegraphics[width=0.30\columnwidth]{PaperA/clean-image-figure/Zucchini_Clean.jpg}}~ \\ %\vspace{-0.2cm}
\subfigure[Apple Juice]{\label{subfig:clean-img-n}\includegraphics[width=0.30\columnwidth]{PaperA/clean-image-figure/Tropicana-Apple-Juice_Clean.jpg}}~
\subfigure[Milk Medium Fat]{\label{subfig:clean-img-q}\includegraphics[width=0.30\columnwidth]{PaperA/clean-image-figure/Arla-Milk-Medium-Fat_Clean.jpg}}~
\subfigure[Yogurt Natural]{\label{subfig:clean-img-r}\includegraphics[width=0.30\columnwidth]{PaperA/clean-image-figure/Arla-Natural-Yoghurt_Clean.jpg}}~
   \caption{Examples of iconic images downloaded from a grocery shopping website, which corresponds to the target items in the images in Figure \ref{fig:PaperA/dataset-figure}.}
\label{fig:PaperA/clean-image-figure}
\end{minipage}
\end{figure*}
\end{comment}

\section{Related Work}\label{sec:related-work}

Many popular image datasets have been collected by downloading images from the web \citeA{deng2009imagenetPaperA,Everingham2010pascal,Gebru2017FineGrainedCD, griffin2007caltech256,Krizhevsky2009cifar100,Lin2014MicrosoftCoco, song2016deep, welinder2010birds,xiao2010sundatabase}. 
If the dataset contains a large amount of images, it is convenient to make use of crowdsourcing to get annotations for recognition tasks \citeA{deng2009imagenetPaperA,Krizhevsky2009cifar100,liu2015faceattributes}. For some datasets, the crowdsourcers are also asked to put bounding boxes around the object to be labeled for object detection tasks \citeA{Everingham2010pascal,Gebru2017FineGrainedCD,welinder2010birds}. In \citeA{griffin2007caltech256} and \citeA{Krizhevsky2009cifar100}, the target objects are usually centered and takes up most content of the image itself. Another significant characteristic is that web images usually are biased in the sense that they have been taken with the object focus in mind; they have good lighting settings and are typically clean from occlusions, since the collectors have used general search words for the object classes, e.g. \textit{car}, \textit{horse}, or \textit{apple}.

Some datasets include additional information about the images beyond the single class label, e.g. text descriptions of what is present in the image and bounding boxes around objects. These datasets can be used in several different computer vision tasks, such as image classification, object detection, and image segmentation. Structured labeling is another important property of a dataset, which provides flexibility when classifying images. In  \citeA{Gebru2017FineGrainedCD,Lin2014MicrosoftCoco}, all of these features exist and moreover they include reference images to each object class, which in \citeA{Lin2014MicrosoftCoco} is used for labeling multiple  categories present in images, while in \citeA{Gebru2017FineGrainedCD} these images are used for fine-grained recognition. 
Our dataset includes a reference image, i.e. the iconic image, and a product description for every class, and we have also labeled the grocery items in a structured manner.

Other image datasets of fruits and vegetables for classification purposes are the FIDS30 database \citeA{marko2013fids30} and the dataset in \citeA{muresan2017fruit}. The images in FIDS30 were downloaded from the web and contain background noise as well as single or multiple instances of the object. In \citeA{muresan2017fruit}, all pixels belonging to the object are extracted from the original image, such that all images have white backgrounds with the same brightness condition. There also exist datasets for detecting fruits in orchards for robotic harvesting purposes, which are very challenging since the images contain plenty of background and various lighting conditions, and the targeted fruits are often occluded or of the same color as the background \citeA{bargoti2017deepfruitdetection,sa2016deepfruits}.

Another dataset that is highly relevant to our application need is presented in  \citeA{waltner2015mango}. They collected a dataset for training and evaluating the image classifier by extracting images from video recordings of 23 main classes, which are subdivided into 98 classes, of raw grocery items (fruits and vegetables) in different grocery stores. Using this dataset, a mobile application was developed to recognize food products in grocery store environments, which provides the user with details and health recommendations about the item along with other proposals of similar food items. For each class, there exists a product description with nutrition values to assist the user in shopping scenarios. The main difference between this work and our dataset is firstly the clean iconic images (visual labels) for each class in our dataset, and secondly that we have also collected images of refrigerated items, such as dairy and juice containers, where visual information is required to distinguish between the products.   

\section{Our Dataset}\label{sec:our-dataset}

We have collected images from fruit and vegetable sections and refrigerated sections with dairy and juice products in 18 different grocery stores. The dataset consists of 5125 images from 81 fine-grained classes, where the number of images in each class range from 30 to 138. Figure \ref{fig:hist} displays a histogram over the number of images per class. As illustrated in Figure \ref{fig:examples}, the class structure is hierarchical, and there are 46 coarse-grained classes. Figure \ref{fig:dataset-figure} shows examples of the collected natural images. For each fine-grained class, we have downloaded an iconic image of the item and also a product description including origin country, an appreciated weight and nutrient values of the item from a grocery store website. Some examples of downloaded iconic images can be seen in Figure \ref{fig:clean-image-figure}. 

Our aim has been to collect the natural images under the same condition as they would be as part of an assistive application on a mobile phone. All images have been taken with a 16-megapixel Android smartphone camera from different distances and angles. Occasionally, the images include other items in the background or even items that have been misplaced in the wrong shelf along with the targeted item. It is important that image classifiers that are used for assisting devices are capable of performing well with such noise since these are typical settings in a grocery store environments. The lighting conditions in the images can also vary depending on where the items are located in the store. 
Sometimes the images are taken while the photographer is holding the item in the hand. This is often the case for refrigerated products since these containers are usually stacked compactly in the refrigerators. For these images, we have consciously varied the position of the object, such that the item is not always centered in the image or present in its entirety. 

We also split the data into a training set and test set based on the application need. Since the images have been taken in several different stores at specific days and time stamps, 
parts of the data will have similar lighting conditions and backgrounds for each photo occasion. To remove any such biasing correlations, all images of a certain class taken at a certain store are assigned to either the test set or training set. Moreover, we balance the class sizes to as large extent as possible in both the training and test set. After the partitioning, the training and test set contains 2640 and 2485 images respectively. Predefining a training and test set also makes it easier for other users to compare their results to the evaluations in this paper.

The task is to classify natural images using mobile devices to aid visually impaired people. The additional information such as the hierarchical structure of the class labels, iconic images, and product descriptions can be used to improve the performance of the computer vision system. Every class label is associated with a product description. Thus, the product description itself can be part of the output for visually impaired persons as they may not be able to read what is printed on a carton box or a label tag on a fruit bin in the store.

%Collecting and labeling real-world data is a time-costly procedure. 
%Therefore, we propose that the natural images can be combined with additional information of the target object in the form of text and iconic images.
The dataset is intended for research purposes and we are open to contributions with more images and new suitable classes. Our dataset is available at \url{https://github.com/marcusklasson/GroceryStoreDataset}. Detailed instructions on how to contribute to the dataset can be found on our dataset webpage.



\begin{figure}[t]
\centering
\includegraphics[width=\columnwidth,height=0.20\paperheight]{figures/hist1_latex_bf14.png}
\caption{Histogram over the number of images in each class in the dataset.}
\label{fig:hist}
\end{figure}


\section{Classification Methods}\label{sec:classification-methods}

We here describe the classification methods and approaches that we have used to provide benchmark results to the dataset. We apply both deterministic deep neural networks as well as a deep generative model used for representation learning to the natural images that we have collected. Furthermore, we utilize the additional information -- iconic images -- from our dataset with a multi-view deep generative model. This model can utilize different data sources and obtain superior representation quality as well as high interpretability. For a fair evaluation, we use a linear classifier with the learned representation from the different methods.

\paragraph*{Deep Neural Networks.} 
CNNs have been the state-of-the-art models in image classification ever since AlexNet \cite{krizhevsky2012imagenet} achieved the best classification accuracy in ILSVRC in 2012.
However, in general, computer vision models require lots of labeled data to achieve satisfactory performance, which has resulted in interest for adapting CNNs that have already been trained on a large amount of training data to other image datasets. When adapting pretrained CNNs to new datasets, we can either use it directly as a feature extractor, a.k.a use the off-the-shelf features,~\cite{donahue2014decaf,razavian2014cnnfeatures}, or fine-tune it~\cite{Girshick2014rich-feature-hierarchies,oquab2014learning-and-transferring,pan2010transferlearning,yosinski2014transferable,Zhang2014PartbasedRCNN}. Using off-the-shelf features, we need to specify which feature representation we should extract from the network and use these for training a new classifier. Fine-tuning a CNN involves adjusting the pretrained model parameters, such that the network can e.g. classify images from a dataset different from what the CNN was trained on before. We can either choose to fine-tune the whole network or select some layer parameters to adjust while keeping the others fixed. One important factor on deciding which approach to choose is the size of the new dataset and how similar the new dataset is to the dataset which the CNN was previously trained on. A rule of thumb here is that the closer the features are to the classification layer, the features become more specific to the training data and task \cite{yosinski2014transferable}. 

Using off-the-shelf CNN features and fine-tuned CNNs have been successfully applied in \cite{donahue2014decaf,razavian2014cnnfeatures} and \cite{Girshick2014rich-feature-hierarchies, oquab2014learning-and-transferring, Zhang2014PartbasedRCNN} respectively.
In \cite{donahue2014decaf,razavian2014cnnfeatures}, it is shown that the pretrained features have sufficient representational power to generalize well to other visual recognition tasks with simple linear classifiers, such as Support Vector Machines (SVMs), without fine-tuning the parameters of the CNN to the new task. In \cite{Girshick2014rich-feature-hierarchies, Zhang2014PartbasedRCNN}, all CNN parameters are fine-tuned, whereas in \cite{oquab2014learning-and-transferring} the pretrained CNN layer parameters are kept fixed and only an adaptation layer of two fully connected layers are trained on the new task. The results from these works motivate why we should evaluate our dataset on fine-tuned CNNs or linear classifiers trained on off-the-shelf feature representations instead of training an image recognition model from scratch. 

%\vspace{-3mm}
\paragraph*{Variational Autoencoders with only natural images.}
Deep generative models, 
e.g. the variational autoencoder (VAE)~\cite{kingma2014autoencoding,Rezende2014StochasticBA,zhang2017advances}, have become widely used in the machine learning community thanks to their generative nature. We thus use VAEs for representation learning as the second benchmarking method. For efficiency, we use low-level pretrained features from a CNN as inputs to the VAE.

The latent representations from VAEs are encodings of the underlying factors for how the data are generated. VAEs belongs to the family of latent variable models, which commonly has the form $p_{\boldsymbol{\theta}}(\mathbf{x},\mathbf{z}) = p(\mathbf{z}) p_{\boldsymbol{\theta}}(\mathbf{x}|\mathbf{z})$, where $p(\mathbf{z})$ is a prior distribution over the latent variables $\mathbf{z}$ and $p_{\boldsymbol{\theta}}(\mathbf{x}|\mathbf{z})$ is the likelihood over the data $\mathbf{x}$ given $\mathbf{z}$. The prior distribution is often assumed to be Gaussian,
$p(\mathbf{z}) = \mathcal{N}(\mathbf{z}\,|\, \boldsymbol{0}, \mathbf{I})$,  
whereas the likelihood distribution depends on the values of $\mathbf{x}$.
The likelihood $p_{\boldsymbol{\theta}}(\mathbf{x}|\mathbf{z})$ is referred to as a decoder represented as a neural network parameterized by $\boldsymbol{\theta}$. An encoder network $q_{\boldsymbol{\phi}}(\mathbf{z}|\mathbf{x})$ parameterized by $\boldsymbol{\phi}$ is introduced as an approximation of the true posterior $p_{\boldsymbol{\theta}}(\mathbf{z}|\mathbf{x})$, which is intractable since it requires computing the integral $p_{\boldsymbol{\theta}}(\mathbf{x}) = \int p_{\boldsymbol{\theta}}(\mathbf{x}, \mathbf{z}) \, d\mathbf{z}$. 
When the prior distribution is a Gaussian, the approximate posterior is also modeled as a Gaussian, $q_{\boldsymbol{\phi}}(\mathbf{z}|\mathbf{x}) = \mathcal{N}(\mathbf{z} \,|\,\boldsymbol{\mu}(\mathbf{x}), \boldsymbol{\sigma}^2(\mathbf{x}) \odot \mathbf{I})$, with some mean $\boldsymbol{\mu}(\mathbf{x})$ and variance $\boldsymbol{\sigma}^2(\mathbf{x})$ computed by the encoder network. The goal is to maximize the marginal log-likelihood by defining a lower bound using $q_{\boldsymbol{\phi}}(\mathbf{z}|\mathbf{x})$:
\begin{align}
\begin{split}\label{eq:vae-loss}
\log p_{\boldsymbol{\theta}}(\mathbf{x}) \geq \mathcal{L}(\boldsymbol{\theta}, \boldsymbol{\phi}; \mathbf{x}) = & \mathbb{E}_{q_{\boldsymbol{\phi}}(\mathbf{z}|\mathbf{x})}\left[\, \log p_{\boldsymbol{\theta}}(\mathbf{x} | \mathbf{z}) \,\right] \\ & -D_{KL}(q_{\boldsymbol{\phi}}(\mathbf{z}|\mathbf{x})\,||\,p(\mathbf{z})) .
\end{split}
\end{align}
The last term is the Kullback-Leibler (KL) divergence of the approximate posterior from the true posterior. The lower bound $\mathcal{L}$ is called the evidence lower bound (ELBO) and can be optimized with stochastic gradient descent via backpropagation \cite{doersch2016tutorialvae,kingma2014autoencoding}. 
VAE is a probabilistic framework. Many extensions such as utilizing structured priors\cite{butepage2018Inform} or using continual learning \cite{nguyen2018variational} have been explored.
%In fully supervised learning settings, we would have to retrain the model when a new class is introduced. 
%Another advantage is that VAEs can be extended to multimodal data and learn joint latent representations between data pairs, such as image--text or image--image. 
In the following method, we describe how to make use of the iconic images while retaining the unsupervised learning setting in VAEs.

\paragraph*{Utilizing iconic images with multi-view VAEs.}
Utilizing extra information has shown to be useful in many applications with various model designs~\cite{butepage2018Inform,vedantam2018generative,vinyals2015show,wang2016vcca,zhang2016inter}. For computer vision tasks, natural language is the most commonly used modality to aid the visual representation learning. However, the consistency of the language and visual embeddings has no guarantee. As an example with our dataset, the product description of a Royal Gala apple explains the appearance of a red apple. But if the description is represented with word embeddings, e.g. word2vec \cite{mikolov2013distributedrepresentations}, the word 'royal' will probably be more similar to the words 'king' and 'queen' than 'apple'. Therefore, if available, additional visual information about objects might be more beneficial for learning meaningful representations instead of text. In this work, with our collected dataset, we propose to utilize the iconic images for the representation learning of natural images using a multi-view VAE. Since the natural images can include background noise and grocery items different from the targeted one, the role of the iconic image will be to guide the model to which features that are of interest in the natural image.

The VAE can be extended to modeling multiple views of data, where a latent variable $\mathbf{z}$ is assumed to have generated the views \cite{vedantam2018generative,wang2016vcca}. Considering two views $\mathbf{x}$ and $\mathbf{y}$, the joint distribution over the paired random variables ($\mathbf{x}$, $\mathbf{y}$) and latent variable $\mathbf{z}$ can be written as $p_{\boldsymbol{\theta}}(\mathbf{x}, \mathbf{y}, \mathbf{z}) = p(\mathbf{z})p_{\boldsymbol{\theta^{(1)}}}(\mathbf{x}\,|\,\mathbf{z})p_{\boldsymbol{\theta^{(2)}}}(\mathbf{y}\,|\,\mathbf{z})$, where both $p_{\boldsymbol{\theta^{(1)}}}(\mathbf{x}\,|\,\mathbf{z})$ and $p_{\boldsymbol{\theta^{(2)}}}(\mathbf{y}\,|\,\mathbf{z})$ are represented as neural networks with parameters $\boldsymbol{\theta^{(1)}}$ and $\boldsymbol{\theta^{(2)}}$. Assuming that the latent variable $\mathbf{z}$ can reconstruct both $\mathbf{x}$ and $\mathbf{y}$ when only $\mathbf{x}$ is encoded into $\mathbf{z}$ by the encoder $q_{\boldsymbol{\phi}}(\mathbf{z}|\mathbf{x})$, then the ELBO is written as 
\begin{align}
\begin{split}\label{eq:vcca-loss}
\log p_{\boldsymbol{\theta}}(\mathbf{x}, \mathbf{y}) \geq & \, \mathcal{L}(\boldsymbol{\theta}, \boldsymbol{\phi}; \mathbf{x}, \mathbf{y}) \\ 
= & \,  \mathbb{E}_{q_{\boldsymbol{\phi}}(\mathbf{z}|\mathbf{x})}\left[\, \log p_{\boldsymbol{\theta^{(1)}}}(\mathbf{x} | \mathbf{z}) + \log p_{\boldsymbol{\theta^{(2)}}}(\mathbf{y} | \mathbf{z}) \,\right] \\ 
& -D_{KL}(q_{\boldsymbol{\phi}}(\mathbf{z}|\mathbf{x})\,||\,p(\mathbf{z})) .
\end{split}
\end{align}  
This model is referred to as variational autoencoder canonical correlation analysis (VAE-CCA) and was introduced in \cite{wang2016vcca}. The main motivation for using VAE-CCA is that the latent representations need to contain information about reconstructing both natural and iconic images.
The main motivation for using VAE-CCA is that 
the latent representation needs to preserve information about how both the natural and iconic images are reconstructed. This also allows us to produce iconic images from new natural images to enhance the interpretability of the latent representation of VAE-CCA (see Section \ref{sec:experimental-results}) \cite{vedantam2018generative}.

\begin{figure*}[t]
    \centering
    \subfigure[CNN]{\label{subfig:cnn}\includegraphics[width=0.35\textwidth]{figures/cnn.pdf}} 
    \subfigure[VAE]{\label{subfig:cnn+vae}\includegraphics[width=0.6\textwidth]{figures/cnn+vae.pdf}}
    \subfigure[VAE-CCA]{\label{subfig:vae-cca}\includegraphics[width=0.7\textwidth]{figures/vae-cca.pdf}}
    \caption{The architectures for the classification methods described in Section \ref{sec:classification-methods}. In this paper, we use either a pretrained AlexNet, VGG16 or DenseNet-169 as the CNN feature extractor, but it may be replaced with any CNN architecture. Note that the pretrained CNN can be fine-tuned. The encoder and decoder of the VAE in \ref{subfig:cnn+vae} consist of two fully-connected layers. VAE-CCA in \ref{subfig:vae-cca} uses the DCGAN architecture as an iconic image decoder and the same encoder and feature vector decoder as the VAE.   }
    \label{fig:classification-methods}
\end{figure*}
\input{PaperA/sections_test/experimental-results}

\section{Conclusions}\label{paperD:sec:conclusions}

We proposed a novel method for learning policies for replay scheduling in CL which previously have been unavailable. The goal of our method is to learn a general policy that can be applied to any CL scenario for replay scheduling without requiring additional training costs at test time. We used RL for learning the policy where we employed DQNs for selecting which tasks to replay at different times. In the experiments, we showed that the replay schedules from the learned policy perform similarly to carefully tuned heuristic scheduling baselines. Furthermore, we gained more insights by visualizing the learned replay schedules and observed that the policy is capable of learning scheduling behaviors similar to human education techniques such as spaced repetition. In future work, we will investigate how alternative RL algorithms than DQN are capable of learning policies in this setting as well as experimenting with continuous action spaces to enable scaling to longer task-horizons. We hope that our work can encourage more research within this CL setting where the limitations are on compute rather than memory storage which stems well with the needs in real-world CL scenarios.  



\renewcommand*{\bibname}{References}
\bibliographystyleA{unsrt}
\bibliographyA{References/paperA_test}
