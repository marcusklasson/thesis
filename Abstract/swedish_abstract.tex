\selectlanguage{swedish}
\begin{abstract}
\noindent De senaste åren har %datorseende-baserade 
teknologiska hjälpmedel baserade på datorseende möjliggjort för synskadade personer att använda sig av automatisk visuell igenkänning på deras mobiltelefoner. 
Dessa system bör kunna %vara kapabla till att 
känna igen objekt på finfördelade % finkornig?
nivåer för att förse användaren med noggranna prediktioner. % förutsägelser?
%Dessutom bör användaren 
Användaren bör även ha möjligheten att uppdatera systemet kontinuerligt till att känna igen nya objekt av intresse. 
Dock finns det flera utmaningar som behöver avklaras för att aktivera dessa funktioner i synhjälpmedelssystem i reella och mycket varierande miljöer.  
Exempelvis behöver finfördelad bildigenkänning vanligtvis stora mängder märkt %betecknad?
data för att vara robust. Dessutom har bildklassificerare besvär med att behålla sin prestanda av tidigare inlärda förmågor när de anpassas till nya uppgifter. 
Denna avhandling är uppdelad i två delar, där vi tar oss an dessa utmaningar. 
Först fokuserar vi på tillämpningen av att använda synhjälpmedelssystem för att handla matvaror, där varorna är naturligt strukturerade enligt finfördelade detaljer. 
Vi påvisar hur bildklassificerare kan tränas med en kombination av naturliga bilder och webbskrapad information om matvarorna för att erhålla mer träffsäker klassificeringsförmåga jämfört med att enbart använda naturliga bilder för träning. 
Därefter lägger vi fram ett nytt tillvägagångssätt för kontinuerlig inlärning som kallas replay scheduling (repris-schemaläggning), där vi väljer vilka uppgifter som ska repeteras vid olika tidpunkter för att förbättra bibehållande av minnen. 
Vi föreslår även ett nytt ramverk för inlärning av policyer för replay scheduling som kan generalisera till nya scenarion för kontinuerlig inlärning för att mildra effekten av katastrofal glömska i bildklassificerare. 
Denna avhandling ger insyn till praktiska utmaningar som behöver lösas för att förbättra användbarheten hos datorseende till att hjälpa synskadade personer i verkliga scenarier.
\end{abstract}
\selectlanguage{english}

