\begin{abstract}
\noindent halloj halloj

\bigskip \bigskip

50\%:
The recent advances in computer vision have made it possible to develop vision-based assistive devices to aid visually impaired people with object recognition in different environments. A particular application where assistive vision would be useful is to help the visually impaired with fine-grained classification of food items since visual information is often required to distinguish between similar items. Enabling image classification models to work in this setting requires collecting large amounts of images of food items which is an expensive process in several ways. In the first part of my work, I have shown how more easily accessible information about food items, such as web-scraped images and text descriptions, can be used for enhancing the classification performance of groceries compared to training using natural images only. 

A valuable feature of assistive vision devices is the capability of adapting to new object classes. The key challenge is to avoid overwriting previous knowledge when the model is updated with new classes, which is called catastrophic forgetting. In the second part of my work, I aim to mitigate this problem using replay methods, which involve mixing previous data samples with incoming samples from new classes during training. 

Finally, I will discuss a future third part consisting of approaches for scaling replay methods to larger datasets and how to deploy our models in mobile devices.
		
\end{abstract}
	
\bigskip \bigskip \bigskip \bigskip \bigskip
	
\setlength{\leftskip}{0.3 cm} \textbf {Keywords:} Lorem, Ipsum, Dolor, Sit, Amet
