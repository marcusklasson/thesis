

%%% Paper content

\chapter{
	\centering{Meta Policy Learning for Replay Scheduling in Continual Learning}
}\label{paperD}
\chaptermark{Meta Policy Learning for Replay Scheduling}
\vspace{-5mm}
\begin{center}
	\large{\textbf{Marcus Klasson$^{*}$, Hedvig Kjellström$^{*}$, Cheng Zhang$^{\dagger}$}} \\[2mm]
	\small{$^{*}$KTH Royal Institute of Technology, Stockholm, Sweden} \\
	\small{$^{\dagger}$Microsoft Research, Cambridge, United Kingdom} \\
\end{center}


%%%%%%%%%%%%%%%%%%%%%%%%%%%%%%%%%%%%%%%%%%%%%%%%%%%%%%%%%%%%%%%%%%%%%%%%%%%%%%%%
%%%%%%%%%%%%%%%%%%%%%%%%%%%%%%%%%%%%%%%%%%%%%%%%%%%%%%%%%%%%%%%%%%%%%%%%%%%%%%%%
\begin{abstract}
	%\printinunitsof{in}\prntlen{\linewidth}
	\noindent Learning the time to replay different tasks have been demonstrated to be important in continual learning. 
	However, a replay scheduling policy that can be applied in any continual learning scenario is currently absent, which makes replay scheduling infeasible in real-world scenarios.  
	To this end, we propose using reinforcement learning to enable learning general policies that can generalize across different data domains. Thus, an efficient replay schedule can be used in any new real-world continual learning scenario to improve the continual learning performance without any additional computational cost.
	We compare the learned policies to several replay scheduling baselines and show that the learned policies can improve the continual learning performance given %any 
	unseen tasks and datasets. 
\end{abstract}


%%% Contents

\section{Introduction}\label{paperD:sec:introduction}

Scheduling over which historical tasks to replay has been shown to mitigate catastrophic forgetting efficiently in continual learning (CL)~\citeD{D:aljundi2019online, D:klasson2021learn}. 
Such replay strategies are important in real-world applications with machine learning systems that needs to learn continuously from streaming data, wherein the incoming data is stored rather than deleted.
In such scenarios, re-training on all stored data is often prohibited due to limitations on computational resources and data transmission times. Therefore, we need a policy that schedules from immense amounts of recorded data which previously learned abilities to replay when learning new tasks. 

Although memory scheduling is essential for real-world CL, an efficient policy that schedules the time to replay which tasks is currently missing from the literature. For example, the policy in~\citeD{D:klasson2021learn} is searched after using multiple CL episodes which is prohibited in the CL setting. Moreover, the method in~\citeD{D:aljundi2019online} performs forward passes on all stored data for selecting the tasks to replay, which becomes challenging in scenarios where the amounts of historical data is huge. 
If we could learn the scheduling policy, we would want the possibility to transfer the policy to new CL scenarios without the need for additional training in the target domain. Ideally, the policy should be domain-agnostic such that it is capable of generalizing to any CL dataset unseen during training. Such policy could potentially enable replay methods to efficiently mitigate catastrophic forgetting in real-world CL applications where the number of seen classes exceeds the replay memory budget. 

In this paper, we propose a framework based on reinforcement learning (RL)~\citeD{D:sutton2018reinforcement} for learning a general replay scheduling policy that selects which tasks to replay at different times. We focus on the CL setting introduced in~\citeD{D:klasson2021learn} which simulates a scenario where a replay memory is sampled from a large historical data storage for mitigating catastrophic forgetting. The policy is learned from multiple episodes in a CL environment by selecting which tasks to replay to efficiently mitigate catastrophic forgetting in the CL network. 
Our goal is to learn a replay scheduling policy that generalizes to new CL scenarios without added computational cost. 
We enable the possibility for generalizing to new domains by using the intuition that the policy should replay tasks that are to be forgotten by the classifier. Hence, we provide the policy with the evaluated task performances of the classifier in the CL environment, such that the policy can select the tasks to efficiently retain overall CL performance.  
In summary, our contributions are:
\begin{itemize}[topsep=1pt,] %noitemsep]
	\item We take the first steps to enable replay scheduling in real-world CL scenarios by proposing an RL-based framework for learning policies that generalize across different CL environments (Section \ref{paperD:sec:methodology}). 
	\item We show that the learned policies can efficiently mitigate catastrophic forgetting in CL scenarios with new task orders and datasets unseen during training without added computational cost (Section \ref{paperD:sec:results_on_policy_generalization}).
\end{itemize}


\section{Preliminaries}\label{paperD:sec:background}

In this section, we recall the CL setting in~\citeD{D:klasson2021learn} that we consider as well as some background on RL and the algorithms Deep Q-Networks (DQNs)~\citeD{D:mnih2013playing, D:mnih2015human}, Advantage Actor-Critic (A2C)~\citeD{D:mnih2016asynchronous}, and Soft Actor-Critic~~\citeD{D:haarnoja2018soft,D:haarnoja2018sac_algo}. 


\subsection{Continual Learning Setting}
%\vspace{-3mm}
%\paragraph{Continual Learning Setting.} 
We consider the CL setting presented in Klasson \etal~\citeD{D:klasson2021learn}. In contrast to the traditional CL setting~\citeD{D:delange2021continual, D:parisi2019continual}, the historical data is assumed to be accessible at any time for replay to mitigate catastrophic forgetting. However, retraining on all seen data is prohibited due to processing time constraints, such as limitations on data transmission times or the allowed time for learning new tasks. The learning setup is the same as for CL in image classification, where a network parameterized by $\vphi$ should learn $T$ tasks arriving sequentially in the form of $T$ datasets $\gD_{1:T} = \{\gD_1, \dots, \gD_T\}$. The dataset for the $t$-th task $\gD_t = \{(\vx_t^{(i)}, y_t^{(i)})\}_{i=1}^{N_t}$ where $\vx_t^{(i)}$ and $y_t^{(i)}$ are the $i$-th data point and corresponding class label among a total of $N_t$ examples in the dataset. Furthermore, each dataset is split into a training, validation, and test set, i.e., $\gD_t = \{\gD_t^{train}, \gD_t^{val}, \gD_t^{test}\}$. The objective at task $t$ is to minimize the loss $\ell(f_{\vphi}(\vx_t), y_{t})$ where $\ell(\cdot)$ is the cross-entropy loss in our case. 

The challenge is for the network $f_{\vphi}$ to retain its performance on the previous tasks. However, as the historical data can be huge, we are only allowed to fill a tiny replay memory $\gM$ of size $M$ with historical data due to limitations on processing time. Hence, we need a method for selecting which tasks to replay and the amount of samples to add to the replay memory from each selected task. We use the same approach as in~\citeD{D:klasson2021learn} to enable the task selection, 
where a discrete action space of task proportions is created. At each task $t$, there is a finite set possible task proportions $\vp_t = (p_1, \dots, p_{T-1})$, where $\sum_j p_j = 1$ with $p_j \geq 0$ if $j < t$ otherwise $p_j = 0$, that can be used for constructing the replay memory for mitigating catastrophic forgetting. 
%See Appendix \ref{paperD:app:action_space} for details on how to construct the action space. 
Replay scheduling involves finding a policy for selecting task proportions that are efficient in mitigating catastrophic forgetting in the CL network. In Section \ref{paperD:sec:methodology}, we describe our RL-based framework for learning such policies. 


\subsection{Background on Reinforcement Learning}

The RL setup considers an agent interacting with an environment $E$ over a number of discrete time steps~\citeD{D:sutton2018reinforcement}. The environment is modeled with a Markov Decision Process (MDP)~\citeD{D:bellman1957markovian} represented as a tuple $E = (\gS, \gA, P, R, \mu, \gamma)$ consisting of the state space $\gS$, action space $\gA$, state transition probability $P(s' | s, a)$, reward function $R(s, a)$, initial state distribution $\mu(s_1)$, and discount factor $\gamma$. At each time step $t$, the agent receives a state $s_t$ from the environment, selects an action $a_t \in \gA$ using a policy $\pi(a | s)$, and enters the next state $s_{t+1}$ with transition probability $P(s_{t+1} | s_t, a_t)$ and receives a numerical reward following $r_t$ from the environment. This procedure is repeated until the agent reaches a terminal state in which the procedure can be restarted. The return $G_t = \sum_{k=0}^{\infty} \gamma^{k} r_{t+k}$ is the discounted accumulated reward from time step $t$. The goal for the agent is to learns  policy that maximizes the expected return.

The action value $Q^{\pi}(s, a) = \E[G_t | s_t=s, a]$  is the expected return for selecting action $a$ in state $s$ and following policy $\pi$. The optimal action value $Q^{*}(s, a) = \max_{\pi} Q^{\pi}(s, a)$ is defined as the maximum action value for state $s$ and action $a$ for any given policy $\pi$. Deep Q-Networks (DQNs)~\citeD{D:mnih2013playing, D:mnih2015human} is a value-based RL algorithm which aims to approximate the optimal action value function as $Q^{*}(s, a) \approx Q_{\vtheta}(s, a)$ with a neural network $Q_{\vtheta}$ parameterized by $\vtheta$. For learning the function $Q_{\vtheta}$, we collect data from the environment with by using an epsilon-greedy policy~\citeD{D:sutton2018reinforcement} to estimate $Q_{\vtheta}$ by minimizing the loss
\begin{align}\label{eq:dqn_loss}
	\gL_{\text{DQN}}(\vtheta) = (y_t - Q_{\vtheta}(s_t, a_t))^2
\end{align}
with $y_t = r_t + \gamma \max_{a'} Q_{\vtheta^{-}}(s_{t+1}, a')$, where $\vtheta^{-}$ is a previous copy of the network $\vtheta$ referred to as the target network. In addition to the introduction of target networks, several methods have been proposed for stabilizing the learning process, such as using experience replay~\citeD{D:lin1992self} to sample training data stored in a replay buffer, applying L1-smoothing to the loss, and correcting the action value estimates~\citeD{D:van2016deep}. 

An alternative to value-based RL is policy gradient methods where the optimal policy is estimated directly with a parameterized form $\pi_{\vtheta}(a | s)$ where $\vtheta$ represents the parameters of a neural network. 
Similar to the action value function, the value function $V^{\pi}(s) = \E[G_t | s_t=s]$ defines the expected return following policy $\pi$ from state $s$. In actor critic methods, such as Advantage Actor-Critic (A2C)~\citeD{D:mnih2016asynchronous} and Proximal Policy Optimization (PPO)~\citeD{D:schulman2017proximal}, we estimate the value function with a neural network $V_{\vtheta_v}$ to reduce variance of the gradients. 
The policy network takes the state $s_t$ as input and outputs a distribution over the possible actions $a_t$. The agent collects experiences from the environment with the current policy to use for updating the parameters $\vtheta$ by minimizing the loss 
\begin{align}\label{eq:pg_loss}
	\gL_{\text{PG}}(\vtheta) = \E[\log \pi_{\vtheta}(a | s) \hat{A}(s_t, a_t)], 
\end{align}
where $\hat{A}_{\vtheta_v}(s_t, a_t)$ is an estimate of the advantage function given by $\sum_{i=0}^{k-1} = \gamma^{i} r_i + \gamma^{k} V_{\vtheta_v}(s_{t+k}) - V_{\vtheta_v}(s_{t})$. The policy and value function can be updated after $t_{max}$ actions or when a terminal state is reached~\citeD{D:mnih2016asynchronous}.

Model-free deep RL algorithms usually suffer from high sample complexity or requires careful hyperparameter tuning. Soft Actor-Critic (SAC)~\citeD{D:haarnoja2018soft,D:haarnoja2018sac_algo} is a maximum entropy RL algorithm that aims to maximize the expected reward and the entropy of the policy, i.e., 
\begin{align}\label{eq:sac_loss}
	\gL_{\text{SAC}} = \E[R(s_t, a_t) + \alpha \gH(\pi(\cdot | s_t))],
\end{align}
where $\alpha$ is a temperature parameter determining the relative importance of the entropy against the reward. The goal for maximum entropy RL is to maximize the reward while maintaining high entropy to encourage exploration of promising actions. SAC has been shown to learn faster than on-policy methods, such as PPO, in tasks with high-dimensional continuous action spaces and being less sensitive to hyperparameter selection~\citeD{D:haarnoja2018soft}. 


\section{Related Work}\label{paperD:sec:related_work}

%In this section, we describe works in continual learning, especially memory-based, generalization in reinforcement learning, and human learning concepts related to our method.

\paragraph{Continual Learning.} One of the main goals of CL methods is to mitigate catastrophically forgetting previous tasks when learning new tasks. Approaches for retaining knowledge of previously learned tasks can be broadly divided into regularization-based methods~\citeD{D:kirkpatrick2017overcoming, D:li2017learning, D:nguyen2017variational, D:schwarz2018progress}, dynamic architectures~\citeD{D:ebrahimi2020adversarial, D:rusu2016progressive, D:yoon2017lifelong}, and memory-based methods~\citeD{D:chaudhry2019tiny, D:hayes2020remind, D:lopez2017gradient}, where the latter is the approach we focus on in this paper. Regularization-based methods put constraints on parameters important for previous tasks to mitigate forgetting and use less salient parameters for learning new tasks. Dynamic architecture approaches increase the capacity of the network by adding more parameters as new tasks are encountered. Memory-based methods mitigate forgetting by replaying samples from old tasks that are stored in a small memory buffer. This work is based on memory-based approaches, but in contrast to the traditional CL setting, we assume that we have a large amount of storage for historical data and can therefore fill the replay memory with historical data from the storage before learning new tasks~\citeD{D:klasson2021learn}. Due to processing time constraints, we need a method for selecting which historical tasks to replay at every time step.   

\paragraph{Large and Changing Action Spaces in RL.} 
Several works focus on bringing RL closer to real-world scenarios where the action spaces are large as well as can change size over time or given certain states. \citeD{D:dulac2015deep} proposes to learn action embeddings where $k$-nearest neighbors are used for selecting valid actions which the policy can select from to enable RL in large action spaces. \citeD{D:chandak2019learning} proposes to let the agent reason about which actions to take in a lower-dimensional representation space and transforms decisions in this space into actual actions which improve generalization in large action spaces when the agent infers outcomes of actions that are similar to actions already taken. \citeD{D:jain2020generalization} takes this a step further and learns action representations that should generalize to completely new actions in a zero-shot setting. \citeD{D:boutilier2018planning, chandak2020reinforcement} present how RL can be applied to stochastic action sets, while \citeD{D:chandak2020lifelong} proposes algorithms that adapt to action spaces where the set of available actions changes over time.  
Several attempts have been made to scale up Q-learning to high-dimensional or continuous action spaces. \citeD{D:he2015deep} applies Q-learning to action spaces for natural language where the Q-function is approximated by the dot-product between state and action embeddings for measuring the relevance of performing the action. \citeD{D:van2020q} comes around the maximization over all actions by instead taking the maximization over a subset of actions sampled from a learned proposal distribution, and \citeD{D:metz2017discrete} discretizes each action dimension into bins where the maximization is performed and predicts Q-value for each dimension sequentially. 
We assume that the structure of the action space is known as the number of tasks to replay grows linearly with every seen task. Since we also assume the task-horizon is known, we know the number of available actions at the end of the episodes which allows us to use a static architecture and use masking for selecting valid actions. 

\paragraph{Generalization in RL.} 
Generalization in RL is an active research field where much focus has been put on developing proper benchmark datasets for evaluating generalization capabilities of RL agents~\citeD{cobbe2019quantifying, cobbe2020leveraging, nichol2018gotta, yu2020meta}. 
Regularization techniques from supervised learning have been used to investigate whether these can enhance generalization in RL~\citeD{D:cobbe2019quantifying, farebrother2018generalization, igl2019generalization}, and also how variations in the environments help to obtain agents that generalize better~\citeD{D:packer2018assessing, zhang2018dissection}. The survey by ~\citeD{D:kirk2021survey} focuses on zero-shot policy transfer where a policy is evaluated on environments different from it was trained on. Furthermore, access to rewards or additional training is disallowed in the test environments with the aim to enable RL in real-world scenarios~\citeD{D:yang2019single}. We evaluate our learned policies on the zero-shot policy transfer setting where the goal is to mitigate catastrophic forgetting in new CL scenarios. The rewards are accessible through accuracies calculated on validation datasets, however, fine-tuning the policy during CL training is prohibited.  


%There are also works that try to learn in large action spaces and also generalize fast to new actions [add references].  

%\MK{Mention zero-shot policy transfer and what it is.}



%\paragraph{Human Learning.} We draw inspiration from human learning techniques, especially spaced repetition~\citep{dempster1989spacing}, in our approach for learning replay scheduling policies. 

%\MK{
%Some links on human learning and ML with spaced repetition 
%\begin{itemize}
%    \item Enhancing human learning via spaced repetition optimization, Tabibian et al. \url{https://www.pnas.org/content/pnas/116/10/3988.full.pdf}, \url{http://learning.mpi-sws.org/memorize/}
%    \item Accelerating Human Learning with Deep Reinforcement Learning, Reddy et al.
%    \item Unbounded Human Learning: Optimal Scheduling for Spaced Repetition Reddy et al. \url{https://dl.acm.org/doi/abs/10.1145/2939672.2939850}
%\end{itemize}}

\section{Methodology}\label{paperD:sec:methodology}

In this section, we describe our problem setting for learning replay scheduling policies and also how we employ DQNs to learn such policies.
%In this section, we describe our approach for learning policies for selecting which tasks to replay at different time steps in a continual learning setting. We apply reinforcement learning methods for learning schedules over which tasks to replay based on the current task performance. The aim is to obtain a scheduler that can be transferred to new domains where classifiers are trained in continual learning settings. We begin by describing the problem setting followed by a description of how reinforcement learning is applied to learn a transferable scheduling policy.  



\subsection{Problem Setting}\label{paperD:sec:problem_setting}

Our goal is to learn a replay scheduling policy for selecting which tasks to replay at different times that can generalize to new CL scenarios without additional training in the new environment. We apply RL for learning the policy where a Deep Q-Network (DQN) acts as a scheduler of the replay tasks for a CL model. We train the policy by collecting experience from multiple learning episodes in different CL environments to improve policy transfer. At test time, we let the policy be used to select replay schedules for new CL scenarios.  

Each environment is modeled by its own MDP $E = (\gS, \gA, P, r, \gS_1, \gamma) \in \gE$ with an accompanied classifier $\vphi^{E}$ and CL datasets $\gD_{1:T}^{E}$ of $T$ tasks. We assume the task-horizon $T$ is finite and known. The agent passes action $a_t^{E}$ at task $t$ to environment $E$ for building the replay memory $\gM_t$ before the classifier $\vphi^{E}$ learns task $t+1$. After learning the task, the agent receives the reward $r_{t}^{E}$ from applying action $a_{t}^{E}$ and the state of the environment $s_{t+1}^{E}$. We calculate the rewards using the average validation accuracy over all seen tasks, i.e.
\begin{align}\label{eq:dense_reward}
    r_t^{E} = \frac{1}{t}\sum_{i=1}^{t} A_{t, i}^{(val)}, 
\end{align} 
where the accuracy $A_{t, i}^{(val)}$ is calculated using the validation dataset $\gV_{i}^{E}$. The goal for the agent is to maximize the sum of rewards after the classifier has learnt the final task $T$. We have summarized the CL procedure in Algorithm \ref{alg:continual_learning_step}.
Next, we describe how the states and action spaces are defined.

%Our goal is to learn a policy for selecting which tasks to replay at different times that can generalize to new continual learning scenarios. We apply reinforcement learning for learning the policy where a Deep Q-Network (DQN) acts as a scheduler of the replay tasks for a continual learning model. At test time, we let the policy from the DQN be used to select replay schedules for new continual learning scenarios.    

We are dealing with state and action spaces that are growing per seen task. We wish for the DQN to select which tasks to replay based on the current performance of the CL model $\vphi$. Therefore, we represent the states with a vector of the currently seen accuracies, such that $s_t \in [0, 1]^{t} = \gS_t$ where the state space domain $\gS_t$ grows linearly with $t$. We keep a validation set for each task to calculate the task accuracies to use for the states. The action space is also growing per task, such that $\gA = (\gA_1, \dots, \gA_{T-1})$ where $|\gA_t| < |\gA_{t+1}|$ for $t=1, \dots, T-1$. Each action $a_t \in \gA_t$ is represented as a vector of task proportions to use for filling the replay memory. More specifically, the action $a_t = [p_1, \dots, p_{t}]$ where $p_i$ is the task proportion for task $i$ which satisfies $\sum_{i=1}^{t} p_i = 1$ for $t \in \{1, \dots, T-1\}$.  We construct the action space and the number of actions to use in the same way as in Klasson \etal~\citeD{D:klasson2021learn} which we describe in more detail in Appendix \ref{paperD:app:construction_of_the_action_space}.   

%The action space is also growing per task similarly as the state space. Since there are an additional task that can be replayed per time step, we have the action space $\gA = (\gA_1, \dots, \gA_{T-1})$ where $|\gA_t| < |\gA_{t+1}|$ for $t=1, \dots, T-1$. We use the same discrete action space as was used in~\cite{klasson2021learn}. An action $\va_t \in \sR^{T-1}$ consists of the task proportions for the $T-1$ tasks that can be replayed during each episode.  

%The learning steps consists of 1) updating the agent $Q_{\vtheta}$ on experiences from continual learning environment, and 2) collect experience by training the continual learning model $\vphi$ on the current task dataset and the replay memory $\gM$ composed from the given action. 

%In this setting, each time step $t \in (1, T)$ consists of the following steps: 1) The DQN interacts with a continual learning environment where a continual learning model $\vphi$ learns task $t$ at time step $t$ 

%Our goal is to learn a policy for selecting which tasks to replay at different times that can generalize to new continual learning scenarios unseen during training. We apply reinforcement learning for learning the policy where the agent acts a scheduler of the replay tasks. To evaluate its generalization capability, we deploy the learned policy to new continual learning environments to be used in a zero-shot setting. Next, we describe the procedure of how we learn the replay scheduling policy. 

%We let the agent interact with environments including a continual learning scenario for learning the replay scheduling policy $\pi$. Each environment includes a classifier $\vphi$ and $T$ datasets of tasks that the classifier should learn. The state $s$ of the environment describes the the performance on every seen task for the classifier. Before training the classifier on task $t$, we must construct a replay memory $\gM_t$ with samples from previous tasks for mitigating catastrophic forgetting. The agent is responsible for selecting actions $\va$ with which tasks to replay and the proportion samples of the selected tasks to compose the replay memory with. After the classifier has learned task $t$ with dataset $\gD_t$ and replay memory $\gM_t$, the agent receives a reward $r_t$ for the performed action as well as the state $s_{t+1}$ for selecting the next action. The goal for the agent is to maximize the sum of rewards after the classifier has learnt the final task $T$. 

%In this setting, the state space is growing as a new task is to be learned. The state space $\gS$ consists of the performance of seen tasks from the classifier, such that $\gS = (\gS_1, \dots, \gS_T)$ where $\gS_t \in \sR^{t}$. We keep a validation set $\gD_t^{(val)}$ for all tasks to evaluate the performance of the classifier during training, and compute the validation accuracy $A_{t, i}^{(val)}$ for all seen tasks $i \leq t$, where $A_{t, i}^{(val)}$ is the validation accuracy for task $i$ after learning task $t$. We use the validation accuracies for representing the state $s_{t}$, such that $s_t = (A_{t, 1}^{(val)}, \dots, A_{t, t}^{(val)}) \in \gS_t$. 

%The action space is also growing per task similarly as the state space. Since there are an additional task that can be replayed per time step, we have the action space $\gA = (\gA_1, \dots, \gA_{T-1})$ where $|\gA_t| < |\gA_{t+1}|$ for $t=1, \dots, T-1$. We use the same discrete action space as was used in~\cite{klasson2021learn}. An action $\va_t \in \sR^{T-1}$ consists of the task proportions for the $T-1$ tasks that can be replayed during each episode.    

%We calculate the rewards using the average validation accuracy over all seen tasks, such that
%\begin{align}
%    r_t = \frac{1}{t}\sum_{i=1}^{t} A_{t, i}^{(val)}.
%\end{align}

%We evaluate the learned policy on new continual learning environments unseen during training. 


%We consider a problem setting where the goal is to learn a policy that selects which tasks to replay at different time steps in the continual learning setting described in Section \ref{sec:background}. We need a method for selecting which tasks to replay at different times to mitigate catastrophic forgetting since all historical data is stored and the replay memory is tiny. Furthermore, we want a method that can generalize to new and unseen continual learning scenarios.   

%Describe our setting where we learn a policy for selecting the tasks to replay and we want thjis policy to generalize to new and unseen CL environments. 

%Let a classifier $\vphi$ learn $T$ tasks from the datasets $\gD_{1:T}$ arriving in a sequence. We assume that there is enough external memory to store the full datasets, such that they can be used for replay to mitigate catastrophically forgetting already seen tasks. However, we are only allowed to fill a limited replay memory $\gM$ of size $M$ with historical samples to use for replay due to processing constraints.        

%\MK{Add info on the RL setting, like describe that action space is increasing, and make connections to our CL setting. Perhaps describe the environment here, that it is the whole CL part.}

%In this paper, we propose to use an RL agent to select which seen tasks a continual learning classifier should replay when training on new tasks. We discretize the action space into selections from a replay memory of how many stored samples from each task that should be used for replay. Our RL agent is a DQN aiming to maximize the classification score of the continual learner by learning action-values estimating the expected discounted return, which is the classification performance. The action-values are learned from observing the current task performance of the continual learner, such that the DQN learns a general policy of selecting replay tasks simply based on how well the classifier is performing on each task. Our goal for this method is that the general policy should be transferable and generalize to new domains where classifiers are trained in a continual learning setting.

%In this paper, the goal is to learn how to select proportions of memory tasks to replay when new tasks arrives. Our intuition behind when replaying tasks is necessary is when the classifier performs worse than earlier on specific tasks. Furthermore, we should be able to balance the proportion of samples per task, such that the classifier can focus on retaining knowledge of tasks that are harder to remember. Constructing rules for selecting the proportion is potentially requires lots of tuning when adapting the same strategies in different environments. Therefore, we suggest to learn the behaviour of controlling the amount of samples from old tasks that the classifier uses for replay. Next, we describe how to construct the problem setting to learn such selection strategies. 

%\subsection{Problem Setting}\label{sec:problem_setting}


\begin{algorithm}[t]
   \caption{Continual Learning Step}
   \label{alg:continual_learning_step}
\begin{algorithmic}[1]
   \Require $\vphi$: Continaul learning model
   \Require $\gD_{1:T}$: Task datasets, $a_t$: action from policy
   \State Sample replay memory $\gM_t$ of historical data using the task proportions in action $a_t$ 
   \State Train $\vphi$ on task $t+1$ with data $\gD_{t+1}^{(train)} \cup \gM_t$
   \State Calculate reward $r_t$ using Eq. \ref{eq:dense_reward}
   \State Get next state $s_{t+1}$ with validation performance of $\vphi$ 
   \Return reward $r_t$ and next state $s_{t+1}$ 
\end{algorithmic}
\end{algorithm}


\begin{algorithm}[tb]
   \caption{Learning Replay Scheduling Policy with DQN}
   \label{alg:learning_replay_scheduling_policy_with_dqn}
\begin{algorithmic}[1]
   \Require $\vtheta$: DQN parameters,
   \Require $\gE$: Set of training environments,
   \State Initialize replay buffer $\gB$
   %\STATE Initialize Q-network $Q_{\vtheta}$ with random weights $\vtheta$ 
   \For{episode$=1, \dots, n_{episodes}$}
        \State Initialize $s_1^{E}$ for every $E \in \gE$
        \For{$t=1, T-1$}
            \For{$E \in \gE$}
            \State Take action $a_t^{E} = \argmax_{a \in \gA_t} Q_{\vtheta}(s_t^{E}, a)$ or random action with probability $\epsilon$
            \State Execute $a_t^{E}$ in $E$ with Alg. \ref{alg:continual_learning_step} and observe reward $r_t^{E}$ and next state $s_{t+1}^{E}$ 
            \State Store transition $(s_t^{E}, a_t^{E}, r_t^{E}, s_{t+1}^{E})$ in $\gB$
            \State Sample mini-batch $(s_j, a_j, r_j, s_{j+1}) \sim \gB$ 
            \State Perform gradient descent step on $(y_j - Q_{\vtheta}(s_j, a_j))^2$ with target values $y_j$ 
            \EndFor
        \EndFor 
   \EndFor
\end{algorithmic}
\end{algorithm}


\subsection{Deep Q-Networks for Learning Replay Scheduling Policy}\label{paperD:sec:dqn_for_learning_replay_scheduling_policy}

We employ Deep Q-Networks~\citeD{D:mnih2013playing} (DQNs) as the RL agent for learning the replay scheduling policy. We assume that the number of tasks $T$ is known which makes it easier to set up the DQN architecture. The output layer size to be the same as the number of actions in the last action space, i.e., $|\gA_{T-1}|$. As earlier action spaces has fewer actions, i.e. $|\gA_{t-1}| < |\gA_{t}|$, we use masking for preventing the DQN to select invalid actions. The input layer is of fixed size $T-1$, such that we can input the task performances of the seen tasks as states. We apply zero-padding on the input state vectors for future tasks that have not been evaluated yet.       

We let the DQN agent collects experience from multiple CL environments for training the policy to enhance the generalization capability. We hypothesize that the policy will get better at learning the tasks to replay based on the task performance through learning experiences from multiple environments with different variations on the datasets in each environment, e.g., different task splits or datasets (Split MNIST, Split FashionMNIST, etc.). 
We store the experiences from all environments in a single replay buffer used for updating the DQN, where we denote the experience from environment $E$ as $(s_{t}^{E}, a_{t}^{E}, r_{t}^{E}, s_{t+1}^{E})$. The replay buffer is shared among the environments to obtain a diverse set of outcomes for learning a general policy that can be used for replay scheduling by only selecting actions with the current task performances of a classifier. See Algorithm \ref{alg:learning_replay_scheduling_policy_with_dqn} for the steps on how to train the DQN. 
%Therefore, if two environments contain the same continual learning dataset, then the labels are split differently with random shuffling to ensure the environments are dissimilar. We can also use datasets with different input distributions in the environments, e.g., MNIST and FashionMNIST, to ease the transferability of the policy to unseen data domains.    

At test time, we evaluate the policy on CL environments with different datasets than seen during training. For example, a different dataset can mean that we have changed the task split of a dataset we used for training, or that we evaluate on a new dataset. We assume that the task-horizons are the same on the datasets used for both training and testing such that we can keep the DQN architecture the same. The DQN acts greedily during testing selecting the next action using
\begin{align}
    a_{t}^{E'} = \argmax_{a} Q_{\theta}(s_{t}^{E'}, a) 
\end{align}
in test environment $E'$ at task $t$. The hope is that we should have obtained a general policy for replay scheduling that generalizes to new CL scenarios.  

%We evaluate the performance of the DQN by using it as the replay scheduler in environments unseen during training. As collecting experience in our setting is expensive, our goal is to learn a general policy that can be used in any continual learning setting without increasing the computational time at test time.    

%We consider the scenario where both the state and action spaces are expanding per time step. 

%\MK{Formal definition of the RL settign we are working with. Add high-level algorithm box explaining the procedure. }

%In this section, we describe in detail how we define the state and action space, reward functions, and environments for our problem setting.

%\paragraph{State Representation.} We represent the states with the performance of each seen task. More specifically, the performance for task $i$ is the validation accuracy evaluated at task $t$ denoted by $A_{t,i}^{(val)}$ where $i \leq t$. We assume that the task horizon is known and use zero-padding for the unseen tasks $t' > t$ at the current task $t$ to be able to use state vectors with fixed size.  

%\paragraph{Action Space.} We use the same action space described in [ICML ws paper]. %Each action represents a proportion of how many samples that should be used for replaying each historical task. The action dimension is therefore increasing linearly with every seen task. We use a binary mask to prevent the agent to select invalid actions. \MK{Describe discretization here too.}

%\paragraph{Reward Function.} We use both sparse and dense rewards based on the task performances. More specifically, at task $t$, we evaluate the average accuracy over all seen tasks up to task $t$ as the reward, such that 
%\begin{align}
%    r_t = \frac{1}{t}\sum_{i=1}^{t} A_{t, i}^{(val)},
%\end{align}
%where $A_{t, i}^{(val)}$ is the validation accuracy for task $i$ after learning task $t$. 



\section{Experiments}\label{paperD:sec:experiments}


We evaluate our RL-based framework using DQN~\citeD{D:mnih2013playing, D:mnih2015human} and A2C~\citeD{D:mnih2016asynchronous} for learning policies that generalize to new CL scenarios. We show that the learned policies can efficiently mitigate catastrophic forgetting in CL environments with new task orders and datasets that are unseen during training. Full details on experimental settings and additional results are in Appendix \ref{paperD:app:experimental_settings} and \ref{paperD:app:additional_experimental_results}. Code will be made publicly available upon acceptance. 


\vspace{-3mm}
\paragraph{Datasets and Network Architectures.} We conduct experiments on four CL benchmark datasets: Split MNIST~\citeD{D:lecun1998gradient, D:zenke2017continual}, FashionMNIST~\citeD{D:xiao2017fashion}, Split notMNIST~\citeD{D:bulatov2011notMNIST}, Permuted MNIST~\citeD{D:goodfellow2013empirical}, and Split CIFAR-10~\citeD{D:krizhevsky2009learning}. We randomly sample 15\% of training data for each task to use for validation, and keep the original test sets intact. 
We use multi-head output layers for all datasets and assume task labels are available at test time~\citeD{D:van2019three}. A 2-layer MLP with 256 hidden units for Split MNIST, Split FashionMNIST, and Split notMNIST. For Split CIFAR-10, we the same ConvNet architecture used in \citeD{D:adel2019continual, D:schwarz2018progress, D:vinyals2016matching}.

\vspace{-3mm}
\paragraph{Generating CL Environments.} We generate multiple CL environments with pre-set random seeds for initializing the network parameters $\vphi$ and shuffling the task order. See Appendix \ref{paperD:app:experimental_settings} for details on the pre-set seeds. 
%The pre-set random seeds are in the range $0-49$, such that we have 50 environments for each dataset. 
We shuffle the task order by permuting the class order and then split the classes into 5 pairs (tasks) with 2 classes/pair. 
%For environments with seed $0$, we keep the original task order in the dataset. 
Taking a step at task $t$ in the CL environments involves training the CL network on the $t$-th dataset with a replay memory $\gM_t$ from the discrete action space described in Appendix \ref{paperD:app:action_space}. 
To speed up the experiments with the RL algorithms, we run a breadth-first search (BFS) through the discrete action space and save the classification results for re-use during policy learning. Note that the action space has 1050 possible paths of replay schedules for the 5-task datasets, which makes the environment generation time-consuming. More specifically, generating a CL environment for one seed with Split MNIST took on around 9.5 hours on average on a NVIDIA GeForce RTW 2080Ti.   
Hence, we only generate environments where the replay memory size $M=10$ have been used, and leave analysis of different memory sizes as future work. 
%The CL environments that we used will be provided in the public code. 

\begin{figure}[t]
	\centering
	\setlength{\figwidth}{0.26\textwidth}
	\setlength{\figheight}{.14\textheight}
	\begin{subfigure}[t]{0.48\textwidth}
		\centering
		
\pgfplotsset{every axis title/.append style={at={(0.5,0.82)}}}
\pgfplotsset{every tick label/.append style={font=\tiny}}
\pgfplotsset{every major tick/.append style={major tick length=2pt}}
\pgfplotsset{every minor tick/.append style={minor tick length=1pt}}
\pgfplotsset{every axis x label/.append style={at={(0.5,-0.22)}}}
\pgfplotsset{every axis y label/.append style={at={(-0.22,0.5)}}}
\begin{tikzpicture}
\tikzstyle{every node}=[font=\scriptsize]
\definecolor{color0}{rgb}{0.12156862745098,0.466666666666667,0.705882352941177}
\definecolor{color1}{rgb}{1,0.498039215686275,0.0549019607843137}
\definecolor{color2}{rgb}{0.172549019607843,0.627450980392157,0.172549019607843}
\definecolor{color3}{rgb}{0.83921568627451,0.152941176470588,0.156862745098039}
\definecolor{color4}{rgb}{0.580392156862745,0.403921568627451,0.741176470588235}

\begin{groupplot}[group style={group size= 4 by 1, horizontal sep=1.40cm, vertical sep=1.00cm}]



\nextgroupplot[title={DQN (ACC: 95.50\%)} ,
height=\figheight,
legend cell align={left},
legend columns=-1,
legend style={
  nodes={scale=0.9},
  fill opacity=0.8,
  draw opacity=1,
  text opacity=1,
  at={(1.20,1.40)},
  anchor=south west,
  draw=white!80!black
},
minor xtick={},
minor ytick={0.55,0.65,0.75,0.85,0.95,1.05},
tick align=outside,
tick pos=left,
width=\figwidth,
x grid style={white!69.0196078431373!black},
xmajorgrids,
xlabel={Task},
xmin=0.8, xmax=5.2,
xtick style={color=black},
xtick={1,2,3,4,5},
xticklabel style={font=\tiny, inner sep=1pt},
xticklabels={1,2,3,4,5},
y grid style={white!69.0196078431373!black},
ymajorgrids,
yminorgrids,
ylabel={Accuracy},
ymin=0.491, ymax=1.01,
ytick style={color=black},
ytick={0.5,0.6,0.7,0.8,0.9,1.0},
yticklabel style={font=\tiny, inner sep=0pt},
yticklabels={50, 60, 70, 80, 90, 100}
]
\addplot [color0, mark=*, mark size=1.5, mark options={solid}]
table {%
1 0.994017958641052
2 0.912761688232422
3 0.868394792079926
4 0.750747740268707
5 0.837986052036285
};\addlegendentry{Task 1}
\addplot [ color1, mark=*, mark size=1.5, mark options={solid}]
table {%
2 0.99297297000885
3 0.918378353118896
4 0.967027008533478
5 0.969729721546173
};\addlegendentry{Task 2}
\addplot [ color2, mark=*, mark size=1.5, mark options={solid}]
table {%
3 0.999067604541779
4 0.987412571907043
5 0.981818199157715
};\addlegendentry{Task 3}
\addplot [ color3, mark=*, mark size=1.5, mark options={solid}]
table {%
4 0.996513962745667
5 0.994521915912628
};\addlegendentry{Task 4}
\addplot [color4, mark=*, mark size=1.5, mark options={solid}]
table {%
5 0.990959346294403
};\addlegendentry{Task 5}



\input{Chapter4/figures/policy_illustrations/mnist/data_seed10/dqn_seed2/bubble_rs_dataset_seed_10_dqn_seed_2}




\nextgroupplot[title={Heur-LD (ACC: 90.77\%)} ,
height=\figheight,
legend cell align={left},
legend style={
  nodes={scale=0.9},
  fill opacity=0.8,
  draw opacity=1,
  text opacity=1,
  at={(3.72,0.28)},
  anchor=south west,
  draw=white!80!black
},
minor xtick={},
minor ytick={0.55,0.65,0.75,0.85,0.95,1.05},
tick align=outside,
tick pos=left,
width=\figwidth,
x grid style={white!69.0196078431373!black},
xmajorgrids,
xlabel={Task},
xmin=0.8, xmax=5.2,
xtick style={color=black},
xtick={1,2,3,4,5},
xticklabels={1,2,3,4,5},
xticklabel style={font=\tiny, inner sep=1pt},
y grid style={white!69.0196078431373!black},
ymajorgrids,
yminorgrids,
ylabel={Accuracy},
ymin=0.491, ymax=1.01,
ytick style={color=black},
ytick={0.5,0.6,0.7,0.8,0.9,1.0},
yticklabels={50, 60, 70, 80, 90, 100},
yticklabel style={font=\tiny, inner sep=0pt}
]
\addplot [color0, mark=*, mark size=1.5, mark options={solid}]
table {%
1 0.994017946161516
2 0.777168494516451
3 0.809571286141575
4 0.567298105682951
5 0.708374875373878
};
%\addlegendentry{Task 1}
\addplot [color1, mark=*, mark size=1.5, mark options={solid}]
table {%
2 0.994054054054054
3 0.970810810810811
4 0.792972972972973
5 0.855675675675676
};
%\addlegendentry{Task 2}
\addplot [color2, mark=*, mark size=1.5, mark options={solid}]
table {%
3 0.998601398601399
4 0.990675990675991
5 0.995804195804196
};
%\addlegendentry{Task 3}
\addplot [color3, mark=*, mark size=1.5, mark options={solid}]
table {%
4 0.99601593625498
5 0.989043824701195
};
%\addlegendentry{Task 4}
\addplot [color4, mark=*, mark size=1.5, mark options={solid}]
table {%
5 0.989452536413862
};
%\addlegendentry{Task 5}
%\addlegendentry{Task 5}




\input{Chapter4/figures/policy_illustrations/mnist/data_seed10/heuristic_local_drop/bubble_plot}

\end{groupplot}

\end{tikzpicture}
		\vspace{-4mm} % hack to get captions aligned...
		\caption{Split MNIST}
		\label{fig:rewards_mnist_2envs}
	\end{subfigure}%
	\begin{subfigure}[t]{0.48\textwidth}
		\centering
		
\pgfplotsset{every axis title/.append style={at={(0.5,0.82)}}}
\pgfplotsset{every tick label/.append style={font=\tiny}}
\pgfplotsset{every major tick/.append style={major tick length=2pt}}
\pgfplotsset{every minor tick/.append style={minor tick length=1pt}}
\pgfplotsset{every axis x label/.append style={at={(0.5,-0.22)}}}
\pgfplotsset{every axis y label/.append style={at={(-0.22,0.5)}}}
\begin{tikzpicture}
\tikzstyle{every node}=[font=\scriptsize]
\definecolor{color0}{rgb}{0.12156862745098,0.466666666666667,0.705882352941177}
\definecolor{color1}{rgb}{1,0.498039215686275,0.0549019607843137}
\definecolor{color2}{rgb}{0.172549019607843,0.627450980392157,0.172549019607843}
\definecolor{color3}{rgb}{0.83921568627451,0.152941176470588,0.156862745098039}
\definecolor{color4}{rgb}{0.580392156862745,0.403921568627451,0.741176470588235}

\begin{groupplot}[group style={group size= 4 by 1, horizontal sep=1.40cm, vertical sep=1.00cm}]



\nextgroupplot[title={DQN (ACC: 95.50\%)} ,
height=\figheight,
legend cell align={left},
legend columns=-1,
legend style={
  nodes={scale=0.9},
  fill opacity=0.8,
  draw opacity=1,
  text opacity=1,
  at={(1.20,1.40)},
  anchor=south west,
  draw=white!80!black
},
minor xtick={},
minor ytick={0.55,0.65,0.75,0.85,0.95,1.05},
tick align=outside,
tick pos=left,
width=\figwidth,
x grid style={white!69.0196078431373!black},
xmajorgrids,
xlabel={Task},
xmin=0.8, xmax=5.2,
xtick style={color=black},
xtick={1,2,3,4,5},
xticklabel style={font=\tiny, inner sep=1pt},
xticklabels={1,2,3,4,5},
y grid style={white!69.0196078431373!black},
ymajorgrids,
yminorgrids,
ylabel={Accuracy},
ymin=0.491, ymax=1.01,
ytick style={color=black},
ytick={0.5,0.6,0.7,0.8,0.9,1.0},
yticklabel style={font=\tiny, inner sep=0pt},
yticklabels={50, 60, 70, 80, 90, 100}
]
\addplot [color0, mark=*, mark size=1.5, mark options={solid}]
table {%
1 0.994017958641052
2 0.912761688232422
3 0.868394792079926
4 0.750747740268707
5 0.837986052036285
};\addlegendentry{Task 1}
\addplot [ color1, mark=*, mark size=1.5, mark options={solid}]
table {%
2 0.99297297000885
3 0.918378353118896
4 0.967027008533478
5 0.969729721546173
};\addlegendentry{Task 2}
\addplot [ color2, mark=*, mark size=1.5, mark options={solid}]
table {%
3 0.999067604541779
4 0.987412571907043
5 0.981818199157715
};\addlegendentry{Task 3}
\addplot [ color3, mark=*, mark size=1.5, mark options={solid}]
table {%
4 0.996513962745667
5 0.994521915912628
};\addlegendentry{Task 4}
\addplot [color4, mark=*, mark size=1.5, mark options={solid}]
table {%
5 0.990959346294403
};\addlegendentry{Task 5}



\input{Chapter4/figures/policy_illustrations/mnist/data_seed10/dqn_seed2/bubble_rs_dataset_seed_10_dqn_seed_2}




\nextgroupplot[title={Heur-LD (ACC: 90.77\%)} ,
height=\figheight,
legend cell align={left},
legend style={
  nodes={scale=0.9},
  fill opacity=0.8,
  draw opacity=1,
  text opacity=1,
  at={(3.72,0.28)},
  anchor=south west,
  draw=white!80!black
},
minor xtick={},
minor ytick={0.55,0.65,0.75,0.85,0.95,1.05},
tick align=outside,
tick pos=left,
width=\figwidth,
x grid style={white!69.0196078431373!black},
xmajorgrids,
xlabel={Task},
xmin=0.8, xmax=5.2,
xtick style={color=black},
xtick={1,2,3,4,5},
xticklabels={1,2,3,4,5},
xticklabel style={font=\tiny, inner sep=1pt},
y grid style={white!69.0196078431373!black},
ymajorgrids,
yminorgrids,
ylabel={Accuracy},
ymin=0.491, ymax=1.01,
ytick style={color=black},
ytick={0.5,0.6,0.7,0.8,0.9,1.0},
yticklabels={50, 60, 70, 80, 90, 100},
yticklabel style={font=\tiny, inner sep=0pt}
]
\addplot [color0, mark=*, mark size=1.5, mark options={solid}]
table {%
1 0.994017946161516
2 0.777168494516451
3 0.809571286141575
4 0.567298105682951
5 0.708374875373878
};
%\addlegendentry{Task 1}
\addplot [color1, mark=*, mark size=1.5, mark options={solid}]
table {%
2 0.994054054054054
3 0.970810810810811
4 0.792972972972973
5 0.855675675675676
};
%\addlegendentry{Task 2}
\addplot [color2, mark=*, mark size=1.5, mark options={solid}]
table {%
3 0.998601398601399
4 0.990675990675991
5 0.995804195804196
};
%\addlegendentry{Task 3}
\addplot [color3, mark=*, mark size=1.5, mark options={solid}]
table {%
4 0.99601593625498
5 0.989043824701195
};
%\addlegendentry{Task 4}
\addplot [color4, mark=*, mark size=1.5, mark options={solid}]
table {%
5 0.989452536413862
};
%\addlegendentry{Task 5}
%\addlegendentry{Task 5}




\input{Chapter4/figures/policy_illustrations/mnist/data_seed10/heuristic_local_drop/bubble_plot}

\end{groupplot}

\end{tikzpicture}
		%\vspace{-2mm}
		\caption{Split CIFAR-10}
		\label{fig:rewards_cifar10_2envs}
	\end{subfigure}
	\vspace{-2mm}
	\caption{Performance progress measured in ACC (\%) for all methods in the 2 test environments with (a) {\bf Split MNIST} and (b) {\bf Split CIFAR-10} in the New Task Orders experiment. We plot the performance progress for the RL algorithms for 100 evaluation steps equidistantly distributed over the training episodes. The performance for the Random, ETS, and Heuristic scheduling baselines are plotted as straight lines.  }
	\label{fig:rewards_new_task_orders_2envs}
	\vspace{-2mm}
\end{figure}


\vspace{-3mm}
\paragraph{DQN and A2C Architectures.}
The input layer has size $T-1$ where each unit is inputting the task performances since the states are represented by the validation accuracies $s_t = [A_{t, 1}^{(val)}, ..., A_{t, t}^{(val)}, 0, ..., 0]$. The current task can therefore be determined by the number of non-zero state inputs. The output layer has 35 units representing the possible actions at $T=5$ in the discrete action space (see Appendix \ref{paperD:app:action_space}). We use action masking on the output units to prevent the network from selection invalid actions for constructing the replay memory at the current task. The DQN is a 2-layer MLP with 512 hidden units and ReLU activations. For A2C, we use separate networks for parameterizing the policy and the value function, where both networks are 2-layer MLPs with 64 hidden units of Tanh activations. 

\vspace{-3mm}
\paragraph{Baselines.} We compare our proposed method to the following scheduling baselines: 
\begin{itemize}[topsep=0pt]%,leftmargin=*, ]
	\item {\bf Random.} Random policy that randomly selects task proportions from the action space on how to structure the replay memory at every task. 
	\item {\bf Equal Task Schedule (ETS).} Policy that selects equal task proportion such that the replay memory aims to fill the memory with an equal number of samples from every seen task. 
	\item {\bf Heuristic Global Drop (Heur-GD).} Heuristic policy that replays tasks with validation accuracy below a certain threshold proportional to the best achieved validation accuracy on the task.
	\item {\bf Heuristic Local Drop (Heur-LD).} Heuristic policy that replays tasks with validation accuracy below a threshold proportional to the previous achieved validation accuracy on the task. 
	\item {\bf Heuristic Accuracy Threshold (Heur-AT).} Heuristic policy that replays tasks with validation accuracy below a fixed threshold. 
\end{itemize}
The heuristics are based on the intuition that forgotten tasks should be replayed. They fill the replay memory with $M/k$ samples per task where $k$ is the number of selected tasks. If $k=0$, then replay is skipped at the current task. 

\vspace{-3mm}
\paragraph{Evaluation Protocol.} We use the average test accuracy over all tasks after learning the final task, i.e., $\ACC = \frac{1}{T} \sum_{i=1}^{T} A_{T, i}^{test}$ where $A_{T, i}^{test}$ is the test accuracy of task $i$ after learning task $T$. 
To assess generalization capability, we use a ranking method based on the $\ACC$ between the methods in every test environment for comparison (see Appendix \ref{paperD:app:ranking_method} for more details). 


\begin{comment}
\begin{table}[t]
	\footnotesize%\small
	\centering
	\caption{Average ranking (lower is better) for experiments on generalizing policies to environments with new task orders or a new dataset. %'S' stands for 'Split'%We train the DQN on 30 training Split MNIST environments and evaluate on 10 test environments. 
		We average the results over 10 test environments. %as well as 3 seeds for DQN and A2C.
	}
	\label{tab:average_ranking_rl_experiment}
	%\resizebox{\textwidth}{!}{
		\begin{tabular}{l c c c c c}
			\toprule %\toprule
			& \multicolumn{3}{c}{ {\bf New Task Order}} & \multicolumn{2}{c}{ {\bf New Dataset}}\\
			\cmidrule(lr){2-4}  \cmidrule(lr){5-6}
			{\bf Method} & S-MNIST & S-FashionMNIST & S-CIFAR-10 & S-notMNIST & S-FashionMNIST  \\ 
			\midrule
			Random         & 4.23   & 3.60 & 5.03 & 3.87 & 3.95\\ 
			ETS            & 3.80 & 4.57 & 5.37 & 4.27 & 3.63 \\
			\midrule 
			Heur-GD        & 4.48 & 4.25 & 3.98 & 4.58 & {\bf 2.77} \\
			Heur-LD        & 4.65 & 3.65 & 3.75 & 4.97 & 5.10 \\
			Heur-AT        & 4.33 & 3.87 & 3.43 & 4.28 & 3.72 \\
			\midrule
			DQN (Ours)     & {\bf 3.23} & {\bf 3.55} & 3.68 & 3.20 & 4.37 \\
			A2C (Ours)     & 3.27 & 4.52 & {\bf 2.75} & {\bf 2.83} & 4.47 \\
			\bottomrule %\bottomrule
		\end{tabular}
		%}
	\vspace{-2mm}
\end{table}
\end{comment}


\begin{table}[t]
	\centering
	\caption{Average ranking (lower is better) across methods in the policy generalization experiments. The best and second-best ranks are colored in \textcolor{forestgreen}{green} and \textcolor{orange}{orange} respectively.}
	\vspace{-3mm}
	\resizebox{0.98\textwidth}{!}{
	\begin{tabular}{lcccccc}
\toprule	
	& \multicolumn{4}{c}{\textbf{New Task Order}}        & \multicolumn{2}{c}{\textbf{New Dataset}} \\
	\cmidrule(lr){2-5} \cmidrule(lr){6-7}
	\textbf{Method} & S-MNIST & S-FashionMNIST & S-notMNIST & S-CIFAR-10 & S-FashionMNIST        & S-notMNIST       \\
	\midrule
	Random          & 4.58    & \textcolor{orange}{3.9}            & 4.14       & 5.57       & 4.53                  & 4.62             \\
	ETS             & 4.5     & 5.24           & 5.0          & 6.2        & \textcolor{orange}{4.14}                  & 4.76             \\
	\midrule
	Heur-GD         & 5.07    & 4.91           & \textcolor{forestgreen}{3.74}       & 4.63       & \textcolor{forestgreen}{3.26}                  & 5.31             \\
	Heur-LD         & 5.27    & 4.17           & 4.38       & 4.21       & 5.68                  & 5.68             \\
	Heur-AT         & 4.92    & 4.6            & 6.28       & 3.89       & 4.21                  & 4.97             \\
	\midrule 
	DQN (Ours)      & \textcolor{orange}{4.0}       & 4.26           & 3.97       & 4.47       & 4.97                  & 4.08             \\
	A2C (Ours)      & \textcolor{forestgreen}{3.56}    & 5.23           & \textcolor{orange}{3.84}       & \textcolor{forestgreen}{3.21}       & 4.76                  & \textcolor{orange}{3.38}             \\
	SAC (Ours)      & 4.1     & \textcolor{forestgreen}{3.69}           & 4.65       & \textcolor{orange}{3.82}       & 4.45                  & \textcolor{forestgreen}{3.2} \\
	\bottomrule             
\end{tabular}
	}
	\label{tab:average_ranking_rl_experiment}
	\vspace{-3mm}
\end{table}


\subsection{Policy Generalization to New Continual Learning Scenarios}\label{paperD:sec:results_on_policy_generalization}

In this section, we show that the policies learned with our RL-based framework using DQN and A2C are capable of generalizing across CL environments with new task orders and datasets unseen during training. 

\vspace{-3mm}
\begin{wrapfigure}{r}{0.5\textwidth}
	%\centering
	\setlength{\figwidth}{0.26\textwidth}
	\setlength{\figheight}{.14\textheight}
	\vspace{-2mm}
	\begin{subfigure}[b]{0.48\textwidth}
		\centering
		
%\pgfplotsset{every axis title/.append style={at={(0.5,0.82)}}} %{at={(0.5,0.84)}}}
%\pgfplotsset{every tick label/.append style={font=\tiny}}
%\pgfplotsset{every major tick/.append style={major tick length=2pt}}
%\pgfplotsset{every minor tick/.append style={minor tick length=1pt}}
%\pgfplotsset{every axis x label/.append style={at={(0.5,-0.27)}}}
%\pgfplotsset{every axis y label/.append style={at={(-0.17,0.5)}}}
%\pgfplotsset{scaled x ticks=false} % switching off scientific notation for x ticks

\begin{tikzpicture}
\tikzstyle{every node}=[font=\scriptsize]
\definecolor{color0}{rgb}{0.12156862745098,0.466666666666667,0.705882352941177}
\definecolor{color1}{rgb}{1,0.498039215686275,0.0549019607843137}
\definecolor{color2}{rgb}{0.172549019607843,0.627450980392157,0.172549019607843}
\definecolor{color3}{rgb}{0.83921568627451,0.152941176470588,0.156862745098039}
\definecolor{color4}{rgb}{0.580392156862745,0.403921568627451,0.741176470588235}
\definecolor{color5}{rgb}{0.549019607843137,0.337254901960784,0.294117647058824}


\begin{groupplot}[group style={group size= 2 by 1, horizontal sep=1.4cm, vertical sep=1.2cm}]

% This file was created with tikzplotlib v0.9.17.
%\begin{tikzpicture}

%\definecolor{color0}{rgb}{0.12156862745098,0.466666666666667,0.705882352941177}
%\definecolor{color1}{rgb}{1,0.498039215686275,0.0549019607843137}
%\definecolor{color2}{rgb}{0.172549019607843,0.627450980392157,0.172549019607843}
%\definecolor{color3}{rgb}{0.83921568627451,0.152941176470588,0.156862745098039}
%\definecolor{color4}{rgb}{0.580392156862745,0.403921568627451,0.741176470588235}

\nextgroupplot[ %begin{axis}[
height=\figheight,
legend cell align={left},
legend columns=-1,
legend style={
  nodes={scale=0.87},
  fill opacity=0.8,
  draw opacity=1,
  text opacity=1,
  at={(-0.43,1.36)},
  anchor=south west,
  draw=white!80!black
},
minor xtick={},
minor ytick={},
tick align=outside,
tick pos=left,
title={},
width=\figwidth,
x grid style={white!69.0196078431373!black},
xlabel={Task},
xmajorgrids,
xmin=0.8, xmax=5.2,
xtick style={color=black},
xtick={1,2,3,4,5},
xticklabels={1,2,3,4,5},
y grid style={white!69.0196078431373!black},
ylabel={Accuracy (\%)},
ymajorgrids,
ymin=0.7, ymax=1.01, %ymin=0.5, ymax=1.01,
ytick style={color=black},
ytick={0.5,0.6,0.7,0.8,0.9,1,1.1},
yticklabels={50,60,70,80,90,100,110}
]
\addplot [line width=1.5pt, color0, mark=*, mark size=1.5, mark options={solid}]
table {%
1 0.973999977111816
2 0.922500014305115
3 0.909500002861023
4 0.846000015735626
5 0.882499992847443
};
\addlegendentry{T1}
\addplot [line width=1.5pt, color1, mark=*, mark size=1.5, mark options={solid}]
table {%
2 0.966000020503998
3 0.828000009059906
4 0.755500018596649
5 0.904500007629395
};
\addlegendentry{T2}
\addplot [line width=1.5pt, color2, mark=*, mark size=1.5, mark options={solid}]
table {%
3 0.986999988555908
4 0.920000016689301
5 0.886500000953674
};
\addlegendentry{T3}
\addplot [line width=1.5pt, color3, mark=*, mark size=1.5, mark options={solid}]
table {%
4 0.94650000333786
5 0.840499997138977
};
\addlegendentry{T4}
\addplot [line width=1.5pt, color4, mark=*, mark size=1.5, mark options={solid}]
table {%
5 0.97350001335144
};
\addlegendentry{T5}
%\end{axis}

%\end{tikzpicture}


\input{PaperD/figures/policy_cifar10/bubble_rs_dataset_seed_19_a2c_seed_2}

\end{groupplot}

\end{tikzpicture}

		\vspace{-4mm}
		\caption{A2C - ACC: 89.75\%}
		\label{fig:policy_cifar10_a2c}
	\end{subfigure}
	\\[1mm]
	\begin{subfigure}[b]{0.48\textwidth}
		\centering
		\input{PaperD/figures/policy_cifar10/groupplot_ets}
		\vspace{-4mm}
		\caption{ETS - ACC: 88.32\%}
		\label{fig:policy_cifar10_ets}
	\end{subfigure}
	%\vspace{-3mm}
	\caption{Task accuracies and replay schedules for A2C and ETS for a Split CIFAR-10 environment.% The replay schedules are visualized as bubble plots and results are taken from 1 seed. %  Task order in environment is [[7, 9], [0, 4], [2, 1], [6, 5], [8, 3]].
	}
	\vspace{-3mm}
	\label{fig:policy_cifar10}
\end{wrapfigure}
\paragraph{Generalization to New Task Orders.} 
We show that the learned replay scheduling policies can generalize to CL environments with previously unseen task orders. We generate training and test environments with unique task orders for three datasets, namely, Split MNIST, FashionMNIST, and CIFAR-10. Table \ref{tab:average_ranking_rl_experiment} shows the average ranking for the DQN, A2C, and the baselines when being applied in 10 test environments. 
Our learned policies obtain the best average ranking across the datasets, where the DQN performs best for Split MNIST and FashionMNIST while A2C outperforms all methods on Split CIFAR-10. Figure \ref{fig:rewards_new_task_orders_2envs}(a) and (b) shows the performance progress for two test environments with Split MNIST and Split CIFAR-10 respectively (see Figure \ref{fig:policy_rewards_mnist_paperD} and \ref{fig:policy_rewards_cifar10_paperD} in Appendix \ref{paperD:app:additional_experimental_results} for all test environments). We observe that DQN exhibits noisier progress of the achieved ACC in these test environments than A2C. 
We provide further insights in the benefits of learning the replay scheduling policy by visualizing the replay schedule and task accuracy progress from A2C in one Split CIFAR-10 test environment in Figure \ref{fig:policy_cifar10}. Additionally, we show the replay schedule and task accuracies from the ETS baseline in the same environment. The replay schedules are visualized with bubble plots showing the selected task proportion to use for composing the replay memories at each task. Each circle color corresponds to a historical task and its size represents the proportion of replay samples at the current task. The sum of points in all circles at each column is fixed at all current tasks. In Figure \ref{fig:policy_cifar10_a2c}, we observe that A2C decides to replay task 2 more than task 1 as the performance on task 2 decreases, which results in a slightly better ACC metric achieved by A2C than ETS. These results show that the learned policy can flexibly consider replaying forgotten tasks to enhance the CL performance.
%or use other advanced RL methods which may generalize better. 

\begin{figure}[t]
	\centering
	\setlength{\figwidth}{0.26\textwidth}
	\setlength{\figheight}{.14\textheight}
	\begin{subfigure}[t]{0.48\textwidth}
		\centering
		
\pgfplotsset{every axis title/.append style={at={(0.5,0.82)}}}
\pgfplotsset{every tick label/.append style={font=\tiny}}
\pgfplotsset{every major tick/.append style={major tick length=2pt}}
\pgfplotsset{every minor tick/.append style={minor tick length=1pt}}
\pgfplotsset{every axis x label/.append style={at={(0.5,-0.22)}}}
\pgfplotsset{every axis y label/.append style={at={(-0.22,0.5)}}}
\begin{tikzpicture}
\tikzstyle{every node}=[font=\scriptsize]
\definecolor{color0}{rgb}{0.12156862745098,0.466666666666667,0.705882352941177}
\definecolor{color1}{rgb}{1,0.498039215686275,0.0549019607843137}
\definecolor{color2}{rgb}{0.172549019607843,0.627450980392157,0.172549019607843}
\definecolor{color3}{rgb}{0.83921568627451,0.152941176470588,0.156862745098039}
\definecolor{color4}{rgb}{0.580392156862745,0.403921568627451,0.741176470588235}

\begin{groupplot}[group style={group size= 4 by 1, horizontal sep=1.40cm, vertical sep=1.00cm}]



\nextgroupplot[title={DQN (ACC: 95.50\%)} ,
height=\figheight,
legend cell align={left},
legend columns=-1,
legend style={
  nodes={scale=0.9},
  fill opacity=0.8,
  draw opacity=1,
  text opacity=1,
  at={(1.20,1.40)},
  anchor=south west,
  draw=white!80!black
},
minor xtick={},
minor ytick={0.55,0.65,0.75,0.85,0.95,1.05},
tick align=outside,
tick pos=left,
width=\figwidth,
x grid style={white!69.0196078431373!black},
xmajorgrids,
xlabel={Task},
xmin=0.8, xmax=5.2,
xtick style={color=black},
xtick={1,2,3,4,5},
xticklabel style={font=\tiny, inner sep=1pt},
xticklabels={1,2,3,4,5},
y grid style={white!69.0196078431373!black},
ymajorgrids,
yminorgrids,
ylabel={Accuracy},
ymin=0.491, ymax=1.01,
ytick style={color=black},
ytick={0.5,0.6,0.7,0.8,0.9,1.0},
yticklabel style={font=\tiny, inner sep=0pt},
yticklabels={50, 60, 70, 80, 90, 100}
]
\addplot [color0, mark=*, mark size=1.5, mark options={solid}]
table {%
1 0.994017958641052
2 0.912761688232422
3 0.868394792079926
4 0.750747740268707
5 0.837986052036285
};\addlegendentry{Task 1}
\addplot [ color1, mark=*, mark size=1.5, mark options={solid}]
table {%
2 0.99297297000885
3 0.918378353118896
4 0.967027008533478
5 0.969729721546173
};\addlegendentry{Task 2}
\addplot [ color2, mark=*, mark size=1.5, mark options={solid}]
table {%
3 0.999067604541779
4 0.987412571907043
5 0.981818199157715
};\addlegendentry{Task 3}
\addplot [ color3, mark=*, mark size=1.5, mark options={solid}]
table {%
4 0.996513962745667
5 0.994521915912628
};\addlegendentry{Task 4}
\addplot [color4, mark=*, mark size=1.5, mark options={solid}]
table {%
5 0.990959346294403
};\addlegendentry{Task 5}



\input{Chapter4/figures/policy_illustrations/mnist/data_seed10/dqn_seed2/bubble_rs_dataset_seed_10_dqn_seed_2}




\nextgroupplot[title={Heur-LD (ACC: 90.77\%)} ,
height=\figheight,
legend cell align={left},
legend style={
  nodes={scale=0.9},
  fill opacity=0.8,
  draw opacity=1,
  text opacity=1,
  at={(3.72,0.28)},
  anchor=south west,
  draw=white!80!black
},
minor xtick={},
minor ytick={0.55,0.65,0.75,0.85,0.95,1.05},
tick align=outside,
tick pos=left,
width=\figwidth,
x grid style={white!69.0196078431373!black},
xmajorgrids,
xlabel={Task},
xmin=0.8, xmax=5.2,
xtick style={color=black},
xtick={1,2,3,4,5},
xticklabels={1,2,3,4,5},
xticklabel style={font=\tiny, inner sep=1pt},
y grid style={white!69.0196078431373!black},
ymajorgrids,
yminorgrids,
ylabel={Accuracy},
ymin=0.491, ymax=1.01,
ytick style={color=black},
ytick={0.5,0.6,0.7,0.8,0.9,1.0},
yticklabels={50, 60, 70, 80, 90, 100},
yticklabel style={font=\tiny, inner sep=0pt}
]
\addplot [color0, mark=*, mark size=1.5, mark options={solid}]
table {%
1 0.994017946161516
2 0.777168494516451
3 0.809571286141575
4 0.567298105682951
5 0.708374875373878
};
%\addlegendentry{Task 1}
\addplot [color1, mark=*, mark size=1.5, mark options={solid}]
table {%
2 0.994054054054054
3 0.970810810810811
4 0.792972972972973
5 0.855675675675676
};
%\addlegendentry{Task 2}
\addplot [color2, mark=*, mark size=1.5, mark options={solid}]
table {%
3 0.998601398601399
4 0.990675990675991
5 0.995804195804196
};
%\addlegendentry{Task 3}
\addplot [color3, mark=*, mark size=1.5, mark options={solid}]
table {%
4 0.99601593625498
5 0.989043824701195
};
%\addlegendentry{Task 4}
\addplot [color4, mark=*, mark size=1.5, mark options={solid}]
table {%
5 0.989452536413862
};
%\addlegendentry{Task 5}
%\addlegendentry{Task 5}




\input{Chapter4/figures/policy_illustrations/mnist/data_seed10/heuristic_local_drop/bubble_plot}

\end{groupplot}

\end{tikzpicture}
		\vspace{-4mm} % hack to get captions aligned..
		%\vspace{-2mm}
		\caption{Split FashionMNIST}
		\label{fig:rewards_fashionmnist_2envs}
	\end{subfigure}%
	\begin{subfigure}[t]{0.48\textwidth}
		\centering
		



\pgfplotsset{every axis title/.append style={at={(0.5,0.85)}}}
\pgfplotsset{every tick label/.append style={font=\scriptsize}}
\pgfplotsset{every major tick/.append style={major tick length=2pt}}
\pgfplotsset{every minor tick/.append style={minor tick length=1pt}}
\pgfplotsset{every axis x label/.append style={at={(0.5,-0.22)}}}
\pgfplotsset{every axis y label/.append style={at={(-0.30,0.5)}}}
\begin{tikzpicture}
\tikzstyle{every node}=[font=\scriptsize]
\definecolor{color0}{rgb}{0.12156862745098,0.466666666666667,0.705882352941177}
\definecolor{color1}{rgb}{1,0.498039215686275,0.0549019607843137}
\definecolor{color2}{rgb}{0.172549019607843,0.627450980392157,0.172549019607843}
\definecolor{color3}{rgb}{0.83921568627451,0.152941176470588,0.156862745098039}
\definecolor{color4}{rgb}{0.580392156862745,0.403921568627451,0.741176470588235}
\definecolor{color5}{rgb}{0.549019607843137,0.337254901960784,0.294117647058824}
\definecolor{color6}{rgb}{0.890196078431372,0.466666666666667,0.76078431372549}

\begin{groupplot}[group style={group size= 2 by 1, horizontal sep=1.00cm, vertical sep=1.20cm}]


\nextgroupplot[title=Seed 1,
height=\figheight,
legend cell align={left},
legend columns=-1,
legend style={
  fill opacity=0.8,
  draw opacity=1,
  text opacity=1,
  at={(0.10,1.29)},
  anchor=south west,
  draw=white!80!black
},
minor xtick={25, 75},
minor ytick={},
tick align=outside,
tick pos=left,
%title={Labels: [[2, 9], [6, 4], [0, 3], [1, 7], [8, 5]]},
width=\figwidth,
x grid style={white!69.0196078431373!black},
xlabel={Eval. Steps},
xminorgrids,
xmajorgrids,
xmin=-3.95, xmax=104.95,
xtick style={color=black},
xtick={-25,0,50,100,125},
xticklabels={-25,0,50,100,125},
y grid style={white!69.0196078431373!black},
%ylabel={ACC},
ymajorgrids,
ymin=0.854905004700025, ymax=0.962595010399818,
ytick style={color=black},
ytick={0.84,0.86,0.88,0.9,0.92,0.94,0.96,0.98},
yticklabels={84,86,88,90,92,94,96,98}
]
\addplot [semithick, color0]
table {%
1 0.924066670735677
2 0.911366661389669
3 0.938500006993612
4 0.941500008106232
5 0.945533339182536
6 0.943866674105326
7 0.947966674963633
8 0.95060000816981
9 0.951466675599416
10 0.947033341725667
11 0.947033341725667
12 0.951000010967255
13 0.951333343982697
14 0.951000010967255
15 0.951000010967255
16 0.951000010967255
17 0.950800009568532
18 0.955333344141642
19 0.951000010967255
20 0.951666676998138
21 0.945600012938182
22 0.952300012111664
23 0.949600013097127
24 0.9550000111262
25 0.951266678174337
26 0.957700010140737
27 0.930266670385997
28 0.945333341757456
29 0.918533333142598
30 0.94900000890096
31 0.926099995772044
32 0.919266664981842
33 0.942000007629395
34 0.905199992656708
35 0.93100000222524
36 0.920233333110809
37 0.919999996821086
38 0.93100000222524
39 0.919999996821086
40 0.931233338514964
41 0.916199998060862
42 0.931466674804687
43 0.931233338514964
44 0.920233333110809
45 0.931233338514964
46 0.920233333110809
47 0.930466667811076
48 0.942466680208842
49 0.931233338514964
50 0.942700016498566
51 0.931466674804687
52 0.931466666857402
53 0.942700016498566
54 0.942466680208842
55 0.930466667811076
56 0.942700016498566
57 0.942466680208842
58 0.94270000855128
59 0.930466667811076
60 0.920233333110809
61 0.930466667811076
62 0.920233325163523
63 0.931466674804687
64 0.931466674804687
65 0.931466674804687
66 0.941700009504954
67 0.941700009504954
68 0.920233333110809
69 0.941700009504954
70 0.931466674804687
71 0.942700016498566
72 0.942700016498566
73 0.930466667811076
74 0.941700009504954
75 0.930100007851919
76 0.941700009504954
77 0.930466667811076
78 0.930466667811076
79 0.930466667811076
80 0.930100007851919
81 0.930466667811076
82 0.941700009504954
83 0.930466667811076
84 0.940700002511342
85 0.941700009504954
86 0.941700009504954
87 0.929466660817464
88 0.939699995517731
89 0.930466667811076
90 0.940700002511342
91 0.929466660817464
92 0.939699995517731
93 0.930466667811076
94 0.919233326117198
95 0.929466660817464
96 0.929466660817464
97 0.920233333110809
98 0.920233333110809
99 0.941700009504954
100 0.919233326117198
};
%\addlegendentry{a2c, gae, envs=10, lr=0.0003, act=tanh}
\addplot [semithick, color1]
table {%
1 0.900799989700317
2 0.912233340740204
3 0.91263333161672
4 0.928866664568583
5 0.908866663773855
6 0.90889999071757
7 0.927533332506816
8 0.891566665967305
9 0.899466673533122
10 0.914566667874654
11 0.937033327420553
12 0.910133334000905
13 0.914766669273376
14 0.907233341534933
15 0.934766670068105
16 0.927966666221618
17 0.921366663773855
18 0.93866666952769
19 0.940600005785624
20 0.939800004164378
21 0.89859999815623
22 0.936033336321513
23 0.93260000149409
24 0.941900006930033
25 0.944600009918213
26 0.859800004959106
27 0.941400003433227
28 0.924700001875559
29 0.941266667842865
30 0.921700004736583
31 0.888233335812887
32 0.931266665458679
33 0.904833336671193
34 0.930966671307882
35 0.895066666603088
36 0.925733335812887
37 0.894133337338765
38 0.911666671435038
39 0.931633329391479
40 0.922200004259745
41 0.9121333360672
42 0.938533333937327
43 0.909999998410543
44 0.90293333530426
45 0.9064333319664
46 0.916133332252502
47 0.922466667493184
48 0.922466667493184
49 0.922466667493184
50 0.916199998060862
51 0.933999999364217
52 0.937000000476837
53 0.893700003623962
54 0.920366668701172
55 0.940266664822896
56 0.906733334064484
57 0.932999996344248
58 0.943666660785675
59 0.923433331648509
60 0.940099994341532
61 0.929866659641266
62 0.941633331775665
63 0.94883333047231
64 0.89816666841507
65 0.925533334414164
66 0.905799996852875
67 0.896833332379659
68 0.890966665744782
69 0.927433335781097
70 0.943099999427795
71 0.941200006008148
72 0.922533329327901
73 0.94256667693456
74 0.913033334414164
75 0.942700012524923
76 0.94699999888738
77 0.938700000445048
78 0.936299999554952
79 0.914599998792013
80 0.915766664346059
81 0.908499999841054
82 0.927266673247019
83 0.927900008360545
84 0.928466665744781
85 0.920866676171621
86 0.928833325703939
87 0.928266668319702
88 0.921433333555857
89 0.933300006389618
90 0.932666671276093
91 0.928233337402344
92 0.928266668319702
93 0.928233337402344
94 0.891733336448669
95 0.913099996248881
96 0.913099996248881
97 0.927066663901011
98 0.927466662724813
99 0.929166666666667
100 0.929166666666667
};
%\addlegendentry{DQN, n train envs=30}
\addplot [semithick, color2]
table {%
1 0.938733333333333
100 0.938733333333333
};
%\addlegendentry{Random}
\addplot [semithick, color3]
table {%
1 0.941
100 0.941
};
%\addlegendentry{ETS}
\addplot [semithick, color4]
table {%
1 0.9501
100 0.9501
};
%\addlegendentry{Heur-GD}
\addplot [semithick, color5, dash pattern=on 1pt off 3pt on 3pt off 3pt]
table {%
1 0.9003
100 0.9003
};
%\addlegendentry{Heur-LD}
\addplot [semithick, color6, dashed]
table {%
1 0.9409
100 0.9409
};
%\addlegendentry{Heur-AT}



\nextgroupplot[title=Seed 19,
height=\figheight,
legend cell align={left},
legend style={
  fill opacity=0.8,
  draw opacity=1,
  text opacity=1,
  at={(0.5,0.09)},
  anchor=south,
  draw=white!80!black
},
minor xtick={25, 75},
minor ytick={},
tick align=outside,
tick pos=left,
%title={Labels: [[1, 7], [9, 6], [8, 4], [3, 0], [2, 5]]},
width=\figwidth,
x grid style={white!69.0196078431373!black},
xlabel={Eval. Steps},
xminorgrids,
xmajorgrids,
xmin=-3.95, xmax=104.95,
xtick style={color=black},
xtick={-25,0,50,100,125},
xticklabels={-25,0,50,100,125},
y grid style={white!69.0196078431373!black},
%ylabel={ACC},
ymajorgrids,
ymin=0.731755009293556, ymax=0.822944995760918,
ytick style={color=black},
ytick={0.72,0.74,0.76,0.78,0.8,0.82,0.84},
yticklabels={72,74,76,78,80,82,84}
]
\addplot [semithick, color0]
table {%
	1 0.770119993686676
	2 0.768799998760223
	3 0.773239998817444
	4 0.778700003623962
	5 0.763700001239777
	6 0.765620002746582
	7 0.773039996623993
	8 0.780080006122589
	9 0.778359997272491
	10 0.781360001564026
	11 0.790859997272491
	12 0.786640000343323
	13 0.780300004482269
	14 0.786000001430512
	15 0.78067999124527
	16 0.778220002651214
	17 0.792040002346039
	18 0.788360004425049
	19 0.793059999942779
	20 0.780739998817444
	21 0.791559996604919
	22 0.789240002632141
	23 0.790059998035431
	24 0.779840002059937
	25 0.78672000169754
	26 0.777060000896454
	27 0.796420001983643
	28 0.79194000005722
	29 0.77565999507904
	30 0.795520000457764
	31 0.794500002861023
	32 0.782959997653961
	33 0.78746000289917
	34 0.782259993553162
	35 0.779920008182526
	36 0.783540003299713
	37 0.799879999160767
	38 0.791479997634888
	39 0.796820001602173
	40 0.800940001010895
	41 0.799299998283386
	42 0.795419998168945
	43 0.800999999046326
	44 0.78731999874115
	45 0.800119996070862
	46 0.782660000324249
	47 0.800239996910095
	48 0.811479997634888
	49 0.803199999332428
	50 0.796839997768402
	51 0.793839998245239
	52 0.791319997310638
	53 0.798840003013611
	54 0.79923999786377
	55 0.789359996318817
	56 0.794979996681213
	57 0.789359996318817
	58 0.795139997005463
	59 0.796240003108978
	60 0.796199998855591
	61 0.790579998493195
	62 0.790579998493195
	63 0.79231999874115
	64 0.792379996776581
	65 0.790579998493195
	66 0.795320000648499
	67 0.787279994487762
	68 0.785639996528625
	69 0.793399999141693
	70 0.791739997863769
	71 0.785639996528625
	72 0.783559994697571
	73 0.784979999065399
	74 0.785719997882843
	75 0.792219998836517
	76 0.783559997081757
	77 0.796579995155335
	78 0.795459997653961
	79 0.786000001430511
	80 0.79173999786377
	81 0.787559993267059
	82 0.801079993247986
	83 0.794759993553162
	84 0.792419996261597
	85 0.787999997138977
	86 0.79191999912262
	87 0.79293999671936
	88 0.780859999656677
	89 0.782959997653961
	90 0.790160000324249
	91 0.786640002727509
	92 0.780199999809265
	93 0.787580006122589
	94 0.788820004463196
	95 0.774800002574921
	96 0.775460002422333
	97 0.788640003204346
	98 0.785080001354218
	99 0.795279996395111
	100 0.799079999923706
};
%\addlegendentry{DQN}
\addplot [semithick, color1]
table {%
	1 0.758300001621246
	2 0.735900008678436
	3 0.775179996490479
	4 0.8
	5 0.807999997138977
	6 0.810799999237061
	7 0.804479999542236
	8 0.810399990081787
	9 0.795520000457764
	10 0.8
	11 0.804479999542236
	12 0.797199997901916
	13 0.792000002861023
	14 0.801679997444153
	15 0.797199997901916
	16 0.81079999923706
	17 0.802800002098084
	18 0.797199997901917
	19 0.805199995040893
	20 0.8
	21 0.807999997138977
	22 0.80239999294281
	23 0.801679997444153
	24 0.81319999217987
	25 0.792000002861023
	26 0.810399990081787
	27 0.807599987983704
	28 0.806879992485046
	29 0.807599987983704
	30 0.813199992179871
	31 0.813199992179871
	32 0.810399990081787
	33 0.794399995803833
	34 0.797119994163513
	35 0.813199992179871
	36 0.807999997138977
	37 0.807599987983704
	38 0.815999994277954
	39 0.796239995956421
	40 0.799599990844727
	41 0.807599987983704
	42 0.80239999294281
	43 0.804239993095398
	44 0.807599987983704
	45 0.801279988288879
	46 0.801439990997314
	47 0.804239993095398
	48 0.801279988288879
	49 0.801439990997314
	50 0.804119989871979
	51 0.806919991970062
	52 0.810399990081787
	53 0.802639994621277
	54 0.80479998588562
	55 0.801439990997314
	56 0.801439990997314
	57 0.807599987983704
	58 0.802319989204407
	59 0.799599990844727
	60 0.804119989871979
	61 0.80479998588562
	62 0.804119989871979
	63 0.80479998588562
	64 0.799839992523193
	65 0.803439993858337
	66 0.801439990997314
	67 0.80479998588562
	68 0.80479998588562
	69 0.80479998588562
	70 0.80479998588562
	71 0.80479998588562
	72 0.80479998588562
	73 0.80479998588562
	74 0.80479998588562
	75 0.80479998588562
	76 0.801439990997314
	77 0.801639993190765
	78 0.80479998588562
	79 0.802319989204407
	80 0.80479998588562
	81 0.80479998588562
	82 0.80479998588562
	83 0.80479998588562
	84 0.80479998588562
	85 0.80479998588562
	86 0.80479998588562
	87 0.80479998588562
	88 0.802319989204407
	89 0.80479998588562
	90 0.80479998588562
	91 0.80479998588562
	92 0.80479998588562
	93 0.80479998588562
	94 0.80479998588562
	95 0.80479998588562
	96 0.80479998588562
	97 0.80479998588562
	98 0.802319989204407
	99 0.80479998588562
	100 0.80479998588562
};
%\addlegendentry{A2C}
\addplot [semithick, color2]
table {%
	1 0.79190000295639
	2 0.79190000295639
	3 0.79190000295639
	4 0.79190000295639
	5 0.79190000295639
	6 0.79190000295639
	7 0.779560005664825
	8 0.777020008563995
	9 0.789179999828339
	10 0.792739999294281
	11 0.792400002479553
	12 0.788620002269745
	13 0.792400002479553
	14 0.792400002479553
	15 0.794879999160767
	16 0.791360001564026
	17 0.79735999584198
	18 0.797980000972748
	19 0.800459997653961
	20 0.797980000972748
	21 0.79735999584198
	22 0.79735999584198
	23 0.79735999584198
	24 0.802319989204407
	25 0.799839992523193
	26 0.799839992523193
	27 0.799839992523193
	28 0.799839992523193
	29 0.800459997653961
	30 0.79735999584198
	31 0.801080002784729
	32 0.797980000972748
	33 0.794639999866486
	34 0.797119996547699
	35 0.797119996547699
	36 0.79581999540329
	37 0.796059994697571
	38 0.796059994697571
	39 0.799399995803833
	40 0.796059994697571
	41 0.796059994697571
	42 0.799399995803833
	43 0.796059994697571
	44 0.796059994697571
	45 0.796059994697571
	46 0.793579998016357
	47 0.793579998016357
	48 0.793579998016357
	49 0.793579998016357
	50 0.796059994697571
	51 0.793339998722076
	52 0.789999997615814
	53 0.79581999540329
	54 0.792380003929138
	55 0.789040002822876
	56 0.792620003223419
	57 0.789280002117157
	58 0.789280002117157
	59 0.79175999879837
	60 0.79175999879837
	61 0.79175999879837
	62 0.789280002117157
	63 0.789040002822876
	64 0.795099999904633
	65 0.7957200050354
	66 0.795099999904633
	67 0.789280002117157
	68 0.7957200050354
	69 0.792980005741119
	70 0.792980005741119
	71 0.792980005741119
	72 0.793419997692108
	73 0.795460002422333
	74 0.792980005741119
	75 0.795460002422333
	76 0.795460002422333
	77 0.792820003032684
	78 0.792980005741119
	79 0.801900005340576
	80 0.798800003528595
	81 0.792980005741119
	82 0.799420008659363
	83 0.792980005741119
	84 0.799420008659363
	85 0.792980005741119
	86 0.792980005741119
	87 0.795460002422333
	88 0.799420008659363
	89 0.792980005741119
	90 0.792980005741119
	91 0.801660006046295
	92 0.792740006446838
	93 0.792980005741119
	94 0.799420008659363
	95 0.792740006446838
	96 0.790700001716614
	97 0.801900005340576
	98 0.798200001716614
	99 0.792980005741119
	100 0.800380005836487
};
%\addlegendentry{SAC}
\addplot [semithick, color3]
table {%
	1 0.74888
	100 0.74888
};
%\addlegendentry{Random}
\addplot [semithick, color4]
table {%
	1 0.7767
	100 0.7767
};
%\addlegendentry{ETS}
\addplot [semithick, color5]
table {%
	1 0.7787
	100 0.7787
};
%\addlegendentry{Heur-GD}
\addplot [semithick, color6, dash pattern=on 1pt off 3pt on 3pt off 3pt]
table {%
	1 0.775
	100 0.775
};
%\addlegendentry{Heur-LD}
\addplot [semithick, white!49.8039215686275!black, dashed]
table {%
	1 0.7787
	100 0.7787
};
%\addlegendentry{Heur-AT}



\end{groupplot}

\end{tikzpicture}
		\caption{Split notMNIST}
		\label{fig:rewards_notmnist_2envs}
	\end{subfigure}
	\vspace{-1mm}
	\caption{Performance progress measured in ACC (\%) for all methods in the 2 test environments with (a) {\bf Split FashionMNIST} and (b) {\bf Split notMNIST} in the New Dataset experiment. We plot the performance progress for the RL algorithms for 100 evaluation steps equidistantly distributed over the training episodes. The performance for the Random, ETS, and Heuristic scheduling baselines are plotted as straight lines.  }
	\label{fig:rewards_new_dataset_2envs}
	\vspace{-2mm}
\end{figure}

\vspace{-3mm}
\paragraph{Generalization to New Datasets.}
We show that the learned replay scheduling policy is capable of generalizing to CL environments with new datasets unseen in the training environments. We perform two sets of experiments:
\begin{enumerate}[topsep=1pt,noitemsep]
	\item Train with environments generated with Split MNIST and FashionMNIST, and test on environments generated with Split notMNIST.
	\item Train with environments generated with Split MNIST and notMNIST, and test on environments generated with Split FashionMNIST.
\end{enumerate}
%, 1) train with environments generated with Split MNIST and FashionMNIST and test on environments generated with Split notMNIST, and 2) train with environments generated with Split MNIST and notMNIST and test on environments generated with Split FashionMNIST.
%We perform experiments where the policies are applied in test environments containing either Split notMNIST or FashionMNIST datasets. To foster generalization, we enhance the diversity of the training environments by generating the environments with two types of datasets. For instance, when Split notMNIST is the target dataset, we train on environments with Split MNIST and FashionMNIST. Similarly, we use Split notMNIST for training when the test environments contain Split FashionMNIST. 
Table \ref{tab:average_ranking_rl_experiment} shows the average ranking for DQN, A2C, and the baselines when generalization to test environments with new datasets. We observe that both A2C and DQN successfully generalize to Split notMNIST environments outperforming all baselines. 
However, the learned policies have difficulties to generalize to Split FashionMNIST environments, which could be due to 
high variations in the dynamics between training and test environments.
To gain insights, we show the performance progress for two test environments with Split notMNIST and Split FashionMNIST datasets in Figure \ref{fig:rewards_new_dataset_2envs}(a) and (b) respectively (see Figure \ref{fig:policy_rewards_notmnist_new_dataset_paperD} and \ref{fig:policy_rewards_fashionmnist_new_dataset_paperD} in Appendix \ref{paperD:app:additional_experimental_results} for all test environments). 
In Figure \ref{fig:rewards_new_dataset_2envs}(b), we observe that both A2C and DQN has difficulties in converging to efficient replay scheduling policies, where A2C even collapses to a suboptimal replay schedule in second test environment (Seed 9).  
A possible reason could be that the policy encounters state transition dynamics in the test environments which has not been experienced during training, which makes generalization difficult for both DQN and A2C.
%The policy may exhibit state transition dynamics which has not been experienced during training, which makes generalization difficult for both DQN and A2C.
Potentially, the performance could be improved by generating more training environments for the agent to exhibit more variations in the CL scenarios or by using other advanced RL methods which may generalize better~\citeD{D:igl2019generalization}.







\section{Discussion}\label{paperD:sec:discussion}

In this section, we provide a discussion of the experimental results in Section \ref{paperD:sec:experiments}. Our goal was to answer whether it was possible to learn a policy for replay scheduling that can be applied in new CL scenarios without additional training cost. We evaluate the policy on two CL scenarios where (i) the training and test environments share the same dataset but the task splits are different, and (ii) the datasets in the training and test environments are completely different. We observed that the DQN receives a similar average ranking of the performance in the test environments as the best heuristic scheduling baselines in both CL scenarios. Furthermore, we analyzed the learned policies by visualizing them together with the test accuracy progression for each task to illustrate the benefits of the DQN. We then showed an example in Figure \ref{fig:fashionmnist_policy_illustration_dqn_vs_heur_gd} where learning the replay scheduling policy indeed opens up for scheduling behaviors similar to human learning insights, such as spaced repetition, that would be difficult to create a heuristic scheduling rule for. We conclude that our results indicate that using RL for learning policies for replay scheduling is a promising direction for making replay scheduling in CL practical for real-world applications where training at test time is prohibited.   

Several challenges need to be tackled to make the best out of our approach to learn replays scheduling policies. As the policy is learned with RL, the training requires careful hyperparameter tuning to succeed to learn a policy that generalizes well to the studied CL scenarios. We will therefore perform more hyperparameter searches for the DQN as well as increase the number of training environments since adding more experience for training should intuitively improve the generalization capabilities. Moreover, each step in the environment requires training the classifier on the current CL task which makes each environment step computationally expensive. Hence, we will apply RL algorithms that can be easier to tune, e.g., PPO~\citeD{D:schulman2017proximal} and A2C~\citeD{D:mnih2016asynchronous}, and investigate whether these can be more sample-efficient than DQN for our application. 

Another challenge is how to learn policies for replay scheduling in CL scenarios with long task-horizons. Continuous action spaces are a scalable option that would be appropriate for representing the task proportions since the action space would simply add one dimension per seen task. 
However, as learning policies for continuous actions are considered to be more difficult than learning how to take discrete actions, we could also look into various ways for discretizing continuous action spaces into finite sets of actions.    

\MK{Add a paragraph for limitations.}

%\begin{itemize}
%    \item Continuous action space for enabling longer task-horizon 
%    \item large action spaces seems to need lots of exploration, transitions are expensive in a CL environment so need more sample-efficient methods here
%    \item potential conflicting environments that disturbs the policy learning
%    \item more hyperparameter tuning and/or other RL algorithm
%\end{itemize}



\section{Conclusions}\label{paperD:sec:conclusions}

We proposed a novel method for learning policies for replay scheduling in CL which previously have been unavailable. The goal of our method is to learn a general policy that can be applied to any CL scenario for replay scheduling without requiring additional training costs at test time. We used RL for learning the policy where we employed DQNs for selecting which tasks to replay at different times. In the experiments, we showed that the replay schedules from the learned policy perform similarly to carefully tuned heuristic scheduling baselines. Furthermore, we gained more insights by visualizing the learned replay schedules and observed that the policy is capable of learning scheduling behaviors similar to human education techniques such as spaced repetition. In future work, we will investigate how alternative RL algorithms than DQN are capable of learning policies in this setting as well as experimenting with continuous action spaces to enable scaling to longer task-horizons. We hope that our work can encourage more research within this CL setting where the limitations are on compute rather than memory storage which stems well with the needs in real-world CL scenarios.  



\begin{appendices}

\section*{Appendix}

This supplementary material is structured as follows:
\begin{itemize}
    \item Appendix \ref{paperD:app:experimental_settings}: Details of the experimental settings on the continual learning (CL) part, hyperparameter settings for the heuristic baselines, and hyperparameters for the Deep Q-Networks (DQNs). 
    
    \item Appendix \ref{paperD:app:construction_of_the_action_space}: Details on the construction of the action space with task proportions. 
    
    \item Appendix \ref{paperD:app:task_split_tables}: Tables over the task splits for Split FashionMNIST and Split CIFAR-10.  
\end{itemize}


\section{Experimental Settings}\label{paperD:app:experimental_settings}

In this section, we describe the full details of the experimental settings used in this paper. We first provide details on hyperparameter settings for the CL and RL parts, including hyperparameters for DQN~\citeD{D:mnih2013playing, D:mnih2015human} and A2C~\citeD{D:mnih2016asynchronous}, followed by description of the heuristic baselines, and the ranking method for assesing the generalization capability of the learned policy.  



\vspace{-3mm}
\paragraph{Datasets.}
We generate CL environments with four datasets commonly used in the CL literature. 
Split MNIST~\citeD{D:zenke2017continual} is a variant of the MNIST~\citeD{D:lecun1998gradient} dataset where the classes have been divided into 5 tasks incoming in the order 0/1, 2/3, 4/5, 6/7, and 8/9. Split FashionMNIST~\citeD{D:xiao2017fashion} is of similar size to MNIST and consists of grayscale images of different clothes, where the classes have been divided into the 5 tasks T-shirt/Trouser, Pullover/Dress, Coat/Sandals, Shirt/Sneaker, and Bag/Ankle boots. Similar to MNIST, Split notMNIST~\citeD{D:bulatov2011notMNIST} consists of 10 classes of the letters A-J with various fonts, where the classes are divided into the 5 tasks A/B, C/D, E/F, G/H, and I/J. We use training/test split provided by \citeD{D:ebrahimi2020adversarial} for Split notMNIST. 
In Split CIFAR-10~\citeD{D:krizhevsky2009learning}, the 10 classes are divided into 5 tasks with 2 classes for each task. 

\vspace{-3mm}
\paragraph{CL Network Architectures.} We use a 2-layer MLP with 256 hidden units and ReLU activation for Split MNIST, Split FashionMNIST, and Split notMNIST. For Split CIFAR-10, we use the ConvNet architecture used in \citeD{D:adel2019continual, D:schwarz2018progress, D:vinyals2016matching}, which consists of four 3x3 convolutional blocks, i.e. convolutional layer followed by batch normalization~\citeD{D:ioffe2015batch}, with 64 filters, ReLU activations, and 2x2 Max-pooling. We use a multi-head output layer for each dataset and assume task labels are available at test time for selecting the correct output head related to the task. 

\vspace{-3mm}
\paragraph{CL Hyperparameters.} We train all networks with the Adam optimizer~\citeD{D:kingma2014adam} with learning rate $\eta = 0.001$ and hyperparameters $\beta_1 = 0.9$ and $\beta_2 = 0.999$. Note that the learning rate for Adam is not reset before training on a new task. Next, we give details on number of training epochs and batch sizes specific for each dataset:
\begin{itemize}[topsep=1pt,noitemsep]
    \item Split MNIST: 10 epochs/task, batch size 128.
    \item Split FashionMNIST: 10 epochs/task, batch size 128.
    \item Split notMNIST: 20 epochs/task, batch size 128.
    \item Split CIFAR-100: 20 epochs/task, batch size 256.
\end{itemize}

\vspace{-3mm}
\paragraph{Generating CL Environments.} We generate multiple CL environments with pre-set random seeds for initializing the network parameters $\vphi$ and shuffling the task order. The pre-set random seeds are in the range $0-49$, such that we have 50 environments for each dataset. We shuffle the task order by permuting the class order and then split the classes into 5 pairs (tasks) with 2 classes/pair. For environments with seed $0$, we keep the original task order in the dataset. 
Taking a step at task $t$ in the CL environments involves training the CL network on the $t$-th dataset with a replay memory $\gM_t$ from the discrete action space described in Section \ref{paperD:app:action_space}. Therefore, to speed up the experiments with the RL algorithms, we run a breadth-first search (BFS) through the discrete action space and save the classification results for re-use during policy learning. Note that the action space has 1050 possible paths of replay schedules for the datasets with $T=5$ tasks, which makes the environment generation time-consuming. Hence, we only generate environments where the replay memory size $M=10$ have been used, and leave analysis of different memory sizes as future work. The CL environments that we used will be provided in the public code. 

\vspace{-3mm}
\paragraph{Computational Cost.} 
All experiments were performed on one NVIDIA GeForce RTW 2080Ti on an internal GPU cluster. Generating a CL environment for one seed with Split MNIST took on around 9.5 hours averaged over 10 runs of BFS. Similarly for Split CIFAR-10, generating one CL environment took on average 16.1 hours. Note that BFS runs 1050 iterations in total for all 5-task datasets. The wall clock time for ETS on Split MNIST was around 1.5 minutes. 


\vspace{-3mm}
\paragraph{DQN and A2C Architectures.}
The input layer has size $T-1$ where each unit is inputting the task performances since the states are represented by the validation accuracies $s_t = [A_{t, 1}^{(val)}, ..., A_{t, t}^{(val)}, 0, ..., 0]$. The current task can therefore be determined by the number of non-zero state inputs. 
The output layer has 35 units representing the possible actions at $T=5$ with the discrete action space we have constructed in Appendix \ref{paperD:app:action_space}. %Section \ref{paperD:sec:methodology}. 
We use action masking on the output units to prevent the network from selection invalid actions for constructing the replay memory at the current task. The DQN is a 2-layer MLP with 512 hidden units and ReLU activations. For A2C, we use separate networks for parameterizing the policy and the value function, where both networks are 2-layer MLPs with 64 hidden units of Tanh activations. 

\vspace{-3mm}
\paragraph{DQN and A2C Hyperparameters.}
We provide the hyperparameters for the both DQN and A2C in Table \ref{tab:dqn_hyperparameters_new_task_orders}-\ref{tab:a2c_hyperparameters_new_dataset}. Table \ref{tab:dqn_hyperparameters_new_task_orders} and \ref{tab:a2c_hyperparameters_new_task_orders} includes the hyperparameters on the New Task Order experiment for DQN and A2C respectively, while Table \ref{tab:dqn_hyperparameters_new_dataset} and \ref{tab:a2c_hyperparameters_new_dataset} includes the hyperparameters on the New Dataset experiment for DQN and A2C respectively. Regarding the training environments in Table \ref{tab:dqn_hyperparameters_new_dataset} and \ref{tab:a2c_hyperparameters_new_dataset}, we use two different datasets in the training environments to increase the diversity. When Spit notMNIST is for testing, half the amount of training environments are using Split MNIST and the other half uses Split FashionMNIST. For example, in Table \ref{tab:a2c_hyperparameters_new_dataset}, A2C uses 10 training environments which means that there are 5 Split MNIST environments and 5 Split FashionMNIST environments. Similarly, half the amount of training environments are using Split MNIST and the other half uses Split notMNIST when the testing environments uses Split FashionMNIST.   

\vspace{-3mm}
\paragraph{Implementations.} 
The code for DQN was adapted from OpenAI baselines~\citeD{D:dhariwal2017baselines} and the PyTorch~\citeD{D:paszke2019pytorch} tutorial on DQN {\small \url{https://pytorch.org/tutorials/intermediate/reinforcement_q_learning.html}}. For A2C, we followed the implementations released by Kostrikov~\citeD{D:kostrikov2018pytorchrl} and Igl \etal~\citeD{D:igl2020transient}. We make the code publicly available upon acceptance. 

%\clearpage

\begin{table}[h]
\small
\centering
\caption{DQN hyperparameters for the experiments on {\bf New Task Orders} in Section \ref{paperD:sec:results_on_policy_generalization}.  
}
\vspace{-2mm}
\label{tab:dqn_hyperparameters_new_task_orders}
\resizebox{0.99\textwidth}{!}{
\begin{tabular}{l c c c}
\toprule
 {\bf Hyperparameters} & {\bf Split MNIST} & {\bf Split FashionMNIST} & {\bf Split CIFAR-10} \\
 \midrule
 Training Environments & 30 & 20 & 10 \\
 Learning Rate  & 0.0001 & 0.0003 & 0.0003 \\
 Optimizer      & Adam & Adam & Adam \\
 Buffer Size    & 10k & 10k & 10k  \\
 Target Update per step  & 500 & 500 & 500 \\
 Batch Size     & 32 & 32 & 32 \\
 Discount Factor $\gamma$ & 1.0 & 1.0 & 1.0 \\
 Exploration Start $\epsilon_{start}$ & 1.0 & 1.0 & 1.0 \\
 Exploration Final $\epsilon_{final}$ & 0.02 & 0.02 & 0.02 \\
 Exploration Annealing (episodes) & 2.5k & 2.5k & 2.5k \\
 Training Episodes & 10k & 10k & 10k \\
\bottomrule
\end{tabular}
}
\end{table}

\begin{table}[h]
\small
\centering
\caption{A2C hyperparameters for the experiments on {\bf New Task Orders} in Section \ref{paperD:sec:results_on_policy_generalization}.  
}
\vspace{-2mm}
\label{tab:a2c_hyperparameters_new_task_orders}
\begin{tabular}{l c c c}
\toprule
 {\bf Hyperparameters} & {\bf Split MNIST} & {\bf Split FashionMNIST} & {\bf Split CIFAR-10} \\
 \midrule
 Training Environments & 10 & 10 & 10 \\
 Learning Rate  & 0.0001 & 0.0003 & 0.00003 \\
 Optimizer      & RMSProp & RMSProp & RMSProp \\
 Gradient Clipping & 0.5 & 0.5 & 0.5 \\
 GAE parameter $\lambda$ & 0.95 & 0.95 & 0.95 \\
 VF coefficient & 0.5 & 0.5 & 0.5 \\
 Entropy coefficient & 0.01 & 0.01 & 0.01 \\
 Number of steps $n_{steps}$ & 5 & 5 & 5 \\
 Discount Factor $\gamma$ & 1.0 & 1.0 & 1.0 \\
 Training Episodes & 100k & 100k & 100k \\
\bottomrule
\end{tabular}
\end{table}

\begin{table}[h]
\small
\centering
\caption{DQN hyperparameters for the experiments on {\bf New Dataset} in Section \ref{paperD:sec:results_on_policy_generalization}. Split notMNIST and Split FashionMNIST indicate the dataset used in the test environments. 
}
\vspace{-2mm}
\label{tab:dqn_hyperparameters_new_dataset}
\begin{tabular}{l c c}
\toprule
 {\bf Hyperparameters} & {\bf Split notMNIST} & {\bf Split FashionMNIST}  \\
 \midrule
 Training Environments & 30 & 30  \\
 Learning Rate  & 0.0001 & 0.0001  \\
 Optimizer      & Adam & Adam  \\
 Buffer Size    & 10k & 10k   \\
 Target Update per step  & 500 & 500  \\
 Batch Size     & 32 & 32  \\
 Discount Factor $\gamma$ & 1.0 & 1.0  \\
 Exploration Start $\epsilon_{start}$ & 1.0 & 1.0 \\
 Exploration Final $\epsilon_{final}$ & 0.02 & 0.02  \\
 Exploration Annealing (episodes) & 2.5k & 2.5k  \\
 Training Episodes & 10k & 10k  \\
\bottomrule
\end{tabular}
\end{table}

\begin{table}[h]
\small
\centering
\caption{A2C hyperparameters for the experiments on {\bf New Dataset} in Section \ref{paperD:sec:results_on_policy_generalization}. Split notMNIST and Split FashionMNIST indicate the dataset used in the test environments. 
}
\vspace{-2mm}
\label{tab:a2c_hyperparameters_new_dataset}
\begin{tabular}{l c c c}
\toprule
 {\bf Hyperparameters} & {\bf Split notMNIST} & {\bf Split FashionMNIST}  \\
 \midrule
 Training Environments & 10 & 10  \\
 Learning Rate  & 0.0001 & 0.0003  \\
 Optimizer      & RMSProp & RMSProp  \\
 Gradient Clipping & 0.5 & 0.5  \\
 GAE parameter $\lambda$ & 0.95 & 0.95  \\
 VF coefficient & 0.5 & 0.5 \\
 Entropy coefficient & 0.01 & 0.01  \\
 Number of steps $n_{steps}$ & 5 & 5  \\
 Discount Factor $\gamma$ & 1.0 & 1.0  \\
 Training Episodes & 100k & 100k  \\
\bottomrule
\end{tabular}
\end{table}


%\clearpage

\subsection*{Heuristic Scheduling Baselines}\label{paperD:app:heuristic_scheduling_baselines}

We implemented three heuristic scheduling baselines to compare against our proposed methods. These heuristics are based on the intuition of re-learning tasks when they have been forgotten. We keep a validation set for each task to determine whether any task should be replayed by comparing the validation accuracy against a hand-tuned threshold. If the validation accuracy is below the threshold, then the corresponding task is replayed. Let $A_{t, i}^{(val)}$ be the validation accuracy for task $t$ evaluated at time step $i$. The threshold is set differently in each of the baselines:
\begin{itemize}[topsep=1pt,noitemsep]%[leftmargin=*, topsep=0pt]
    \item {\bf Heuristic Global Drop (Heur-GD).} Heuristic policy that replays tasks  
    with validation accuracy below a certain threshold proportional to the best achieved validation accuracy on the task. The best achieved validation accuracy for task $i$ is given by $A_{t, i}^{(best)} = \max\{(A_{1, i}^{(val)}, \dots, A_{t, i}^{(val)})\}$. Task $i$ is replayed if $A_{t, i}^{(val)} < \tau A_{t, i}^{(best)}$ where $\tau \in [0, 1]$ is a ratio representing the degree of how much the validation accuracy of a task is allowed to drop. 

    \item {\bf Heuristic Local Drop (Heur-LD).} Heuristic policy that replays tasks with validation accuracy below a threshold proportional to the previous achieved validation accuracy on the task. Task $i$ is replayed if $A_{t, i}^{(val)} < \tau A_{t-1, i}^{(val)}$ where $tau$ again represents the degree of how much the validation accuracy of a task is allowed to drop. 

    \item {\bf Heuristic Accuracy Threshold (Heur-AT).} Heuristic policy that replays tasks with validation accuracy below a fixed threshold. Task $i$ is replayed if if $A_{t, i}^{(val)} < \tau$ where $\tau \in [0, 1]$ represents the least tolerated accuracy before we need to replay the task. 
    %The third heuristic has a fixed threshold $\tau \in [0, 1]$ on the accuracy such that task $i$ is replayed after learning task $t$ if $A_{t, i} < \tau$.
\end{itemize}
The replay memory is filled with $M/k$ samples from each selected task, where $k$ is the number of tasks that need to be replayed according to their decrease in validation accuracy. We skip replaying any tasks if no tasks are selected for replay, i.e., $k=0$. 


\vspace{-3mm}
\paragraph{Grid search for $\tau$.}
We performed a grid search for the parameter $\tau$ for the three heuristic scheduling baselines for each experiment to compare against the learned replay scheduling policies. The validation set consists of 15\% of the training data and is identical to the validation set used for learning the RL policies. 
We select the parameter based on ACC scores achieved in the same number of training environments used by either DQN or A2C. The search range we use is $\tau \in \{0.90, 0.95, 0.999\}$. In Table \ref{tab:grid_search_heuristics_multiple_cl_environments}, we show the selected parameter value of $\tau$ and the number of environments used for selecting the value for each method and experiment in Section \ref{sec:results_on_policy_generalization}. The same parameters are used to generate the results on the heuristics in Table \ref{tab:average_ranking_rl_experiment}. 

\begin{table}[t]
%\footnotesize%\small
\centering
\caption{The threshold parameter $\tau$ used in the heuristic scheudling baselines Heuristic Global Drop (Heur-GD), Heuristic Local Drop (Heur-LD), and Heuristic Accuracy Threshold (Heur-AT). The search range is $\tau \in \{0.90, 0.95, 0.999\}$ for all methods and we display the number of environments used for selecting the parameter used at test time.  
}
\vspace{-2mm}
\label{tab:grid_search_heuristics_multiple_cl_environments}
\resizebox{\textwidth}{!}{
\begin{tabular}{l c c c c c c c c c c}
\toprule %\toprule
 & \multicolumn{6}{c}{ {\bf New Task Order}} & \multicolumn{4}{c}{ {\bf New Dataset}}\\
 \cmidrule(lr){2-7}  \cmidrule(lr){8-11}
 %{\bf Method} & S-MNIST & S-FashionMNIST & S-CIFAR-10 & S-notMNIST & S-FashionMNIST  \\
  & \multicolumn{2}{c}{ {\bf S-MNIST}} & \multicolumn{2}{c}{ {\bf S-FashionMNIST}} & \multicolumn{2}{c}{ {\bf S-CIFAR-10}} & \multicolumn{2}{c}{ {\bf S-notMNIST}} & \multicolumn{2}{c}{ {\bf S-FashionMNIST}}   \\
 \cmidrule(lr){2-3} \cmidrule(lr){4-5} \cmidrule(lr){6-7} \cmidrule(lr){8-9} \cmidrule(lr){10-11} 
 {\bf Method} & $\tau$ & \#Envs & $\tau$ & \#Envs & $\tau$ & \#Envs & $\tau$ & \#Envs & $\tau$ &  \#Envs \\
 \midrule
    Heur-GD & 0.9 & 10 & 0.95  & 20 & 0.9   & 10 & 0.9  & 10 & 0.9   & 10 \\ 
    Heur-LD & 0.9 & 10 & 0.999 & 20 & 0.999 & 10 & 0.95 & 10 & 0.999 & 10 \\ 
    Heur-AT & 0.9 & 10 & 0.999 & 20 & 0.9   & 10 & 0.9 & 10 & 0.95  & 10 \\
    %Heur-GD        & \{0.90, 0.95, 0.999\} & \{0.90, 0.95, 0.999\} & \{0.90, 0.95, 0.999\} & \{0.90, 0.95, 0.999\} & \{0.90, 0.95, 0.999\} \\
    %Heur-LD        & \{0.90, 0.95, 0.999\} & \{0.90, 0.95, 0.999\} & \{0.90, 0.95, 0.999\} & \{0.90, 0.95, 0.999\} & \{0.90, 0.95, 0.999\} \\
    %Heur-AT        & \{0.90, 0.95, 0.999\} & \{0.90, 0.95, 0.999\} & \{0.90, 0.95, 0.999\} & \{0.90, 0.95, 0.999\} & \{0.90, 0.95, 0.999\} \\
\bottomrule %\bottomrule
\end{tabular}
} 
%\vspace{-3mm}
\end{table}





\subsection*{Assessing Generalization with Ranking Method}\label{paperD:app:ranking_method}

We use a ranking method based on the CL performance in every test environment for performance comparison between the methods in Section \ref{paperD:sec:results_on_policy_generalization}. We use rankings because the performances can vary greatly between environments with different task orders and datasets. To measure the CL performance in the environments, we use the average test accuracy over all tasks after learning the final task, i.e.,
\begin{align*}
    \ACC = \frac{1}{T} \sum_{i=1}^{T} A_{T, i}^{(test)},
\end{align*}
where $A_{t, i}^{(test)}$ is the test accuracy of task $i$ after learning task $t$. Each method are ranked in descending order based on the ACC achieved in an environment. For example, assume that we want to compare the CL performance from using learned replay scheduling policies with DQN and A2C against a Random scheduling policy in one environment. The CL performances achieved for each method are given by
\vspace{2mm}
\begin{align*}
    [\ACC_{\Random}, \ACC_{\DQN}, \ACC_{\text{A2C}}] = [90\%, 99\%, 95\%].
\end{align*}

We get the following ranking order between the methods based on their corresponding ACC:
\vspace{2mm}
\begin{align*}
    \texttt{ranking}([\ACC_{\Random}, \ACC_{\DQN}, \ACC_{\text{A2C}}]) = [3, 1, 2],
\end{align*} 

where DQN is ranked in 1st place, A2C in 2nd, and Random in 3rd. When there are multiple environments for evaluation, we compute the average ranking across the ranking positions in every environment for each method to compare.

The average ranking for DQN and A2C are computed over the seed for initializing the network parameters as well as the seed of the environment. Similarly, the Random baseline is affected by the seed setting the random selection of actions and the environment seed. However, the performance of the ETS and Heuristic baselines are affected by the seed of the environment as these policies are fixed. We use copied values of the performance in environments for the ETS and Heuristic baselines when we need to compare across different random seeds for Random, DQN, and A2C. We show an example of such ranking calculation for ETS, a Heuristic baseline, DQN, and A2C. Consider the following performances for one environment:
\vspace{2mm}
\begin{align*}
\begin{bmatrix}
\ACC_{\ETS}^{1} & \ACC_{\Heur}^{1} & \ACC_{\DQN}^{1} & \ACC_{\text{A2C}}^{1} \\[3pt] \ACC_{\ETS}^{2}  & \ACC_{\Heur}^{2} & \ACC_{\DQN}^{2} & \ACC_{\text{A2C}}^{2} 
\end{bmatrix}  
=
\begin{bmatrix}
90\% & 95\% & 95\% & 99\% \\[3pt]
 * & * & 97\% & 98\%  
\end{bmatrix}  ,
\end{align*}

where $*$ denotes a copy of the ACC  value in the first row. The subscript on ACC denotes the method and the superscript the seed used for initializing the policy network $\vtheta$. Therefore, we copy the values for ETS and Heur such that the $\ACC_{\DQN}^2$ for seed 2 can be compared against ETS and Heur. Note that there is a tie between $\ACC_{\Heur}^{1}$ and $\ACC_{\DQN}^{1}$ as they have ACC $95\%$. We handle ties by assigning tied methods the average of their ranks, such that the ranks for both seeds will be
\vspace{2mm}
\begin{align*}
&
\texttt{ranking}
\left(
\begin{bmatrix}
\ACC_{\ETS}^{1} & \ACC_{\Heur}^{1} & \ACC_{\DQN}^{1} & \ACC_{\text{A2C}}^{1} \\[3pt] \ACC_{\ETS}^{2}  & \ACC_{\Heur}^{2} & \ACC_{\DQN}^{2} & \ACC_{\text{A2C}}^{2} 
\end{bmatrix}, \texttt{axis=-1, keepdim=True}  \right) \\[3pt]
= & 
\texttt{ranking}
\left(
\begin{bmatrix}
90\% & 95\% & 95\% & 99\% \\[3pt]
90\% & 95\% & 97\% & 98\%  
\end{bmatrix}, \texttt{axis=-1, keepdim=True}  \right) \\[3pt]
= & 
\begin{bmatrix}
4 & 2.5 & 2.5 & 1 \\[3pt] 
4 & 3 & 2 & 1 
\end{bmatrix},
\end{align*}

where we inserted the copied values, such that $\ACC_{\ETS}^{1} = \ACC_{\ETS}^{2} = 90\%$ and $\ACC_{\Heur}^{1} = \ACC_{\Heur}^{2} = 95\%$. The mean ranking across the seeds thus becomes
\vspace{2mm}
\begin{align*}
\texttt{mean}
\left(
\begin{bmatrix}
4 & 2.5 & 2.5 & 1 \\[3pt] 
4 & 3 & 2 & 1 
\end{bmatrix}, \texttt{axis=0} \right)
= 
\begin{bmatrix} 4 & 2.75 & 2.25 & 1  \end{bmatrix} 
\end{align*}

where A2C comes in 1st place, DQN in 2nd, Heur. in 3rd, and ETS on 4th place. We average across seeds and environments to obtain the final ranking score for each method for comparison. 




\section{Construction of the Action Space}\label{app:construction_of_the_action_space}

In this section, we describe the action space construction used in~\cite{klasson2021learn} that we use in our experiments. Each action $a_t = (p_1, \dots, p_{T-1}) \in \gA_t$ consists of a sequence of task proportions $p$ where the valid proportions sum to one, i.e., $\sum_{i=1}^{t} p_i = 1$, and invalid proportions are set to zero as $p_i = 0$ for $i>t$. The simplify the selection of task proportions, we create a discrete number of choices for the proportions of valid tasks. We use a binning method to construct the discrete action space. At task $t$, there are $t-1$ historical tasks that can be replayed. Thus, we create $t-1$ bins $\vb_t = [b_1, \dots, b_{t-1}]$ and choose a task index to sample for each bin $b_i \in \{1, \dots, t-1 \}$. The bins are treated as interchangeable and only the unique choices are kept. As an example, the unique choices of vectors at task 3 are $[1,1], [1,2], [2,2]$, where $[1,1]$ indicates that all memory samples are from task 1, $[1,2]$ indicates that half memory is from task 1 and the other half are from task etc. The task proportions for each choice are then computed by counting the number of occurrences of each task index in $\vb_t$ and dividing by $t-1$, such that $a_t = \texttt{bincount}(\vb_t) / (t-1)$. The task proportions are used to determine the number of samples to place in the replay memory from each task, such that $\texttt{round}(p_t \cdot M)$ are the number of samples from task $t$ where we round up or down accordingly while keeping the memory size $M$ fixed. From this specification, the whole action space over the task-horizon becomes a tree of different replay schedules that can be evaluated by the network.


\newpage

\section{Task Split Tables}\label{paperD:app:task_split_tables}

Here, we provide tables of the task splits for Split FashionMNIST and split CIFAR-10. 
We provide the random seeds that we used for shuffling the label splits as well as the class names for the datasets concerned. 


\begin{table}[h]
    \centering
    \caption{Split FashionMNIST task splits with their corresponding seed. }
    \resizebox{1.0\textwidth}{!}{
    \begin{tabular}{l c c c c c}
        \toprule \toprule
         {\bf Seed} & {\bf Task 1} & {\bf Task 2} & {\bf Task 3} & {\bf Task 4} & {\bf Task 5} \\
        \midrule
        0 & T-shirt/top, Trouser & Pullover, Dress & Coat, Sandal & Shirt, Sneaker & Bag, Ankle boot \\ \midrule 
        1 & Pullover, Ankle boot & Shirt, Coat & T-shirt/top, Dress & Trouser, Sneaker & Bag, Sandal \\ \midrule 
        2 & Coat, Trouser & Sandal, T-shirt/top & Sneaker, Pullover & Dress, Shirt & Ankle boot, Bag \\ \midrule 
        3 & Sandal, Coat & Trouser, Pullover & Ankle boot, Shirt & Sneaker, T-shirt/top & Dress, Bag \\ \midrule
        4 & Dress, Bag & Coat, Ankle boot & Pullover, Shirt & T-shirt/top, Trouser & Sandal, Sneaker\\ \midrule 
        5 & Ankle boot, Sandal & Pullover, Coat & Sneaker, Trouser & T-shirt/top, Bag & Shirt, Dress \\ \midrule
        6 & Bag, Trouser & Sneaker, T-shirt/top & Shirt, Sandal & Pullover, Coat & Dress, Ankle boot \\ \midrule
        7 & Bag, Sandal & T-shirt/top, Pullover & Trouser, Ankle boot & Sneaker, Dress & Shirt, Coat \\ \midrule 
        8 & Bag, Shirt & Ankle boot, T-shirt/top & Pullover, Sandal & Sneaker, Trouser & Coat, Dress \\ \midrule
        9 & Bag, Coat & Sneaker, Pullover & Trouser, Ankle boot & Dress, T-shirt/top & Shirt, Sandal \\
        \bottomrule \bottomrule
    \end{tabular}
    }
    \label{tab:label_splits_fashionmnist}
\end{table}



\begin{table}[h]
    \centering
    \caption{Split CIFAR-10 task splits with their corresponding seed. }
    \resizebox{1.0\textwidth}{!}{
    \begin{tabular}{l c c c c c}
        \toprule \toprule
         {\bf Seed} & {\bf Task 1} & {\bf Task 2} & {\bf Task 3} & {\bf Task 4} & {\bf Task 5} \\
        \midrule
        0 & Airplane, Automobile & Bird, Cat & Deer, Dog & Frog, Horse & Ship, Truck \\
        \midrule
        1 & Bird, Truck & Frog, Deer & Airplane, Cat & Automobile, Horse & Ship, Dog \\
        \midrule
        2 & Deer, Automobile & Dog, Airplane & Horse, Bird & Cat, Frog & Truck, Ship \\
        \midrule
        3 & Dog, Deer & Automobile, Bird & Truck, Frog & Horse, Airplane & Cat, Ship \\ 
        \midrule
        4 & Cat, Ship & Deer, Truck & Bird, Frog & Airplane, Automobile & Dog, Horse \\ 
        \midrule
        5 & Truck, Dog & Bird, Deer & Horse, Automobile & Airplane, Ship & Frog, Cat \\ 
        \midrule
        6 & Ship, Automobile & Horse, Airplane & Frog, Dog & Bird, Deer & Cat, Truck \\ 
        \midrule
        7 & Ship, Dog & Airplane, Bird & Automobile, Truck & Horse, Cat & Frog, Deer \\ 
        \midrule
        8 & Ship, Frog & Truck, Airplane & Bird, Dog & Horse, Automobile & Deer, Cat \\ 
        \midrule
        9 & Ship, Deer & Horse, Bird & Automobile, Truck & Cat, Airplane & Frog, Dog \\ 
        \midrule
        10 & Ship, Bird & Dog, Frog & Cat, Automobile & Airplane, Horse & Deer, Truck \\ 
        \midrule
        11 & Horse, Ship & Bird, Frog & Deer, Dog & Automobile, Cat & Airplane, Truck \\ 
        \midrule
        12 & Dog, Ship & Horse, Airplane & Deer, Truck & Cat, Bird & Automobile, Frog \\ 
        \midrule
        13 & Cat, Dog & Frog, Automobile & Deer, Horse & Ship, Truck & Airplane, Bird \\ 
        \midrule
        14 & Cat, Truck & Airplane,  Dog & Deer, Bird & Automobile, Horse & Frog, Ship \\ 
        \midrule
        15 & Bird, Frog & Automobile, Cat & Horse, Airplane & Truck, Deer & Dog, Ship \\ 
        \midrule
        16 & Frog, Bird & Airplane, Horse & Ship, Deer & Cat, Automobile & Dog, Truck \\ 
        \midrule
        17 & Horse, Bird & Dog, Cat & Deer, Airplane & Truck, Ship & Frog, Automobile \\ 
        \midrule
        18 & Horse, Truck & Airplane, Deer & Bird, Automobile & Frog, Dog & Ship, Cat  \\ 
        \midrule
        19 & Automobile, Horse & Truck, Frog & Ship, Deer & Cat, Airplane & Bird, Dog  \\
        \bottomrule \bottomrule
    \end{tabular}
    }
    \label{tab:label_splits_cifar10}
\end{table}



%\newpage
%
\section{Additional Experimental Results}
\label{app:additional_experimental_results}

In this section, we bring more insights to the benefits of replay scheduling in Section \ref{app:replay_schedule_visualization_for_split_mnist} as well as provide metrics for catastrophic forgetting in Section \ref{app:analysis_of_catastrophic_forgetting}.

\subsection{Replay Schedule Visualization for Split MNIST}
\label{app:replay_schedule_visualization_for_split_mnist}

In Figure \ref{fig:split_mnist_task_accuracies_and_bubble_plot}, we show the progress in test classification performance for each task when using ETS and RS-MCTS with memory size $M=10$ on Split MNIST. For comparison, we also show the performance from a network that is fine-tuning on the current task without using replay. Both ETS and RS-MCTS overcome catastrophic forgetting to a large degree compared to the fine-tuning network. Our method RS-MCTS further improves the performance compared to ETS with the same memory, which indicates that learning the time to learn can be more efficient against catastrophic forgetting. Especially, Task 1 and 2 seems to be the most difficult task to remember since it has the lowest final performance using the fine-tuning network. Both ETS and RS-MCTS manage to retain their performance on Task 1 using replay, however, RS-MCTS remembers Task 2 better than ETS by around 5\%. 

To bring more insights to this behavior, we have visualized the task proportions of the replay examples using a bubble plot showing the corresponding replay schedule from RS-MCTS in Figure \ref{fig:split_mnist_task_accuracies_and_bubble_plot}(right). At Task 3 and 4, we see that the schedule fills the memory with data from Task 2 and discards replaying Task 1. This helps the network to retain knowledge about Task 2 better than ETS at the cost of forgetting Task 3 slightly when learning Task 4. This shows that the learned policy has considered the difficulty level of different tasks. At the next task, the RS-MCTS schedule has decided to rehearse Task 3 and reduces replaying Task 2 when learning Task 5. This behavior is similar to spaced repetition, where increasing the time interval between rehearsals helps memory retention.
We emphasize that even on datasets with few tasks, using learned replay schedules can overcome catastrophic forgetting better than standard ETS approaches.



\begin{figure}[t]
  \centering
  \setlength{\figwidth}{0.25\textwidth}
  \setlength{\figheight}{.14\textheight}
  \input{PaperC/appendix/figures/bubble_plots/mnist/seed3/mnist_accuracy_and_bubble_plot}
  \vspace{-3mm}
  \caption{ Comparison of test classification accuracies for Task 1-5 on Split MNIST from a network trained without replay (Fine-tuning), ETS, and RS-MCTS. The ACC metric for each method is shown on top of each figure. We also visualize the replay schedule found by RS-MCTS as a bubble plot to the right. The memory size is set to $M=10$ with uniform memory selection for ETS and RS-MCTS. Results are shown for 1 seed. 
  }
  \vspace{-3mm}
  \label{fig:split_mnist_task_accuracies_and_bubble_plot}
\end{figure}

\subsection{Analysis of Catastrophic Forgetting}\label{app:analysis_of_catastrophic_forgetting}

We have compared the degree of catastrophic forgetting for our method against the baselines by measuring the backward transfer (BWT) metric from \citet{lopez2017gradient}, which is given by
\begin{align}
    \textnormal{BWT} = \frac{1}{T-1} \sum_{i=1}^{T-1} A_{T, i} - A_{i, i},
\end{align}
where $A_{t, i}$ is the test accuracy for task $t$ after learning task $i$. Table \ref{tab:bwt_alternative_memory_selection} shows the ACC and BWT metrics for the experiments in Section \ref{sec:alternative_memory_selection_methods}. In general, the BWT metric is consistently better when the corresponding ACC is better. We find an exception in Table \ref{tab:bwt_alternative_memory_selection} on Split CIFAR-100 and Split miniImagenet between Ours and Heuristic with uniform selection method, where Heuristic has better BWT while its mean of ACC is slightly lower than ACC for Ours. Table \ref{tab:bwt_efficiency_of_replay_scheduling} shows the ACC and BWT metrics for the experiments in Section \ref{sec:efficiency_of_replay_scheduling}, where we see a similar pattern that better ACC yields better BWT. The BWT of RS-MCTS is on par with the other baselines except on Split CIFAR-100 where the ACC on our method was a bit lower than the best baselines.



\subsection{Applying Scheduling to Recent Replay Methods}
\label{app:apply_scheduling_to_recent_replay_methods}

In Section \ref{sec:applying_scheduling_to_recent_replay_methods}, we showed that RS-MCTS can be applied to any replay method. We combined RS-MCTS together with four recent replay methods, namely Hindsight Anchor Learning (HAL)~\citeC{C:chaudhry2021using}, Meta Experience Replay (MER)~\citeC{C:riemer2018learning}, and Dark Experience Replay (DER)~\citeC{C:buzzega2020dark}. Table \ref{tab:bwt_sota_models_applied_to_rsmcts} shows the ACC and BWT for all methods combined with the scheduling from Random, ETS, Heuristic, and RS-MCTS. We observe that RS-MCTS can further improve the performance for each of the replay methods across the different datasets.  

\paragraph{Hyperparameters.} We present the hyperparameters used for each method below. We used the same architectures and hyperparameters as described in Appendix \ref{app:experimental_settings} for all datasets if not mentioned otherwise. We used the Adam optimizer with learning rate $\eta=0.001$ for MER, DER, and DER++. For HAL, we used the SGD optimizer since using Adam made the model diverge in our experiments.  

\begin{itemize}
    \item Hindsight Anchor Learning (HAL):
    \begin{itemize}
        \item Split MNIST: learning rate $\eta = 0.1$, regularization $\lambda=0.1$, mean embedding strength $\gamma = 0.5$, decay rate $\beta=0.7$, gradient steps on anchors $k=100$
        \item Split FashionMNIST: learning rate $\eta = 0.1$, regularization $\lambda=0.1$, mean embedding strength $\gamma = 0.1$, decay rate $\beta=0.5$, gradient steps on anchors $k=100$
        \item Split notMNIST: learning rate $\eta = 0.1$, regularization $\lambda=0.1$, mean embedding strength $\gamma = 0.1$, decay rate $\beta=0.5$, gradient steps on anchors $k=100$
        \item Permuted MNIST: learning rate $\eta = 0.1$, regularization $\lambda=0.1$, mean embedding strength $\gamma = 0.1$, decay rate $\beta=0.5$, gradient steps on anchors $k=100$
        \item Split CIFAR-100: learning rate $\eta = 0.03$, regularization $\lambda=1.0$, mean embedding strength $\gamma = 0.1$, decay rate $\beta=0.5$, gradient steps on anchors $k=100$
        \item Split miniImagenet: learning rate $\eta = 0.03$, regularization $\lambda=0.3$, mean embedding strength $\gamma = 0.1$, decay rate $\beta=0.5$, gradient steps on anchors $k=100$
    \end{itemize}
    \item Meta Experience Replay (MER):
    \begin{itemize}
        \item Split MNIST: across batch meta-learning rate $\gamma = 1.0$, within batch meta-learning rate $\beta=1.0$ 
        \item Split FashionMNIST: across batch meta-learning rate $\gamma = 1.0$, within batch meta-learning rate $\beta=0.01$ 
        \item Split notMNIST: across batch meta-learning rate $\gamma = 1.0$, within batch meta-learning rate $\beta=1.0$ 
        \item Permuted MNIST: across batch meta-learning rate $\gamma = 1.0$, within batch meta-learning rate $\beta=1.0$ 
        \item Split CIFAR-100: across batch meta-learning rate $\gamma = 1.0$, within batch meta-learning rate $\beta=0.1$ 
        \item Split miniImagenet: across batch meta-learning rate $\gamma = 1.0$, within batch meta-learning rate $\beta=0.1$ 
    \end{itemize}
    \item Dark Experience Replay (DER):
    \begin{itemize}
        \item Split MNIST: loss coefficient memory logits $\alpha=0.2$
        \item Split FashionMNIST: loss coefficient memory logits $\alpha=0.2$
        \item Split notMNIST: loss coefficient memory logits $\alpha=0.1$
        \item Permuted MNIST: loss coefficient memory logits $\alpha=1.0$
        \item Split CIFAR-100: loss coefficient memory logits $\alpha=1.0$
        \item Split miniImagenet: loss coefficient memory logits $\alpha=0.1$
    \end{itemize}
    \item Dark Experience Replay++ (DER++):
    \begin{itemize}
        \item Split MNIST: loss coefficient memory logits $\alpha=0.2$, loss coefficient memory labels $\beta=1.0$
        \item Split FashionMNIST: loss coefficient memory logits $\alpha=0.2$, loss coefficient memory labels $\beta=1.0$
        \item Split notMNIST: loss coefficient memory logits $\alpha=0.1$, loss coefficient memory labels $\beta=1.0$
        \item Permuted MNIST: loss coefficient memory logits $\alpha=1.0$, loss coefficient memory labels $\beta=1.0$
        \item Split CIFAR-100: loss coefficient memory logits $\alpha=1.0$, loss coefficient memory labels $\beta=1.0$
        \item Split miniImagenet: loss coefficient memory logits $\alpha=0.1$, loss coefficient memory labels $\beta=1.0$
    \end{itemize}
\end{itemize}



%%% TABLE


\begin{table}[t]
    %\footnotesize
    \centering
    \caption{
    Performance comparison with ACC and BWT metrics for all datasets between MCTS (Ours) and the baselines with various memory selection methods. We provide the metrics for training on all seen task datasets jointly (Joint) as an upper bound. Furthermore, we include the results from a breadth-first search (BFS) with Uniform memory selection for the 5-task datasets. 
	The memory size is set to $M=10$ and $M=100$ for the 5-task and 10/20-task datasets respectively. We report the mean and standard deviation of ACC and BWT, where all results have been averaged over 5 seeds. MCTS performs better or on par than the baselines on most datasets and selection methods, where MoF yields the best performance in general.
    }
    \vspace{-3mm}
    \resizebox{0.95\textwidth}{!}{
    \begin{tabular}{l l c c c c c c}
        \toprule
         & & \multicolumn{2}{c}{{\bf Split MNIST} } & \multicolumn{2}{c}{{\bf Split FashionMNIST} } & \multicolumn{2}{c}{{\bf Split notMNIST} } \\
         \cmidrule(lr){3-4} \cmidrule(lr){5-6} \cmidrule(lr){7-8} %\cline{3-8}
         {\bf Selection} & {\bf Method} & ACC(\%)$\uparrow$ & BWT(\%)$\uparrow$ & ACC(\%)$\uparrow$ & BWT(\%)$\uparrow$ & ACC(\%)$\uparrow$ & BWT(\%)$\uparrow$ \\
        \midrule 
        \multicolumn{1}{c}{--} & Joint & 99.75 ($\pm$ 0.06) & 0.01 ($\pm$ 0.06) & 99.34 ($\pm$ 0.08) & -0.01 ($\pm$ 0.14) & 96.12 ($\pm$ 0.57) & -0.21 ($\pm$ 0.71) \\
		Uniform & BFS & 98.28 ($\pm$ 0.49) & -1.84 ($\pm$ 0.63) & 98.51 ($\pm$ 0.23) & -1.03 ($\pm$ 0.28) & 95.54 ($\pm$ 0.67) & -1.04 ($\pm$ 0.87) \\
		\midrule 
        \multirow{4}{*}{Uniform} & Random & 95.91 ($\pm$ 1.56) & -4.79 ($\pm$ 1.95) & 95.82 ($\pm$ 1.45) & -4.35 ($\pm$ 1.79 ) & 92.39 ($\pm$ 1.29) & -4.56 ($\pm$ 1.29) \\
         & ETS & 94.02 ($\pm$ 4.25) & -7.22 ($\pm$ 5.33) & 95.81 ($\pm$ 3.53) & -4.45 ($\pm$ 4.34) & 91.01 ($\pm$ 1.39) & -6.16 ($\pm$ 1.82) \\
         & Heuristic & 96.02 ($\pm$ 2.32) & -4.64 ($\pm$ 2.90) & 97.09 ($\pm$ 0.62) & -2.82 ($\pm$ 0.84) & 91.26 ($\pm$ 3.99) & -6.06 ($\pm$ 4.70) \\
         & Ours & 97.93 ($\pm$ 0.56) & -2.27 ($\pm$ 0.71) & 98.27 ($\pm$ 0.17) & -1.29 ($\pm$ 0.20) & 94.64 ($\pm$ 0.39) & -1.47 ($\pm$ 0.79) \\
        \midrule 
        \multirow{4}{*}{$k$-means} & Random & 94.24 ($\pm$ 3.20) & -6.88 ($\pm$ 4.00) & 96.30 ($\pm$ 1.62) & -3.77 ($\pm$ 2.05) & 91.64 ($\pm$ 1.39) & -5.64 ($\pm$ 1.77) \\
         & ETS & 92.89 ($\pm$ 3.53) & -8.66 ($\pm$ 4.42) & 96.47 ($\pm$ 0.85) & -3.55 ($\pm$ 1.07) & 93.80 ($\pm$ 0.82) & -2.84 ($\pm$ 0.81) \\
         & Heuristic & 96.28 ($\pm$ 1.68) & -4.32 ($\pm$ 2.11) & 95.78 ($\pm$ 1.50) & -4.46 ($\pm$ 1.87) & 91.75 ($\pm$ 0.94) & -5.60 ($\pm$ 2.07) \\
         & Ours & 98.20 ($\pm$ 0.16) & -1.94 ($\pm$ 0.22) & 98.48 ($\pm$ 0.26) & -1.04 ($\pm$ 0.31) & 93.61 ($\pm$ 0.71) & -3.11 ($\pm$ 0.55) \\
        \midrule 
        \multirow{4}{*}{$k$-center} & Random & 96.40 ($\pm$ 0.68) & -4.21 ($\pm$ 0.84) & 95.57 ($\pm$ 3.16) & -7.20 ($\pm$ 3.93) & 92.61 ($\pm$ 1.70) & -4.14 ($\pm$ 2.37) \\
         & ETS & 94.84 ($\pm$ 1.40) & -6.20 ($\pm$ 1.77) & 97.28 ($\pm$ 0.50) & -2.58 ($\pm$ 0.66) & 91.08 ($\pm$ 2.48) & -6.39 ($\pm$ 3.46) \\
         & Heuristic & 94.55 ($\pm$ 2.79) & -6.47 ($\pm$ 3.50) & 94.08 ($\pm$ 3.72) & -6.59 ($\pm$ 4.57) & 92.06 ($\pm$ 1.20) & -4.70 ($\pm$ 2.09) \\
         & Ours & 98.24 ($\pm$ 0.36) & -1.93 ($\pm$ 0.44) & 98.06 ($\pm$ 0.35) & -1.59 ($\pm$ 0.45) & 94.26 ($\pm$ 0.37) & -1.97 ($\pm$ 1.02) \\
        \midrule
        \multirow{4}{*}{MoF} & Random & 95.18 ($\pm$ 3.18) & -5.73 ($\pm$ 3.95) & 95.76 ($\pm$ 1.41) & -4.41 ($\pm$ 1.75) & 91.33 ($\pm$ 1.75) & -5.94 ($\pm$ 1.92) \\
         & ETS & 97.04 ($\pm$ 1.23) & -3.46 ($\pm$ 1.50) & 96.48 ($\pm$ 1.33) & -3.55 ($\pm$ 1.73) & 92.64 ($\pm$ 0.87) & -4.57 ($\pm$ 1.59) \\
         & Heuristic & 96.46 ($\pm$ 2.41) & -4.09 ($\pm$ 3.01) & 95.84 ($\pm$ 0.89) & -4.39 ($\pm$ 1.15) & 93.24 ($\pm$ 0.77) & -3.48 ($\pm$ 1.37) \\
         & Ours & 98.37 ($\pm$ 0.24) & -1.70 ($\pm$ 0.28) & 97.84 ($\pm$ 0.32) & -1.81 ($\pm$ 0.39) & 94.62 ($\pm$ 0.42) & -1.80 ($\pm$ 0.56) \\
        \bottomrule
        \toprule
         & & \multicolumn{2}{c}{{\bf Permuted MNIST} } & \multicolumn{2}{c}{{\bf Split CIFAR-100} } & \multicolumn{2}{c}{{\bf Split miniImagenet} } \\
         \cmidrule(lr){3-4} \cmidrule(lr){5-6} \cmidrule(lr){7-8} %\cline{3-8}
         {\bf Selection} & {\bf Method} & ACC(\%)$\uparrow$ & BWT(\%)$\uparrow$ & ACC(\%)$\uparrow$ & BWT(\%)$\uparrow$ & ACC(\%)$\uparrow$ & BWT(\%)$\uparrow$ \\
        \midrule 
        \multicolumn{1}{c}{--} & Joint & 95.34 ($\pm$ 0.13) & 0.17 ($\pm$ 0.18) & 84.73 ($\pm$ 0.81) & -1.06 ($\pm$ 0.81) & 74.03 ($\pm$ 0.83) & 9.70 ($\pm$ 0.68) \\
		\midrule
        \multirow{4}{*}{Uniform} & Random & 72.44 ($\pm$ 1.15) & -25.56 ($\pm$ 1.39) & 53.99 ($\pm$ 0.51) & -34.19 ($\pm$ 0.66) & 48.08 ($\pm$ 1.36) & -15.98 ($\pm$ 1.08) \\
         & ETS & 71.09 ($\pm$ 2.31) & -27.39 ($\pm$ 2.59) & 47.70 ($\pm$ 2.16) & -41.68 ($\pm$ 2.37) & 46.97 ($\pm$ 1.24) & -18.32 ($\pm$ 1.34) \\
         & Heuristic & 76.68 ($\pm$ 2.13) & -20.82 ($\pm$ 2.41) & 57.31 ($\pm$ 1.21) & -30.76 ($\pm$ 1.45) & 49.66 ($\pm$ 1.10) & -12.04 ($\pm$ 0.59) \\
         & Ours & 76.34 ($\pm$ 0.98) & -21.21 ($\pm$ 1.16) & 56.60 ($\pm$ 1.13) & -31.39 ($\pm$ 1.11) & 50.20 ($\pm$ 0.72) & -13.46 ($\pm$ 1.22) \\
        \midrule 
        \multirow{4}{*}{$k$-means} & Random & 74.30 ($\pm$ 1.43) & -23.50 ($\pm$ 1.64) & 53.18 ($\pm$ 1.66) & -35.15 ($\pm$ 1.61) & 49.47 ($\pm$ 2.70) & -14.10 ($\pm$ 2.71) \\
         & ETS & 69.40 ($\pm$ 1.32) & -29.23 ($\pm$ 1.47) & 47.51 ($\pm$ 1.14) & -41.77 ($\pm$ 1.30) & 45.82 ($\pm$ 0.92) & -19.53 ($\pm$ 1.10) \\
         & Heuristic & 75.57 ($\pm$ 1.18) & -22.11 ($\pm$ 1.22) & 54.31 ($\pm$ 3.94) & -33.80 ($\pm$ 4.24) & 49.25 ($\pm$ 1.00) & -12.92 ($\pm$ 1.22) \\
         & Ours & 77.74 ($\pm$ 0.80) & -19.66 ($\pm$ 0.95) & 56.95 ($\pm$ 0.92) & -30.92 ($\pm$ 0.83) & 50.47 ($\pm$ 0.85) & -13.31 ($\pm$ 1.24) \\
        \midrule 
        \multirow{4}{*}{$k$-center} & Random & 71.41 ($\pm$ 2.75) & -26.73 ($\pm$ 3.11) & 48.46 ($\pm$ 0.31) & -39.89 ($\pm$ 0.27) & 44.76 ($\pm$ 0.96) & -18.72 ($\pm$ 1.17) \\
         & ETS & 69.11 ($\pm$ 1.69) & -29.58 ($\pm$ 1.81) & 44.13 ($\pm$ 1.06) & -45.28 ($\pm$ 1.04) & 41.35 ($\pm$ 0.96) & -23.71 ($\pm$ 1.45) \\
         & Heuristic & 74.33 ($\pm$ 2.00) & -23.45 ($\pm$ 2.27) & 50.32 ($\pm$ 1.97) & -37.99 ($\pm$ 2.14) & 44.13 ($\pm$ 0.95) & -18.26 ($\pm$ 1.05) \\
         & Ours & 76.55 ($\pm$ 1.16) & -21.06 ($\pm$ 1.32) & 51.37 ($\pm$ 1.63) & -37.01 ($\pm$ 1.62) & 46.76 ($\pm$ 0.96) & -16.56 ($\pm$ 0.90) \\
        \midrule
        \multirow{4}{*}{MoF} & Random & 77.96 ($\pm$ 1.84) & -19.44 ($\pm$ 2.13) & 61.93 ($\pm$ 1.05) & -25.89 ($\pm$ 1.07) & 54.50 ($\pm$ 1.33) & -8.64 ($\pm$ 1.26) \\
         & ETS & 77.62 ($\pm$ 1.12) & -20.10 ($\pm$ 1.26) & 60.43 ($\pm$ 1.17) & -28.22 ($\pm$ 1.26) & 56.12 ($\pm$ 1.12) & -8.93 ($\pm$ 0.83) \\
         & Heuristic & 77.27 ($\pm$ 1.45) & -20.15 ($\pm$ 1.63) & 55.60 ($\pm$ 2.70) & -32.57 ($\pm$ 2.77) & 52.30 ($\pm$ 0.59) & -9.61 ($\pm$ 0.67) \\
         & Ours & 81.58 ($\pm$ 0.75) & -15.41 ($\pm$ 0.86) & 64.22 ($\pm$ 0.65) & -23.48 ($\pm$ 1.02) & 57.70 ($\pm$ 0.51) & -5.31 ($\pm$ 0.55) \\
        \bottomrule
    \end{tabular}
    }
    \vspace{-2mm}
    \label{tab:bwt_alternative_memory_selection}
\end{table}


\begin{comment}
\begin{table}[h]
    \footnotesize
    \centering
    \caption{
    Performance comparison with ACC and BWT metrics for all datasets between RS-MCTS and the baselines with various memory selection methods. The memory size is set to $M=10$ and $M=100$ for the 5-task and 10/20-task datasets respectively. We report the mean and standard deviation of ACC and BWT, where all results have been averaged over 5 seeds. RS-MCTS performs better or on par than the baselines on most datasets and selection methods, where MoF yields the best performance in general.
    }
    %\vspace{-3mm}
    \begin{tabular}{l l c c c c c c}
        \toprule
         & & \multicolumn{2}{c}{{\bf Split MNIST} } & \multicolumn{2}{c}{{\bf Split FashionMNIST} } & \multicolumn{2}{c}{{\bf Split notMNIST} } \\
         \cmidrule(lr){3-4} \cmidrule(lr){5-6} \cmidrule(lr){7-8} %\cline{3-8}
         {\bf Selection} & {\bf Method} & ACC(\%)$\uparrow$ & BWT(\%)$\uparrow$ & ACC(\%)$\uparrow$ & BWT(\%)$\uparrow$ & ACC(\%)$\uparrow$ & BWT(\%)$\uparrow$ \\
        \midrule 
        \multirow{4}{*}{Uniform} & Random & 95.91 $\pm$ 1.56 & -4.79 $\pm$ 1.95 & 95.82 $\pm$ 1.45 & -4.35 $\pm$ 1.79  & 92.39 $\pm$ 1.29 & -4.56 $\pm$ 1.29 \\
         & ETS & 94.02 $\pm$ 4.25 & -7.22 $\pm$ 5.33 & 95.81 $\pm$ 3.53 & -4.45 $\pm$ 4.34  & 91.01 $\pm$ 1.39 & -6.16 $\pm$ 1.82 \\
         & Heuristic & 96.02 $\pm$ 2.32 & -4.64 $\pm$ 2.90 & 97.09 $\pm$ 0.62 & -2.82 $\pm$ 0.84  & 91.26 $\pm$ 3.99 & -6.06 $\pm$ 4.70 \\
         & Ours & 97.93 $\pm$ 0.56 & -2.27 $\pm$ 0.71 & 98.27 $\pm$ 0.17 & -1.29 $\pm$ 0.20  & 94.64 $\pm$ 0.39 & -1.47 $\pm$ 0.79 \\
        \midrule 
        \multirow{4}{*}{$k$-means} & Random & 94.24 $\pm$ 3.20 & -6.88 $\pm$ 4.00 & 96.30 $\pm$ 1.62 & -3.77 $\pm$ 2.05  & 91.64 $\pm$ 1.39 & -5.64 $\pm$ 1.77 \\
         & ETS & 92.89 $\pm$ 3.53 & -8.66 $\pm$ 4.42 & 96.47 $\pm$ 0.85 & -3.55 $\pm$ 1.07  & 93.80 $\pm$ 0.82 & -2.84 $\pm$ 0.81 \\
         & Heuristic & 96.28 $\pm$ 1.68 & -4.32 $\pm$ 2.11 & 95.78 $\pm$ 1.50 & -4.46 $\pm$ 1.87  & 91.75 $\pm$ 0.94 & -5.60 $\pm$ 2.07 \\
         & Ours & 98.20 $\pm$ 0.16 & -1.94 $\pm$ 0.22 & 98.48 $\pm$ 0.26 & -1.04 $\pm$ 0.31  & 93.61 $\pm$ 0.71 & -3.11 $\pm$ 0.55 \\
        \midrule 
        \multirow{4}{*}{$k$-center} & Random & 96.40 $\pm$ 0.68 & -4.21 $\pm$ 0.84 & 95.57 $\pm$ 3.16 & -7.20 $\pm$ 3.93  & 92.61 $\pm$ 1.70 & -4.14 $\pm$ 2.37 \\
         & ETS & 94.84 $\pm$ 1.40 & -6.20 $\pm$ 1.77 & 97.28 $\pm$ 0.50 & -2.58 $\pm$ 0.66  & 91.08 $\pm$ 2.48 & -6.39 $\pm$ 3.46 \\
         & Heuristic & 94.55 $\pm$ 2.79 & -6.47 $\pm$ 3.50 & 94.08 $\pm$ 3.72 & -6.59 $\pm$ 4.57  & 92.06 $\pm$ 1.20 & -4.70 $\pm$ 2.09 \\
         & Ours & 98.24 $\pm$ 0.36 & -1.93 $\pm$ 0.44 & 98.06 $\pm$ 0.35 & -1.59 $\pm$ 0.45  & 94.26 $\pm$ 0.37 & -1.97 $\pm$ 1.02 \\
        \midrule
        \multirow{4}{*}{MoF} & Random & 95.18 $\pm$ 3.18 & -5.73 $\pm$ 3.95 & 95.76 $\pm$ 1.41 & -4.41 $\pm$ 1.75  & 91.33 $\pm$ 1.75 & -5.94 $\pm$ 1.92 \\
         & ETS & 97.04 $\pm$ 1.23 & -3.46 $\pm$ 1.50 & 96.48 $\pm$ 1.33 & -3.55 $\pm$ 1.73  & 92.64 $\pm$ 0.87 & -4.57 $\pm$ 1.59 \\
         & Heuristic & 96.46 $\pm$ 2.41 & -4.09 $\pm$ 3.01 & 95.84 $\pm$ 0.89 & -4.39 $\pm$ 1.15  & 93.24 $\pm$ 0.77 & -3.48 $\pm$ 1.37 \\
         & Ours & 98.37 $\pm$ 0.24 & -1.70 $\pm$ 0.28 & 97.84 $\pm$ 0.32 & -1.81 $\pm$ 0.39  & 94.62 $\pm$ 0.42 & -1.80 $\pm$ 0.56 \\
        \bottomrule
        \toprule
         & & \multicolumn{2}{c}{{\bf Permuted MNIST} } & \multicolumn{2}{c}{{\bf Split CIFAR-100} } & \multicolumn{2}{c}{{\bf Split miniImagenet} } \\
         \cmidrule(lr){3-4} \cmidrule(lr){5-6} \cmidrule(lr){7-8} %\cline{3-8}
         {\bf Selection} & {\bf Method} & ACC(\%)$\uparrow$ & BWT(\%)$\uparrow$ & ACC(\%)$\uparrow$ & BWT(\%)$\uparrow$ & ACC(\%)$\uparrow$ & BWT(\%)$\uparrow$ \\
        \midrule 
        \multirow{4}{*}{Uniform} & Random & 72.44 $\pm$ 1.15 & -25.56 $\pm$ 1.39 & 53.99 $\pm$ 0.51 & -34.19 $\pm$ 0.66  & 48.08 $\pm$ 1.36 & -15.98 $\pm$ 1.08 \\
         & ETS & 71.09 $\pm$ 2.31 & -27.39 $\pm$ 2.59 & 47.70 $\pm$ 2.16 & -41.68 $\pm$ 2.37  & 46.97 $\pm$ 1.24 & -18.32 $\pm$ 1.34 \\
         & Heuristic & 76.68 $\pm$ 2.13 & -20.82 $\pm$ 2.41 & 57.31 $\pm$ 1.21 & -30.76 $\pm$ 1.45  & 49.66 $\pm$ 1.10 & -12.04 $\pm$ 0.59 \\
         & Ours & 76.34 $\pm$ 0.98 & -21.21 $\pm$ 1.16 & 56.60 $\pm$ 1.13 & -31.39 $\pm$ 1.11  & 50.20 $\pm$ 0.72 & -13.46 $\pm$ 1.22 \\
        \midrule 
        \multirow{4}{*}{$k$-means} & Random & 74.30 $\pm$ 1.43 & -23.50 $\pm$ 1.64 & 53.18 $\pm$ 1.66 & -35.15 $\pm$ 1.61  & 49.47 $\pm$ 2.70 & -14.10 $\pm$ 2.71 \\
         & ETS & 69.40 $\pm$ 1.32 & -29.23 $\pm$ 1.47 & 47.51 $\pm$ 1.14 & -41.77 $\pm$ 1.30  & 45.82 $\pm$ 0.92 & -19.53 $\pm$ 1.10 \\
         & Heuristic & 75.57 $\pm$ 1.18 & -22.11 $\pm$ 1.22 & 54.31 $\pm$ 3.94 & -33.80 $\pm$ 4.24  & 49.25 $\pm$ 1.00 & -12.92 $\pm$ 1.22 \\
         & Ours & 77.74 $\pm$ 0.80 & -19.66 $\pm$ 0.95 & 56.95 $\pm$ 0.92 & -30.92 $\pm$ 0.83  & 50.47 $\pm$ 0.85 & -13.31 $\pm$ 1.24 \\
        \midrule 
        \multirow{4}{*}{$k$-center} & Random & 71.41 $\pm$ 2.75 & -26.73 $\pm$ 3.11 & 48.46 $\pm$ 0.31 & -39.89 $\pm$ 0.27  & 44.76 $\pm$ 0.96 & -18.72 $\pm$ 1.17  \\
         & ETS & 69.11 $\pm$ 1.69 & -29.58 $\pm$ 1.81 & 44.13 $\pm$ 1.06 & -45.28 $\pm$ 1.04  & 41.35 $\pm$ 0.96 & -23.71 $\pm$ 1.45 \\
         & Heuristic & 74.33 $\pm$ 2.00 & -23.45 $\pm$ 2.27 & 50.32 $\pm$ 1.97 & -37.99 $\pm$ 2.14  & 44.13 $\pm$ 0.95 & -18.26 $\pm$ 1.05 \\
         & Ours & 76.55 $\pm$ 1.16 & -21.06 $\pm$ 1.32 & 51.37 $\pm$ 1.63 & -37.01 $\pm$ 1.62  & 46.76 $\pm$ 0.96 & -16.56 $\pm$ 0.90 \\
        \midrule
        \multirow{4}{*}{MoF} & Random & 77.96 $\pm$ 1.84 & -19.44 $\pm$ 2.13 & 61.93 $\pm$ 1.05 & -25.89 $\pm$ 1.07  & 54.50 $\pm$ 1.33 & -8.64 $\pm$ 1.26 \\
         & ETS & 77.62 $\pm$ 1.12 & -20.10 $\pm$ 1.26 & 60.43 $\pm$ 1.17 & -28.22 $\pm$ 1.26  & 56.12 $\pm$ 1.12 & -8.93 $\pm$ 0.83 \\
         & Heuristic & 77.27 $\pm$ 1.45 & -20.15 $\pm$ 1.63 & 55.60 $\pm$ 2.70 & -32.57 $\pm$ 2.77  & 52.30 $\pm$ 0.59 & -9.61 $\pm$ 0.67 \\
         & Ours & 81.58 $\pm$ 0.75 & -15.41 $\pm$ 0.86 & 64.22 $\pm$ 0.65 & -23.48 $\pm$ 1.02  & 57.70 $\pm$ 0.51 & -5.31 $\pm$ 0.55 \\
        \bottomrule
    \end{tabular}
    \vspace{-3mm}
    \label{tab:bwt_alternative_memory_selection}
\end{table}
\end{comment}


\begin{table}[t]
    \footnotesize
    \centering
    \caption{
    Performance comparison with ACC and BWT metrics for all datasets between RS-MCTS and the baselines in the setting where only 1 sample per class can be replayed. The memory sizes are set to $M=10$ and $M=100$ for the 5-task and 10/20-task datasets respectively. We report the mean and standard deviation of ACC and BWT, where all results have been averaged over 5 seeds. RS-MCTS performs on par with the best baselines for both metrics on all datasets except S-CIFAR-100.
    }
    %\vspace{-3mm}
    \resizebox{\textwidth}{!}{
    \begin{tabular}{l c c c c c c}
        \toprule
         & \multicolumn{2}{c}{{\bf Split MNIST} } & \multicolumn{2}{c}{{\bf Split FashionMNIST} } & \multicolumn{2}{c}{{\bf Split notMNIST} } \\
         \cmidrule(lr){2-3} \cmidrule(lr){4-5} \cmidrule(lr){6-7} %\cline{2-7}
         {\bf Method} & ACC(\%)$\uparrow$ & BWT(\%)$\uparrow$ & ACC(\%)$\uparrow$ & BWT(\%)$\uparrow$ & ACC(\%)$\uparrow$ & BWT(\%)$\uparrow$ \\
        \midrule 
        Random & 92.56 ($\pm$ 2.90) & -8.97 ($\pm$ 3.62) & 92.70 ($\pm$ 3.78) & -8.24 ($\pm$ 4.75) & 89.53 ($\pm$ 3.96) & -8.13 ($\pm$ 5.02) \\
        A-GEM & 94.97 ($\pm$ 1.50) & -6.03 ($\pm$ 1.87) & 94.81 ($\pm$ 0.86) & -5.65 ($\pm$ 1.06) & 92.27 ($\pm$ 1.16) & -4.17 ($\pm$ 1.39) \\
        ER-Ring & 94.94 ($\pm$ 1.56) & -6.07 ($\pm$ 1.92) & 95.83 ($\pm$ 2.15) & -4.38 ($\pm$ 2.59) & 91.10 ($\pm$ 1.89) & -6.27 ($\pm$ 2.35) \\ 
        Uniform & 95.77 ($\pm$ 1.12) & -5.02 ($\pm$ 1.39) & 97.12 ($\pm$ 1.57) & -2.79 ($\pm$ 1.98) & 92.14 ($\pm$ 1.45) & -4.90 ($\pm$ 1.41) \\ 
        \midrule 
        RS-MCTS (Ours) & 96.07 ($\pm$ 1.60) & -4.59 ($\pm$ 2.01) & 97.17 ($\pm$ 0.78) & -2.64 ($\pm$ 0.99) & 93.41 ($\pm$ 1.11) & -3.36 ($\pm$ 1.56)  \\ 
        \bottomrule
        \toprule
         & \multicolumn{2}{c}{{\bf Permuted MNIST} } & \multicolumn{2}{c}{{\bf Split CIFAR-100} } & \multicolumn{2}{c}{{\bf Split miniImagenet} } \\
         \cmidrule(lr){2-3} \cmidrule(lr){4-5} \cmidrule(lr){6-7} %\cline{2-7}
         {\bf Method} & ACC(\%)$\uparrow$ & BWT(\%)$\uparrow$ & ACC(\%)$\uparrow$ & BWT(\%)$\uparrow$ & ACC(\%)$\uparrow$ & BWT(\%)$\uparrow$ \\
        \midrule 
        Random & 70.02 ($\pm$ 1.76) & -28.22 ($\pm$ 1.92) & 48.62 ($\pm$ 1.02) & -39.95 ($\pm$ 1.10) & 48.85 ($\pm$ 1.38) & -14.55 ($\pm$ 1.86) \\
        A-GEM & 64.71 ($\pm$ 1.78) & -34.41 ($\pm$ 2.05) & 42.22 ($\pm$ 2.13) & -46.90 ($\pm$ 2.21) & 32.06 ($\pm$ 1.83) & -30.81 ($\pm$ 1.79) \\
        ER-Ring & 69.73 ($\pm$ 1.13) & -28.87 ($\pm$ 1.29) & 53.93 ($\pm$ 1.13) & -34.91 ($\pm$ 1.18) & 49.82 ($\pm$ 1.69) & -14.38 ($\pm$ 1.57) \\ 
        Uniform & 69.85 ($\pm$ 1.01) & -28.74 ($\pm$ 1.17) & 52.63 ($\pm$ 1.62) & -36.43 ($\pm$ 1.81) &  50.56 ($\pm$ 1.07) & -13.52 ($\pm$ 1.34) \\ 
        \midrule 
        RS-MCTS (Ours) & 72.52 ($\pm$ 0.54) & -25.43 ($\pm$ 0.65) & 51.50 ($\pm$ 1.19) & -37.01 ($\pm$ 1.08) & 50.70 ($\pm$ 0.54) & -12.60 ($\pm$ 1.13)  \\ 
        \bottomrule
    \end{tabular}
    }
    \vspace{-3mm}
    \label{tab:bwt_efficiency_of_replay_scheduling}
\end{table}

\begin{comment}

\begin{table}[h]
    \footnotesize
    \centering
    \caption{
    Performance comparison with ACC and BWT metrics for all datasets between RS-MCTS and the baselines in the setting where only 1 sample per class can be replayed. The memory sizes are set to $M=10$ and $M=100$ for the 5-task and 10/20-task datasets respectively. We report the mean and standard deviation of ACC and BWT, where all results have been averaged over 5 seeds. RS-MCTS performs on par with the best baselines for both metrics on all datasets except S-CIFAR-100.
    }
    %\vspace{-3mm}
    \begin{tabular}{l c c c c c c}
        \toprule
         & \multicolumn{2}{c}{{\bf Split MNIST} } & \multicolumn{2}{c}{{\bf Split FashionMNIST} } & \multicolumn{2}{c}{{\bf Split notMNIST} } \\
         \cmidrule(lr){2-3} \cmidrule(lr){4-5} \cmidrule(lr){6-7} %\cline{2-7}
         {\bf Method} & ACC(\%)$\uparrow$ & BWT(\%)$\uparrow$ & ACC(\%)$\uparrow$ & BWT(\%)$\uparrow$ & ACC(\%)$\uparrow$ & BWT(\%)$\uparrow$ \\
        \midrule 
        Random & 92.56 $\pm$ 2.90 & -8.97 $\pm$ 3.62 & 92.70 $\pm$ 3.78 & -8.24 $\pm$ 4.75 & 89.53 $\pm$ 3.96 & -8.13 $\pm$ 5.02  \\
        A-GEM & 94.97 $\pm$ 1.50 & -6.03 $\pm$ 1.87 & 94.81 $\pm$ 0.86 & -5.65 $\pm$ 1.06 & 92.27 $\pm$ 1.16 & -4.17 $\pm$ 1.39  \\
        ER-Ring & 94.94 $\pm$ 1.56 & -6.07 $\pm$ 1.92 & 95.83 $\pm$ 2.15 & -4.38 $\pm$ 2.59 & 91.10 $\pm$ 1.89 & -6.27 $\pm$ 2.35 \\ 
        Uniform & 95.77 $\pm$ 1.12 & -5.02 $\pm$ 1.39 & 97.12 $\pm$ 1.57 & -2.79 $\pm$ 1.98 & 92.14 $\pm$ 1.45 & -4.90 $\pm$ 1.41 \\ 
        \midrule 
        RS-MCTS (Ours) & 96.07 $\pm$ 1.60 & -4.59 $\pm$ 2.01  & 97.17 $\pm$ 0.78 & -2.64 $\pm$ 0.99 & 93.41 $\pm$ 1.11 & -3.36 $\pm$ 1.56  \\ 
        \bottomrule
        \toprule
         & \multicolumn{2}{c}{{\bf Permuted MNIST} } & \multicolumn{2}{c}{{\bf Split CIFAR-100} } & \multicolumn{2}{c}{{\bf Split miniImagenet} } \\
         \cmidrule(lr){2-3} \cmidrule(lr){4-5} \cmidrule(lr){6-7} %\cline{2-7}
         {\bf Method} & ACC(\%)$\uparrow$ & BWT(\%)$\uparrow$ & ACC(\%)$\uparrow$ & BWT(\%)$\uparrow$ & ACC(\%)$\uparrow$ & BWT(\%)$\uparrow$ \\
        \midrule 
        Random & 70.02 $\pm$ 1.76 & -28.22 $\pm$ 1.92 & 48.62 $\pm$ 1.02 & -39.95 $\pm$ 1.10 & 48.85 $\pm$ 1.38 & -14.55 $\pm$ 1.86  \\
        A-GEM & 64.71 $\pm$ 1.78 & -34.41 $\pm$ 2.05 & 42.22 $\pm$ 2.13 & -46.90 $\pm$ 2.21 & 32.06 $\pm$ 1.83 & -30.81 $\pm$ 1.79  \\
        ER-Ring & 69.73 $\pm$ 1.13 & -28.87 $\pm$ 1.29 & 53.93 $\pm$ 1.13 & -34.91 $\pm$ 1.18 & 49.82 $\pm$ 1.69 & -14.38 $\pm$ 1.57 \\ 
        Uniform & 69.85 $\pm$ 1.01 & -28.74 $\pm$ 1.17 & 52.63 $\pm$ 1.62 & -36.43 $\pm$ 1.81 &  50.56 $\pm$ 1.07 & -13.52 $\pm$ 1.34 \\ 
        \midrule 
        RS-MCTS (Ours) & 72.52 $\pm$ 0.54 & -25.43 $\pm$ 0.65 & 51.50 $\pm$ 1.19 & -37.01 $\pm$ 1.08 & 50.70 $\pm$ 0.54 & -12.60 $\pm$ 1.13  \\ 
        \bottomrule
    \end{tabular}
    \vspace{-3mm}
    \label{tab:bwt_efficiency_of_replay_scheduling}
\end{table}
\end{comment}





\begin{table}[t]
    \footnotesize
    \centering
    \caption{
    Performance comparison with ACC and BWT metrics between scheduling methods RS-MCTS (Ours), Random, ETS, and Heuristic when combining them with replay-based methods Hindsight Anchor Learning (HAL), Meta Experience Replay (MER), Dark Experience Replay (DER), and DER++. 
    Replay memory sizes are $M=10$ and $M=100$ for the 5-task and 10/20-task datasets respectively. We report the mean and standard deviation averaged over 5 seeds. Results on Heuristic where some seed did not converge is denoted by $^{*}$. Applying RS-MCTS to each method can enhance the performance compared to using the baseline schedules. 
    }
    %\vspace{-3mm}
    \scalebox{0.91}{
    \begin{tabular}{l l c c c c c c}
        \toprule
         & & \multicolumn{2}{c}{{\bf Split MNIST} } & \multicolumn{2}{c}{{\bf Split FashionMNIST} } & \multicolumn{2}{c}{{\bf Split notMNIST} } \\
         \cmidrule{3-4} \cmidrule{5-6} \cmidrule{7-8} %\cline{3-8}
         {\bf Method} & {\bf Schedule} & ACC(\%)$\uparrow$ & BWT(\%)$\uparrow$ & ACC(\%)$\uparrow$ & BWT(\%)$\uparrow$ & ACC(\%)$\uparrow$ & BWT(\%)$\uparrow$ \\
        \midrule 

        \multirow{4}{*}{HAL} & Random & 97.24 ($\pm$ 0.70) & -2.77 ($\pm$ 0.90) & 86.74 ($\pm$ 6.05) & -15.54 ($\pm$ 7.58) & 93.61 ($\pm$ 1.31) & -2.73 ($\pm$ 1.33) \\
         & ETS & 97.21 ($\pm$ 1.25) & -2.80 ($\pm$ 1.59) & 96.75 ($\pm$ 0.50) & -2.84 ($\pm$ 0.75) & 92.16 ($\pm$ 1.82) &  -5.04 ($\pm$ 2.24) \\
         & Heuristic & 97.69 ($\pm$ 0.19) & -2.22 ($\pm$ 0.24) & $^{*}$74.16 ($\pm$ 11.19) & -31.26 ($\pm$ 14.00) & 93.64 ($\pm$ 0.93) & -2.80 ($\pm$ 1.20) \\
         & Ours & 97.96 ($\pm$ 0.15) & -1.85 ($\pm$ 0.18) & 97.56 ($\pm$ 0.51) & -2.02 ($\pm$ 0.63) & 94.47 ($\pm$ 0.82) &  -1.67 ($\pm$ 0.64) \\
        \midrule 
        \multirow{4}{*}{MER} & Random & 93.07 ($\pm$ 0.81)  & -8.36 ($\pm$ 0.99) & 85.53 ($\pm$ 3.30) & -0.56 ($\pm$ 3.36) & 91.13 ($\pm$ 0.86) & -5.32 ($\pm$ 0.95) \\
         & ETS & 92.97 ($\pm$ 1.73) &  -8.52 ($\pm$ 2.15) & 84.88 ($\pm$ 3.85) & -3.34 ($\pm$ 5.59) & 90.56 ($\pm$ 0.83) & -6.11 ($\pm$ 1.06) \\
         & Heuristic & 94.30 ($\pm$ 2.79) & -6.46 ($\pm$ 3.50) & 96.91 ($\pm$ 0.62) & -1.34 ($\pm$ 0.76) & 90.90 ($\pm$ 1.30) & -6.24 ($\pm$ 1.96) \\
         & Ours & 96.44 ($\pm$ 0.72)  & -4.14 ($\pm$ 0.94)  & 86.67 ($\pm$ 4.09) & 0.85 ($\pm$ 3.85)  & 92.44 ($\pm$ 0.77) & -3.63 ($\pm$ 1.06) \\
        \midrule 
        \multirow{4}{*}{DER} & Random & 98.23 ($\pm$ 0.53) & -1.89 ($\pm$ 0.65) & 96.56 ($\pm$ 1.79) & -3.48 ($\pm$ 2.22) & 92.89 ($\pm$ 0.86) & -3.75 ($\pm$ 1.16) \\
         & ETS & 98.17 ($\pm$ 0.35) & -2.00 ($\pm$ 0.42) & 97.69 ($\pm$ 0.58) & -2.05 ($\pm$ 0.71) & 94.74 ($\pm$ 1.05) & -1.94 ($\pm$ 1.17) \\
         & Heuristic & 94.57 ($\pm$ 1.71) & -6.08 ($\pm$ 2.09) & $^{*}$72.49 ($\pm$ 19.32) & -20.88 ($\pm$ 11.46) & $^{*}$77.88 ($\pm$ 12.58) & -12.66 ($\pm$ 4.17) \\
         & Ours & 99.02 ($\pm$ 0.10) & -0.91 ($\pm$ 0.13) & 98.33 ($\pm$ 0.51) & -1.26 ($\pm$ 0.63) & 95.02 ($\pm$ 0.33) & -0.97 ($\pm$ 0.81) \\
        \midrule
        \multirow{4}{*}{DER++} & Random & 97.90 ($\pm$ 0.52) & -2.32 ($\pm$ 0.67) & 97.10 ($\pm$ 1.03) & -2.77 ($\pm$ 1.29) & 93.29 ($\pm$ 1.43) & -3.11 ($\pm$ 1.60 ) \\
         & ETS & 97.98 ($\pm$ 0.52) & -2.24 ($\pm$ 0.66) & 98.12 ($\pm$ 0.40) & -1.59 ($\pm$ 0.52) & 94.53 ($\pm$ 1.02) & -1.82 ($\pm$ 1.02) \\
         & Heuristic & 92.35 ($\pm$ 2.42) & -8.83 ($\pm$ 2.99) & $^{*}$67.31 ($\pm$ 21.20) & -24.86 ($\pm$ 16.34) & 93.88 ($\pm$ 1.33) & -2.86 ($\pm$ 1.49) \\
         & Ours & 98.84 ($\pm$ 0.21)  & -1.14 ($\pm$ 0.26)  & 98.38 ($\pm$ 0.43) & -1.17 ($\pm$ 0.51) & 94.73 ($\pm$ 0.20) & -1.21 ($\pm$ 1.12) \\
        \bottomrule
        \toprule
         & & \multicolumn{2}{c}{{\bf Permuted MNIST} } & \multicolumn{2}{c}{{\bf Split CIFAR-100} } & \multicolumn{2}{c}{{\bf Split miniImagenet} } \\
         \cmidrule{3-4} \cmidrule{5-6} \cmidrule{7-8} %\cline{3-8}
         {\bf Method} & {\bf Schedule} & ACC(\%)$\uparrow$ & BWT(\%)$\uparrow$ & ACC(\%)$\uparrow$ & BWT(\%)$\uparrow$ & ACC(\%)$\uparrow$ & BWT(\%)$\uparrow$ \\
        \midrule 
        \multirow{4}{*}{HAL} & Random & 88.49 ($\pm$ 0.99) & -7.03 ($\pm$ 1.05) & 36.09 ($\pm$ 1.77) & -17.49 ($\pm$ 1.78 ) & 38.51 ($\pm$ 2.22) & -6.65 ($\pm$ 1.43) \\
         & ETS & 88.46 ($\pm$ 0.86) & -7.26 ($\pm$ 0.90) & 34.90 ($\pm$ 2.02) & -18.92 ($\pm$ 0.91) & 38.13 ($\pm$ 1.18) & -8.19 ($\pm$ 1.73) \\
         & Heuristic & $^{*}$66.63 ($\pm$ 28.50) & -29.68 ($\pm$ 27.90) & 35.07 ($\pm$ 1.29) & -24.76 ($\pm$ 2.41) & 39.51 ($\pm$ 1.49) & -5.65 ($\pm$ 0.77) \\
         & Ours & 89.14 ($\pm$ 0.74) & -6.29 ($\pm$ 0.74) & 40.22 ($\pm$ 1.57) & -12.77 ($\pm$ 1.30) & 41.39 ($\pm$ 1.15) & -3.69 ($\pm$ 1.86) \\
        \midrule 
        \multirow{4}{*}{MER} & Random & 75.90 ($\pm$ 1.34) & -21.69 ($\pm$ 1.47) & 42.96 ($\pm$ 1.70) & -34.01 ($\pm$ 2.07) & 31.48 ($\pm$ 1.65) & -6.99 ($\pm$ 1.27) \\
         & ETS & 73.01 ($\pm$ 0.96) & -25.19 ($\pm$ 1.10) & 43.38 ($\pm$ 1.81) & -34.84 ($\pm$ 1.98) & 33.58 ($\pm$ 1.53) & -6.80 ($\pm$ 1.46) \\
         & Heuristic & 83.86 ($\pm$ 3.19) & -12.48 ($\pm$ 3.60) & 40.90 ($\pm$ 1.70) & -44.10 ($\pm$ 2.03) & 34.22 ($\pm$ 1.93) & -7.57 ($\pm$ 1.63) \\
         & Ours & 79.72 ($\pm$ 0.71) & -17.42 ($\pm$ 0.78) & 44.29 ($\pm$ 0.69) & -32.73 ($\pm$ 0.88) & 32.74 ($\pm$ 1.29) & -5.77 ($\pm$ 1.04) \\
        \midrule 
        \multirow{4}{*}{DER} & Random & 87.51 ($\pm$ 1.10) & -8.81 ($\pm$ 1.28) & 56.83 ($\pm$ 0.76) & -27.34 ($\pm$ 0.63) & 42.19  ($\pm$ 0.67) & -10.60  ($\pm$ 1.28) \\
         & ETS & 85.71 ($\pm$ 0.75) & -11.15 ($\pm$ 0.87) & 52.58 ($\pm$ 1.49) & -32.93 ($\pm$ 2.04) & 35.50  ($\pm$ 2.84) & -10.94  ($\pm$ 2.21) \\
         & Heuristic & 81.56 ($\pm$ 2.28) & -15.06 ($\pm$ 2.51) & 55.75 ($\pm$ 1.08) & -31.27 ($\pm$ 1.02) & 43.62 ($\pm$ 0.88) & -8.18 ($\pm$ 1.16) \\
         & Ours & 90.11 ($\pm$ 0.18) & -5.89 ($\pm$ 0.23) & 58.99 ($\pm$ 0.98) & -24.95 ($\pm$ 0.64) & 43.46  ($\pm$ 0.95) & -9.32  ($\pm$ 1.37) \\
        \midrule
        \multirow{4}{*}{DER++} & Random & 87.89 ($\pm$ 1.10& -8.35 ($\pm$ 1.33) & 58.49  ($\pm$ 1.44) & -25.48  ($\pm$ 1.55) & 48.40  ($\pm$ 0.69) & -4.20  ($\pm$ 0.86) \\
         & ETS & 85.25 ($\pm$ 0.88) & -11.60 ($\pm$ 1.03) & 52.54  ($\pm$ 1.06) & -33.22  ($\pm$ 1.51) & 41.36  ($\pm$ 2.90) & -4.07  ($\pm$ 2.28) \\
         & Heuristic & 79.17 ($\pm$ 2.44) & -17.68 ($\pm$ 2.68) & 56.70 ($\pm$ 1.27) & -30.33 ($\pm$ 1.41) & 45.73 ($\pm$ 0.84) & -6.09 ($\pm$ 1.24) \\
         & Ours & 89.84 ($\pm$ 0.22) & -6.13 ($\pm$ 0.29) & 59.23  ($\pm$ 0.83) & -24.61  ($\pm$ 0.91) & 49.45  ($\pm$ 0.68) & -3.12  ($\pm$ 0.89) \\
        \bottomrule
    \end{tabular}
    }
    \vspace{-3mm}
    \label{tab:bwt_sota_models_applied_to_rsmcts}
\end{table}


\begin{comment}
\begin{table}[h]
    \footnotesize
    \centering
    \caption{
    Performance comparison between scheduling methods RS-MCTS (Ours), Random, ETS, and Heuristic when combining them with replay-based methods Hindsight Anchor Learning (HAL), Meta Experience Replay (MER), Dark Experience Replay (DER), and DER++. 
    Replay memory sizes are $M=10$ and $M=100$ for the 5-task and 10/20-task datasets respectively. We report the mean and standard deviation averaged over 5 seeds. Results on Heuristic where some seed did not converge is denoted by $^{*}$. Applying RS-MCTS to each method can enhance the performance compared to using the baseline schedules. 
    }
    %\vspace{-3mm}
    \scalebox{0.9}{
    \begin{tabular}{l l c c c c c c}
        \toprule
         & & \multicolumn{2}{c}{{\bf Split MNIST} } & \multicolumn{2}{c}{{\bf Split FashionMNIST} } & \multicolumn{2}{c}{{\bf Split notMNIST} } \\
         \cmidrule{3-4} \cmidrule{5-6} \cmidrule{7-8} %\cline{3-8}
         {\bf Method} & {\bf Schedule} & ACC(\%)$\uparrow$ & BWT(\%)$\uparrow$ & ACC(\%)$\uparrow$ & BWT(\%)$\uparrow$ & ACC(\%)$\uparrow$ & BWT(\%)$\uparrow$ \\
        \midrule 

        \multirow{4}{*}{HAL} & Random & 97.24 $\pm$ 0.70 & -2.77 $\pm$ 0.90 & 86.74 $\pm$ 6.05 & -15.54 $\pm$ 7.58 & 93.61 $\pm$ 1.31 & -2.73 $\pm$ 1.33 \\
         & ETS & 97.21 $\pm$ 1.25 & -2.80 $\pm$ 1.59 & 96.75 $\pm$ 0.50 & -2.84 $\pm$ 0.75 & 92.16 $\pm$ 1.82 &  -5.04 $\pm$ 2.24 \\
         & Heuristic & 97.69 $\pm$ 0.19 & -2.22 $\pm$ 0.24 & $^{*}$74.16 $\pm$ 11.19 & -31.26 $\pm$ 14.00 & 93.64 $\pm$ 0.93 & -2.80 $\pm$ 1.20 \\
         & Ours & 97.96 $\pm$ 0.15 & -1.85 $\pm$ 0.18 & 97.56 $\pm$ 0.51 & -2.02 $\pm$ 0.63 & 94.47 $\pm$ 0.82 &  -1.67 $\pm$ 0.64 \\
        \midrule 
        \multirow{4}{*}{MER} & Random & 93.07 $\pm$ 0.81  & -8.36 $\pm$ 0.99 & 85.53 $\pm$ 3.30 & -0.56 $\pm$ 3.36 & 91.13 $\pm$ 0.86 & -5.32 $\pm$ 0.95 \\
         & ETS & 92.97 $\pm$ 1.73 &  -8.52 $\pm$ 2.15 & 84.88 $\pm$ 3.85 & -3.34 $\pm$ 5.59 & 90.56 $\pm$ 0.83 & -6.11 $\pm$ 1.06 \\
         & Heuristic & 94.30 $\pm$ 2.79 & -6.46 $\pm$ 3.50 & 96.91 $\pm$ 0.62 & -1.34 $\pm$ 0.76 & 90.90 $\pm$ 1.30 & -6.24 $\pm$ 1.96 \\
         & Ours & {96.44 $\pm$ 0.72}  & {-4.14 $\pm$ 0.94}  & 86.67 $\pm$ 4.09 & 0.85 $\pm$ 3.85  & 92.44 $\pm$ 0.77 & -3.63 $\pm$ 1.06 \\
        \midrule 
        \multirow{4}{*}{DER} & Random & 98.23 $\pm$ 0.53 & -1.89 $\pm$ 0.65 & 96.56 $\pm$ 1.79 & -3.48 $\pm$ 2.22 & 92.89 $\pm$ 0.86 & -3.75 $\pm$ 1.16 \\
         & ETS & 98.17 $\pm$ 0.35 & -2.00 $\pm$ 0.42 & 97.69 $\pm$ 0.58 & -2.05 $\pm$ 0.71 & 94.74 $\pm$ 1.05 & -1.94 $\pm$ 1.17 \\
         & Heuristic & 94.57 $\pm$ 1.71 & -6.08 $\pm$ 2.09 & $^{*}$72.49 $\pm$ 19.32 & -20.88 $\pm$ 11.46 & $^{*}$77.88 $\pm$ 12.58 & -12.66 $\pm$ 4.17 \\
         & Ours & {99.02 $\pm$ 0.10} & { -0.91 $\pm$ 0.13} & 98.33 $\pm$ 0.51 & -1.26 $\pm$ 0.63 & 95.02 $\pm$ 0.33 & -0.97 $\pm$ 0.81 \\
        \midrule
        \multirow{4}{*}{DER++} & Random & 97.90 $\pm$ 0.52 & -2.32 $\pm$ 0.67 & 97.10 $\pm$ 1.03 & -2.77 $\pm$ 1.29 & 93.29 $\pm$ 1.43 & -3.11 $\pm$ 1.60  \\
         & ETS & 97.98 $\pm$ 0.52 & -2.24 $\pm$ 0.66 & 98.12 $\pm$ 0.40 & -1.59 $\pm$ 0.52 & 94.53 $\pm$ 1.02 & -1.82 $\pm$ 1.02 \\
         & Heuristic & 92.35 $\pm$ 2.42 & -8.83 $\pm$ 2.99 & $^{*}$67.31 $\pm$ 21.20 & -24.86 $\pm$ 16.34 & 93.88 $\pm$ 1.33 & -2.86 $\pm$ 1.49 \\
         & Ours & {98.84 $\pm$ 0.21}  & {-1.14 $\pm$ 0.26}  & 98.38 $\pm$ 0.43 & -1.17 $\pm$ 0.51 & 94.73 $\pm$ 0.20 & -1.21 $\pm$ 1.12 \\
        \bottomrule
        \toprule
         & & \multicolumn{2}{c}{{\bf Permuted MNIST} } & \multicolumn{2}{c}{{\bf Split CIFAR-100} } & \multicolumn{2}{c}{{\bf Split miniImagenet} } \\
         \cmidrule{3-4} \cmidrule{5-6} \cmidrule{7-8} %\cline{3-8}
         {\bf Method} & {\bf Schedule} & ACC(\%)$\uparrow$ & BWT(\%)$\uparrow$ & ACC(\%)$\uparrow$ & BWT(\%)$\uparrow$ & ACC(\%)$\uparrow$ & BWT(\%)$\uparrow$ \\
        \midrule 
        \multirow{4}{*}{HAL} & Random & 88.49 $\pm$ 0.99 & -7.03 $\pm$ 1.05 & 36.09 $\pm$ 1.77 & -17.49 $\pm$ 1.78  & 38.51 $\pm$ 2.22 & -6.65 $\pm$ 1.43 \\
         & ETS & 88.46 $\pm$ 0.86 & -7.26 $\pm$ 0.90 & 34.90 $\pm$ 2.02 & -18.92 $\pm$ 0.91  & 38.13 $\pm$ 1.18 & -8.19 $\pm$ 1.73 \\
         & Heuristic & $^{*}$66.63 $\pm$ 28.50 & -29.68 $\pm$ 27.90 & 35.07 $\pm$ 1.29 & -24.76 $\pm$ 2.41 & 39.51 $\pm$ 1.49 & -5.65 $\pm$ 0.77 \\
         & Ours & 89.14 $\pm$ 0.74 & -6.29 $\pm$ 0.74 & 40.22 $\pm$ 1.57 & -12.77 $\pm$ 1.30  & 41.39 $\pm$ 1.15 & -3.69 $\pm$ 1.86\\
        \midrule 
        \multirow{4}{*}{MER} & Random & 75.90 $\pm$ 1.34  & -21.69 $\pm$ 1.47 & 42.96 $\pm$ 1.70 & -34.01 $\pm$ 2.07  & 31.48 $\pm$ 1.65 & -6.99 $\pm$ 1.27 \\
         & ETS & 73.01 $\pm$ 0.96 & -25.19 $\pm$ 1.10 & 43.38 $\pm$ 1.81 & -34.84 $\pm$ 1.98  & 33.58 $\pm$ 1.53 & -6.80 $\pm$ 1.46 \\
         & Heuristic & 83.86 $\pm$ 3.19 & -12.48 $\pm$ 3.60 & 40.90 $\pm$ 1.70 & -44.10 $\pm$ 2.03 & 34.22 $\pm$ 1.93 & -7.57 $\pm$ 1.63 \\
         & Ours & 79.72 $\pm$ 0.71 & -17.42 $\pm$ 0.78 & 44.29 $\pm$ 0.69 & -32.73 $\pm$ 0.88  & 32.74 $\pm$ 1.29 & -5.77 $\pm$ 1.04 \\
        \midrule 
        \multirow{4}{*}{DER} & Random & 87.51 $\pm$ 1.10 & -8.81 $\pm$ 1.28 & 56.83 $\pm$ 0.76 & -27.34 $\pm$ 0.63 & 42.19  $\pm$ 0.67 & -10.60  $\pm$ 1.28 \\
         & ETS & 85.71 $\pm$ 0.75 & -11.15 $\pm$ 0.87 & 52.58 $\pm$ 1.49 & -32.93 $\pm$ 2.04 & 35.50  $\pm$ 2.84 & -10.94  $\pm$ 2.21 \\
         & Heuristic & 81.56 $\pm$ 2.28 & -15.06 $\pm$ 2.51 & 55.75 $\pm$ 1.08 & -31.27 $\pm$ 1.02 & 43.62 $\pm$ 0.88 & -8.18 $\pm$ 1.16 \\
         & Ours & 90.11 $\pm$ 0.18 & -5.89 $\pm$ 0.23 & 58.99 $\pm$ 0.98 & -24.95 $\pm$ 0.64 & 43.46  $\pm$ 0.95 & -9.32  $\pm$ 1.37 \\
        \midrule
        \multirow{4}{*}{DER++} & Random & 87.89 $\pm$ 1.10& -8.35 $\pm$ 1.33  & 58.49  $\pm$ 1.44 & -25.48  $\pm$ 1.55  & 48.40  $\pm$ 0.69 & -4.20  $\pm$ 0.86 \\
         & ETS & 85.25 $\pm$ 0.88  & -11.60 $\pm$ 1.03  & 52.54  $\pm$ 1.06 & -33.22  $\pm$ 1.51 & 41.36  $\pm$ 2.90 & -4.07  $\pm$ 2.28 \\
         & Heuristic & 79.17 $\pm$ 2.44 & -17.68 $\pm$ 2.68 & 56.70 $\pm$ 1.27 & -30.33 $\pm$ 1.41 & 45.73 $\pm$ 0.84 & -6.09 $\pm$ 1.24 \\
         & Ours & 89.84 $\pm$ 0.22 & -6.13 $\pm$ 0.29 & 59.23  $\pm$ 0.83 & -24.61  $\pm$ 0.91  & 49.45  $\pm$ 0.68 & -3.12  $\pm$ 0.89 \\
        \bottomrule
    \end{tabular}
    }
    \vspace{-3mm}
    \label{tab:bwt_sota_models_applied_to_rsmcts}
\end{table}
\end{comment}


%\clearpage
%\clearpage


\begin{table}[h]
%\small
\centering
%\caption{Avg Rank  }
\label{tab:testing}
\resizebox{\textwidth}{!}{
\begin{tabular}{l c c c}
\toprule
 & \multicolumn{2}{c}{{\bf New Task Order}} & {\bf New Dataset} \\
 \toprule 
 {\bf Method} & {\bf Split MNIST} & {\bf Split CIFAR-10} & {\bf Split FashionMNIST}\\
 \midrule
 Random     & 3.70 & 3.98 & 3.80\\
 ETS        & 4.38 & 3.74 & 4.60 \\
 Heur-Global Drop    & 4.16 & 3.84 & {\bf 2.56} \\
 Heur-Local Drop    & {\bf 2.80} & 3.28 & 4.12  \\
 Heur-Fixed    & 2.86 & 3.26 & 2.96 \\
 \midrule
  DQN       & {\bf 3.10} & {\bf 2.90} & {\bf 2.96} \\
\bottomrule
\end{tabular}
}
\end{table}
\end{appendices}
%\clearpage

%\renewcommand*{\bibname}{References}
\bibliographystyleD{unsrt}
{\small
	\bibliographyD{PaperD/refD}
}