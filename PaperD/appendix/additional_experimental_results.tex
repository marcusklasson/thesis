
\section{Additional Experimental Results}\label{paperD:app:additional_experimental_results}

%\subsection{Complementary Experimental Results for Policy Generalization} 

Here, we provide complementary results for the policy generalization experiments in Section \ref{paperD:sec:results_on_policy_generalization}. Figure \ref{fig:policy_rewards_mnist_paperD}-\ref{fig:policy_rewards_fashionmnist_new_dataset_paperD} shows the performance progress measured in ACC in the test environments during training for the DQN and A2C. We also show the ACC achieved by the scheduling baselines as straight lines as these policies are fixed. Figure \ref{fig:policy_rewards_mnist_paperD} and \ref{fig:policy_rewards_cifar10_paperD} shows the performance progress for test environments with Split MNIST and Split CIFAR-10 respectively in the New Task Orders experiment. Figure \ref{fig:policy_rewards_notmnist_new_dataset_paperD} and \ref{fig:policy_rewards_fashionmnist_new_dataset_paperD} shows the performance progress for test environments with Split notMNIST and Split FashionMNIST respectively in the New Dataset experiment. In general, we observe that DQN exhibits noisier progress of the achieved ACC in the test environments than A2C. 

In Table \ref{tab:rankings_mnist_new_task_order}-\ref{tab:rankings_fashionmnist_new_dataset}, we display the ACC and rank in every test environment that was used for generating the average rankings in Table \ref{tab:average_ranking_rl_experiment} (see Section \ref{paperD:sec:results_on_policy_generalization}). Table \ref{tab:rankings_mnist_new_task_order}, \ref{tab:rankings_fashionmnist_new_task_order}, and \ref{tab:rankings_cifar10_new_task_order} shows the ACCs and ranks for the New Task Orders experiments with datasets Split MNIST, FashionMNIST, and CIFAR-10 respectively. Table \ref{tab:rankings_notmnist_new_dataset} and \ref{tab:rankings_fashionmnist_new_dataset} shows the ACCs and ranks for the New Dataset experiments with datasets Split notMNIST and FashionMNIST respectively. The numbers for the average rankings in Table \ref{tab:average_ranking_rl_experiment} are computed by averaging over the ranks in each test environment for every method separately. 


\clearpage
%%% Figures 
\begin{figure}[t]
  \centering
  \setlength{\figwidth}{0.26\textwidth}
  \setlength{\figheight}{.14\textheight}
  
\pgfplotsset{every axis title/.append style={at={(0.5,0.82)}}}
\pgfplotsset{every tick label/.append style={font=\tiny}}
\pgfplotsset{every major tick/.append style={major tick length=2pt}}
\pgfplotsset{every minor tick/.append style={minor tick length=1pt}}
\pgfplotsset{every axis x label/.append style={at={(0.5,-0.22)}}}
\pgfplotsset{every axis y label/.append style={at={(-0.22,0.5)}}}
\begin{tikzpicture}
\tikzstyle{every node}=[font=\scriptsize]
\definecolor{color0}{rgb}{0.12156862745098,0.466666666666667,0.705882352941177}
\definecolor{color1}{rgb}{1,0.498039215686275,0.0549019607843137}
\definecolor{color2}{rgb}{0.172549019607843,0.627450980392157,0.172549019607843}
\definecolor{color3}{rgb}{0.83921568627451,0.152941176470588,0.156862745098039}
\definecolor{color4}{rgb}{0.580392156862745,0.403921568627451,0.741176470588235}

\begin{groupplot}[group style={group size= 4 by 1, horizontal sep=1.40cm, vertical sep=1.00cm}]



\nextgroupplot[title={DQN (ACC: 95.50\%)} ,
height=\figheight,
legend cell align={left},
legend columns=-1,
legend style={
  nodes={scale=0.9},
  fill opacity=0.8,
  draw opacity=1,
  text opacity=1,
  at={(1.20,1.40)},
  anchor=south west,
  draw=white!80!black
},
minor xtick={},
minor ytick={0.55,0.65,0.75,0.85,0.95,1.05},
tick align=outside,
tick pos=left,
width=\figwidth,
x grid style={white!69.0196078431373!black},
xmajorgrids,
xlabel={Task},
xmin=0.8, xmax=5.2,
xtick style={color=black},
xtick={1,2,3,4,5},
xticklabel style={font=\tiny, inner sep=1pt},
xticklabels={1,2,3,4,5},
y grid style={white!69.0196078431373!black},
ymajorgrids,
yminorgrids,
ylabel={Accuracy},
ymin=0.491, ymax=1.01,
ytick style={color=black},
ytick={0.5,0.6,0.7,0.8,0.9,1.0},
yticklabel style={font=\tiny, inner sep=0pt},
yticklabels={50, 60, 70, 80, 90, 100}
]
\addplot [color0, mark=*, mark size=1.5, mark options={solid}]
table {%
1 0.994017958641052
2 0.912761688232422
3 0.868394792079926
4 0.750747740268707
5 0.837986052036285
};\addlegendentry{Task 1}
\addplot [ color1, mark=*, mark size=1.5, mark options={solid}]
table {%
2 0.99297297000885
3 0.918378353118896
4 0.967027008533478
5 0.969729721546173
};\addlegendentry{Task 2}
\addplot [ color2, mark=*, mark size=1.5, mark options={solid}]
table {%
3 0.999067604541779
4 0.987412571907043
5 0.981818199157715
};\addlegendentry{Task 3}
\addplot [ color3, mark=*, mark size=1.5, mark options={solid}]
table {%
4 0.996513962745667
5 0.994521915912628
};\addlegendentry{Task 4}
\addplot [color4, mark=*, mark size=1.5, mark options={solid}]
table {%
5 0.990959346294403
};\addlegendentry{Task 5}



\input{Chapter4/figures/policy_illustrations/mnist/data_seed10/dqn_seed2/bubble_rs_dataset_seed_10_dqn_seed_2}




\nextgroupplot[title={Heur-LD (ACC: 90.77\%)} ,
height=\figheight,
legend cell align={left},
legend style={
  nodes={scale=0.9},
  fill opacity=0.8,
  draw opacity=1,
  text opacity=1,
  at={(3.72,0.28)},
  anchor=south west,
  draw=white!80!black
},
minor xtick={},
minor ytick={0.55,0.65,0.75,0.85,0.95,1.05},
tick align=outside,
tick pos=left,
width=\figwidth,
x grid style={white!69.0196078431373!black},
xmajorgrids,
xlabel={Task},
xmin=0.8, xmax=5.2,
xtick style={color=black},
xtick={1,2,3,4,5},
xticklabels={1,2,3,4,5},
xticklabel style={font=\tiny, inner sep=1pt},
y grid style={white!69.0196078431373!black},
ymajorgrids,
yminorgrids,
ylabel={Accuracy},
ymin=0.491, ymax=1.01,
ytick style={color=black},
ytick={0.5,0.6,0.7,0.8,0.9,1.0},
yticklabels={50, 60, 70, 80, 90, 100},
yticklabel style={font=\tiny, inner sep=0pt}
]
\addplot [color0, mark=*, mark size=1.5, mark options={solid}]
table {%
1 0.994017946161516
2 0.777168494516451
3 0.809571286141575
4 0.567298105682951
5 0.708374875373878
};
%\addlegendentry{Task 1}
\addplot [color1, mark=*, mark size=1.5, mark options={solid}]
table {%
2 0.994054054054054
3 0.970810810810811
4 0.792972972972973
5 0.855675675675676
};
%\addlegendentry{Task 2}
\addplot [color2, mark=*, mark size=1.5, mark options={solid}]
table {%
3 0.998601398601399
4 0.990675990675991
5 0.995804195804196
};
%\addlegendentry{Task 3}
\addplot [color3, mark=*, mark size=1.5, mark options={solid}]
table {%
4 0.99601593625498
5 0.989043824701195
};
%\addlegendentry{Task 4}
\addplot [color4, mark=*, mark size=1.5, mark options={solid}]
table {%
5 0.989452536413862
};
%\addlegendentry{Task 5}
%\addlegendentry{Task 5}




\input{Chapter4/figures/policy_illustrations/mnist/data_seed10/heuristic_local_drop/bubble_plot}

\end{groupplot}

\end{tikzpicture}
  \vspace{-3mm}
  \caption{Performance progress measured in ACC (\%) for all methods in the 10 test environments for {\bf Split MNIST} in the New Task Orders experiment. We plot the performance progress for A2C and DQN for 100 evaluation steps equidistantly distributed over the training episodes. The performance for the Random, ETS, and Heuristic scheduling baselines are plotted as straight lines.  }
  \label{fig:policy_rewards_mnist_paperD}
  \vspace{-3mm}
\end{figure}


\begin{figure}[t]
  \centering
  \setlength{\figwidth}{0.26\textwidth}
  \setlength{\figheight}{.14\textheight}
  
\pgfplotsset{every axis title/.append style={at={(0.5,0.82)}}}
\pgfplotsset{every tick label/.append style={font=\tiny}}
\pgfplotsset{every major tick/.append style={major tick length=2pt}}
\pgfplotsset{every minor tick/.append style={minor tick length=1pt}}
\pgfplotsset{every axis x label/.append style={at={(0.5,-0.22)}}}
\pgfplotsset{every axis y label/.append style={at={(-0.22,0.5)}}}
\begin{tikzpicture}
\tikzstyle{every node}=[font=\scriptsize]
\definecolor{color0}{rgb}{0.12156862745098,0.466666666666667,0.705882352941177}
\definecolor{color1}{rgb}{1,0.498039215686275,0.0549019607843137}
\definecolor{color2}{rgb}{0.172549019607843,0.627450980392157,0.172549019607843}
\definecolor{color3}{rgb}{0.83921568627451,0.152941176470588,0.156862745098039}
\definecolor{color4}{rgb}{0.580392156862745,0.403921568627451,0.741176470588235}

\begin{groupplot}[group style={group size= 4 by 1, horizontal sep=1.40cm, vertical sep=1.00cm}]



\nextgroupplot[title={DQN (ACC: 95.50\%)} ,
height=\figheight,
legend cell align={left},
legend columns=-1,
legend style={
  nodes={scale=0.9},
  fill opacity=0.8,
  draw opacity=1,
  text opacity=1,
  at={(1.20,1.40)},
  anchor=south west,
  draw=white!80!black
},
minor xtick={},
minor ytick={0.55,0.65,0.75,0.85,0.95,1.05},
tick align=outside,
tick pos=left,
width=\figwidth,
x grid style={white!69.0196078431373!black},
xmajorgrids,
xlabel={Task},
xmin=0.8, xmax=5.2,
xtick style={color=black},
xtick={1,2,3,4,5},
xticklabel style={font=\tiny, inner sep=1pt},
xticklabels={1,2,3,4,5},
y grid style={white!69.0196078431373!black},
ymajorgrids,
yminorgrids,
ylabel={Accuracy},
ymin=0.491, ymax=1.01,
ytick style={color=black},
ytick={0.5,0.6,0.7,0.8,0.9,1.0},
yticklabel style={font=\tiny, inner sep=0pt},
yticklabels={50, 60, 70, 80, 90, 100}
]
\addplot [color0, mark=*, mark size=1.5, mark options={solid}]
table {%
1 0.994017958641052
2 0.912761688232422
3 0.868394792079926
4 0.750747740268707
5 0.837986052036285
};\addlegendentry{Task 1}
\addplot [ color1, mark=*, mark size=1.5, mark options={solid}]
table {%
2 0.99297297000885
3 0.918378353118896
4 0.967027008533478
5 0.969729721546173
};\addlegendentry{Task 2}
\addplot [ color2, mark=*, mark size=1.5, mark options={solid}]
table {%
3 0.999067604541779
4 0.987412571907043
5 0.981818199157715
};\addlegendentry{Task 3}
\addplot [ color3, mark=*, mark size=1.5, mark options={solid}]
table {%
4 0.996513962745667
5 0.994521915912628
};\addlegendentry{Task 4}
\addplot [color4, mark=*, mark size=1.5, mark options={solid}]
table {%
5 0.990959346294403
};\addlegendentry{Task 5}



\input{Chapter4/figures/policy_illustrations/mnist/data_seed10/dqn_seed2/bubble_rs_dataset_seed_10_dqn_seed_2}




\nextgroupplot[title={Heur-LD (ACC: 90.77\%)} ,
height=\figheight,
legend cell align={left},
legend style={
  nodes={scale=0.9},
  fill opacity=0.8,
  draw opacity=1,
  text opacity=1,
  at={(3.72,0.28)},
  anchor=south west,
  draw=white!80!black
},
minor xtick={},
minor ytick={0.55,0.65,0.75,0.85,0.95,1.05},
tick align=outside,
tick pos=left,
width=\figwidth,
x grid style={white!69.0196078431373!black},
xmajorgrids,
xlabel={Task},
xmin=0.8, xmax=5.2,
xtick style={color=black},
xtick={1,2,3,4,5},
xticklabels={1,2,3,4,5},
xticklabel style={font=\tiny, inner sep=1pt},
y grid style={white!69.0196078431373!black},
ymajorgrids,
yminorgrids,
ylabel={Accuracy},
ymin=0.491, ymax=1.01,
ytick style={color=black},
ytick={0.5,0.6,0.7,0.8,0.9,1.0},
yticklabels={50, 60, 70, 80, 90, 100},
yticklabel style={font=\tiny, inner sep=0pt}
]
\addplot [color0, mark=*, mark size=1.5, mark options={solid}]
table {%
1 0.994017946161516
2 0.777168494516451
3 0.809571286141575
4 0.567298105682951
5 0.708374875373878
};
%\addlegendentry{Task 1}
\addplot [color1, mark=*, mark size=1.5, mark options={solid}]
table {%
2 0.994054054054054
3 0.970810810810811
4 0.792972972972973
5 0.855675675675676
};
%\addlegendentry{Task 2}
\addplot [color2, mark=*, mark size=1.5, mark options={solid}]
table {%
3 0.998601398601399
4 0.990675990675991
5 0.995804195804196
};
%\addlegendentry{Task 3}
\addplot [color3, mark=*, mark size=1.5, mark options={solid}]
table {%
4 0.99601593625498
5 0.989043824701195
};
%\addlegendentry{Task 4}
\addplot [color4, mark=*, mark size=1.5, mark options={solid}]
table {%
5 0.989452536413862
};
%\addlegendentry{Task 5}
%\addlegendentry{Task 5}




\input{Chapter4/figures/policy_illustrations/mnist/data_seed10/heuristic_local_drop/bubble_plot}

\end{groupplot}

\end{tikzpicture}
  \vspace{-3mm}
  \caption{Performance progress measured in ACC (\%) for all methods in the 10 test environments for {\bf Split CIFAR-10} in the New Task Orders experiment. We plot the performance progress for A2C and DQN for 100 evaluation steps equidistantly distributed over the training episodes. The performance for the Random, ETS, and Heuristic scheduling baselines are plotted as straight lines.  }
  \label{fig:policy_rewards_cifar10_paperD}
  \vspace{-3mm}
\end{figure}

\clearpage

\begin{figure}[t]
  \centering
  \setlength{\figwidth}{0.26\textwidth}
  \setlength{\figheight}{.14\textheight}
  
\pgfplotsset{every axis title/.append style={at={(0.5,0.82)}}}
\pgfplotsset{every tick label/.append style={font=\tiny}}
\pgfplotsset{every major tick/.append style={major tick length=2pt}}
\pgfplotsset{every minor tick/.append style={minor tick length=1pt}}
\pgfplotsset{every axis x label/.append style={at={(0.5,-0.22)}}}
\pgfplotsset{every axis y label/.append style={at={(-0.22,0.5)}}}
\begin{tikzpicture}
\tikzstyle{every node}=[font=\scriptsize]
\definecolor{color0}{rgb}{0.12156862745098,0.466666666666667,0.705882352941177}
\definecolor{color1}{rgb}{1,0.498039215686275,0.0549019607843137}
\definecolor{color2}{rgb}{0.172549019607843,0.627450980392157,0.172549019607843}
\definecolor{color3}{rgb}{0.83921568627451,0.152941176470588,0.156862745098039}
\definecolor{color4}{rgb}{0.580392156862745,0.403921568627451,0.741176470588235}

\begin{groupplot}[group style={group size= 4 by 1, horizontal sep=1.40cm, vertical sep=1.00cm}]



\nextgroupplot[title={DQN (ACC: 95.50\%)} ,
height=\figheight,
legend cell align={left},
legend columns=-1,
legend style={
  nodes={scale=0.9},
  fill opacity=0.8,
  draw opacity=1,
  text opacity=1,
  at={(1.20,1.40)},
  anchor=south west,
  draw=white!80!black
},
minor xtick={},
minor ytick={0.55,0.65,0.75,0.85,0.95,1.05},
tick align=outside,
tick pos=left,
width=\figwidth,
x grid style={white!69.0196078431373!black},
xmajorgrids,
xlabel={Task},
xmin=0.8, xmax=5.2,
xtick style={color=black},
xtick={1,2,3,4,5},
xticklabel style={font=\tiny, inner sep=1pt},
xticklabels={1,2,3,4,5},
y grid style={white!69.0196078431373!black},
ymajorgrids,
yminorgrids,
ylabel={Accuracy},
ymin=0.491, ymax=1.01,
ytick style={color=black},
ytick={0.5,0.6,0.7,0.8,0.9,1.0},
yticklabel style={font=\tiny, inner sep=0pt},
yticklabels={50, 60, 70, 80, 90, 100}
]
\addplot [color0, mark=*, mark size=1.5, mark options={solid}]
table {%
1 0.994017958641052
2 0.912761688232422
3 0.868394792079926
4 0.750747740268707
5 0.837986052036285
};\addlegendentry{Task 1}
\addplot [ color1, mark=*, mark size=1.5, mark options={solid}]
table {%
2 0.99297297000885
3 0.918378353118896
4 0.967027008533478
5 0.969729721546173
};\addlegendentry{Task 2}
\addplot [ color2, mark=*, mark size=1.5, mark options={solid}]
table {%
3 0.999067604541779
4 0.987412571907043
5 0.981818199157715
};\addlegendentry{Task 3}
\addplot [ color3, mark=*, mark size=1.5, mark options={solid}]
table {%
4 0.996513962745667
5 0.994521915912628
};\addlegendentry{Task 4}
\addplot [color4, mark=*, mark size=1.5, mark options={solid}]
table {%
5 0.990959346294403
};\addlegendentry{Task 5}



\input{Chapter4/figures/policy_illustrations/mnist/data_seed10/dqn_seed2/bubble_rs_dataset_seed_10_dqn_seed_2}




\nextgroupplot[title={Heur-LD (ACC: 90.77\%)} ,
height=\figheight,
legend cell align={left},
legend style={
  nodes={scale=0.9},
  fill opacity=0.8,
  draw opacity=1,
  text opacity=1,
  at={(3.72,0.28)},
  anchor=south west,
  draw=white!80!black
},
minor xtick={},
minor ytick={0.55,0.65,0.75,0.85,0.95,1.05},
tick align=outside,
tick pos=left,
width=\figwidth,
x grid style={white!69.0196078431373!black},
xmajorgrids,
xlabel={Task},
xmin=0.8, xmax=5.2,
xtick style={color=black},
xtick={1,2,3,4,5},
xticklabels={1,2,3,4,5},
xticklabel style={font=\tiny, inner sep=1pt},
y grid style={white!69.0196078431373!black},
ymajorgrids,
yminorgrids,
ylabel={Accuracy},
ymin=0.491, ymax=1.01,
ytick style={color=black},
ytick={0.5,0.6,0.7,0.8,0.9,1.0},
yticklabels={50, 60, 70, 80, 90, 100},
yticklabel style={font=\tiny, inner sep=0pt}
]
\addplot [color0, mark=*, mark size=1.5, mark options={solid}]
table {%
1 0.994017946161516
2 0.777168494516451
3 0.809571286141575
4 0.567298105682951
5 0.708374875373878
};
%\addlegendentry{Task 1}
\addplot [color1, mark=*, mark size=1.5, mark options={solid}]
table {%
2 0.994054054054054
3 0.970810810810811
4 0.792972972972973
5 0.855675675675676
};
%\addlegendentry{Task 2}
\addplot [color2, mark=*, mark size=1.5, mark options={solid}]
table {%
3 0.998601398601399
4 0.990675990675991
5 0.995804195804196
};
%\addlegendentry{Task 3}
\addplot [color3, mark=*, mark size=1.5, mark options={solid}]
table {%
4 0.99601593625498
5 0.989043824701195
};
%\addlegendentry{Task 4}
\addplot [color4, mark=*, mark size=1.5, mark options={solid}]
table {%
5 0.989452536413862
};
%\addlegendentry{Task 5}
%\addlegendentry{Task 5}




\input{Chapter4/figures/policy_illustrations/mnist/data_seed10/heuristic_local_drop/bubble_plot}

\end{groupplot}

\end{tikzpicture}
  \vspace{-3mm}
  \caption{Performance progress measured in ACC (\%) for all methods in the 10 test environments for {\bf Split notMNIST} in the New Dataset experiment. We plot the performance progress for A2C and DQN for 100 evaluation steps equidistantly distributed over the training episodes. The performance for the Random, ETS, and Heuristic scheduling baselines are plotted as straight lines.  }
  \label{fig:policy_rewards_notmnist_new_dataset_paperD}
  \vspace{-3mm}
\end{figure}

\begin{figure}[t]
  \centering
  \setlength{\figwidth}{0.26\textwidth}
  \setlength{\figheight}{.14\textheight}
  
\pgfplotsset{every axis title/.append style={at={(0.5,0.82)}}}
\pgfplotsset{every tick label/.append style={font=\tiny}}
\pgfplotsset{every major tick/.append style={major tick length=2pt}}
\pgfplotsset{every minor tick/.append style={minor tick length=1pt}}
\pgfplotsset{every axis x label/.append style={at={(0.5,-0.22)}}}
\pgfplotsset{every axis y label/.append style={at={(-0.22,0.5)}}}
\begin{tikzpicture}
\tikzstyle{every node}=[font=\scriptsize]
\definecolor{color0}{rgb}{0.12156862745098,0.466666666666667,0.705882352941177}
\definecolor{color1}{rgb}{1,0.498039215686275,0.0549019607843137}
\definecolor{color2}{rgb}{0.172549019607843,0.627450980392157,0.172549019607843}
\definecolor{color3}{rgb}{0.83921568627451,0.152941176470588,0.156862745098039}
\definecolor{color4}{rgb}{0.580392156862745,0.403921568627451,0.741176470588235}

\begin{groupplot}[group style={group size= 4 by 1, horizontal sep=1.40cm, vertical sep=1.00cm}]



\nextgroupplot[title={DQN (ACC: 95.50\%)} ,
height=\figheight,
legend cell align={left},
legend columns=-1,
legend style={
  nodes={scale=0.9},
  fill opacity=0.8,
  draw opacity=1,
  text opacity=1,
  at={(1.20,1.40)},
  anchor=south west,
  draw=white!80!black
},
minor xtick={},
minor ytick={0.55,0.65,0.75,0.85,0.95,1.05},
tick align=outside,
tick pos=left,
width=\figwidth,
x grid style={white!69.0196078431373!black},
xmajorgrids,
xlabel={Task},
xmin=0.8, xmax=5.2,
xtick style={color=black},
xtick={1,2,3,4,5},
xticklabel style={font=\tiny, inner sep=1pt},
xticklabels={1,2,3,4,5},
y grid style={white!69.0196078431373!black},
ymajorgrids,
yminorgrids,
ylabel={Accuracy},
ymin=0.491, ymax=1.01,
ytick style={color=black},
ytick={0.5,0.6,0.7,0.8,0.9,1.0},
yticklabel style={font=\tiny, inner sep=0pt},
yticklabels={50, 60, 70, 80, 90, 100}
]
\addplot [color0, mark=*, mark size=1.5, mark options={solid}]
table {%
1 0.994017958641052
2 0.912761688232422
3 0.868394792079926
4 0.750747740268707
5 0.837986052036285
};\addlegendentry{Task 1}
\addplot [ color1, mark=*, mark size=1.5, mark options={solid}]
table {%
2 0.99297297000885
3 0.918378353118896
4 0.967027008533478
5 0.969729721546173
};\addlegendentry{Task 2}
\addplot [ color2, mark=*, mark size=1.5, mark options={solid}]
table {%
3 0.999067604541779
4 0.987412571907043
5 0.981818199157715
};\addlegendentry{Task 3}
\addplot [ color3, mark=*, mark size=1.5, mark options={solid}]
table {%
4 0.996513962745667
5 0.994521915912628
};\addlegendentry{Task 4}
\addplot [color4, mark=*, mark size=1.5, mark options={solid}]
table {%
5 0.990959346294403
};\addlegendentry{Task 5}



\input{Chapter4/figures/policy_illustrations/mnist/data_seed10/dqn_seed2/bubble_rs_dataset_seed_10_dqn_seed_2}




\nextgroupplot[title={Heur-LD (ACC: 90.77\%)} ,
height=\figheight,
legend cell align={left},
legend style={
  nodes={scale=0.9},
  fill opacity=0.8,
  draw opacity=1,
  text opacity=1,
  at={(3.72,0.28)},
  anchor=south west,
  draw=white!80!black
},
minor xtick={},
minor ytick={0.55,0.65,0.75,0.85,0.95,1.05},
tick align=outside,
tick pos=left,
width=\figwidth,
x grid style={white!69.0196078431373!black},
xmajorgrids,
xlabel={Task},
xmin=0.8, xmax=5.2,
xtick style={color=black},
xtick={1,2,3,4,5},
xticklabels={1,2,3,4,5},
xticklabel style={font=\tiny, inner sep=1pt},
y grid style={white!69.0196078431373!black},
ymajorgrids,
yminorgrids,
ylabel={Accuracy},
ymin=0.491, ymax=1.01,
ytick style={color=black},
ytick={0.5,0.6,0.7,0.8,0.9,1.0},
yticklabels={50, 60, 70, 80, 90, 100},
yticklabel style={font=\tiny, inner sep=0pt}
]
\addplot [color0, mark=*, mark size=1.5, mark options={solid}]
table {%
1 0.994017946161516
2 0.777168494516451
3 0.809571286141575
4 0.567298105682951
5 0.708374875373878
};
%\addlegendentry{Task 1}
\addplot [color1, mark=*, mark size=1.5, mark options={solid}]
table {%
2 0.994054054054054
3 0.970810810810811
4 0.792972972972973
5 0.855675675675676
};
%\addlegendentry{Task 2}
\addplot [color2, mark=*, mark size=1.5, mark options={solid}]
table {%
3 0.998601398601399
4 0.990675990675991
5 0.995804195804196
};
%\addlegendentry{Task 3}
\addplot [color3, mark=*, mark size=1.5, mark options={solid}]
table {%
4 0.99601593625498
5 0.989043824701195
};
%\addlegendentry{Task 4}
\addplot [color4, mark=*, mark size=1.5, mark options={solid}]
table {%
5 0.989452536413862
};
%\addlegendentry{Task 5}
%\addlegendentry{Task 5}




\input{Chapter4/figures/policy_illustrations/mnist/data_seed10/heuristic_local_drop/bubble_plot}

\end{groupplot}

\end{tikzpicture}
  \vspace{-3mm}
  \caption{Performance progress measured in ACC (\%) for all methods in the 10 test environments for {\bf Split FashionMNIST} in the New Dataset experiment. We plot the performance progress for A2C and DQN for 100 evaluation steps equidistantly distributed over the training episodes. The performance for the Random, ETS, and Heuristic scheduling baselines are plotted as straight lines.  }
  \label{fig:policy_rewards_fashionmnist_new_dataset_paperD}
  \vspace{-3mm}
\end{figure}

%%% Tables
\clearpage

\begin{table}[t]
\caption{The ACC and rank for each seed (10-19) in every test environment with {\bf Split MNIST} for New Task Order experiment. Under each column named 'Seed X', we show the ACC (\%) and, in parenthesis, the rank for the corresponding method for the specific seed. We also show the average rank of each method in the test environment under 'Rank'. Note that the ACC for ETS, Heur-GD, Heur-LD, and Heur-AT are copied across the seeds, since the seed only affects the policy initialization in DQN and A2C and the random selection in Random.}
\label{tab:rankings_mnist_new_task_order}
\resizebox{\textwidth}{!}{
\begin{tabular}{lcccccccc}
\toprule
 \multicolumn{1}{l}{}   & \multicolumn{4}{c}{{\bf Test Env. Seed 10}}             & \multicolumn{4}{c}{{\bf Test Env. Seed 11}} \\
    \cmidrule(lr){2-5} \cmidrule(lr){6-9} 
Method     & Seed 1     & Seed 2     & Seed 3     & Rank & Seed 1     & Seed 2     & Seed 3     & Rank \\ \midrule
Random     & 93.2 (6.0) & 93.7 (6.0) & 96.2 (1.0) & 4.33 & 92.2 (3.0) & 90.8 (7.0) & 90.5 (7.0) & 5.67 \\
ETS        & 89.9 (7.0) & 89.9 (7.0) & 89.9 (7.0) & 7    & 93.8 (2.0) & 93.8 (3.0) & 93.8 (3.0) & 2.67 \\
Heur-GD & 94.6 (3.5) & 94.6 (4.5) & 94.6 (5.5) & 4.5  & 91.3 (6.0) & 91.3 (5.0) & 91.3 (5.0) & 5.33 \\
Heur-LD & 95.6 (1.0) & 95.6 (1.0) & 95.6 (3.0) & 1.67 & 91.3 (6.0) & 91.3 (5.0) & 91.3 (5.0) & 5.33 \\
Heur-AT & 94.6 (3.5) & 94.6 (4.5) & 94.6 (5.5) & 4.5  & 91.3 (6.0) & 91.3 (5.0) & 91.3 (5.0) & 5.33 \\
DQN        & 93.2 (5.0) & 95.5 (2.0) & 95.7 (2.0) & 3    & 91.8 (4.0) & 96.0 (2.0) & 95.7 (2.0) & 2.67 \\
A2C        & 95.3 (2.0) & 95.3 (3.0) & 95.3 (4.0) & 3    & 96.5 (1.0) & 96.5 (1.0) & 96.5 (1.0) & 1    \\ \midrule
    & \multicolumn{4}{c}{{\bf Test Env. Seed 12}}             & \multicolumn{4}{c}{{\bf Test Env. Seed 13}} \\
    \cmidrule(lr){2-5} \cmidrule(lr){6-9} 
Method     & Seed 1     & Seed 2     & Seed 3     & Rank & Seed 1     & Seed 2     & Seed 3     & Rank \\ \midrule
Random     & 90.7 (7.0) & 88.8 (7.0) & 95.2 (1.0) & 5    & 93.4 (5.0) & 94.4 (6.0) & 89.9 (7.0) & 6    \\
ETS        & 91.7 (6.0) & 91.7 (6.0) & 91.7 (7.0) & 6.33 & 95.0 (2.0) & 95.0 (3.0) & 95.0 (3.0) & 2.67 \\
Heur-GD & 94.8 (1.5) & 94.8 (1.5) & 94.8 (2.5) & 1.83 & 94.7 (3.5) & 94.7 (4.5) & 94.7 (4.5) & 4.17 \\
Heur-LD & 91.9 (5.0) & 91.9 (5.0) & 91.9 (6.0) & 5.33 & 93.2 (7.0) & 93.2 (7.0) & 93.2 (6.0) & 6.67 \\
Heur-AT & 94.8 (1.5) & 94.8 (1.5) & 94.8 (2.5) & 1.83 & 94.7 (3.5) & 94.7 (4.5) & 94.7 (4.5) & 4.17 \\
DQN        & 94.2 (3.0) & 94.5 (3.0) & 92.8 (5.0) & 3.67 & 93.2 (6.0) & 96.6 (1.0) & 96.2 (1.0) & 2.67 \\
A2C        & 93.6 (4.0) & 93.6 (4.0) & 93.6 (4.0) & 4    & 95.6 (1.0) & 95.6 (2.0) & 95.6 (2.0) & 1.67 \\ \midrule
    & \multicolumn{4}{c}{{\bf Test Env. Seed 14}}             & \multicolumn{4}{c}{{\bf Test Env. Seed 15}} \\
    \cmidrule(lr){2-5} \cmidrule(lr){6-9} 
Method     & Seed 1     & Seed 2     & Seed 3     & Rank & Seed 1     & Seed 2     & Seed 3     & Rank \\ \midrule
Random     & 84.4 (3.0) & 80.5 (7.0) & 86.6 (3.0) & 4.33 & 93.8 (7.0) & 95.4 (4.0) & 91.9 (7.0) & 6    \\
ETS        & 87.3 (2.0) & 87.3 (2.0) & 87.3 (2.0) & 2    & 95.3 (4.0) & 95.3 (5.0) & 95.3 (4.0) & 4.33 \\
Heur-GD & 81.2 (6.5) & 81.2 (5.5) & 81.2 (6.5) & 6.17 & 95.9 (2.5) & 95.9 (2.5) & 95.9 (2.5) & 2.5  \\
Heur-LD & 81.2 (6.5) & 81.2 (5.5) & 81.2 (6.5) & 6.17 & 96.1 (1.0) & 96.1 (1.0) & 96.1 (1.0) & 1    \\
Heur-AT & 82.4 (5.0) & 82.4 (4.0) & 82.4 (5.0) & 4.67 & 95.9 (2.5) & 95.9 (2.5) & 95.9 (2.5) & 2.5  \\
DQN        & 92.8 (1.0) & 86.8 (3.0) & 88.1 (1.0) & 1.67 & 94.8 (6.0) & 94.8 (7.0) & 94.3 (6.0) & 6.33 \\
A2C        & 83.4 (4.0) & 95.3 (1.0) & 83.4 (4.0) & 3    & 94.8 (5.0) & 94.8 (6.0) & 94.8 (5.0) & 5.33 \\ \midrule
    & \multicolumn{4}{c}{{\bf Test Env. Seed 16}}             & \multicolumn{4}{c}{{\bf Test Env. Seed 17}} \\ 
    \cmidrule(lr){2-5} \cmidrule(lr){6-9} 
Method     & Seed 1     & Seed 2     & Seed 3     & Rank & Seed 1     & Seed 2     & Seed 3     & Rank \\ \midrule
Random     & 80.5 (6.0) & 87.5 (6.0) & 83.0 (6.0) & 6    & 96.4 (1.0) & 96.1 (3.0) & 96.1 (1.0) & 1.67 \\
ETS        & 79.4 (7.0) & 79.4 (7.0) & 79.4 (7.0) & 7    & 95.7 (3.0) & 95.7 (4.0) & 95.7 (3.0) & 3.33 \\
Heur-GD & 91.2 (3.0) & 91.2 (4.0) & 91.2 (3.0) & 3.33 & 93.5 (6.0) & 93.5 (6.0) & 93.5 (6.0) & 6    \\
Heur-LD & 91.2 (3.0) & 91.2 (4.0) & 91.2 (3.0) & 3.33 & 93.5 (6.0) & 93.5 (6.0) & 93.5 (6.0) & 6    \\
Heur-AT & 91.2 (3.0) & 91.2 (4.0) & 91.2 (3.0) & 3.33 & 93.5 (6.0) & 93.5 (6.0) & 93.5 (6.0) & 6    \\
DQN        & 93.9 (1.0) & 92.9 (1.0) & 94.5 (1.0) & 1    & 95.7 (4.0) & 96.2 (2.0) & 95.7 (4.0) & 3.33 \\
A2C        & 90.3 (5.0) & 92.3 (2.0) & 90.3 (5.0) & 4    & 96.1 (2.0) & 96.5 (1.0) & 96.1 (2.0) & 1.67 \\ \midrule
    & \multicolumn{4}{c}{{\bf Test Env. Seed 18}}             & \multicolumn{4}{c}{{\bf Test Env. Seed 19}} \\
    \cmidrule(lr){2-5} \cmidrule(lr){6-9} 
Method     & Seed 1     & Seed 2     & Seed 3     & Rank & Seed 1     & Seed 2     & Seed 3     & Rank \\ \midrule
Random     & 92.7 (2.0) & 94.2 (1.0) & 93.2 (1.0) & 1.33 & 97.3 (2.0) & 95.9 (2.0) & 91.6 (2.0) & 2    \\
ETS        & 92.9 (1.0) & 92.9 (2.0) & 92.9 (2.0) & 1.67 & 97.4 (1.0) & 97.4 (1.0) & 97.4 (1.0) & 1    \\
Heur-GD & 87.8 (6.0) & 87.8 (6.0) & 87.8 (6.0) & 6    & 88.3 (5.0) & 88.3 (6.0) & 88.3 (4.0) & 5    \\
Heur-LD & 87.8 (6.0) & 87.8 (6.0) & 87.8 (6.0) & 6    & 88.3 (5.0) & 88.3 (6.0) & 88.3 (4.0) & 5    \\
Heur-AT & 87.8 (6.0) & 87.8 (6.0) & 87.8 (6.0) & 6    & 88.3 (5.0) & 88.3 (6.0) & 88.3 (4.0) & 5    \\
DQN        & 89.5 (4.0) & 89.4 (4.0) & 90.5 (4.0) & 4    & 89.4 (3.0) & 90.5 (3.0) & 87.5 (6.0) & 4    \\
A2C        & 91.6 (3.0) & 91.6 (3.0) & 91.6 (3.0) & 3    & 86.1 (7.0) & 89.9 (4.0) & 86.1 (7.0) & 6   \\ 
\bottomrule
\end{tabular}
}
\end{table}
\clearpage

\begin{table}[t]
\caption{The ACC and rank for each seed (10-19) in every test environment with {\bf Split FashionMNIST} for New Task Order experiment. Under each column named 'Seed X', we show the ACC (\%) and, in parenthesis, the rank for the corresponding method for the specific seed. We also show the average rank of each method under 'Rank' in the corresponding test environment. Note that the ACC for ETS, Heur-GD, Heur-LD, and Heur-AT are copied across the seeds, since the seed only affects the policy initialization in DQN and A2C and the random selection in Random.}
\vspace{-2mm}
\label{tab:rankings_fashionmnist_new_task_order}
\resizebox{\textwidth}{!}{
\begin{tabular}{lcccccccc}
\toprule
    & \multicolumn{4}{c}{\textbf{Test Env. Seed 10}}    & \multicolumn{4}{c}{\textbf{Test Env. Seed 11}}    \\
    \cmidrule(lr){2-5} \cmidrule(lr){6-9} 
Method     & Seed 1     & Seed 2     & Seed 3     & Rank & Seed 1     & Seed 2     & Seed 3     & Rank \\ \midrule
Random     & 98.4 (1.0) & 90.8 (7.0) & 98.0 (1.0) & 3    & 88.9 (5.0) & 93.4 (3.0) & 89.2 (5.0) & 4.33 \\
ETS        & 96.1 (6.0) & 96.1 (5.0) & 96.1 (5.0) & 5.33 & 88.8 (6.0) & 88.8 (6.0) & 88.8 (6.0) & 6    \\
Heur-GD & 98.0 (2.5) & 98.0 (1.5) & 98.0 (2.5) & 2.17 & 93.2 (3.0) & 93.2 (4.0) & 93.2 (4.0) & 3.67 \\
Heur-LD & 98.0 (2.5) & 98.0 (1.5) & 98.0 (2.5) & 2.17 & 94.7 (1.0) & 94.7 (1.0) & 94.7 (1.0) & 1    \\
Heur-AT & 92.0 (7.0) & 92.0 (6.0) & 92.0 (7.0) & 6.67 & 93.5 (2.0) & 93.5 (2.0) & 93.5 (2.0) & 2    \\
DQN        & 96.9 (4.5) & 97.0 (4.0) & 95.4 (6.0) & 4.83 & 91.3 (4.0) & 93.2 (5.0) & 93.3 (3.0) & 4    \\
A2C        & 96.9 (4.5) & 97.1 (3.0) & 96.9 (4.0) & 3.83 & 85.3 (7.0) & 86.9 (7.0) & 85.3 (7.0) & 7    \\ \midrule
\textbf{}  & \multicolumn{4}{c}{\textbf{Test Env. Seed 12}}    & \multicolumn{4}{c}{\textbf{Test Env. Seed 13}}    \\
    \cmidrule(lr){2-5} \cmidrule(lr){6-9} 
Method     & Seed 1     & Seed 2     & Seed 3     & Rank & Seed 1     & Seed 2     & Seed 3     & Rank \\ \midrule
Random     & 97.6 (1.0) & 96.8 (1.0) & 93.4 (4.0) & 2    & 94.0 (6.0) & 88.9 (7.0) & 93.4 (6.0) & 6.33 \\
ETS        & 91.2 (6.0) & 91.2 (6.0) & 91.2 (5.0) & 5.67 & 91.2 (7.0) & 91.2 (6.0) & 91.2 (7.0) & 6.67 \\
Heur-GD & 95.0 (5.0) & 95.0 (4.0) & 95.0 (3.0) & 4    & 95.1 (5.0) & 95.1 (5.0) & 95.1 (5.0) & 5    \\
Heur-LD & 96.1 (2.0) & 96.1 (2.0) & 96.1 (1.0) & 1.67 & 96.2 (4.0) & 96.2 (4.0) & 96.2 (4.0) & 4    \\
Heur-AT & 95.8 (3.0) & 95.8 (3.0) & 95.8 (2.0) & 2.67 & 98.1 (1.0) & 98.1 (2.0) & 98.1 (1.0) & 1.33 \\
DQN        & 95.0 (4.0) & 94.8 (5.0) & 88.6 (6.0) & 5    & 96.6 (3.0) & 98.4 (1.0) & 97.4 (2.0) & 2    \\
A2C        & 87.7 (7.0) & 87.7 (7.0) & 87.7 (7.0) & 7    & 98.0 (2.0) & 98.0 (3.0) & 96.7 (3.0) & 2.67 \\ \midrule
\textbf{}  & \multicolumn{4}{c}{\textbf{Test Env. Seed 14}}    & \multicolumn{4}{c}{\textbf{Test Env. Seed 15}}    \\
    \cmidrule(lr){2-5} \cmidrule(lr){6-9} 
Method     & Seed 1     & Seed 2     & Seed 3     & Rank & Seed 1     & Seed 2     & Seed 3     & Rank \\ \midrule
Random     & 94.2 (5.0) & 96.0 (2.0) & 93.0 (5.0) & 4    & 94.3 (1.0) & 92.6 (2.0) & 93.8 (1.0) & 1.33 \\
ETS        & 90.0 (6.0) & 90.0 (6.0) & 90.0 (6.0) & 6    & 93.5 (2.0) & 93.5 (1.0) & 93.5 (2.0) & 1.67 \\
Heur-GD & 95.4 (1.0) & 95.4 (3.0) & 95.4 (2.0) & 2    & 79.3 (7.0) & 79.3 (7.0) & 79.3 (7.0) & 7    \\
Heur-LD & 95.1 (3.0) & 95.1 (4.0) & 95.1 (3.0) & 3.33 & 92.6 (3.0) & 92.6 (3.0) & 92.6 (3.0) & 3    \\
Heur-AT & 82.0 (7.0) & 82.0 (7.0) & 82.0 (7.0) & 7    & 88.5 (5.0) & 88.5 (5.0) & 88.5 (4.0) & 4.67 \\
DQN        & 95.3 (2.0) & 96.1 (1.0) & 95.5 (1.0) & 1.33 & 89.6 (4.0) & 90.1 (4.0) & 88.4 (5.0) & 4.33 \\
A2C        & 94.7 (4.0) & 94.7 (5.0) & 94.7 (4.0) & 4.33 & 81.0 (6.0) & 81.0 (6.0) & 80.5 (6.0) & 6    \\ \midrule
\textbf{}  & \multicolumn{4}{c}{\textbf{Test Env. Seed 16}}    & \multicolumn{4}{c}{\textbf{Test Env. Seed 17}}    \\
    \cmidrule(lr){2-5} \cmidrule(lr){6-9} 
Method     & Seed 1     & Seed 2     & Seed 3     & Rank & Seed 1     & Seed 2     & Seed 3     & Rank \\ \midrule
Random     & 90.0 (2.0) & 88.1 (3.0) & 92.3 (3.0) & 2.67 & 99.1 (3.0) & 98.6 (5.0) & 99.2 (4.0) & 4    \\
ETS        & 94.4 (1.0) & 94.4 (1.0) & 94.4 (1.0) & 1    & 98.1 (7.0) & 98.1 (7.0) & 98.1 (7.0) & 7    \\
Heur-GD & 73.8 (7.0) & 73.8 (7.0) & 73.8 (7.0) & 7    & 99.4 (1.0) & 99.4 (1.0) & 99.4 (1.0) & 1    \\
Heur-LD & 80.4 (6.0) & 80.4 (6.0) & 80.4 (6.0) & 6    & 99.3 (2.0) & 99.3 (2.0) & 99.3 (2.0) & 2    \\
Heur-AT & 89.2 (4.0) & 89.2 (2.0) & 89.2 (4.0) & 3.33 & 98.7 (4.0) & 98.7 (4.0) & 98.7 (6.0) & 4.67 \\
DQN        & 87.3 (5.0) & 85.7 (4.0) & 92.3 (2.0) & 3.67 & 98.7 (5.0) & 99.1 (3.0) & 99.0 (5.0) & 4.33 \\
A2C        & 89.3 (3.0) & 83.2 (5.0) & 87.7 (5.0) & 4.33 & 98.7 (6.0) & 98.5 (6.0) & 99.2 (3.0) & 5    \\ \midrule
\textbf{}  & \multicolumn{4}{c}{\textbf{Test Env. Seed 18}}    & \multicolumn{4}{c}{\textbf{Test Env. Seed 19}}    \\
    \cmidrule(lr){2-5} \cmidrule(lr){6-9} 
Method     & Seed 1     & Seed 2     & Seed 3     & Rank & Seed 1     & Seed 2     & Seed 3     & Rank \\ \midrule
Random     & 87.2 (7.0) & 87.2 (7.0) & 94.2 (4.0) & 6    & 96.2 (5.0) & 97.8 (1.0) & 97.7 (1.0) & 2.33 \\
ETS        & 93.6 (4.0) & 93.6 (4.0) & 93.6 (5.0) & 4.33 & 97.5 (1.0) & 97.5 (3.0) & 97.5 (2.0) & 2    \\
Heur-GD & 92.9 (5.0) & 92.9 (5.0) & 92.9 (6.0) & 5.33 & 95.8 (6.0) & 95.8 (5.0) & 95.8 (5.0) & 5.33 \\
Heur-LD & 92.1 (6.0) & 92.1 (6.0) & 92.1 (7.0) & 6.33 & 93.4 (7.0) & 93.4 (7.0) & 93.4 (7.0) & 7    \\
Heur-AT & 94.2 (2.0) & 94.2 (2.0) & 94.2 (3.0) & 2.33 & 96.6 (4.0) & 96.6 (4.0) & 96.6 (4.0) & 4    \\
DQN        & 95.3 (1.0) & 94.7 (1.0) & 96.0 (1.0) & 1    & 96.9 (3.0) & 95.7 (6.0) & 95.7 (6.0) & 5    \\
A2C        & 93.8 (3.0) & 93.8 (3.0) & 95.0 (2.0) & 2.67 & 97.0 (2.0) & 97.7 (2.0) & 96.9 (3.0) & 2.33 \\
\bottomrule
\end{tabular}
}
\end{table}
\clearpage

\begin{table}[t]
\caption{The ACC and rank for each seed (10-19) in every test environment with {\bf Split CIFAR-10} for New Task Order experiment. Under each column named 'Seed X', we show the ACC (\%) and, in parenthesis, the rank for the corresponding method for the specific seed. We also show the average rank of each method in the test environment under 'Rank'. Note that the ACC for ETS, Heur-GD, Heur-LD, and Heur-AT are copied across the seeds, since the seed only affects the policy initialization in DQN and A2C and the random selection in Random.}
\label{tab:rankings_cifar10_new_task_order}
\resizebox{\textwidth}{!}{
\begin{tabular}{lcccccccc}
\toprule
\textbf{}  & \multicolumn{4}{c}{\textbf{Test Env. Seed 10}} & \multicolumn{4}{c}{\textbf{Test Env. Seed 11}} \\
\cmidrule(lr){2-5} \cmidrule(lr){6-9} 
Method     & Seed 1      & Seed 2      & Seed 3      & Rank & Seed 1      & Seed 2      & Seed 3      & Rank \\ \midrule
Random     & 87.3 (2.0)  & 85.2 (6.0)  & 85.7 (6.0)  & 4.67 & 75.3 (6.0)  & 71.1 (7.0)  & 72.8 (7.0)  & 6.67 \\
ETS        & 87.1 (3.0)  & 87.1 (1.0)  & 87.1 (1.0)  & 1.67 & 75.2 (7.0)  & 75.2 (6.0)  & 75.2 (6.0)  & 6.33 \\
Heur-GD & 86.3 (5.0)  & 86.3 (3.0)  & 86.3 (4.0)  & 4    & 79.5 (4.0)  & 79.5 (5.0)  & 79.5 (4.0)  & 4.33 \\
Heur-LD & 86.7 (4.0)  & 86.7 (2.0)  & 86.7 (3.0)  & 3    & 81.6 (2.0)  & 81.6 (2.0)  & 81.6 (2.0)  & 2    \\
Heur-AT & 85.9 (6.0)  & 85.9 (4.0)  & 85.9 (5.0)  & 5    & 81.1 (3.0)  & 81.1 (3.0)  & 81.1 (3.0)  & 3    \\
DQN        & 88.9 (1.0)  & 85.5 (5.0)  & 86.9 (2.0)  & 2.67 & 78.2 (5.0)  & 79.9 (4.0)  & 79.3 (5.0)  & 4.67 \\
A2C        & 84.6 (7.0)  & 84.6 (7.0)  & 84.6 (7.0)  & 7    & 82.9 (1.0)  & 82.9 (1.0)  & 82.9 (1.0)  & 1    \\ \midrule
\textbf{}  & \multicolumn{4}{c}{\textbf{Test Env. Seed 12}} & \multicolumn{4}{c}{\textbf{Test Env. Seed 13}} \\
\cmidrule(lr){2-5} \cmidrule(lr){6-9} 
Method     & Seed 1      & Seed 2      & Seed 3      & Rank & Seed 1      & Seed 2      & Seed 3      & Rank \\ \midrule
Random     & 86.2 (4.0)  & 88.9 (3.0)  & 87.6 (3.0)  & 3.33 & 77.3 (6.0)  & 81.3 (2.0)  & 80.9 (2.0)  & 3.33 \\
ETS        & 85.5 (7.0)  & 85.5 (6.0)  & 85.5 (7.0)  & 6.67 & 76.9 (7.0)  & 76.9 (7.0)  & 76.9 (7.0)  & 7    \\
Heur-GD & 89.8 (1.0)  & 89.8 (1.0)  & 89.8 (1.0)  & 1    & 78.9 (5.0)  & 78.9 (6.0)  & 78.9 (6.0)  & 5.67 \\
Heur-LD & 87.5 (3.0)  & 87.5 (4.0)  & 87.5 (5.0)  & 4    & 80.8 (3.0)  & 80.8 (4.0)  & 80.8 (3.0)  & 3.33 \\
Heur-AT & 89.6 (2.0)  & 89.6 (2.0)  & 89.6 (2.0)  & 2    & 80.2 (4.0)  & 80.2 (5.0)  & 80.2 (5.0)  & 4.67 \\
DQN        & 86.0 (6.0)  & 84.4 (7.0)  & 85.7 (6.0)  & 6.33 & 80.8 (2.0)  & 81.3 (1.0)  & 80.2 (4.0)  & 2.33 \\
A2C        & 86.1 (5.0)  & 86.1 (5.0)  & 87.5 (4.0)  & 4.67 & 81.3 (1.0)  & 81.0 (3.0)  & 81.3 (1.0)  & 1.67 \\ \midrule
\textbf{}  & \multicolumn{4}{c}{\textbf{Test Env. Seed 14}} & \multicolumn{4}{c}{\textbf{Test Env. Seed 15}} \\
\cmidrule(lr){2-5} \cmidrule(lr){6-9} 
Method     & Seed 1      & Seed 2      & Seed 3      & Rank & Seed 1      & Seed 2      & Seed 3      & Rank \\ \midrule
Random     & 81.7 (6.0)  & 84.8 (5.0)  & 83.0 (6.0)  & 5.67 & 83.8 (4.0)  & 85.0 (2.0)  & 85.8 (1.0)  & 2.33 \\
ETS        & 77.8 (7.0)  & 77.8 (7.0)  & 77.8 (7.0)  & 7    & 83.8 (5.0)  & 83.8 (6.0)  & 83.8 (5.0)  & 5.33 \\
Heur-GD & 86.9 (1.0)  & 86.9 (1.0)  & 86.9 (1.0)  & 1    & 83.5 (6.0)  & 83.5 (7.0)  & 83.5 (6.0)  & 6.33 \\
Heur-LD & 84.0 (5.0)  & 84.0 (6.0)  & 84.0 (4.0)  & 5    & 84.1 (2.5)  & 84.1 (4.5)  & 84.1 (3.5)  & 3.5  \\
Heur-AT & 86.6 (2.0)  & 86.6 (2.0)  & 86.6 (2.0)  & 2    & 84.1 (2.5)  & 84.1 (4.5)  & 84.1 (3.5)  & 3.5  \\
DQN        & 85.0 (4.0)  & 85.2 (4.0)  & 83.8 (5.0)  & 4.33 & 83.2 (7.0)  & 85.4 (1.0)  & 81.7 (7.0)  & 5    \\
A2C        & 86.2 (3.0)  & 86.2 (3.0)  & 85.5 (3.0)  & 3    & 84.6 (1.0)  & 84.6 (3.0)  & 84.6 (2.0)  & 2    \\ \midrule
\textbf{}  & \multicolumn{4}{c}{\textbf{Test Env. Seed 16}} & \multicolumn{4}{c}{\textbf{Test Env. Seed 17}} \\
\cmidrule(lr){2-5} \cmidrule(lr){6-9} 
Method     & Seed 1      & Seed 2      & Seed 3      & Rank & Seed 1      & Seed 2      & Seed 3      & Rank \\ \midrule
Random     & 85.2 (7.0)  & 89.4 (1.0)  & 87.6 (4.0)  & 4    & 68.9 (7.0)  & 72.1 (6.0)  & 74.0 (6.0)  & 6.33 \\
ETS        & 87.8 (3.0)  & 87.8 (3.0)  & 87.8 (2.0)  & 2.67 & 70.3 (5.0)  & 70.3 (7.0)  & 70.3 (7.0)  & 6.33 \\
Heur-GD & 86.8 (4.5)  & 86.8 (5.5)  & 86.8 (5.5)  & 5.17 & 74.1 (4.0)  & 74.1 (5.0)  & 74.1 (5.0)  & 4.67 \\
Heur-LD & 86.5 (6.0)  & 86.5 (7.0)  & 86.5 (7.0)  & 6.67 & 76.1 (2.0)  & 76.1 (3.0)  & 76.1 (3.0)  & 2.67 \\
Heur-AT & 86.8 (4.5)  & 86.8 (5.5)  & 86.8 (5.5)  & 5.17 & 76.4 (1.0)  & 76.4 (1.0)  & 76.4 (2.0)  & 1.33 \\
DQN        & 87.9 (2.0)  & 87.4 (4.0)  & 87.6 (3.0)  & 3    & 69.9 (6.0)  & 76.4 (2.0)  & 79.2 (1.0)  & 3    \\
A2C        & 88.0 (1.0)  & 88.0 (2.0)  & 88.0 (1.0)  & 1.33 & 75.7 (3.0)  & 75.7 (4.0)  & 75.7 (4.0)  & 3.67 \\ \midrule
\textbf{}  & \multicolumn{4}{c}{\textbf{Test Env. Seed 18}} & \multicolumn{4}{c}{\textbf{Test Env. Seed 19}} \\
\cmidrule(lr){2-5} \cmidrule(lr){6-9} 
Method     & Seed 1      & Seed 2      & Seed 3      & Rank & Seed 1      & Seed 2      & Seed 3      & Rank \\ \midrule
Random     & 86.9 (7.0)  & 87.6 (7.0)  & 86.4 (7.0)  & 7    & 71.2 (7.0)  & 77.5 (7.0)  & 73.2 (7.0)  & 7    \\
ETS        & 88.3 (6.0)  & 88.3 (6.0)  & 88.3 (5.0)  & 5.67 & 77.7 (5.0)  & 77.7 (5.0)  & 77.7 (5.0)  & 5    \\
Heur-GD & 88.6 (4.5)  & 88.6 (4.5)  & 88.6 (3.5)  & 4.17 & 77.9 (3.5)  & 77.9 (3.5)  & 77.9 (3.5)  & 3.5  \\
Heur-LD & 89.8 (1.0)  & 89.8 (2.0)  & 89.8 (1.0)  & 1.33 & 77.5 (6.0)  & 77.5 (6.0)  & 77.5 (6.0)  & 6    \\
Heur-AT & 88.6 (4.5)  & 88.6 (4.5)  & 88.6 (3.5)  & 4.17 & 77.9 (3.5)  & 77.9 (3.5)  & 77.9 (3.5)  & 3.5  \\
DQN        & 89.1 (2.0)  & 88.6 (3.0)  & 86.8 (6.0)  & 3.67 & 80.5 (2.0)  & 80.5 (1.5)  & 78.5 (2.0)  & 1.83 \\
A2C        & 88.7 (3.0)  & 89.9 (1.0)  & 89.4 (2.0)  & 2    & 80.5 (1.0)  & 80.5 (1.5)  & 80.5 (1.0)  & 1.17 \\
\bottomrule
\end{tabular}
}
\end{table}

\clearpage


\begin{table}[t]
\caption{The ACC and rank for each seed (0-9) in every test environment with {\bf Split notMNIST} for New Dataset experiment. Under each column named 'Seed X', we show the ACC (\%) and, in parenthesis, the rank for the corresponding method for the specific seed. We also show the average rank of each method in the test environment under 'Rank'. Note that the ACC for ETS, Heur-GD, Heur-LD, and Heur-AT are copied across the seeds, since the seed only affects the policy initialization in DQN and A2C and the random selection in Random.}
\label{tab:rankings_notmnist_new_dataset}
\resizebox{\textwidth}{!}{
\begin{tabular}{lcccccccc}
\toprule
\textbf{} & \multicolumn{4}{c}{\textbf{Test Env. Seed 0}} & \multicolumn{4}{c}{\textbf{Test Env. Seed 1}} \\
\cmidrule(lr){2-5} \cmidrule(lr){6-9} 
Method    & Seed 1      & Seed 2      & Seed 3     & Rank & Seed 1      & Seed 2      & Seed 3     & Rank \\ \midrule
Random    & 93.8 (1.0)  & 94.9 (1.0)  & 93.2 (2.0) & 1.33 & 93.5 (2.0)  & 92.0 (4.0)  & 91.0 (7.0) & 4.33 \\
ETS       & 91.1 (5.0)  & 91.1 (7.0)  & 91.1 (6.0) & 6    & 92.5 (4.0)  & 92.5 (3.0)  & 92.5 (3.0) & 3.33 \\
Heur-GD   & 92.4 (3.0)  & 92.4 (5.0)  & 92.4 (4.0) & 4    & 91.2 (6.0)  & 91.2 (6.0)  & 91.2 (5.0) & 5.67 \\
Heur-LD   & 92.4 (3.0)  & 92.4 (5.0)  & 92.4 (4.0) & 4    & 91.2 (6.0)  & 91.2 (6.0)  & 91.2 (5.0) & 5.67 \\
Heur-AT   & 92.4 (3.0)  & 92.4 (5.0)  & 92.4 (4.0) & 4    & 91.2 (6.0)  & 91.2 (6.0)  & 91.2 (5.0) & 5.67 \\
DQN       & 90.7 (6.0)  & 94.5 (2.0)  & 85.3 (7.0) & 5    & 94.2 (1.0)  & 94.6 (1.0)  & 94.5 (1.0) & 1    \\
A2C       & 90.6 (7.0)  & 92.7 (3.0)  & 93.3 (1.0) & 3.67 & 93.5 (3.0)  & 93.9 (2.0)  & 93.9 (2.0) & 2.33 \\ \midrule
\textbf{} & \multicolumn{4}{c}{\textbf{Test Env. Seed 2}} & \multicolumn{4}{c}{\textbf{Test Env. Seed 3}} \\
\cmidrule(lr){2-5} \cmidrule(lr){6-9} 
Method    & Seed 1      & Seed 2      & Seed 3     & Rank & Seed 1      & Seed 2      & Seed 3     & Rank \\ \midrule
A2C       & 91.7 (2.0)  & 89.7 (4.0)  & 91.7 (2.0) & 2.67 & 95.0 (4.0)  & 95.0 (4.0)  & 95.0 (4.0) & 4    \\
DQN       & 93.9 (1.0)  & 94.2 (1.0)  & 92.0 (1.0) & 1    & 93.3 (6.0)  & 92.8 (5.0)  & 92.3 (5.0) & 5.33 \\
Random    & 82.1 (7.0)  & 90.5 (3.0)  & 88.5 (4.0) & 4.67 & 94.2 (5.0)  & 89.5 (6.0)  & 85.2 (7.0) & 6    \\
ETS       & 91.2 (3.0)  & 91.2 (2.0)  & 91.2 (3.0) & 2.67 & 89.3 (7.0)  & 89.3 (7.0)  & 89.3 (6.0) & 6.67 \\
Heur-GD   & 82.6 (5.0)  & 82.6 (6.0)  & 82.6 (6.0) & 5.67 & 95.3 (2.0)  & 95.3 (2.0)  & 95.3 (2.0) & 2    \\
Heur-LD   & 82.6 (5.0)  & 82.6 (6.0)  & 82.6 (6.0) & 5.67 & 95.3 (2.0)  & 95.3 (2.0)  & 95.3 (2.0) & 2    \\
Heur-AT   & 82.6 (5.0)  & 82.6 (6.0)  & 82.6 (6.0) & 5.67 & 95.3 (2.0)  & 95.3 (2.0)  & 95.3 (2.0) & 2    \\
DQN       & 93.9 (1.0)  & 94.2 (1.0)  & 92.0 (1.0) & 1    & 93.3 (6.0)  & 92.8 (5.0)  & 92.3 (5.0) & 5.33 \\
A2C       & 91.7 (2.0)  & 89.7 (4.0)  & 91.7 (2.0) & 2.67 & 95.0 (4.0)  & 95.0 (4.0)  & 95.0 (4.0) & 4    \\ \midrule
\textbf{} & \multicolumn{4}{c}{\textbf{Test Env. Seed 4}} & \multicolumn{4}{c}{\textbf{Test Env. Seed 5}} \\
\cmidrule(lr){2-5} \cmidrule(lr){6-9} 
Method    & Seed 1      & Seed 2      & Seed 3     & Rank & Seed 1      & Seed 2      & Seed 3     & Rank \\ \midrule
Random    & 90.2 (7.0)  & 93.2 (1.0)  & 90.1 (7.0) & 5    & 90.7 (7.0)  & 93.7 (1.0)  & 91.4 (6.0) & 4.67 \\
ETS       & 90.5 (6.0)  & 90.5 (7.0)  & 90.5 (6.0) & 6.33 & 90.7 (6.0)  & 90.7 (7.0)  & 90.7 (7.0) & 6.67 \\
Heur-GD   & 91.6 (3.5)  & 91.6 (4.5)  & 91.6 (4.5) & 4.17 & 91.8 (4.0)  & 91.8 (5.0)  & 91.8 (4.0) & 4.33 \\
Heur-LD   & 92.7 (2.0)  & 92.7 (3.0)  & 92.7 (2.0) & 2.33 & 91.8 (4.0)  & 91.8 (5.0)  & 91.8 (4.0) & 4.33 \\
Heur-AT   & 91.6 (3.5)  & 91.6 (4.5)  & 91.6 (4.5) & 4.17 & 91.8 (4.0)  & 91.8 (5.0)  & 91.8 (4.0) & 4.33 \\
DQN       & 91.3 (5.0)  & 91.2 (6.0)  & 92.0 (3.0) & 4.67 & 93.2 (1.0)  & 92.1 (3.0)  & 93.0 (1.0) & 1.67 \\
A2C       & 93.1 (1.0)  & 93.1 (2.0)  & 93.0 (1.0) & 1.33 & 92.6 (2.0)  & 93.1 (2.0)  & 92.8 (2.0) & 2    \\ \midrule
\textbf{} & \multicolumn{4}{c}{\textbf{Test Env. Seed 6}} & \multicolumn{4}{c}{\textbf{Test Env. Seed 7}} \\
\cmidrule(lr){2-5} \cmidrule(lr){6-9} 
Method    & Seed 1      & Seed 2      & Seed 3     & Rank & Seed 1      & Seed 2      & Seed 3     & Rank \\ \midrule
Random    & 91.6 (4.0)  & 87.2 (6.0)  & 91.9 (3.0) & 4.33 & 95.0 (1.0)  & 94.3 (1.0)  & 93.3 (3.0) & 1.67 \\
ETS       & 92.0 (3.0)  & 92.0 (2.0)  & 92.0 (2.0) & 2.33 & 94.0 (2.0)  & 94.0 (2.0)  & 94.0 (1.0) & 1.67 \\
Heur-GD   & 91.6 (5.5)  & 91.6 (3.5)  & 91.6 (4.5) & 4.5  & 88.8 (7.0)  & 88.8 (7.0)  & 88.8 (7.0) & 7    \\
Heur-LD   & 87.0 (7.0)  & 87.0 (7.0)  & 87.0 (7.0) & 7    & 90.6 (6.0)  & 90.6 (6.0)  & 90.6 (6.0) & 6    \\
Heur-AT   & 91.6 (5.5)  & 91.6 (3.5)  & 91.6 (4.5) & 4.5  & 93.1 (4.0)  & 93.1 (4.0)  & 93.1 (4.0) & 4    \\
DQN       & 93.1 (2.0)  & 95.2 (1.0)  & 93.8 (1.0) & 1.33 & 94.0 (3.0)  & 92.6 (5.0)  & 93.7 (2.0) & 3.33 \\
A2C       & 93.3 (1.0)  & 90.0 (5.0)  & 90.0 (6.0) & 4    & 92.6 (5.0)  & 93.7 (3.0)  & 92.6 (5.0) & 4.33 \\ \midrule
\textbf{} & \multicolumn{4}{c}{\textbf{Test Env. Seed 8}} & \multicolumn{4}{c}{\textbf{Test Env. Seed 9}} \\
\cmidrule(lr){2-5} \cmidrule(lr){6-9} 
Method    & Seed 1      & Seed 2      & Seed 3     & Rank & Seed 1      & Seed 2      & Seed 3     & Rank \\ \midrule
Random    & 92.3 (5.0)  & 92.6 (4.0)  & 92.4 (4.0) & 4.33 & 90.7 (3.0)  & 91.2 (2.0)  & 93.4 (2.0) & 2.33 \\
ETS       & 89.0 (6.0)  & 89.0 (6.0)  & 89.0 (6.0) & 6    & 94.3 (1.0)  & 94.3 (1.0)  & 94.3 (1.0) & 1    \\
Heur-GD   & 93.6 (3.5)  & 93.6 (2.5)  & 93.6 (2.5) & 2.83 & 79.1 (6.0)  & 79.1 (5.0)  & 79.1 (6.0) & 5.67 \\
Heur-LD   & 88.2 (7.0)  & 88.2 (7.0)  & 88.2 (7.0) & 7    & 79.1 (6.0)  & 79.1 (5.0)  & 79.1 (6.0) & 5.67 \\
Heur-AT   & 93.6 (3.5)  & 93.6 (2.5)  & 93.6 (2.5) & 2.83 & 79.1 (6.0)  & 79.1 (5.0)  & 79.1 (6.0) & 5.67 \\
DQN       & 94.6 (1.0)  & 91.6 (5.0)  & 92.1 (5.0) & 3.67 & 82.7 (4.0)  & 79.1 (7.0)  & 82.9 (4.0) & 5    \\
A2C       & 93.9 (2.0)  & 93.9 (1.0)  & 93.9 (1.0) & 1.33 & 90.8 (2.0)  & 90.8 (3.0)  & 90.8 (3.0) & 2.67 \\
\bottomrule
\end{tabular}
}
\end{table}
\clearpage


\begin{table}[t]
\caption{The ACC and rank for each seed (0-9) in every test environment with {\bf Split FashionMNIST} for New Dataset experiment. Under each column named 'Seed X', we show the ACC (\%) and, in parenthesis, the rank for the corresponding method for the specific seed. We also show the average rank of each method in the test environment under 'Rank'. Note that the ACC for ETS, Heur-GD, Heur-LD, and Heur-AT are copied across the seeds, since the seed only affects the policy initialization in DQN and A2C and the random selection in Random.}
\label{tab:rankings_fashionmnist_new_dataset}
\resizebox{\textwidth}{!}{
\begin{tabular}{lcccccccc}
\toprule
\textbf{}  & \multicolumn{4}{c}{\textbf{Test Env. Seed 0}} & \multicolumn{4}{c}{\textbf{Test Env. Seed 1}} \\ 
    \cmidrule(lr){2-5} \cmidrule(lr){6-9} 
Method     & Seed 1      & Seed 2      & Seed 3     & Rank & Seed 1      & Seed 2      & Seed 3     & Rank \\ \midrule
Random     & 94.7 (7.0)  & 97.7 (2.0)  & 91.5 (7.0) & 5.33 & 92.8 (5.0)  & 95.1 (1.0)  & 93.7 (6.0) & 4    \\
ETS        & 97.1 (6.0)  & 97.1 (6.0)  & 97.1 (6.0) & 6    & 94.1 (2.0)  & 94.1 (3.0)  & 94.1 (3.0) & 2.67 \\
Heur-GD & 97.9 (1.0)  & 97.9 (1.0)  & 97.9 (1.0) & 1    & 95.0 (1.0)  & 95.0 (2.0)  & 95.0 (1.0) & 1.33 \\
Heur-LD & 97.6 (3.0)  & 97.6 (3.0)  & 97.6 (3.0) & 3    & 90.0 (7.0)  & 90.0 (7.0)  & 90.0 (7.0) & 7    \\
Heur-AT & 97.4 (4.0)  & 97.4 (5.0)  & 97.4 (5.0) & 4.67 & 94.1 (3.0)  & 94.1 (4.0)  & 94.1 (4.0) & 3.67 \\
DQN        & 97.7 (2.0)  & 97.4 (4.0)  & 97.6 (4.0) & 3.33 & 94.0 (4.0)  & 90.1 (6.0)  & 94.6 (2.0) & 4    \\
A2C        & 97.3 (5.0)  & 89.1 (7.0)  & 97.7 (2.0) & 4.67 & 90.9 (6.0)  & 90.9 (5.0)  & 94.0 (5.0) & 5.33 \\ \midrule
\textbf{}  & \multicolumn{4}{c}{\textbf{Test Env. Seed 2}} & \multicolumn{4}{c}{\textbf{Test Env. Seed 3}} \\
    \cmidrule(lr){2-5} \cmidrule(lr){6-9} 
Method     & Seed 1      & Seed 2      & Seed 3     & Rank & Seed 1      & Seed 2      & Seed 3     & Rank \\ \midrule
Random     & 94.6 (6.0)  & 95.1 (6.0)  & 94.8 (5.0) & 5.67 & 89.5 (6.0)  & 99.4 (1.5)  & 93.9 (4.0) & 3.83 \\
ETS        & 86.7 (7.0)  & 86.7 (7.0)  & 86.7 (7.0) & 7    & 89.4 (7.0)  & 89.4 (6.0)  & 89.4 (6.0) & 6.33 \\
Heur-GD & 97.4 (2.0)  & 97.4 (3.0)  & 97.4 (3.0) & 2.67 & 96.7 (3.0)  & 96.7 (3.0)  & 96.7 (2.0) & 2.67 \\
Heur-LD & 97.3 (3.0)  & 97.3 (4.0)  & 97.3 (4.0) & 3.67 & 90.6 (5.0)  & 90.6 (5.0)  & 90.6 (5.0) & 5    \\
Heur-AT & 97.7 (1.0)  & 97.7 (1.0)  & 97.7 (1.0) & 1    & 99.4 (1.0)  & 99.4 (1.5)  & 99.4 (1.0) & 1.17 \\
DQN        & 96.6 (4.0)  & 95.9 (5.0)  & 94.3 (6.0) & 5    & 97.3 (2.0)  & 96.6 (4.0)  & 96.0 (3.0) & 3    \\
A2C        & 96.2 (5.0)  & 97.6 (2.0)  & 97.6 (2.0) & 3    & 93.8 (4.0)  & 88.6 (7.0)  & 89.2 (7.0) & 6    \\ \midrule
\textbf{}  & \multicolumn{4}{c}{\textbf{Test Env. Seed 4}} & \multicolumn{4}{c}{\textbf{Test Env. Seed 5}} \\
    \cmidrule(lr){2-5} \cmidrule(lr){6-9} 
Method     & Seed 1      & Seed 2      & Seed 3     & Rank & Seed 1      & Seed 2      & Seed 3     & Rank \\ \midrule
Random     & 88.6 (4.0)  & 84.6 (6.0)  & 80.3 (7.0) & 5.67 & 88.7 (5.0)  & 93.9 (1.0)  & 88.0 (5.0) & 3.67 \\
ETS        & 87.1 (5.0)  & 87.1 (4.0)  & 87.1 (4.0) & 4.33 & 91.5 (1.0)  & 91.5 (2.0)  & 91.5 (1.0) & 1.33 \\
Heur-GD & 91.3 (3.0)  & 91.3 (3.0)  & 91.3 (2.0) & 2.67 & 90.3 (3.0)  & 90.3 (4.0)  & 90.3 (3.0) & 3.33 \\
Heur-LD & 86.8 (6.0)  & 86.8 (5.0)  & 86.8 (5.0) & 5.33 & 88.0 (6.0)  & 88.0 (5.0)  & 88.0 (4.0) & 5    \\
Heur-AT & 83.3 (7.0)  & 83.3 (7.0)  & 83.3 (6.0) & 6.67 & 90.5 (2.0)  & 90.5 (3.0)  & 90.5 (2.0) & 2.33 \\
DQN        & 91.3 (2.0)  & 92.2 (1.0)  & 90.5 (3.0) & 2    & 89.2 (4.0)  & 87.1 (7.0)  & 86.1 (7.0) & 6    \\
A2C        & 91.8 (1.0)  & 91.8 (2.0)  & 91.8 (1.0) & 1.33 & 87.9 (7.0)  & 87.9 (6.0)  & 87.9 (6.0) & 6.33 \\ \midrule
\textbf{}  & \multicolumn{4}{c}{\textbf{Test Env. Seed 6}} & \multicolumn{4}{c}{\textbf{Test Env. Seed 7}} \\
    \cmidrule(lr){2-5} \cmidrule(lr){6-9} 
Method     & Seed 1      & Seed 2      & Seed 3     & Rank & Seed 1      & Seed 2      & Seed 3     & Rank \\ \midrule
Random     & 95.3 (2.0)  & 94.0 (2.0)  & 96.3 (1.0) & 1.67 & 93.3 (4.0)  & 94.0 (4.0)  & 94.4 (5.0) & 4.33 \\
ETS        & 95.5 (1.0)  & 95.5 (1.0)  & 95.5 (2.0) & 1.33 & 95.3 (2.0)  & 95.3 (2.0)  & 95.3 (3.0) & 2.33 \\
Heur-GD & 90.0 (3.0)  & 90.0 (4.0)  & 90.0 (4.0) & 3.67 & 91.8 (5.0)  & 91.8 (5.0)  & 91.8 (6.0) & 5.33 \\
Heur-LD & 85.4 (4.0)  & 85.4 (5.0)  & 85.4 (5.0) & 4.67 & 95.1 (3.0)  & 95.1 (3.0)  & 95.1 (4.0) & 3.33 \\
Heur-AT & 76.0 (7.0)  & 76.0 (7.0)  & 76.0 (7.0) & 7    & 96.8 (1.0)  & 96.8 (1.0)  & 96.8 (1.0) & 1    \\
DQN        & 80.1 (6.0)  & 84.3 (6.0)  & 81.6 (6.0) & 6    & 87.4 (7.0)  & 88.8 (7.0)  & 87.4 (7.0) & 7    \\
A2C        & 81.9 (5.0)  & 91.7 (3.0)  & 91.5 (3.0) & 3.67 & 91.7 (6.0)  & 91.3 (6.0)  & 96.1 (2.0) & 4.67 \\ \midrule
\textbf{}  & \multicolumn{4}{c}{\textbf{Test Env. Seed 8}} & \multicolumn{4}{c}{\textbf{Test Env. Seed 9}} \\
    \cmidrule(lr){2-5} \cmidrule(lr){6-9} 
Method     & Seed 1      & Seed 2      & Seed 3     & Rank & Seed 1      & Seed 2      & Seed 3     & Rank \\ \midrule
Random     & 88.6 (6.0)  & 87.9 (6.0)  & 95.8 (1.0) & 4.33 & 97.2 (1.0)  & 97.0 (1.0)  & 97.7 (1.0) & 1    \\
ETS        & 94.8 (2.0)  & 94.8 (3.0)  & 94.8 (3.0) & 2.67 & 96.7 (3.0)  & 96.7 (2.0)  & 96.7 (2.0) & 2.33 \\
Heur-GD & 95.0 (1.0)  & 95.0 (1.0)  & 95.0 (2.0) & 1.33 & 95.3 (4.0)  & 95.3 (3.0)  & 95.3 (4.0) & 3.67 \\
Heur-LD & 87.9 (7.0)  & 87.9 (7.0)  & 87.9 (7.0) & 7    & 90.0 (7.0)  & 90.0 (7.0)  & 90.0 (7.0) & 7    \\
Heur-AT & 93.9 (3.0)  & 93.9 (4.0)  & 93.9 (4.0) & 3.67 & 90.3 (6.0)  & 90.3 (6.0)  & 90.3 (6.0) & 6    \\
DQN        & 91.8 (5.0)  & 95.0 (2.0)  & 91.0 (6.0) & 4.33 & 96.9 (2.0)  & 93.7 (4.0)  & 95.9 (3.0) & 3    \\
A2C        & 93.6 (4.0)  & 93.6 (5.0)  & 93.6 (5.0) & 4.67 & 90.4 (5.0)  & 92.6 (5.0)  & 90.4 (5.0) & 5 \\
\bottomrule
\end{tabular}
}
\end{table}
\clearpage


