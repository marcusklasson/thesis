
\section{Additional Experimental Results}\label{paperD:app:additional_experimental_results}

%\subsection{Complementary Experimental Results for Policy Generalization} 

Here, we provide complementary results for the policy generalization experiments in Section \ref{paperD:sec:results_on_policy_generalization}. Figure \ref{fig:policy_rewards_mnist_paperD}-\ref{fig:policy_rewards_fashionmnist_new_dataset_paperD} shows the performance progress measured in ACC in the test environments during training for the DQN and A2C. We also show the ACC achieved by the scheduling baselines as straight lines as these policies are fixed. Figure \ref{fig:policy_rewards_mnist_paperD} and \ref{fig:policy_rewards_cifar10_paperD} shows the performance progress for test environments with Split MNIST and Split CIFAR-10 respectively in the New Task Orders experiment. Figure \ref{fig:policy_rewards_notmnist_new_dataset_paperD} and \ref{fig:policy_rewards_fashionmnist_new_dataset_paperD} shows the performance progress for test environments with Split notMNIST and Split FashionMNIST respectively in the New Dataset experiment. In general, we observe that DQN exhibits noisier progress of the achieved ACC in the test environments than A2C. 

In Table \ref{tab:rankings_mnist_new_task_order}-\ref{tab:rankings_fashionmnist_new_dataset}, we display the ACC and rank in every test environment that was used for generating the average rankings in Table \ref{tab:average_ranking_rl_experiment} (see Section \ref{paperD:sec:results_on_policy_generalization}). Table \ref{tab:rankings_mnist_new_task_order}, \ref{tab:rankings_fashionmnist_new_task_order}, and \ref{tab:rankings_cifar10_new_task_order} shows the ACCs and ranks for the New Task Orders experiments with datasets Split MNIST, FashionMNIST, and CIFAR-10 respectively. Table \ref{tab:rankings_notmnist_new_dataset} and \ref{tab:rankings_fashionmnist_new_dataset} shows the ACCs and ranks for the New Dataset experiments with datasets Split notMNIST and FashionMNIST respectively. The numbers for the average rankings in Table \ref{tab:average_ranking_rl_experiment} are computed by averaging over the ranks in each test environment for every method separately. 


\clearpage
%%% Figures 
\begin{figure}[t]
  \centering
  \setlength{\figwidth}{0.26\textwidth}
  \setlength{\figheight}{.14\textheight}
  



\pgfplotsset{every axis title/.append style={at={(0.5,0.85)}}}
\pgfplotsset{every tick label/.append style={font=\scriptsize}}
\pgfplotsset{every major tick/.append style={major tick length=2pt}}
\pgfplotsset{every minor tick/.append style={minor tick length=1pt}}
\pgfplotsset{every axis x label/.append style={at={(0.5,-0.22)}}}
\pgfplotsset{every axis y label/.append style={at={(-0.30,0.5)}}}
\begin{tikzpicture}
\tikzstyle{every node}=[font=\scriptsize]
\definecolor{color0}{rgb}{0.12156862745098,0.466666666666667,0.705882352941177}
\definecolor{color1}{rgb}{1,0.498039215686275,0.0549019607843137}
\definecolor{color2}{rgb}{0.172549019607843,0.627450980392157,0.172549019607843}
\definecolor{color3}{rgb}{0.83921568627451,0.152941176470588,0.156862745098039}
\definecolor{color4}{rgb}{0.580392156862745,0.403921568627451,0.741176470588235}
\definecolor{color5}{rgb}{0.549019607843137,0.337254901960784,0.294117647058824}
\definecolor{color6}{rgb}{0.890196078431372,0.466666666666667,0.76078431372549}

\begin{groupplot}[group style={group size= 4 by 3, horizontal sep=1.00cm, vertical sep=1.20cm}]

\nextgroupplot[title=Seed 10,
height=\figheight,
legend cell align={left},
legend columns=-1,
legend style={
  fill opacity=0.8,
  draw opacity=1,
  text opacity=1,
  at={(-0.20,1.36)},%at={(0.10,1.29)},
  anchor=south west,
  draw=white!80!black
},
minor xtick={25, 75},
minor ytick={},
tick align=outside,
tick pos=left,
%title={Labels: [[8, 2], [5, 6], [3, 1], [0, 7], [4, 9]]},
width=\figwidth,
x grid style={white!69.0196078431373!black},
xlabel={Eval. Steps},
xminorgrids,
xmajorgrids,
xmin=-3.95, xmax=104.95,
xtick style={color=black},
xtick={-25,0,50,100,125},
xticklabels={-25,0,50,100,125},
y grid style={white!69.0196078431373!black},
ylabel={ACC (\%)},
ymajorgrids,
ymin=0.895982454991159, ymax=0.960314079751296,
ytick style={color=black},
ytick={0.89,0.9,0.91,0.92,0.93,0.94,0.95,0.96,0.97},
yticklabels={89,90,91,92,93,94,95,96,97}
]
\addplot [semithick, color0]
table {%
1 0.928916958967845
2 0.951259243488312
3 0.951259243488312
4 0.957143481572469
5 0.957389914989471
6 0.950179465611776
7 0.935758566856384
8 0.935758566856384
9 0.935758566856384
10 0.947296047210693
11 0.953064787387848
12 0.953064787387848
13 0.953064787387848
14 0.953064787387848
15 0.953064787387848
16 0.953064787387848
17 0.953064787387848
18 0.953064787387848
19 0.953064787387848
20 0.953064787387848
21 0.953064787387848
22 0.953064787387848
23 0.953064787387848
24 0.953064787387848
25 0.953064787387848
26 0.953064787387848
27 0.953064787387848
28 0.953064787387848
29 0.953064787387848
30 0.953064787387848
31 0.953064787387848
32 0.953064787387848
33 0.953064787387848
34 0.953064787387848
35 0.953064787387848
36 0.953064787387848
37 0.953064787387848
38 0.953064787387848
39 0.953064787387848
40 0.953064787387848
41 0.953064787387848
42 0.953064787387848
43 0.953064787387848
44 0.953064787387848
45 0.953064787387848
46 0.953064787387848
47 0.947296047210693
48 0.947296047210693
49 0.953064787387848
50 0.953064787387848
51 0.953064787387848
52 0.953064787387848
53 0.953064787387848
54 0.953064787387848
55 0.953064787387848
56 0.953064787387848
57 0.953064787387848
58 0.953064787387848
59 0.953064787387848
60 0.947296047210693
61 0.953064787387848
62 0.953064787387848
63 0.953064787387848
64 0.953064787387848
65 0.953064787387848
66 0.947296047210693
67 0.953064787387848
68 0.947296047210693
69 0.953064787387848
70 0.953064787387848
71 0.953064787387848
72 0.941527307033539
73 0.953064787387848
74 0.947296047210693
75 0.953064787387848
76 0.953064787387848
77 0.953064787387848
78 0.947296047210693
79 0.953064787387848
80 0.953064787387848
81 0.953064787387848
82 0.953064787387848
83 0.953064787387848
84 0.953064787387848
85 0.953064787387848
86 0.953064787387848
87 0.953064787387848
88 0.947296047210693
89 0.947296047210693
90 0.947296047210693
91 0.953064787387848
92 0.953064787387848
93 0.953064787387848
94 0.953064787387848
95 0.953064787387848
96 0.953064787387848
97 0.947296047210693
98 0.953064787387848
99 0.947296047210693
100 0.953064787387848
};
\addlegendentry{A2C}
\addplot [semithick, color1]
table {%
1 0.946875449021657
2 0.943612333138784
3 0.939076503117879
4 0.943958342075348
5 0.951629606882731
6 0.949499154090881
7 0.938134737809499
8 0.947858540217082
9 0.93805935382843
10 0.947194731235504
11 0.938645708560944
12 0.944262035687764
13 0.930862271785736
14 0.941974437236786
15 0.938590168952942
16 0.943620713551839
17 0.937054864565531
18 0.93942437171936
19 0.940189401308695
20 0.938030183315277
21 0.936716564496358
22 0.932151925563812
23 0.934587570031484
24 0.936899375915527
25 0.936111251513163
26 0.945568529764811
27 0.931886915365855
28 0.944513730208079
29 0.937833281358083
30 0.941893573602041
31 0.939429183801015
32 0.927043116092682
33 0.933157527446747
34 0.935961580276489
35 0.934696674346924
36 0.945227158069611
37 0.946772507826487
38 0.936443185806274
39 0.944100288550059
40 0.939507095019023
41 0.939349623521169
42 0.935918192068736
43 0.943360650539398
44 0.940596298376719
45 0.942300256093343
46 0.941220355033874
47 0.943428015708923
48 0.945602977275848
49 0.949816147486369
50 0.940117915471395
51 0.947238286336263
52 0.934195733070374
53 0.935175971190135
54 0.944136706988017
55 0.939367651939392
56 0.948354355494181
57 0.948414611816406
58 0.94425938129425
59 0.938315244515737
60 0.943881408373515
61 0.938720309734345
62 0.940491338570913
63 0.937058353424072
64 0.946232680479685
65 0.950825242201487
66 0.947540696461995
67 0.946412448088328
68 0.938605499267578
69 0.950295746326447
70 0.944109499454498
71 0.947393588225047
72 0.952604659398397
73 0.949641068776449
74 0.94713325103124
75 0.95142068862915
76 0.94816845258077
77 0.943809986114502
78 0.941947905222575
79 0.945736765861511
80 0.943446473280589
81 0.94816845258077
82 0.951026980082194
83 0.948867444197337
84 0.947391430536906
85 0.940476810932159
86 0.949786361058553
87 0.947496966520945
88 0.945095586776733
89 0.936391858259837
90 0.944066290060679
91 0.945736765861511
92 0.942870044708252
93 0.945836798350016
94 0.943091146151225
95 0.938331906000773
96 0.943332389990489
97 0.928321810563405
98 0.945367765426636
99 0.9357603708903
100 0.947986404101054
};
\addlegendentry{DQN}
\addplot [semithick, color2]
table {%
1 0.943769861300433
100 0.943769861300433
};
\addlegendentry{Random}
\addplot [semithick, color3]
table {%
1 0.898906619752983
100 0.898906619752983
};
\addlegendentry{ETS}
\addplot [semithick, color4]
table {%
1 0.946386095853952
100 0.946386095853952
};
\addlegendentry{Heur-GD}
\addplot [semithick, color5, dash pattern=on 1pt off 3pt on 3pt off 3pt]
table {%
1 0.956332387217083
100 0.956332387217083
};
\addlegendentry{Heur-LD}
\addplot [semithick, color6, dashed]
table {%
1 0.946386095853952
100 0.946386095853952
};
\addlegendentry{Heur-AT}



\nextgroupplot[title=Seed 11,
height=\figheight,
legend cell align={left},
legend style={
  fill opacity=0.8,
  draw opacity=1,
  text opacity=1,
  at={(0.03,0.03)},
  anchor=south west,
  draw=white!80!black
},
minor xtick={25, 75},
minor ytick={},
tick align=outside,
tick pos=left,
%title={Labels: [[7, 8], [2, 6], [4, 5], [1, 3], [0, 9]]},
width=\figwidth,
x grid style={white!69.0196078431373!black},
xlabel={Eval. Steps},
xminorgrids,
xmajorgrids,
xmin=-3.95, xmax=104.95,
xtick style={color=black},
xtick={-25,0,50,100,125},
xticklabels={-25,0,50,100,125},
y grid style={white!69.0196078431373!black},
%ylabel={ACC},
ymajorgrids,
ymin=0.908595944220984, ymax=0.972076868974256,
ytick style={color=black},
ytick={0.9,0.91,0.92,0.93,0.94,0.95,0.96,0.97,0.98},
yticklabels={90,91,92,93,94,95,96,97,98}
]
\addplot [semithick, color0]
table {%
	1 0.949812457561493
	2 0.949775466918945
	3 0.960892503261566
	4 0.956866431236267
	5 0.945231194496155
	6 0.945613367557526
	7 0.94686802148819
	8 0.952512142658234
	9 0.943398241996765
	10 0.944289662837982
	11 0.930401303768158
	12 0.950187075138092
	13 0.946467769145966
	14 0.943979327678681
	15 0.942831747531891
	16 0.943993356227875
	17 0.944837226867676
	18 0.941551671028137
	19 0.931446702480316
	20 0.949583513736725
	21 0.953074641227722
	22 0.952990527153015
	23 0.939547181129456
	24 0.942692797183991
	25 0.956659049987793
	26 0.954558956623077
	27 0.956728446483612
	28 0.952419261932373
	29 0.953414702415466
	30 0.949406821727753
	31 0.96032018661499
	32 0.938822565078735
	33 0.941302559375763
	34 0.945325396060944
	35 0.949670062065124
	36 0.940685608386993
	37 0.946110789775848
	38 0.936599233150482
	39 0.952321965694427
	40 0.948934230804443
	41 0.942440538406372
	42 0.938124449253082
	43 0.949964995384216
	44 0.942903997898102
	45 0.954234335422516
	46 0.95132431268692
	47 0.948090310096741
	48 0.938373746871948
	49 0.942048320770264
	50 0.950774230957031
	51 0.948220450878143
	52 0.944915144443512
	53 0.934470863342285
	54 0.946825733184814
	55 0.947917900085449
	56 0.931744492053986
	57 0.948268630504608
	58 0.940266535282135
	59 0.949667906761169
	60 0.941861910820007
	61 0.946911418437958
	62 0.949336833953857
	63 0.93532523393631
	64 0.944827678203583
	65 0.944855451583862
	66 0.943393461704254
	67 0.941029033660889
	68 0.939231975078583
	69 0.938024179935455
	70 0.940667865276337
	71 0.933951034545898
	72 0.942696425914764
	73 0.946916031837463
	74 0.937888109683991
	75 0.940196809768677
	76 0.936573145389557
	77 0.932672383785248
	78 0.942083840370178
	79 0.936137390136719
	80 0.941790580749512
	81 0.941255300045013
	82 0.937389364242554
	83 0.903695275783539
	84 0.940915014743805
	85 0.938173484802246
	86 0.934200079441071
	87 0.943309750556946
	88 0.940444509983063
	89 0.93537590265274
	90 0.931826453208923
	91 0.936913092136383
	92 0.938212811946869
	93 0.946724779605866
	94 0.928359398841858
	95 0.941560471057892
	96 0.930597925186157
	97 0.93455456495285
	98 0.934019336700439
	99 0.928332870006561
	100 0.940782828330994
};
%\addlegendentry{DQN}
\addplot [semithick, color1]
table {%
	1 0.930601587295532
	2 0.966550159454346
	3 0.966550159454346
	4 0.966192562580109
	5 0.969191372394562
	6 0.96504182100296
	7 0.948443615436554
	8 0.948443615436554
	9 0.948443615436554
	10 0.948443615436554
	11 0.961694481372833
	12 0.958381764888763
	13 0.965007197856903
	14 0.965007197856903
	15 0.961694481372833
	16 0.965007197856903
	17 0.961694481372833
	18 0.958381764888763
	19 0.961694481372833
	20 0.965007197856903
	21 0.958381764888763
	22 0.958381764888763
	23 0.955069048404694
	24 0.958381764888763
	25 0.965007197856903
	26 0.958381764888763
	27 0.958381764888763
	28 0.961694481372833
	29 0.961694481372833
	30 0.955069048404694
	31 0.965007197856903
	32 0.958381764888763
	33 0.958381764888763
	34 0.958381764888763
	35 0.955069048404694
	36 0.958381764888763
	37 0.955069048404694
	38 0.955069048404694
	39 0.951756331920624
	40 0.955069048404694
	41 0.948443615436554
	42 0.955069048404694
	43 0.955069048404694
	44 0.955069048404694
	45 0.948443615436554
	46 0.951756331920624
	47 0.948443615436554
	48 0.955069048404694
	49 0.958381764888763
	50 0.951756331920624
	51 0.951756331920624
	52 0.958381764888763
	53 0.951756331920624
	54 0.958381764888763
	55 0.955069048404694
	56 0.951756331920624
	57 0.958381764888763
	58 0.955069048404694
	59 0.958381764888763
	60 0.958381764888763
	61 0.961694481372833
	62 0.958381764888763
	63 0.958381764888763
	64 0.958381764888763
	65 0.958381764888763
	66 0.961694481372833
	67 0.961694481372833
	68 0.951756331920624
	69 0.958381764888763
	70 0.961694481372833
	71 0.961694481372833
	72 0.958381764888763
	73 0.961694481372833
	74 0.961694481372833
	75 0.960777387619019
	76 0.961694481372833
	77 0.961694481372833
	78 0.958381764888763
	79 0.961694481372833
	80 0.961694481372833
	81 0.961694481372833
	82 0.960777387619019
	83 0.965007197856903
	84 0.965007197856903
	85 0.964090104103088
	86 0.958381764888763
	87 0.964090104103088
	88 0.964090104103088
	89 0.962255916595459
	90 0.965007197856903
	91 0.961694481372833
	92 0.964090104103088
	93 0.964090104103088
	94 0.965007197856903
	95 0.963173010349274
	96 0.964090104103088
	97 0.963173010349274
	98 0.964090104103088
	99 0.959860293865204
	100 0.964090104103088
};
%\addlegendentry{A2C}
\addplot [semithick, color2]
table {%
	1 0.941761815547943
	2 0.941761815547943
	3 0.941761815547943
	4 0.943232760429382
	5 0.943232760429382
	6 0.956506490707397
	7 0.957273149490356
	8 0.958714089393616
	9 0.958714089393616
	10 0.956937248706818
	11 0.95984382390976
	12 0.958714089393616
	13 0.958714089393616
	14 0.958714089393616
	15 0.958714089393616
	16 0.960152294635773
	17 0.95927184343338
	18 0.95927184343338
	19 0.954713544845581
	20 0.956052875518799
	21 0.9589133477211
	22 0.9589133477211
	23 0.9589133477211
	24 0.9589133477211
	25 0.9589133477211
	26 0.9589133477211
	27 0.9589133477211
	28 0.958693928718567
	29 0.956346516609192
	30 0.9589133477211
	31 0.9589133477211
	32 0.9589133477211
	33 0.9589133477211
	34 0.9589133477211
	35 0.9589133477211
	36 0.960173325538635
	37 0.9589133477211
	38 0.953893489837647
	39 0.952666237354279
	40 0.954494125843048
	41 0.958400287628174
	42 0.957150495052338
	43 0.958693928718567
	44 0.958693928718567
	45 0.958400287628174
	46 0.958400287628174
	47 0.958693928718567
	48 0.958693928718567
	49 0.958693928718567
	50 0.958693928718567
	51 0.958693928718567
	52 0.958693928718567
	53 0.958693928718567
	54 0.95981064081192
	55 0.955153148174286
	56 0.958693928718567
	57 0.958343198299408
	58 0.958693928718567
	59 0.958773958683014
	60 0.959471101760864
	61 0.959471101760864
	62 0.958773958683014
	63 0.957524166107178
	64 0.959471101760864
	65 0.958773958683014
	66 0.958773958683014
	67 0.955211129188538
	68 0.958773958683014
	69 0.958773958683014
	70 0.955211129188538
	71 0.958773958683014
	72 0.958773958683014
	73 0.959526281356812
	74 0.958856613636017
	75 0.959702248573303
	76 0.958773958683014
	77 0.956346516609192
	78 0.955211129188538
	79 0.958773958683014
	80 0.956346516609192
	81 0.951848232746124
	82 0.96026933670044
	83 0.959973311424255
	84 0.961511354446411
	85 0.958299555778503
	86 0.960509932041168
	87 0.958011574745178
	88 0.958094229698181
	89 0.958551387786865
	90 0.955022370815277
	91 0.956802217960358
	92 0.955577042102814
	93 0.956882247924805
	94 0.956882247924805
	95 0.957371439933777
	96 0.957371439933777
	97 0.956955509185791
	98 0.955022370815277
	99 0.956882247924805
	100 0.958062195777893
};
%\addlegendentry{SAC}
\addplot [semithick, color3]
table {%
	1 0.92125960952966
	100 0.92125960952966
};
%\addlegendentry{Random}
\addplot [semithick, color4]
table {%
	1 0.937675110453656
	100 0.937675110453656
};
%\addlegendentry{ETS}
\addplot [semithick, color5]
table {%
	1 0.913359966355126
	100 0.913359966355126
};
%\addlegendentry{Heur-GD}
\addplot [semithick, color6, dash pattern=on 1pt off 3pt on 3pt off 3pt]
table {%
	1 0.913359966355126
	100 0.913359966355126
};
%\addlegendentry{Heur-LD}
\addplot [semithick, white!49.8039215686275!black, dashed]
table {%
	1 0.913359966355126
	100 0.913359966355126
};
%\addlegendentry{Heur-AT}



\nextgroupplot[title=Seed 12,
height=\figheight,
legend cell align={left},
legend style={
  fill opacity=0.8,
  draw opacity=1,
  text opacity=1,
  at={(0.97,0.03)},
  anchor=south east,
  draw=white!80!black
},
minor xtick={25, 75},
minor ytick={},
tick align=outside,
tick pos=left,
%title={Labels: [[5, 8], [7, 0], [4, 9], [3, 2], [1, 6]]},
width=\figwidth,
x grid style={white!69.0196078431373!black},
xlabel={Eval. Steps},
xminorgrids,
xmajorgrids,
xmin=-3.95, xmax=104.95,
xtick style={color=black},
xtick={-25,0,50,100,125},
xticklabels={-25,0,50,100,125},
y grid style={white!69.0196078431373!black},
%ylabel={ACC},
ymajorgrids,
ymin=0.91, ymax=0.95,
ytick style={color=black},
ytick={0.91,0.92,0.93,0.94,0.95},
yticklabels={91,92,93,94,95}
]
\addplot [semithick, color0]
table {%
	1 0.924207570552826
	2 0.929173192977905
	3 0.933564960956573
	4 0.928047413825989
	5 0.930393221378326
	6 0.925607509613037
	7 0.922258069515228
	8 0.927588517665863
	9 0.926537489891052
	10 0.916732404232025
	11 0.920992476940155
	12 0.92415479183197
	13 0.935238471031189
	14 0.936645829677582
	15 0.915031096935272
	16 0.93369217634201
	17 0.929549508094788
	18 0.928418254852295
	19 0.92591046333313
	20 0.922384634017944
	21 0.931622250080109
	22 0.932581198215485
	23 0.931437931060791
	24 0.929808349609375
	25 0.936558699607849
	26 0.934100792407989
	27 0.938888564109802
	28 0.937725448608398
	29 0.935701014995575
	30 0.939993996620178
	31 0.935162389278412
	32 0.937869670391083
	33 0.94358188867569
	34 0.932286295890808
	35 0.942554121017456
	36 0.931464703083038
	37 0.919817514419556
	38 0.927928068637848
	39 0.928479843139648
	40 0.936659381389618
	41 0.934711079597473
	42 0.935029854774475
	43 0.937896454334259
	44 0.938827860355377
	45 0.936895394325256
	46 0.908302109241486
	47 0.927564733028412
	48 0.929870774745941
	49 0.924899156093597
	50 0.930377097129822
	51 0.928829081058502
	52 0.93658331155777
	53 0.936232712268829
	54 0.926392548084259
	55 0.925232834815979
	56 0.938509414196014
	57 0.93367520570755
	58 0.942024314403534
	59 0.932734744548798
	60 0.929209289550781
	61 0.9211811876297
	62 0.927589473724365
	63 0.91637229681015
	64 0.9146497797966
	65 0.931868298053741
	66 0.931552817821503
	67 0.927083780765533
	68 0.92749582529068
	69 0.925627110004425
	70 0.93371896982193
	71 0.932625524997711
	72 0.935283114910126
	73 0.930264873504639
	74 0.922552454471588
	75 0.931776270866394
	76 0.928163428306579
	77 0.92653840303421
	78 0.935957896709442
	79 0.924751200675964
	80 0.931757996082306
	81 0.928001294136047
	82 0.931330494880676
	83 0.932585844993591
	84 0.937321374416351
	85 0.922634980678558
	86 0.930933086872101
	87 0.932527911663055
	88 0.925494439601898
	89 0.931979060173035
	90 0.935957896709442
	91 0.934714961051941
	92 0.929686982631683
	93 0.926328732967377
	94 0.930632083415985
	95 0.935140006542206
	96 0.93530912399292
	97 0.934028418064117
	98 0.931712002754212
	99 0.933255157470703
	100 0.930540452003479
};
%\addlegendentry{DQN}
\addplot [semithick, color1]
table {%
	1 0.927715406417847
	2 0.93209730386734
	3 0.93209730386734
	4 0.937182974815369
	5 0.942491888999939
	6 0.943392896652222
	7 0.946996927261353
	8 0.946996927261353
	9 0.946996927261353
	10 0.938371496200562
	11 0.936215138435364
	12 0.936215138435364
	13 0.936215138435364
	14 0.936215138435364
	15 0.936215138435364
	16 0.936215138435364
	17 0.936215138435364
	18 0.936215138435364
	19 0.936215138435364
	20 0.936215138435364
	21 0.936215138435364
	22 0.936215138435364
	23 0.936215138435364
	24 0.936215138435364
	25 0.936215138435364
	26 0.936215138435364
	27 0.936215138435364
	28 0.936215138435364
	29 0.936215138435364
	30 0.936215138435364
	31 0.936215138435364
	32 0.936215138435364
	33 0.936215138435364
	34 0.936215138435364
	35 0.936215138435364
	36 0.936215138435364
	37 0.936215138435364
	38 0.936215138435364
	39 0.936215138435364
	40 0.936215138435364
	41 0.936215138435364
	42 0.936215138435364
	43 0.936215138435364
	44 0.936215138435364
	45 0.936215138435364
	46 0.936215138435364
	47 0.936215138435364
	48 0.936215138435364
	49 0.936215138435364
	50 0.936215138435364
	51 0.936215138435364
	52 0.936215138435364
	53 0.936215138435364
	54 0.936215138435364
	55 0.936215138435364
	56 0.936215138435364
	57 0.936215138435364
	58 0.936215138435364
	59 0.936215138435364
	60 0.936215138435364
	61 0.936215138435364
	62 0.936215138435364
	63 0.936215138435364
	64 0.936215138435364
	65 0.936215138435364
	66 0.936215138435364
	67 0.936215138435364
	68 0.936215138435364
	69 0.936215138435364
	70 0.936215138435364
	71 0.936215138435364
	72 0.936215138435364
	73 0.936215138435364
	74 0.936215138435364
	75 0.936215138435364
	76 0.936215138435364
	77 0.936215138435364
	78 0.936215138435364
	79 0.936215138435364
	80 0.936215138435364
	81 0.936215138435364
	82 0.936215138435364
	83 0.936215138435364
	84 0.936215138435364
	85 0.936215138435364
	86 0.936215138435364
	87 0.936215138435364
	88 0.936215138435364
	89 0.936215138435364
	90 0.936215138435364
	91 0.936215138435364
	92 0.936215138435364
	93 0.936215138435364
	94 0.936215138435364
	95 0.936215138435364
	96 0.936215138435364
	97 0.936215138435364
	98 0.936215138435364
	99 0.936215138435364
	100 0.936215138435364
};
%\addlegendentry{A2C}
\addplot [semithick, color2]
table {%
	1 0.927580988407135
	2 0.927580988407135
	3 0.927580988407135
	4 0.92012232542038
	5 0.916206045150757
	6 0.916206214427948
	7 0.908623313903809
	8 0.908623144626617
	9 0.912414679527283
	10 0.912414848804474
	11 0.912414848804474
	12 0.908623144626617
	13 0.916247909069061
	14 0.916247909069061
	15 0.916247909069061
	16 0.916247909069061
	17 0.916247909069061
	18 0.916247909069061
	19 0.916090621948242
	20 0.916247909069061
	21 0.917300915718079
	22 0.921092450618744
	23 0.916090621948242
	24 0.916247909069061
	25 0.916247909069061
	26 0.919882156848908
	27 0.920039443969727
	28 0.919882156848908
	29 0.920039443969727
	30 0.924613606929779
	31 0.924770894050598
	32 0.924770894050598
	33 0.920979359149933
	34 0.920979359149933
	35 0.924770894050598
	36 0.924770894050598
	37 0.924729368686676
	38 0.920937833786011
	39 0.924034337997437
	40 0.924034337997437
	41 0.920242803096771
	42 0.923992812633514
	43 0.923992812633514
	44 0.924034337997437
	45 0.924034337997437
	46 0.923992812633514
	47 0.923992812633514
	48 0.923997430801392
	49 0.923997430801392
	50 0.92395590543747
	51 0.923997430801392
	52 0.923562579154968
	53 0.927752058506012
	54 0.923997430801392
	55 0.922097291946411
	56 0.927788965702057
	57 0.923562579154968
	58 0.92395590543747
	59 0.927788965702057
	60 0.919646117687225
	61 0.921092450618744
	62 0.924842460155487
	63 0.920201277732849
	64 0.915813057422638
	65 0.923400745391846
	66 0.924770894050598
	67 0.923992812633514
	68 0.919211266040802
	69 0.923562579154968
	70 0.919692189693451
	71 0.924449133872986
	72 0.924336543083191
	73 0.922239291667938
	74 0.926970911026001
	75 0.917187824249268
	76 0.917146298885345
	77 0.917146129608154
	78 0.916169307231903
	79 0.916206383705139
	80 0.919604592323303
	81 0.917986667156219
	82 0.914195132255554
	83 0.925731539726257
	84 0.917851316928864
	85 0.922055766582489
	86 0.92213445186615
	87 0.922212884426117
	88 0.924518406391144
	89 0.928309941291809
	90 0.924559931755066
	91 0.931691765785217
	92 0.928273034095764
	93 0.920292520523071
	94 0.928309941291809
	95 0.924559762477875
	96 0.924518406391144
	97 0.924559931755066
	98 0.924559931755066
	99 0.928309941291809
	100 0.928351466655731
};
%\addlegendentry{SAC}
\addplot [semithick, color3]
table {%
	1 0.922176351419478
	100 0.922176351419478
};
%\addlegendentry{Random}
\addplot [semithick, color4]
table {%
	1 0.916884203700455
	100 0.916884203700455
};
%\addlegendentry{ETS}
\addplot [semithick, color5]
table {%
	1 0.9481932676297
	100 0.9481932676297
};
%\addlegendentry{Heur-GD}
\addplot [semithick, color6, dash pattern=on 1pt off 3pt on 3pt off 3pt]
table {%
	1 0.919327072719136
	100 0.919327072719136
};
%\addlegendentry{Heur-LD}
\addplot [semithick, white!49.8039215686275!black, dashed]
table {%
	1 0.9481932676297
	100 0.9481932676297
};
%\addlegendentry{Heur-AT}



\nextgroupplot[title=Seed 13,
height=\figheight,
legend cell align={left},
legend style={
  fill opacity=0.8,
  draw opacity=1,
  text opacity=1,
  at={(0.97,0.03)},
  anchor=south east,
  draw=white!80!black
},
minor xtick={25, 75},
minor ytick={},
tick align=outside,
tick pos=left,
%title={Labels: [[3, 5], [6, 1], [4, 7], [8, 9], [0, 2]]},
width=\figwidth,
x grid style={white!69.0196078431373!black},
xlabel={Eval. Steps},
xminorgrids,
xmajorgrids,
xmin=-3.95, xmax=104.95,
xtick style={color=black},
xtick={-25,0,50,100,125},
xticklabels={-25,0,50,100,125},
y grid style={white!69.0196078431373!black},
%ylabel={ACC},
ymajorgrids,
ymin=0.91, ymax=0.965336399277051,
ytick style={color=black},
ytick={0.91,0.92,0.93,0.94,0.95,0.96,0.97},
yticklabels={91,92,93,94,95,96,97}
]
\addplot [semithick, color0]
table {%
	1 0.94407958984375
	2 0.936318454742432
	3 0.947445096969604
	4 0.946504876613617
	5 0.944883387088776
	6 0.942895493507385
	7 0.953645408153534
	8 0.956253445148468
	9 0.942046263217926
	10 0.955559017658233
	11 0.959160966873169
	12 0.946965930461883
	13 0.951435182094574
	14 0.935411803722382
	15 0.928654100894928
	16 0.944909629821777
	17 0.9371990275383
	18 0.947445607185364
	19 0.948051402568817
	20 0.95228022813797
	21 0.947926814556122
	22 0.95360830783844
	23 0.944389894008636
	24 0.958125319480896
	25 0.940531749725342
	26 0.954730541706085
	27 0.956690919399261
	28 0.957412021160126
	29 0.941255302429199
	30 0.951203615665436
	31 0.955989248752594
	32 0.960922265052795
	33 0.957080857753754
	34 0.954466605186462
	35 0.96093279838562
	36 0.961934671401978
	37 0.955538847446442
	38 0.959152238368988
	39 0.949690964221954
	40 0.961638457775116
	41 0.960985457897186
	42 0.961805782318115
	43 0.964418361186981
	44 0.955766880512238
	45 0.947572944164276
	46 0.957326240539551
	47 0.955383372306824
	48 0.946119985580444
	49 0.955771377086639
	50 0.954218716621399
	51 0.954565784931183
	52 0.961631319522858
	53 0.962026016712189
	54 0.962096524238586
	55 0.958655517101288
	56 0.954666602611542
	57 0.956585283279419
	58 0.954020566940308
	59 0.953756141662598
	60 0.953353888988495
	61 0.948602015972138
	62 0.953735647201538
	63 0.955932664871216
	64 0.957354364395142
	65 0.954616487026215
	66 0.954616487026215
	67 0.954616487026215
	68 0.954179191589355
	69 0.951076197624207
	70 0.945662777423859
	71 0.940962488651276
	72 0.945419273376465
	73 0.945866641998291
	74 0.950721111297608
	75 0.954346742630005
	76 0.961702144145966
	77 0.952450807094574
	78 0.960819141864777
	79 0.961248767375946
	80 0.961818206310272
	81 0.961489198207855
	82 0.953463044166565
	83 0.958222694396973
	84 0.960931057929993
	85 0.946363959312439
	86 0.956439669132233
	87 0.955210733413696
	88 0.955375337600708
	89 0.961282110214233
	90 0.952119126319885
	91 0.954131832122803
	92 0.952647817134857
	93 0.956467955112457
	94 0.953886413574219
	95 0.954470994472504
	96 0.955059974193573
	97 0.954469792842865
	98 0.951070246696472
	99 0.959129455089569
	100 0.955949988365173
};
%\addlegendentry{DQN}
\addplot [semithick, color1]
table {%
	1 0.944150683879852
	2 0.927731931209564
	3 0.927731931209564
	4 0.920855252742767
	5 0.951843798160553
	6 0.952179987430572
	7 0.953524744510651
	8 0.953524744510651
	9 0.953524744510651
	10 0.954770562648773
	11 0.955601108074188
	12 0.954770562648773
	13 0.955601108074188
	14 0.955601108074188
	15 0.955601108074188
	16 0.955601108074188
	17 0.955601108074188
	18 0.955601108074188
	19 0.955185835361481
	20 0.955601108074188
	21 0.955601108074188
	22 0.955601108074188
	23 0.955601108074188
	24 0.955601108074188
	25 0.955601108074188
	26 0.955601108074188
	27 0.955601108074188
	28 0.955601108074188
	29 0.955185835361481
	30 0.955185835361481
	31 0.955601108074188
	32 0.955601108074188
	33 0.955601108074188
	34 0.955601108074188
	35 0.955185835361481
	36 0.955601108074188
	37 0.955601108074188
	38 0.955601108074188
	39 0.955185835361481
	40 0.955185835361481
	41 0.955601108074188
	42 0.955601108074188
	43 0.955601108074188
	44 0.955601108074188
	45 0.955601108074188
	46 0.955601108074188
	47 0.955601108074188
	48 0.955601108074188
	49 0.955601108074188
	50 0.955601108074188
	51 0.955601108074188
	52 0.955601108074188
	53 0.955601108074188
	54 0.955601108074188
	55 0.955601108074188
	56 0.955601108074188
	57 0.955601108074188
	58 0.955601108074188
	59 0.955601108074188
	60 0.955601108074188
	61 0.955601108074188
	62 0.955601108074188
	63 0.955601108074188
	64 0.955601108074188
	65 0.955601108074188
	66 0.955601108074188
	67 0.955601108074188
	68 0.955601108074188
	69 0.955601108074188
	70 0.955601108074188
	71 0.955601108074188
	72 0.955601108074188
	73 0.955601108074188
	74 0.955601108074188
	75 0.955601108074188
	76 0.955601108074188
	77 0.955601108074188
	78 0.955601108074188
	79 0.955601108074188
	80 0.955601108074188
	81 0.955601108074188
	82 0.955601108074188
	83 0.955601108074188
	84 0.955601108074188
	85 0.951601815223694
	86 0.955601108074188
	87 0.955601108074188
	88 0.955601108074188
	89 0.955601108074188
	90 0.955601108074188
	91 0.955601108074188
	92 0.955601108074188
	93 0.955601108074188
	94 0.955601108074188
	95 0.951601815223694
	96 0.955601108074188
	97 0.955601108074188
	98 0.955601108074188
	99 0.951601815223694
	100 0.955601108074188
};
%\addlegendentry{A2C}
\addplot [semithick, color2]
table {%
	1 0.930442702770233
	2 0.930442702770233
	3 0.930442702770233
	4 0.952474148273468
	5 0.950748896598816
	6 0.949119729995727
	7 0.950843458175659
	8 0.950136177539825
	9 0.951857967376709
	10 0.95024603843689
	11 0.950173184871674
	12 0.950748896598816
	13 0.945724830627441
	14 0.94609114408493
	15 0.94626357793808
	16 0.951838912963867
	17 0.948444013595581
	18 0.950290250778198
	19 0.945353112220764
	20 0.949614534378052
	21 0.941955187320709
	22 0.944135622978211
	23 0.944207589626312
	24 0.94626357793808
	25 0.947478041648865
	26 0.943668842315674
	27 0.945539875030518
	28 0.94571346282959
	29 0.946388025283813
	30 0.945297605991363
	31 0.943910708427429
	32 0.947478041648865
	33 0.946926772594452
	34 0.944342844486237
	35 0.937194030284882
	36 0.946979253292084
	37 0.944207589626312
	38 0.949217069149017
	39 0.943359439373016
	40 0.943359439373016
	41 0.944369931221008
	42 0.938633408546448
	43 0.940450468063355
	44 0.941231484413147
	45 0.945649735927582
	46 0.948648562431335
	47 0.945929379463196
	48 0.945649735927582
	49 0.949092161655426
	50 0.946497886180878
	51 0.948755192756653
	52 0.947648365497589
	53 0.941329603195191
	54 0.949217069149017
	55 0.950193929672241
	56 0.948012070655823
	57 0.947492482662201
	58 0.946053104400635
	59 0.948648562431335
	60 0.948469343185425
	61 0.946058089733124
	62 0.944457406997681
	63 0.94754955291748
	64 0.947878115177155
	65 0.939720222949982
	66 0.947029964923859
	67 0.946058089733124
	68 0.94754955291748
	69 0.945841383934021
	70 0.937805330753326
	71 0.953019323348999
	72 0.946058089733124
	73 0.94698915719986
	74 0.951645109653473
	75 0.942948222160339
	76 0.949625422954559
	77 0.946293520927429
	78 0.946058089733124
	79 0.945314204692841
	80 0.937803988456726
	81 0.942382569313049
	82 0.940201199054718
	83 0.942329154014587
	84 0.943834238052368
	85 0.942792782783508
	86 0.935607047080994
	87 0.934986724853516
	88 0.927268800735474
	89 0.920120820999145
	90 0.919708261489868
	91 0.9261123919487
	92 0.929396755695343
	93 0.926806282997131
	94 0.931878700256348
	95 0.94122706413269
	96 0.94122706413269
	97 0.9406689286232
	98 0.930901839733124
	99 0.928182656764984
	100 0.930901839733124
};
%\addlegendentry{SAC}
\addplot [semithick, color3]
table {%
	1 0.936537616435732
	100 0.936537616435732
};
%\addlegendentry{Random}
\addplot [semithick, color4]
table {%
	1 0.949541405667153
	100 0.949541405667153
};
%\addlegendentry{ETS}
\addplot [semithick, color5]
table {%
	1 0.946626678641995
	100 0.946626678641995
};
%\addlegendentry{Heur-GD}
\addplot [semithick, color6, dash pattern=on 1pt off 3pt on 3pt off 3pt]
table {%
	1 0.931740421926683
	100 0.931740421926683
};
%\addlegendentry{Heur-LD}
\addplot [semithick, white!49.8039215686275!black, dashed]
table {%
	1 0.946626678641995
	100 0.946626678641995
};
%\addlegendentry{Heur-AT}


\nextgroupplot[title=Seed 14,
height=\figheight,
legend cell align={left},
legend style={fill opacity=0.8, draw opacity=1, text opacity=1, draw=white!80!black},
minor xtick={25, 75},
minor ytick={},
tick align=outside,
tick pos=left,
%title={Labels: [[3, 9], [0, 5], [4, 2], [1, 7], [6, 8]]},
width=\figwidth,
x grid style={white!69.0196078431373!black},
xlabel={Eval. Steps},
xminorgrids,
xmajorgrids,
xmin=-3.95, xmax=104.95,
xtick style={color=black},
xtick={-25,0,50,100,125},
xticklabels={-25,0,50,100,125},
y grid style={white!69.0196078431373!black},
ylabel={ACC (\%)},
ymajorgrids,
ymin=0.80, ymax=0.95714084326699,
ytick style={color=black},
ytick={0.8,0.82,0.84,0.86,0.88,0.9,0.92,0.94,0.96},
yticklabels={80,82,84,86,88,90,92,94,96}
]
\addplot [semithick, color0]
table {%
	1 0.916396675109863
	2 0.896205940246582
	3 0.891117959022522
	4 0.916551446914673
	5 0.896587288379669
	6 0.918176510334015
	7 0.882758994102478
	8 0.850276832580567
	9 0.908061904907227
	10 0.894339718818665
	11 0.875018074512482
	12 0.924485545158386
	13 0.902640979290009
	14 0.893863232135773
	15 0.904817912578583
	16 0.918725221157074
	17 0.876341035366058
	18 0.868708076477051
	19 0.884978587627411
	20 0.901387987136841
	21 0.915615966320038
	22 0.915445804595947
	23 0.912999403476715
	24 0.911855869293213
	25 0.918023388385773
	26 0.869092843532562
	27 0.874445083141327
	28 0.905952234268189
	29 0.911653981208801
	30 0.913164944648743
	31 0.923977439403534
	32 0.9057031083107
	33 0.938780770301819
	34 0.918868775367737
	35 0.908285293579101
	36 0.926242802143097
	37 0.93048898935318
	38 0.932678084373474
	39 0.920158889293671
	40 0.881258022785187
	41 0.894296562671661
	42 0.898353590965271
	43 0.914174480438232
	44 0.926796355247498
	45 0.90830549955368
	46 0.882054014205933
	47 0.924008729457855
	48 0.927835884094238
	49 0.921014764308929
	50 0.932188739776611
	51 0.934300398826599
	52 0.905171036720276
	53 0.932568740844727
	54 0.924054172039032
	55 0.930006926059723
	56 0.913129487037659
	57 0.918123376369476
	58 0.926608290672302
	59 0.932085626125336
	60 0.91831383228302
	61 0.937865898609161
	62 0.902103989124298
	63 0.922160756587982
	64 0.908948993682861
	65 0.896896452903748
	66 0.917829582691193
	67 0.934444105625153
	68 0.921569378376007
	69 0.934439039230347
	70 0.916621522903442
	71 0.931531209945679
	72 0.934873199462891
	73 0.916299495697021
	74 0.917058711051941
	75 0.915784816741943
	76 0.904028639793396
	77 0.919655315876007
	78 0.905303547382355
	79 0.929512922763825
	80 0.904111721515656
	81 0.903080062866211
	82 0.916673884391785
	83 0.907977528572082
	84 0.923186464309692
	85 0.894923555850983
	86 0.915620608329773
	87 0.920473303794861
	88 0.917172136306763
	89 0.903990876674652
	90 0.895423247814178
	91 0.904028639793396
	92 0.918265881538391
	93 0.912154824733734
	94 0.88551604270935
	95 0.92120441198349
	96 0.907404839992523
	97 0.902782588005066
	98 0.915665168762207
	99 0.913756396770477
	100 0.912190101146698
};
%\addlegendentry{DQN}
\addplot [semithick, color1]
table {%
	1 0.890334053039551
	2 0.911138689517975
	3 0.911138689517975
	4 0.911307096481323
	5 0.94194061756134
	6 0.920390133857727
	7 0.834188199043274
	8 0.834188199043274
	9 0.834188199043274
	10 0.869349603652954
	11 0.904511008262634
	12 0.886930305957794
	13 0.922091710567474
	14 0.922091710567474
	15 0.922091710567474
	16 0.922091710567474
	17 0.904511008262634
	18 0.904511008262634
	19 0.904511008262634
	20 0.922091710567474
	21 0.886930305957794
	22 0.886930305957794
	23 0.886930305957794
	24 0.886930305957794
	25 0.922091710567474
	26 0.886930305957794
	27 0.886930305957794
	28 0.904511008262634
	29 0.886930305957794
	30 0.869349603652954
	31 0.922091710567474
	32 0.886930305957794
	33 0.886930305957794
	34 0.886930305957794
	35 0.869349603652954
	36 0.869349603652954
	37 0.851768901348114
	38 0.851768901348114
	39 0.851768901348114
	40 0.851768901348114
	41 0.834188199043274
	42 0.851768901348114
	43 0.834188199043274
	44 0.851768901348114
	45 0.834188199043274
	46 0.851768901348114
	47 0.834188199043274
	48 0.834188199043274
	49 0.834188199043274
	50 0.834188199043274
	51 0.834188199043274
	52 0.834188199043274
	53 0.834188199043274
	54 0.834188199043274
	55 0.834188199043274
	56 0.834188199043274
	57 0.834188199043274
	58 0.834188199043274
	59 0.834188199043274
	60 0.834188199043274
	61 0.834188199043274
	62 0.834188199043274
	63 0.851768901348114
	64 0.834188199043274
	65 0.834188199043274
	66 0.834188199043274
	67 0.834188199043274
	68 0.834188199043274
	69 0.834188199043274
	70 0.834188199043274
	71 0.834188199043274
	72 0.834188199043274
	73 0.834188199043274
	74 0.834188199043274
	75 0.834188199043274
	76 0.834188199043274
	77 0.834188199043274
	78 0.834188199043274
	79 0.834188199043274
	80 0.857894058227539
	81 0.857894058227539
	82 0.834188199043274
	83 0.857894058227539
	84 0.857894058227539
	85 0.857894058227539
	86 0.834188199043274
	87 0.834188199043274
	88 0.834188199043274
	89 0.857894058227539
	90 0.834188199043274
	91 0.834188199043274
	92 0.857894058227539
	93 0.857894058227539
	94 0.857894058227539
	95 0.857894058227539
	96 0.857894058227539
	97 0.857894058227539
	98 0.881599917411804
	99 0.881599917411804
	100 0.881599917411804
};
%\addlegendentry{A2C}
\addplot [semithick, color2]
table {%
	1 0.913314700126648
	2 0.913314700126648
	3 0.913314700126648
	4 0.939180500507355
	5 0.92199241399765
	6 0.945216856002808
	7 0.942206349372864
	8 0.950941164493561
	9 0.931096231937408
	10 0.924309854507446
	11 0.928900640010834
	12 0.937530643939972
	13 0.944648070335388
	14 0.950820319652557
	15 0.948624727725983
	16 0.934927937984466
	17 0.951645224094391
	18 0.946652956008911
	19 0.912691481113434
	20 0.920337865352631
	21 0.914513950347901
	22 0.921825230121613
	23 0.917171883583069
	24 0.931745939254761
	25 0.92791672706604
	26 0.937047877311707
	27 0.917366900444031
	28 0.925311527252197
	29 0.949100112915039
	30 0.9206361079216
	31 0.92117484331131
	32 0.926226260662079
	33 0.942813296318054
	34 0.93536817073822
	35 0.911138460636139
	36 0.9209454870224
	37 0.925996904373169
	38 0.939474647045135
	39 0.92442661523819
	40 0.932371242046356
	41 0.922900016307831
	42 0.916742844581604
	43 0.929281651973724
	44 0.939546916484833
	45 0.927275693416595
	46 0.917622117996216
	47 0.956747059822083
	48 0.927930977344513
	49 0.9414408826828
	50 0.935867738723755
	51 0.928396031856537
	52 0.944497742652893
	53 0.953567383289337
	54 0.958897848129273
	55 0.942302150726319
	56 0.930916912555695
	57 0.943461775779724
	58 0.943362760543823
	59 0.941862514019012
	60 0.951601147651672
	61 0.942741787433624
	62 0.95167341709137
	63 0.929740695953369
	64 0.943043413162232
	65 0.951371791362763
	66 0.951371791362763
	67 0.947908344268799
	68 0.949828207492828
	69 0.938247663974762
	70 0.928819739818573
	71 0.918329968452454
	72 0.947468707561493
	73 0.944375019073486
	74 0.930600316524506
	75 0.931039953231812
	76 0.941862514019012
	77 0.948137700557709
	78 0.941631283760071
	79 0.938838703632355
	80 0.936358132362366
	81 0.93994775056839
	82 0.939926476478577
	83 0.944654684066773
	84 0.940106024742126
	85 0.955012166500092
	86 0.948577754497528
	87 0.948807110786438
	88 0.940339367389679
	89 0.918703589439392
	90 0.939899730682373
	91 0.931269726753235
	92 0.919143226146698
	93 0.932681791782379
	94 0.940890436172485
	95 0.931510202884674
	96 0.932893748283386
	97 0.931269726753235
	98 0.931269726753235
	99 0.916769435405731
	100 0.932010989189148
};
%\addlegendentry{SAC}
\addplot [semithick, color3]
table {%
	1 0.857374093030356
	100 0.857374093030356
};
%\addlegendentry{Random}
\addplot [semithick, color4]
table {%
	1 0.872864057756591
	100 0.872864057756591
};
%\addlegendentry{ETS}
\addplot [semithick, color5]
table {%
	1 0.812038970097982
	100 0.812038970097982
};
%\addlegendentry{Heur-GD}
\addplot [semithick, color6, dash pattern=on 1pt off 3pt on 3pt off 3pt]
table {%
	1 0.812038970097982
	100 0.812038970097982
};
%\addlegendentry{Heur-LD}
\addplot [semithick, white!49.8039215686275!black, dashed]
table {%
	1 0.823641171956639
	100 0.823641171956639
};
%\addlegendentry{Heur-AT}



\nextgroupplot[title=Seed 15,
height=\figheight,
legend cell align={left},
legend style={
  fill opacity=0.8,
  draw opacity=1,
  text opacity=1,
  at={(0.97,0.03)},
  anchor=south east,
  draw=white!80!black
},
minor xtick={25, 75},
minor ytick={},
tick align=outside,
tick pos=left,
%title={Labels: [[2, 6], [1, 3], [7, 0], [9, 4], [5, 8]]},
width=\figwidth,
x grid style={white!69.0196078431373!black},
xlabel={Eval. Steps},
xminorgrids,
xmajorgrids,
xmin=-3.95, xmax=104.95,
xtick style={color=black},
xtick={-25,0,50,100,125},
xticklabels={-25,0,50,100,125},
y grid style={white!69.0196078431373!black},
%ylabel={ACC},
ymajorgrids,
ymin=0.917591079316641, ymax=0.962546271995501,
ytick style={color=black},
ytick={0.89,0.90,0.91,0.92,0.93,0.94,0.95,0.96,0.97},
yticklabels={89,90,91,92,93,94,95,96,97}
]
\addplot [semithick, color0]
table {%
1 0.951650253931681
2 0.940867698192596
3 0.940867698192596
4 0.948078032334645
5 0.948131513595581
6 0.944906004269918
7 0.938454985618591
8 0.938454985618591
9 0.938454985618591
10 0.944964607556661
11 0.948219418525696
12 0.944964607556661
13 0.948219418525696
14 0.948219418525696
15 0.948219418525696
16 0.948219418525696
17 0.948219418525696
18 0.948219418525696
19 0.948219418525696
20 0.948219418525696
21 0.948219418525696
22 0.948219418525696
23 0.948219418525696
24 0.948219418525696
25 0.948219418525696
26 0.948219418525696
27 0.948219418525696
28 0.948219418525696
29 0.948219418525696
30 0.944964607556661
31 0.948219418525696
32 0.948219418525696
33 0.948219418525696
34 0.948219418525696
35 0.948219418525696
36 0.948219418525696
37 0.948219418525696
38 0.948219418525696
39 0.948219418525696
40 0.948219418525696
41 0.948219418525696
42 0.948219418525696
43 0.948219418525696
44 0.948219418525696
45 0.948219418525696
46 0.948219418525696
47 0.944964607556661
48 0.944964607556661
49 0.948219418525696
50 0.948219418525696
51 0.948219418525696
52 0.948219418525696
53 0.948219418525696
54 0.948219418525696
55 0.948219418525696
56 0.948219418525696
57 0.948219418525696
58 0.948219418525696
59 0.948219418525696
60 0.944964607556661
61 0.948219418525696
62 0.948219418525696
63 0.948219418525696
64 0.948219418525696
65 0.948219418525696
66 0.948219418525696
67 0.948219418525696
68 0.948219418525696
69 0.948219418525696
70 0.948219418525696
71 0.948219418525696
72 0.944964607556661
73 0.948219418525696
74 0.948219418525696
75 0.948219418525696
76 0.948219418525696
77 0.948219418525696
78 0.948219418525696
79 0.948219418525696
80 0.948219418525696
81 0.948219418525696
82 0.948219418525696
83 0.948219418525696
84 0.948219418525696
85 0.948219418525696
86 0.948219418525696
87 0.948219418525696
88 0.948219418525696
89 0.948219418525696
90 0.948219418525696
91 0.948219418525696
92 0.948219418525696
93 0.948219418525696
94 0.948219418525696
95 0.948219418525696
96 0.948219418525696
97 0.948219418525696
98 0.948219418525696
99 0.948219418525696
100 0.948219418525696
};
%\addlegendentry{a2c, gae, envs=10, lr=0.0001, act=tanh}
\addplot [semithick, color1]
table {%
1 0.955312685171763
2 0.954510879516601
3 0.950444503625234
4 0.950363171100617
5 0.953102123737335
6 0.952889839808146
7 0.947375639279683
8 0.942846246560415
9 0.953287172317505
10 0.946167325973511
11 0.929952383041382
12 0.95445925394694
13 0.94297255674998
14 0.947870596249898
15 0.942315165201823
16 0.952336267630259
17 0.946282267570496
18 0.94942049185435
19 0.944900278250376
20 0.941963708400726
21 0.944639849662781
22 0.946689927577972
23 0.943459892272949
24 0.940949189662933
25 0.942495286464691
26 0.937013586362203
27 0.944521121184031
28 0.945434542496999
29 0.944001150131226
30 0.946851003170013
31 0.948784677187602
32 0.944281967480977
33 0.94452440738678
34 0.933093698819478
35 0.935895168781281
36 0.91963449716568
37 0.948877461751302
38 0.941479408740997
39 0.953234493732452
40 0.942806319395701
41 0.94194803237915
42 0.940852161248525
43 0.94433708190918
44 0.944278434912364
45 0.937031189600627
46 0.947922241687775
47 0.942180744806925
48 0.947116935253143
49 0.944611314932505
50 0.934834929307302
51 0.937804273764292
52 0.948080412546794
53 0.944634008407593
54 0.945612168312073
55 0.948067510128021
56 0.94510854880015
57 0.946430079142253
58 0.935739835103353
59 0.94833820660909
60 0.950487144788106
61 0.939296062787374
62 0.946170834700267
63 0.95057110786438
64 0.94088382323583
65 0.944602783521016
66 0.941458721955617
67 0.944401093324025
68 0.946956646442413
69 0.942628570397695
70 0.944659399986267
71 0.943167996406555
72 0.946911919116974
73 0.944707687695821
74 0.952685157457987
75 0.942673154671987
76 0.936400751272837
77 0.944943662484487
78 0.943593974908193
79 0.948078417778015
80 0.949201222260793
81 0.94095420440038
82 0.952070224285126
83 0.941591954231262
84 0.951161714394887
85 0.948766322930654
86 0.945157579580943
87 0.945880568027496
88 0.943079562981923
89 0.952671849727631
90 0.945184195041657
91 0.94071048895518
92 0.943040283521016
93 0.945184195041657
94 0.95034921169281
95 0.947856406370799
96 0.950465142726898
97 0.942603389422099
98 0.942355012893677
99 0.953934605916341
100 0.946316822369893
};
%\addlegendentry{DQN, n train envs=30, lr=0.0001}
\addplot [semithick, color2]
table {%
1 0.937166291149101
100 0.937166291149101
};
%\addlegendentry{Random}
\addplot [semithick, color3]
table {%
1 0.953225047502535
100 0.953225047502535
};
%\addlegendentry{ETS}
\addplot [semithick, color4]
table {%
1 0.959211434368453
100 0.959211434368453
};
%\addlegendentry{Heur-GD}
\addplot [semithick, color5, dash pattern=on 1pt off 3pt on 3pt off 3pt]
table {%
1 0.960502854146462
100 0.960502854146462
};
%\addlegendentry{Heur-LD}
\addplot [semithick, color6, dashed]
table {%
1 0.959211434368453
100 0.959211434368453
};
%\addlegendentry{Heur-AT}



\nextgroupplot[title=Seed 16,
height=\figheight,
legend cell align={left},
legend style={
  fill opacity=0.8,
  draw opacity=1,
  text opacity=1,
  at={(0.5,0.09)},
  anchor=south,
  draw=white!80!black
},
minor xtick={25, 75},
minor ytick={},
tick align=outside,
tick pos=left,
%title={Labels: [[6, 2], [0, 7], [8, 4], [3, 1], [5, 9]]},
width=\figwidth,
x grid style={white!69.0196078431373!black},
xlabel={Eval. Steps},
xminorgrids,
xmajorgrids,
xmin=-3.95, xmax=104.95,
xtick style={color=black},
xtick={-25,0,50,100,125},
xticklabels={-25,0,50,100,125},
y grid style={white!69.0196078431373!black},
%ylabel={ACC},
ymajorgrids,
ymin=0.786342887914087, ymax=0.950188479875955,
ytick style={color=black},
ytick={0.78,0.8,0.82,0.84,0.86,0.88,0.9,0.92,0.94,0.96},
yticklabels={78,80,82,84,86,88,90,92,94,96}
]
\addplot [semithick, color0]
table {%
1 0.89561520020167
2 0.842172574996948
3 0.842172574996948
4 0.870251055558522
5 0.886999356746674
6 0.892352298895518
7 0.903058183193207
8 0.903058183193207
9 0.903058183193207
10 0.912424429257711
11 0.921790675322215
12 0.921790675322215
13 0.931156921386719
14 0.931156921386719
15 0.931156921386719
16 0.931156921386719
17 0.921790675322215
18 0.912424429257711
19 0.921790675322215
20 0.931156921386719
21 0.912424429257711
22 0.912424429257711
23 0.921790675322215
24 0.921790675322215
25 0.931156921386719
26 0.921790675322215
27 0.921790675322215
28 0.921790675322215
29 0.912424429257711
30 0.903058183193207
31 0.931156921386719
32 0.912424429257711
33 0.912424429257711
34 0.912424429257711
35 0.912424429257711
36 0.912424429257711
37 0.903058183193207
38 0.903058183193207
39 0.903058183193207
40 0.903058183193207
41 0.903058183193207
42 0.903058183193207
43 0.903058183193207
44 0.903058183193207
45 0.903058183193207
46 0.903058183193207
47 0.903058183193207
48 0.903058183193207
49 0.903058183193207
50 0.903058183193207
51 0.903058183193207
52 0.903058183193207
53 0.903058183193207
54 0.903058183193207
55 0.903058183193207
56 0.903058183193207
57 0.903058183193207
58 0.903058183193207
59 0.903058183193207
60 0.903058183193207
61 0.903058183193207
62 0.903058183193207
63 0.903058183193207
64 0.903058183193207
65 0.903058183193207
66 0.903058183193207
67 0.903058183193207
68 0.903058183193207
69 0.903058183193207
70 0.903058183193207
71 0.903058183193207
72 0.903058183193207
73 0.903058183193207
74 0.903058183193207
75 0.903058183193207
76 0.903058183193207
77 0.903058183193207
78 0.903058183193207
79 0.903058183193207
80 0.903058183193207
81 0.909743805726369
82 0.903058183193207
83 0.909743805726369
84 0.903058183193207
85 0.903058183193207
86 0.903058183193207
87 0.903058183193207
88 0.903058183193207
89 0.903058183193207
90 0.903058183193207
91 0.903058183193207
92 0.903058183193207
93 0.903058183193207
94 0.903058183193207
95 0.903058183193207
96 0.903058183193207
97 0.909743805726369
98 0.909743805726369
99 0.909743805726369
100 0.909743805726369
};
%\addlegendentry{a2c, gae, envs=10, lr=0.0001, act=tanh}
\addplot [semithick, color1]
table {%
1 0.91650405327479
2 0.904043237368266
3 0.913246476650238
4 0.870602770646413
5 0.886044383049011
6 0.922824776172638
7 0.894750980536143
8 0.892096320788066
9 0.922540358702342
10 0.905100329717
11 0.933740385373433
12 0.899058349927266
13 0.889970529079437
14 0.892187794049581
15 0.895830094814301
16 0.90184234380722
17 0.902475380897522
18 0.892988248666128
19 0.915492935975393
20 0.878717788060506
21 0.904989989598592
22 0.914053905010224
23 0.887466553846995
24 0.917130720615387
25 0.922701911131541
26 0.89799503882726
27 0.924551383654276
28 0.921796270211538
29 0.896597564220428
30 0.917688274383545
31 0.931189711888631
32 0.912550703684489
33 0.926023821036021
34 0.921841239929199
35 0.933752107620239
36 0.913284889856974
37 0.930583624045054
38 0.919631548722585
39 0.934001803398132
40 0.923692639668783
41 0.915505905946096
42 0.913579150040944
43 0.923930390675863
44 0.93449072043101
45 0.941302156448364
46 0.925169106324514
47 0.922530794143677
48 0.942740952968597
49 0.929424242178599
50 0.897265724341075
51 0.92692547639211
52 0.919515085220337
53 0.904395190874736
54 0.919797885417938
55 0.903674515088399
56 0.918722422917684
57 0.882863481839498
58 0.90570692618688
59 0.883204686641693
60 0.911331284046173
61 0.89586668809255
62 0.900129838784536
63 0.880136934916178
64 0.885089035828908
65 0.875284890333811
66 0.885857550303141
67 0.894540385405223
68 0.883335530757904
69 0.905438296000163
70 0.903595383961995
71 0.890290224552155
72 0.902651099363963
73 0.916800002257029
74 0.913987970352173
75 0.892326617240906
76 0.903872009118398
77 0.912424445152283
78 0.893903156121572
79 0.896357595920563
80 0.901084935665131
81 0.89387929836909
82 0.909589604536692
83 0.91767992178599
84 0.910073781013489
85 0.899599913756053
86 0.897598250706991
87 0.915892783800761
88 0.898820344607035
89 0.918456892172495
90 0.93744740486145
91 0.923053538799286
92 0.931306978066762
93 0.930489540100098
94 0.898673872152964
95 0.888887651761373
96 0.920562887191772
97 0.911262706915537
98 0.910187097390493
99 0.899929853280385
100 0.93744740486145
};
%\addlegendentry{DQN, n train envs=30, lr=0.0001}
\addplot [semithick, color2]
table {%
1 0.8366389860345
100 0.8366389860345
};
%\addlegendentry{Random}
\addplot [semithick, color3]
table {%
1 0.793790414821445
100 0.793790414821445
};
%\addlegendentry{ETS}
\addplot [semithick, color4]
table {%
1 0.911572701428831
100 0.911572701428831
};
%\addlegendentry{Heur-GD}
\addplot [semithick, color5, dash pattern=on 1pt off 3pt on 3pt off 3pt]
table {%
1 0.911572701428831
100 0.911572701428831
};
%\addlegendentry{Heur-LD}
\addplot [semithick, color6, dashed]
table {%
1 0.911572701428831
100 0.911572701428831
};
%\addlegendentry{Heur-AT}



\nextgroupplot[title=Seed 17,
height=\figheight,
legend cell align={left},
legend style={
  fill opacity=0.8,
  draw opacity=1,
  text opacity=1,
  at={(0.97,0.03)},
  anchor=south east,
  draw=white!80!black
},
minor xtick={25, 75},
minor ytick={},
tick align=outside,
tick pos=left,
%title={Labels: [[7, 2], [5, 3], [4, 0], [9, 8], [6, 1]]},
width=\figwidth,
x grid style={white!69.0196078431373!black},
xlabel={Eval. Steps},
xminorgrids,
xmajorgrids,
xmin=-3.95, xmax=104.95,
xtick style={color=black},
xtick={-25,0,50,100,125},
xticklabels={-25,0,50,100,125},
y grid style={white!69.0196078431373!black},
%ylabel={ACC},
ymajorgrids,
ymin=0.904683085481326, ymax=0.96807973364989,
ytick style={color=black},
ytick={0.9,0.91,0.92,0.93,0.94,0.95,0.96,0.97},
yticklabels={90,91,92,93,94,95,96,97}
]
\addplot [semithick, color0]
table {%
	1 0.925646488666535
	2 0.933946962356567
	3 0.918928203582764
	4 0.953521945476532
	5 0.950605626106262
	6 0.933354690074921
	7 0.923654346466064
	8 0.948091323375702
	9 0.936637659072876
	10 0.933624348640442
	11 0.924818322658539
	12 0.937614045143127
	13 0.934010100364685
	14 0.94721786737442
	15 0.923204612731934
	16 0.945419938564301
	17 0.936714193820953
	18 0.931443445682526
	19 0.93220050573349
	20 0.939460294246674
	21 0.93577828168869
	22 0.944345276355744
	23 0.932533686161041
	24 0.948526587486267
	25 0.946431591510773
	26 0.932007641792297
	27 0.93585037946701
	28 0.923234033584595
	29 0.95537572145462
	30 0.947152307033539
	31 0.951709620952606
	32 0.950887589454651
	33 0.94084244966507
	34 0.964170005321503
	35 0.937397184371948
	36 0.933455040454864
	37 0.939046754837036
	38 0.941606221199036
	39 0.940591585636139
	40 0.933696081638336
	41 0.937507472038269
	42 0.938939354419708
	43 0.939813299179077
	44 0.938939354419708
	45 0.946320216655731
	46 0.943472123146057
	47 0.945719342231751
	48 0.944556543827057
	49 0.932703199386597
	50 0.941797780990601
	51 0.929217126369476
	52 0.925934135913849
	53 0.945016052722931
	54 0.951934041976929
	55 0.952278971672058
	56 0.91905175447464
	57 0.943824701309204
	58 0.942804131507874
	59 0.9330486369133
	60 0.942298591136932
	61 0.927589254379272
	62 0.937228734493255
	63 0.94397191286087
	64 0.939463796615601
	65 0.946885869503021
	66 0.930947642326355
	67 0.939136025905609
	68 0.921112067699432
	69 0.924169294834137
	70 0.9234294962883
	71 0.936647589206695
	72 0.934901647567749
	73 0.943510844707489
	74 0.923181402683258
	75 0.905027749538422
	76 0.94205638885498
	77 0.94766884803772
	78 0.940032765865326
	79 0.951093370914459
	80 0.919179995059967
	81 0.941371216773987
	82 0.945668501853943
	83 0.940956971645355
	84 0.939194812774658
	85 0.934193849563599
	86 0.934940989017486
	87 0.933694803714752
	88 0.93156135559082
	89 0.939906461238861
	90 0.929434428215027
	91 0.921583020687103
	92 0.919113039970398
	93 0.928214464187622
	94 0.94719854593277
	95 0.932164080142975
	96 0.939837822914124
	97 0.950224962234497
	98 0.927816519737244
	99 0.920654098987579
	100 0.946744928359985
};
%\addlegendentry{DQN}
\addplot [semithick, color1]
table {%
	1 0.927530746459961
	2 0.913990163803101
	3 0.913990163803101
	4 0.932377617359161
	5 0.964568865299225
	6 0.963869073390961
	7 0.961069905757904
	8 0.961069905757904
	9 0.961069905757904
	10 0.948192956447601
	11 0.935316007137299
	12 0.94175448179245
	13 0.928877532482147
	14 0.928877532482147
	15 0.928877532482147
	16 0.928877532482147
	17 0.935316007137299
	18 0.935316007137299
	19 0.935316007137299
	20 0.928877532482147
	21 0.94175448179245
	22 0.94175448179245
	23 0.94175448179245
	24 0.94175448179245
	25 0.928877532482147
	26 0.94175448179245
	27 0.94175448179245
	28 0.935316007137299
	29 0.94175448179245
	30 0.948192956447601
	31 0.928877532482147
	32 0.94175448179245
	33 0.94175448179245
	34 0.94175448179245
	35 0.948192956447601
	36 0.94175448179245
	37 0.954631431102753
	38 0.954631431102753
	39 0.954631431102753
	40 0.954631431102753
	41 0.961069905757904
	42 0.954631431102753
	43 0.954631431102753
	44 0.954631431102753
	45 0.961069905757904
	46 0.954631431102753
	47 0.961069905757904
	48 0.954631431102753
	49 0.961069905757904
	50 0.961069905757904
	51 0.961069905757904
	52 0.961069905757904
	53 0.961069905757904
	54 0.961069905757904
	55 0.961069905757904
	56 0.961069905757904
	57 0.961069905757904
	58 0.961069905757904
	59 0.961069905757904
	60 0.961069905757904
	61 0.954631431102753
	62 0.961069905757904
	63 0.954631431102753
	64 0.961069905757904
	65 0.961069905757904
	66 0.961069905757904
	67 0.961069905757904
	68 0.961069905757904
	69 0.961069905757904
	70 0.954631431102753
	71 0.961069905757904
	72 0.961947584152222
	73 0.961069905757904
	74 0.948192956447601
	75 0.954631431102753
	76 0.961069905757904
	77 0.961069905757904
	78 0.961947584152222
	79 0.961069905757904
	80 0.961947584152222
	81 0.961947584152222
	82 0.961069905757904
	83 0.961947584152222
	84 0.948192956447601
	85 0.961947584152222
	86 0.961069905757904
	87 0.961069905757904
	88 0.954631431102753
	89 0.961947584152222
	90 0.961069905757904
	91 0.961069905757904
	92 0.961947584152222
	93 0.954631431102753
	94 0.961947584152222
	95 0.961947584152222
	96 0.961947584152222
	97 0.95550910949707
	98 0.95550910949707
	99 0.963702940940857
	100 0.962825262546539
};
%\addlegendentry{A2C}
\addplot [semithick, color2]
table {%
	1 0.932587051391602
	2 0.932587051391602
	3 0.932587051391602
	4 0.957632751464844
	5 0.949594004154205
	6 0.957228562831879
	7 0.966377611160278
	8 0.959233660697937
	9 0.957228562831879
	10 0.960530476570129
	11 0.957228562831879
	12 0.949964604377747
	13 0.940426235198975
	14 0.957649013996124
	15 0.965764093399048
	16 0.949964604377747
	17 0.958930151462555
	18 0.960229139328003
	19 0.956248672008514
	20 0.965513854026794
	21 0.950452921390533
	22 0.957356672286987
	23 0.960093796253204
	24 0.965061016082764
	25 0.964821667671204
	26 0.960644598007202
	27 0.960805094242096
	28 0.968509438037872
	29 0.950871255397797
	30 0.953344149589539
	31 0.964376595020294
	32 0.96689204454422
	33 0.966726760864258
	34 0.949022362232208
	35 0.966001296043396
	36 0.959935886859894
	37 0.96864021062851
	38 0.967538607120514
	39 0.960839281082153
	40 0.966936988830567
	41 0.957383751869202
	42 0.967538607120514
	43 0.966679439544678
	44 0.966936988830567
	45 0.966359839439392
	46 0.966936988830567
	47 0.96696145772934
	48 0.966936988830567
	49 0.966936988830567
	50 0.966936988830567
	51 0.966936988830567
	52 0.966936988830567
	53 0.966936988830567
	54 0.966936988830567
	55 0.967538607120514
	56 0.966936988830567
	57 0.95923264503479
	58 0.966361033916473
	59 0.966936988830567
	60 0.966359839439392
	61 0.966359839439392
	62 0.966359839439392
	63 0.96696145772934
	64 0.966359839439392
	65 0.966359839439392
	66 0.965783884525299
	67 0.965783884525299
	68 0.965783884525299
	69 0.966359839439392
	70 0.966359839439392
	71 0.966359839439392
	72 0.966359839439392
	73 0.963273327350617
	74 0.964977233409882
	75 0.964977233409882
	76 0.966102290153503
	77 0.966972692012787
	78 0.958655495643616
	79 0.966359839439392
	80 0.967230241298676
	81 0.967241468429565
	82 0.966948215961456
	83 0.964437658786774
	84 0.966397414207459
	85 0.966397414207459
	86 0.963816230297089
	87 0.96781861782074
	88 0.966116411685944
	89 0.966116411685944
	90 0.965308060646057
	91 0.965565609931946
	92 0.965565609931946
	93 0.965565609931946
	94 0.964549260139465
	95 0.965216345787048
	96 0.965565609931946
	97 0.965014808177948
	98 0.964602766036987
	99 0.965216345787048
	100 0.965216345787048
};
%\addlegendentry{SAC}
\addplot [semithick, color3]
table {%
	1 0.958628828739618
	100 0.958628828739618
};
%\addlegendentry{Random}
\addplot [semithick, color4]
table {%
	1 0.956852431378258
	100 0.956852431378258
};
%\addlegendentry{ETS}
\addplot [semithick, color5]
table {%
	1 0.934796203829509
	100 0.934796203829509
};
%\addlegendentry{Heur-GD}
\addplot [semithick, color6, dash pattern=on 1pt off 3pt on 3pt off 3pt]
table {%
	1 0.934796203829509
	100 0.934796203829509
};
%\addlegendentry{Heur-LD}
\addplot [semithick, white!49.8039215686275!black, dashed]
table {%
	1 0.934796203829509
	100 0.934796203829509
};
%\addlegendentry{Heur-AT}



\nextgroupplot[title=Seed 18,
height=\figheight,
legend cell align={left},
legend style={
  fill opacity=0.8,
  draw opacity=1,
  text opacity=1,
  at={(0.97,0.03)},
  anchor=south east,
  draw=white!80!black
},
minor xtick={25, 75},
minor ytick={},
tick align=outside,
tick pos=left,
%title={Labels: [[7, 9], [0, 4], [2, 1], [6, 5], [8, 3]]},
width=\figwidth,
x grid style={white!69.0196078431373!black},
xlabel={Eval. Steps},
xminorgrids,
xmajorgrids,
xmin=-3.95, xmax=104.95,
xtick style={color=black},
xtick={-25,0,50,100,125},
xticklabels={-25,0,50,100,125},
y grid style={white!69.0196078431373!black},
ylabel={ACC (\%)},
ymajorgrids,
ymin=0.830549303416718, ymax=0.938523592342451,
ytick style={color=black},
ytick={0.82,0.84,0.86,0.88,0.9,0.92,0.94},
yticklabels={82,84,86,88,90,92,94}
]
\addplot [semithick, color0]
table {%
	1 0.857881555557251
	2 0.862911217212677
	3 0.880801177024841
	4 0.852274785041809
	5 0.839353015422821
	6 0.886810147762299
	7 0.868168838024139
	8 0.890694448947907
	9 0.85710554599762
	10 0.882744507789612
	11 0.892343473434448
	12 0.890326836109161
	13 0.880862910747528
	14 0.887716913223267
	15 0.889768481254578
	16 0.896898374557495
	17 0.885303859710693
	18 0.8976713347435
	19 0.887434437274933
	20 0.888865554332733
	21 0.889848833084106
	22 0.875329923629761
	23 0.89411866903305
	24 0.883421454429626
	25 0.890048928260803
	26 0.883032774925232
	27 0.891479640007019
	28 0.886578013896942
	29 0.897927131652832
	30 0.882720291614532
	31 0.883947806358337
	32 0.884861583709717
	33 0.888930234909058
	34 0.883521864414215
	35 0.898145618438721
	36 0.890077469348907
	37 0.892419769763947
	38 0.88220472574234
	39 0.883382301330566
	40 0.890429756641388
	41 0.896969592571259
	42 0.895179805755615
	43 0.887995812892914
	44 0.891491417884827
	45 0.894793207645416
	46 0.894793207645416
	47 0.90239488363266
	48 0.892253420352936
	49 0.895408749580383
	50 0.894540975093842
	51 0.90491996049881
	52 0.901241390705109
	53 0.898203792572022
	54 0.892136952877045
	55 0.894898736476898
	56 0.897854144573212
	57 0.903333587646484
	58 0.900333921909332
	59 0.901387317180633
	60 0.899059288501739
	61 0.894965295791626
	62 0.896144385337829
	63 0.890166785717011
	64 0.895818769931793
	65 0.880279972553253
	66 0.900819313526154
	67 0.886906223297119
	68 0.893759305477142
	69 0.894183056354523
	70 0.897377274036407
	71 0.891073155403137
	72 0.886168007850647
	73 0.891984388828278
	74 0.890592336654663
	75 0.899135134220123
	76 0.897326889038086
	77 0.897774376869202
	78 0.899057879447937
	79 0.899667007923126
	80 0.903986179828644
	81 0.900342876911163
	82 0.899143760204315
	83 0.896699621677399
	84 0.902075884342194
	85 0.898507866859436
	86 0.899143760204315
	87 0.900952005386353
	88 0.894040162563324
	89 0.89722286939621
	90 0.887207639217377
	91 0.894272327423096
	92 0.894238612651825
	93 0.899135134220123
	94 0.899135134220123
	95 0.898610391616821
	96 0.89762193441391
	97 0.899135134220123
	98 0.89811182975769
	99 0.900857961177826
	100 0.900520915985108
};
%\addlegendentry{DQN}
\addplot [semithick, color1]
table {%
	1 0.873820266723633
	2 0.913773119449615
	3 0.913773119449615
	4 0.884403381347656
	5 0.896446752548218
	6 0.894138422012329
	7 0.884905099868774
	8 0.884905099868774
	9 0.884905099868774
	10 0.891209287643433
	11 0.910121850967407
	12 0.903817663192749
	13 0.916426038742065
	14 0.916426038742065
	15 0.916426038742065
	16 0.916426038742065
	17 0.910121850967407
	18 0.903817663192749
	19 0.910121850967407
	20 0.916426038742065
	21 0.903817663192749
	22 0.903817663192749
	23 0.903817663192749
	24 0.903817663192749
	25 0.916426038742065
	26 0.903817663192749
	27 0.903817663192749
	28 0.910121850967407
	29 0.910121850967407
	30 0.897513475418091
	31 0.916426038742065
	32 0.903817663192749
	33 0.903817663192749
	34 0.903817663192749
	35 0.903817663192749
	36 0.910121850967407
	37 0.897513475418091
	38 0.897513475418091
	39 0.891209287643433
	40 0.897513475418091
	41 0.884905099868774
	42 0.903817663192749
	43 0.897513475418091
	44 0.897513475418091
	45 0.884905099868774
	46 0.891209287643433
	47 0.884905099868774
	48 0.897513475418091
	49 0.903817663192749
	50 0.891209287643433
	51 0.891209287643433
	52 0.910121850967407
	53 0.891209287643433
	54 0.903817663192749
	55 0.891209287643433
	56 0.891209287643433
	57 0.903817663192749
	58 0.897513475418091
	59 0.903817663192749
	60 0.903817663192749
	61 0.910121850967407
	62 0.897513475418091
	63 0.897513475418091
	64 0.903817663192749
	65 0.903817663192749
	66 0.910121850967407
	67 0.910121850967407
	68 0.891209287643433
	69 0.897513475418091
	70 0.903817663192749
	71 0.903817663192749
	72 0.903817663192749
	73 0.903817663192749
	74 0.903817663192749
	75 0.900945055484772
	76 0.910121850967407
	77 0.896333913803101
	78 0.897513475418091
	79 0.896333913803101
	80 0.903817663192749
	81 0.903817663192749
	82 0.888336679935455
	83 0.916426038742065
	84 0.916426038742065
	85 0.894640867710114
	86 0.903817663192749
	87 0.900945055484772
	88 0.900945055484772
	89 0.898072447776795
	90 0.916426038742065
	91 0.910121850967407
	92 0.900945055484772
	93 0.90724924325943
	94 0.916426038742065
	95 0.90724924325943
	96 0.90724924325943
	97 0.893461306095123
	98 0.910121850967407
	99 0.891768260002136
	100 0.916426038742065
};
%\addlegendentry{A2C}
\addplot [semithick, color2]
table {%
	1 0.773968613147736
	2 0.773968613147736
	3 0.773968613147736
	4 0.882205860614777
	5 0.887453289031982
	6 0.903275337219238
	7 0.898865914344788
	8 0.905354263782501
	9 0.905354263782501
	10 0.898710308074951
	11 0.903467628955841
	12 0.903467628955841
	13 0.900752549171448
	14 0.902639183998108
	15 0.902639183998108
	16 0.899133808612823
	17 0.886604671478272
	18 0.880059134960175
	19 0.896379916667938
	20 0.898428559303284
	21 0.896379916667938
	22 0.896379916667938
	23 0.892702353000641
	24 0.894160566329956
	25 0.896047201156616
	26 0.894160566329956
	27 0.896047201156616
	28 0.894160566329956
	29 0.893136837482452
	30 0.887901804447174
	31 0.902662827968597
	32 0.894160566329956
	33 0.904549462795257
	34 0.900845956802368
	35 0.896047201156616
	36 0.900845956802368
	37 0.885741250514984
	38 0.874719784259796
	39 0.889282548427582
	40 0.895565459728241
	41 0.895769443511963
	42 0.885519645214081
	43 0.895769443511963
	44 0.887847130298614
	45 0.88154613494873
	46 0.892795343399048
	47 0.878572034835815
	48 0.881541841030121
	49 0.88154613494873
	50 0.884457972049713
	51 0.886581473350525
	52 0.886785457134247
	53 0.886581473350525
	54 0.877084100246429
	55 0.869154098033905
	56 0.877288084030151
	57 0.87341983795166
	58 0.87341983795166
	59 0.876407561302185
	60 0.879485287666321
	61 0.881037011146545
	62 0.882808203697204
	63 0.884181451797485
	64 0.875712018013
	65 0.874284060001373
	66 0.879026756286621
	67 0.877140121459961
	68 0.884266078472137
	69 0.884266078472137
	70 0.882923645973206
	71 0.879455516338348
	72 0.880800144672394
	73 0.880082778930664
	74 0.88090665102005
	75 0.896610949039459
	76 0.889513597488403
	77 0.86997421503067
	78 0.884274275302887
	79 0.880800144672394
	80 0.889582562446594
	81 0.876331751346588
	82 0.871047737598419
	83 0.873487391471863
	84 0.887424786090851
	85 0.865461130142212
	86 0.885928812026978
	87 0.878377678394318
	88 0.871484293937683
	89 0.860263788700104
	90 0.886209063529968
	91 0.867418384552002
	92 0.884487280845642
	93 0.884736850261688
	94 0.874537386894226
	95 0.882988181114197
	96 0.870961918830872
	97 0.87908091545105
	98 0.883503904342651
	99 0.878725159168243
	100 0.872657706737518
};
%\addlegendentry{SAC}
\addplot [semithick, color3]
table {%
	1 0.912964921095126
	100 0.912964921095126
};
%\addlegendentry{Random}
\addplot [semithick, color4]
table {%
	1 0.928894772109996
	100 0.928894772109996
};
%\addlegendentry{ETS}
\addplot [semithick, color5]
table {%
	1 0.878206432691232
	100 0.878206432691232
};
%\addlegendentry{Heur-GD}
\addplot [semithick, color6, dash pattern=on 1pt off 3pt on 3pt off 3pt]
table {%
	1 0.878206432691232
	100 0.878206432691232
};
%\addlegendentry{Heur-LD}
\addplot [semithick, white!49.8039215686275!black, dashed]
table {%
	1 0.878206432691232
	100 0.878206432691232
};
%\addlegendentry{Heur-AT}



\nextgroupplot[title=Seed 19,
height=\figheight,
legend cell align={left},
legend style={
  fill opacity=0.8,
  draw opacity=1,
  text opacity=1,
  at={(0.5,0.91)},
  anchor=north,
  draw=white!80!black
},
minor xtick={25, 75},
minor ytick={},
tick align=outside,
tick pos=left,
%title={Labels: [[1, 7], [9, 6], [8, 4], [3, 0], [2, 5]]},
width=\figwidth,
x grid style={white!69.0196078431373!black},
xlabel={Eval. Steps},
xminorgrids,
xmajorgrids,
xmin=-3.95, xmax=104.95,
xtick style={color=black},
xtick={-25,0,50,100,125},
xticklabels={-25,0,50,100,125},
y grid style={white!69.0196078431373!black},
%ylabel={ACC},
ymajorgrids,
ymin=0.851598749903499, ymax=0.981,%0.979828229247698,
ytick style={color=black},
ytick={0.82,0.84,0.86,0.88,0.9,0.92,0.94,0.96,0.98},
yticklabels={82,84,86,88,90,92,94,96,98}
]
\addplot [semithick, color0]
table {%
	1 0.891900000572205
	2 0.870728073120117
	3 0.887302560806275
	4 0.891249873638153
	5 0.890086531639099
	6 0.899092857837677
	7 0.882806446552277
	8 0.873749299049378
	9 0.904258868694306
	10 0.894001138210297
	11 0.899233901500702
	12 0.899340629577637
	13 0.897505233287811
	14 0.876691434383392
	15 0.884888064861298
	16 0.887577679157257
	17 0.893109476566315
	18 0.884327914714813
	19 0.887249743938446
	20 0.896651532649994
	21 0.887953219413757
	22 0.890935604572296
	23 0.866674072742462
	24 0.866147949695587
	25 0.867637567520142
	26 0.886910104751587
	27 0.879507422447205
	28 0.884166300296783
	29 0.877913863658905
	30 0.88883780002594
	31 0.890576314926147
	32 0.876163997650147
	33 0.883837101459503
	34 0.854316961765289
	35 0.868526153564453
	36 0.86820508480072
	37 0.888266439437866
	38 0.873623299598694
	39 0.876741054058075
	40 0.890799646377564
	41 0.868758587837219
	42 0.871008818149567
	43 0.86053964138031
	44 0.869284734725952
	45 0.866789548397064
	46 0.871869289875031
	47 0.872515361309052
	48 0.88690370798111
	49 0.88885507106781
	50 0.881400172710419
	51 0.873593800067902
	52 0.86505907535553
	53 0.874167947769165
	54 0.869896783828735
	55 0.899356143474579
	56 0.888880620002747
	57 0.894088606834412
	58 0.882235479354858
	59 0.889430191516876
	60 0.906564958095551
	61 0.866823582649231
	62 0.86843896150589
	63 0.891059186458588
	64 0.887329840660095
	65 0.899667947292328
	66 0.893594002723694
	67 0.898137345314026
	68 0.894892835617065
	69 0.886050415039062
	70 0.870453736782074
	71 0.883400464057922
	72 0.909472041130066
	73 0.902736728191376
	74 0.88432245016098
	75 0.882327282428741
	76 0.880106899738312
	77 0.896692199707031
	78 0.889570889472961
	79 0.904215469360352
	80 0.892619745731354
	81 0.890838108062744
	82 0.896593768596649
	83 0.889551441669464
	84 0.902189273834228
	85 0.894054191112518
	86 0.867003726959229
	87 0.897551684379578
	88 0.885523025989533
	89 0.893696343898773
	90 0.900682573318482
	91 0.887681894302368
	92 0.888847486972809
	93 0.891881983280182
	94 0.893643395900726
	95 0.89449863910675
	96 0.892744927406311
	97 0.893643395900726
	98 0.886455552577972
	99 0.874034440517426
	100 0.890434644222259
};
%\addlegendentry{DQN}
\addplot [semithick, color1]
table {%
	1 0.892099900245666
	2 0.919530880451202
	3 0.919530880451202
	4 0.907443795204163
	5 0.89366180896759
	6 0.887145669460297
	7 0.861081111431122
	8 0.861081111431122
	9 0.861081111431122
	10 0.871089012622833
	11 0.881096913814545
	12 0.876092963218689
	13 0.8861008644104
	14 0.8861008644104
	15 0.8861008644104
	16 0.8861008644104
	17 0.881096913814545
	18 0.876092963218689
	19 0.881096913814545
	20 0.8861008644104
	21 0.876092963218689
	22 0.871089012622833
	23 0.876092963218689
	24 0.876092963218689
	25 0.8861008644104
	26 0.876092963218689
	27 0.876092963218689
	28 0.881096913814545
	29 0.876092963218689
	30 0.871089012622833
	31 0.8861008644104
	32 0.866085062026978
	33 0.871089012622833
	34 0.871089012622833
	35 0.866085062026978
	36 0.871089012622833
	37 0.866085062026978
	38 0.866085062026978
	39 0.866085062026978
	40 0.861081111431122
	41 0.861081111431122
	42 0.861081111431122
	43 0.861081111431122
	44 0.861081111431122
	45 0.861081111431122
	46 0.866085062026978
	47 0.861081111431122
	48 0.861081111431122
	49 0.861081111431122
	50 0.861081111431122
	51 0.861081111431122
	52 0.861081111431122
	53 0.861081111431122
	54 0.861081111431122
	55 0.861081111431122
	56 0.861081111431122
	57 0.861081111431122
	58 0.861081111431122
	59 0.861081111431122
	60 0.861081111431122
	61 0.861081111431122
	62 0.861081111431122
	63 0.861081111431122
	64 0.861081111431122
	65 0.861081111431122
	66 0.861081111431122
	67 0.861081111431122
	68 0.861081111431122
	69 0.861081111431122
	70 0.861081111431122
	71 0.861081111431122
	72 0.861081111431122
	73 0.861081111431122
	74 0.861081111431122
	75 0.861081111431122
	76 0.861081111431122
	77 0.861081111431122
	78 0.861081111431122
	79 0.861081111431122
	80 0.861081111431122
	81 0.861081111431122
	82 0.861081111431122
	83 0.868747181892395
	84 0.868747181892395
	85 0.868747181892395
	86 0.861081111431122
	87 0.861081111431122
	88 0.861081111431122
	89 0.861081111431122
	90 0.861081111431122
	91 0.861081111431122
	92 0.868747181892395
	93 0.861081111431122
	94 0.868747181892395
	95 0.868747181892395
	96 0.868747181892395
	97 0.868747181892395
	98 0.876413252353668
	99 0.868747181892395
	100 0.876413252353668
};
%\addlegendentry{A2C}
\addplot [semithick, color2]
table {%
	1 0.853549993038177
	2 0.853549993038177
	3 0.853549993038177
	4 0.885178153514862
	5 0.869715263843536
	6 0.863489193916321
	7 0.870485198497772
	8 0.875043992996216
	9 0.865453994274139
	10 0.872058234214783
	11 0.865453994274139
	12 0.865453994274139
	13 0.870397670269012
	14 0.865453994274139
	15 0.865838875770569
	16 0.8751890873909
	17 0.872378726005554
	18 0.872378726005554
	19 0.866147243976593
	20 0.872058234214783
	21 0.867338578701019
	22 0.867338578701019
	23 0.86659437417984
	24 0.86659437417984
	25 0.86659437417984
	26 0.866255724430084
	27 0.86659437417984
	28 0.873519105911255
	29 0.873249568939209
	30 0.861948051452637
	31 0.86659437417984
	32 0.86659437417984
	33 0.862692255973816
	34 0.871090919971466
	35 0.871199400424957
	36 0.867338578701019
	37 0.866255724430084
	38 0.873180456161499
	39 0.882511873245239
	40 0.878124132156372
	41 0.878124132156372
	42 0.873180456161499
	43 0.878462781906128
	44 0.873519105911255
	45 0.878124132156372
	46 0.868207812309265
	47 0.869404976367951
	48 0.868999421596527
	49 0.878462781906128
	50 0.873604447841644
	51 0.874687302112579
	52 0.8739430975914
	53 0.8739430975914
	54 0.876800272464752
	55 0.875945796966553
	56 0.875945796966553
	57 0.885747807025909
	58 0.868999421596527
	59 0.88046548128128
	60 0.886385252475739
	61 0.881548335552215
	62 0.885747807025909
	63 0.885747807025909
	64 0.873519105911255
	65 0.880804131031036
	66 0.881441576480865
	67 0.880804131031036
	68 0.885747807025909
	69 0.881548335552215
	70 0.890904936790466
	71 0.881441576480865
	72 0.880804131031036
	73 0.873265798091888
	74 0.880889472961426
	75 0.876244592666626
	76 0.880804131031036
	77 0.880804131031036
	78 0.880804131031036
	79 0.885747807025909
	80 0.884395415782928
	81 0.877608728408813
	82 0.882128412723541
	83 0.885939948558807
	84 0.883519718647003
	85 0.887072088718414
	86 0.880382609367371
	87 0.877608728408813
	88 0.877355420589447
	89 0.877270078659058
	90 0.875524275302887
	91 0.875524275302887
	92 0.87275039434433
	93 0.872411744594574
	94 0.875185625553131
	95 0.877994508743286
	96 0.87275039434433
	97 0.87275039434433
	98 0.875524275302887
	99 0.875524275302887
	100 0.875524275302887
};
%\addlegendentry{SAC}
\addplot [semithick, color3]
table {%
	1 0.958486085985684
	100 0.958486085985684
};
%\addlegendentry{Random}
\addplot [semithick, color4]
table {%
	1 0.973999616550235
	100 0.973999616550235
};
%\addlegendentry{ETS}
\addplot [semithick, color5]
table {%
	1 0.883410454122764
	100 0.883410454122764
};
%\addlegendentry{Heur-GD}
\addplot [semithick, color6, dash pattern=on 1pt off 3pt on 3pt off 3pt]
table {%
	1 0.883410454122764
	100 0.883410454122764
};
%\addlegendentry{Heur-LD}
\addplot [semithick, white!49.8039215686275!black, dashed]
table {%
	1 0.883410454122764
	100 0.883410454122764
};
%\addlegendentry{Heur-AT}



\end{groupplot}

\end{tikzpicture}
  \vspace{-3mm}
  \caption{Performance progress measured in ACC (\%) for all methods in the 10 test environments for {\bf Split MNIST} in the New Task Orders experiment. We plot the performance progress for A2C and DQN for 100 evaluation steps equidistantly distributed over the training episodes. The performance for the Random, ETS, and Heuristic scheduling baselines are plotted as straight lines.  }
  \label{fig:policy_rewards_mnist_paperD}
  \vspace{-3mm}
\end{figure}


\begin{figure}[t]
  \centering
  \setlength{\figwidth}{0.26\textwidth}
  \setlength{\figheight}{.14\textheight}
  



\pgfplotsset{every axis title/.append style={at={(0.5,0.85)}}}
\pgfplotsset{every tick label/.append style={font=\scriptsize}}
\pgfplotsset{every major tick/.append style={major tick length=2pt}}
\pgfplotsset{every minor tick/.append style={minor tick length=1pt}}
\pgfplotsset{every axis x label/.append style={at={(0.5,-0.22)}}}
\pgfplotsset{every axis y label/.append style={at={(-0.30,0.5)}}}
\begin{tikzpicture}
\tikzstyle{every node}=[font=\scriptsize]
\definecolor{color0}{rgb}{0.12156862745098,0.466666666666667,0.705882352941177}
\definecolor{color1}{rgb}{1,0.498039215686275,0.0549019607843137}
\definecolor{color2}{rgb}{0.172549019607843,0.627450980392157,0.172549019607843}
\definecolor{color3}{rgb}{0.83921568627451,0.152941176470588,0.156862745098039}
\definecolor{color4}{rgb}{0.580392156862745,0.403921568627451,0.741176470588235}
\definecolor{color5}{rgb}{0.549019607843137,0.337254901960784,0.294117647058824}
\definecolor{color6}{rgb}{0.890196078431372,0.466666666666667,0.76078431372549}

\begin{groupplot}[group style={group size= 4 by 3, horizontal sep=1.00cm, vertical sep=1.20cm}]

\nextgroupplot[title=Seed 10,
height=\figheight,
legend cell align={left},
legend columns=-1,
legend style={
  fill opacity=0.8,
  draw opacity=1,
  text opacity=1,
  at={(-0.20,1.36)},%at={(0.10,1.29)},
  anchor=south west,
  draw=white!80!black
},
minor xtick={25, 75},
minor ytick={},
tick align=outside,
tick pos=left,
%title={Labels: [[8, 2], [5, 6], [3, 1], [0, 7], [4, 9]]},
width=\figwidth,
x grid style={white!69.0196078431373!black},
xlabel={Eval. Steps},
xminorgrids,
xmajorgrids,
xmin=-3.95, xmax=104.95,
xtick style={color=black},
xtick={-25,0,50,100,125},
xticklabels={-25,0,50,100,125},
y grid style={white!69.0196078431373!black},
ylabel={ACC (\%)},
ymajorgrids,
ymin=0.83, ymax=0.89,
ytick style={color=black},
ytick={0.83,0.84,0.85,0.86,0.87,0.88,0.89},
yticklabels={83,84,85,86,87,88,89}
]
\addplot [semithick, color0]
table {%
1 0.862333333492279
2 0.887299990653992
3 0.860200003782908
4 0.847733334700267
5 0.844633336861928
6 0.846200001239776
7 0.846200001239776
8 0.845266668001811
9 0.84663333495458
10 0.845500004291534
11 0.845500004291534
12 0.845266668001811
13 0.845500004291534
14 0.847800000508626
15 0.845500004291534
16 0.845500004291534
17 0.845500004291534
18 0.845500004291534
19 0.845500004291534
20 0.845500004291534
21 0.845500004291534
22 0.845500004291534
23 0.845500004291534
24 0.845500004291534
25 0.845500004291534
26 0.845500004291534
27 0.845500004291534
28 0.845500004291534
29 0.850333340962728
30 0.845500004291534
31 0.845500004291534
32 0.845500004291534
33 0.845500004291534
34 0.850333340962728
35 0.850333340962728
36 0.845500004291534
37 0.845500004291534
38 0.845500004291534
39 0.850333340962728
40 0.845500004291534
41 0.850333340962728
42 0.845500004291534
43 0.845500004291534
44 0.845500004291534
45 0.850333340962728
46 0.845500004291534
47 0.845500004291534
48 0.850333340962728
49 0.845500004291534
50 0.850333340962728
51 0.845500004291534
52 0.845500004291534
53 0.850333340962728
54 0.854500007629394
55 0.845500004291534
56 0.845500004291534
57 0.845500004291534
58 0.850333340962728
59 0.850333340962728
60 0.845500004291534
61 0.845500004291534
62 0.845500004291534
63 0.845500004291534
64 0.854500007629395
65 0.850333340962728
66 0.845500004291534
67 0.845500004291534
68 0.845500004291534
69 0.845500004291534
70 0.845500004291534
71 0.850333340962728
72 0.850333340962728
73 0.850333340962728
74 0.845500004291534
75 0.850333340962728
76 0.845500004291534
77 0.850333340962728
78 0.845500004291534
79 0.850333340962728
80 0.845500004291534
81 0.853833337624868
82 0.845500004291534
83 0.855166677633921
84 0.859333344300588
85 0.845500004291534
86 0.845500004291534
87 0.854500007629395
88 0.850333340962728
89 0.850333340962728
90 0.850333340962728
91 0.850333340962728
92 0.855166677633921
93 0.849666670958201
94 0.845500004291534
95 0.840733333428701
96 0.840966669718424
97 0.845266668001811
98 0.854500007629395
99 0.854500007629394
100 0.845500004291534
};
\addlegendentry{A2C}
\addplot [semithick, color1]
table {%
1 0.864133334159851
2 0.855000003178914
3 0.849366660912832
4 0.846766670544942
5 0.863299997647603
6 0.845399995644887
7 0.85489999850591
8 0.84576667547226
9 0.839133330186208
10 0.854566669464111
11 0.85103333791097
12 0.850766674677531
13 0.857833325862885
14 0.849833337465922
15 0.849399999777476
16 0.847166665395101
17 0.847900001207987
18 0.840466674168905
19 0.845766667524974
20 0.840499997138977
21 0.844099998474121
22 0.845766667524974
23 0.841500000158946
24 0.850033338864644
25 0.84363333384196
26 0.850633343060811
27 0.845866668224335
28 0.849966669082642
29 0.845799998442332
30 0.83846667210261
31 0.834366663297017
32 0.844533328215281
33 0.839699999491374
34 0.851266662279765
35 0.844333334763845
36 0.844100002447764
37 0.846200001239777
38 0.845100005467733
39 0.858233332633972
40 0.838366663455963
41 0.846200001239777
42 0.847100003560384
43 0.8439333319664
44 0.844600001970927
45 0.850166666507721
46 0.84600000778834
47 0.84396667877833
48 0.83793332974116
49 0.842866675059001
50 0.84796667098999
51 0.841199998060862
52 0.862266663710276
53 0.845433330535889
54 0.858399999141693
55 0.856466666857401
56 0.851366674900055
57 0.850500003496806
58 0.862299998601278
59 0.851266662279765
60 0.867766666412354
61 0.841233328978221
62 0.858699997266134
63 0.857266668478648
64 0.858699997266134
65 0.85760000149409
66 0.849400007724762
67 0.837433342138926
68 0.847133338451385
69 0.870166671276093
70 0.846300001939138
71 0.856600002447764
72 0.844500001271566
73 0.848533328374227
74 0.866333329677582
75 0.86416666507721
76 0.853733336925507
77 0.860399993260701
78 0.867066657543182
79 0.860633325576782
80 0.865333330631256
81 0.865333330631256
82 0.865766664346059
83 0.857899995644887
84 0.843533333142598
85 0.851166661580404
86 0.866466657320658
87 0.850066657861074
88 0.873899992307027
89 0.846233328183492
90 0.865333330631256
91 0.867533326148987
92 0.865333330631256
93 0.860633325576782
94 0.856766668955485
95 0.861833341916402
96 0.851366670926412
97 0.865333330631256
98 0.878599997361501
99 0.873899992307027
100 0.871166662375132
};
\addlegendentry{DQN}
\addplot [semithick, color2]
table {%
1 0.860766666666667
100 0.860766666666667
};
\addlegendentry{Random}
\addplot [semithick, color3]
table {%
1 0.871
100 0.871
};
\addlegendentry{ETS}
\addplot [semithick, color4]
table {%
1 0.8631
100 0.8631
};
\addlegendentry{Heur-GD}
\addplot [semithick, color5, dash pattern=on 1pt off 3pt on 3pt off 3pt]
table {%
1 0.8667
100 0.8667
};
\addlegendentry{Heur-LD}
\addplot [semithick, color6, dashed]
table {%
1 0.8594
100 0.8594
};
\addlegendentry{Heur-AT}



\nextgroupplot[title=Seed 11,
height=\figheight,
legend cell align={left},
legend style={
  fill opacity=0.8,
  draw opacity=1,
  text opacity=1,
  at={(0.97,0.03)},
  anchor=south east,
  draw=white!80!black
},
minor xtick={25, 75},
minor ytick={},
tick align=outside,
tick pos=left,
%title={Labels: [[7, 8], [2, 6], [4, 5], [1, 3], [0, 9]]},
width=\figwidth,
x grid style={white!69.0196078431373!black},
xlabel={Eval. Steps},
xminorgrids,
xmajorgrids,
xmin=-3.95, xmax=104.95,
xtick style={color=black},
xtick={-25,0,50,100,125},
xticklabels={-25,0,50,100,125},
y grid style={white!69.0196078431373!black},
%ylabel={ACC},
ymajorgrids,
ymin=0.725908333004316, ymax=0.84,
ytick style={color=black},
ytick={0.72,0.74,0.76,0.78,0.8,0.82,0.84},
yticklabels={72,74,76,78,80,82,84}
]
\addplot [semithick, color0]
table {%
	1 0.79815999507904
	2 0.803639996051788
	3 0.795359995365143
	4 0.819599995613098
	5 0.794980001449585
	6 0.796699995994568
	7 0.797300002574921
	8 0.802079994678497
	9 0.79502001285553
	10 0.799319999217987
	11 0.78978000164032
	12 0.789420001506805
	13 0.791539998054504
	14 0.802660000324249
	15 0.789139995574951
	16 0.800860004425049
	17 0.793599994182587
	18 0.79345999956131
	19 0.791599988937378
	20 0.792399997711182
	21 0.786159992218017
	22 0.784080002307892
	23 0.797299997806549
	24 0.806459996700287
	25 0.78163999080658
	26 0.790539994239807
	27 0.798259990215301
	28 0.794819989204407
	29 0.797479996681213
	30 0.810739994049072
	31 0.799099993705749
	32 0.790179998874664
	33 0.802179996967316
	34 0.792799997329712
	35 0.803599996566772
	36 0.792120006084442
	37 0.786080002784729
	38 0.791560001373291
	39 0.799479997158051
	40 0.791099998950958
	41 0.796200001239777
	42 0.806480002403259
	43 0.800020000934601
	44 0.797440001964569
	45 0.802099997997284
	46 0.797800006866455
	47 0.795460007190704
	48 0.798199999332428
	49 0.797720000743866
	50 0.797099995613098
	51 0.794659996032715
	52 0.789999995231629
	53 0.799519996643066
	54 0.796799998283386
	55 0.800099999904633
	56 0.797319993972778
	57 0.788359994888306
	58 0.799759998321533
	59 0.79819999217987
	60 0.794039990901947
	61 0.793519992828369
	62 0.794839992523193
	63 0.795479991436005
	64 0.802559998035431
	65 0.803799993991852
	66 0.804899990558624
	67 0.80344000339508
	68 0.800999994277954
	69 0.802599990367889
	70 0.803659994602203
	71 0.801399993896484
	72 0.803659994602203
	73 0.803659994602203
	74 0.79759999036789
	75 0.796499993801117
	76 0.802559998035431
	77 0.803799993991852
	78 0.795399997234344
	79 0.795399997234344
	80 0.795399997234344
	81 0.793699994087219
	82 0.796640000343323
	83 0.795399997234344
	84 0.79309999704361
	85 0.792599997520447
	86 0.795399997234344
	87 0.79771999835968
	88 0.794199993610382
	89 0.796499993801117
	90 0.80099999666214
	91 0.794199993610382
	92 0.791899993419647
	93 0.796059994697571
	94 0.798359994888306
	95 0.791399993896484
	96 0.794199993610382
	97 0.794199993610382
	98 0.794199993610382
	99 0.795299990177155
	100 0.794199993610382
};
%\addlegendentry{DQN}
\addplot [semithick, color1]
table {%
	1 0.797659995555878
	2 0.798399996757507
	3 0.799599995613098
	4 0.780660004615784
	5 0.782640001773834
	6 0.776480002403259
	7 0.792839999198914
	8 0.806539998054504
	9 0.815999994277954
	10 0.82279999256134
	11 0.81121999502182
	12 0.800379998683929
	13 0.81121999502182
	14 0.792839999198913
	15 0.817379994392395
	16 0.81121999502182
	17 0.7996399974823
	18 0.817379994392395
	19 0.804419996738434
	20 0.794219999313355
	21 0.78805999994278
	22 0.805799996852875
	23 0.815999994277954
	24 0.794219999313354
	25 0.7996399974823
	26 0.794219999313354
	27 0.793580000400543
	28 0.787420001029968
	29 0.812459998130798
	30 0.769680004119873
	31 0.790540006160736
	32 0.819860000610351
	33 0.813199999332428
	34 0.824560005664825
	35 0.803360006809235
	36 0.795460002422333
	37 0.809020006656647
	38 0.796200006008148
	39 0.824560005664825
	40 0.817640006542206
	41 0.820600004196167
	42 0.817640006542206
	43 0.829000008106232
	44 0.823800005912781
	45 0.816180007457733
	46 0.807520005702972
	47 0.813940002918243
	48 0.809120001792908
	49 0.827180004119873
	50 0.811740002632141
	51 0.829000008106232
	52 0.801640000343323
	53 0.824560005664825
	54 0.808519999980926
	55 0.821840007305145
	56 0.829000008106232
	57 0.816180007457733
	58 0.829000008106232
	59 0.816180007457733
	60 0.829000008106232
	61 0.829000008106232
	62 0.807520005702972
	63 0.824560005664825
	64 0.824560005664826
	65 0.814680006504059
	66 0.829000008106232
	67 0.820840010643005
	68 0.817400004863739
	69 0.822340006828308
	70 0.829000008106232
	71 0.820840010643005
	72 0.829000008106232
	73 0.815020008087158
	74 0.829000008106232
	75 0.816020004749298
	76 0.829000008106232
	77 0.821840007305145
	78 0.829000008106232
	79 0.829000008106232
	80 0.829000008106232
	81 0.824560005664825
	82 0.818900003433228
	83 0.829000008106232
	84 0.829000008106232
	85 0.829000008106232
	86 0.829000008106232
	87 0.824560005664825
	88 0.821840007305145
	89 0.829000008106232
	90 0.806080002784729
	91 0.829000008106232
	92 0.818900003433227
	93 0.818900003433228
	94 0.829000008106232
	95 0.818900003433228
	96 0.821840007305145
	97 0.829000008106232
	98 0.818900003433228
	99 0.829000008106232
	100 0.829000008106232
};
%\addlegendentry{A2C}
\addplot [semithick, color2]
table {%
	1 0.799399995803833
	2 0.799399995803833
	3 0.799399995803833
	4 0.799399995803833
	5 0.799399995803833
	6 0.798319997787476
	7 0.792860000133514
	8 0.801279995441437
	9 0.8067999958992
	10 0.8067999958992
	11 0.8067999958992
	12 0.811239998340607
	13 0.811239998340607
	14 0.811239998340607
	15 0.809859998226166
	16 0.811239998340607
	17 0.811239998340607
	18 0.811239998340607
	19 0.811239998340607
	20 0.811239998340607
	21 0.815680000782013
	22 0.815680000782013
	23 0.809859998226166
	24 0.80655999660492
	25 0.809859998226166
	26 0.808479998111725
	27 0.816820001602173
	28 0.814300000667572
	29 0.807099997997284
	30 0.824560005664826
	31 0.815680000782013
	32 0.824560005664826
	33 0.820120003223419
	34 0.818740003108978
	35 0.818740003108978
	36 0.820120003223419
	37 0.812920000553131
	38 0.815680000782013
	39 0.824560005664826
	40 0.818740003108978
	41 0.814300000667572
	42 0.824560005664826
	43 0.824560005664826
	44 0.815680000782013
	45 0.814300000667572
	46 0.815680000782013
	47 0.814300000667572
	48 0.814300000667572
	49 0.808479998111725
	50 0.808479998111725
	51 0.80655999660492
	52 0.815680000782013
	53 0.807139999866486
	54 0.815680000782013
	55 0.809859998226166
	56 0.812379999160767
	57 0.808479998111725
	58 0.814300000667572
	59 0.811580002307892
	60 0.812920000553131
	61 0.820120003223419
	62 0.809859998226166
	63 0.808479998111725
	64 0.807139999866486
	65 0.818740003108978
	66 0.809859998226166
	67 0.814300000667572
	68 0.808479998111725
	69 0.809859998226166
	70 0.807139999866486
	71 0.802659995555878
	72 0.804039995670319
	73 0.809859998226166
	74 0.805759999752045
	75 0.80541999578476
	76 0.809859998226166
	77 0.808519999980926
	78 0.809859998226166
	79 0.80541999578476
	80 0.809859998226166
	81 0.808519999980926
	82 0.809859998226166
	83 0.8067999958992
	84 0.809859998226166
	85 0.811239998340607
	86 0.80541999578476
	87 0.809859998226166
	88 0.804039995670319
	89 0.8067999958992
	90 0.809859998226166
	91 0.8067999958992
	92 0.801319997310638
	93 0.8067999958992
	94 0.804039995670319
	95 0.80541999578476
	96 0.809859998226166
	97 0.809859998226166
	98 0.804039995670319
	99 0.8067999958992
	100 0.8067999958992
};
%\addlegendentry{SAC}
\addplot [semithick, color3]
table {%
	1 0.73052
	100 0.73052
};
%\addlegendentry{Random}
\addplot [semithick, color4]
table {%
	1 0.7516
	100 0.7516
};
%\addlegendentry{ETS}
\addplot [semithick, color5]
table {%
	1 0.7945
	100 0.7945
};
%\addlegendentry{Heur-GD}
\addplot [semithick, color6, dash pattern=on 1pt off 3pt on 3pt off 3pt]
table {%
	1 0.8164
	100 0.8164
};
%\addlegendentry{Heur-LD}
\addplot [semithick, white!49.8039215686275!black, dashed]
table {%
	1 0.8112
	100 0.8112
};
%\addlegendentry{Heur-AT}


\nextgroupplot[title=Seed 12,
height=\figheight,
legend cell align={left},
legend style={
  fill opacity=0.8,
  draw opacity=1,
  text opacity=1,
  at={(0.5,0.91)},
  anchor=north,
  draw=white!80!black
},
minor xtick={25, 75},
minor ytick={},
tick align=outside,
tick pos=left,
%title={Labels: [[5, 8], [7, 0], [4, 9], [3, 2], [1, 6]]},
width=\figwidth,
x grid style={white!69.0196078431373!black},
xlabel={Eval. Steps},
xminorgrids,
xmajorgrids,
xmin=-3.95, xmax=104.95,
xtick style={color=black},
xtick={-25,0,50,100,125},
xticklabels={-25,0,50,100,125},
y grid style={white!69.0196078431373!black},
%ylabel={ACC},
ymajorgrids,
ymin=0.85, ymax=0.90,
ytick style={color=black},
ytick={0.85,0.86,0.87,0.88,0.89,0.9},
yticklabels={85,86,87,88,89,90}
]
\addplot [semithick, color0]
table {%
	1 0.864440002441406
	2 0.864460000991821
	3 0.866780002117157
	4 0.867700002193451
	5 0.867260003089905
	6 0.868540003299713
	7 0.865099999904632
	8 0.867340009212494
	9 0.870860002040863
	10 0.865220005512238
	11 0.864160003662109
	12 0.866160001754761
	13 0.863999998569489
	14 0.857459998130798
	15 0.869420001506805
	16 0.856319999694824
	17 0.86555999994278
	18 0.871280000209808
	19 0.85478000164032
	20 0.870300002098083
	21 0.864079999923706
	22 0.871039998531342
	23 0.862680003643036
	24 0.858879997730255
	25 0.868380000591278
	26 0.859960000514984
	27 0.863400001525879
	28 0.862380001544952
	29 0.871219999790192
	30 0.871399998664856
	31 0.852339994907379
	32 0.871999995708466
	33 0.865180003643036
	34 0.858700001239777
	35 0.856920001506805
	36 0.866579999923706
	37 0.867880001068115
	38 0.858740003108978
	39 0.866160004138947
	40 0.865980002880096
	41 0.864099996089935
	42 0.865820000171661
	43 0.865400002002716
	44 0.866800000667572
	45 0.864120001792908
	46 0.855740003585815
	47 0.875099999904632
	48 0.865160005092621
	49 0.862380001544952
	50 0.854979999065399
	51 0.859179999828339
	52 0.856319999694824
	53 0.864240005016327
	54 0.853480002880096
	55 0.851160001754761
	56 0.8557000041008
	57 0.862300002574921
	58 0.856160001754761
	59 0.856940002441406
	60 0.859320003986359
	61 0.854420001506805
	62 0.859320003986359
	63 0.858560004234314
	64 0.857900002002716
	65 0.855019998550415
	66 0.852940001487732
	67 0.850820002555847
	68 0.855520000457764
	69 0.859139997959137
	70 0.854740002155304
	71 0.857360000610351
	72 0.850280003547669
	73 0.850560007095337
	74 0.856980001926422
	75 0.852920000553131
	76 0.854659998416901
	77 0.862500002384186
	78 0.859120001792908
	79 0.856900005340576
	80 0.853400006294251
	81 0.853399999141693
	82 0.856980001926422
	83 0.84918000459671
	84 0.850200002193451
	85 0.86110000371933
	86 0.852780003547668
	87 0.850200002193451
	88 0.852960004806519
	89 0.849160006046295
	90 0.854519999027252
	91 0.853559999465942
	92 0.855520000457764
	93 0.857579998970032
	94 0.856659998893738
	95 0.858359999656677
	96 0.851860003471374
	97 0.855120000839233
	98 0.851959998607635
	99 0.850800001621246
	100 0.851820006370544
};
%\addlegendentry{DQN}
\addplot [semithick, color1]
table {%
	1 0.866880002021789
	2 0.866699993610382
	3 0.87338000535965
	4 0.869360010623932
	5 0.863380010128021
	6 0.863880014419556
	7 0.866360015869141
	8 0.862400004863739
	9 0.866380004882813
	10 0.863380010128021
	11 0.865860011577606
	12 0.8628600025177
	13 0.862880005836487
	14 0.865840008258819
	15 0.8628600025177
	16 0.863400013446808
	17 0.863880014419556
	18 0.8628600025177
	19 0.862400004863739
	20 0.863360006809235
	21 0.863380010128021
	22 0.862380001544953
	23 0.862400004863739
	24 0.862900009155273
	25 0.862880005836487
	26 0.862400004863739
	27 0.862400004863739
	28 0.865360007286072
	29 0.864000000953674
	30 0.863400013446808
	31 0.865680010318756
	32 0.862400004863739
	33 0.862380001544952
	34 0.867139999866485
	35 0.864500005245209
	36 0.862900009155273
	37 0.861900000572205
	38 0.866500010490417
	39 0.865699999332428
	40 0.861879997253418
	41 0.864680001735687
	42 0.862400004863739
	43 0.866840000152588
	44 0.861900000572205
	45 0.869800002574921
	46 0.86139999628067
	47 0.867599999904633
	48 0.867139999866485
	49 0.86139999628067
	50 0.864919996261597
	51 0.864179997444153
	52 0.86746000289917
	53 0.867220005989075
	54 0.872400002479553
	55 0.866840000152588
	56 0.864999997615814
	57 0.864059998989105
	58 0.866840000152588
	59 0.870099999904633
	60 0.869380004405975
	61 0.869759998321533
	62 0.86139999628067
	63 0.864059998989105
	64 0.869620001316071
	65 0.870240001678467
	66 0.867999999523163
	67 0.866959998607636
	68 0.864059998989105
	69 0.869620001316071
	70 0.86139999628067
	71 0.866840000152588
	72 0.869620001316071
	73 0.864179997444153
	74 0.866959998607636
	75 0.866840000152588
	76 0.86521999835968
	77 0.872400002479553
	78 0.86139999628067
	79 0.866959998607636
	80 0.864179997444153
	81 0.866959998607636
	82 0.866959998607636
	83 0.866959998607636
	84 0.872520000934601
	85 0.869739999771118
	86 0.866959998607636
	87 0.866959998607636
	88 0.866959998607636
	89 0.866959998607636
	90 0.869739999771118
	91 0.864179997444153
	92 0.872520000934601
	93 0.866959998607636
	94 0.869620001316071
	95 0.866959998607636
	96 0.866959998607636
	97 0.875300002098083
	98 0.869739999771118
	99 0.872520000934601
	100 0.864179997444153
};
%\addlegendentry{A2C}
\addplot [semithick, color2]
table {%
	1 0.847200012207031
	2 0.847200012207031
	3 0.847200012207031
	4 0.847200012207031
	5 0.847200012207031
	6 0.847200012207031
	7 0.859560000896454
	8 0.875719993114471
	9 0.867139999866486
	10 0.873340001106262
	11 0.875300002098083
	12 0.875300002098083
	13 0.875300002098083
	14 0.875300002098083
	15 0.875300002098083
	16 0.875300002098083
	17 0.875300002098083
	18 0.875300002098083
	19 0.875300002098083
	20 0.878580000400543
	21 0.875660002231598
	22 0.875660002231598
	23 0.872520000934601
	24 0.872520000934601
	25 0.872880001068115
	26 0.869739999771118
	27 0.872880001068115
	28 0.869740002155304
	29 0.86938000202179
	30 0.86938000202179
	31 0.872520003318787
	32 0.86938000202179
	33 0.875660002231598
	34 0.875660002231598
	35 0.872520000934601
	36 0.875660002231598
	37 0.875660002231598
	38 0.872520003318787
	39 0.87554000377655
	40 0.872400004863739
	41 0.869740002155304
	42 0.875660002231598
	43 0.87554000377655
	44 0.87554000377655
	45 0.87554000377655
	46 0.872880001068115
	47 0.872400004863739
	48 0.87554000377655
	49 0.87554000377655
	50 0.87554000377655
	51 0.87554000377655
	52 0.869740002155304
	53 0.872400004863739
	54 0.87554000377655
	55 0.873300004005432
	56 0.873300004005432
	57 0.87554000377655
	58 0.87554000377655
	59 0.872880001068115
	60 0.87554000377655
	61 0.87554000377655
	62 0.869260005950928
	63 0.873300004005432
	64 0.87554000377655
	65 0.867500002384186
	66 0.873300004005432
	67 0.87554000377655
	68 0.87554000377655
	69 0.87554000377655
	70 0.870480000972748
	71 0.87554000377655
	72 0.869740002155304
	73 0.872400004863739
	74 0.872520003318787
	75 0.87456000328064
	76 0.871420004367828
	77 0.871420004367828
	78 0.872400004863739
	79 0.871420004367828
	80 0.868760001659393
	81 0.875660002231598
	82 0.875700001716614
	83 0.872400004863739
	84 0.87481999874115
	85 0.876680002212524
	86 0.875700001716614
	87 0.875700001716614
	88 0.872400004863739
	89 0.876680002212524
	90 0.873780002593994
	91 0.881859998703003
	92 0.877599999904633
	93 0.875320003032684
	94 0.874460000991821
	95 0.878460001945496
	96 0.877599999904633
	97 0.875320003032684
	98 0.875320003032684
	99 0.875320003032684
	100 0.875320003032684
};
%\addlegendentry{SAC}
\addplot [semithick, color3]
table {%
	1 0.87862
	100 0.87862
};
%\addlegendentry{Random}
\addplot [semithick, color4]
table {%
	1 0.8553
	100 0.8553
};
%\addlegendentry{ETS}
\addplot [semithick, color5]
table {%
	1 0.8976
	100 0.8976
};
%\addlegendentry{Heur-GD}
\addplot [semithick, color6, dash pattern=on 1pt off 3pt on 3pt off 3pt]
table {%
	1 0.8752
	100 0.8752
};
%\addlegendentry{Heur-LD}
\addplot [semithick, white!49.8039215686275!black, dashed]
table {%
	1 0.8957
	100 0.8957
};
%\addlegendentry{Heur-AT}



\nextgroupplot[title=Seed 13,
height=\figheight,
legend cell align={left},
legend style={
  fill opacity=0.8,
  draw opacity=1,
  text opacity=1,
  at={(0.5,0.09)},
  anchor=south,
  draw=white!80!black
},
minor xtick={25, 75},
minor ytick={},
tick align=outside,
tick pos=left,
%title={Labels: [[3, 5], [6, 1], [4, 7], [8, 9], [0, 2]]},
width=\figwidth,
x grid style={white!69.0196078431373!black},
xlabel={Eval. Steps},
xminorgrids,
xmajorgrids,
xmin=-3.95, xmax=104.95,
xtick style={color=black},
xtick={-25,0,50,100,125},
xticklabels={-25,0,50,100,125},
y grid style={white!69.0196078431373!black},
%ylabel={ACC},
ymajorgrids,
ymin=0.766089999723435, ymax=0.82,
ytick style={color=black},
ytick={0.76,0.77,0.78,0.79,0.8,0.81,0.82},
yticklabels={76,77,78,79,80,81,82}
]
\addplot [semithick, color0]
table {%
1 0.795866672197978
2 0.78270001411438
3 0.797366674741109
4 0.812333337465922
5 0.807566654682159
6 0.809599999586741
7 0.809599999586741
8 0.812633331616719
9 0.811666671435038
10 0.811733333269755
11 0.81170000632604
12 0.810566659768422
13 0.813033334414164
14 0.81453332901001
15 0.816333333651225
16 0.81453332901001
17 0.815499997138977
18 0.815499997138977
19 0.815499997138977
20 0.815499997138977
21 0.815499997138977
22 0.815499997138977
23 0.815499997138977
24 0.815499997138977
25 0.815499997138977
26 0.815499997138977
27 0.815499997138977
28 0.815499997138977
29 0.815499997138977
30 0.815499997138977
31 0.815499997138977
32 0.815499997138977
33 0.816300002733866
34 0.816700005531311
35 0.816700005531311
36 0.816700005531311
37 0.816700005531311
38 0.816700005531311
39 0.816700005531311
40 0.816700005531311
41 0.816700005531311
42 0.816700005531311
43 0.816700005531311
44 0.816700005531311
45 0.816700005531311
46 0.816700005531311
47 0.816700005531311
48 0.816700005531311
49 0.816700005531311
50 0.816700005531311
51 0.816700005531311
52 0.816700005531311
53 0.816700005531311
54 0.816700005531311
55 0.816700005531311
56 0.816700005531311
57 0.816700005531311
58 0.816700005531311
59 0.811466670036316
60 0.816700005531311
61 0.816700005531311
62 0.815466666221619
63 0.811466670036316
64 0.815466666221619
65 0.815466666221619
66 0.812999987602234
67 0.815466666221619
68 0.814233326911926
69 0.815466666221619
70 0.814233326911926
71 0.816266667842865
72 0.814233326911926
73 0.812999987602234
74 0.814233326911926
75 0.814233326911926
76 0.812999987602234
77 0.814233326911926
78 0.812999987602234
79 0.8128666639328
80 0.811933322747548
81 0.812099993228912
82 0.813866662979126
83 0.814233326911926
84 0.811933322747548
85 0.812999987602234
86 0.813166662057241
87 0.814233326911926
88 0.812999987602234
89 0.813166658083598
90 0.812999987602234
91 0.812633323669434
92 0.812999987602234
93 0.812999987602234
94 0.815033328533173
95 0.812999987602234
96 0.812999987602234
97 0.812999987602234
98 0.812999987602234
99 0.812999987602234
100 0.811933322747548
};
%\addlegendentry{a2c, gae, envs=10, lr=0.0003, act=tanh}
\addplot [semithick, color1]
table {%
1 0.799166663487752
2 0.800333336989085
3 0.804266663392385
4 0.790233329931895
5 0.804466660817464
6 0.808233336607615
7 0.805666665236155
8 0.810966674486796
9 0.804799997806549
10 0.803966661294301
11 0.809100008010864
12 0.81096666653951
13 0.809766670068105
14 0.805333340167999
15 0.809500002861023
16 0.813900009791056
17 0.809100008010864
18 0.805400005976359
19 0.812933333714803
20 0.810566675662994
21 0.812766671180725
22 0.809799993038178
23 0.811933322747548
24 0.809799993038178
25 0.814133334159851
26 0.813966663678487
27 0.809799993038178
28 0.816000004609426
29 0.811033328374227
30 0.811366657416026
31 0.809966659545898
32 0.812933325767517
33 0.812933325767517
34 0.813333328564962
35 0.812833329041799
36 0.811699990431468
37 0.814533336957296
38 0.810666664441427
39 0.812933325767517
40 0.814133334159851
41 0.811066667238871
42 0.807533331712087
43 0.807766664028168
44 0.812966664632161
45 0.811233337720235
46 0.81096666653951
47 0.811033328374227
48 0.812666658560435
49 0.81306666135788
50 0.812133324146271
51 0.811466658115387
52 0.811633328596751
53 0.809266666571299
54 0.81096666653951
55 0.811199998855591
56 0.811199998855591
57 0.810466667016347
58 0.810466667016347
59 0.810466667016347
60 0.802399996916453
61 0.80313333272934
62 0.800266667207082
63 0.796600008010864
64 0.808199997742971
65 0.807833329836527
66 0.802066663901011
67 0.810466667016347
68 0.807833329836527
69 0.810466667016347
70 0.802399996916453
71 0.812866671880086
72 0.809233331680298
73 0.809233331680298
74 0.800266667207082
75 0.801500002543131
76 0.800266667207082
77 0.809233331680298
78 0.809233331680298
79 0.812333329518636
80 0.806166660785675
81 0.800266667207082
82 0.805699992179871
83 0.814299996693929
84 0.801500002543131
85 0.810466667016347
86 0.812966668605804
87 0.801500002543131
88 0.810466667016347
89 0.810433328151703
90 0.810466667016347
91 0.810466667016347
92 0.810466667016347
93 0.809933332602183
94 0.805166665712992
95 0.810799996058146
96 0.800199997425079
97 0.801566664377848
98 0.799500000476837
99 0.803733336925507
100 0.808066662152608
};
%\addlegendentry{DQN, n train envs=10, lr=0.00003}
\addplot [semithick, color2]
table {%
1 0.7985
100 0.7985
};
%\addlegendentry{Random}
\addplot [semithick, color3]
table {%
1 0.7685
100 0.7685
};
%\addlegendentry{ETS}
\addplot [semithick, color4]
table {%
1 0.7893
100 0.7893
};
%\addlegendentry{Heur-GD}
\addplot [semithick, color5, dash pattern=on 1pt off 3pt on 3pt off 3pt]
table {%
1 0.8081
100 0.8081
};
%\addlegendentry{Heur-LD}
\addplot [semithick, color6, dashed]
table {%
1 0.8019
100 0.8019
};
%\addlegendentry{Heur-AT}



\nextgroupplot[title=Seed 14,
height=\figheight,
legend cell align={left},
legend style={
  fill opacity=0.8,
  draw opacity=1,
  text opacity=1,
  at={(0.5,0.09)},
  anchor=south,
  draw=white!80!black
},
minor xtick={25, 75},
minor ytick={},
tick align=outside,
tick pos=left,
%title={Labels: [[3, 9], [0, 5], [4, 2], [1, 7], [6, 8]]},
width=\figwidth,
x grid style={white!69.0196078431373!black},
xlabel={Eval. Steps},
xminorgrids,
xmajorgrids,
xmin=-3.95, xmax=104.95,
xtick style={color=black},
xtick={-25,0,50,100,125},
xticklabels={-25,0,50,100,125},
y grid style={white!69.0196078431373!black},
ylabel={ACC (\%)},
ymajorgrids,
ymin=0.773365, ymax=0.873135,
ytick style={color=black},
ytick={0.76,0.78,0.8,0.82,0.84,0.86,0.88},
yticklabels={76,78,80,82,84,86,88}
]
\addplot [semithick, color0]
table {%
1 0.850099996725718
2 0.834599995613098
3 0.863500003019969
4 0.862599996725718
5 0.857299995422363
6 0.855499994754791
7 0.860233330726624
8 0.861633332570394
9 0.862033331394196
10 0.861766664187113
11 0.860233330726624
12 0.859466663996378
13 0.860899996757507
14 0.857799994945526
15 0.859466663996378
16 0.855499994754791
17 0.857299995422363
18 0.861266664663951
19 0.860599998633067
20 0.855499994754791
21 0.855499994754791
22 0.857666663328806
23 0.863200000921885
24 0.855499994754791
25 0.857666663328806
26 0.857666663328806
27 0.857666663328806
28 0.855499994754791
29 0.858433330059052
30 0.857666663328806
31 0.855499994754791
32 0.855499994754791
33 0.855499994754791
34 0.862033331394196
35 0.855499994754791
36 0.855499994754791
37 0.857666663328806
38 0.855499994754791
39 0.855499994754791
40 0.855499994754791
41 0.855499994754791
42 0.855499994754791
43 0.855499994754791
44 0.859833331902822
45 0.857666663328806
46 0.857666663328806
47 0.857666663328806
48 0.855499994754791
49 0.857666663328806
50 0.857666663328806
51 0.857666663328806
52 0.855499994754791
53 0.861266664663951
54 0.862000000476837
55 0.857666663328806
56 0.855499994754791
57 0.855499994754791
58 0.855499994754791
59 0.857666663328806
60 0.855499994754791
61 0.859833331902822
62 0.859833331902822
63 0.857666663328806
64 0.861266664663951
65 0.855499994754791
66 0.855499994754791
67 0.857666663328806
68 0.857666663328806
69 0.855499994754791
70 0.857666663328806
71 0.862000000476837
72 0.855499994754791
73 0.855499994754791
74 0.855499994754791
75 0.855499994754791
76 0.862000000476837
77 0.859833331902822
78 0.855499994754791
79 0.856633325417837
80 0.859833331902822
81 0.859833331902822
82 0.857666663328806
83 0.856633325417837
84 0.857666663328806
85 0.855499994754791
86 0.859833331902822
87 0.859833331902822
88 0.858799993991852
89 0.859833331902822
90 0.857666663328806
91 0.859833331902822
92 0.856633325417837
93 0.855933332443237
94 0.857666663328806
95 0.859833331902822
96 0.857666663328806
97 0.855499994754791
98 0.857666663328806
99 0.859833331902822
100 0.859833331902822
};
%\addlegendentry{a2c, gae, envs=10, lr=0.0003, act=tanh}
\addplot [semithick, color1]
table {%
1 0.814466667175293
2 0.849566666285197
3 0.844299999872844
4 0.843799996376038
5 0.849666666984558
6 0.848233334223429
7 0.850866667429606
8 0.845033331712087
9 0.855433336893717
10 0.84766667286555
11 0.845833337306976
12 0.84600000778834
13 0.853500000635783
14 0.845766663551331
15 0.848666659990946
16 0.856899996598562
17 0.853966669241587
18 0.838900009791056
19 0.851900001366933
20 0.853166667620341
21 0.840933338801066
22 0.850733331839243
23 0.849966661135356
24 0.841633331775665
25 0.849433334668477
26 0.845766671498617
27 0.849533331394196
28 0.854766662915548
29 0.849233337243398
30 0.853166659673055
31 0.840866669019063
32 0.848500005404154
33 0.846366671721141
34 0.848466670513153
35 0.848266673088074
36 0.845333329836528
37 0.855733331044515
38 0.85313333272934
39 0.849766663710276
40 0.83939999739329
41 0.85313333272934
42 0.848499997456869
43 0.854299994309743
44 0.845766659577687
45 0.843699995676676
46 0.845866668224335
47 0.847966663042704
48 0.848199995358785
49 0.848666659990946
50 0.848399992783864
51 0.849199994405111
52 0.846899998188019
53 0.835300004482269
54 0.846999994913737
55 0.841533327102661
56 0.846733327706655
57 0.848766668637594
58 0.843666660785675
59 0.846466660499573
60 0.849933326244354
61 0.846466660499573
62 0.850499999523163
63 0.827633333206177
64 0.844799999396006
65 0.848933327198028
66 0.834533329804738
67 0.846599996089935
68 0.847599995136261
69 0.844766660531362
70 0.841999999682108
71 0.843799996376038
72 0.832866664727529
73 0.844333326816559
74 0.846133335431417
75 0.848633337020874
76 0.840899995962779
77 0.841899995009105
78 0.842199997107188
79 0.845233333110809
80 0.844599997997284
81 0.842799993356069
82 0.854866663614909
83 0.835533328851064
84 0.849333329995473
85 0.850833336512248
86 0.841899995009105
87 0.846533330281576
88 0.847799996534983
89 0.847499998410543
90 0.852333331108093
91 0.846799997488658
92 0.857099997997284
93 0.846799997488658
94 0.844899996121725
95 0.855200000603994
96 0.847866662343343
97 0.849399995803833
98 0.846799997488658
99 0.846799997488658
100 0.846799997488658
};
%\addlegendentry{DQN, n train envs=10, lr=0.00003}
\addplot [semithick, color2]
table {%
1 0.831866666666667
100 0.831866666666667
};
%\addlegendentry{Random}
\addplot [semithick, color3]
table {%
1 0.7779
100 0.7779
};
%\addlegendentry{ETS}
\addplot [semithick, color4]
table {%
1 0.8686
100 0.8686
};
%\addlegendentry{Heur-GD}
\addplot [semithick, color5, dash pattern=on 1pt off 3pt on 3pt off 3pt]
table {%
1 0.84
100 0.84
};
%\addlegendentry{Heur-LD}
\addplot [semithick, color6, dashed]
table {%
1 0.8657
100 0.8657
};
%\addlegendentry{Heur-AT}



\nextgroupplot[title=Seed 15,
height=\figheight,
legend cell align={left},
legend style={
  fill opacity=0.8,
  draw opacity=1,
  text opacity=1,
  at={(0.97,0.03)},
  anchor=south east,
  draw=white!80!black
},
minor xtick={25, 75},
minor ytick={},
tick align=outside,
tick pos=left,
%title={Labels: [[2, 6], [1, 3], [7, 0], [9, 4], [5, 8]]},
width=\figwidth,
x grid style={white!69.0196078431373!black},
xlabel={Eval. Steps},
xminorgrids,
xmajorgrids,
xmin=-3.95, xmax=104.95,
xtick style={color=black},
xtick={-25,0,50,100,125},
xticklabels={-25,0,50,100,125},
y grid style={white!69.0196078431373!black},
%ylabel={ACC},
ymajorgrids,
ymin=0.795768337051074, ymax=0.851465010046959,
ytick style={color=black},
ytick={0.79,0.8,0.81,0.82,0.83,0.84,0.85,0.86},
yticklabels={79,80,81,82,83,84,85,86}
]
\addplot [semithick, color0]
table {%
1 0.809100000063578
2 0.798300004005432
3 0.821366667747498
4 0.826866670449575
5 0.826533329486847
6 0.823666667938232
7 0.825433333714803
8 0.823833334445953
9 0.818466671307882
10 0.822533333301544
11 0.829399995009104
12 0.829666662216186
13 0.835233322779338
14 0.837833329041799
15 0.837299990653992
16 0.837833329041799
17 0.837299990653992
18 0.837299990653992
19 0.837299990653992
20 0.837299990653992
21 0.837299990653992
22 0.837299990653992
23 0.837299990653992
24 0.837299990653992
25 0.837299990653992
26 0.837299990653992
27 0.837299990653992
28 0.837299990653992
29 0.837299990653992
30 0.84296665986379
31 0.840133325258891
32 0.845799994468689
33 0.845799994468689
34 0.845799994468689
35 0.845799994468689
36 0.845799994468689
37 0.845799994468689
38 0.845799994468689
39 0.845799994468689
40 0.845799994468689
41 0.845799994468689
42 0.845799994468689
43 0.845799994468689
44 0.845799994468689
45 0.845799994468689
46 0.845799994468689
47 0.845799994468689
48 0.845799994468689
49 0.845799994468689
50 0.845799994468689
51 0.845799994468689
52 0.845799994468689
53 0.845799994468689
54 0.845799994468689
55 0.845799994468689
56 0.845799994468689
57 0.845799994468689
58 0.845799994468689
59 0.842599999904632
60 0.845799994468689
61 0.845799994468689
62 0.845799994468689
63 0.845799994468689
64 0.846300001939138
65 0.846300001939138
66 0.847300016880035
67 0.846300001939138
68 0.846333336830139
69 0.846300001939138
70 0.846300001939138
71 0.846699996789297
72 0.846800009409587
73 0.845866664250692
74 0.847400005658468
75 0.846333336830139
76 0.845899999141693
77 0.845899999141693
78 0.847433340549469
79 0.845866664250692
80 0.848466670513153
81 0.845899999141693
82 0.846966667970022
83 0.845866664250692
84 0.847099999586741
85 0.845899999141693
86 0.845899999141693
87 0.848933343092601
88 0.845899999141693
89 0.848466670513153
90 0.846366671721141
91 0.845899999141693
92 0.845866664250692
93 0.845899999141693
94 0.846366671721141
95 0.845899999141693
96 0.845899999141693
97 0.848466670513153
98 0.846366671721141
99 0.845899999141693
100 0.845899999141693
};
%\addlegendentry{a2c, gae, envs=10, lr=0.0003, act=tanh}
\addplot [semithick, color1]
table {%
1 0.835166668891907
2 0.847099999586741
3 0.842033330599467
4 0.839433340231578
5 0.848699998855591
6 0.848800003528595
7 0.844233330090841
8 0.846799993515015
9 0.847366670767466
10 0.840366665522258
11 0.848299996058146
12 0.844866669178009
13 0.838833336035411
14 0.841166671117147
15 0.845366664727529
16 0.844833334287008
17 0.843866662184397
18 0.839366674423218
19 0.848933343092601
20 0.846600000063578
21 0.846566673119863
22 0.845899999141693
23 0.842200001080831
24 0.845899999141693
25 0.848466670513153
26 0.847200004259745
27 0.845899999141693
28 0.833266667524974
29 0.846600000063578
30 0.8349999944369
31 0.829833332697551
32 0.833766671021779
33 0.845333337783813
34 0.826366666952769
35 0.83933333158493
36 0.831199999650319
37 0.831600002447764
38 0.837400003274282
39 0.831633329391479
40 0.837300002574921
41 0.837033335367839
42 0.826799996693929
43 0.827566667397817
44 0.830833331743876
45 0.82966666618983
46 0.822699999809265
47 0.834266670544942
48 0.833766671021779
49 0.835066672166188
50 0.83846667210261
51 0.837833333015442
52 0.825766666730245
53 0.82813333272934
54 0.831733338038127
55 0.833766671021779
56 0.833766671021779
57 0.831866669654846
58 0.84203333457311
59 0.842933332920074
60 0.838933332761129
61 0.837833329041799
62 0.837833329041799
63 0.839733330408732
64 0.839733330408732
65 0.840299999713898
66 0.837833329041799
67 0.843933335940043
68 0.839733330408732
69 0.843933335940043
70 0.839733330408732
71 0.83909999926885
72 0.84056666692098
73 0.838666665554047
74 0.839533332983653
75 0.84313333829244
76 0.84056666692098
77 0.84099999666214
78 0.840866669019063
79 0.840866669019063
80 0.841933333873749
81 0.843833335240682
82 0.841866668065389
83 0.841933333873749
84 0.844500005245209
85 0.844500005245209
86 0.844500005245209
87 0.844500005245209
88 0.844500005245209
89 0.844500005245209
90 0.844500005245209
91 0.844500005245209
92 0.844500005245209
93 0.843000002702077
94 0.841933333873749
95 0.841933333873749
96 0.843833335240682
97 0.841366676489512
98 0.83663334051768
99 0.83663334051768
100 0.834166669845581
};
%\addlegendentry{DQN, n train envs=10, lr=0.00003}
\addplot [semithick, color2]
table {%
1 0.8486
100 0.8486
};
%\addlegendentry{Random}
\addplot [semithick, color3]
table {%
1 0.8378
100 0.8378
};
%\addlegendentry{ETS}
\addplot [semithick, color4]
table {%
1 0.8352
100 0.8352
};
%\addlegendentry{Heur-GD}
\addplot [semithick, color5, dash pattern=on 1pt off 3pt on 3pt off 3pt]
table {%
1 0.8408
100 0.8408
};
%\addlegendentry{Heur-LD}
\addplot [semithick, color6, dashed]
table {%
1 0.8408
100 0.8408
};
%\addlegendentry{Heur-AT}



\nextgroupplot[title=Seed 16,
height=\figheight,
legend cell align={left},
legend style={
  fill opacity=0.8,
  draw opacity=1,
  text opacity=1,
  at={(0.97,0.03)},
  anchor=south east,
  draw=white!80!black
},
minor xtick={25, 75},
minor ytick={},
tick align=outside,
tick pos=left,
%title={Labels: [[6, 2], [0, 7], [8, 4], [3, 1], [5, 9]]},
width=\figwidth,
x grid style={white!69.0196078431373!black},
xlabel={Eval. Steps},
xminorgrids,
xmajorgrids,
xmin=-3.95, xmax=104.95,
xtick style={color=black},
xtick={-25,0,50,100,125},
xticklabels={-25,0,50,100,125},
y grid style={white!69.0196078431373!black},
%ylabel={ACC},
ymajorgrids,
ymin=0.84, ymax=0.89,
ytick style={color=black},
ytick={0.84,0.85,0.86,0.87,0.88,0.89},
yticklabels={84,85,86,87,88,89}
]
\addplot [semithick, color0]
table {%
1 0.850666666030884
2 0.846299993991852
3 0.865166668097178
4 0.870766667524974
5 0.862966660658519
6 0.864999997615814
7 0.864999997615814
8 0.861966661612193
9 0.86256666580836
10 0.856666668256124
11 0.856666668256124
12 0.861966661612193
13 0.854866663614909
14 0.855766665935516
15 0.853433334827423
16 0.854066661993663
17 0.848399996757507
18 0.848399996757507
19 0.848399996757507
20 0.848399996757507
21 0.848399996757507
22 0.848399996757507
23 0.848399996757507
24 0.848399996757507
25 0.848399996757507
26 0.848399996757507
27 0.848399996757507
28 0.848399996757507
29 0.848399996757507
30 0.854066661993663
31 0.854066661993663
32 0.848399996757507
33 0.846200001239777
34 0.848399996757507
35 0.848399996757507
36 0.857566662629445
37 0.85043332974116
38 0.85146666765213
39 0.850299998124441
40 0.855966663360596
41 0.855700000127157
42 0.853799998760223
43 0.858633335431417
44 0.862833329041799
45 0.858899998664856
46 0.855700000127157
47 0.858633335431417
48 0.855700000127157
49 0.854100000858307
50 0.855700000127157
51 0.854100000858307
52 0.862833329041799
53 0.863999998569489
54 0.871566657225291
55 0.854100000858307
56 0.860233334700267
57 0.856433339913686
58 0.856866669654846
59 0.859800004959106
60 0.855266670385996
61 0.871566657225291
62 0.862833329041799
63 0.854100000858307
64 0.863999998569489
65 0.854100000858307
66 0.865166668097178
67 0.863999998569489
68 0.871566657225291
69 0.862833329041799
70 0.880299985408783
71 0.880299985408783
72 0.856433339913686
73 0.871566657225291
74 0.872733326752981
75 0.866333333651225
76 0.880299985408783
77 0.880299985408783
78 0.880299985408783
79 0.871566657225291
80 0.880299985408783
81 0.880299985408783
82 0.880299985408783
83 0.873899992307027
84 0.880299985408783
85 0.880299985408783
86 0.880299985408783
87 0.872733326752981
88 0.880299985408783
89 0.880299985408783
90 0.872733326752981
91 0.880299985408783
92 0.871566657225291
93 0.880299985408783
94 0.872733326752981
95 0.880299985408783
96 0.880299985408783
97 0.880299985408783
98 0.873899992307027
99 0.880299985408783
100 0.880299985408783
};
%\addlegendentry{a2c, gae, envs=10, lr=0.0003, act=tanh}
\addplot [semithick, color1]
table {%
1 0.863933332761129
2 0.873733325799306
3 0.870266664028168
4 0.864566667874654
5 0.860766669114431
6 0.878533327579498
7 0.868366666634877
8 0.869766668478648
9 0.867200005054474
10 0.862699997425079
11 0.863933332761129
12 0.864833331108093
13 0.859266670544942
14 0.864966654777527
15 0.864566663901011
16 0.862433330217997
17 0.863700000445048
18 0.865400000413259
19 0.868600006898244
20 0.873099990685781
21 0.861100002129873
22 0.876599987347921
23 0.876533321539561
24 0.880299985408783
25 0.877033325036367
26 0.871533326307933
27 0.880299985408783
28 0.874366652965546
29 0.879666654268901
30 0.879599992434184
31 0.873766664663951
32 0.877699991067251
33 0.877699991067251
34 0.87683333158493
35 0.88003333012263
36 0.880133326848348
37 0.877866657574972
38 0.88076665798823
39 0.879233316580454
40 0.872199988365173
41 0.882499996821086
42 0.878666655222575
43 0.880233323574066
44 0.881966662406921
45 0.875633331139882
46 0.87383333047231
47 0.880233323574066
48 0.880233323574066
49 0.880666661262512
50 0.880599991480509
51 0.880666661262512
52 0.878433318932851
53 0.869833330313365
54 0.880233323574066
55 0.879033323129018
56 0.878166663646698
57 0.877099998792013
58 0.878166663646698
59 0.878166663646698
60 0.876166665554047
61 0.876366670926412
62 0.880233323574066
63 0.880233323574066
64 0.880233323574066
65 0.875433337688446
66 0.877233330408732
67 0.878166663646698
68 0.878166663646698
69 0.878166663646698
70 0.877966658274333
71 0.878999996185303
72 0.874366664886475
73 0.880233323574066
74 0.880233323574066
75 0.872333339850108
76 0.87949998776118
77 0.885500001907349
78 0.880299985408783
79 0.882533331712087
80 0.881300000349681
81 0.879299994309743
82 0.877833342552185
83 0.876633338133494
84 0.872933332125346
85 0.879200005531311
86 0.877900000413259
87 0.877900000413259
88 0.876500006516774
89 0.880433336893717
90 0.880433336893717
91 0.87590000629425
92 0.880233323574066
93 0.880233323574066
94 0.882499996821086
95 0.879966660340627
96 0.878933330376943
97 0.871200009187063
98 0.879200005531311
99 0.878166663646698
100 0.876233327388763
};
%\addlegendentry{DQN, n train envs=10, lr=0.00003}
\addplot [semithick, color2]
table {%
1 0.874133333333333
100 0.874133333333333
};
%\addlegendentry{Random}
\addplot [semithick, color3]
table {%
1 0.878
100 0.878
};
%\addlegendentry{ETS}
\addplot [semithick, color4]
table {%
1 0.8677
100 0.8677
};
%\addlegendentry{Heur-GD}
\addplot [semithick, color5, dash pattern=on 1pt off 3pt on 3pt off 3pt]
table {%
1 0.8651
100 0.8651
};
%\addlegendentry{Heur-LD}
\addplot [semithick, color6, dashed]
table {%
1 0.8677
100 0.8677
};
%\addlegendentry{Heur-AT}



\nextgroupplot[title=Seed 17,
height=\figheight,
legend cell align={left},
legend style={
  fill opacity=0.8,
  draw opacity=1,
  text opacity=1,
  at={(0.5,0.09)},
  anchor=south,
  draw=white!80!black
},
minor xtick={25, 75},
minor ytick={},
tick align=outside,
tick pos=left,
%title={Labels: [[7, 2], [5, 3], [4, 0], [9, 8], [6, 1]]},
width=\figwidth,
x grid style={white!69.0196078431373!black},
xlabel={Eval. Steps},
xminorgrids,
xmajorgrids,
xmin=-3.95, xmax=104.95,
xtick style={color=black},
xtick={-25,0,50,100,125},
xticklabels={-25,0,50,100,125},
y grid style={white!69.0196078431373!black},
%ylabel={ACC},
ymajorgrids,
ymin=0.699, ymax=0.775004995894432,
ytick style={color=black},
ytick={0.7,0.71,0.72,0.73,0.74,0.75,0.76,0.77,0.78},
yticklabels={70,71,72,73,74,75,76,77,78}
]
\addplot [semithick, color0]
table {%
	1 0.744239995479584
	2 0.743399996757507
	3 0.749260003566742
	4 0.744819996356964
	5 0.75095999956131
	6 0.759700002670288
	7 0.756399998664856
	8 0.756700000762939
	9 0.757900002002716
	10 0.762019999027252
	11 0.766700003147125
	12 0.760359995365143
	13 0.763759994506836
	14 0.750460000038147
	15 0.756039998531342
	16 0.757139995098114
	17 0.753199996948242
	18 0.753239994049072
	19 0.749899995326996
	20 0.757779994010925
	21 0.75503999710083
	22 0.758399996757507
	23 0.754900000095367
	24 0.747000000476837
	25 0.753559999465942
	26 0.750839996337891
	27 0.75289999961853
	28 0.753819997310638
	29 0.754619998931885
	30 0.753299994468689
	31 0.749959998130798
	32 0.752859997749329
	33 0.754659998416901
	34 0.755699996948242
	35 0.762420001029968
	36 0.748239996433258
	37 0.758399999141693
	38 0.759719996452332
	39 0.758619997501373
	40 0.7525
	41 0.756520001888275
	42 0.742119998931885
	43 0.742119998931885
	44 0.75594000339508
	45 0.734539999961853
	46 0.738839998245239
	47 0.753339998722076
	48 0.749880006313324
	49 0.744960007667541
	50 0.743820004463196
	51 0.739260001182556
	52 0.734360003471375
	53 0.725680000782013
	54 0.752539992332458
	55 0.749879994392395
	56 0.748419997692108
	57 0.753919994831085
	58 0.751559994220734
	59 0.760239996910095
	60 0.745479996204376
	61 0.753139998912811
	62 0.756619999408722
	63 0.740759999752045
	64 0.75945999622345
	65 0.753519997596741
	66 0.747559998035431
	67 0.740060000419617
	68 0.738559999465942
	69 0.745620000362396
	70 0.746239995956421
	71 0.747699997425079
	72 0.747260000705719
	73 0.745759999752045
	74 0.753519997596741
	75 0.740139997005463
	76 0.738559999465942
	77 0.730399997234345
	78 0.735759992599487
	79 0.724499995708465
	80 0.744019994735718
	81 0.735179996490478
	82 0.727619998455048
	83 0.734379999637604
	84 0.741099996566773
	85 0.742200000286102
	86 0.737339999675751
	87 0.756599996089935
	88 0.737579996585846
	89 0.728380000591278
	90 0.744019994735718
	91 0.727219998836517
	92 0.736640005111694
	93 0.743780002593994
	94 0.752000002861023
	95 0.750560004711151
	96 0.732000000476837
	97 0.732700004577637
	98 0.736839997768402
	99 0.761980006694794
	100 0.75128000497818
};
%\addlegendentry{DQN}
\addplot [semithick, color1]
table {%
	1 0.747979998588562
	2 0.736539998054504
	3 0.762419996261597
	4 0.756520001888275
	5 0.75084000825882
	6 0.748240013122559
	7 0.74850001335144
	8 0.74850001335144
	9 0.752440009117126
	10 0.746060009002686
	11 0.747280011177063
	12 0.74850001335144
	13 0.746060009002685
	14 0.746580004692078
	15 0.753720006942749
	16 0.755460004806519
	17 0.753720006942749
	18 0.757900002002716
	19 0.759300000667572
	20 0.758600001335144
	21 0.758600001335144
	22 0.761099998950958
	23 0.759300000667572
	24 0.758600001335144
	25 0.759300000667572
	26 0.758600001335144
	27 0.759300000667572
	28 0.757900002002716
	29 0.758920001983643
	30 0.757900002002716
	31 0.757200002670288
	32 0.757200002670288
	33 0.757200002670288
	34 0.757200002670288
	35 0.757200002670288
	36 0.757200002670288
	37 0.757200002670288
	38 0.757200002670288
	39 0.757200002670288
	40 0.757200002670288
	41 0.757200002670288
	42 0.757200002670288
	43 0.757200002670288
	44 0.757200002670288
	45 0.757200002670288
	46 0.757200002670288
	47 0.757200002670288
	48 0.757200002670288
	49 0.757200002670288
	50 0.757200002670288
	51 0.757200002670288
	52 0.757200002670288
	53 0.757200002670288
	54 0.757200002670288
	55 0.757200002670288
	56 0.757200002670288
	57 0.757200002670288
	58 0.757200002670288
	59 0.757200002670288
	60 0.757200002670288
	61 0.757200002670288
	62 0.757200002670288
	63 0.757200002670288
	64 0.757200002670288
	65 0.757200002670288
	66 0.757200002670288
	67 0.757200002670288
	68 0.757200002670288
	69 0.757200002670288
	70 0.757200002670288
	71 0.757200002670288
	72 0.757200002670288
	73 0.757200002670288
	74 0.757200002670288
	75 0.757200002670288
	76 0.757200002670288
	77 0.757200002670288
	78 0.757200002670288
	79 0.757200002670288
	80 0.757200002670288
	81 0.757200002670288
	82 0.757200002670288
	83 0.757200002670288
	84 0.757200002670288
	85 0.757200002670288
	86 0.757200002670288
	87 0.757200002670288
	88 0.757200002670288
	89 0.757200002670288
	90 0.757200002670288
	91 0.757200002670288
	92 0.757200002670288
	93 0.757200002670288
	94 0.757200002670288
	95 0.757200002670288
	96 0.757200002670288
	97 0.757200002670288
	98 0.757200002670288
	99 0.757200002670288
	100 0.757200002670288
};
%\addlegendentry{A2C}
\addplot [semithick, color2]
table {%
	1 0.787199985980988
	2 0.787199985980988
	3 0.787199985980988
	4 0.787199985980988
	5 0.787199985980988
	6 0.787199985980988
	7 0.785039992332458
	8 0.778899986743927
	9 0.749999992847443
	10 0.751560001373291
	11 0.753619997501373
	12 0.754279997348785
	13 0.747620000839233
	14 0.757999999523163
	15 0.743299999237061
	16 0.743299999237061
	17 0.743299999237061
	18 0.743299999237061
	19 0.743299999237061
	20 0.743299999237061
	21 0.743299999237061
	22 0.743299999237061
	23 0.747079999446869
	24 0.744700000286102
	25 0.747079999446869
	26 0.747079999446869
	27 0.747079999446869
	28 0.747079999446869
	29 0.747079999446869
	30 0.747079999446869
	31 0.747079999446869
	32 0.747079999446869
	33 0.747079999446869
	34 0.747079999446869
	35 0.747079999446869
	36 0.758999998569489
	37 0.747079999446869
	38 0.747079999446869
	39 0.744700000286102
	40 0.745679998397827
	41 0.75059999704361
	42 0.743299999237061
	43 0.743299999237061
	44 0.745679998397827
	45 0.745679998397827
	46 0.745679998397827
	47 0.745679998397827
	48 0.744700000286102
	49 0.744700000286102
	50 0.744700000286102
	51 0.744700000286102
	52 0.744700000286102
	53 0.743299999237061
	54 0.750759997367859
	55 0.744700000286102
	56 0.744700000286102
	57 0.754900000095367
	58 0.768220000267029
	59 0.754900000095367
	60 0.759479999542236
	61 0.773699998855591
	62 0.771979999542236
	63 0.771979999542236
	64 0.771979999542236
	65 0.771979999542236
	66 0.771979999542236
	67 0.771979999542236
	68 0.771979999542236
	69 0.771979999542236
	70 0.771979999542236
	71 0.771979999542236
	72 0.771979999542236
	73 0.771979999542236
	74 0.771979999542236
	75 0.771979999542236
	76 0.771979999542236
	77 0.771979999542236
	78 0.771979999542236
	79 0.771979999542236
	80 0.771979999542236
	81 0.771979999542236
	82 0.771979999542236
	83 0.771979999542236
	84 0.773480000495911
	85 0.771979999542236
	86 0.771979999542236
	87 0.771979999542236
	88 0.771979999542236
	89 0.771979999542236
	90 0.771979999542236
	91 0.771979999542236
	92 0.771979999542236
	93 0.771979999542236
	94 0.771979999542236
	95 0.771979999542236
	96 0.771979999542236
	97 0.771979999542236
	98 0.771979999542236
	99 0.771979999542236
	100 0.771979999542236
};
%\addlegendentry{SAC}
\addplot [semithick, color3]
table {%
	1 0.7233
	100 0.7233
};
%\addlegendentry{Random}
\addplot [semithick, color4]
table {%
	1 0.7035
	100 0.7035
};
%\addlegendentry{ETS}
\addplot [semithick, color5]
table {%
	1 0.7414
	100 0.7414
};
%\addlegendentry{Heur-GD}
\addplot [semithick, color6, dash pattern=on 1pt off 3pt on 3pt off 3pt]
table {%
	1 0.7608
	100 0.7608
};
%\addlegendentry{Heur-LD}
\addplot [semithick, white!49.8039215686275!black, dashed]
table {%
	1 0.7644
	100 0.7644
};
%\addlegendentry{Heur-AT}



\nextgroupplot[title=Seed 18,
height=\figheight,
legend cell align={left},
legend style={
  fill opacity=0.8,
  draw opacity=1,
  text opacity=1,
  at={(0.03,0.03)},
  anchor=south west,
  draw=white!80!black
},
minor xtick={25, 75},
minor ytick={},
tick align=outside,
tick pos=left,
%title={Labels: [[7, 9], [0, 4], [2, 1], [6, 5], [8, 3]]},
width=\figwidth,
x grid style={white!69.0196078431373!black},
xlabel={Eval. Steps},
xminorgrids,
xmajorgrids,
xmin=-3.95, xmax=104.95,
xtick style={color=black},
xtick={-25,0,50,100,125},
xticklabels={-25,0,50,100,125},
y grid style={white!69.0196078431373!black},
ylabel={ACC (\%)},
ymajorgrids,
ymin=0.867104990553856, ymax=0.899995000449816,
ytick style={color=black},
ytick={0.87,0.88,0.89,0.9,0.91},
yticklabels={87,88,89,90,91}
]
\addplot [semithick, color0]
table {%
	1 0.88367999792099
	2 0.888259999752045
	3 0.889179999828339
	4 0.892259995937347
	5 0.882100002765656
	6 0.891120004653931
	7 0.885059998035431
	8 0.890080001354218
	9 0.89150000333786
	10 0.889939999580383
	11 0.896399993896485
	12 0.887880005836487
	13 0.886579999923706
	14 0.890820000171662
	15 0.88308000087738
	16 0.889860000610352
	17 0.888180000782013
	18 0.882180001735687
	19 0.892920000553131
	20 0.889739999771118
	21 0.892420003414154
	22 0.893500001430511
	23 0.885360000133514
	24 0.884519996643066
	25 0.885520000457764
	26 0.889079999923706
	27 0.884240002632141
	28 0.885520000457764
	29 0.882379996776581
	30 0.882299997806549
	31 0.884559998512268
	32 0.881440000534057
	33 0.885539999008179
	34 0.883760001659393
	35 0.886239998340607
	36 0.885979998111725
	37 0.880799996852875
	38 0.881019997596741
	39 0.888079998493195
	40 0.884740002155304
	41 0.880419998168945
	42 0.876459996700287
	43 0.88289999961853
	44 0.878579998016357
	45 0.887079997062683
	46 0.884979999065399
	47 0.878639998435974
	48 0.885199997425079
	49 0.880499997138977
	50 0.886539995670319
	51 0.882239999771118
	52 0.88061999797821
	53 0.883860001564026
	54 0.884159998893738
	55 0.881839997768402
	56 0.882599999904633
	57 0.879720001220703
	58 0.881859998703003
	59 0.883619997501373
	60 0.880400004386902
	61 0.881559998989105
	62 0.878739998340607
	63 0.884740002155304
	64 0.886080005168915
	65 0.880759999752045
	66 0.88268000125885
	67 0.883539996147156
	68 0.879079992771149
	69 0.877339999675751
	70 0.882480001449585
	71 0.885560004711151
	72 0.884439997673035
	73 0.877219994068146
	74 0.882760004997253
	75 0.882479996681213
	76 0.878459997177124
	77 0.883079996109009
	78 0.880099999904633
	79 0.876659998893738
	80 0.877039997577667
	81 0.872979998588562
	82 0.875179994106293
	83 0.872539994716644
	84 0.872539994716644
	85 0.880379996299744
	86 0.879799997806549
	87 0.875739998817444
	88 0.88407999753952
	89 0.881580002307892
	90 0.881920001506805
	91 0.887499995231628
	92 0.886100001335144
	93 0.884619998931885
	94 0.882139992713928
	95 0.880519993305206
	96 0.876819999217987
	97 0.882019999027252
	98 0.886019997596741
	99 0.887980003356934
	100 0.884139995574951
};
%\addlegendentry{DQN}
\addplot [semithick, color1]
table {%
	1 0.878299994468689
	2 0.874599993228912
	3 0.886639997959137
	4 0.888540005683899
	5 0.889360001087189
	6 0.88882000207901
	7 0.893339998722076
	8 0.896479997634888
	9 0.89559999704361
	10 0.895799994468689
	11 0.895799994468689
	12 0.896139996051788
	13 0.895799994468689
	14 0.895939998626709
	15 0.895799994468689
	16 0.895799994468689
	17 0.895799994468689
	18 0.895799994468689
	19 0.895799994468689
	20 0.895799994468689
	21 0.895799994468689
	22 0.895799994468689
	23 0.895799994468689
	24 0.895799994468689
	25 0.895799994468689
	26 0.895799994468689
	27 0.894239995479584
	28 0.89541999578476
	29 0.894779996871948
	30 0.8946599984169
	31 0.894659998416901
	32 0.894279999732971
	33 0.8946599984169
	34 0.893900001049042
	35 0.894279999732971
	36 0.893900001049042
	37 0.893900001049042
	38 0.893900001049042
	39 0.893900001049042
	40 0.893900001049042
	41 0.893900001049042
	42 0.893519997596741
	43 0.893900001049042
	44 0.893900001049042
	45 0.893900001049042
	46 0.893900001049042
	47 0.894279999732971
	48 0.893900001049042
	49 0.893900001049042
	50 0.893900001049042
	51 0.893900001049042
	52 0.893900001049042
	53 0.893900001049042
	54 0.893900001049042
	55 0.893900001049042
	56 0.893900001049042
	57 0.893900001049042
	58 0.893900001049042
	59 0.893900001049042
	60 0.893900001049042
	61 0.893900001049042
	62 0.893900001049042
	63 0.894920001029968
	64 0.894920001029968
	65 0.893900001049042
	66 0.893900001049042
	67 0.894920001029968
	68 0.894920001029968
	69 0.893900001049042
	70 0.893900001049042
	71 0.894920001029968
	72 0.893900001049042
	73 0.893900001049042
	74 0.893900001049042
	75 0.893900001049042
	76 0.893900001049042
	77 0.893900001049042
	78 0.893900001049042
	79 0.893900001049042
	80 0.891540002822876
	81 0.894920001029968
	82 0.894920001029968
	83 0.893900001049042
	84 0.893900001049042
	85 0.893900001049042
	86 0.895940001010895
	87 0.895940001010895
	88 0.893519999980927
	89 0.894920001029968
	90 0.893519999980927
	91 0.894920001029968
	92 0.896960000991821
	93 0.894920001029968
	94 0.894920001029968
	95 0.895559999942779
	96 0.893519999980927
	97 0.895940001010895
	98 0.894600002765655
	99 0.895940001010895
	100 0.893519999980927
};
%\addlegendentry{A2C}
\addplot [semithick, color2]
table {%
	1 0.877400004863739
	2 0.877400004863739
	3 0.877400004863739
	4 0.877400004863739
	5 0.877400004863739
	6 0.877400004863739
	7 0.884920001029968
	8 0.889900004863739
	9 0.892560007572174
	10 0.88864000082016
	11 0.891439998149872
	12 0.89135999917984
	13 0.887780001163483
	14 0.892820003032684
	15 0.8928600025177
	16 0.892560002803803
	17 0.892820003032684
	18 0.895080003738403
	19 0.889200005531311
	20 0.89146000623703
	21 0.895080003738403
	22 0.893920001983643
	23 0.897539999485016
	24 0.899000000953674
	25 0.895999999046326
	26 0.895920000076294
	27 0.895920000076294
	28 0.895920000076294
	29 0.897460000514984
	30 0.897460000514984
	31 0.895380003452301
	32 0.897460000514984
	33 0.893840003013611
	34 0.892820003032684
	35 0.892820003032684
	36 0.893759999275208
	37 0.889119999408722
	38 0.894200000762939
	39 0.891360001564026
	40 0.891959998607635
	41 0.890099999904633
	42 0.891959998607635
	43 0.889820001125336
	44 0.892660000324249
	45 0.890099999904633
	46 0.893499999046326
	47 0.891360001564026
	48 0.889820001125336
	49 0.892660000324249
	50 0.889820001125336
	51 0.890460000038147
	52 0.889820001125336
	53 0.89039999961853
	54 0.889820001125336
	55 0.889759998321533
	56 0.891119999885559
	57 0.889579999446869
	58 0.892419998645782
	59 0.891959998607635
	60 0.889579999446869
	61 0.890879998207092
	62 0.892419998645782
	63 0.889859998226166
	64 0.889859998226166
	65 0.893959999084473
	66 0.890959997177124
	67 0.890959997177124
	68 0.890719995498657
	69 0.89025999546051
	70 0.889419996738434
	71 0.886080002784729
	72 0.889419996738434
	73 0.889419996738434
	74 0.889419996738434
	75 0.889419996738434
	76 0.886659998893738
	77 0.889419996738434
	78 0.89025999546051
	79 0.889179995059967
	80 0.887560002803802
	81 0.889419996738434
	82 0.887699999809265
	83 0.885840001106262
	84 0.884380004405975
	85 0.888420000076294
	86 0.885420000553131
	87 0.884760000705719
	88 0.885660002231598
	89 0.885660002231598
	90 0.885660002231598
	91 0.885660002231598
	92 0.887120001316071
	93 0.884360003471375
	94 0.884360003471375
	95 0.885660002231598
	96 0.884360003471375
	97 0.885200002193451
	98 0.885660002231598
	99 0.886500000953674
	100 0.885660002231598
};
%\addlegendentry{SAC}
\addplot [semithick, color3]
table {%
	1 0.87434
	100 0.87434
};
%\addlegendentry{Random}
\addplot [semithick, color4]
table {%
	1 0.8832
	100 0.8832
};
%\addlegendentry{ETS}
\addplot [semithick, color5]
table {%
	1 0.8859
	100 0.8859
};
%\addlegendentry{Heur-GD}
\addplot [semithick, color6, dash pattern=on 1pt off 3pt on 3pt off 3pt]
table {%
	1 0.8985
	100 0.8985
};
%\addlegendentry{Heur-LD}
\addplot [semithick, white!49.8039215686275!black, dashed]
table {%
	1 0.8859
	100 0.8859
};
%\addlegendentry{Heur-AT}


\nextgroupplot[title=Seed 19,
height=\figheight,
legend cell align={left},
legend style={
  fill opacity=0.8,
  draw opacity=1,
  text opacity=1,
  at={(0.5,0.09)},
  anchor=south,
  draw=white!80!black
},
minor xtick={25, 75},
minor ytick={},
tick align=outside,
tick pos=left,
%title={Labels: [[1, 7], [9, 6], [8, 4], [3, 0], [2, 5]]},
width=\figwidth,
x grid style={white!69.0196078431373!black},
xlabel={Eval. Steps},
xminorgrids,
xmajorgrids,
xmin=-3.95, xmax=104.95,
xtick style={color=black},
xtick={-25,0,50,100,125},
xticklabels={-25,0,50,100,125},
y grid style={white!69.0196078431373!black},
%ylabel={ACC},
ymajorgrids,
ymin=0.731755009293556, ymax=0.822944995760918,
ytick style={color=black},
ytick={0.72,0.74,0.76,0.78,0.8,0.82,0.84},
yticklabels={72,74,76,78,80,82,84}
]
\addplot [semithick, color0]
table {%
	1 0.770119993686676
	2 0.768799998760223
	3 0.773239998817444
	4 0.778700003623962
	5 0.763700001239777
	6 0.765620002746582
	7 0.773039996623993
	8 0.780080006122589
	9 0.778359997272491
	10 0.781360001564026
	11 0.790859997272491
	12 0.786640000343323
	13 0.780300004482269
	14 0.786000001430512
	15 0.78067999124527
	16 0.778220002651214
	17 0.792040002346039
	18 0.788360004425049
	19 0.793059999942779
	20 0.780739998817444
	21 0.791559996604919
	22 0.789240002632141
	23 0.790059998035431
	24 0.779840002059937
	25 0.78672000169754
	26 0.777060000896454
	27 0.796420001983643
	28 0.79194000005722
	29 0.77565999507904
	30 0.795520000457764
	31 0.794500002861023
	32 0.782959997653961
	33 0.78746000289917
	34 0.782259993553162
	35 0.779920008182526
	36 0.783540003299713
	37 0.799879999160767
	38 0.791479997634888
	39 0.796820001602173
	40 0.800940001010895
	41 0.799299998283386
	42 0.795419998168945
	43 0.800999999046326
	44 0.78731999874115
	45 0.800119996070862
	46 0.782660000324249
	47 0.800239996910095
	48 0.811479997634888
	49 0.803199999332428
	50 0.796839997768402
	51 0.793839998245239
	52 0.791319997310638
	53 0.798840003013611
	54 0.79923999786377
	55 0.789359996318817
	56 0.794979996681213
	57 0.789359996318817
	58 0.795139997005463
	59 0.796240003108978
	60 0.796199998855591
	61 0.790579998493195
	62 0.790579998493195
	63 0.79231999874115
	64 0.792379996776581
	65 0.790579998493195
	66 0.795320000648499
	67 0.787279994487762
	68 0.785639996528625
	69 0.793399999141693
	70 0.791739997863769
	71 0.785639996528625
	72 0.783559994697571
	73 0.784979999065399
	74 0.785719997882843
	75 0.792219998836517
	76 0.783559997081757
	77 0.796579995155335
	78 0.795459997653961
	79 0.786000001430511
	80 0.79173999786377
	81 0.787559993267059
	82 0.801079993247986
	83 0.794759993553162
	84 0.792419996261597
	85 0.787999997138977
	86 0.79191999912262
	87 0.79293999671936
	88 0.780859999656677
	89 0.782959997653961
	90 0.790160000324249
	91 0.786640002727509
	92 0.780199999809265
	93 0.787580006122589
	94 0.788820004463196
	95 0.774800002574921
	96 0.775460002422333
	97 0.788640003204346
	98 0.785080001354218
	99 0.795279996395111
	100 0.799079999923706
};
%\addlegendentry{DQN}
\addplot [semithick, color1]
table {%
	1 0.758300001621246
	2 0.735900008678436
	3 0.775179996490479
	4 0.8
	5 0.807999997138977
	6 0.810799999237061
	7 0.804479999542236
	8 0.810399990081787
	9 0.795520000457764
	10 0.8
	11 0.804479999542236
	12 0.797199997901916
	13 0.792000002861023
	14 0.801679997444153
	15 0.797199997901916
	16 0.81079999923706
	17 0.802800002098084
	18 0.797199997901917
	19 0.805199995040893
	20 0.8
	21 0.807999997138977
	22 0.80239999294281
	23 0.801679997444153
	24 0.81319999217987
	25 0.792000002861023
	26 0.810399990081787
	27 0.807599987983704
	28 0.806879992485046
	29 0.807599987983704
	30 0.813199992179871
	31 0.813199992179871
	32 0.810399990081787
	33 0.794399995803833
	34 0.797119994163513
	35 0.813199992179871
	36 0.807999997138977
	37 0.807599987983704
	38 0.815999994277954
	39 0.796239995956421
	40 0.799599990844727
	41 0.807599987983704
	42 0.80239999294281
	43 0.804239993095398
	44 0.807599987983704
	45 0.801279988288879
	46 0.801439990997314
	47 0.804239993095398
	48 0.801279988288879
	49 0.801439990997314
	50 0.804119989871979
	51 0.806919991970062
	52 0.810399990081787
	53 0.802639994621277
	54 0.80479998588562
	55 0.801439990997314
	56 0.801439990997314
	57 0.807599987983704
	58 0.802319989204407
	59 0.799599990844727
	60 0.804119989871979
	61 0.80479998588562
	62 0.804119989871979
	63 0.80479998588562
	64 0.799839992523193
	65 0.803439993858337
	66 0.801439990997314
	67 0.80479998588562
	68 0.80479998588562
	69 0.80479998588562
	70 0.80479998588562
	71 0.80479998588562
	72 0.80479998588562
	73 0.80479998588562
	74 0.80479998588562
	75 0.80479998588562
	76 0.801439990997314
	77 0.801639993190765
	78 0.80479998588562
	79 0.802319989204407
	80 0.80479998588562
	81 0.80479998588562
	82 0.80479998588562
	83 0.80479998588562
	84 0.80479998588562
	85 0.80479998588562
	86 0.80479998588562
	87 0.80479998588562
	88 0.802319989204407
	89 0.80479998588562
	90 0.80479998588562
	91 0.80479998588562
	92 0.80479998588562
	93 0.80479998588562
	94 0.80479998588562
	95 0.80479998588562
	96 0.80479998588562
	97 0.80479998588562
	98 0.802319989204407
	99 0.80479998588562
	100 0.80479998588562
};
%\addlegendentry{A2C}
\addplot [semithick, color2]
table {%
	1 0.79190000295639
	2 0.79190000295639
	3 0.79190000295639
	4 0.79190000295639
	5 0.79190000295639
	6 0.79190000295639
	7 0.779560005664825
	8 0.777020008563995
	9 0.789179999828339
	10 0.792739999294281
	11 0.792400002479553
	12 0.788620002269745
	13 0.792400002479553
	14 0.792400002479553
	15 0.794879999160767
	16 0.791360001564026
	17 0.79735999584198
	18 0.797980000972748
	19 0.800459997653961
	20 0.797980000972748
	21 0.79735999584198
	22 0.79735999584198
	23 0.79735999584198
	24 0.802319989204407
	25 0.799839992523193
	26 0.799839992523193
	27 0.799839992523193
	28 0.799839992523193
	29 0.800459997653961
	30 0.79735999584198
	31 0.801080002784729
	32 0.797980000972748
	33 0.794639999866486
	34 0.797119996547699
	35 0.797119996547699
	36 0.79581999540329
	37 0.796059994697571
	38 0.796059994697571
	39 0.799399995803833
	40 0.796059994697571
	41 0.796059994697571
	42 0.799399995803833
	43 0.796059994697571
	44 0.796059994697571
	45 0.796059994697571
	46 0.793579998016357
	47 0.793579998016357
	48 0.793579998016357
	49 0.793579998016357
	50 0.796059994697571
	51 0.793339998722076
	52 0.789999997615814
	53 0.79581999540329
	54 0.792380003929138
	55 0.789040002822876
	56 0.792620003223419
	57 0.789280002117157
	58 0.789280002117157
	59 0.79175999879837
	60 0.79175999879837
	61 0.79175999879837
	62 0.789280002117157
	63 0.789040002822876
	64 0.795099999904633
	65 0.7957200050354
	66 0.795099999904633
	67 0.789280002117157
	68 0.7957200050354
	69 0.792980005741119
	70 0.792980005741119
	71 0.792980005741119
	72 0.793419997692108
	73 0.795460002422333
	74 0.792980005741119
	75 0.795460002422333
	76 0.795460002422333
	77 0.792820003032684
	78 0.792980005741119
	79 0.801900005340576
	80 0.798800003528595
	81 0.792980005741119
	82 0.799420008659363
	83 0.792980005741119
	84 0.799420008659363
	85 0.792980005741119
	86 0.792980005741119
	87 0.795460002422333
	88 0.799420008659363
	89 0.792980005741119
	90 0.792980005741119
	91 0.801660006046295
	92 0.792740006446838
	93 0.792980005741119
	94 0.799420008659363
	95 0.792740006446838
	96 0.790700001716614
	97 0.801900005340576
	98 0.798200001716614
	99 0.792980005741119
	100 0.800380005836487
};
%\addlegendentry{SAC}
\addplot [semithick, color3]
table {%
	1 0.74888
	100 0.74888
};
%\addlegendentry{Random}
\addplot [semithick, color4]
table {%
	1 0.7767
	100 0.7767
};
%\addlegendentry{ETS}
\addplot [semithick, color5]
table {%
	1 0.7787
	100 0.7787
};
%\addlegendentry{Heur-GD}
\addplot [semithick, color6, dash pattern=on 1pt off 3pt on 3pt off 3pt]
table {%
	1 0.775
	100 0.775
};
%\addlegendentry{Heur-LD}
\addplot [semithick, white!49.8039215686275!black, dashed]
table {%
	1 0.7787
	100 0.7787
};
%\addlegendentry{Heur-AT}


\end{groupplot}

\end{tikzpicture}
  \vspace{-3mm}
  \caption{Performance progress measured in ACC (\%) for all methods in the 10 test environments for {\bf Split CIFAR-10} in the New Task Orders experiment. We plot the performance progress for A2C and DQN for 100 evaluation steps equidistantly distributed over the training episodes. The performance for the Random, ETS, and Heuristic scheduling baselines are plotted as straight lines.  }
  \label{fig:policy_rewards_cifar10_paperD}
  \vspace{-3mm}
\end{figure}

\clearpage

\begin{figure}[t]
  \centering
  \setlength{\figwidth}{0.26\textwidth}
  \setlength{\figheight}{.14\textheight}
  



\pgfplotsset{every axis title/.append style={at={(0.5,0.85)}}}
\pgfplotsset{every tick label/.append style={font=\tiny}}
\pgfplotsset{every major tick/.append style={major tick length=2pt}}
\pgfplotsset{every minor tick/.append style={minor tick length=1pt}}
\pgfplotsset{every axis x label/.append style={at={(0.5,-0.22)}}}
\pgfplotsset{every axis y label/.append style={at={(-0.30,0.5)}}}
\begin{tikzpicture}
\tikzstyle{every node}=[font=\scriptsize]
\definecolor{color0}{rgb}{0.12156862745098,0.466666666666667,0.705882352941177}
\definecolor{color1}{rgb}{1,0.498039215686275,0.0549019607843137}
\definecolor{color2}{rgb}{0.172549019607843,0.627450980392157,0.172549019607843}
\definecolor{color3}{rgb}{0.83921568627451,0.152941176470588,0.156862745098039}
\definecolor{color4}{rgb}{0.580392156862745,0.403921568627451,0.741176470588235}
\definecolor{color5}{rgb}{0.549019607843137,0.337254901960784,0.294117647058824}
\definecolor{color6}{rgb}{0.890196078431372,0.466666666666667,0.76078431372549}

\begin{groupplot}[group style={group size= 4 by 3, horizontal sep=1.00cm, vertical sep=1.20cm}]

\nextgroupplot[title=Seed 0,
height=\figheight,
legend cell align={left},
legend columns=-1,
legend style={
  fill opacity=0.8,
  draw opacity=1,
  text opacity=1,
  at={(0.10,1.29)},
  anchor=south west,
  draw=white!80!black
},
minor xtick={25, 75},
minor ytick={},
tick align=outside,
tick pos=left,
%title={Labels: [[0, 1], [2, 3], [4, 5], [6, 7], [8, 9]]},
width=\figwidth,
x grid style={white!69.0196078431373!black},
xlabel={Eval. Steps},
xminorgrids,
xmajorgrids,
xmin=-3.95, xmax=104.95,
xtick style={color=black},
xtick={-25,0,50,100,125},
xticklabels={-25,0,50,100,125},
y grid style={white!69.0196078431373!black},
ylabel={ACC (\%)},
ymajorgrids,
ymin=0.874000021219254, ymax=0.944911737839381,
ytick style={color=black},
ytick={0.87,0.88,0.89,0.9,0.91,0.92,0.93,0.94,0.95},
yticklabels={87,88,89,90,91,92,93,94,95}
]
\addplot [semithick, color0]
table {%
1 0.920442871252696
2 0.941688477993011
3 0.941688477993011
4 0.941688477993011
5 0.92857813835144
6 0.90269779364268
7 0.899552301565806
8 0.89826672077179
9 0.901317000389099
10 0.90269779364268
11 0.899552301565806
12 0.893451742331187
13 0.899552301565806
14 0.911943844954173
15 0.90712886651357
16 0.896502021948497
17 0.897882815202077
18 0.902313888072968
19 0.90712886651357
20 0.89826672077179
21 0.892070949077606
22 0.896502021948497
23 0.897882815202077
24 0.90712886651357
25 0.900933094819387
26 0.897882815202077
27 0.911943844954173
28 0.90574807325999
29 0.89826672077179
30 0.903991615772247
31 0.897794369856517
32 0.902313888072968
33 0.90712886651357
34 0.911943844954173
35 0.911943844954173
36 0.911943844954173
37 0.902313888072968
38 0.896561602751414
39 0.902313888072968
40 0.909403892358144
41 0.909403892358144
42 0.914218870798747
43 0.916493896643321
44 0.910741611321767
45 0.914218870798747
46 0.909403892358144
47 0.923292430241903
48 0.923583900928497
49 0.916493896643321
50 0.923583900928497
51 0.925210293134053
52 0.924621899922689
53 0.926582682132721
54 0.923292430241903
55 0.923583900928497
56 0.923292430241903
57 0.925210293134053
58 0.923583900928497
59 0.925210293134053
60 0.923292430241903
61 0.92683668533961
62 0.923583900928497
63 0.924627351760864
64 0.923000959555308
65 0.923000959555308
66 0.923583900928497
67 0.923000959555308
68 0.923292430241903
69 0.923583900928497
70 0.923583900928497
71 0.924621733029683
72 0.923583900928497
73 0.923583900928497
74 0.924918822447459
75 0.924330429236094
76 0.911563050746918
77 0.926291211446126
78 0.913772384325663
79 0.923583900928497
80 0.919341317812602
81 0.925210293134053
82 0.924918822447459
83 0.924918822447459
84 0.913480913639069
85 0.92595682144165
86 0.92595682144165
87 0.925999740759532
88 0.912892520427704
89 0.915398776531219
90 0.913480913639069
91 0.926218418280284
92 0.900708083311717
93 0.915398776531219
94 0.913480913639069
95 0.922340099016825
96 0.921751705805461
97 0.917131984233856
98 0.915398776531219
99 0.915398776531219
100 0.921751705805461
};
\addlegendentry{A2C}
\addplot [semithick, color1]
table {%
1 0.917391939957937
2 0.893735134601593
3 0.928369176387787
4 0.913403868675232
5 0.881280032793681
6 0.904842758178711
7 0.922092696030935
8 0.913173941771189
9 0.911552508672078
10 0.882053593794505
11 0.919121166070302
12 0.877223281065623
13 0.903644506136576
14 0.895649520556132
15 0.924893915653229
16 0.926588149865468
17 0.926732567946116
18 0.92084907690684
19 0.928024895985921
20 0.90300509929657
21 0.915032398700714
22 0.900353260835012
23 0.902990074952443
24 0.91404904127121
25 0.899091947078705
26 0.925257849693298
27 0.922489154338837
28 0.895459878444672
29 0.902996039390564
30 0.894334733486176
31 0.915931260585785
32 0.911435166994731
33 0.921298980712891
34 0.909860912958781
35 0.916818722089132
36 0.919512649377187
37 0.918125911553701
38 0.926833562056224
39 0.922511267662048
40 0.912653028964996
41 0.917856554190318
42 0.904125285148621
43 0.928729323546092
44 0.923400549093882
45 0.906525619824727
46 0.910272725423177
47 0.907428586483002
48 0.910797242323558
49 0.918778196970622
50 0.900753668944041
51 0.917465869585673
52 0.922927192846934
53 0.913506857554118
54 0.918377463022868
55 0.918654251098633
56 0.924610034624735
57 0.930768668651581
58 0.926225193341573
59 0.910062948862712
60 0.930805810292562
61 0.918196562925975
62 0.902306791146596
63 0.908601939678192
64 0.908501255512237
65 0.931351455052694
66 0.910565503438314
67 0.897773396968842
68 0.910506268342336
69 0.919329957167308
70 0.904659565289815
71 0.924862730503082
72 0.901529077688853
73 0.903433068593343
74 0.91377325852712
75 0.914650988578796
76 0.892892404397329
77 0.91377325852712
78 0.898701965808868
79 0.895259539286296
80 0.926143173376719
81 0.922568714618683
82 0.925139685471853
83 0.925793282190959
84 0.91958281993866
85 0.922568714618683
86 0.923876885573069
87 0.908007089296977
88 0.912167000770569
89 0.93069638411204
90 0.926091678937276
91 0.908002463976542
92 0.904659565289815
93 0.926143173376719
94 0.91958281993866
95 0.919823940594991
96 0.901666402816772
97 0.909426828225454
98 0.923898514111837
99 0.904934374491374
100 0.901666402816772
};
\addlegendentry{DQN}
\addplot [semithick, color2]
table {%
1 0.939648303765951
100 0.939648303765951
};
\addlegendentry{Random}
\addplot [semithick, color3]
table {%
1 0.910772966283802
100 0.910772966283802
};
\addlegendentry{ETS}
\addplot [semithick, color4]
table {%
1 0.924058274027314
100 0.924058274027314
};
\addlegendentry{Heur-GD}
\addplot [semithick, color5, dash pattern=on 1pt off 3pt on 3pt off 3pt]
table {%
1 0.924058274027314
100 0.924058274027314
};
\addlegendentry{Heur-LD}
\addplot [semithick, color6, dashed]
table {%
1 0.924058274027314
100 0.924058274027314
};
\addlegendentry{Heur-AT}



\nextgroupplot[title=Seed 1,
height=\figheight,
legend cell align={left},
legend style={
  fill opacity=0.8,
  draw opacity=1,
  text opacity=1,
  at={(0.5,0.09)},
  anchor=south,
  draw=white!80!black
},
minor xtick={25, 75},
minor ytick={},
tick align=outside,
tick pos=left,
%title={Labels: [[2, 9], [6, 4], [0, 3], [1, 7], [8, 5]]},
width=\figwidth,
x grid style={white!69.0196078431373!black},
xlabel={Eval. Steps},
xminorgrids,
xmajorgrids,
xmin=-3.95, xmax=104.95,
xtick style={color=black},
xtick={-25,0,50,100,125},
xticklabels={-25,0,50,100,125},
y grid style={white!69.0196078431373!black},
%ylabel={ACC},
ymajorgrids,
ymin=0.90963120704857, ymax=0.956657568043202,
ytick style={color=black},
ytick={0.89,0.9,0.91,0.92,0.93,0.94,0.95},
yticklabels={89,90,91,92,93,94,95}
]
\addplot [semithick, color0]
table {%
	1 0.926526477336884
	2 0.939247980117798
	3 0.935187644958496
	4 0.925645167827606
	5 0.935822265148163
	6 0.939189372062683
	7 0.941273808479309
	8 0.928815503120422
	9 0.931343376636505
	10 0.922736911773682
	11 0.937997064590454
	12 0.922509007453919
	13 0.938627438545227
	14 0.937414453029633
	15 0.928753695487976
	16 0.939214766025543
	17 0.931400475502014
	18 0.935029242038727
	19 0.935885074138641
	20 0.934611978530884
	21 0.932901191711426
	22 0.935550451278687
	23 0.93708690404892
	24 0.932933146953583
	25 0.936250669956207
	26 0.931883718967438
	27 0.937011017799377
	28 0.932685158252716
	29 0.929382679462433
	30 0.936874041557312
	31 0.932706274986267
	32 0.933693010807037
	33 0.935215861797333
	34 0.934235816001892
	35 0.935582404136658
	36 0.933670988082886
	37 0.931162846088409
	38 0.93810830116272
	39 0.93292852640152
	40 0.9316530418396
	41 0.932063701152802
	42 0.935461931228638
	43 0.935344345569611
	44 0.937480161190033
	45 0.934545674324036
	46 0.932980155944824
	47 0.930925192832947
	48 0.932422370910645
	49 0.928665611743927
	50 0.935880389213562
	51 0.936276979446411
	52 0.937418785095215
	53 0.936205298900604
	54 0.935460548400879
	55 0.935683145523071
	56 0.937961993217468
	57 0.935662438869476
	58 0.933530776500702
	59 0.933766539096832
	60 0.934950053691864
	61 0.935966205596924
	62 0.936140537261963
	63 0.938224914073944
	64 0.931714975833893
	65 0.933175549507141
	66 0.937365276813507
	67 0.933276283740997
	68 0.93775288105011
	69 0.934669761657715
	70 0.937860724925995
	71 0.93797271490097
	72 0.938591463565826
	73 0.939373638629913
	74 0.937480571269989
	75 0.937196063995361
	76 0.936227269172668
	77 0.935242385864258
	78 0.934577858448029
	79 0.935856518745422
	80 0.935378816127777
	81 0.937817645072937
	82 0.932884953022003
	83 0.936093189716339
	84 0.934577858448029
	85 0.938488738536835
	86 0.938327097892761
	87 0.937906632423401
	88 0.935758130550385
	89 0.936054382324219
	90 0.933709387779236
	91 0.934736795425415
	92 0.937888407707214
	93 0.935846648216248
	94 0.931536915302277
	95 0.937313776016235
	96 0.937313776016235
	97 0.938091804981232
	98 0.939286563396454
	99 0.935835523605347
	100 0.937874391078949
};
%\addlegendentry{DQN}
\addplot [semithick, color1]
table {%
	1 0.930518407821655
	2 0.921653783321381
	3 0.921653783321381
	4 0.921653783321381
	5 0.915469143390655
	6 0.922698998451233
	7 0.93472268819809
	8 0.925607373714447
	9 0.929355580806732
	10 0.929980828762054
	11 0.942404277324677
	12 0.926232621669769
	13 0.93372903585434
	14 0.924245316982269
	15 0.93369292974472
	16 0.934373958110809
	17 0.932124648094177
	18 0.932962191104889
	19 0.93327481508255
	20 0.939917254447937
	21 0.947221047878265
	22 0.935524125099182
	23 0.936901099681854
	24 0.936937205791473
	25 0.935106010437012
	26 0.939463033676147
	27 0.932613461017609
	28 0.932649567127228
	29 0.939881148338318
	30 0.941601326465607
	31 0.937355320453644
	32 0.934339165687561
	33 0.937319214344025
	34 0.937809541225433
	35 0.934757280349731
	36 0.940066323280334
	37 0.939463033676147
	38 0.937319214344025
	39 0.935186104774475
	40 0.936121282577515
	41 0.932672092914581
	42 0.933566796779633
	43 0.938965327739716
	44 0.934983205795288
	45 0.937080209255218
	46 0.937912096977234
	47 0.935219883918762
	48 0.936588568687439
	49 0.93774514913559
	50 0.936622190475464
	51 0.937250332832336
	52 0.937912096977234
	53 0.937936952114105
	54 0.935994048118591
	55 0.936622190475464
	56 0.938357827663422
	57 0.93588164806366
	58 0.938573861122131
	59 0.93064710855484
	60 0.936431012153625
	61 0.937092776298523
	62 0.937912096977234
	63 0.935253505706787
	64 0.935994048118591
	65 0.935994048118591
	66 0.937912096977234
	67 0.930266420841217
	68 0.932955932617187
	69 0.937833318710327
	70 0.936519591808319
	71 0.937912096977234
	72 0.937283954620361
	73 0.938573861122131
	74 0.935802869796753
	75 0.936568267345428
	76 0.933339321613312
	77 0.935141105651856
	78 0.931308872699737
	79 0.935700271129608
	80 0.935611691474915
	81 0.931206274032593
	82 0.931858236789703
	83 0.927364239692688
	84 0.926487762928009
	85 0.933505296707153
	86 0.933364176750183
	87 0.935175762176514
	88 0.926487762928009
	89 0.930075242519379
	90 0.933176734447479
	91 0.93049058675766
	92 0.927968847751617
	93 0.931907947063446
	94 0.9317471575737
	95 0.93264848947525
	96 0.932623634338379
	97 0.925333790779114
	98 0.927389094829559
	99 0.924521102905273
	100 0.93264848947525
};
%\addlegendentry{A2C}
\addplot [semithick, color2]
table {%
	1 0.93659999370575
	2 0.93659999370575
	3 0.93659999370575
	4 0.93659999370575
	5 0.93659999370575
	6 0.936879994869232
	7 0.937159996032715
	8 0.938099999427795
	9 0.93771999835968
	10 0.94579999923706
	11 0.947039999961853
	12 0.944980001449585
	13 0.940780005455017
	14 0.949740006923676
	15 0.952040002346039
	16 0.954140002727509
	17 0.942260000705719
	18 0.94646000623703
	19 0.947360005378723
	20 0.946820006370544
	21 0.947160000801086
	22 0.946400001049042
	23 0.940440006256104
	24 0.949100005626679
	25 0.943520004749298
	26 0.944160003662109
	27 0.946660006046295
	28 0.941040005683899
	29 0.943300006389618
	30 0.946700005531311
	31 0.943880004882813
	32 0.942480001449585
	33 0.95452000617981
	34 0.953320002555847
	35 0.953720002174377
	36 0.954120006561279
	37 0.947700002193451
	38 0.95190000295639
	39 0.95190000295639
	40 0.946280002593994
	41 0.947700002193451
	42 0.95190000295639
	43 0.947700002193451
	44 0.953320002555847
	45 0.947700002193451
	46 0.947700002193451
	47 0.946280002593994
	48 0.953320002555847
	49 0.948140006065369
	50 0.95190000295639
	51 0.94824000120163
	52 0.95190000295639
	53 0.95190000295639
	54 0.953860001564026
	55 0.95190000295639
	56 0.946280002593994
	57 0.951040008068085
	58 0.951040008068085
	59 0.951040008068085
	60 0.951040008068085
	61 0.95190000295639
	62 0.946280002593994
	63 0.946280002593994
	64 0.943360006809235
	65 0.943420007228851
	66 0.951040000915527
	67 0.945680005550384
	68 0.945680005550384
	69 0.945680005550384
	70 0.942800006866455
	71 0.945680005550384
	72 0.948500006198883
	73 0.95190000295639
	74 0.946280002593994
	75 0.948220002651215
	76 0.946280002593994
	77 0.940920007228851
	78 0.95190000295639
	79 0.950480003356934
	80 0.948160004615784
	81 0.948160004615784
	82 0.948160004615784
	83 0.93988000869751
	84 0.950860004425049
	85 0.949440004825592
	86 0.948020005226135
	87 0.950480003356934
	88 0.949620008468628
	89 0.95190000295639
	90 0.951040008068085
	91 0.950480003356934
	92 0.946280002593994
	93 0.951040008068085
	94 0.945420007705688
	95 0.95190000295639
	96 0.949620008468628
	97 0.95190000295639
	98 0.949620008468628
	99 0.951040008068085
	100 0.945420007705688
};
%\addlegendentry{SAC}
\addplot [semithick, color3]
table {%
	1 0.92133256344386
	100 0.92133256344386
};
%\addlegendentry{Random}
\addplot [semithick, color4]
table {%
	1 0.924611906166864
	100 0.924611906166864
};
%\addlegendentry{ETS}
\addplot [semithick, color5]
table {%
	1 0.911768768911962
	100 0.911768768911962
};
%\addlegendentry{Heur-GD}
\addplot [semithick, color6, dash pattern=on 1pt off 3pt on 3pt off 3pt]
table {%
	1 0.911768768911962
	100 0.911768768911962
};
%\addlegendentry{Heur-LD}
\addplot [semithick, white!49.8039215686275!black, dashed]
table {%
	1 0.911768768911962
	100 0.911768768911962
};
%\addlegendentry{Heur-AT}



\nextgroupplot[title=Seed 2,
height=\figheight,
legend cell align={left},
legend style={
  fill opacity=0.8,
  draw opacity=1,
  text opacity=1,
  at={(0.5,0.09)},
  anchor=south,
  draw=white!80!black
},
minor xtick={25, 75},
minor ytick={},
tick align=outside,
tick pos=left,
%title={Labels: [[4, 1], [5, 0], [7, 2], [3, 6], [9, 8]]},
width=\figwidth,
x grid style={white!69.0196078431373!black},
xlabel={Eval. Steps},
xminorgrids,
xmajorgrids,
xmin=-3.95, xmax=104.95,
xtick style={color=black},
xtick={-25,0,50,100,125},
xticklabels={-25,0,50,100,125},
y grid style={white!69.0196078431373!black},
%ylabel={ACC},
ymajorgrids,
ymin=0.818460931005355, ymax=0.980568533888777,
ytick style={color=black},
ytick={0.8,0.85,0.9,0.95,1},
yticklabels={80,85,90,95,100}
]
\addplot [semithick, color0]
table {%
	1 0.89190726518631
	2 0.894684348106384
	3 0.871349685192108
	4 0.913724598884582
	5 0.895439357757568
	6 0.902479364871979
	7 0.913152024745941
	8 0.908342905044556
	9 0.908540561199188
	10 0.89989856004715
	11 0.902726781368256
	12 0.904293346405029
	13 0.911792125701904
	14 0.908807907104492
	15 0.919487283229828
	16 0.915252845287323
	17 0.913990349769592
	18 0.908616099357605
	19 0.910581438541412
	20 0.929837033748627
	21 0.926937355995178
	22 0.921002459526062
	23 0.922406945228577
	24 0.938619968891144
	25 0.925057516098022
	26 0.927420122623444
	27 0.9340021443367
	28 0.929742772579193
	29 0.924154653549194
	30 0.916724565029144
	31 0.934186916351318
	32 0.928935904502869
	33 0.937045972347259
	34 0.919203629493713
	35 0.934599163532257
	36 0.930349986553192
	37 0.929940717220306
	38 0.935961058139801
	39 0.941150102615356
	40 0.941334874629974
	41 0.941334874629974
	42 0.925100202560425
	43 0.934414391517639
	44 0.935860841274262
	45 0.940927081108093
	46 0.906733903884888
	47 0.933768031597137
	48 0.907821133136749
	49 0.918878283500671
	50 0.907409791946411
	51 0.928369624614716
	52 0.904618701934814
	53 0.913865900039673
	54 0.937952260971069
	55 0.919390075206757
	56 0.90059268951416
	57 0.92766886472702
	58 0.916191494464874
	59 0.916844551563263
	60 0.927786371707916
	61 0.907945125102997
	62 0.916304395198822
	63 0.922693629264831
	64 0.919690434932709
	65 0.913751182556152
	66 0.914343748092651
	67 0.919875206947327
	68 0.926154470443726
	69 0.914098827838898
	70 0.930913228988647
	71 0.917029323577881
	72 0.917225339412689
	73 0.92079293012619
	74 0.929744064807892
	75 0.919875206947327
	76 0.919875206947327
	77 0.914188704490662
	78 0.919875206947327
	79 0.919282641410828
	80 0.918877391815186
	81 0.919875206947327
	82 0.925365345478058
	83 0.934803745746613
	84 0.902926790714264
	85 0.912245481014252
	86 0.912341561317444
	87 0.928114969730377
	88 0.91116482257843
	89 0.912341561317444
	90 0.919934480190277
	91 0.923131837844849
	92 0.910565910339355
	93 0.927411417961121
	94 0.906706285476685
	95 0.916403455734253
	96 0.920284912586212
	97 0.897139272689819
	98 0.911598973274231
	99 0.9198291015625
	100 0.919211175441742
};
%\addlegendentry{DQN}
\addplot [semithick, color1]
table {%
	1 0.905392324924469
	2 0.902243876457214
	3 0.902243876457214
	4 0.902243876457214
	5 0.885631222724915
	6 0.910980079174042
	7 0.890888421535492
	8 0.915248990058899
	9 0.89980030298233
	10 0.900934250354767
	11 0.857669031620026
	12 0.917516884803772
	13 0.884351615905762
	14 0.904443273544312
	15 0.899566195011139
	16 0.904969053268433
	17 0.901283674240112
	18 0.893962345123291
	19 0.895096292495728
	20 0.912862303256989
	21 0.892770645618439
	22 0.898823692798615
	23 0.908741824626923
	24 0.893527245521545
	25 0.909526348114014
	26 0.913561151027679
	27 0.908042976856232
	28 0.890410294532776
	29 0.913050043582916
	30 0.906248371601105
	31 0.910546510219574
	32 0.901504395008087
	33 0.901154971122742
	34 0.908641793727875
	35 0.891122567653656
	36 0.911454093456268
	37 0.904216663837433
	38 0.901798560619354
	39 0.904216663837433
	40 0.912523989677429
	41 0.916748428344727
	42 0.916643054485321
	43 0.916748428344727
	44 0.911765139102936
	45 0.916748428344727
	46 0.916748428344727
	47 0.916748428344727
	48 0.91270932674408
	49 0.912434725761414
	50 0.912434725761414
	51 0.907477056980133
	52 0.916748428344727
	53 0.914911937713623
	54 0.906559133529663
	55 0.916748428344727
	56 0.91270932674408
	57 0.91270932674408
	58 0.910872836112976
	59 0.905800344944
	60 0.91270932674408
	61 0.904895572662354
	62 0.906833734512329
	63 0.908670225143433
	64 0.906833734512329
	65 0.902794632911682
	66 0.906833734512329
	67 0.904631123542786
	68 0.902886152267456
	69 0.904631123542786
	70 0.900592021942139
	71 0.91270932674408
	72 0.904631123542786
	73 0.904631123542786
	74 0.908670225143433
	75 0.901242806911469
	76 0.900317420959473
	77 0.908670225143433
	78 0.900592021942139
	79 0.908670225143433
	80 0.900592021942139
	81 0.900592021942139
	82 0.900592021942139
	83 0.900592021942139
	84 0.892778267860413
	85 0.908670225143433
	86 0.90435652256012
	87 0.900592021942139
	88 0.896552920341492
	89 0.900317420959473
	90 0.90435652256012
	91 0.896552920341492
	92 0.900317420959473
	93 0.897196242809296
	94 0.906284470558166
	95 0.880357837677002
	96 0.891824357509613
	97 0.897414388656616
	98 0.893157141208649
	99 0.883746154308319
	100 0.904631123542786
};
%\addlegendentry{A2C}
\addplot [semithick, color2]
table {%
	1 0.968599998950958
	2 0.968599998950958
	3 0.968599998950958
	4 0.968599998950958
	5 0.968599998950958
	6 0.968599998950958
	7 0.968599998950958
	8 0.968580000400543
	9 0.971700003147125
	10 0.973200006484985
	11 0.970200006961822
	12 0.969860005378723
	13 0.963300008773804
	14 0.963300008773804
	15 0.969120004177094
	16 0.969120004177094
	17 0.972720003128052
	18 0.970620000362396
	19 0.972380001544952
	20 0.968460001945496
	21 0.968780002593994
	22 0.968800003528595
	23 0.972380001544952
	24 0.968460001945496
	25 0.966800005435944
	26 0.969500002861023
	27 0.973100001811981
	28 0.968800003528595
	29 0.968460001945496
	30 0.968800003528595
	31 0.968800003528595
	32 0.968800003528595
	33 0.968800003528595
	34 0.968460001945496
	35 0.968800003528595
	36 0.968460001945496
	37 0.967840001583099
	38 0.971760001182556
	39 0.968160002231598
	40 0.967840001583099
	41 0.968160002231598
	42 0.968160002231598
	43 0.968160002231598
	44 0.968160002231598
	45 0.968160002231598
	46 0.968160002231598
	47 0.967340002059937
	48 0.967040002346039
	49 0.9678600025177
	50 0.967120001316071
	51 0.966220002174378
	52 0.967040002346039
	53 0.967340002059937
	54 0.966220002174378
	55 0.967040002346039
	56 0.967360002994538
	57 0.967040002346039
	58 0.967040002346039
	59 0.967820003032684
	60 0.967040002346039
	61 0.969700000286102
	62 0.970640001296997
	63 0.966220002174378
	64 0.966220002174378
	65 0.967040002346039
	66 0.967040002346039
	67 0.967040002346039
	68 0.967040002346039
	69 0.967040002346039
	70 0.966780002117157
	71 0.967040002346039
	72 0.966520001888275
	73 0.966520001888275
	74 0.967340002059937
	75 0.967340002059937
	76 0.966520001888275
	77 0.967040002346039
	78 0.967040002346039
	79 0.967040002346039
	80 0.967120001316071
	81 0.966780002117157
	82 0.970640001296997
	83 0.967040002346039
	84 0.966780002117157
	85 0.966780002117157
	86 0.966420001983643
	87 0.967999999523163
	88 0.969439997673035
	89 0.971120002269745
	90 0.970120000839233
	91 0.969579999446869
	92 0.969820001125336
	93 0.970020000934601
	94 0.969760000705719
	95 0.969579999446869
	96 0.970720000267029
	97 0.969959998130798
	98 0.970479998588562
	99 0.966420001983643
	100 0.970020000934601
};
%\addlegendentry{SAC}
\addplot [semithick, color3]
table {%
	1 0.871296851535353
	100 0.871296851535353
};
%\addlegendentry{Random}
\addplot [semithick, color4]
table {%
	1 0.912342344433749
	100 0.912342344433749
};
%\addlegendentry{ETS}
\addplot [semithick, color5]
table {%
	1 0.825829458409147
	100 0.825829458409147
};
%\addlegendentry{Heur-GD}
\addplot [semithick, color6, dash pattern=on 1pt off 3pt on 3pt off 3pt]
table {%
	1 0.825829458409147
	100 0.825829458409147
};
%\addlegendentry{Heur-LD}
\addplot [semithick, white!49.8039215686275!black, dashed]
table {%
	1 0.825829458409147
	100 0.825829458409147
};
%\addlegendentry{Heur-AT}


\nextgroupplot[title=Seed 3,
height=\figheight,
legend cell align={left},
legend style={
  fill opacity=0.8,
  draw opacity=1,
  text opacity=1,
  at={(0.5,0.09)},
  anchor=south,
  draw=white!80!black
},
minor xtick={25, 75},
minor ytick={},
tick align=outside,
tick pos=left,
%title={Labels: [[5, 4], [1, 2], [9, 6], [7, 0], [3, 8]]},
width=\figwidth,
x grid style={white!69.0196078431373!black},
xlabel={Eval. Steps},
xminorgrids,
xmajorgrids,
xmin=-3.95, xmax=104.95,
xtick style={color=black},
xtick={-25,0,50,100,125},
xticklabels={-25,0,50,100,125},
y grid style={white!69.0196078431373!black},
%ylabel={ACC},
ymajorgrids,
ymin=0.885511360243864, ymax=0.993257558909867,
ytick style={color=black},
ytick={0.88,0.9,0.92,0.94,0.96,0.98,1},
yticklabels={88,90,92,94,96,98,100}
]
\addplot [semithick, color0]
table {%
	1 0.908985142707825
	2 0.919273247718811
	3 0.907157618999481
	4 0.916004521846771
	5 0.931805555820465
	6 0.92931459903717
	7 0.92381007194519
	8 0.918906972408295
	9 0.927060074806213
	10 0.926106586456299
	11 0.928337202072144
	12 0.915986435413361
	13 0.933045864105225
	14 0.937406969070435
	15 0.927328813076019
	16 0.917096893787384
	17 0.929847545623779
	18 0.916021957397461
	19 0.921013565063477
	20 0.928847546577453
	21 0.933138239383698
	22 0.912666659355164
	23 0.924259040355682
	24 0.91892053604126
	25 0.935601418018341
	26 0.932813947200775
	27 0.925939271450043
	28 0.924759037494659
	29 0.918305554389954
	30 0.920958008766174
	31 0.916564600467682
	32 0.924642763137817
	33 0.92555942773819
	34 0.918184750080109
	35 0.918499994277954
	36 0.91073578119278
	37 0.919244828224182
	38 0.921805546283722
	39 0.920670540332794
	40 0.921105940341949
	41 0.928342378139496
	42 0.920231261253357
	43 0.927985785007477
	44 0.921559426784516
	45 0.919930231571197
	46 0.919421188831329
	47 0.923694446086884
	48 0.921361112594605
	49 0.921990303993225
	50 0.930231266021729
	51 0.920115633010864
	52 0.92181459903717
	53 0.928082683086395
	54 0.916846895217896
	55 0.928560078144073
	56 0.936323640346527
	57 0.931708009243011
	58 0.924319767951965
	59 0.911971576213837
	60 0.919671189785004
	61 0.934582679271698
	62 0.929800381660461
	63 0.916634364128113
	64 0.928573639392853
	65 0.924568467140198
	66 0.927411494255066
	67 0.932360460758209
	68 0.927781648635864
	69 0.930388236045837
	70 0.929388236999512
	71 0.913985137939453
	72 0.918763558864594
	73 0.927869501113892
	74 0.915749349594116
	75 0.924430227279663
	76 0.930351414680481
	77 0.931322989463806
	78 0.927772603034973
	79 0.913370156288147
	80 0.910801029205322
	81 0.92433268070221
	82 0.930532295703888
	83 0.920244178771973
	84 0.911082680225372
	85 0.924652450084686
	86 0.919282295703888
	87 0.936642756462097
	88 0.909902451038361
	89 0.932744824886322
	90 0.912406973838806
	91 0.925721569061279
	92 0.931481261253357
	93 0.910402452945709
	94 0.922689919471741
	95 0.910786168575287
	96 0.930499351024628
	97 0.911560075283051
	98 0.916027777194977
	99 0.914430229663849
	100 0.924851415157318
};
%\addlegendentry{DQN}
\addplot [semithick, color1]
table {%
	1 0.903962535858154
	2 0.901369512081146
	3 0.901369512081146
	4 0.901369512081146
	5 0.919872088432312
	6 0.932754521369934
	7 0.933726096153259
	8 0.925295865535736
	9 0.921295866966248
	10 0.933240308761597
	11 0.923596253395081
	12 0.937240307331085
	13 0.929240310192108
	14 0.928268735408783
	15 0.933018088340759
	16 0.931531653404236
	17 0.938212530612946
	18 0.932726745605469
	19 0.938698966503143
	20 0.92982364654541
	21 0.9300865650177
	22 0.933851420879364
	23 0.940476746559143
	24 0.936698968410492
	25 0.941476745605469
	26 0.939768090248108
	27 0.930532302856445
	28 0.926754524707794
	29 0.937114987373352
	30 0.935573644638061
	31 0.932851421833038
	32 0.941185402870178
	33 0.931796514987946
	34 0.931032302379608
	35 0.93579651594162
	36 0.930203490257263
	37 0.936421189308166
	38 0.931379849910736
	39 0.936213181018829
	40 0.938226745128632
	41 0.93579651594162
	42 0.93776421546936
	43 0.940352072715759
	44 0.940352072715759
	45 0.933208658695221
	46 0.942583334445953
	47 0.941481266021728
	48 0.939666669368744
	49 0.940795862674713
	50 0.944814596176147
	51 0.940768737792969
	52 0.940671186447143
	53 0.941897931098938
	54 0.941013562679291
	55 0.946629192829132
	56 0.942999999523163
	57 0.948443789482117
	58 0.942999999523163
	59 0.940795862674713
	60 0.945527124404907
	61 0.943925056457519
	62 0.941411490440369
	63 0.944328155517578
	64 0.943226087093353
	65 0.948443789482117
	66 0.944814596176147
	67 0.950258386135101
	68 0.942027118206024
	69 0.943226087093353
	70 0.942027118206024
	71 0.944328155517578
	72 0.942027118206024
	73 0.950258386135101
	74 0.946142752170563
	75 0.942027118206024
	76 0.950258386135101
	77 0.943226087093353
	78 0.946142752170563
	79 0.946142752170563
	80 0.942027118206024
	81 0.946142752170563
	82 0.950258386135101
	83 0.942027118206024
	84 0.950258386135101
	85 0.946142752170563
	86 0.946142752170563
	87 0.946142752170563
	88 0.950258386135101
	89 0.946142752170563
	90 0.946142752170563
	91 0.946142752170563
	92 0.950258386135101
	93 0.946142752170563
	94 0.950258386135101
	95 0.950258386135101
	96 0.946142752170563
	97 0.946142752170563
	98 0.950258386135101
	99 0.942027118206024
	100 0.946142752170563
};
%\addlegendentry{A2C}
\addplot [semithick, color2]
table {%
	1 0.909700000286102
	2 0.909700000286102
	3 0.909700000286102
	4 0.909700000286102
	5 0.909700000286102
	6 0.909700000286102
	7 0.909700000286102
	8 0.909700000286102
	9 0.905379998683929
	10 0.914039998054504
	11 0.91404000043869
	12 0.949640002250671
	13 0.946459996700287
	14 0.919659998416901
	15 0.948959999084473
	16 0.966519999504089
	17 0.968720004558563
	18 0.983560004234314
	19 0.968720004558563
	20 0.986180005073547
	21 0.986180005073547
	22 0.986180005073547
	23 0.980299999713898
	24 0.981400001049042
	25 0.982500004768372
	26 0.982500004768372
	27 0.985300004482269
	28 0.948160002231598
	29 0.986020004749298
	30 0.96724000453949
	31 0.972600004673004
	32 0.973880002498627
	33 0.972600004673004
	34 0.96724000453949
	35 0.972600004673004
	36 0.984740006923676
	37 0.972600004673004
	38 0.972600004673004
	39 0.972600004673004
	40 0.972600004673004
	41 0.973880002498627
	42 0.973880002498627
	43 0.973880002498627
	44 0.973880002498627
	45 0.975400004386902
	46 0.975400004386902
	47 0.976260001659393
	48 0.976260001659393
	49 0.975400004386902
	50 0.976260001659393
	51 0.985780003070831
	52 0.95789999961853
	53 0.959680001735687
	54 0.969200003147125
	55 0.984120004177094
	56 0.983400003910065
	57 0.983140001296997
	58 0.971980001926422
	59 0.986780004501343
	60 0.984120004177094
	61 0.987600002288818
	62 0.986780004501343
	63 0.986780004501343
	64 0.974640002250671
	65 0.974640002250671
	66 0.966980001926422
	67 0.974640002250671
	68 0.986780004501343
	69 0.98650000333786
	70 0.986780004501343
	71 0.986780004501343
	72 0.986780004501343
	73 0.974640002250671
	74 0.986780004501343
	75 0.986780004501343
	76 0.974640002250671
	77 0.986780004501343
	78 0.972720003128052
	79 0.983880002498627
	80 0.985800001621246
	81 0.972720003128052
	82 0.985800001621246
	83 0.985800001621246
	84 0.983600001335144
	85 0.985800001621246
	86 0.984580004215241
	87 0.987560002803803
	88 0.983359999656677
	89 0.95364000082016
	90 0.973659999370575
	91 0.974640002250671
	92 0.971980001926422
	93 0.979880001544952
	94 0.983140001296997
	95 0.971980001926422
	96 0.971740000247955
	97 0.986160004138947
	98 0.972579998970032
	99 0.988360004425049
	100 0.972579998970032
};
%\addlegendentry{SAC}
\addplot [semithick, color3]
table {%
	1 0.890408914728682
	100 0.890408914728682
};
%\addlegendentry{Random}
\addplot [semithick, color4]
table {%
	1 0.893317183462532
	100 0.893317183462532
};
%\addlegendentry{ETS}
\addplot [semithick, color5]
table {%
	1 0.952642118863049
	100 0.952642118863049
};
%\addlegendentry{Heur-GD}
\addplot [semithick, color6, dash pattern=on 1pt off 3pt on 3pt off 3pt]
table {%
	1 0.952642118863049
	100 0.952642118863049
};
%\addlegendentry{Heur-LD}
\addplot [semithick, white!49.8039215686275!black, dashed]
table {%
	1 0.952642118863049
	100 0.952642118863049
};
%\addlegendentry{Heur-AT}



\nextgroupplot[title=Seed 4,
height=\figheight,
legend cell align={left},
legend style={
  fill opacity=0.8,
  draw opacity=1,
  text opacity=1,
  at={(0.03,0.03)},
  anchor=south west,
  draw=white!80!black
},
minor xtick={25, 75},
minor ytick={},
tick align=outside,
tick pos=left,
%title={Labels: [[3, 8], [4, 9], [2, 6], [0, 1], [5, 7]]},
width=\figwidth,
x grid style={white!69.0196078431373!black},
xlabel={Eval. Steps},
xminorgrids,
xmajorgrids,
xmin=-3.95, xmax=104.95,
xtick style={color=black},
xtick={-25,0,50,100,125},
xticklabels={-25,0,50,100,125},
y grid style={white!69.0196078431373!black},
ylabel={ACC (\%)},
ymajorgrids,
ymin=0.858147337913513, ymax=0.95, 
ytick style={color=black},
ytick={0.86, 0.88, 0.9,0.92,0.94,0.96},
yticklabels={86, 88, 90,92,94,96}
]
\addplot [semithick, color0]
table {%
	1 0.920202186107635
	2 0.921896319389343
	3 0.924375891685486
	4 0.925927443504334
	5 0.926288747787476
	6 0.93278945684433
	7 0.932207942008972
	8 0.932470872402191
	9 0.928564040660858
	10 0.924558382034302
	11 0.931553471088409
	12 0.929582829475403
	13 0.929671201705933
	14 0.925383093357086
	15 0.926511151790619
	16 0.930267097949982
	17 0.92332243680954
	18 0.933574614524841
	19 0.933791589736938
	20 0.931290335655212
	21 0.923562700748443
	22 0.924367020130157
	23 0.929790318012238
	24 0.932659602165222
	25 0.929018821716308
	26 0.925916228294372
	27 0.929801135063171
	28 0.926916947364807
	29 0.923028447628021
	30 0.925937695503235
	31 0.925885019302368
	32 0.931518025398254
	33 0.931299753189087
	34 0.931116714477539
	35 0.928093838691711
	36 0.928416225910187
	37 0.923616757392883
	38 0.920336740016937
	39 0.928972868919373
	40 0.933254318237305
	41 0.92645672082901
	42 0.924253034591675
	43 0.9258686876297
	44 0.925684938430786
	45 0.926427676677704
	46 0.923387560844421
	47 0.927771117687225
	48 0.933750169277191
	49 0.922857069969177
	50 0.918554248809815
	51 0.923891603946686
	52 0.917176074981689
	53 0.929895966053009
	54 0.924373099803925
	55 0.930455913543701
	56 0.929074063301086
	57 0.929046902656555
	58 0.92842798948288
	59 0.92399480342865
	60 0.924296052455902
	61 0.926871256828308
	62 0.919563875198364
	63 0.925717544555664
	64 0.932200129032135
	65 0.926814641952515
	66 0.931867277622223
	67 0.925625643730164
	68 0.924564962387085
	69 0.909733281135559
	70 0.921791617870331
	71 0.912906556129455
	72 0.925297529697418
	73 0.909957332611084
	74 0.908516972064972
	75 0.908309705257416
	76 0.908466589450836
	77 0.908894512653351
	78 0.910878083705902
	79 0.929756581783295
	80 0.921753494739532
	81 0.908974652290344
	82 0.92508712053299
	83 0.912866084575653
	84 0.909458940029144
	85 0.91332715511322
	86 0.915056965351105
	87 0.925276401042938
	88 0.925370342731476
	89 0.912643926143646
	90 0.930628039836884
	91 0.923951199054718
	92 0.926084415912628
	93 0.922337806224823
	94 0.916916291713714
	95 0.905735847949982
	96 0.90629634141922
	97 0.922304117679596
	98 0.928993213176727
	99 0.918944504261017
	100 0.907640323638916
};
%\addlegendentry{DQN}
\addplot [semithick, color1]
table {%
	1 0.911717159748077
	2 0.902619230747223
	3 0.902619230747223
	4 0.902619230747223
	5 0.929462602138519
	6 0.934100966453552
	7 0.936752285957337
	8 0.935110723972321
	9 0.935704402923584
	10 0.935426626205444
	11 0.937959182262421
	12 0.934832947254181
	13 0.936020305156708
	14 0.933368985652924
	15 0.937185189723969
	16 0.938056058883667
	17 0.936560714244843
	18 0.937101709842682
	19 0.934694645404816
	20 0.939761939048767
	21 0.942413258552551
	22 0.938458166122437
	23 0.940673658847809
	24 0.93837468624115
	25 0.937101709842682
	26 0.940896027088165
	27 0.937271394729614
	28 0.937240598201752
	29 0.94225248336792
	30 0.935704407691956
	31 0.937714927196503
	32 0.939317202568054
	33 0.938405482769012
	34 0.936106510162354
	35 0.938405482769012
	36 0.939761939048767
	37 0.938183114528656
	38 0.939539570808411
	39 0.935290467739105
	40 0.939400682449341
	41 0.937049026489258
	42 0.934780850410461
	43 0.940673658847809
	44 0.940673658847809
	45 0.938183114528656
	46 0.937792809009552
	47 0.934775376319885
	48 0.936658720970154
	49 0.940673658847809
	50 0.938044226169586
	51 0.937931697368622
	52 0.935039050579071
	53 0.936266074180603
	54 0.938843321800232
	55 0.93906578540802
	56 0.934017491340637
	57 0.937680280208588
	58 0.936517491340637
	59 0.934672260284424
	60 0.932380568981171
	61 0.934272541999817
	62 0.93906578540802
	63 0.935131986141205
	64 0.934644808769226
	65 0.937222056388855
	66 0.938183114528656
	67 0.933052496910095
	68 0.932339510917664
	69 0.926305689811707
	70 0.928882937431336
	71 0.924539849758148
	72 0.922387547492981
	73 0.923652741909027
	74 0.937625160217285
	75 0.930223832130432
	76 0.931294977664948
	77 0.922387547492981
	78 0.926046407222748
	79 0.922387547492981
	80 0.926305689811706
	81 0.920370290279388
	82 0.930223832130432
	83 0.937463295459747
	84 0.929277720451355
	85 0.920881445407867
	86 0.936719222068787
	87 0.920881445407867
	88 0.937790367603302
	89 0.927004568576813
	90 0.924378440380096
	91 0.919999570846558
	92 0.930611205101013
	93 0.926300504207611
	94 0.928087728023529
	95 0.91743827342987
	96 0.932666461467743
	97 0.927797186374664
	98 0.92842342376709
	99 0.929202520847321
	100 0.927955701351166
};
%\addlegendentry{A2C}
\addplot [semithick, color2]
table {%
	1 0.868099999427795
	2 0.868099999427795
	3 0.868099999427795
	4 0.868099999427795
	5 0.868099999427795
	6 0.868099999427795
	7 0.862160000801086
	8 0.887760002613068
	9 0.90552000284195
	10 0.887999999523163
	11 0.887999999523163
	12 0.894880001544952
	13 0.890399997234345
	14 0.889199998378754
	15 0.902859997749329
	16 0.906180000305176
	17 0.904620001316071
	18 0.901820001602173
	19 0.8978600025177
	20 0.891159996986389
	21 0.893639998435974
	22 0.897340004444122
	23 0.896120004653931
	24 0.897480003833771
	25 0.897380006313324
	26 0.906020007133484
	27 0.897060000896454
	28 0.903220002651215
	29 0.913839998245239
	30 0.908959999084473
	31 0.8842799949646
	32 0.877599997520447
	33 0.893940000534058
	34 0.900339996814728
	35 0.899059996604919
	36 0.906119995117187
	37 0.906119995117187
	38 0.87945999622345
	39 0.862419993877411
	40 0.882439994812012
	41 0.882799997329712
	42 0.880959994792938
	43 0.881159994602203
	44 0.885919995307922
	45 0.884739995002747
	46 0.873699996471405
	47 0.878159997463226
	48 0.897699999809265
	49 0.891220002174377
	50 0.894019997119904
	51 0.89422000169754
	52 0.903800003528595
	53 0.899859998226166
	54 0.895020000934601
	55 0.885139999389648
	56 0.899539999961853
	57 0.888339998722076
	58 0.881800005435944
	59 0.889100003242493
	60 0.87400000333786
	61 0.884420001506805
	62 0.87960000038147
	63 0.894879999160767
	64 0.873040001392364
	65 0.880979998111725
	66 0.878039999008179
	67 0.894459998607635
	68 0.862599999904633
	69 0.889219999313354
	70 0.871380002498627
	71 0.878559999465942
	72 0.86518000125885
	73 0.864760000705719
	74 0.887380001544952
	75 0.883300001621246
	76 0.872779998779297
	77 0.889559998512268
	78 0.885939998626709
	79 0.895639998912811
	80 0.893999998569489
	81 0.886800000667572
	82 0.889400000572204
	83 0.888680000305176
	84 0.893999998569489
	85 0.894379999637604
	86 0.894379999637604
	87 0.89039999961853
	88 0.897479999065399
	89 0.893039999008179
	90 0.889559998512268
	91 0.889559998512268
	92 0.896400001049042
	93 0.882740004062653
	94 0.889559998512268
	95 0.901220002174377
	96 0.893039999008179
	97 0.898880000114441
	98 0.895939998626709
	99 0.899300000667572
	100 0.895939998626709
};
%\addlegendentry{SAC}
\addplot [semithick, color3]
table {%
	1 0.913697124266064
	100 0.913697124266064
};
%\addlegendentry{Random}
\addplot [semithick, color4]
table {%
	1 0.905164085947425
	100 0.905164085947425
};
%\addlegendentry{ETS}
\addplot [semithick, color5]
table {%
	1 0.916400743643145
	100 0.916400743643145
};
%\addlegendentry{Heur-GD}
\addplot [semithick, color6, dash pattern=on 1pt off 3pt on 3pt off 3pt]
table {%
	1 0.92662679292127
	100 0.92662679292127
};
%\addlegendentry{Heur-LD}
\addplot [semithick, white!49.8039215686275!black, dashed]
table {%
	1 0.916400743643145
	100 0.916400743643145
};
%\addlegendentry{Heur-AT}



\nextgroupplot[title=Seed 5,
height=\figheight,
legend cell align={left},
legend style={
  fill opacity=0.8,
  draw opacity=1,
  text opacity=1,
  at={(0.97,0.03)},
  anchor=south east,
  draw=white!80!black
},
minor xtick={25, 75},
minor ytick={},
tick align=outside,
tick pos=left,
%title={Labels: [[9, 5], [2, 4], [7, 1], [0, 8], [6, 3]]},
width=\figwidth,
x grid style={white!69.0196078431373!black},
xlabel={Eval. Steps},
xminorgrids,
xmajorgrids,
xmin=-3.95, xmax=104.95,
xtick style={color=black},
xtick={-25,0,50,100,125},
xticklabels={-25,0,50,100,125},
y grid style={white!69.0196078431373!black},
%ylabel={ACC},
ymajorgrids,
ymin=0.90, ymax=0.95,
ytick style={color=black},
ytick={0.9,0.91,0.92,0.93,0.94,0.95},
yticklabels={90,91,92,93,94,95}
]
\addplot [semithick, color0]
table {%
1 0.931729527314504
2 0.937117218971252
3 0.937117218971252
4 0.937117218971252
5 0.942877070109049
6 0.934570733706157
7 0.929683101177216
8 0.942782874902089
9 0.937807408968608
10 0.934570733706157
11 0.929683101177216
12 0.939634033044179
13 0.929683101177216
14 0.929507434368134
15 0.926358592510223
16 0.938103334108988
17 0.93698548078537
18 0.93433692852656
19 0.928773339589437
20 0.939721477031708
21 0.939809310436249
22 0.93869145711263
23 0.941018370787303
24 0.939900517463684
25 0.941018370787303
26 0.942136224110921
27 0.937485770384471
28 0.935158856709798
29 0.942136224110921
30 0.937474048137665
31 0.939809310436249
32 0.942136224110921
33 0.937573603789012
34 0.936276710033417
35 0.937485770384471
36 0.939809310436249
37 0.94325407743454
38 0.939721477031708
39 0.938206644852956
40 0.940855197111766
41 0.934674044450124
42 0.939324498176575
43 0.93539491891861
44 0.935028020540873
45 0.931465339660644
46 0.931465339660644
47 0.9312779545784
48 0.931465339660644
49 0.934177509943644
50 0.931465339660644
51 0.931465339660644
52 0.931465339660644
53 0.931465339660644
54 0.9312779545784
55 0.931465339660644
56 0.931465339660644
57 0.931465339660644
58 0.931465339660644
59 0.931465339660644
60 0.931465339660644
61 0.933546725908915
62 0.931465339660644
63 0.931090569496155
64 0.931090569496155
65 0.9312779545784
66 0.931465339660644
67 0.931090569496155
68 0.9312779545784
69 0.931465339660644
70 0.931465339660644
71 0.931465339660644
72 0.931465339660644
73 0.931465339660644
74 0.9312779545784
75 0.9312779545784
76 0.929871662457784
77 0.9312779545784
78 0.9312779545784
79 0.931465339660644
80 0.931465339660644
81 0.925737639268239
82 0.9312779545784
83 0.931465339660644
84 0.933171955744426
85 0.93335934082667
86 0.9312779545784
87 0.931090569496155
88 0.931090569496155
89 0.935440727074941
90 0.9312779545784
91 0.9312779545784
92 0.931090569496155
93 0.933546725908915
94 0.9312779545784
95 0.930262192090352
96 0.929502296447754
97 0.929127526283264
98 0.933665068944295
99 0.931583682696025
100 0.928180805842082
};
%\addlegendentry{a2c, gae, envs=10, lr=0.0001, act=tanh}
\addplot [semithick, color1]
table {%
1 0.938462579250336
2 0.929322858651479
3 0.93185932636261
4 0.919784116744995
5 0.923728128274282
6 0.919772783915202
7 0.940860076745351
8 0.926340981324514
9 0.935865338643392
10 0.943523891766866
11 0.928873646259308
12 0.932237732410431
13 0.946447324752808
14 0.938856673240662
15 0.935454722245534
16 0.926658300558726
17 0.931981674830119
18 0.935190240542094
19 0.916622002919515
20 0.919737533728282
21 0.913096984227498
22 0.922481528917948
23 0.908880762259165
24 0.911864976088206
25 0.908803669611613
26 0.9075230717659
27 0.914152542750041
28 0.916906170050303
29 0.938629158337911
30 0.921688520908356
31 0.931728839874268
32 0.942857221762339
33 0.93785594701767
34 0.930912121136983
35 0.932871409257253
36 0.927952154477437
37 0.927952154477437
38 0.932871409257253
39 0.938629158337911
40 0.934254634380341
41 0.934254634380341
42 0.924203590552012
43 0.91836435397466
44 0.937225651741028
45 0.93217324813207
46 0.933058794339498
47 0.93217324813207
48 0.940622886021932
49 0.91836435397466
50 0.934254634380341
51 0.928101881345113
52 0.928415397802989
53 0.937099472681681
54 0.933337430159251
55 0.935110624631246
56 0.932451883951823
57 0.914488629500071
58 0.926863654454549
59 0.925524310270945
60 0.92686740954717
61 0.936768265565236
62 0.925134134292603
63 0.928985778490702
64 0.932591048876445
65 0.932025988896688
66 0.917664090792338
67 0.939527535438538
68 0.933797895908356
69 0.918236402670542
70 0.93553394873937
71 0.937721188863118
72 0.944841833909353
73 0.928742865721385
74 0.944841833909353
75 0.934656186898549
76 0.921211969852448
77 0.929280662536621
78 0.937721188863118
79 0.942302890618642
80 0.937181663513183
81 0.92693882783254
82 0.943035491307577
83 0.934801348050435
84 0.943035491307577
85 0.925608591238658
86 0.935625203450521
87 0.935903628667196
88 0.923314925034841
89 0.934097286065419
90 0.917586783568064
91 0.931163505713145
92 0.923850417137146
93 0.925524310270945
94 0.930219646294912
95 0.923604933420817
96 0.934097286065419
97 0.925742328166962
98 0.927723395824433
99 0.927723395824433
100 0.927723395824433
};
%\addlegendentry{DQN, n train envs=30}
\addplot [semithick, color2]
table {%
1 0.919256607570398
100 0.919256607570398
};
%\addlegendentry{Random}
\addplot [semithick, color3]
table {%
1 0.907289510340143
100 0.907289510340143
};
%\addlegendentry{ETS}
\addplot [semithick, color4]
table {%
1 0.917972439382205
100 0.917972439382205
};
%\addlegendentry{Heur-GD}
\addplot [semithick, color5, dash pattern=on 1pt off 3pt on 3pt off 3pt]
table {%
1 0.917972439382205
100 0.917972439382205
};
%\addlegendentry{Heur-LD}
\addplot [semithick, color6, dashed]
table {%
1 0.917972439382205
100 0.917972439382205
};
%\addlegendentry{Heur-AT}



\nextgroupplot[title=Seed 6,
height=\figheight,
legend cell align={left},
legend style={fill opacity=0.8, draw opacity=1, text opacity=1, draw=white!80!black},
minor xtick={25, 75},
minor ytick={},
tick align=outside,
tick pos=left,
%title={Labels: [[8, 1], [7, 0], [6, 5], [2, 4], [3, 9]]},
width=\figwidth,
x grid style={white!69.0196078431373!black},
xlabel={Eval. Steps},
xminorgrids,
xmajorgrids,
xmin=-3.95, xmax=104.95,
xtick style={color=black},
xtick={-25,0,50,100,125},
xticklabels={-25,0,50,100,125},
y grid style={white!69.0196078431373!black},
%ylabel={ACC},
ymajorgrids,
ymin=0.86, ymax=0.96,
ytick style={color=black},
ytick={0.86,0.88,0.9,0.92,0.94,0.96},
yticklabels={86,88,90,92,94,96}
]
\addplot [semithick, color0]
table {%
	1 0.930722200870514
	2 0.936096985340118
	3 0.935214531421661
	4 0.939554686546326
	5 0.944134604930878
	6 0.950024144649506
	7 0.937071206569672
	8 0.943242862224579
	9 0.936882400512695
	10 0.944773888587952
	11 0.939938724040985
	12 0.952454426288605
	13 0.943014719486236
	14 0.940661566257477
	15 0.915400938987732
	16 0.947147722244263
	17 0.937870402336121
	18 0.9355149269104
	19 0.934913132190704
	20 0.931308434009552
	21 0.930855939388275
	22 0.939125275611877
	23 0.928815424442291
	24 0.935912284851074
	25 0.933418810367584
	26 0.932542247772217
	27 0.932683966159821
	28 0.932457284927368
	29 0.936435561180115
	30 0.931336631774902
	31 0.934936127662659
	32 0.925628607273102
	33 0.931905825138092
	34 0.914039967060089
	35 0.907361762523651
	36 0.925383667945862
	37 0.941073174476623
	38 0.941863934993744
	39 0.937441923618317
	40 0.934209570884705
	41 0.930895130634308
	42 0.920576612949371
	43 0.92158807516098
	44 0.93119909286499
	45 0.92411144733429
	46 0.933895845413208
	47 0.922361342906952
	48 0.906788775920868
	49 0.925705206394196
	50 0.921476957798004
	51 0.919678130149841
	52 0.939757430553436
	53 0.933425309658051
	54 0.936025657653809
	55 0.934545586109161
	56 0.93272438287735
	57 0.921624245643616
	58 0.906200218200684
	59 0.937418992519379
	60 0.921995735168457
	61 0.936298561096191
	62 0.918848567008972
	63 0.929568939208984
	64 0.932894215583801
	65 0.922026011943817
	66 0.92334733247757
	67 0.913518941402435
	68 0.888998968601227
	69 0.922282812595368
	70 0.935202374458313
	71 0.923606991767883
	72 0.911089503765106
	73 0.9185382604599
	74 0.907463524341583
	75 0.934335405826568
	76 0.937859942913055
	77 0.935603830814362
	78 0.92369713306427
	79 0.898801875114441
	80 0.936217942237854
	81 0.934787509441376
	82 0.906857922077179
	83 0.914701709747314
	84 0.914245340824127
	85 0.927465226650238
	86 0.933781156539917
	87 0.915534055233002
	88 0.923473563194275
	89 0.938763310909271
	90 0.911316254138947
	91 0.928034269809723
	92 0.939117288589478
	93 0.928278830051422
	94 0.935780630111694
	95 0.930896911621094
	96 0.938319563865662
	97 0.930362174510956
	98 0.937758581638336
	99 0.938752465248108
	100 0.93318400144577
};
%\addlegendentry{DQN}
\addplot [semithick, color1]
table {%
	1 0.907577283382416
	2 0.894921326637268
	3 0.894921326637268
	4 0.894921326637268
	5 0.90094810962677
	6 0.903636569976807
	7 0.906610579490662
	8 0.919651484489441
	9 0.915575406551361
	10 0.907605164051056
	11 0.921398355960846
	12 0.902807016372681
	13 0.909952504634857
	14 0.896133997440338
	15 0.91039128780365
	16 0.913302409648895
	17 0.902850298881531
	18 0.903880734443664
	19 0.898092579841614
	20 0.90524023771286
	21 0.914563782215118
	22 0.907416124343872
	23 0.898092579841614
	24 0.898092579841614
	25 0.907416124343872
	26 0.907416124343872
	27 0.898092579841614
	28 0.898092579841614
	29 0.898092579841614
	30 0.898092579841614
	31 0.898683142662048
	32 0.898092579841614
	33 0.898092579841614
	34 0.898092579841614
	35 0.898387861251831
	36 0.898683142662048
	37 0.898683142662048
	38 0.898978424072266
	39 0.899273705482483
	40 0.899273705482483
	41 0.899273705482483
	42 0.898683142662048
	43 0.899273705482483
	44 0.899273705482483
	45 0.898683142662048
	46 0.898978424072266
	47 0.898683142662048
	48 0.883306429386139
	49 0.893655998706818
	50 0.883306429386139
	51 0.88833357334137
	52 0.883011147975922
	53 0.888628854751587
	54 0.883601710796356
	55 0.877984004020691
	56 0.888642721176148
	57 0.883011147975922
	58 0.879848880767822
	59 0.8801580286026
	60 0.878293151855469
	61 0.8801580286026
	62 0.882185921669006
	63 0.878293151855469
	64 0.878293151855469
	65 0.883320295810699
	66 0.87295686006546
	67 0.87295686006546
	68 0.87295686006546
	69 0.87295686006546
	70 0.87295686006546
	71 0.878293151855469
	72 0.887522213459015
	73 0.884962451457977
	74 0.890298743247986
	75 0.896751275062561
	76 0.887522213459015
	77 0.878293151855469
	78 0.890298743247986
	79 0.878293151855469
	80 0.890298743247986
	81 0.87295686006546
	82 0.883629443645477
	83 0.890298743247986
	84 0.899639635086059
	85 0.882297751903534
	86 0.899527804851532
	87 0.890298743247986
	88 0.904864096641541
	89 0.902304334640503
	90 0.887522213459015
	91 0.890298743247986
	92 0.907640626430511
	93 0.890298743247986
	94 0.90404483795166
	95 0.916869688034058
	96 0.916869688034058
	97 0.908049437999725
	98 0.915537996292114
	99 0.920874288082123
	100 0.916869688034058
};
%\addlegendentry{A2C}
\addplot [semithick, color2]
table {%
	1 0.899300003051758
	2 0.899300003051758
	3 0.899300003051758
	4 0.899300003051758
	5 0.899300003051758
	6 0.899300003051758
	7 0.899300003051758
	8 0.853739993572235
	9 0.853459997177124
	10 0.902619996070862
	11 0.92793999671936
	12 0.917359993457794
	13 0.858919994831085
	14 0.897079994678497
	15 0.880999999046326
	16 0.904799995422363
	17 0.877219994068146
	18 0.868479988574982
	19 0.89931999206543
	20 0.897179989814758
	21 0.909519994258881
	22 0.906639993190765
	23 0.913039991855621
	24 0.902699987888336
	25 0.897219989299774
	26 0.886939992904663
	27 0.893839991092682
	28 0.898939988613129
	29 0.888659989833832
	30 0.885199995040894
	31 0.881839995384216
	32 0.888779995441437
	33 0.928639998435974
	34 0.887259993553162
	35 0.89813999414444
	36 0.887479994297028
	37 0.89813999414444
	38 0.887279994487762
	39 0.887279994487762
	40 0.895599994659424
	41 0.900919992923736
	42 0.894879992008209
	43 0.894879992008209
	44 0.887279994487762
	45 0.887279994487762
	46 0.894879992008209
	47 0.88163999080658
	48 0.903839988708496
	49 0.908139991760254
	50 0.903839988708496
	51 0.897419991493225
	52 0.889999992847443
	53 0.914259986877441
	54 0.914259986877441
	55 0.916319990158081
	56 0.925
	57 0.916319990158081
	58 0.899779994487763
	59 0.903359994888306
	60 0.896299993991852
	61 0.887459995746613
	62 0.881339993476868
	63 0.882519993782044
	64 0.927739999294281
	65 0.910619993209839
	66 0.900739996433258
	67 0.899619996547699
	68 0.891359996795654
	69 0.890179991722107
	70 0.899739997386932
	71 0.899739997386932
	72 0.899619996547699
	73 0.899619996547699
	74 0.911259996891022
	75 0.908059995174408
	76 0.876059999465942
	77 0.884879992008209
	78 0.860339994430542
	79 0.917979991436005
	80 0.897479989528656
	81 0.8964399933815
	82 0.887139992713928
	83 0.925439989566803
	84 0.924199991226196
	85 0.916719992160797
	86 0.894779989719391
	87 0.918839991092682
	88 0.919279987812042
	89 0.927759997844696
	90 0.918839991092682
	91 0.917979991436005
	92 0.903659992218018
	93 0.900939991474152
	94 0.917979991436005
	95 0.8785799908638
	96 0.91585999250412
	97 0.884499995708466
	98 0.899999992847442
	99 0.898439991474152
	100 0.922399995326996
};
%\addlegendentry{SAC}
\addplot [semithick, color3]
table {%
	1 0.917958133258011
	100 0.917958133258011
};
%\addlegendentry{Random}
\addplot [semithick, color4]
table {%
	1 0.919890937296078
	100 0.919890937296078
};
%\addlegendentry{ETS}
\addplot [semithick, color5]
table {%
	1 0.915537550984307
	100 0.915537550984307
};
%\addlegendentry{Heur-GD}
\addplot [semithick, color6, dash pattern=on 1pt off 3pt on 3pt off 3pt]
table {%
	1 0.870281256890804
	100 0.870281256890804
};
%\addlegendentry{Heur-LD}
\addplot [semithick, white!49.8039215686275!black, dashed]
table {%
	1 0.915537550984307
	100 0.915537550984307
};
%\addlegendentry{Heur-AT}



\nextgroupplot[title=Seed 7,
height=\figheight,
legend cell align={left},
legend style={
  fill opacity=0.8,
  draw opacity=1,
  text opacity=1,
  at={(0.5,0.09)},
  anchor=south,
  draw=white!80!black
},
minor xtick={25, 75},
minor ytick={},
tick align=outside,
tick pos=left,
%title={Labels: [[8, 5], [0, 2], [1, 9], [7, 3], [6, 4]]},
width=\figwidth,
x grid style={white!69.0196078431373!black},
xlabel={Eval. Steps},
xminorgrids,
xmajorgrids,
xmin=-3.95, xmax=104.95,
xtick style={color=black},
xtick={-25,0,50,100,125},
xticklabels={-25,0,50,100,125},
y grid style={white!69.0196078431373!black},
%ylabel={ACC},
ymajorgrids,
ymin=0.885420464722211, ymax=0.965022746722308,
ytick style={color=black},
ytick={0.88,0.9,0.92,0.94,0.96,0.98},
yticklabels={88,90,92,94,96,98}
]
\addplot [semithick, color0]
table {%
	1 0.924841277599335
	2 0.929526827335358
	3 0.930327990055084
	4 0.931702213287353
	5 0.921860957145691
	6 0.928373532295227
	7 0.927709972858429
	8 0.925899450778961
	9 0.921747510433197
	10 0.927996134757996
	11 0.918691599369049
	12 0.93013334274292
	13 0.92781106710434
	14 0.926097729206085
	15 0.930164132118225
	16 0.927510731220245
	17 0.922988827228546
	18 0.929132158756256
	19 0.928696069717407
	20 0.929051394462585
	21 0.93341301202774
	22 0.926546959877014
	23 0.925016405582428
	24 0.927078635692596
	25 0.924322869777679
	26 0.928027443885803
	27 0.933737735748291
	28 0.931074163913727
	29 0.92974860906601
	30 0.931776692867279
	31 0.930958535671234
	32 0.926067311763763
	33 0.931577570438385
	34 0.932911400794983
	35 0.922479078769684
	36 0.930874993801117
	37 0.935327386856079
	38 0.927650244235992
	39 0.929340658187866
	40 0.920480332374573
	41 0.927025694847107
	42 0.932649915218353
	43 0.931115479469299
	44 0.930591354370117
	45 0.930914335250854
	46 0.934185612201691
	47 0.926754698753357
	48 0.932573564052582
	49 0.930073335170746
	50 0.932503733634949
	51 0.931569747924805
	52 0.930127799510956
	53 0.931331701278686
	54 0.930460381507874
	55 0.92915563583374
	56 0.934733779430389
	57 0.933253374099732
	58 0.931107740402222
	59 0.935002777576446
	60 0.930212564468384
	61 0.933653779029846
	62 0.935274319648743
	63 0.934486033916473
	64 0.932758479118347
	65 0.931562802791595
	66 0.932697465419769
	67 0.930892131328583
	68 0.929596350193024
	69 0.933314387798309
	70 0.932974398136139
	71 0.933283398151398
	72 0.933976709842682
	73 0.930373513698578
	74 0.933058695793152
	75 0.931262860298157
	76 0.928498425483704
	77 0.928552968502045
	78 0.927358248233795
	79 0.932741942405701
	80 0.931948025226593
	81 0.931422970294952
	82 0.930589919090271
	83 0.93123738527298
	84 0.930667450428009
	85 0.933668229579925
	86 0.933005537986755
	87 0.931892201900482
	88 0.929935178756714
	89 0.93190877199173
	90 0.932719867229462
	91 0.93035774230957
	92 0.93263302564621
	93 0.931960406303406
	94 0.933042597770691
	95 0.93335902929306
	96 0.934617991447449
	97 0.931215949058533
	98 0.930342876911163
	99 0.929229986667633
	100 0.931668019294739
};
%\addlegendentry{DQN}
\addplot [semithick, color1]
table {%
	1 0.926097009181976
	2 0.913716804981232
	3 0.913716804981232
	4 0.913716804981232
	5 0.922926385402679
	6 0.921183180809021
	7 0.922541751861572
	8 0.922627608776093
	9 0.920675175189972
	10 0.921430766582489
	11 0.923760540485382
	12 0.92338320016861
	13 0.919478332996368
	14 0.917256362438202
	15 0.920828096866608
	16 0.922163956165314
	17 0.923490879535675
	18 0.922109777927399
	19 0.922487573623657
	20 0.923482382297516
	21 0.925704352855682
	22 0.922510414123535
	23 0.927201972007751
	24 0.925852208137512
	25 0.924169890880585
	26 0.927602608203888
	27 0.923081746101379
	28 0.92173198223114
	29 0.928003244400024
	30 0.925859272480011
	31 0.924192731380463
	32 0.926801335811615
	33 0.924741222858429
	34 0.924655618667603
	35 0.924741222858429
	36 0.924655618667603
	37 0.927534875869751
	38 0.925620746612549
	39 0.927449271678925
	40 0.931199057102203
	41 0.926373314857483
	42 0.929353160858154
	43 0.926458919048309
	44 0.926868491172791
	45 0.926373314857483
	46 0.926373314857483
	47 0.929043922424316
	48 0.926373314857483
	49 0.929043922424316
	50 0.929043922424316
	51 0.934855244159698
	52 0.929066760540009
	53 0.929066760540009
	54 0.931737368106842
	55 0.929043922424316
	56 0.929267556667328
	57 0.929066760540009
	58 0.929066760540009
	59 0.929043922424316
	60 0.926373314857483
	61 0.92845737695694
	62 0.931150822639465
	63 0.927268335819244
	64 0.928163356781006
	65 0.935905492305756
	66 0.931150822639465
	67 0.932023005485535
	68 0.939794867038727
	69 0.930541439056396
	70 0.931150822639465
	71 0.92845737695694
	72 0.931737368106842
	73 0.92845737695694
	74 0.932655227184296
	75 0.92845737695694
	76 0.93321204662323
	77 0.933234884738922
	78 0.933821430206299
	79 0.92845737695694
	80 0.936537714004517
	81 0.930541439056396
	82 0.931737368106842
	83 0.93321204662323
	84 0.933798592090607
	85 0.92845737695694
	86 0.931127984523773
	87 0.937989554405212
	88 0.933798592090607
	89 0.935296108722687
	90 0.931127984523773
	91 0.935296108722687
	92 0.931127984523773
	93 0.931127984523773
	94 0.933844268321991
	95 0.931127984523773
	96 0.931127984523773
	97 0.939386279582977
	98 0.935296108722687
	99 0.937411160469055
	100 0.93321204662323
};
%\addlegendentry{A2C}
\addplot [semithick, color2]
table {%
	1 0.913600015640259
	2 0.913600015640259
	3 0.913600015640259
	4 0.913600015640259
	5 0.913600015640259
	6 0.913600015640259
	7 0.913600015640259
	8 0.911679997444153
	9 0.947660000324249
	10 0.948780000209808
	11 0.951260006427765
	12 0.953840005397797
	13 0.951820011138916
	14 0.961360013484955
	15 0.961300015449524
	16 0.942340006828308
	17 0.951820011138916
	18 0.9328600025177
	19 0.9328600025177
	20 0.942340006828308
	21 0.9328600025177
	22 0.926460001468658
	23 0.916979997158051
	24 0.9328600025177
	25 0.947700006961822
	26 0.941300005912781
	27 0.947700006961822
	28 0.945420010089874
	29 0.940840003490448
	30 0.937520005702972
	31 0.93936000585556
	32 0.932960004806519
	33 0.934580001831055
	34 0.943920006752014
	35 0.943920006752014
	36 0.934440002441406
	37 0.943920006752014
	38 0.942380006313324
	39 0.942380006313324
	40 0.944780006408691
	41 0.948780007362366
	42 0.948780007362366
	43 0.948780007362366
	44 0.942380006313324
	45 0.940600008964539
	46 0.94700001001358
	47 0.94700001001358
	48 0.94700001001358
	49 0.946480000019074
	50 0.953400011062622
	51 0.951260004043579
	52 0.951260004043579
	53 0.939240005016327
	54 0.939479999542236
	55 0.933259997367859
	56 0.932680003643036
	57 0.932480003833771
	58 0.939479999542236
	59 0.941940002441406
	60 0.943559999465942
	61 0.952260000705719
	62 0.948340003490448
	63 0.948459997177124
	64 0.937999997138977
	65 0.942720005512238
	66 0.949960000514984
	67 0.948080003261566
	68 0.946840000152588
	69 0.948080003261566
	70 0.949960000514984
	71 0.946840000152588
	72 0.948459997177124
	73 0.948080003261566
	74 0.946840000152588
	75 0.946840000152588
	76 0.954480004310608
	77 0.950759997367859
	78 0.954480004310608
	79 0.95324000120163
	80 0.95324000120163
	81 0.948860006332398
	82 0.95324000120163
	83 0.95324000120163
	84 0.954480004310608
	85 0.954480004310608
	86 0.95324000120163
	87 0.950620000362396
	88 0.954480004310608
	89 0.949700000286102
	90 0.954780006408691
	91 0.954480004310608
	92 0.954480004310608
	93 0.953540003299713
	94 0.953540003299713
	95 0.950000002384186
	96 0.952300002574921
	97 0.939100005626679
	98 0.945020005702972
	99 0.949760007858276
	100 0.949400007724762
};
%\addlegendentry{SAC}
\addplot [semithick, color3]
table {%
	1 0.937237713758487
	100 0.937237713758487
};
%\addlegendentry{Random}
\addplot [semithick, color4]
table {%
	1 0.94031006304327
	100 0.94031006304327
};
%\addlegendentry{ETS}
\addplot [semithick, color5]
table {%
	1 0.888105348737885
	100 0.888105348737885
};
%\addlegendentry{Heur-GD}
\addplot [semithick, color6, dash pattern=on 1pt off 3pt on 3pt off 3pt]
table {%
	1 0.906470737356177
	100 0.906470737356177
};
%\addlegendentry{Heur-LD}
\addplot [semithick, white!49.8039215686275!black, dashed]
table {%
	1 0.93077412279482
	100 0.93077412279482
};
%\addlegendentry{Heur-AT}



\nextgroupplot[title=Seed 8,
height=\figheight,
legend cell align={left},
legend style={
  fill opacity=0.8,
  draw opacity=1,
  text opacity=1,
  at={(0.5,0.09)},
  anchor=south,
  draw=white!80!black
},
minor xtick={25, 75},
minor ytick={},
tick align=outside,
tick pos=left,
%title={Labels: [[8, 6], [9, 0], [2, 5], [7, 1], [4, 3]]},
width=\figwidth,
x grid style={white!69.0196078431373!black},
xlabel={Eval. Steps},
xminorgrids,
xmajorgrids,
xmin=-3.95, xmax=104.95,
xtick style={color=black},
xtick={-25,0,50,100,125},
xticklabels={-25,0,50,100,125},
y grid style={white!69.0196078431373!black},
ylabel={ACC (\%)},
ymajorgrids,
ymin=0.87912724410805, ymax=0.945090560499017,
ytick style={color=black},
ytick={0.87,0.88,0.89,0.9,0.91,0.92,0.93,0.94,0.95},
yticklabels={87,88,89,90,91,92,93,94,95}
]
\addplot [semithick, color0]
table {%
1 0.932959767182668
2 0.9315957903862
3 0.9315957903862
4 0.9315957903862
5 0.929590439796448
6 0.922864870230357
7 0.920010852813721
8 0.926947601636251
9 0.921921781698863
10 0.922864870230357
11 0.920010852813721
12 0.928529818852743
13 0.920010852813721
14 0.917199921607971
15 0.91954847574234
16 0.92719247341156
17 0.925871590773265
18 0.923213362693787
19 0.923213362693787
20 0.926513294378916
21 0.927834177017212
22 0.930987938245138
23 0.925192411740621
24 0.925748900572459
25 0.927069783210754
26 0.926513294378916
27 0.925748900572459
28 0.924147645632426
29 0.932865309715271
30 0.927069783210754
31 0.931189453601837
32 0.927626272042592
33 0.927626272042592
34 0.933421798547109
35 0.927292303244273
36 0.927851009368897
37 0.926682889461517
38 0.926761345068614
39 0.925616447130839
40 0.922710601488749
41 0.928392314910889
42 0.925537991523743
43 0.932428169250488
44 0.928470770517985
45 0.932959127426147
46 0.928392314910889
47 0.933311065038045
48 0.937698856989543
49 0.931516695022583
50 0.937698856989543
51 0.934844533602397
52 0.931516695022583
53 0.936165388425191
54 0.937698856989543
55 0.933035643895467
56 0.937698856989543
57 0.93590448697408
58 0.939232325553894
59 0.939232325553894
60 0.93590448697408
61 0.93590448697408
62 0.939232325553894
63 0.939232325553894
64 0.939232325553894
65 0.93590448697408
66 0.939232325553894
67 0.93590448697408
68 0.939232325553894
69 0.93590448697408
70 0.939232325553894
71 0.939232325553894
72 0.939232325553894
73 0.939232325553894
74 0.93590448697408
75 0.93590448697408
76 0.939232325553894
77 0.93590448697408
78 0.93590448697408
79 0.939232325553894
80 0.93590448697408
81 0.939232325553894
82 0.932576648394267
83 0.932576648394267
84 0.93590448697408
85 0.939232325553894
86 0.932576648394267
87 0.939232325553894
88 0.932576648394267
89 0.93590448697408
90 0.93590448697408
91 0.939232325553894
92 0.932576648394267
93 0.93590448697408
94 0.93590448697408
95 0.929248809814453
96 0.932576648394267
97 0.93590448697408
98 0.939232325553894
99 0.9348570783933
100 0.939232325553894
};
%\addlegendentry{a2c, gae, envs=10, lr=0.0001, act=tanh}
\addplot [semithick, color1]
table {%
1 0.932049596309662
2 0.941636677583059
3 0.934989941120148
4 0.921661857763926
5 0.930970370769501
6 0.930268514156341
7 0.935579820473989
8 0.931672505537669
9 0.922811106840769
10 0.923790995279948
11 0.927957705656687
12 0.933957719802856
13 0.926543176174164
14 0.936040687561035
15 0.932267272472382
16 0.926834412415822
17 0.93060261408488
18 0.929786193370819
19 0.939294250806173
20 0.936575829982758
21 0.936856269836426
22 0.938313861687978
23 0.93715136051178
24 0.942092227935791
25 0.938181002934774
26 0.931000852584839
27 0.929074800014496
28 0.920625591278076
29 0.926675526301066
30 0.928688132762909
31 0.927901291847229
32 0.927641900380452
33 0.938032786051432
34 0.925219917297363
35 0.938511923948924
36 0.927879174550374
37 0.935892975330353
38 0.938010668754578
39 0.927901291847229
40 0.930322178204854
41 0.927901291847229
42 0.929819822311401
43 0.942092227935791
44 0.923540131251017
45 0.927879174550374
46 0.93443036476771
47 0.927879174550374
48 0.933671625455221
49 0.933671625455221
50 0.936981530984243
51 0.941434061527252
52 0.936981530984243
53 0.931788198153178
54 0.934452482064565
55 0.932146255175273
56 0.937352502346039
57 0.933951226870219
58 0.931788198153178
59 0.931960733731588
60 0.938010668754578
61 0.931302567323049
62 0.935870858033498
63 0.931302567323049
64 0.933401230971019
65 0.931459482510885
66 0.939952417214712
67 0.936575829982758
68 0.93447459936142
69 0.925738263130188
70 0.931960733731588
71 0.931960733731588
72 0.933902482191722
73 0.921810984611511
74 0.929820923010508
75 0.925910798708598
76 0.931960733731588
77 0.935870858033498
78 0.938010668754578
79 0.935870858033498
80 0.937316373984019
81 0.931960733731588
82 0.927621690432231
83 0.930005343755086
84 0.940824639797211
85 0.933244315783183
86 0.921071171760559
87 0.924524704615275
88 0.927571841080983
89 0.927621690432231
90 0.91920108795166
91 0.922336483001709
92 0.929489497343699
93 0.923298939069112
94 0.921399219830831
95 0.928038068612417
96 0.933508288860321
97 0.926905278364817
98 0.935325880845388
99 0.925916190942128
100 0.927621690432231
};
%\addlegendentry{DQN, n train envs=30}
\addplot [semithick, color2]
table {%
1 0.924340868446401
100 0.924340868446401
};
%\addlegendentry{Random}
\addplot [semithick, color3]
table {%
1 0.890415249936047
100 0.890415249936047
};
%\addlegendentry{ETS}
\addplot [semithick, color4]
table {%
1 0.935567569236973
100 0.935567569236973
};
%\addlegendentry{Heur-GD}
\addplot [semithick, color5, dash pattern=on 1pt off 3pt on 3pt off 3pt]
table {%
1 0.882125576671276
100 0.882125576671276
};
%\addlegendentry{Heur-LD}
\addplot [semithick, color6, dashed]
table {%
1 0.935567569236973
100 0.935567569236973
};
%\addlegendentry{Heur-AT}



\nextgroupplot[title=Seed 9,
height=\figheight,
legend cell align={left},
legend style={
  fill opacity=0.8,
  draw opacity=1,
  text opacity=1,
  at={(0.03,0.97)},
  anchor=north west,
  draw=white!80!black
},
minor xtick={25, 75},
minor ytick={},
tick align=outside,
tick pos=left,
%title={Labels: [[8, 4], [7, 2], [1, 9], [3, 0], [6, 5]]},
width=\figwidth,
x grid style={white!69.0196078431373!black},
xlabel={Eval. Steps},
xminorgrids,
xmajorgrids,
xmin=-3.95, xmax=104.95,
xtick style={color=black},
xtick={-25,0,50,100,125},
xticklabels={-25,0,50,100,125},
y grid style={white!69.0196078431373!black},
%ylabel={ACC},
ymajorgrids,
ymin=0.737830088791097, ymax=0.952696562415609,
ytick style={color=black},
ytick={0.7,0.75,0.8,0.85,0.9,0.95},
yticklabels={70,75,80,85,90,95}
]
\addplot [semithick, color0]
table {%
1 0.864580869674683
2 0.890737676620483
3 0.890737676620483
4 0.890737676620483
5 0.875396760304769
6 0.854225154717763
7 0.868369205792745
8 0.847537930806478
9 0.852605259418488
10 0.854225154717763
11 0.868369205792745
12 0.858234548568726
13 0.868369205792745
14 0.850215760866801
15 0.860912378629049
16 0.826918341716131
17 0.823584355910619
18 0.788934163252513
19 0.830271579821904
20 0.754264718294144
21 0.757598704099655
22 0.754264718294144
23 0.750930732488632
24 0.747596746683121
25 0.750930732488632
26 0.754264718294144
27 0.747596746683121
28 0.787314267953237
29 0.754264718294144
30 0.792268149058024
31 0.754264718294144
32 0.747596746683121
33 0.747596746683121
34 0.795471624533335
35 0.778237545490265
36 0.747596746683121
37 0.795471624533335
38 0.817509053150813
39 0.795471624533335
40 0.836809041102727
41 0.891221380233765
42 0.84334650238355
43 0.84334650238355
44 0.847002758582433
45 0.891221380233765
46 0.891221380233765
47 0.903750479221344
48 0.902299348513285
49 0.902299348513285
50 0.902299348513285
51 0.902299348513285
52 0.902299348513285
53 0.902299348513285
54 0.903779264291127
55 0.907838332653046
56 0.907838332653046
57 0.907838332653046
58 0.907838332653046
59 0.907838332653046
60 0.907838332653046
61 0.907838332653046
62 0.907838332653046
63 0.909289463361104
64 0.910740594069163
65 0.909289463361104
66 0.905085090796153
67 0.910740594069163
68 0.909289463361104
69 0.907838332653046
70 0.905085090796153
71 0.907838332653046
72 0.907838332653046
73 0.907838332653046
74 0.909289463361104
75 0.905085090796153
76 0.906536221504211
77 0.90233184893926
78 0.909289463361104
79 0.905085090796153
80 0.907838332653046
81 0.905085090796153
82 0.909289463361104
83 0.907838332653046
84 0.906536221504211
85 0.907838332653046
86 0.909289463361104
87 0.906536221504211
88 0.910740594069163
89 0.909289463361104
90 0.906536221504211
91 0.909289463361104
92 0.910740594069163
93 0.907838332653046
94 0.909289463361104
95 0.909289463361104
96 0.907838332653046
97 0.912191724777222
98 0.909289463361104
99 0.906536221504211
100 0.907838332653046
};
%\addlegendentry{a2c, gae, envs=10, lr=0.0001, act=tanh}
\addplot [semithick, color1]
table {%
1 0.82398406068484
2 0.819590324163437
3 0.805619784196218
4 0.842099557320277
5 0.847449709971746
6 0.862992805242539
7 0.862818151712418
8 0.815620054801305
9 0.838435190916062
10 0.828719919919968
11 0.837833952903748
12 0.83437171181043
13 0.809042167663574
14 0.811566481987635
15 0.802314128478368
16 0.768619201580683
17 0.76221762696902
18 0.775513849655787
19 0.821620704730352
20 0.788659310340881
21 0.755245093504588
22 0.80042626063029
23 0.796455224355062
24 0.784978888432185
25 0.830409242709478
26 0.766947452227275
27 0.794648689031601
28 0.819959839185079
29 0.821937431891759
30 0.774133729934692
31 0.78890484770139
32 0.820355375607808
33 0.78890484770139
34 0.824500423669815
35 0.754527072111766
36 0.763281941413879
37 0.754268834988276
38 0.772821555534999
39 0.763281941413879
40 0.783529005448024
41 0.784274286031723
42 0.797616587082545
43 0.799387564261754
44 0.808817382653554
45 0.822072672843933
46 0.780172135432561
47 0.798758753140767
48 0.767593346039454
49 0.810347509384155
50 0.82993229230245
51 0.814422140518824
52 0.817668300867081
53 0.815709577004115
54 0.830520842472712
55 0.818913745880127
56 0.76557908654213
57 0.833521642287572
58 0.832544392347336
59 0.863667800029119
60 0.826702016592026
61 0.792052586873372
62 0.78453009724617
63 0.859827069441478
64 0.797461128234863
65 0.803748973210653
66 0.814349021514257
67 0.833525441090266
68 0.822327238321304
69 0.836811979611715
70 0.765421374638875
71 0.785058705012004
72 0.801098960638046
73 0.83154648343722
74 0.805435031652451
75 0.83154648343722
76 0.787536863485972
77 0.849995064735413
78 0.786743950843811
79 0.784296961625417
80 0.799439497788747
81 0.787542967001597
82 0.794803486267726
83 0.787542967001597
84 0.808187987407048
85 0.799438212315242
86 0.810539462169012
87 0.748970190684001
88 0.801836121082306
89 0.846956340471903
90 0.841425973176956
91 0.829928729931514
92 0.833457630872727
93 0.815532783667246
94 0.828041472037633
95 0.815532783667246
96 0.819186894098918
97 0.828615017731984
98 0.794450050592422
99 0.810692767302195
100 0.815824488798777
};
%\addlegendentry{DQN, n train envs=30}
\addplot [semithick, color2]
table {%
1 0.917741544938917
100 0.917741544938917
};
%\addlegendentry{Random}
\addplot [semithick, color3]
table {%
1 0.942929904523586
100 0.942929904523586
};
%\addlegendentry{ETS}
\addplot [semithick, color4]
table {%
1 0.790914223232021
100 0.790914223232021
};
%\addlegendentry{Heur-GD}
\addplot [semithick, color5, dash pattern=on 1pt off 3pt on 3pt off 3pt]
table {%
1 0.790914223232021
100 0.790914223232021
};
%\addlegendentry{Heur-LD}
\addplot [semithick, color6, dashed]
table {%
1 0.790914223232021
100 0.790914223232021
};
%\addlegendentry{Heur-AT}



\end{groupplot}

\end{tikzpicture}
  \vspace{-3mm}
  \caption{Performance progress measured in ACC (\%) for all methods in the 10 test environments for {\bf Split notMNIST} in the New Dataset experiment. We plot the performance progress for A2C and DQN for 100 evaluation steps equidistantly distributed over the training episodes. The performance for the Random, ETS, and Heuristic scheduling baselines are plotted as straight lines.  }
  \label{fig:policy_rewards_notmnist_new_dataset_paperD}
  \vspace{-3mm}
\end{figure}

\begin{figure}[t]
  \centering
  \setlength{\figwidth}{0.26\textwidth}
  \setlength{\figheight}{.14\textheight}
  



\pgfplotsset{every axis title/.append style={at={(0.5,0.85)}}}
\pgfplotsset{every tick label/.append style={font=\tiny}}
\pgfplotsset{every major tick/.append style={major tick length=2pt}}
\pgfplotsset{every minor tick/.append style={minor tick length=1pt}}
\pgfplotsset{every axis x label/.append style={at={(0.5,-0.22)}}}
\pgfplotsset{every axis y label/.append style={at={(-0.30,0.5)}}}
\begin{tikzpicture}
\tikzstyle{every node}=[font=\scriptsize]
\definecolor{color0}{rgb}{0.12156862745098,0.466666666666667,0.705882352941177}
\definecolor{color1}{rgb}{1,0.498039215686275,0.0549019607843137}
\definecolor{color2}{rgb}{0.172549019607843,0.627450980392157,0.172549019607843}
\definecolor{color3}{rgb}{0.83921568627451,0.152941176470588,0.156862745098039}
\definecolor{color4}{rgb}{0.580392156862745,0.403921568627451,0.741176470588235}
\definecolor{color5}{rgb}{0.549019607843137,0.337254901960784,0.294117647058824}
\definecolor{color6}{rgb}{0.890196078431372,0.466666666666667,0.76078431372549}

\begin{groupplot}[group style={group size= 4 by 3, horizontal sep=1.00cm, vertical sep=1.20cm}]

\nextgroupplot[title=Seed 0,
height=\figheight,
legend cell align={left},
legend columns=-1,
legend style={
  nodes={scale=0.85},
  fill opacity=0.8,
  draw opacity=1,
  text opacity=1,
  at={(-0.20,1.36)},%at={(0.10,1.29)},
  anchor=south west,
  draw=white!80!black
},
minor xtick={25, 75},
minor ytick={},
tick align=outside,
tick pos=left,
%title={Labels: [[0, 1], [2, 3], [4, 5], [6, 7], [8, 9]]},
width=\figwidth,
x grid style={white!69.0196078431373!black},
xlabel={Eval. Steps},
xminorgrids,
xmajorgrids,
xmin=-3.95, xmax=104.95,
xtick style={color=black},
xtick={-25,0,50,100,125},
xticklabels={-25,0,50,100,125},
y grid style={white!69.0196078431373!black},
ylabel={ACC (\%)},
ymajorgrids,
ymin=0.88, ymax=0.982,
ytick style={color=black},
ytick={0.88,0.9,0.92,0.94,0.96,0.98,1},
yticklabels={88,90,92,94,96,98,100}
]
\addplot [semithick, color0]
table {%
	1 0.973060004711151
	2 0.948860001564026
	3 0.934120001792908
	4 0.94382000207901
	5 0.955320000648499
	6 0.974020001888275
	7 0.935279998779297
	8 0.924340000152588
	9 0.925099999904633
	10 0.95688000202179
	11 0.972660002708435
	12 0.973719999790192
	13 0.974520001411438
	14 0.942260000705719
	15 0.943160004615784
	16 0.924800004959107
	17 0.971780006885529
	18 0.940880002975464
	19 0.942779998779297
	20 0.974239995479584
	21 0.955839993953705
	22 0.95675999879837
	23 0.941459999084473
	24 0.958460001945496
	25 0.938839998245239
	26 0.958220002651215
	27 0.943180003166199
	28 0.907200000286102
	29 0.956900000572205
	30 0.909420001506805
	31 0.938919999599457
	32 0.942260003089905
	33 0.93941999912262
	34 0.892059998512268
	35 0.891940002441406
	36 0.920140001773834
	37 0.939760003089905
	38 0.959319996833801
	39 0.906999998092651
	40 0.906960000991821
	41 0.933239998817444
	42 0.907880003452301
	43 0.924180002212524
	44 0.940959997177124
	45 0.890039999485016
	46 0.956579999923706
	47 0.924640004634857
	48 0.941100001335144
	49 0.906900005340576
	50 0.939460000991821
	51 0.924660000801086
	52 0.938880000114441
	53 0.921559996604919
	54 0.898559999465942
	55 0.956360001564026
	56 0.938040001392364
	57 0.924359998703003
	58 0.95058000087738
	59 0.924519999027252
	60 0.907219998836517
	61 0.924640002250671
	62 0.956900000572205
	63 0.950679998397827
	64 0.951800003051758
	65 0.941820001602173
	66 0.958000001907349
	67 0.924860000610352
	68 0.974360001087189
	69 0.957059998512268
	70 0.926699995994568
	71 0.93561999797821
	72 0.948559997081757
	73 0.969419996738434
	74 0.939100003242493
	75 0.95905999660492
	76 0.926259999275208
	77 0.952299995422363
	78 0.952159996032715
	79 0.939939997196197
	80 0.971360001564026
	81 0.960399997234344
	82 0.942219998836517
	83 0.949459993839264
	84 0.927979996204376
	85 0.946919994354248
	86 0.948779993057251
	87 0.968079996109009
	88 0.957759997844696
	89 0.948079996109009
	90 0.95039999961853
	91 0.950940001010895
	92 0.945299997329712
	93 0.945819993019104
	94 0.95545999288559
	95 0.967979991436005
	96 0.968199994564056
	97 0.951139996051788
	98 0.93585999250412
	99 0.965319995880127
	100 0.968179996013641
};
\addlegendentry{DQN}
\addplot [semithick, color1]
table {%
	1 0.972999999523163
	2 0.975260000228882
	3 0.975319995880127
	4 0.97279999256134
	5 0.941779992580414
	6 0.910139999389649
	7 0.910160000324249
	8 0.895559999942779
	9 0.911100001335144
	10 0.895539996623993
	11 0.893779997825622
	12 0.895579998493195
	13 0.894719996452332
	14 0.894719996452331
	15 0.895579998493195
	16 0.895579998493195
	17 0.895579998493195
	18 0.892959995269775
	19 0.895579998493195
	20 0.894699997901917
	21 0.894759998321533
	22 0.894719996452332
	23 0.894719996452332
	24 0.894719996452332
	25 0.894719996452332
	26 0.892959995269775
	27 0.892959995269775
	28 0.892959995269775
	29 0.908399994373321
	30 0.892959995269775
	31 0.923919994831085
	32 0.907499997615814
	33 0.908399994373322
	34 0.942820000648499
	35 0.941119997501373
	36 0.908439996242523
	37 0.924759995937347
	38 0.941999998092651
	39 0.939359993934631
	40 0.954879994392395
	41 0.955719995498657
	42 0.954819996356964
	43 0.956759996414185
	44 0.971379997730255
	45 0.973840000629425
	46 0.971979997158051
	47 0.972019996643066
	48 0.954779994487762
	49 0.971239993572235
	50 0.974159998893738
	51 0.973880000114441
	52 0.972019996643066
	53 0.972819998264313
	54 0.955579996109009
	55 0.972119996547699
	56 0.959200000762939
	57 0.970319991111755
	58 0.957379999160767
	59 0.972039995193481
	60 0.956019995212555
	61 0.974519996643066
	62 0.973319997787476
	63 0.973880000114441
	64 0.975479998588562
	65 0.976180000305176
	66 0.971919994354248
	67 0.973780002593994
	68 0.939139997959137
	69 0.974559996128082
	70 0.974679999351502
	71 0.974720001220703
	72 0.957439997196198
	73 0.973780002593994
	74 0.973600001335144
	75 0.973600001335144
	76 0.971939995288849
	77 0.972039995193481
	78 0.956559996604919
	79 0.976280000209808
	80 0.957860004901886
	81 0.975560002326965
	82 0.975560002326965
	83 0.973680000305176
	84 0.956479997634888
	85 0.973700001239777
	86 0.97456000328064
	87 0.959100000858307
	88 0.975640003681183
	89 0.972700002193451
	90 0.972999999523163
	91 0.9753600025177
	92 0.975560002326965
	93 0.975380001068115
	94 0.956440000534058
	95 0.975299997329712
	96 0.975480000972748
	97 0.973179996013641
	98 0.959979999065399
	99 0.958139998912811
	100 0.957439997196198
};
\addlegendentry{A2C}
\addplot [semithick, color2]
table {%
	1 0.969300007820129
	2 0.969300007820129
	3 0.969300007820129
	4 0.970620007514954
	5 0.972400002479553
	6 0.970939998626709
	7 0.97308000087738
	8 0.971780002117157
	9 0.969760000705719
	10 0.971160001754761
	11 0.968860001564026
	12 0.969600002765655
	13 0.969359998703003
	14 0.970879998207092
	15 0.971679999828339
	16 0.970699999332428
	17 0.970820000171661
	18 0.969979999065399
	19 0.969800000190735
	20 0.971000001430511
	21 0.969800002574921
	22 0.955979998111725
	23 0.970079998970032
	24 0.970640001296997
	25 0.970200002193451
	26 0.970920000076294
	27 0.97079999923706
	28 0.970460002422333
	29 0.954420003890991
	30 0.971860001087189
	31 0.972000000476837
	32 0.971640000343323
	33 0.972780001163483
	34 0.971560003757477
	35 0.97114000082016
	36 0.971920003890991
	37 0.969880001544952
	38 0.968940002918244
	39 0.969400000572205
	40 0.968780002593994
	41 0.970060005187988
	42 0.954580006599426
	43 0.95510000705719
	44 0.969400000572205
	45 0.970140001773834
	46 0.96978000164032
	47 0.970160002708435
	48 0.968780002593994
	49 0.969580001831055
	50 0.968780002593994
	51 0.969800002574921
	52 0.968780002593994
	53 0.968780002593994
	54 0.968780002593994
	55 0.969980001449585
	56 0.968780002593994
	57 0.970300002098084
	58 0.968360002040863
	59 0.969780004024506
	60 0.969680001735687
	61 0.955120000839233
	62 0.969780004024506
	63 0.969160001277924
	64 0.969780004024506
	65 0.954700005054474
	66 0.969780004024506
	67 0.969780004024506
	68 0.969780004024506
	69 0.955840003490448
	70 0.969780004024506
	71 0.955620002746582
	72 0.969780004024506
	73 0.97076000213623
	74 0.969780004024506
	75 0.970179998874664
	76 0.972020001411438
	77 0.968440003395081
	78 0.970900001525879
	79 0.969780004024506
	80 0.969780004024506
	81 0.969780004024506
	82 0.971080000400543
	83 0.971900005340576
	84 0.96996000289917
	85 0.969780004024506
	86 0.971020004749298
	87 0.971260004043579
	88 0.971260004043579
	89 0.971260004043579
	90 0.971260004043579
	91 0.971260004043579
	92 0.971260004043579
	93 0.974040002822876
	94 0.971260004043579
	95 0.971360003948212
	96 0.971360003948212
	97 0.970400002002716
	98 0.971120004653931
	99 0.955240006446838
	100 0.971360003948212
};
\addlegendentry{SAC}
\addplot [semithick, color3]
table {%
	1 0.94328
	100 0.94328
};
\addlegendentry{Random}
\addplot [semithick, color4]
table {%
	1 0.971
	100 0.971
};
\addlegendentry{ETS}
\addplot [semithick, color5]
table {%
	1 0.979
	100 0.979
};
\addlegendentry{Heur-GD}
\addplot [semithick, color6, dash pattern=on 1pt off 3pt on 3pt off 3pt]
table {%
	1 0.9759
	100 0.9759
};
\addlegendentry{Heur-LD}
\addplot [semithick, white!49.8039215686275!black, dashed]
table {%
	1 0.9741
	100 0.9741
};
\addlegendentry{Heur-AT}



\nextgroupplot[title=Seed 1,
height=\figheight,
legend cell align={left},
legend style={
  fill opacity=0.8,
  draw opacity=1,
  text opacity=1,
  at={(0.97,0.03)},
  anchor=south east,
  draw=white!80!black
},
minor xtick={25, 75},
minor ytick={},
tick align=outside,
tick pos=left,
%title={Labels: [[2, 9], [6, 4], [0, 3], [1, 7], [8, 5]]},
width=\figwidth,
x grid style={white!69.0196078431373!black},
xlabel={Eval. Steps},
xminorgrids,
xmajorgrids,
xmin=-3.95, xmax=104.95,
xtick style={color=black},
xtick={-25,0,50,100,125},
xticklabels={-25,0,50,100,125},
y grid style={white!69.0196078431373!black},
%ylabel={ACC},
ymajorgrids,
ymin=0.854905004700025, ymax=0.962595010399818,
ytick style={color=black},
ytick={0.84,0.86,0.88,0.9,0.92,0.94,0.96,0.98},
yticklabels={84,86,88,90,92,94,96,98}
]
\addplot [semithick, color0]
table {%
	1 0.917259993553162
	2 0.924879999160767
	3 0.916719996929169
	4 0.925719995498657
	5 0.920299999713898
	6 0.883439996242523
	7 0.916079998016357
	8 0.881360001564026
	9 0.91722000837326
	10 0.90210000038147
	11 0.939179999828339
	12 0.911920003890991
	13 0.901460003852844
	14 0.902080008983612
	15 0.938520004749298
	16 0.933319997787476
	17 0.920500001907349
	18 0.939060006141663
	19 0.93204000711441
	20 0.942940003871918
	21 0.896760001182556
	22 0.926840002536774
	23 0.920659999847412
	24 0.919160003662109
	25 0.941320009231567
	26 0.893200001716614
	27 0.944400000572204
	28 0.916319999694824
	29 0.944599997997284
	30 0.926540002822876
	31 0.886360003948212
	32 0.926839997768402
	33 0.89592000246048
	34 0.93146000623703
	35 0.897980000972748
	36 0.926440002918243
	37 0.90826000213623
	38 0.907240002155304
	39 0.93864000082016
	40 0.93088000535965
	41 0.92206000328064
	42 0.938940002918243
	43 0.920299999713898
	44 0.91978000164032
	45 0.907599995136261
	46 0.90021999835968
	47 0.906599998474121
	48 0.907820000648499
	49 0.912699999809265
	50 0.908939998149872
	51 0.93558000087738
	52 0.923519999980927
	53 0.89058000087738
	54 0.92804000377655
	55 0.920520000457764
	56 0.900400002002716
	57 0.934879999160766
	58 0.937099997997284
	59 0.92944000005722
	60 0.916579995155334
	61 0.929479994773865
	62 0.93059999704361
	63 0.94061999797821
	64 0.88923999786377
	65 0.915999999046326
	66 0.897479994297028
	67 0.895119998455048
	68 0.904300000667572
	69 0.926180002689362
	70 0.940860004425049
	71 0.929560000896454
	72 0.916819996833801
	73 0.93566000699997
	74 0.928280003070831
	75 0.925940003395081
	76 0.942720000743866
	77 0.937740001678467
	78 0.936300001144409
	79 0.921600000858307
	80 0.9142999958992
	81 0.919479999542236
	82 0.931720006465912
	83 0.935860006809235
	84 0.936880004405975
	85 0.930080008506775
	86 0.931819996833801
	87 0.931480002403259
	88 0.932660005092621
	89 0.935340006351471
	90 0.939400007724762
	91 0.936740007400513
	92 0.936760005950928
	93 0.936740007400513
	94 0.910380005836487
	95 0.922379999160767
	96 0.922379999160767
	97 0.931099998950958
	98 0.930999999046326
	99 0.93364000082016
	100 0.937420003414154
};
%\addlegendentry{DQN}
\addplot [semithick, color1]
table {%
	1 0.914759998321533
	2 0.913479995727539
	3 0.937900006771088
	4 0.941500008106232
	5 0.946800005435944
	6 0.946800007820129
	7 0.949060008525848
	8 0.950760009288788
	9 0.951480009555817
	10 0.946360008716583
	11 0.947020008563995
	12 0.951000010967255
	13 0.95120001077652
	14 0.95120001077652
	15 0.951000010967255
	16 0.951000010967255
	17 0.950880010128021
	18 0.95620001077652
	19 0.953400011062622
	20 0.951400010585785
	21 0.948360011577606
	22 0.953560011386871
	23 0.950940012931824
	24 0.942760007381439
	25 0.933920004367828
	26 0.9514200091362
	27 0.935140006542206
	28 0.937400004863739
	29 0.914279997348785
	30 0.939600005149841
	31 0.932460000514984
	32 0.92175999879837
	33 0.928800001144409
	34 0.906719992160797
	35 0.922199997901916
	36 0.915739996433258
	37 0.915599994659424
	38 0.922199997901916
	39 0.915599994659424
	40 0.922339999675751
	41 0.920060000419617
	42 0.935820009708404
	43 0.928940002918243
	44 0.922480001449585
	45 0.922339999675751
	46 0.915739996433258
	47 0.921879997253418
	48 0.929080004692078
	49 0.929080004692078
	50 0.942560014724731
	51 0.935960011482239
	52 0.934800002574921
	53 0.929220006465912
	54 0.929080004692078
	55 0.928619997501373
	56 0.935960011482239
	57 0.94256000995636
	58 0.935960001945496
	59 0.9353600025177
	60 0.922480001449585
	61 0.927459998130798
	62 0.922479996681213
	63 0.935960011482239
	64 0.929220006465912
	65 0.9353600025177
	66 0.941500008106232
	67 0.935360007286072
	68 0.922480001449585
	69 0.935360007286072
	70 0.929220006465912
	71 0.942700016498566
	72 0.935960011482239
	73 0.935360007286072
	74 0.935360007286072
	75 0.928400006294251
	76 0.942100012302399
	77 0.928620002269745
	78 0.934760003089905
	79 0.928620002269745
	80 0.93454000711441
	81 0.928620002269745
	82 0.941500008106232
	83 0.934159998893738
	84 0.940299999713898
	85 0.941500008106232
	86 0.941500008106232
	87 0.928019998073578
	88 0.940299999713898
	89 0.928019998073578
	90 0.934159998893738
	91 0.927419993877411
	92 0.933559994697571
	93 0.934159998893738
	94 0.921879997253418
	95 0.932999994754791
	96 0.934159998893738
	97 0.928019998073578
	98 0.927459998130798
	99 0.935220005512238
	100 0.928019998073578
};
%\addlegendentry{A2C}
\addplot [semithick, color2]
table {%
	1 0.93659999370575
	2 0.93659999370575
	3 0.936879994869232
	4 0.937499997615814
	5 0.947880003452301
	6 0.950119998455048
	7 0.947580001354217
	8 0.953160002231598
	9 0.946860001087189
	10 0.946600005626678
	11 0.949720001220703
	12 0.949320001602173
	13 0.949180002212524
	14 0.949720001220703
	15 0.95084000825882
	16 0.949320001602173
	17 0.948520002365112
	18 0.945940008163452
	19 0.945900001525879
	20 0.946340000629425
	21 0.950719997882843
	22 0.949860002994537
	23 0.945799999237061
	24 0.947140002250671
	25 0.945880000591278
	26 0.94922000169754
	27 0.946619999408722
	28 0.947960002422333
	29 0.947960002422333
	30 0.942500007152557
	31 0.926960000991821
	32 0.945699999332428
	33 0.948320000171661
	34 0.948320000171661
	35 0.948320000171661
	36 0.947160000801086
	37 0.94460000038147
	38 0.947460005283356
	39 0.948320000171661
	40 0.948320000171661
	41 0.948320000171661
	42 0.945400004386902
	43 0.946540005207062
	44 0.947460005283356
	45 0.947460005283356
	46 0.947460005283356
	47 0.944240000247955
	48 0.942720003128052
	49 0.94686000585556
	50 0.940960001945496
	51 0.941680002212524
	52 0.943940002918243
	53 0.946260006427765
	54 0.944540004730225
	55 0.947460005283356
	56 0.946760005950928
	57 0.943940002918243
	58 0.946880002021789
	59 0.937200002670288
	60 0.937420003414154
	61 0.933480005264282
	62 0.933480005264282
	63 0.933480005264282
	64 0.933260004520416
	65 0.937380001544952
	66 0.93226000547409
	67 0.941920006275177
	68 0.933480005264282
	69 0.942740001678467
	70 0.933260004520416
	71 0.942740001678467
	72 0.941920006275177
	73 0.93164000749588
	74 0.942740001678467
	75 0.942740001678467
	76 0.942740001678467
	77 0.93164000749588
	78 0.942740001678467
	79 0.93164000749588
	80 0.941540002822876
	81 0.932280006408691
	82 0.941540002822876
	83 0.941920006275177
	84 0.940340003967285
	85 0.941540002822876
	86 0.935800006389618
	87 0.935580005645752
	88 0.943860003948212
	89 0.943840003013611
	90 0.935800006389618
	91 0.943860003948212
	92 0.935800006389618
	93 0.935580005645752
	94 0.935800006389618
	95 0.943860003948212
	96 0.943860003948212
	97 0.943860003948212
	98 0.943800003528595
	99 0.934460003376007
	100 0.943719999790191
};
%\addlegendentry{SAC}
\addplot [semithick, color3]
table {%
	1 0.92262
	100 0.92262
};
%\addlegendentry{Random}
\addplot [semithick, color4]
table {%
	1 0.941
	100 0.941
};
%\addlegendentry{ETS}
\addplot [semithick, color5]
table {%
	1 0.9501
	100 0.9501
};
%\addlegendentry{Heur-GD}
\addplot [semithick, color6, dash pattern=on 1pt off 3pt on 3pt off 3pt]
table {%
	1 0.9003
	100 0.9003
};
%\addlegendentry{Heur-LD}
\addplot [semithick, white!49.8039215686275!black, dashed]
table {%
	1 0.9409
	100 0.9409
};
%\addlegendentry{Heur-AT}



\nextgroupplot[title=Seed 2,
height=\figheight,
legend cell align={left},
legend style={
  fill opacity=0.8,
  draw opacity=1,
  text opacity=1,
  at={(0.5,0.09)},
  anchor=south,
  draw=white!80!black
},
minor xtick={25, 75},
minor ytick={},
tick align=outside,
tick pos=left,
%title={Labels: [[4, 1], [5, 0], [7, 2], [3, 6], [9, 8]]},
width=\figwidth,
x grid style={white!69.0196078431373!black},
xlabel={Eval. Steps},
xminorgrids,
xmajorgrids,
xmin=-3.95, xmax=104.95,
xtick style={color=black},
xtick={-25,0,50,100,125},
xticklabels={-25,0,50,100,125},
y grid style={white!69.0196078431373!black},
%ylabel={ACC},
ymajorgrids,
ymin=0.86, ymax=0.982280000700951,
ytick style={color=black},
ytick={0.86,0.88,0.9,0.92,0.94,0.96,0.98,1},
yticklabels={86,88,90,92,94,96,98,100}
]
\addplot [semithick, color0]
table {%
	1 0.943180005550385
	2 0.937100005149841
	3 0.944380004405975
	4 0.961840000152588
	5 0.958739995956421
	6 0.956720006465912
	7 0.965880002975464
	8 0.968760004043579
	9 0.969200000762939
	10 0.93650000333786
	11 0.947040002346039
	12 0.969380002021789
	13 0.960360009670257
	14 0.968719997406006
	15 0.969860000610352
	16 0.940640003681183
	17 0.954779999256134
	18 0.966980001926422
	19 0.971000001430512
	20 0.965660002231598
	21 0.96411999464035
	22 0.967679998874664
	23 0.963800001144409
	24 0.969360003471375
	25 0.966480007171631
	26 0.963359999656677
	27 0.968840005397797
	28 0.970020003318787
	29 0.96731999874115
	30 0.969739997386932
	31 0.968000001907349
	32 0.966839997768402
	33 0.966720006465912
	34 0.96694000005722
	35 0.965199999809265
	36 0.965560004711151
	37 0.960200002193451
	38 0.960760004520416
	39 0.961940002441406
	40 0.959740002155304
	41 0.960620007514954
	42 0.965520005226135
	43 0.947940006256104
	44 0.959880003929138
	45 0.946320009231567
	46 0.948780007362366
	47 0.951940007209778
	48 0.967980008125305
	49 0.950940008163452
	50 0.951080005168915
	51 0.946840009689331
	52 0.967580003738403
	53 0.965960009098053
	54 0.95584000825882
	55 0.931300013065338
	56 0.962420010566711
	57 0.945800006389618
	58 0.95186000585556
	59 0.944480011463165
	60 0.968540003299713
	61 0.966140007972717
	62 0.945440006256103
	63 0.966120007038117
	64 0.961300010681152
	65 0.969080009460449
	66 0.963840007781982
	67 0.964840006828308
	68 0.969040005207062
	69 0.946760008335114
	70 0.946180009841919
	71 0.945220007896423
	72 0.968180010318756
	73 0.940920009613037
	74 0.967320008277893
	75 0.966920006275177
	76 0.968720006942749
	77 0.948320009708405
	78 0.946260004043579
	79 0.97010000705719
	80 0.949080004692078
	81 0.947540001869202
	82 0.947460007667541
	83 0.95028000831604
	84 0.949220006465912
	85 0.962460007667542
	86 0.968380002975464
	87 0.959800007343292
	88 0.963680005073547
	89 0.954700002670288
	90 0.96614000082016
	91 0.950360000133514
	92 0.961020004749298
	93 0.960580005645752
	94 0.959220006465912
	95 0.964260001182556
	96 0.961220006942749
	97 0.958740005493164
	98 0.956220004558563
	99 0.963340005874634
	100 0.961320006847382
};
%\addlegendentry{DQN}
\addplot [semithick, color1]
table {%
	1 0.969500000476837
	2 0.961380000114441
	3 0.965039999485016
	4 0.976800000667572
	5 0.97132000207901
	6 0.940560004711151
	7 0.951600005626678
	8 0.949860005378723
	9 0.95702000617981
	10 0.95206000328064
	11 0.958840003013611
	12 0.955980005264282
	13 0.964860005378723
	14 0.962100005149841
	15 0.955980005264282
	16 0.955980005264282
	17 0.949860005378723
	18 0.970380005836487
	19 0.955980005264282
	20 0.958740005493164
	21 0.96090000629425
	22 0.964860005378723
	23 0.964860005378723
	24 0.964860005378723
	25 0.964860005378723
	26 0.967620005607605
	27 0.967620005607605
	28 0.967620005607605
	29 0.967620005607605
	30 0.967620005607605
	31 0.970380005836487
	32 0.968920004367828
	33 0.970380005836487
	34 0.963400003910065
	35 0.962100005149841
	36 0.967620005607605
	37 0.967620005607605
	38 0.967620005607605
	39 0.967620005607605
	40 0.967620005607605
	41 0.967620005607605
	42 0.970380005836487
	43 0.970380005836487
	44 0.970380005836487
	45 0.964860005378723
	46 0.960640003681183
	47 0.970560004711151
	48 0.964860005378723
	49 0.970560004711151
	50 0.973320004940033
	51 0.967620005607605
	52 0.970560004711151
	53 0.964860005378723
	54 0.967620005607605
	55 0.970380005836487
	56 0.967620005607605
	57 0.964860005378723
	58 0.963400003910065
	59 0.970380005836487
	60 0.964700002670288
	61 0.970380005836487
	62 0.960640003681183
	63 0.970380005836487
	64 0.97590000629425
	65 0.970380005836487
	66 0.964860005378723
	67 0.963400003910065
	68 0.968920004367828
	69 0.967620005607605
	70 0.973140006065369
	71 0.967620005607605
	72 0.973140006065369
	73 0.963400003910065
	74 0.961940002441406
	75 0.961940002441406
	76 0.962100005149841
	77 0.970380005836487
	78 0.970380005836487
	79 0.970380005836487
	80 0.962100005149841
	81 0.964860005378723
	82 0.967620005607605
	83 0.973140006065369
	84 0.973140006065369
	85 0.963400003910065
	86 0.970380005836487
	87 0.97168000459671
	88 0.967620005607605
	89 0.968920004367828
	90 0.970380005836487
	91 0.96324000120163
	92 0.967800004482269
	93 0.970380005836487
	94 0.964880001544952
	95 0.970380005836487
	96 0.967620005607605
	97 0.964700002670288
	98 0.973140006065369
	99 0.971860003471374
	100 0.973140006065369
};
%\addlegendentry{A2C}
\addplot [semithick, color2]
table {%
	1 0.968599998950958
	2 0.968599998950958
	3 0.968599998950958
	4 0.967260005474091
	5 0.969080004692078
	6 0.969600005149841
	7 0.970340006351471
	8 0.97050000667572
	9 0.972720003128052
	10 0.972400002479553
	11 0.972560002803803
	12 0.972240002155304
	13 0.971080002784729
	14 0.972720003128052
	15 0.972720003128052
	16 0.972720003128052
	17 0.972400002479553
	18 0.971800003051758
	19 0.971800003051758
	20 0.972480001449585
	21 0.972280004024506
	22 0.967120001316071
	23 0.967120001316071
	24 0.970640001296997
	25 0.972400002479553
	26 0.967519998550415
	27 0.967519998550415
	28 0.967519998550415
	29 0.97210000038147
	30 0.971619999408722
	31 0.971779999732971
	32 0.971779999732971
	33 0.97175999879837
	34 0.967980003356934
	35 0.968800003528595
	36 0.968180003166199
	37 0.961600005626678
	38 0.971099998950958
	39 0.9675
	40 0.9675
	41 0.963840003013611
	42 0.960620002746582
	43 0.963840003013611
	44 0.9675
	45 0.966439998149872
	46 0.966139998435974
	47 0.962839999198914
	48 0.965059995651245
	49 0.962760000228882
	50 0.962160000801086
	51 0.966199994087219
	52 0.964759995937347
	53 0.966119997501373
	54 0.967279992103577
	55 0.964759995937347
	56 0.964679996967316
	57 0.965739998817444
	58 0.964919998645782
	59 0.96444000005722
	60 0.96425999879837
	61 0.96425999879837
	62 0.96444000005722
	63 0.966359996795654
	64 0.966659996509552
	65 0.96521999835968
	66 0.964079999923706
	67 0.967139995098114
	68 0.96481999874115
	69 0.967439994812012
	70 0.964299998283386
	71 0.964919998645782
	72 0.967339994907379
	73 0.964299998283386
	74 0.964299998283386
	75 0.964599997997284
	76 0.964299998283386
	77 0.964299998283386
	78 0.964299998283386
	79 0.966519994735718
	80 0.964299998283386
	81 0.964299998283386
	82 0.966519994735718
	83 0.966519994735718
	84 0.966519994735718
	85 0.966519994735718
	86 0.965579996109009
	87 0.965879995822907
	88 0.96639999628067
	89 0.963359999656677
	90 0.963359999656677
	91 0.965279996395111
	92 0.96565999507904
	93 0.966139996051788
	94 0.963919999599457
	95 0.966139996051788
	96 0.966139996051788
	97 0.965839996337891
	98 0.96621999502182
	99 0.965839996337891
	100 0.963919999599457
};
%\addlegendentry{SAC}
\addplot [semithick, color3]
table {%
	1 0.93738
	100 0.93738
};
%\addlegendentry{Random}
\addplot [semithick, color4]
table {%
	1 0.8672
	100 0.8672
};
%\addlegendentry{ETS}
\addplot [semithick, color5]
table {%
	1 0.9741
	100 0.9741
};
%\addlegendentry{Heur-GD}
\addplot [semithick, color6, dash pattern=on 1pt off 3pt on 3pt off 3pt]
table {%
	1 0.973
	100 0.973
};
%\addlegendentry{Heur-LD}
\addplot [semithick, white!49.8039215686275!black, dashed]
table {%
	1 0.9765
	100 0.9765
};
%\addlegendentry{Heur-AT}



\nextgroupplot[title=Seed 3,
height=\figheight,
legend cell align={left},
legend style={
  fill opacity=0.8,
  draw opacity=1,
  text opacity=1,
  at={(0.97,0.03)},
  anchor=south east,
  draw=white!80!black
},
minor xtick={25, 75},
minor ytick={},
tick align=outside,
tick pos=left,
%title={Labels: [[5, 4], [1, 2], [9, 6], [7, 0], [3, 8]]},
width=\figwidth,
x grid style={white!69.0196078431373!black},
xlabel={Eval. Steps},
xminorgrids,
xmajorgrids,
xmin=-3.95, xmax=104.95,
xtick style={color=black},
xtick={-25,0,50,100,125},
xticklabels={-25,0,50,100,125},
y grid style={white!69.0196078431373!black},
%ylabel={ACC},
ymajorgrids,
ymin=0.86, ymax=1.0,
ytick style={color=black},
ytick={0.86,0.88,0.9,0.92,0.94,0.96,0.98,1,1.02},
yticklabels={86,88,90,92,94,96,98,100,102}
]
\addplot [semithick, color0]
table {%
	1 0.952180006504059
	2 0.933399996757507
	3 0.94289999961853
	4 0.956359996795654
	5 0.907979996204376
	6 0.895579998493195
	7 0.92173999786377
	8 0.95326000213623
	9 0.972060000896454
	10 0.931539993286133
	11 0.928119995594025
	12 0.944759998321533
	13 0.932160007953644
	14 0.952580003738403
	15 0.944879999160767
	16 0.961920003890991
	17 0.940499999523163
	18 0.940099999904633
	19 0.940160007476807
	20 0.952679996490478
	21 0.947040002346039
	22 0.950359995365143
	23 0.936679997444153
	24 0.975840001106262
	25 0.923599996566772
	26 0.951719996929169
	27 0.954539999961853
	28 0.964379999637604
	29 0.957740001678467
	30 0.947319996356964
	31 0.911820003986359
	32 0.947479996681213
	33 0.958439998626709
	34 0.988700003623962
	35 0.953660001754761
	36 0.980480003356934
	37 0.96978000164032
	38 0.985699996948242
	39 0.966219999790192
	40 0.947019999027252
	41 0.963280003070831
	42 0.983599991798401
	43 0.974579994678497
	44 0.984020001888275
	45 0.980880000591278
	46 0.98746000289917
	47 0.953519997596741
	48 0.958500001430511
	49 0.956940002441406
	50 0.950720002651215
	51 0.965940008163452
	52 0.974959995746612
	53 0.947380001544952
	54 0.966560001373291
	55 0.977839999198914
	56 0.962700006961822
	57 0.980200006961823
	58 0.965700008869171
	59 0.961240000724792
	60 0.973900003433228
	61 0.984300003051758
	62 0.980639996528626
	63 0.981520001888275
	64 0.982180004119873
	65 0.97313999414444
	66 0.983039996623993
	67 0.953179998397827
	68 0.985279998779297
	69 0.98102000951767
	70 0.971279995441437
	71 0.964960000514984
	72 0.988399999141693
	73 0.960680003166199
	74 0.979140000343323
	75 0.959800002574921
	76 0.979800000190735
	77 0.960480008125305
	78 0.95632000207901
	79 0.969220004081726
	80 0.975459997653961
	81 0.986259999275208
	82 0.964359998703003
	83 0.984360003471375
	84 0.982200005054474
	85 0.952859997749329
	86 0.965359995365143
	87 0.959519996643066
	88 0.98360000371933
	89 0.965919992923737
	90 0.961040000915527
	91 0.975239996910095
	92 0.971559998989105
	93 0.972739996910095
	94 0.974960000514984
	95 0.978260004520416
	96 0.975959992408752
	97 0.977100005149841
	98 0.979539995193481
	99 0.971539995670319
	100 0.946600000858307
};
%\addlegendentry{DQN}
\addplot [semithick, color1]
table {%
	1 0.979080007076263
	2 0.944800004959107
	3 0.950440003871918
	4 0.986600005626678
	5 0.984600005149841
	6 0.97338000535965
	7 0.981860003471374
	8 0.96738000869751
	9 0.974400005340576
	10 0.960760002136231
	11 0.96610000371933
	12 0.968300008773804
	13 0.967880005836487
	14 0.968300008773804
	15 0.968300008773804
	16 0.968300008773804
	17 0.967840008735657
	18 0.967039999961853
	19 0.968300008773804
	20 0.967880005836487
	21 0.959680004119873
	22 0.967880005836487
	23 0.972060008049011
	24 0.959680004119873
	25 0.962500002384186
	26 0.961659996509552
	27 0.966619997024536
	28 0.963020000457764
	29 0.9646399974823
	30 0.953559999465942
	31 0.953039994239807
	32 0.983360004425049
	33 0.968319997787476
	34 0.958759996891022
	35 0.958059995174408
	36 0.96195999622345
	37 0.964320001602173
	38 0.964619996547699
	39 0.967299997806549
	40 0.986660006046295
	41 0.960359997749329
	42 0.987100005149841
	43 0.955179994106293
	44 0.94423999786377
	45 0.955499999523163
	46 0.965219995975495
	47 0.967480001449585
	48 0.984740002155304
	49 0.948560001850128
	50 0.958440001010895
	51 0.956060001850128
	52 0.974480004310608
	53 0.941440002918243
	54 0.965420000553131
	55 0.959140002727509
	56 0.935459997653961
	57 0.960240001678467
	58 0.935459997653961
	59 0.966220002174378
	60 0.934100000858307
	61 0.980540006160736
	62 0.95342000246048
	63 0.949420001506805
	64 0.929859998226166
	65 0.916760001182556
	66 0.93985999584198
	67 0.972460000514984
	68 0.946060004234314
	69 0.953680002689362
	70 0.948340003490448
	71 0.985500004291535
	72 0.966420001983643
	73 0.954119997024536
	74 0.943539996147156
	75 0.925320003032684
	76 0.92039999961853
	77 0.930680000782013
	78 0.93827999830246
	79 0.931140003204346
	80 0.97210000038147
	81 0.957359998226166
	82 0.936040000915527
	83 0.96900000333786
	84 0.946019999980927
	85 0.936279997825623
	86 0.947360000610352
	87 0.938780002593994
	88 0.938819997310638
	89 0.939140002727509
	90 0.954940001964569
	91 0.974379999637604
	92 0.949920001029968
	93 0.94
	94 0.981459999084473
	95 0.946920001506805
	96 0.950620002746582
	97 0.944679999351501
	98 0.937499997615814
	99 0.945899996757507
	100 0.921219999790192
};
%\addlegendentry{A2C}
\addplot [semithick, color2]
table {%
	1 0.909700000286102
	2 0.909700000286102
	3 0.909700000286102
	4 0.891019999980927
	5 0.935939998626709
	6 0.957439999580383
	7 0.935280001163483
	8 0.965960004329681
	9 0.95554000377655
	10 0.981359994411468
	11 0.971040003299713
	12 0.97043999671936
	13 0.977580001354218
	14 0.98460000038147
	15 0.953379995822907
	16 0.98698000907898
	17 0.978180005550385
	18 0.986080005168915
	19 0.971580007076263
	20 0.948539998531342
	21 0.936059997081757
	22 0.973099992275238
	23 0.974759998321533
	24 0.974759998321533
	25 0.967539999485016
	26 0.946299998760223
	27 0.96675999879837
	28 0.967279999256134
	29 0.958880002498627
	30 0.961100006103516
	31 0.954699995517731
	32 0.956900000572205
	33 0.96842000246048
	34 0.98425999879837
	35 0.986460003852844
	36 0.953839998245239
	37 0.967000002861023
	38 0.973740000724792
	39 0.983280003070831
	40 0.953559999465942
	41 0.98418000459671
	42 0.952960000038147
	43 0.983380000591278
	44 0.973679995536804
	45 0.967100002765656
	46 0.95558000087738
	47 0.966800005435944
	48 0.986360008716583
	49 0.986260008811951
	50 0.986360008716583
	51 0.984680004119873
	52 0.984680004119873
	53 0.984160003662109
	54 0.966840004920959
	55 0.983980000019073
	56 0.987320005893707
	57 0.986180005073547
	58 0.985660004615784
	59 0.984900004863739
	60 0.984480004310608
	61 0.984900004863739
	62 0.983960001468658
	63 0.985380005836487
	64 0.9810600066185
	65 0.981579999923706
	66 0.985640006065369
	67 0.984380004405975
	68 0.984200000762939
	69 0.986860008239746
	70 0.965119998455048
	71 0.984900004863739
	72 0.986400005817413
	73 0.985080008506775
	74 0.980679998397827
	75 0.984380004405975
	76 0.986580009460449
	77 0.984380004405975
	78 0.984380004405975
	79 0.984499998092651
	80 0.984200000762939
	81 0.954140002727509
	82 0.984200000762939
	83 0.984200000762939
	84 0.969179999828339
	85 0.97558000087738
	86 0.986000006198883
	87 0.968860001564026
	88 0.983799998760224
	89 0.987900006771088
	90 0.986199998855591
	91 0.98760000705719
	92 0.98521999835968
	93 0.986380002498627
	94 0.985820002555847
	95 0.968960001468658
	96 0.98778000831604
	97 0.968960001468658
	98 0.976080005168915
	99 0.98539999961853
	100 0.985340001583099
};
%\addlegendentry{SAC}
\addplot [semithick, color3]
table {%
	1 0.9412
	100 0.9412
};
%\addlegendentry{Random}
\addplot [semithick, color4]
table {%
	1 0.8944
	100 0.8944
};
%\addlegendentry{ETS}
\addplot [semithick, color5]
table {%
	1 0.9669
	100 0.9669
};
%\addlegendentry{Heur-GD}
\addplot [semithick, color6, dash pattern=on 1pt off 3pt on 3pt off 3pt]
table {%
	1 0.9061
	100 0.9061
};
%\addlegendentry{Heur-LD}
\addplot [semithick, white!49.8039215686275!black, dashed]
table {%
	1 0.994
	100 0.994
};
%\addlegendentry{Heur-AT}



\nextgroupplot[title=Seed 4,
height=\figheight,
legend cell align={left},
legend style={
  fill opacity=0.8,
  draw opacity=1,
  text opacity=1,
  at={(0.97,0.03)},
  anchor=south east,
  draw=white!80!black
},
minor xtick={25, 75},
minor ytick={0.775, 0.825, 0.875, 0.925, 0.975},
tick align=outside,
tick pos=left,
%title={Labels: [[3, 8], [4, 9], [2, 6], [0, 1], [5, 7]]},
width=\figwidth,
x grid style={white!69.0196078431373!black},
xlabel={Eval. Steps},
xminorgrids,
xmajorgrids,
xmin=-3.95, xmax=104.95,
xtick style={color=black},
xtick={-25,0,50,100,125},
xticklabels={-25,0,50,100,125},
y grid style={white!69.0196078431373!black},
ylabel={ACC (\%)},
yminorgrids,
ymajorgrids,
ymin=0.75, ymax=0.929473329186439,
ytick style={color=black},
ytick={0.75,0.8,0.85,0.9,0.95},
yticklabels={75,80,85,90,95}
]
\addplot [semithick, color0]
table {%
	1 0.869040002822876
	2 0.84954000711441
	3 0.891940000057221
	4 0.854859998226166
	5 0.82840000629425
	6 0.899639999866486
	7 0.871580007076263
	8 0.880440003871918
	9 0.901979997158051
	10 0.879519999027252
	11 0.902059998512268
	12 0.898919994831085
	13 0.873899998664856
	14 0.906419999599457
	15 0.853799993991852
	16 0.880699996948242
	17 0.91746000289917
	18 0.877380001544952
	19 0.873319997787476
	20 0.871840002536774
	21 0.907159991264343
	22 0.884239997863769
	23 0.912879996299744
	24 0.852420001029968
	25 0.889100000858307
	26 0.859200000762939
	27 0.864160003662109
	28 0.88702000617981
	29 0.849040002822876
	30 0.872339994907379
	31 0.910639996528625
	32 0.912379996776581
	33 0.904359998703003
	34 0.900979998111725
	35 0.886580002307892
	36 0.898239996433258
	37 0.896280000209808
	38 0.890200002193451
	39 0.882100005149841
	40 0.880900003910065
	41 0.911279997825623
	42 0.884380004405975
	43 0.882060000896454
	44 0.875199995040894
	45 0.885720000267029
	46 0.888220002651215
	47 0.880660002231598
	48 0.886380000114441
	49 0.909659993648529
	50 0.86547999382019
	51 0.865660002231598
	52 0.86282000541687
	53 0.88289999961853
	54 0.899680004119873
	55 0.884080002307892
	56 0.903379998207092
	57 0.880659997463226
	58 0.901899991035461
	59 0.884919998645782
	60 0.883179993629456
	61 0.878419997692108
	62 0.90349999666214
	63 0.881000003814697
	64 0.880260000228882
	65 0.901740002632141
	66 0.885619995594025
	67 0.88353999376297
	68 0.861179995536804
	69 0.8910799908638
	70 0.888579995632172
	71 0.910039994716644
	72 0.907859990596771
	73 0.887579996585846
	74 0.873239998817444
	75 0.885579996109009
	76 0.889339995384216
	77 0.882899994850159
	78 0.881679992675781
	79 0.886299991607666
	80 0.883819999694824
	81 0.89681999206543
	82 0.88007999420166
	83 0.895379993915558
	84 0.861159994602203
	85 0.881439998149872
	86 0.868359997272491
	87 0.866299998760223
	88 0.866299998760223
	89 0.866800000667572
	90 0.859379999637604
	91 0.881359996795654
	92 0.879179999828339
	93 0.859499998092651
	94 0.861839997768402
	95 0.882400000095367
	96 0.864639999866486
	97 0.863439996242523
	98 0.866719996929169
	99 0.861760003566742
	100 0.887639994621277
};
%\addlegendentry{DQN}
\addplot [semithick, color1]
table {%
	1 0.900859997272491
	2 0.914959995746612
	3 0.846099998950958
	4 0.817900002002716
	5 0.816979999542236
	6 0.819060003757477
	7 0.820220005512238
	8 0.820840010643005
	9 0.830500004291534
	10 0.820840010643005
	11 0.824760007858276
	12 0.82370001077652
	13 0.82370001077652
	14 0.82370001077652
	15 0.82370001077652
	16 0.818220009803772
	17 0.818220009803772
	18 0.82688000202179
	19 0.818220009803772
	20 0.802480006217957
	21 0.808320004940033
	22 0.807140007019043
	23 0.786740002632141
	24 0.807140007019043
	25 0.781960000991821
	26 0.777179999351502
	27 0.767439999580383
	28 0.777179999351502
	29 0.772399997711182
	30 0.772399997711182
	31 0.772399997711182
	32 0.772399997711182
	33 0.772399997711182
	34 0.772399997711182
	35 0.772399997711182
	36 0.772399997711182
	37 0.772399997711182
	38 0.772399997711182
	39 0.775019998550415
	40 0.767439999580383
	41 0.777639999389648
	42 0.772399997711182
	43 0.817360000610352
	44 0.772399997711182
	45 0.772399997711182
	46 0.772399997711182
	47 0.772399997711182
	48 0.817360000610351
	49 0.772399997711182
	50 0.794879999160767
	51 0.794879999160767
	52 0.817360000610351
	53 0.839840002059936
	54 0.862320003509521
	55 0.839840002059936
	56 0.847260000705719
	57 0.862320003509521
	58 0.885660002231598
	59 0.794879999160767
	60 0.863180000782013
	61 0.869740002155304
	62 0.839840002059936
	63 0.892220003604889
	64 0.877160000801086
	65 0.889780004024506
	66 0.839840002059936
	67 0.863180000782013
	68 0.876439995765686
	69 0.884800004959106
	70 0.892220003604889
	71 0.844820001125336
	72 0.899640002250671
	73 0.862320003509521
	74 0.884800004959106
	75 0.862320003509521
	76 0.884800004959106
	77 0.892380001544952
	78 0.884800004959106
	79 0.892220003604889
	80 0.884800004959106
	81 0.892220003604889
	82 0.884800004959106
	83 0.847260000705719
	84 0.898940002918243
	85 0.898940002918243
	86 0.87576000213623
	87 0.869040002822876
	88 0.891520004272461
	89 0.906360001564026
	90 0.892220003604889
	91 0.900499999523163
	92 0.898240003585815
	93 0.905660002231598
	94 0.892380001544952
	95 0.90496000289917
	96 0.898240003585815
	97 0.90496000289917
	98 0.898240003585815
	99 0.911680002212525
	100 0.919800000190735
};
%\addlegendentry{A2C}
\addplot [semithick, color2]
table {%
	1 0.868099999427795
	2 0.868099999427795
	3 0.864140000343323
	4 0.86614000082016
	5 0.890659999847412
	6 0.863700001239777
	7 0.867039999961853
	8 0.895039999485016
	9 0.873440001010895
	10 0.867660000324249
	11 0.875800001621246
	12 0.87826000213623
	13 0.879539999961853
	14 0.856399998664856
	15 0.853960001468658
	16 0.849240002632141
	17 0.854760003089905
	18 0.846640002727509
	19 0.856960003376007
	20 0.855020000934601
	21 0.847480003833771
	22 0.851279997825623
	23 0.874520003795624
	24 0.888560001850128
	25 0.873120005130768
	26 0.864060001373291
	27 0.892379999160767
	28 0.870800006389618
	29 0.882200000286102
	30 0.87996000289917
	31 0.871880006790161
	32 0.871880006790161
	33 0.871880006790161
	34 0.862600004673004
	35 0.863500006198883
	36 0.873480002880096
	37 0.849739999771118
	38 0.856820003986358
	39 0.868480002880096
	40 0.8482200050354
	41 0.846560006141663
	42 0.831820001602173
	43 0.831820001602173
	44 0.849700002670288
	45 0.831820001602173
	46 0.831820001602173
	47 0.848699998855591
	48 0.864760000705719
	49 0.867479999065399
	50 0.845759997367859
	51 0.86055999994278
	52 0.834720003604889
	53 0.840700006484985
	54 0.828300004005432
	55 0.820940003395081
	56 0.825020005702972
	57 0.828460006713867
	58 0.873020005226135
	59 0.854040005207062
	60 0.863820004463196
	61 0.865480010509491
	62 0.859500007629395
	63 0.86204000711441
	64 0.871260006427765
	65 0.847760007381439
	66 0.857140009403229
	67 0.878480005264282
	68 0.865480010509491
	69 0.873060007095337
	70 0.888640003204346
	71 0.879100008010864
	72 0.865880010128021
	73 0.865720007419586
	74 0.855400006771088
	75 0.859420008659363
	76 0.855020005702972
	77 0.86242000579834
	78 0.869240007400513
	79 0.86438000202179
	80 0.869640007019043
	81 0.856980006694794
	82 0.849260001182556
	83 0.875640001296997
	84 0.854080004692078
	85 0.856760008335113
	86 0.865780005455017
	87 0.868280005455017
	88 0.876220002174377
	89 0.870720007419586
	90 0.855040009021759
	91 0.849840004444122
	92 0.868280005455017
	93 0.86308000087738
	94 0.863780000209808
	95 0.883720002174378
	96 0.860140001773834
	97 0.878099999427795
	98 0.894960000514984
	99 0.881940004825592
	100 0.893839998245239
};
%\addlegendentry{SAC}
\addplot [semithick, color3]
table {%
	1 0.83638
	100 0.83638
};
%\addlegendentry{Random}
\addplot [semithick, color4]
table {%
	1 0.8707
	100 0.8707
};
%\addlegendentry{ETS}
\addplot [semithick, color5]
table {%
	1 0.9129
	100 0.9129
};
%\addlegendentry{Heur-GD}
\addplot [semithick, color6, dash pattern=on 1pt off 3pt on 3pt off 3pt]
table {%
	1 0.8675
	100 0.8675
};
%\addlegendentry{Heur-LD}
\addplot [semithick, white!49.8039215686275!black, dashed]
table {%
	1 0.8328
	100 0.8328
};
%\addlegendentry{Heur-AT}



\nextgroupplot[title=Seed 5,
height=\figheight,
legend cell align={left},
legend style={fill opacity=0.8, draw opacity=1, text opacity=1, draw=white!80!black},
minor xtick={25, 75},
minor ytick={},
tick align=outside,
tick pos=left,
%title={Labels: [[9, 5], [2, 4], [7, 1], [0, 8], [6, 3]]},
width=\figwidth,
x grid style={white!69.0196078431373!black},
xlabel={Eval. Steps},
xminorgrids,
xmajorgrids,
xmin=-3.95, xmax=104.95,
xtick style={color=black},
xtick={-25,0,50,100,125},
xticklabels={-25,0,50,100,125},
y grid style={white!69.0196078431373!black},
%ylabel={ACC},
ymajorgrids,
ymin=0.835075005292892, ymax=0.933891669511795,
ytick style={color=black},
ytick={0.82,0.84,0.86,0.88,0.9,0.92,0.94},
yticklabels={82,84,86,88,90,92,94}
]
\addplot [semithick, color0]
table {%
	1 0.90804000377655
	2 0.873699996471405
	3 0.900719997882843
	4 0.888240003585815
	5 0.886600000858307
	6 0.88945999622345
	7 0.914419996738434
	8 0.88154000043869
	9 0.897220003604889
	10 0.876819999217987
	11 0.894019997119904
	12 0.878660001754761
	13 0.887699999809265
	14 0.900120003223419
	15 0.894539999961853
	16 0.885040001869202
	17 0.879940006732941
	18 0.880059998035431
	19 0.888939995765686
	20 0.880759999752045
	21 0.896639997959137
	22 0.867980000972748
	23 0.869920001029968
	24 0.860199997425079
	25 0.861919994354248
	26 0.852920002937317
	27 0.863280000686646
	28 0.849819993972778
	29 0.875839998722076
	30 0.890359995365143
	31 0.867059998512268
	32 0.88581999540329
	33 0.872799997329712
	34 0.85753999710083
	35 0.894099998474121
	36 0.88407999753952
	37 0.877460000514984
	38 0.866539993286133
	39 0.86757999420166
	40 0.858980000019073
	41 0.859979996681213
	42 0.864139995574951
	43 0.872259995937347
	44 0.860299999713898
	45 0.879379990100861
	46 0.871239998340607
	47 0.862759997844696
	48 0.869879999160767
	49 0.876959998607636
	50 0.87595999956131
	51 0.88345999956131
	52 0.866579999923706
	53 0.87617999792099
	54 0.869960000514984
	55 0.846919994354248
	56 0.870619993209839
	57 0.872219996452332
	58 0.888239994049072
	59 0.88471999168396
	60 0.876419992446899
	61 0.887199990749359
	62 0.881679992675781
	63 0.851819996833801
	64 0.89385999917984
	65 0.879899997711182
	66 0.863599994182587
	67 0.90918000459671
	68 0.882880001068115
	69 0.894020001888275
	70 0.863459997177124
	71 0.863559997081757
	72 0.885920007228851
	73 0.87055999994278
	74 0.876219999790192
	75 0.874939994812012
	76 0.865959997177124
	77 0.882059996128082
	78 0.878299994468689
	79 0.868619995117187
	80 0.863000001907349
	81 0.865859999656677
	82 0.898460004329681
	83 0.882840001583099
	84 0.860359997749329
	85 0.874679992198944
	86 0.89210000038147
	87 0.890079998970032
	88 0.870359992980957
	89 0.867919995784759
	90 0.886640002727509
	91 0.877219998836517
	92 0.909559991359711
	93 0.89121999502182
	94 0.865279996395111
	95 0.879179999828339
	96 0.877079997062683
	97 0.883899998664856
	98 0.872279994487762
	99 0.881779997348785
	100 0.873100001811981
};
%\addlegendentry{DQN}
\addplot [semithick, color1]
table {%
	1 0.903960003852844
	2 0.897739996910095
	3 0.87493999004364
	4 0.882000000476837
	5 0.891300001144409
	6 0.901579999923706
	7 0.907359998226166
	8 0.906140003204346
	9 0.903000001907349
	10 0.902579998970032
	11 0.908439998626709
	12 0.898639998435974
	13 0.905400002002716
	14 0.905620000362396
	15 0.910639998912811
	16 0.908239998817444
	17 0.912600002288818
	18 0.91708000421524
	19 0.918599998950958
	20 0.918599998950958
	21 0.916220002174377
	22 0.899140002727508
	23 0.908800003528595
	24 0.906660001277923
	25 0.903120000362396
	26 0.903120000362396
	27 0.866259999275207
	28 0.879959998130798
	29 0.866819996833801
	30 0.867260000705719
	31 0.866819996833801
	32 0.866819996833801
	33 0.866819996833801
	34 0.866819996833801
	35 0.869379999637604
	36 0.869939997196198
	37 0.866819996833801
	38 0.869939997196198
	39 0.869939997196198
	40 0.873059997558594
	41 0.879299998283386
	42 0.879299998283386
	43 0.873059997558594
	44 0.87617999792099
	45 0.879299998283386
	46 0.875759997367859
	47 0.879299998283386
	48 0.879299998283386
	49 0.879299998283386
	50 0.879299998283386
	51 0.879299998283386
	52 0.879299998283386
	53 0.879299998283386
	54 0.879299998283386
	55 0.879299998283386
	56 0.879299998283386
	57 0.879299998283386
	58 0.879299998283386
	59 0.879299998283386
	60 0.879299998283386
	61 0.879299998283386
	62 0.879299998283386
	63 0.879299998283386
	64 0.879299998283386
	65 0.879299998283386
	66 0.879299998283386
	67 0.879299998283386
	68 0.879299998283386
	69 0.879299998283386
	70 0.879299998283386
	71 0.879299998283386
	72 0.879299998283386
	73 0.879299998283386
	74 0.879299998283386
	75 0.879299998283386
	76 0.879299998283386
	77 0.879299998283386
	78 0.879299998283386
	79 0.879299998283386
	80 0.879299998283386
	81 0.879299998283386
	82 0.879299998283386
	83 0.879299998283386
	84 0.879299998283386
	85 0.879299998283386
	86 0.879299998283386
	87 0.879299998283386
	88 0.879299998283386
	89 0.879299998283386
	90 0.879299998283386
	91 0.879299998283386
	92 0.879299998283386
	93 0.879299998283386
	94 0.879299998283386
	95 0.879299998283386
	96 0.879299998283386
	97 0.879299998283386
	98 0.879299998283386
	99 0.879299998283386
	100 0.879299998283386
};
%\addlegendentry{A2C}
\addplot [semithick, color2]
table {%
	1 0.876499998569489
	2 0.876499998569489
	3 0.871379997730255
	4 0.874659996032715
	5 0.884819996356964
	6 0.880719997882843
	7 0.892520000934601
	8 0.887299997806549
	9 0.892039997577667
	10 0.889559998512268
	11 0.879339995384216
	12 0.882099997997284
	13 0.880839998722076
	14 0.89345999956131
	15 0.894120001792908
	16 0.889280002117157
	17 0.889280002117157
	18 0.896760001182556
	19 0.892440001964569
	20 0.896760001182556
	21 0.897620000839233
	22 0.896099998950958
	23 0.894260001182556
	24 0.888600001335144
	25 0.886820001602173
	26 0.885900001525879
	27 0.895840001106262
	28 0.88135999917984
	29 0.887139999866486
	30 0.886339995861054
	31 0.889360001087189
	32 0.884240000247955
	33 0.897820000648499
	34 0.901239998340607
	35 0.891440005302429
	36 0.897500004768371
	37 0.897500004768371
	38 0.897500004768371
	39 0.89342000246048
	40 0.897500004768371
	41 0.903580002784729
	42 0.904860002994537
	43 0.902680003643036
	44 0.897500004768371
	45 0.897500004768371
	46 0.896640005111694
	47 0.897500004768371
	48 0.888060007095337
	49 0.897500004768371
	50 0.897500004768371
	51 0.898220000267029
	52 0.89650000333786
	53 0.895900001525879
	54 0.895900001525879
	55 0.895900001525879
	56 0.893440001010895
	57 0.895260004997253
	58 0.891920006275177
	59 0.889100000858307
	60 0.886840007305145
	61 0.886840007305145
	62 0.885000004768371
	63 0.885240004062653
	64 0.886840007305145
	65 0.885600004196167
	66 0.885240004062653
	67 0.886840007305145
	68 0.885100002288818
	69 0.880240006446838
	70 0.878640003204346
	71 0.880240006446838
	72 0.880240006446838
	73 0.878640003204346
	74 0.878640003204346
	75 0.878640003204346
	76 0.882320005893707
	77 0.882320005893707
	78 0.882320005893707
	79 0.883400003910065
	80 0.88860000371933
	81 0.879640002250671
	82 0.884740002155304
	83 0.883320004940033
	84 0.883320004940033
	85 0.884740002155304
	86 0.885499999523163
	87 0.879040002822876
	88 0.884140002727508
	89 0.884200003147125
	90 0.882720005512237
	91 0.888000004291534
	92 0.885620000362396
	93 0.882720005512237
	94 0.884140002727508
	95 0.893120005130768
	96 0.894540002346039
	97 0.882380003929138
	98 0.882380003929138
	99 0.883800001144409
	100 0.894940001964569
};
%\addlegendentry{SAC}
\addplot [semithick, color3]
table {%
	1 0.89762
	100 0.89762
};
%\addlegendentry{Random}
\addplot [semithick, color4]
table {%
	1 0.9153
	100 0.9153
};
%\addlegendentry{ETS}
\addplot [semithick, color5]
table {%
	1 0.9033
	100 0.9033
};
%\addlegendentry{Heur-GD}
\addplot [semithick, color6, dash pattern=on 1pt off 3pt on 3pt off 3pt]
table {%
	1 0.8801
	100 0.8801
};
%\addlegendentry{Heur-LD}
\addplot [semithick, white!49.8039215686275!black, dashed]
table {%
	1 0.9053
	100 0.9053
};
%\addlegendentry{Heur-AT}



\nextgroupplot[title=Seed 6,
height=\figheight,
legend cell align={left},
legend style={
  fill opacity=0.8,
  draw opacity=1,
  text opacity=1,
  at={(0.5,0.91)},
  anchor=north,
  draw=white!80!black
},
minor xtick={25, 75},
minor ytick={0.775, 0.825, 0.875, 0.925, 0.975},
tick align=outside,
tick pos=left,
%title={Labels: [[8, 1], [7, 0], [6, 5], [2, 4], [3, 9]]},
width=\figwidth,
x grid style={white!69.0196078431373!black},
xlabel={Eval. Steps},
xminorgrids,
xmajorgrids,
xmin=-3.95, xmax=104.95,
xtick style={color=black},
xtick={-25,0,50,100,125},
xticklabels={-25,0,50,100,125},
y grid style={white!69.0196078431373!black},
%ylabel={ACC},
yminorgrids,
ymajorgrids,
ymin=0.75047, ymax=0.96453,
ytick style={color=black},
ytick={0.75,0.8,0.85,0.9,0.95,0.975},
yticklabels={75,80,85,90,95,975}
]
\addplot [semithick, color0]
table {%
1 0.912899998823802
2 0.907566662629445
3 0.927833330631256
4 0.933099997043609
5 0.915600001811981
6 0.933099997043609
7 0.915600001811981
8 0.92199999888738
9 0.912899994850159
10 0.92199999888738
11 0.9182000041008
12 0.9182000041008
13 0.863533333937327
14 0.9182000041008
15 0.9182000041008
16 0.92316666841507
17 0.870333333810171
18 0.808866663773855
19 0.8653666694959
20 0.8653666694959
21 0.812533334891001
22 0.777799999713898
23 0.777799999713898
24 0.8653666694959
25 0.779633335272471
26 0.779633335272471
27 0.79956667025884
28 0.79956667025884
29 0.819500005245209
30 0.797733334700267
31 0.797733334700267
32 0.797733334700267
33 0.819500005245209
34 0.819500005245209
35 0.819500005245209
36 0.819500005245209
37 0.819500005245209
38 0.819500005245209
39 0.819500005245209
40 0.819500005245209
41 0.797733334700267
42 0.819500005245209
43 0.819500005245209
44 0.819500005245209
45 0.819500005245209
46 0.819500005245209
47 0.819500005245209
48 0.819500005245209
49 0.819500005245209
50 0.819500005245209
51 0.819500005245209
52 0.819500005245209
53 0.819500005245209
54 0.819500005245209
55 0.819500005245209
56 0.819500005245209
57 0.819500005245209
58 0.861299995581309
59 0.819500005245209
60 0.819500005245209
61 0.819500005245209
62 0.819500005245209
63 0.819500005245209
64 0.819500005245209
65 0.850366667906443
66 0.819500005245209
67 0.819500005245209
68 0.852033332983653
69 0.819500005245209
70 0.824133336544037
71 0.819500005245209
72 0.914799992243449
73 0.800233336289724
74 0.881233330567678
75 0.847100003560384
76 0.852033332983653
77 0.852033332983653
78 0.850366667906443
79 0.850366667906443
80 0.852033332983653
81 0.913766658306122
82 0.882899995644887
83 0.85140000184377
84 0.850366667906443
85 0.850299998124441
86 0.850366667906443
87 0.847266670068105
88 0.85196666320165
89 0.850366667906443
90 0.894999996821086
91 0.862466669082642
92 0.862399999300639
93 0.879099996884664
94 0.85196666320165
95 0.910033329327901
96 0.847200000286102
97 0.893266661961873
98 0.8471333305041
99 0.879799997806549
100 0.883866659800212
};
%\addlegendentry{a2c, gae, envs=10, lr=0.0003, act=tanh}
\addplot [semithick, color1]
table {%
1 0.905266670385997
2 0.864233334859212
3 0.810099999109904
4 0.839866662025452
5 0.885899996757507
6 0.876900005340576
7 0.765333326657613
8 0.874233333269755
9 0.885866665840149
10 0.794099994500478
11 0.765233329931895
12 0.886333334445953
13 0.813366663455963
14 0.820233333110809
15 0.80789999961853
16 0.815499989191691
17 0.82433332602183
18 0.88896666765213
19 0.841799994309743
20 0.869766668478648
21 0.896999994913737
22 0.854566665490468
23 0.787966664632161
24 0.834399998188019
25 0.780200000603994
26 0.843666664759318
27 0.826933332284292
28 0.815366665522257
29 0.844133337338765
30 0.869633332888285
31 0.83806666135788
32 0.798933331171672
33 0.872233331203461
34 0.827833330631256
35 0.805966663360596
36 0.838566664854686
37 0.805966663360596
38 0.872966659069061
39 0.872966659069061
40 0.872966659069061
41 0.866699993610382
42 0.861866664886474
43 0.851199992497762
44 0.810466659069061
45 0.810433328151703
46 0.860266657670339
47 0.81146666208903
48 0.790466662247976
49 0.851266658306122
50 0.810433328151703
51 0.790466662247976
52 0.851199992497762
53 0.810433328151703
54 0.810433328151703
55 0.819299991925557
56 0.822200000286102
57 0.841799994309743
58 0.851199992497762
59 0.790466662247976
60 0.79933332602183
61 0.79933332602183
62 0.822200000286102
63 0.801233327388763
64 0.810433328151703
65 0.848233326276143
66 0.810433328151703
67 0.810433328151703
68 0.822200000286102
69 0.822200000286102
70 0.780799992879232
71 0.810433328151703
72 0.790466662247976
73 0.812266663710276
74 0.790466662247976
75 0.856033329168956
76 0.822200000286102
77 0.801599995295207
78 0.780799992879232
79 0.803066662947337
80 0.812999999523163
81 0.810433328151703
82 0.856033329168956
83 0.83506666024526
84 0.812266663710276
85 0.815099994341532
86 0.846499995390574
87 0.84609999259313
88 0.848933323224386
89 0.819299991925557
90 0.821133327484131
91 0.835833326975504
92 0.821899994214376
93 0.850833328564962
94 0.820066658655802
95 0.820066658655802
96 0.820066658655802
97 0.79953332344691
98 0.834266658624013
99 0.781566659609477
100 0.820066658655802
};
%\addlegendentry{DQN, n train envs=30}
\addplot [semithick, color2]
table {%
1 0.9521
100 0.9521
};
%\addlegendentry{Random}
\addplot [semithick, color3]
table {%
1 0.9548
100 0.9548
};
%\addlegendentry{ETS}
\addplot [semithick, color4]
table {%
1 0.9001
100 0.9001
};
%\addlegendentry{Heur-GD}
\addplot [semithick, color5, dash pattern=on 1pt off 3pt on 3pt off 3pt]
table {%
1 0.8544
100 0.8544
};
%\addlegendentry{Heur-LD}
\addplot [semithick, color6, dashed]
table {%
1 0.7602
100 0.7602
};
%\addlegendentry{Heur-AT}



\nextgroupplot[title=Seed 7,
height=\figheight,
legend cell align={left},
legend style={
  fill opacity=0.8,
  draw opacity=1,
  text opacity=1,
  at={(0.5,0.91)},
  anchor=north,
  draw=white!80!black
},
minor xtick={25, 75},
minor ytick={},
tick align=outside,
tick pos=left,
%title={Labels: [[8, 5], [0, 2], [1, 9], [7, 3], [6, 4]]},
width=\figwidth,
x grid style={white!69.0196078431373!black},
xlabel={Eval. Steps},
xminorgrids,
xmajorgrids,
xmin=-3.95, xmax=104.95,
xtick style={color=black},
xtick={-25,0,50,100,125},
xticklabels={-25,0,50,100,125},
y grid style={white!69.0196078431373!black},
%ylabel={ACC},
ymajorgrids,
ymin=0.869310013518333, ymax=0.97248999935627,
ytick style={color=black},
ytick={0.86,0.88,0.9,0.92,0.94,0.96,0.98},
yticklabels={86,88,90,92,94,96,98}
]
\addplot [semithick, color0]
table {%
	1 0.913319997787476
	2 0.918559999465942
	3 0.927160005569458
	4 0.920380001068115
	5 0.913320000171661
	6 0.906619997024536
	7 0.9353600025177
	8 0.927519998550415
	9 0.907300002574921
	10 0.925239999294281
	11 0.913660001754761
	12 0.926619997024536
	13 0.925860002040863
	14 0.932279996871948
	15 0.906640002727509
	16 0.922220001220703
	17 0.940580003261566
	18 0.928900001049042
	19 0.902419998645782
	20 0.929979994297028
	21 0.927580001354218
	22 0.924640002250671
	23 0.92922000169754
	24 0.93168000459671
	25 0.919200003147125
	26 0.943560004234314
	27 0.92842000246048
	28 0.929679999351501
	29 0.921260006427765
	30 0.908080008029938
	31 0.93172000169754
	32 0.916879999637604
	33 0.892380011081696
	34 0.924140002727509
	35 0.883840007781982
	36 0.895860006809235
	37 0.90300000667572
	38 0.91521999835968
	39 0.886400010585785
	40 0.893160009384155
	41 0.892720007896423
	42 0.903680005073547
	43 0.88664000749588
	44 0.916860003471374
	45 0.895640006065369
	46 0.909840004444122
	47 0.897999999523163
	48 0.890040006637573
	49 0.887800006866455
	50 0.9085600066185
	51 0.907280006408691
	52 0.915840005874634
	53 0.892960004806519
	54 0.895940005779266
	55 0.915560002326965
	56 0.880320010185242
	57 0.907240006923675
	58 0.901520006656647
	59 0.891360003948212
	60 0.892280006408691
	61 0.890800004005432
	62 0.885960009098053
	63 0.899160003662109
	64 0.906220009326935
	65 0.892840006351471
	66 0.87954000711441
	67 0.892160007953644
	68 0.90084000825882
	69 0.884080009460449
	70 0.893880009651184
	71 0.906500005722046
	72 0.893880009651184
	73 0.898360006809235
	74 0.901200008392334
	75 0.880320010185242
	76 0.89300000667572
	77 0.901320006847382
	78 0.886680006980896
	79 0.892300002574921
	80 0.90110000371933
	81 0.883060007095337
	82 0.909080007076263
	83 0.893660008907318
	84 0.885180006027222
	85 0.903380007743836
	86 0.898080008029938
	87 0.90510000705719
	88 0.894680006504059
	89 0.908280005455017
	90 0.906720008850098
	91 0.903060007095337
	92 0.887460005283356
	93 0.909320006370544
	94 0.903300001621246
	95 0.886700005531311
	96 0.900260007381439
	97 0.899200000762939
	98 0.925260002613068
	99 0.913320004940033
	100 0.892160003185272
};
%\addlegendentry{DQN}
\addplot [semithick, color1]
table {%
	1 0.895880002975464
	2 0.890360004901886
	3 0.902779991626739
	4 0.896099984645844
	5 0.910119993686676
	6 0.903539998531341
	7 0.909159996509552
	8 0.904319994449616
	9 0.921339995861054
	10 0.902319996356964
	11 0.913199992179871
	12 0.903599989414215
	13 0.910199990272522
	14 0.903599989414215
	15 0.903599989414215
	16 0.903959991931915
	17 0.903959991931915
	18 0.923399991989136
	19 0.903599989414215
	20 0.910199990272522
	21 0.910199990272522
	22 0.910199990272522
	23 0.910199990272522
	24 0.910199990272522
	25 0.910199990272522
	26 0.916799991130829
	27 0.923399991989136
	28 0.923399991989136
	29 0.921239991188049
	30 0.916799991130829
	31 0.924459993839264
	32 0.924599997997284
	33 0.911419997215271
	34 0.907119991779327
	35 0.914699990749359
	36 0.91815999507904
	37 0.921299991607666
	38 0.920499994754791
	39 0.923119995594025
	40 0.929759998321533
	41 0.909039993286133
	42 0.936899998188019
	43 0.920299997329712
	44 0.922260000705719
	45 0.923639991283417
	46 0.911199994087219
	47 0.940060000419617
	48 0.924499998092651
	49 0.931080000400543
	50 0.940920004844666
	51 0.92529999256134
	52 0.930220003128052
	53 0.913799996376037
	54 0.919679996967316
	55 0.92559999704361
	56 0.941960000991821
	57 0.911899993419647
	58 0.925339992046356
	59 0.933540003299713
	60 0.922500002384186
	61 0.932120001316071
	62 0.921299998760223
	63 0.920280001163483
	64 0.935200006961822
	65 0.932940008640289
	66 0.918440001010895
	67 0.926019995212555
	68 0.914140002727509
	69 0.928560001850128
	70 0.926840002536774
	71 0.933600001335144
	72 0.933040008544922
	73 0.935560002326965
	74 0.917020001411438
	75 0.917040002346039
	76 0.917580003738403
	77 0.922880005836487
	78 0.934060006141663
	79 0.931760003566742
	80 0.919899997711182
	81 0.936620004177094
	82 0.916160004138947
	83 0.935100011825562
	84 0.914500007629395
	85 0.933360004425049
	86 0.917560007572174
	87 0.932880005836487
	88 0.922180004119873
	89 0.929680001735687
	90 0.915780005455017
	91 0.926980004310608
	92 0.919020009040832
	93 0.945880012512207
	94 0.911440000534058
	95 0.9353600025177
	96 0.919940004348755
	97 0.922740006446838
	98 0.918200006484985
	99 0.91646000623703
	100 0.934880006313324
};
%\addlegendentry{A2C}
\addplot [semithick, color2]
table {%
	1 0.913600015640259
	2 0.913600015640259
	3 0.913600015640259
	4 0.914260010719299
	5 0.946700003147125
	6 0.925860011577606
	7 0.93814001083374
	8 0.940800008773804
	9 0.940180008411408
	10 0.931320009231567
	11 0.932700006961823
	12 0.933200006484985
	13 0.939340007305145
	14 0.932700006961823
	15 0.936260006427765
	16 0.932860009670258
	17 0.939440004825592
	18 0.934320003986359
	19 0.932860009670258
	20 0.934480006694794
	21 0.934820003509522
	22 0.939600007534027
	23 0.939440004825592
	24 0.941220004558563
	25 0.941060001850128
	26 0.941020004749298
	27 0.940440006256103
	28 0.939480004310608
	29 0.940040009021759
	30 0.948520002365112
	31 0.944600005149841
	32 0.946380004882812
	33 0.946220002174377
	34 0.946240003108978
	35 0.946240003108978
	36 0.944460003376007
	37 0.940340006351471
	38 0.941800000667572
	39 0.934440009593964
	40 0.9360600066185
	41 0.937420008182526
	42 0.933900005817413
	43 0.933900005817413
	44 0.938000009059906
	45 0.9360600066185
	46 0.9360600066185
	47 0.935360000133514
	48 0.941960003376007
	49 0.942160003185272
	50 0.941320004463196
	51 0.937840006351471
	52 0.936220009326935
	53 0.934140009880066
	54 0.940900003910065
	55 0.940900003910065
	56 0.9360600066185
	57 0.940340006351471
	58 0.944880003929138
	59 0.939920003414154
	60 0.930540008544922
	61 0.938140003681183
	62 0.938060007095337
	63 0.945440003871918
	64 0.938320004940033
	65 0.9375
	66 0.943960003852844
	67 0.939840006828308
	68 0.938620004653931
	69 0.943799998760223
	70 0.939940001964569
	71 0.939680004119873
	72 0.94518000125885
	73 0.93954000711441
	74 0.941460003852844
	75 0.943960003852844
	76 0.937760007381439
	77 0.936660008430481
	78 0.946040003299713
	79 0.948460001945496
	80 0.925340006351471
	81 0.929800007343292
	82 0.941780004501343
	83 0.939920003414154
	84 0.931880009174347
	85 0.941760003566742
	86 0.937340006828308
	87 0.937280006408691
	88 0.94014000415802
	89 0.941780002117157
	90 0.929600002765656
	91 0.940560007095337
	92 0.939640002250671
	93 0.93110000371933
	94 0.939980006217957
	95 0.938140006065369
	96 0.938920006752014
	97 0.940480005741119
	98 0.940020005702973
	99 0.936260006427765
	100 0.939120006561279
};
%\addlegendentry{SAC}
\addplot [semithick, color3]
table {%
	1 0.94402
	100 0.94402
};
%\addlegendentry{Random}
\addplot [semithick, color4]
table {%
	1 0.9531
	100 0.9531
};
%\addlegendentry{ETS}
\addplot [semithick, color5]
table {%
	1 0.9176
	100 0.9176
};
%\addlegendentry{Heur-GD}
\addplot [semithick, color6, dash pattern=on 1pt off 3pt on 3pt off 3pt]
table {%
	1 0.9514
	100 0.9514
};
%\addlegendentry{Heur-LD}
\addplot [semithick, white!49.8039215686275!black, dashed]
table {%
	1 0.9678
	100 0.9678
};
%\addlegendentry{Heur-AT}



\nextgroupplot[title=Seed 8,
height=\figheight,
legend cell align={left},
legend style={
  fill opacity=0.8,
  draw opacity=1,
  text opacity=1,
  at={(0.03,0.03)},
  anchor=south west,
  draw=white!80!black
},
minor xtick={25, 75},
minor ytick={},
tick align=outside,
tick pos=left,
%title={Labels: [[8, 6], [9, 0], [2, 5], [7, 1], [4, 3]]},
width=\figwidth,
x grid style={white!69.0196078431373!black},
xlabel={Eval. Steps},
xminorgrids,
xmajorgrids,
xmin=-3.95, xmax=104.95,
xtick style={color=black},
xtick={-25,0,50,100,125},
xticklabels={-25,0,50,100,125},
y grid style={white!69.0196078431373!black},
ylabel={ACC (\%)},
ymajorgrids,
ymin=0.848394991755486, ymax=0.96,
ytick style={color=black},
ytick={0.84,0.86,0.88,0.9,0.92,0.94,0.96},
yticklabels={84,86,88,90,92,94,96}
]
\addplot [semithick, color0]
table {%
	1 0.900880002975464
	2 0.912259998321533
	3 0.904700000286102
	4 0.915420002937317
	5 0.914839999675751
	6 0.905720002651215
	7 0.919560000896454
	8 0.90458000421524
	9 0.938380002975464
	10 0.910859999656677
	11 0.900859999656677
	12 0.872179992198944
	13 0.921759994029999
	14 0.924419991970062
	15 0.917960000038147
	16 0.907239995002747
	17 0.916419997215271
	18 0.941619997024536
	19 0.893900001049042
	20 0.923179998397827
	21 0.900799999237061
	22 0.922160000801087
	23 0.918139996528626
	24 0.922660000324249
	25 0.921239998340607
	26 0.923660004138947
	27 0.914500000476837
	28 0.928399996757507
	29 0.93425999879837
	30 0.93074000120163
	31 0.932740008831024
	32 0.921120007038117
	33 0.925960009098053
	34 0.926600003242493
	35 0.933480007648468
	36 0.926240000724792
	37 0.941160001754761
	38 0.926840002536774
	39 0.943620002269745
	40 0.925399997234345
	41 0.902780005931854
	42 0.910520002841949
	43 0.936420001983643
	44 0.900639998912811
	45 0.912560000419617
	46 0.908819999694824
	47 0.929080004692078
	48 0.913500001430511
	49 0.928140006065369
	50 0.922260003089905
	51 0.909619996547699
	52 0.925
	53 0.929880003929138
	54 0.913540003299713
	55 0.910520005226135
	56 0.902100005149841
	57 0.898940005302429
	58 0.902479999065399
	59 0.925340003967285
	60 0.905240004062653
	61 0.897380003929138
	62 0.915880002975464
	63 0.881899995803833
	64 0.906680002212524
	65 0.922120006084442
	66 0.926960005760193
	67 0.927160005569458
	68 0.910760002136231
	69 0.890300004482269
	70 0.91150000333786
	71 0.895880000591278
	72 0.910460004806519
	73 0.915480000972748
	74 0.902900006771088
	75 0.917259998321533
	76 0.884019997119904
	77 0.907779998779297
	78 0.903619997501373
	79 0.902679994106293
	80 0.898859996795654
	81 0.89731999874115
	82 0.909739995002747
	83 0.928619999885559
	84 0.904839994907379
	85 0.922619998455048
	86 0.907579998970032
	87 0.886920006275177
	88 0.898959999084473
	89 0.904360001087189
	90 0.898959999084473
	91 0.89364000082016
	92 0.888140001296997
	93 0.910639998912811
	94 0.923340003490448
	95 0.89527999162674
	96 0.904720001220703
	97 0.911979997158051
	98 0.900299999713898
	99 0.911719999313355
	100 0.916680002212524
};
%\addlegendentry{DQN}
\addplot [semithick, color1]
table {%
	1 0.929460017681122
	2 0.93730001449585
	3 0.92974000453949
	4 0.917700004577637
	5 0.919140000343323
	6 0.939460000991821
	7 0.937739999294281
	8 0.933620002269745
	9 0.930639998912811
	10 0.909820001125336
	11 0.925339999198913
	12 0.92610000371933
	13 0.932980003356934
	14 0.92610000371933
	15 0.92610000371933
	16 0.929860002994537
	17 0.929860002994537
	18 0.946740002632141
	19 0.929860002994537
	20 0.929860002994537
	21 0.931900000572205
	22 0.932980003356934
	23 0.932980003356934
	24 0.92610000371933
	25 0.92610000371933
	26 0.92610000371933
	27 0.92610000371933
	28 0.92610000371933
	29 0.939860002994537
	30 0.932980003356934
	31 0.932980003356934
	32 0.932980003356934
	33 0.92610000371933
	34 0.932980003356934
	35 0.92610000371933
	36 0.92610000371933
	37 0.92610000371933
	38 0.92610000371933
	39 0.92610000371933
	40 0.92610000371933
	41 0.932980003356934
	42 0.92610000371933
	43 0.936820001602173
	44 0.932980003356934
	45 0.92610000371933
	46 0.92610000371933
	47 0.92610000371933
	48 0.92610000371933
	49 0.92610000371933
	50 0.92802000284195
	51 0.92802000284195
	52 0.92802000284195
	53 0.92802000284195
	54 0.929940001964569
	55 0.92802000284195
	56 0.92610000371933
	57 0.933780000209808
	58 0.935699999332428
	59 0.92802000284195
	60 0.92802000284195
	61 0.933780000209808
	62 0.92610000371933
	63 0.933780000209808
	64 0.933780000209808
	65 0.929940001964569
	66 0.931860001087189
	67 0.933780000209808
	68 0.929940001964569
	69 0.933780000209808
	70 0.931860001087189
	71 0.92802000284195
	72 0.933780000209808
	73 0.933780000209808
	74 0.933780000209808
	75 0.933780000209808
	76 0.933780000209808
	77 0.935699999332428
	78 0.935699999332428
	79 0.933780000209808
	80 0.933780000209808
	81 0.935699999332428
	82 0.935699999332428
	83 0.929940001964569
	84 0.929940001964569
	85 0.935699999332428
	86 0.933780000209808
	87 0.933780000209808
	88 0.935699999332428
	89 0.929940001964569
	90 0.935699999332428
	91 0.935699999332428
	92 0.933780000209808
	93 0.935699999332428
	94 0.933780000209808
	95 0.935699999332428
	96 0.933780000209808
	97 0.935699999332428
	98 0.935699999332428
	99 0.933780000209808
	100 0.935699999332428
};
%\addlegendentry{A2C}
\addplot [semithick, color2]
table {%
	1 0.92279999256134
	2 0.92279999256134
	3 0.92279999256134
	4 0.922440001964569
	5 0.915979998111725
	6 0.91945999622345
	7 0.931519999504089
	8 0.928900001049042
	9 0.913780002593994
	10 0.916660001277923
	11 0.913980002403259
	12 0.924640004634857
	13 0.916460001468658
	14 0.914980003833771
	15 0.921080002784729
	16 0.921980001926422
	17 0.917400002479553
	18 0.908060002326965
	19 0.91978000164032
	20 0.921559998989105
	21 0.905060000419617
	22 0.918680002689362
	23 0.915900001525879
	24 0.922980005741119
	25 0.92172000169754
	26 0.930820000171661
	27 0.935240001678467
	28 0.923880000114441
	29 0.910220003128052
	30 0.919620001316071
	31 0.929140000343323
	32 0.914880003929138
	33 0.930740003585815
	34 0.919180002212524
	35 0.92826000213623
	36 0.919739999771118
	37 0.930460002422333
	38 0.934520001411438
	39 0.920640003681183
	40 0.92940000295639
	41 0.928500003814697
	42 0.928500001430512
	43 0.929520003795624
	44 0.924080004692078
	45 0.931280002593994
	46 0.929679999351501
	47 0.941920001506805
	48 0.941440005302429
	49 0.940160000324249
	50 0.939879996776581
	51 0.932119998931885
	52 0.93456000328064
	53 0.937000000476837
	54 0.9357200050354
	55 0.940000002384186
	56 0.941060001850128
	57 0.938980000019073
	58 0.923219997882843
	59 0.933960001468659
	60 0.935940001010895
	61 0.935320000648498
	62 0.935119998455048
	63 0.924140002727509
	64 0.930019998550415
	65 0.91978000164032
	66 0.927820003032684
	67 0.937740001678467
	68 0.923259997367859
	69 0.937500002384186
	70 0.933880002498627
	71 0.933280003070831
	72 0.931460003852844
	73 0.938720004558563
	74 0.935740003585815
	75 0.935420000553131
	76 0.935040001869202
	77 0.940480003356934
	78 0.939140005111694
	79 0.938820004463196
	80 0.935740003585815
	81 0.937700002193451
	82 0.940700001716614
	83 0.936640002727509
	84 0.930720000267029
	85 0.943140001296997
	86 0.942920002937317
	87 0.941440002918243
	88 0.940000004768372
	89 0.941840002536774
	90 0.935720002651215
	91 0.933159999847412
	92 0.938180003166199
	93 0.939080002307892
	94 0.943180003166199
	95 0.939660005569458
	96 0.936820001602173
	97 0.942820003032684
	98 0.931519997119904
	99 0.943180003166199
	100 0.938500003814697
};
%\addlegendentry{SAC}
\addplot [semithick, color3]
table {%
	1 0.93148
	100 0.93148
};
%\addlegendentry{Random}
\addplot [semithick, color4]
table {%
	1 0.948
	100 0.948
};
%\addlegendentry{ETS}
\addplot [semithick, color5]
table {%
	1 0.95
	100 0.95
};
%\addlegendentry{Heur-GD}
\addplot [semithick, color6, dash pattern=on 1pt off 3pt on 3pt off 3pt]
table {%
	1 0.8786
	100 0.8786
};
%\addlegendentry{Heur-LD}
\addplot [semithick, white!49.8039215686275!black, dashed]
table {%
	1 0.9393
	100 0.9393
};
%\addlegendentry{Heur-AT}



\nextgroupplot[title=Seed 9,
height=\figheight,
legend cell align={left},
legend style={
  fill opacity=0.8,
  draw opacity=1,
  text opacity=1,
  at={(0.5,0.91)},
  anchor=north,
  draw=white!80!black
},
minor xtick={25, 75},
minor ytick={},
tick align=outside,
tick pos=left,
%title={Labels: [[8, 4], [7, 2], [1, 9], [3, 0], [6, 5]]},
width=\figwidth,
x grid style={white!69.0196078431373!black},
xlabel={Eval. Steps},
xminorgrids,
xmajorgrids,
xmin=-3.95, xmax=104.95,
xtick style={color=black},
xtick={-25,0,50,100,125},
xticklabels={-25,0,50,100,125},
y grid style={white!69.0196078431373!black},
%ylabel={ACC},
ymajorgrids,
ymin=0.889279995059967, ymax=0.98,
ytick style={color=black},
ytick={0.88,0.9,0.92,0.94,0.96,0.98},
yticklabels={88,90,92,94,96,98}
]
\addplot [semithick, color0]
table {%
	1 0.925939996242523
	2 0.921499993801117
	3 0.937920000553131
	4 0.933399994373321
	5 0.928640003204346
	6 0.946819994449615
	7 0.948119995594025
	8 0.959260008335113
	9 0.941440005302429
	10 0.960839996337891
	11 0.947199997901917
	12 0.961319997310638
	13 0.923739998340607
	14 0.935899999141693
	15 0.935279994010925
	16 0.946059997081756
	17 0.907939994335175
	18 0.947039997577667
	19 0.919839999675751
	20 0.899259996414185
	21 0.955639996528626
	22 0.939659996032715
	23 0.963559987545013
	24 0.928619992733002
	25 0.953659992218017
	26 0.9196399974823
	27 0.949759998321533
	28 0.957060000896454
	29 0.958579998016358
	30 0.931440000534058
	31 0.933619999885559
	32 0.947920002937317
	33 0.959499995708466
	34 0.953819997310638
	35 0.942039999961853
	36 0.933619997501373
	37 0.952139999866486
	38 0.942719995975494
	39 0.93691999912262
	40 0.934539995193482
	41 0.928699996471405
	42 0.927340002059937
	43 0.922279996871948
	44 0.952459998130798
	45 0.950479996204376
	46 0.941559994220734
	47 0.943599996566772
	48 0.930439999103546
	49 0.93481999874115
	50 0.957319991588593
	51 0.917779998779297
	52 0.939059998989105
	53 0.947619998455048
	54 0.951699998378754
	55 0.931860001087189
	56 0.916979997158051
	57 0.951199994087219
	58 0.926960000991821
	59 0.949520001411438
	60 0.952779996395111
	61 0.958399996757507
	62 0.941259996891022
	63 0.932260003089905
	64 0.946459996700287
	65 0.953639996051788
	66 0.946840000152588
	67 0.947380001544952
	68 0.941039998531342
	69 0.935839993953705
	70 0.930980002880096
	71 0.933880002498627
	72 0.958019995689392
	73 0.945940001010895
	74 0.936260004043579
	75 0.953799996376038
	76 0.953839995861054
	77 0.959119999408722
	78 0.909100000858307
	79 0.939499998092651
	80 0.957119998931885
	81 0.929820001125336
	82 0.950239999294281
	83 0.942099997997284
	84 0.928659996986389
	85 0.937000005245209
	86 0.942519998550415
	87 0.955179994106293
	88 0.947699995040894
	89 0.941080000400543
	90 0.934680004119873
	91 0.928780000209808
	92 0.949599997997284
	93 0.939839999675751
	94 0.956979992389679
	95 0.934199995994568
	96 0.948380000591278
	97 0.949100000858307
	98 0.949499995708466
	99 0.945340001583099
	100 0.953159997463226
};
%\addlegendentry{DQN}
\addplot [semithick, color1]
table {%
	1 0.952539992332458
	2 0.94775999546051
	3 0.917819998264313
	4 0.914699995517731
	5 0.911399993896484
	6 0.942999994754791
	7 0.932299995422363
	8 0.925699996948242
	9 0.926859998703003
	10 0.924039995670319
	11 0.907859997749329
	12 0.914999997615814
	13 0.904299998283386
	14 0.904299998283386
	15 0.914999997615814
	16 0.914999997615814
	17 0.937299997806549
	18 0.896080000400543
	19 0.900539999008179
	20 0.914999997615814
	21 0.914099998474121
	22 0.900539999008179
	23 0.904299998283386
	24 0.900539999008179
	25 0.904299998283386
	26 0.896779999732971
	27 0.893020000457764
	28 0.900539999008179
	29 0.904299998283386
	30 0.900539999008179
	31 0.904299998283386
	32 0.896779999732971
	33 0.904299998283386
	34 0.904299998283386
	35 0.900539999008179
	36 0.904299998283386
	37 0.901439995765686
	38 0.904299998283386
	39 0.900539999008179
	40 0.896779999732971
	41 0.900539999008179
	42 0.897679996490478
	43 0.900539999008179
	44 0.904299998283386
	45 0.904299998283386
	46 0.904299998283386
	47 0.904299998283386
	48 0.904299998283386
	49 0.904299998283386
	50 0.904299998283386
	51 0.904299998283386
	52 0.904299998283386
	53 0.904299998283386
	54 0.904299998283386
	55 0.904299998283386
	56 0.904299998283386
	57 0.904299998283386
	58 0.904299998283386
	59 0.904299998283386
	60 0.904299998283386
	61 0.904299998283386
	62 0.904299998283386
	63 0.904299998283386
	64 0.904299998283386
	65 0.904299998283386
	66 0.904299998283386
	67 0.904299998283386
	68 0.904299998283386
	69 0.904299998283386
	70 0.904299998283386
	71 0.904299998283386
	72 0.904299998283386
	73 0.904299998283386
	74 0.904299998283386
	75 0.904299998283386
	76 0.904299998283386
	77 0.904299998283386
	78 0.904299998283386
	79 0.904299998283386
	80 0.904299998283386
	81 0.904299998283386
	82 0.904299998283386
	83 0.904299998283386
	84 0.904299998283386
	85 0.904299998283386
	86 0.904299998283386
	87 0.904299998283386
	88 0.904299998283386
	89 0.904299998283386
	90 0.904299998283386
	91 0.904299998283386
	92 0.904299998283386
	93 0.904299998283386
	94 0.904299998283386
	95 0.908699998855591
	96 0.904299998283386
	97 0.904299998283386
	98 0.904299998283386
	99 0.904299998283386
	100 0.913099999427796
};
%\addlegendentry{A2C}
\addplot [semithick, color2]
table {%
	1 0.958700001239777
	2 0.958700001239777
	3 0.957340004444122
	4 0.940739996433258
	5 0.931439998149872
	6 0.928639996051788
	7 0.92331999540329
	8 0.92331999540329
	9 0.922579996585846
	10 0.920219995975494
	11 0.92139999628067
	12 0.929859998226166
	13 0.925539996623993
	14 0.922579996585846
	15 0.922579996585846
	16 0.922579996585846
	17 0.922499997615814
	18 0.922579996585846
	19 0.922579996585846
	20 0.92139999628067
	21 0.92139999628067
	22 0.932759997844696
	23 0.932759997844696
	24 0.924139997959137
	25 0.922579996585846
	26 0.936499996185303
	27 0.936499996185303
	28 0.936499996185303
	29 0.919259996414185
	30 0.919259996414185
	31 0.919259996414185
	32 0.919259996414185
	33 0.927879996299744
	34 0.931199996471405
	35 0.931199996471405
	36 0.931199996471405
	37 0.948439996242523
	38 0.931199996471405
	39 0.939819996356964
	40 0.948439996242523
	41 0.948439996242523
	42 0.957059996128082
	43 0.948439996242523
	44 0.939819996356964
	45 0.947699997425079
	46 0.93907999753952
	47 0.956319997310638
	48 0.956319997310638
	49 0.956319997310638
	50 0.957059996128082
	51 0.955319998264313
	52 0.946959998607636
	53 0.947699997425079
	54 0.954839999675751
	55 0.946959998607636
	56 0.956319997310638
	57 0.946959998607636
	58 0.948519999980926
	59 0.955879998207092
	60 0.948519999980926
	61 0.948519999980926
	62 0.955879998207092
	63 0.955879998207092
	64 0.956619997024536
	65 0.948519999980926
	66 0.956619997024536
	67 0.947780001163483
	68 0.948519999980926
	69 0.947780001163483
	70 0.955879998207092
	71 0.948519999980926
	72 0.955879998207092
	73 0.948519999980926
	74 0.948519999980926
	75 0.948519999980926
	76 0.948519999980926
	77 0.94925999879837
	78 0.948519999980926
	79 0.948519999980926
	80 0.948519999980926
	81 0.948519999980926
	82 0.947780001163483
	83 0.947780001163483
	84 0.948519999980926
	85 0.948519999980926
	86 0.95311999797821
	87 0.952379999160767
	88 0.960479996204376
	89 0.95311999797821
	90 0.95311999797821
	91 0.952379999160767
	92 0.95311999797821
	93 0.952379999160767
	94 0.95311999797821
	95 0.952379999160767
	96 0.952379999160767
	97 0.95311999797821
	98 0.95311999797821
	99 0.95311999797821
	100 0.95311999797821
};
%\addlegendentry{SAC}
\addplot [semithick, color3]
table {%
	1 0.94812
	100 0.94812
};
%\addlegendentry{Random}
\addplot [semithick, color4]
table {%
	1 0.9671
	100 0.9671
};
%\addlegendentry{ETS}
\addplot [semithick, color5]
table {%
	1 0.9527
	100 0.9527
};
%\addlegendentry{Heur-GD}
\addplot [semithick, color6, dash pattern=on 1pt off 3pt on 3pt off 3pt]
table {%
	1 0.9002
	100 0.9002
};
%\addlegendentry{Heur-LD}
\addplot [semithick, white!49.8039215686275!black, dashed]
table {%
	1 0.9029
	100 0.9029
};
%\addlegendentry{Heur-AT}



\end{groupplot}

\end{tikzpicture}
  \vspace{-3mm}
  \caption{Performance progress measured in ACC (\%) for all methods in the 10 test environments for {\bf Split FashionMNIST} in the New Dataset experiment. We plot the performance progress for A2C and DQN for 100 evaluation steps equidistantly distributed over the training episodes. The performance for the Random, ETS, and Heuristic scheduling baselines are plotted as straight lines.  }
  \label{fig:policy_rewards_fashionmnist_new_dataset_paperD}
  \vspace{-3mm}
\end{figure}

%%% Tables
\clearpage

\begin{table}[t]
\caption{The ACC and rank for each seed (10-19) in every test environment with {\bf Split MNIST} for New Task Order experiment. Under each column named 'Seed X', we show the ACC (\%) and, in parenthesis, the rank for the corresponding method for the specific seed. We also show the average rank of each method in the test environment under 'Rank'. Note that the ACC for ETS, Heur-GD, Heur-LD, and Heur-AT are copied across the seeds, since the seed only affects the policy initialization in DQN and A2C and the random selection in Random.}
\vspace{-2mm}
\label{tab:rankings_mnist_new_task_order}
\resizebox{\textwidth}{!}{
\begin{tabular}{lcccccccc}
\toprule
 \multicolumn{1}{l}{}   & \multicolumn{4}{c}{{\bf Test Env. Seed 10}}             & \multicolumn{4}{c}{{\bf Test Env. Seed 11}} \\
    \cmidrule(lr){2-5} \cmidrule(lr){6-9} 
Method     & Seed 1     & Seed 2     & Seed 3     & Rank & Seed 1     & Seed 2     & Seed 3     & Rank \\ \midrule
Random     & 93.2 (6.0) & 93.7 (6.0) & 96.2 (1.0) & 4.33 & 92.2 (3.0) & 90.8 (7.0) & 90.5 (7.0) & 5.67 \\
ETS        & 89.9 (7.0) & 89.9 (7.0) & 89.9 (7.0) & 7    & 93.8 (2.0) & 93.8 (3.0) & 93.8 (3.0) & 2.67 \\
Heur-GD & 94.6 (3.5) & 94.6 (4.5) & 94.6 (5.5) & 4.5  & 91.3 (6.0) & 91.3 (5.0) & 91.3 (5.0) & 5.33 \\
Heur-LD & 95.6 (1.0) & 95.6 (1.0) & 95.6 (3.0) & 1.67 & 91.3 (6.0) & 91.3 (5.0) & 91.3 (5.0) & 5.33 \\
Heur-AT & 94.6 (3.5) & 94.6 (4.5) & 94.6 (5.5) & 4.5  & 91.3 (6.0) & 91.3 (5.0) & 91.3 (5.0) & 5.33 \\
DQN        & 93.2 (5.0) & 95.5 (2.0) & 95.7 (2.0) & 3    & 91.8 (4.0) & 96.0 (2.0) & 95.7 (2.0) & 2.67 \\
A2C        & 95.3 (2.0) & 95.3 (3.0) & 95.3 (4.0) & 3    & 96.5 (1.0) & 96.5 (1.0) & 96.5 (1.0) & 1    \\ \midrule
    & \multicolumn{4}{c}{{\bf Test Env. Seed 12}}             & \multicolumn{4}{c}{{\bf Test Env. Seed 13}} \\
    \cmidrule(lr){2-5} \cmidrule(lr){6-9} 
Method     & Seed 1     & Seed 2     & Seed 3     & Rank & Seed 1     & Seed 2     & Seed 3     & Rank \\ \midrule
Random     & 90.7 (7.0) & 88.8 (7.0) & 95.2 (1.0) & 5    & 93.4 (5.0) & 94.4 (6.0) & 89.9 (7.0) & 6    \\
ETS        & 91.7 (6.0) & 91.7 (6.0) & 91.7 (7.0) & 6.33 & 95.0 (2.0) & 95.0 (3.0) & 95.0 (3.0) & 2.67 \\
Heur-GD & 94.8 (1.5) & 94.8 (1.5) & 94.8 (2.5) & 1.83 & 94.7 (3.5) & 94.7 (4.5) & 94.7 (4.5) & 4.17 \\
Heur-LD & 91.9 (5.0) & 91.9 (5.0) & 91.9 (6.0) & 5.33 & 93.2 (7.0) & 93.2 (7.0) & 93.2 (6.0) & 6.67 \\
Heur-AT & 94.8 (1.5) & 94.8 (1.5) & 94.8 (2.5) & 1.83 & 94.7 (3.5) & 94.7 (4.5) & 94.7 (4.5) & 4.17 \\
DQN        & 94.2 (3.0) & 94.5 (3.0) & 92.8 (5.0) & 3.67 & 93.2 (6.0) & 96.6 (1.0) & 96.2 (1.0) & 2.67 \\
A2C        & 93.6 (4.0) & 93.6 (4.0) & 93.6 (4.0) & 4    & 95.6 (1.0) & 95.6 (2.0) & 95.6 (2.0) & 1.67 \\ \midrule
    & \multicolumn{4}{c}{{\bf Test Env. Seed 14}}             & \multicolumn{4}{c}{{\bf Test Env. Seed 15}} \\
    \cmidrule(lr){2-5} \cmidrule(lr){6-9} 
Method     & Seed 1     & Seed 2     & Seed 3     & Rank & Seed 1     & Seed 2     & Seed 3     & Rank \\ \midrule
Random     & 84.4 (3.0) & 80.5 (7.0) & 86.6 (3.0) & 4.33 & 93.8 (7.0) & 95.4 (4.0) & 91.9 (7.0) & 6    \\
ETS        & 87.3 (2.0) & 87.3 (2.0) & 87.3 (2.0) & 2    & 95.3 (4.0) & 95.3 (5.0) & 95.3 (4.0) & 4.33 \\
Heur-GD & 81.2 (6.5) & 81.2 (5.5) & 81.2 (6.5) & 6.17 & 95.9 (2.5) & 95.9 (2.5) & 95.9 (2.5) & 2.5  \\
Heur-LD & 81.2 (6.5) & 81.2 (5.5) & 81.2 (6.5) & 6.17 & 96.1 (1.0) & 96.1 (1.0) & 96.1 (1.0) & 1    \\
Heur-AT & 82.4 (5.0) & 82.4 (4.0) & 82.4 (5.0) & 4.67 & 95.9 (2.5) & 95.9 (2.5) & 95.9 (2.5) & 2.5  \\
DQN        & 92.8 (1.0) & 86.8 (3.0) & 88.1 (1.0) & 1.67 & 94.8 (6.0) & 94.8 (7.0) & 94.3 (6.0) & 6.33 \\
A2C        & 83.4 (4.0) & 95.3 (1.0) & 83.4 (4.0) & 3    & 94.8 (5.0) & 94.8 (6.0) & 94.8 (5.0) & 5.33 \\ \midrule
    & \multicolumn{4}{c}{{\bf Test Env. Seed 16}}             & \multicolumn{4}{c}{{\bf Test Env. Seed 17}} \\ 
    \cmidrule(lr){2-5} \cmidrule(lr){6-9} 
Method     & Seed 1     & Seed 2     & Seed 3     & Rank & Seed 1     & Seed 2     & Seed 3     & Rank \\ \midrule
Random     & 80.5 (6.0) & 87.5 (6.0) & 83.0 (6.0) & 6    & 96.4 (1.0) & 96.1 (3.0) & 96.1 (1.0) & 1.67 \\
ETS        & 79.4 (7.0) & 79.4 (7.0) & 79.4 (7.0) & 7    & 95.7 (3.0) & 95.7 (4.0) & 95.7 (3.0) & 3.33 \\
Heur-GD & 91.2 (3.0) & 91.2 (4.0) & 91.2 (3.0) & 3.33 & 93.5 (6.0) & 93.5 (6.0) & 93.5 (6.0) & 6    \\
Heur-LD & 91.2 (3.0) & 91.2 (4.0) & 91.2 (3.0) & 3.33 & 93.5 (6.0) & 93.5 (6.0) & 93.5 (6.0) & 6    \\
Heur-AT & 91.2 (3.0) & 91.2 (4.0) & 91.2 (3.0) & 3.33 & 93.5 (6.0) & 93.5 (6.0) & 93.5 (6.0) & 6    \\
DQN        & 93.9 (1.0) & 92.9 (1.0) & 94.5 (1.0) & 1    & 95.7 (4.0) & 96.2 (2.0) & 95.7 (4.0) & 3.33 \\
A2C        & 90.3 (5.0) & 92.3 (2.0) & 90.3 (5.0) & 4    & 96.1 (2.0) & 96.5 (1.0) & 96.1 (2.0) & 1.67 \\ \midrule
    & \multicolumn{4}{c}{{\bf Test Env. Seed 18}}             & \multicolumn{4}{c}{{\bf Test Env. Seed 19}} \\
    \cmidrule(lr){2-5} \cmidrule(lr){6-9} 
Method     & Seed 1     & Seed 2     & Seed 3     & Rank & Seed 1     & Seed 2     & Seed 3     & Rank \\ \midrule
Random     & 92.7 (2.0) & 94.2 (1.0) & 93.2 (1.0) & 1.33 & 97.3 (2.0) & 95.9 (2.0) & 91.6 (2.0) & 2    \\
ETS        & 92.9 (1.0) & 92.9 (2.0) & 92.9 (2.0) & 1.67 & 97.4 (1.0) & 97.4 (1.0) & 97.4 (1.0) & 1    \\
Heur-GD & 87.8 (6.0) & 87.8 (6.0) & 87.8 (6.0) & 6    & 88.3 (5.0) & 88.3 (6.0) & 88.3 (4.0) & 5    \\
Heur-LD & 87.8 (6.0) & 87.8 (6.0) & 87.8 (6.0) & 6    & 88.3 (5.0) & 88.3 (6.0) & 88.3 (4.0) & 5    \\
Heur-AT & 87.8 (6.0) & 87.8 (6.0) & 87.8 (6.0) & 6    & 88.3 (5.0) & 88.3 (6.0) & 88.3 (4.0) & 5    \\
DQN        & 89.5 (4.0) & 89.4 (4.0) & 90.5 (4.0) & 4    & 89.4 (3.0) & 90.5 (3.0) & 87.5 (6.0) & 4    \\
A2C        & 91.6 (3.0) & 91.6 (3.0) & 91.6 (3.0) & 3    & 86.1 (7.0) & 89.9 (4.0) & 86.1 (7.0) & 6   \\ 
\bottomrule
\end{tabular}
}
\end{table}
\clearpage

\begin{table}[t]
\caption{The ACC and rank for each seed (10-19) in every test environment with {\bf Split FashionMNIST} for New Task Order experiment. Under each column named 'Seed X', we show the ACC (\%) and, in parenthesis, the rank for the corresponding method for the specific seed. We also show the average rank of each method under 'Rank' in the corresponding test environment. Note that the ACC for ETS, Heur-GD, Heur-LD, and Heur-AT are copied across the seeds, since the seed only affects the policy initialization in DQN and A2C and the random selection in Random.}
\label{tab:rankings_fashionmnist_new_task_order}
\resizebox{\textwidth}{!}{
\begin{tabular}{lcccccccc}
\toprule
    & \multicolumn{4}{c}{\textbf{Test Env. Seed 10}}    & \multicolumn{4}{c}{\textbf{Test Env. Seed 11}}    \\
    \cmidrule(lr){2-5} \cmidrule(lr){6-9} 
Method     & Seed 1     & Seed 2     & Seed 3     & Rank & Seed 1     & Seed 2     & Seed 3     & Rank \\ \midrule
Random     & 98.4 (1.0) & 90.8 (7.0) & 98.0 (1.0) & 3    & 88.9 (5.0) & 93.4 (3.0) & 89.2 (5.0) & 4.33 \\
ETS        & 96.1 (6.0) & 96.1 (5.0) & 96.1 (5.0) & 5.33 & 88.8 (6.0) & 88.8 (6.0) & 88.8 (6.0) & 6    \\
Heur-GD & 98.0 (2.5) & 98.0 (1.5) & 98.0 (2.5) & 2.17 & 93.2 (3.0) & 93.2 (4.0) & 93.2 (4.0) & 3.67 \\
Heur-LD & 98.0 (2.5) & 98.0 (1.5) & 98.0 (2.5) & 2.17 & 94.7 (1.0) & 94.7 (1.0) & 94.7 (1.0) & 1    \\
Heur-AT & 92.0 (7.0) & 92.0 (6.0) & 92.0 (7.0) & 6.67 & 93.5 (2.0) & 93.5 (2.0) & 93.5 (2.0) & 2    \\
DQN        & 96.9 (4.5) & 97.0 (4.0) & 95.4 (6.0) & 4.83 & 91.3 (4.0) & 93.2 (5.0) & 93.3 (3.0) & 4    \\
A2C        & 96.9 (4.5) & 97.1 (3.0) & 96.9 (4.0) & 3.83 & 85.3 (7.0) & 86.9 (7.0) & 85.3 (7.0) & 7    \\ \midrule
\textbf{}  & \multicolumn{4}{c}{\textbf{Test Env. Seed 12}}    & \multicolumn{4}{c}{\textbf{Test Env. Seed 13}}    \\
    \cmidrule(lr){2-5} \cmidrule(lr){6-9} 
Method     & Seed 1     & Seed 2     & Seed 3     & Rank & Seed 1     & Seed 2     & Seed 3     & Rank \\ \midrule
Random     & 97.6 (1.0) & 96.8 (1.0) & 93.4 (4.0) & 2    & 94.0 (6.0) & 88.9 (7.0) & 93.4 (6.0) & 6.33 \\
ETS        & 91.2 (6.0) & 91.2 (6.0) & 91.2 (5.0) & 5.67 & 91.2 (7.0) & 91.2 (6.0) & 91.2 (7.0) & 6.67 \\
Heur-GD & 95.0 (5.0) & 95.0 (4.0) & 95.0 (3.0) & 4    & 95.1 (5.0) & 95.1 (5.0) & 95.1 (5.0) & 5    \\
Heur-LD & 96.1 (2.0) & 96.1 (2.0) & 96.1 (1.0) & 1.67 & 96.2 (4.0) & 96.2 (4.0) & 96.2 (4.0) & 4    \\
Heur-AT & 95.8 (3.0) & 95.8 (3.0) & 95.8 (2.0) & 2.67 & 98.1 (1.0) & 98.1 (2.0) & 98.1 (1.0) & 1.33 \\
DQN        & 95.0 (4.0) & 94.8 (5.0) & 88.6 (6.0) & 5    & 96.6 (3.0) & 98.4 (1.0) & 97.4 (2.0) & 2    \\
A2C        & 87.7 (7.0) & 87.7 (7.0) & 87.7 (7.0) & 7    & 98.0 (2.0) & 98.0 (3.0) & 96.7 (3.0) & 2.67 \\ \midrule
\textbf{}  & \multicolumn{4}{c}{\textbf{Test Env. Seed 14}}    & \multicolumn{4}{c}{\textbf{Test Env. Seed 15}}    \\
    \cmidrule(lr){2-5} \cmidrule(lr){6-9} 
Method     & Seed 1     & Seed 2     & Seed 3     & Rank & Seed 1     & Seed 2     & Seed 3     & Rank \\ \midrule
Random     & 94.2 (5.0) & 96.0 (2.0) & 93.0 (5.0) & 4    & 94.3 (1.0) & 92.6 (2.0) & 93.8 (1.0) & 1.33 \\
ETS        & 90.0 (6.0) & 90.0 (6.0) & 90.0 (6.0) & 6    & 93.5 (2.0) & 93.5 (1.0) & 93.5 (2.0) & 1.67 \\
Heur-GD & 95.4 (1.0) & 95.4 (3.0) & 95.4 (2.0) & 2    & 79.3 (7.0) & 79.3 (7.0) & 79.3 (7.0) & 7    \\
Heur-LD & 95.1 (3.0) & 95.1 (4.0) & 95.1 (3.0) & 3.33 & 92.6 (3.0) & 92.6 (3.0) & 92.6 (3.0) & 3    \\
Heur-AT & 82.0 (7.0) & 82.0 (7.0) & 82.0 (7.0) & 7    & 88.5 (5.0) & 88.5 (5.0) & 88.5 (4.0) & 4.67 \\
DQN        & 95.3 (2.0) & 96.1 (1.0) & 95.5 (1.0) & 1.33 & 89.6 (4.0) & 90.1 (4.0) & 88.4 (5.0) & 4.33 \\
A2C        & 94.7 (4.0) & 94.7 (5.0) & 94.7 (4.0) & 4.33 & 81.0 (6.0) & 81.0 (6.0) & 80.5 (6.0) & 6    \\ \midrule
\textbf{}  & \multicolumn{4}{c}{\textbf{Test Env. Seed 16}}    & \multicolumn{4}{c}{\textbf{Test Env. Seed 17}}    \\
    \cmidrule(lr){2-5} \cmidrule(lr){6-9} 
Method     & Seed 1     & Seed 2     & Seed 3     & Rank & Seed 1     & Seed 2     & Seed 3     & Rank \\ \midrule
Random     & 90.0 (2.0) & 88.1 (3.0) & 92.3 (3.0) & 2.67 & 99.1 (3.0) & 98.6 (5.0) & 99.2 (4.0) & 4    \\
ETS        & 94.4 (1.0) & 94.4 (1.0) & 94.4 (1.0) & 1    & 98.1 (7.0) & 98.1 (7.0) & 98.1 (7.0) & 7    \\
Heur-GD & 73.8 (7.0) & 73.8 (7.0) & 73.8 (7.0) & 7    & 99.4 (1.0) & 99.4 (1.0) & 99.4 (1.0) & 1    \\
Heur-LD & 80.4 (6.0) & 80.4 (6.0) & 80.4 (6.0) & 6    & 99.3 (2.0) & 99.3 (2.0) & 99.3 (2.0) & 2    \\
Heur-AT & 89.2 (4.0) & 89.2 (2.0) & 89.2 (4.0) & 3.33 & 98.7 (4.0) & 98.7 (4.0) & 98.7 (6.0) & 4.67 \\
DQN        & 87.3 (5.0) & 85.7 (4.0) & 92.3 (2.0) & 3.67 & 98.7 (5.0) & 99.1 (3.0) & 99.0 (5.0) & 4.33 \\
A2C        & 89.3 (3.0) & 83.2 (5.0) & 87.7 (5.0) & 4.33 & 98.7 (6.0) & 98.5 (6.0) & 99.2 (3.0) & 5    \\ \midrule
\textbf{}  & \multicolumn{4}{c}{\textbf{Test Env. Seed 18}}    & \multicolumn{4}{c}{\textbf{Test Env. Seed 19}}    \\
    \cmidrule(lr){2-5} \cmidrule(lr){6-9} 
Method     & Seed 1     & Seed 2     & Seed 3     & Rank & Seed 1     & Seed 2     & Seed 3     & Rank \\ \midrule
Random     & 87.2 (7.0) & 87.2 (7.0) & 94.2 (4.0) & 6    & 96.2 (5.0) & 97.8 (1.0) & 97.7 (1.0) & 2.33 \\
ETS        & 93.6 (4.0) & 93.6 (4.0) & 93.6 (5.0) & 4.33 & 97.5 (1.0) & 97.5 (3.0) & 97.5 (2.0) & 2    \\
Heur-GD & 92.9 (5.0) & 92.9 (5.0) & 92.9 (6.0) & 5.33 & 95.8 (6.0) & 95.8 (5.0) & 95.8 (5.0) & 5.33 \\
Heur-LD & 92.1 (6.0) & 92.1 (6.0) & 92.1 (7.0) & 6.33 & 93.4 (7.0) & 93.4 (7.0) & 93.4 (7.0) & 7    \\
Heur-AT & 94.2 (2.0) & 94.2 (2.0) & 94.2 (3.0) & 2.33 & 96.6 (4.0) & 96.6 (4.0) & 96.6 (4.0) & 4    \\
DQN        & 95.3 (1.0) & 94.7 (1.0) & 96.0 (1.0) & 1    & 96.9 (3.0) & 95.7 (6.0) & 95.7 (6.0) & 5    \\
A2C        & 93.8 (3.0) & 93.8 (3.0) & 95.0 (2.0) & 2.67 & 97.0 (2.0) & 97.7 (2.0) & 96.9 (3.0) & 2.33 \\
\bottomrule
\end{tabular}
}
\end{table}
\clearpage

\begin{table}[t]
\caption{The ACC and rank for each seed (10-19) in every test environment with {\bf Split CIFAR-10} for New Task Order experiment. Under each column named 'Seed X', we show the ACC (\%) and, in parenthesis, the rank for the corresponding method for the specific seed. We also show the average rank of each method in the test environment under 'Rank'. Note that the ACC for ETS, Heur-GD, Heur-LD, and Heur-AT are copied across the seeds, since the seed only affects the policy initialization in DQN and A2C and the random selection in Random.}
\vspace{-2mm}
\label{tab:rankings_cifar10_new_task_order}
\resizebox{\textwidth}{!}{
\begin{tabular}{lcccccccc}
\toprule
\textbf{}  & \multicolumn{4}{c}{\textbf{Test Env. Seed 10}} & \multicolumn{4}{c}{\textbf{Test Env. Seed 11}} \\
\cmidrule(lr){2-5} \cmidrule(lr){6-9} 
Method     & Seed 1      & Seed 2      & Seed 3      & Rank & Seed 1      & Seed 2      & Seed 3      & Rank \\ \midrule
Random     & 87.3 (2.0)  & 85.2 (6.0)  & 85.7 (6.0)  & 4.67 & 75.3 (6.0)  & 71.1 (7.0)  & 72.8 (7.0)  & 6.67 \\
ETS        & 87.1 (3.0)  & 87.1 (1.0)  & 87.1 (1.0)  & 1.67 & 75.2 (7.0)  & 75.2 (6.0)  & 75.2 (6.0)  & 6.33 \\
Heur-GD & 86.3 (5.0)  & 86.3 (3.0)  & 86.3 (4.0)  & 4    & 79.5 (4.0)  & 79.5 (5.0)  & 79.5 (4.0)  & 4.33 \\
Heur-LD & 86.7 (4.0)  & 86.7 (2.0)  & 86.7 (3.0)  & 3    & 81.6 (2.0)  & 81.6 (2.0)  & 81.6 (2.0)  & 2    \\
Heur-AT & 85.9 (6.0)  & 85.9 (4.0)  & 85.9 (5.0)  & 5    & 81.1 (3.0)  & 81.1 (3.0)  & 81.1 (3.0)  & 3    \\
DQN        & 88.9 (1.0)  & 85.5 (5.0)  & 86.9 (2.0)  & 2.67 & 78.2 (5.0)  & 79.9 (4.0)  & 79.3 (5.0)  & 4.67 \\
A2C        & 84.6 (7.0)  & 84.6 (7.0)  & 84.6 (7.0)  & 7    & 82.9 (1.0)  & 82.9 (1.0)  & 82.9 (1.0)  & 1    \\ \midrule
\textbf{}  & \multicolumn{4}{c}{\textbf{Test Env. Seed 12}} & \multicolumn{4}{c}{\textbf{Test Env. Seed 13}} \\
\cmidrule(lr){2-5} \cmidrule(lr){6-9} 
Method     & Seed 1      & Seed 2      & Seed 3      & Rank & Seed 1      & Seed 2      & Seed 3      & Rank \\ \midrule
Random     & 86.2 (4.0)  & 88.9 (3.0)  & 87.6 (3.0)  & 3.33 & 77.3 (6.0)  & 81.3 (2.0)  & 80.9 (2.0)  & 3.33 \\
ETS        & 85.5 (7.0)  & 85.5 (6.0)  & 85.5 (7.0)  & 6.67 & 76.9 (7.0)  & 76.9 (7.0)  & 76.9 (7.0)  & 7    \\
Heur-GD & 89.8 (1.0)  & 89.8 (1.0)  & 89.8 (1.0)  & 1    & 78.9 (5.0)  & 78.9 (6.0)  & 78.9 (6.0)  & 5.67 \\
Heur-LD & 87.5 (3.0)  & 87.5 (4.0)  & 87.5 (5.0)  & 4    & 80.8 (3.0)  & 80.8 (4.0)  & 80.8 (3.0)  & 3.33 \\
Heur-AT & 89.6 (2.0)  & 89.6 (2.0)  & 89.6 (2.0)  & 2    & 80.2 (4.0)  & 80.2 (5.0)  & 80.2 (5.0)  & 4.67 \\
DQN        & 86.0 (6.0)  & 84.4 (7.0)  & 85.7 (6.0)  & 6.33 & 80.8 (2.0)  & 81.3 (1.0)  & 80.2 (4.0)  & 2.33 \\
A2C        & 86.1 (5.0)  & 86.1 (5.0)  & 87.5 (4.0)  & 4.67 & 81.3 (1.0)  & 81.0 (3.0)  & 81.3 (1.0)  & 1.67 \\ \midrule
\textbf{}  & \multicolumn{4}{c}{\textbf{Test Env. Seed 14}} & \multicolumn{4}{c}{\textbf{Test Env. Seed 15}} \\
\cmidrule(lr){2-5} \cmidrule(lr){6-9} 
Method     & Seed 1      & Seed 2      & Seed 3      & Rank & Seed 1      & Seed 2      & Seed 3      & Rank \\ \midrule
Random     & 81.7 (6.0)  & 84.8 (5.0)  & 83.0 (6.0)  & 5.67 & 83.8 (4.0)  & 85.0 (2.0)  & 85.8 (1.0)  & 2.33 \\
ETS        & 77.8 (7.0)  & 77.8 (7.0)  & 77.8 (7.0)  & 7    & 83.8 (5.0)  & 83.8 (6.0)  & 83.8 (5.0)  & 5.33 \\
Heur-GD & 86.9 (1.0)  & 86.9 (1.0)  & 86.9 (1.0)  & 1    & 83.5 (6.0)  & 83.5 (7.0)  & 83.5 (6.0)  & 6.33 \\
Heur-LD & 84.0 (5.0)  & 84.0 (6.0)  & 84.0 (4.0)  & 5    & 84.1 (2.5)  & 84.1 (4.5)  & 84.1 (3.5)  & 3.5  \\
Heur-AT & 86.6 (2.0)  & 86.6 (2.0)  & 86.6 (2.0)  & 2    & 84.1 (2.5)  & 84.1 (4.5)  & 84.1 (3.5)  & 3.5  \\
DQN        & 85.0 (4.0)  & 85.2 (4.0)  & 83.8 (5.0)  & 4.33 & 83.2 (7.0)  & 85.4 (1.0)  & 81.7 (7.0)  & 5    \\
A2C        & 86.2 (3.0)  & 86.2 (3.0)  & 85.5 (3.0)  & 3    & 84.6 (1.0)  & 84.6 (3.0)  & 84.6 (2.0)  & 2    \\ \midrule
\textbf{}  & \multicolumn{4}{c}{\textbf{Test Env. Seed 16}} & \multicolumn{4}{c}{\textbf{Test Env. Seed 17}} \\
\cmidrule(lr){2-5} \cmidrule(lr){6-9} 
Method     & Seed 1      & Seed 2      & Seed 3      & Rank & Seed 1      & Seed 2      & Seed 3      & Rank \\ \midrule
Random     & 85.2 (7.0)  & 89.4 (1.0)  & 87.6 (4.0)  & 4    & 68.9 (7.0)  & 72.1 (6.0)  & 74.0 (6.0)  & 6.33 \\
ETS        & 87.8 (3.0)  & 87.8 (3.0)  & 87.8 (2.0)  & 2.67 & 70.3 (5.0)  & 70.3 (7.0)  & 70.3 (7.0)  & 6.33 \\
Heur-GD & 86.8 (4.5)  & 86.8 (5.5)  & 86.8 (5.5)  & 5.17 & 74.1 (4.0)  & 74.1 (5.0)  & 74.1 (5.0)  & 4.67 \\
Heur-LD & 86.5 (6.0)  & 86.5 (7.0)  & 86.5 (7.0)  & 6.67 & 76.1 (2.0)  & 76.1 (3.0)  & 76.1 (3.0)  & 2.67 \\
Heur-AT & 86.8 (4.5)  & 86.8 (5.5)  & 86.8 (5.5)  & 5.17 & 76.4 (1.0)  & 76.4 (1.0)  & 76.4 (2.0)  & 1.33 \\
DQN        & 87.9 (2.0)  & 87.4 (4.0)  & 87.6 (3.0)  & 3    & 69.9 (6.0)  & 76.4 (2.0)  & 79.2 (1.0)  & 3    \\
A2C        & 88.0 (1.0)  & 88.0 (2.0)  & 88.0 (1.0)  & 1.33 & 75.7 (3.0)  & 75.7 (4.0)  & 75.7 (4.0)  & 3.67 \\ \midrule
\textbf{}  & \multicolumn{4}{c}{\textbf{Test Env. Seed 18}} & \multicolumn{4}{c}{\textbf{Test Env. Seed 19}} \\
\cmidrule(lr){2-5} \cmidrule(lr){6-9} 
Method     & Seed 1      & Seed 2      & Seed 3      & Rank & Seed 1      & Seed 2      & Seed 3      & Rank \\ \midrule
Random     & 86.9 (7.0)  & 87.6 (7.0)  & 86.4 (7.0)  & 7    & 71.2 (7.0)  & 77.5 (7.0)  & 73.2 (7.0)  & 7    \\
ETS        & 88.3 (6.0)  & 88.3 (6.0)  & 88.3 (5.0)  & 5.67 & 77.7 (5.0)  & 77.7 (5.0)  & 77.7 (5.0)  & 5    \\
Heur-GD & 88.6 (4.5)  & 88.6 (4.5)  & 88.6 (3.5)  & 4.17 & 77.9 (3.5)  & 77.9 (3.5)  & 77.9 (3.5)  & 3.5  \\
Heur-LD & 89.8 (1.0)  & 89.8 (2.0)  & 89.8 (1.0)  & 1.33 & 77.5 (6.0)  & 77.5 (6.0)  & 77.5 (6.0)  & 6    \\
Heur-AT & 88.6 (4.5)  & 88.6 (4.5)  & 88.6 (3.5)  & 4.17 & 77.9 (3.5)  & 77.9 (3.5)  & 77.9 (3.5)  & 3.5  \\
DQN        & 89.1 (2.0)  & 88.6 (3.0)  & 86.8 (6.0)  & 3.67 & 80.5 (2.0)  & 80.5 (1.5)  & 78.5 (2.0)  & 1.83 \\
A2C        & 88.7 (3.0)  & 89.9 (1.0)  & 89.4 (2.0)  & 2    & 80.5 (1.0)  & 80.5 (1.5)  & 80.5 (1.0)  & 1.17 \\
\bottomrule
\end{tabular}
}
\end{table}

\clearpage


\begin{table}[t]
\caption{The ACC and rank for each seed (0-9) in every test environment with {\bf Split notMNIST} for New Dataset experiment. Under each column named 'Seed X', we show the ACC (\%) and, in parenthesis, the rank for the corresponding method for the specific seed. We also show the average rank of each method in the test environment under 'Rank'. Note that the ACC for ETS, Heur-GD, Heur-LD, and Heur-AT are copied across the seeds, since the seed only affects the policy initialization in DQN and A2C and the random selection in Random.}
\vspace{-2mm}
\label{tab:rankings_notmnist_new_dataset}
\resizebox{\textwidth}{!}{
\begin{tabular}{lcccccccc}
\toprule
\textbf{} & \multicolumn{4}{c}{\textbf{Test Env. Seed 0}} & \multicolumn{4}{c}{\textbf{Test Env. Seed 1}} \\
\cmidrule(lr){2-5} \cmidrule(lr){6-9} 
Method    & Seed 1      & Seed 2      & Seed 3     & Rank & Seed 1      & Seed 2      & Seed 3     & Rank \\ \midrule
Random    & 93.8 (1.0)  & 94.9 (1.0)  & 93.2 (2.0) & 1.33 & 93.5 (2.0)  & 92.0 (4.0)  & 91.0 (7.0) & 4.33 \\
ETS       & 91.1 (5.0)  & 91.1 (7.0)  & 91.1 (6.0) & 6    & 92.5 (4.0)  & 92.5 (3.0)  & 92.5 (3.0) & 3.33 \\
Heur-GD   & 92.4 (3.0)  & 92.4 (5.0)  & 92.4 (4.0) & 4    & 91.2 (6.0)  & 91.2 (6.0)  & 91.2 (5.0) & 5.67 \\
Heur-LD   & 92.4 (3.0)  & 92.4 (5.0)  & 92.4 (4.0) & 4    & 91.2 (6.0)  & 91.2 (6.0)  & 91.2 (5.0) & 5.67 \\
Heur-AT   & 92.4 (3.0)  & 92.4 (5.0)  & 92.4 (4.0) & 4    & 91.2 (6.0)  & 91.2 (6.0)  & 91.2 (5.0) & 5.67 \\
DQN       & 90.7 (6.0)  & 94.5 (2.0)  & 85.3 (7.0) & 5    & 94.2 (1.0)  & 94.6 (1.0)  & 94.5 (1.0) & 1    \\
A2C       & 90.6 (7.0)  & 92.7 (3.0)  & 93.3 (1.0) & 3.67 & 93.5 (3.0)  & 93.9 (2.0)  & 93.9 (2.0) & 2.33 \\ \midrule
\textbf{} & \multicolumn{4}{c}{\textbf{Test Env. Seed 2}} & \multicolumn{4}{c}{\textbf{Test Env. Seed 3}} \\
\cmidrule(lr){2-5} \cmidrule(lr){6-9} 
Method    & Seed 1      & Seed 2      & Seed 3     & Rank & Seed 1      & Seed 2      & Seed 3     & Rank \\ \midrule
Random    & 82.1 (7.0)  & 90.5 (3.0)  & 88.5 (4.0) & 4.67 & 94.2 (5.0)  & 89.5 (6.0)  & 85.2 (7.0) & 6    \\
ETS       & 91.2 (3.0)  & 91.2 (2.0)  & 91.2 (3.0) & 2.67 & 89.3 (7.0)  & 89.3 (7.0)  & 89.3 (6.0) & 6.67 \\
Heur-GD   & 82.6 (5.0)  & 82.6 (6.0)  & 82.6 (6.0) & 5.67 & 95.3 (2.0)  & 95.3 (2.0)  & 95.3 (2.0) & 2    \\
Heur-LD   & 82.6 (5.0)  & 82.6 (6.0)  & 82.6 (6.0) & 5.67 & 95.3 (2.0)  & 95.3 (2.0)  & 95.3 (2.0) & 2    \\
Heur-AT   & 82.6 (5.0)  & 82.6 (6.0)  & 82.6 (6.0) & 5.67 & 95.3 (2.0)  & 95.3 (2.0)  & 95.3 (2.0) & 2    \\
DQN       & 93.9 (1.0)  & 94.2 (1.0)  & 92.0 (1.0) & 1    & 93.3 (6.0)  & 92.8 (5.0)  & 92.3 (5.0) & 5.33 \\
A2C       & 91.7 (2.0)  & 89.7 (4.0)  & 91.7 (2.0) & 2.67 & 95.0 (4.0)  & 95.0 (4.0)  & 95.0 (4.0) & 4    \\ \midrule
\textbf{} & \multicolumn{4}{c}{\textbf{Test Env. Seed 4}} & \multicolumn{4}{c}{\textbf{Test Env. Seed 5}} \\
\cmidrule(lr){2-5} \cmidrule(lr){6-9} 
Method    & Seed 1      & Seed 2      & Seed 3     & Rank & Seed 1      & Seed 2      & Seed 3     & Rank \\ \midrule
Random    & 90.2 (7.0)  & 93.2 (1.0)  & 90.1 (7.0) & 5    & 90.7 (7.0)  & 93.7 (1.0)  & 91.4 (6.0) & 4.67 \\
ETS       & 90.5 (6.0)  & 90.5 (7.0)  & 90.5 (6.0) & 6.33 & 90.7 (6.0)  & 90.7 (7.0)  & 90.7 (7.0) & 6.67 \\
Heur-GD   & 91.6 (3.5)  & 91.6 (4.5)  & 91.6 (4.5) & 4.17 & 91.8 (4.0)  & 91.8 (5.0)  & 91.8 (4.0) & 4.33 \\
Heur-LD   & 92.7 (2.0)  & 92.7 (3.0)  & 92.7 (2.0) & 2.33 & 91.8 (4.0)  & 91.8 (5.0)  & 91.8 (4.0) & 4.33 \\
Heur-AT   & 91.6 (3.5)  & 91.6 (4.5)  & 91.6 (4.5) & 4.17 & 91.8 (4.0)  & 91.8 (5.0)  & 91.8 (4.0) & 4.33 \\
DQN       & 91.3 (5.0)  & 91.2 (6.0)  & 92.0 (3.0) & 4.67 & 93.2 (1.0)  & 92.1 (3.0)  & 93.0 (1.0) & 1.67 \\
A2C       & 93.1 (1.0)  & 93.1 (2.0)  & 93.0 (1.0) & 1.33 & 92.6 (2.0)  & 93.1 (2.0)  & 92.8 (2.0) & 2    \\ \midrule
\textbf{} & \multicolumn{4}{c}{\textbf{Test Env. Seed 6}} & \multicolumn{4}{c}{\textbf{Test Env. Seed 7}} \\
\cmidrule(lr){2-5} \cmidrule(lr){6-9} 
Method    & Seed 1      & Seed 2      & Seed 3     & Rank & Seed 1      & Seed 2      & Seed 3     & Rank \\ \midrule
Random    & 91.6 (4.0)  & 87.2 (6.0)  & 91.9 (3.0) & 4.33 & 95.0 (1.0)  & 94.3 (1.0)  & 93.3 (3.0) & 1.67 \\
ETS       & 92.0 (3.0)  & 92.0 (2.0)  & 92.0 (2.0) & 2.33 & 94.0 (2.0)  & 94.0 (2.0)  & 94.0 (1.0) & 1.67 \\
Heur-GD   & 91.6 (5.5)  & 91.6 (3.5)  & 91.6 (4.5) & 4.5  & 88.8 (7.0)  & 88.8 (7.0)  & 88.8 (7.0) & 7    \\
Heur-LD   & 87.0 (7.0)  & 87.0 (7.0)  & 87.0 (7.0) & 7    & 90.6 (6.0)  & 90.6 (6.0)  & 90.6 (6.0) & 6    \\
Heur-AT   & 91.6 (5.5)  & 91.6 (3.5)  & 91.6 (4.5) & 4.5  & 93.1 (4.0)  & 93.1 (4.0)  & 93.1 (4.0) & 4    \\
DQN       & 93.1 (2.0)  & 95.2 (1.0)  & 93.8 (1.0) & 1.33 & 94.0 (3.0)  & 92.6 (5.0)  & 93.7 (2.0) & 3.33 \\
A2C       & 93.3 (1.0)  & 90.0 (5.0)  & 90.0 (6.0) & 4    & 92.6 (5.0)  & 93.7 (3.0)  & 92.6 (5.0) & 4.33 \\ \midrule
\textbf{} & \multicolumn{4}{c}{\textbf{Test Env. Seed 8}} & \multicolumn{4}{c}{\textbf{Test Env. Seed 9}} \\
\cmidrule(lr){2-5} \cmidrule(lr){6-9} 
Method    & Seed 1      & Seed 2      & Seed 3     & Rank & Seed 1      & Seed 2      & Seed 3     & Rank \\ \midrule
Random    & 92.3 (5.0)  & 92.6 (4.0)  & 92.4 (4.0) & 4.33 & 90.7 (3.0)  & 91.2 (2.0)  & 93.4 (2.0) & 2.33 \\
ETS       & 89.0 (6.0)  & 89.0 (6.0)  & 89.0 (6.0) & 6    & 94.3 (1.0)  & 94.3 (1.0)  & 94.3 (1.0) & 1    \\
Heur-GD   & 93.6 (3.5)  & 93.6 (2.5)  & 93.6 (2.5) & 2.83 & 79.1 (6.0)  & 79.1 (5.0)  & 79.1 (6.0) & 5.67 \\
Heur-LD   & 88.2 (7.0)  & 88.2 (7.0)  & 88.2 (7.0) & 7    & 79.1 (6.0)  & 79.1 (5.0)  & 79.1 (6.0) & 5.67 \\
Heur-AT   & 93.6 (3.5)  & 93.6 (2.5)  & 93.6 (2.5) & 2.83 & 79.1 (6.0)  & 79.1 (5.0)  & 79.1 (6.0) & 5.67 \\
DQN       & 94.6 (1.0)  & 91.6 (5.0)  & 92.1 (5.0) & 3.67 & 82.7 (4.0)  & 79.1 (7.0)  & 82.9 (4.0) & 5    \\
A2C       & 93.9 (2.0)  & 93.9 (1.0)  & 93.9 (1.0) & 1.33 & 90.8 (2.0)  & 90.8 (3.0)  & 90.8 (3.0) & 2.67 \\
\bottomrule
\end{tabular}
}
\end{table}
\clearpage


\begin{table}[t]
\caption{The ACC and rank for each seed (0-9) in every test environment with {\bf Split FashionMNIST} for New Dataset experiment. Under each column named 'Seed X', we show the ACC (\%) and, in parenthesis, the rank for the corresponding method for the specific seed. We also show the average rank of each method in the test environment under 'Rank'. Note that the ACC for ETS, Heur-GD, Heur-LD, and Heur-AT are copied across the seeds, since the seed only affects the policy initialization in DQN and A2C and the random selection in Random.}
\vspace{-2mm}
\label{tab:rankings_fashionmnist_new_dataset}
\resizebox{\textwidth}{!}{
\begin{tabular}{lcccccccc}
\toprule
\textbf{}  & \multicolumn{4}{c}{\textbf{Test Env. Seed 0}} & \multicolumn{4}{c}{\textbf{Test Env. Seed 1}} \\ 
    \cmidrule(lr){2-5} \cmidrule(lr){6-9} 
Method     & Seed 1      & Seed 2      & Seed 3     & Rank & Seed 1      & Seed 2      & Seed 3     & Rank \\ \midrule
Random     & 94.7 (7.0)  & 97.7 (2.0)  & 91.5 (7.0) & 5.33 & 92.8 (5.0)  & 95.1 (1.0)  & 93.7 (6.0) & 4    \\
ETS        & 97.1 (6.0)  & 97.1 (6.0)  & 97.1 (6.0) & 6    & 94.1 (2.0)  & 94.1 (3.0)  & 94.1 (3.0) & 2.67 \\
Heur-GD & 97.9 (1.0)  & 97.9 (1.0)  & 97.9 (1.0) & 1    & 95.0 (1.0)  & 95.0 (2.0)  & 95.0 (1.0) & 1.33 \\
Heur-LD & 97.6 (3.0)  & 97.6 (3.0)  & 97.6 (3.0) & 3    & 90.0 (7.0)  & 90.0 (7.0)  & 90.0 (7.0) & 7    \\
Heur-AT & 97.4 (4.0)  & 97.4 (5.0)  & 97.4 (5.0) & 4.67 & 94.1 (3.0)  & 94.1 (4.0)  & 94.1 (4.0) & 3.67 \\
DQN        & 97.7 (2.0)  & 97.4 (4.0)  & 97.6 (4.0) & 3.33 & 94.0 (4.0)  & 90.1 (6.0)  & 94.6 (2.0) & 4    \\
A2C        & 97.3 (5.0)  & 89.1 (7.0)  & 97.7 (2.0) & 4.67 & 90.9 (6.0)  & 90.9 (5.0)  & 94.0 (5.0) & 5.33 \\ \midrule
\textbf{}  & \multicolumn{4}{c}{\textbf{Test Env. Seed 2}} & \multicolumn{4}{c}{\textbf{Test Env. Seed 3}} \\
    \cmidrule(lr){2-5} \cmidrule(lr){6-9} 
Method     & Seed 1      & Seed 2      & Seed 3     & Rank & Seed 1      & Seed 2      & Seed 3     & Rank \\ \midrule
Random     & 94.6 (6.0)  & 95.1 (6.0)  & 94.8 (5.0) & 5.67 & 89.5 (6.0)  & 99.4 (1.5)  & 93.9 (4.0) & 3.83 \\
ETS        & 86.7 (7.0)  & 86.7 (7.0)  & 86.7 (7.0) & 7    & 89.4 (7.0)  & 89.4 (6.0)  & 89.4 (6.0) & 6.33 \\
Heur-GD & 97.4 (2.0)  & 97.4 (3.0)  & 97.4 (3.0) & 2.67 & 96.7 (3.0)  & 96.7 (3.0)  & 96.7 (2.0) & 2.67 \\
Heur-LD & 97.3 (3.0)  & 97.3 (4.0)  & 97.3 (4.0) & 3.67 & 90.6 (5.0)  & 90.6 (5.0)  & 90.6 (5.0) & 5    \\
Heur-AT & 97.7 (1.0)  & 97.7 (1.0)  & 97.7 (1.0) & 1    & 99.4 (1.0)  & 99.4 (1.5)  & 99.4 (1.0) & 1.17 \\
DQN        & 96.6 (4.0)  & 95.9 (5.0)  & 94.3 (6.0) & 5    & 97.3 (2.0)  & 96.6 (4.0)  & 96.0 (3.0) & 3    \\
A2C        & 96.2 (5.0)  & 97.6 (2.0)  & 97.6 (2.0) & 3    & 93.8 (4.0)  & 88.6 (7.0)  & 89.2 (7.0) & 6    \\ \midrule
\textbf{}  & \multicolumn{4}{c}{\textbf{Test Env. Seed 4}} & \multicolumn{4}{c}{\textbf{Test Env. Seed 5}} \\
    \cmidrule(lr){2-5} \cmidrule(lr){6-9} 
Method     & Seed 1      & Seed 2      & Seed 3     & Rank & Seed 1      & Seed 2      & Seed 3     & Rank \\ \midrule
Random     & 88.6 (4.0)  & 84.6 (6.0)  & 80.3 (7.0) & 5.67 & 88.7 (5.0)  & 93.9 (1.0)  & 88.0 (5.0) & 3.67 \\
ETS        & 87.1 (5.0)  & 87.1 (4.0)  & 87.1 (4.0) & 4.33 & 91.5 (1.0)  & 91.5 (2.0)  & 91.5 (1.0) & 1.33 \\
Heur-GD & 91.3 (3.0)  & 91.3 (3.0)  & 91.3 (2.0) & 2.67 & 90.3 (3.0)  & 90.3 (4.0)  & 90.3 (3.0) & 3.33 \\
Heur-LD & 86.8 (6.0)  & 86.8 (5.0)  & 86.8 (5.0) & 5.33 & 88.0 (6.0)  & 88.0 (5.0)  & 88.0 (4.0) & 5    \\
Heur-AT & 83.3 (7.0)  & 83.3 (7.0)  & 83.3 (6.0) & 6.67 & 90.5 (2.0)  & 90.5 (3.0)  & 90.5 (2.0) & 2.33 \\
DQN        & 91.3 (2.0)  & 92.2 (1.0)  & 90.5 (3.0) & 2    & 89.2 (4.0)  & 87.1 (7.0)  & 86.1 (7.0) & 6    \\
A2C        & 91.8 (1.0)  & 91.8 (2.0)  & 91.8 (1.0) & 1.33 & 87.9 (7.0)  & 87.9 (6.0)  & 87.9 (6.0) & 6.33 \\ \midrule
\textbf{}  & \multicolumn{4}{c}{\textbf{Test Env. Seed 6}} & \multicolumn{4}{c}{\textbf{Test Env. Seed 7}} \\
    \cmidrule(lr){2-5} \cmidrule(lr){6-9} 
Method     & Seed 1      & Seed 2      & Seed 3     & Rank & Seed 1      & Seed 2      & Seed 3     & Rank \\ \midrule
Random     & 95.3 (2.0)  & 94.0 (2.0)  & 96.3 (1.0) & 1.67 & 93.3 (4.0)  & 94.0 (4.0)  & 94.4 (5.0) & 4.33 \\
ETS        & 95.5 (1.0)  & 95.5 (1.0)  & 95.5 (2.0) & 1.33 & 95.3 (2.0)  & 95.3 (2.0)  & 95.3 (3.0) & 2.33 \\
Heur-GD & 90.0 (3.0)  & 90.0 (4.0)  & 90.0 (4.0) & 3.67 & 91.8 (5.0)  & 91.8 (5.0)  & 91.8 (6.0) & 5.33 \\
Heur-LD & 85.4 (4.0)  & 85.4 (5.0)  & 85.4 (5.0) & 4.67 & 95.1 (3.0)  & 95.1 (3.0)  & 95.1 (4.0) & 3.33 \\
Heur-AT & 76.0 (7.0)  & 76.0 (7.0)  & 76.0 (7.0) & 7    & 96.8 (1.0)  & 96.8 (1.0)  & 96.8 (1.0) & 1    \\
DQN        & 80.1 (6.0)  & 84.3 (6.0)  & 81.6 (6.0) & 6    & 87.4 (7.0)  & 88.8 (7.0)  & 87.4 (7.0) & 7    \\
A2C        & 81.9 (5.0)  & 91.7 (3.0)  & 91.5 (3.0) & 3.67 & 91.7 (6.0)  & 91.3 (6.0)  & 96.1 (2.0) & 4.67 \\ \midrule
\textbf{}  & \multicolumn{4}{c}{\textbf{Test Env. Seed 8}} & \multicolumn{4}{c}{\textbf{Test Env. Seed 9}} \\
    \cmidrule(lr){2-5} \cmidrule(lr){6-9} 
Method     & Seed 1      & Seed 2      & Seed 3     & Rank & Seed 1      & Seed 2      & Seed 3     & Rank \\ \midrule
Random     & 88.6 (6.0)  & 87.9 (6.0)  & 95.8 (1.0) & 4.33 & 97.2 (1.0)  & 97.0 (1.0)  & 97.7 (1.0) & 1    \\
ETS        & 94.8 (2.0)  & 94.8 (3.0)  & 94.8 (3.0) & 2.67 & 96.7 (3.0)  & 96.7 (2.0)  & 96.7 (2.0) & 2.33 \\
Heur-GD & 95.0 (1.0)  & 95.0 (1.0)  & 95.0 (2.0) & 1.33 & 95.3 (4.0)  & 95.3 (3.0)  & 95.3 (4.0) & 3.67 \\
Heur-LD & 87.9 (7.0)  & 87.9 (7.0)  & 87.9 (7.0) & 7    & 90.0 (7.0)  & 90.0 (7.0)  & 90.0 (7.0) & 7    \\
Heur-AT & 93.9 (3.0)  & 93.9 (4.0)  & 93.9 (4.0) & 3.67 & 90.3 (6.0)  & 90.3 (6.0)  & 90.3 (6.0) & 6    \\
DQN        & 91.8 (5.0)  & 95.0 (2.0)  & 91.0 (6.0) & 4.33 & 96.9 (2.0)  & 93.7 (4.0)  & 95.9 (3.0) & 3    \\
A2C        & 93.6 (4.0)  & 93.6 (5.0)  & 93.6 (5.0) & 4.67 & 90.4 (5.0)  & 92.6 (5.0)  & 90.4 (5.0) & 5 \\
\bottomrule
\end{tabular}
}
\end{table}
\clearpage


