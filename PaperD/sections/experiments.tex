
\section{Experiments}\label{paperD:sec:experiments}


We evaluate our RL-based framework using DQN~\citeD{D:mnih2013playing, D:mnih2015human} and A2C~\citeD{D:mnih2016asynchronous} for learning policies that generalize to new CL scenarios. We show that the learned policies can efficiently mitigate catastrophic forgetting in CL environments with new task orders and datasets that are unseen during training. Full details on experimental settings and additional results are in Appendix \ref{paperD:app:experimental_settings} and \ref{paperD:app:additional_experimental_results}. Code will be made publicly available upon acceptance. 


\vspace{-3mm}
\paragraph{Datasets and Network Architectures.} We conduct experiments on four CL benchmark datasets: Split MNIST~\citeD{D:lecun1998gradient, D:zenke2017continual}, FashionMNIST~\citeD{D:xiao2017fashion}, Split notMNIST~\citeD{D:bulatov2011notMNIST}, Permuted MNIST~\citeD{D:goodfellow2013empirical}, and Split CIFAR-10~\citeD{D:krizhevsky2009learning}. We randomly sample 15\% of training data for each task to use for validation, and keep the original test sets intact. 
We use multi-head output layers for all datasets and assume task labels are available at test time~\citeD{D:van2019three}. A 2-layer MLP with 256 hidden units for Split MNIST, Split FashionMNIST, and Split notMNIST. For Split CIFAR-10, we the same ConvNet architecture used in \citeD{D:adel2019continual, D:schwarz2018progress, D:vinyals2016matching}.

\vspace{-3mm}
\paragraph{Generating CL Environments.} We generate multiple CL environments with pre-set random seeds for initializing the network parameters $\vphi$ and shuffling the task order. See Appendix \ref{paperD:app:experimental_settings} for details on the pre-set seeds. 
%The pre-set random seeds are in the range $0-49$, such that we have 50 environments for each dataset. 
We shuffle the task order by permuting the class order and then split the classes into 5 pairs (tasks) with 2 classes/pair. 
%For environments with seed $0$, we keep the original task order in the dataset. 
Taking a step at task $t$ in the CL environments involves training the CL network on the $t$-th dataset with a replay memory $\gM_t$ from the discrete action space described in Appendix \ref{paperD:app:action_space}. 
To speed up the experiments with the RL algorithms, we run a breadth-first search (BFS) through the discrete action space and save the classification results for re-use during policy learning. Note that the action space has 1050 possible paths of replay schedules for the 5-task datasets, which makes the environment generation time-consuming. More specifically, generating a CL environment for one seed with Split MNIST took on around 9.5 hours on average on a NVIDIA GeForce RTW 2080Ti.   
Hence, we only generate environments where the replay memory size $M=10$ have been used, and leave analysis of different memory sizes as future work. 
%The CL environments that we used will be provided in the public code. 

\begin{figure}[t]
	\centering
	\setlength{\figwidth}{0.26\textwidth}
	\setlength{\figheight}{.14\textheight}
	\begin{subfigure}[t]{0.48\textwidth}
		\centering
		
\pgfplotsset{every axis title/.append style={at={(0.5,0.82)}}}
\pgfplotsset{every tick label/.append style={font=\tiny}}
\pgfplotsset{every major tick/.append style={major tick length=2pt}}
\pgfplotsset{every minor tick/.append style={minor tick length=1pt}}
\pgfplotsset{every axis x label/.append style={at={(0.5,-0.22)}}}
\pgfplotsset{every axis y label/.append style={at={(-0.22,0.5)}}}
\begin{tikzpicture}
\tikzstyle{every node}=[font=\scriptsize]
\definecolor{color0}{rgb}{0.12156862745098,0.466666666666667,0.705882352941177}
\definecolor{color1}{rgb}{1,0.498039215686275,0.0549019607843137}
\definecolor{color2}{rgb}{0.172549019607843,0.627450980392157,0.172549019607843}
\definecolor{color3}{rgb}{0.83921568627451,0.152941176470588,0.156862745098039}
\definecolor{color4}{rgb}{0.580392156862745,0.403921568627451,0.741176470588235}

\begin{groupplot}[group style={group size= 4 by 1, horizontal sep=1.40cm, vertical sep=1.00cm}]



\nextgroupplot[title={DQN (ACC: 95.50\%)} ,
height=\figheight,
legend cell align={left},
legend columns=-1,
legend style={
  nodes={scale=0.9},
  fill opacity=0.8,
  draw opacity=1,
  text opacity=1,
  at={(1.20,1.40)},
  anchor=south west,
  draw=white!80!black
},
minor xtick={},
minor ytick={0.55,0.65,0.75,0.85,0.95,1.05},
tick align=outside,
tick pos=left,
width=\figwidth,
x grid style={white!69.0196078431373!black},
xmajorgrids,
xlabel={Task},
xmin=0.8, xmax=5.2,
xtick style={color=black},
xtick={1,2,3,4,5},
xticklabel style={font=\tiny, inner sep=1pt},
xticklabels={1,2,3,4,5},
y grid style={white!69.0196078431373!black},
ymajorgrids,
yminorgrids,
ylabel={Accuracy},
ymin=0.491, ymax=1.01,
ytick style={color=black},
ytick={0.5,0.6,0.7,0.8,0.9,1.0},
yticklabel style={font=\tiny, inner sep=0pt},
yticklabels={50, 60, 70, 80, 90, 100}
]
\addplot [color0, mark=*, mark size=1.5, mark options={solid}]
table {%
1 0.994017958641052
2 0.912761688232422
3 0.868394792079926
4 0.750747740268707
5 0.837986052036285
};\addlegendentry{Task 1}
\addplot [ color1, mark=*, mark size=1.5, mark options={solid}]
table {%
2 0.99297297000885
3 0.918378353118896
4 0.967027008533478
5 0.969729721546173
};\addlegendentry{Task 2}
\addplot [ color2, mark=*, mark size=1.5, mark options={solid}]
table {%
3 0.999067604541779
4 0.987412571907043
5 0.981818199157715
};\addlegendentry{Task 3}
\addplot [ color3, mark=*, mark size=1.5, mark options={solid}]
table {%
4 0.996513962745667
5 0.994521915912628
};\addlegendentry{Task 4}
\addplot [color4, mark=*, mark size=1.5, mark options={solid}]
table {%
5 0.990959346294403
};\addlegendentry{Task 5}



\input{Chapter4/figures/policy_illustrations/mnist/data_seed10/dqn_seed2/bubble_rs_dataset_seed_10_dqn_seed_2}




\nextgroupplot[title={Heur-LD (ACC: 90.77\%)} ,
height=\figheight,
legend cell align={left},
legend style={
  nodes={scale=0.9},
  fill opacity=0.8,
  draw opacity=1,
  text opacity=1,
  at={(3.72,0.28)},
  anchor=south west,
  draw=white!80!black
},
minor xtick={},
minor ytick={0.55,0.65,0.75,0.85,0.95,1.05},
tick align=outside,
tick pos=left,
width=\figwidth,
x grid style={white!69.0196078431373!black},
xmajorgrids,
xlabel={Task},
xmin=0.8, xmax=5.2,
xtick style={color=black},
xtick={1,2,3,4,5},
xticklabels={1,2,3,4,5},
xticklabel style={font=\tiny, inner sep=1pt},
y grid style={white!69.0196078431373!black},
ymajorgrids,
yminorgrids,
ylabel={Accuracy},
ymin=0.491, ymax=1.01,
ytick style={color=black},
ytick={0.5,0.6,0.7,0.8,0.9,1.0},
yticklabels={50, 60, 70, 80, 90, 100},
yticklabel style={font=\tiny, inner sep=0pt}
]
\addplot [color0, mark=*, mark size=1.5, mark options={solid}]
table {%
1 0.994017946161516
2 0.777168494516451
3 0.809571286141575
4 0.567298105682951
5 0.708374875373878
};
%\addlegendentry{Task 1}
\addplot [color1, mark=*, mark size=1.5, mark options={solid}]
table {%
2 0.994054054054054
3 0.970810810810811
4 0.792972972972973
5 0.855675675675676
};
%\addlegendentry{Task 2}
\addplot [color2, mark=*, mark size=1.5, mark options={solid}]
table {%
3 0.998601398601399
4 0.990675990675991
5 0.995804195804196
};
%\addlegendentry{Task 3}
\addplot [color3, mark=*, mark size=1.5, mark options={solid}]
table {%
4 0.99601593625498
5 0.989043824701195
};
%\addlegendentry{Task 4}
\addplot [color4, mark=*, mark size=1.5, mark options={solid}]
table {%
5 0.989452536413862
};
%\addlegendentry{Task 5}
%\addlegendentry{Task 5}




\input{Chapter4/figures/policy_illustrations/mnist/data_seed10/heuristic_local_drop/bubble_plot}

\end{groupplot}

\end{tikzpicture}
		\vspace{-4mm} % hack to get captions aligned...
		\caption{Split MNIST}
		\label{fig:rewards_mnist_2envs}
	\end{subfigure}%
	\begin{subfigure}[t]{0.48\textwidth}
		\centering
		
\pgfplotsset{every axis title/.append style={at={(0.5,0.82)}}}
\pgfplotsset{every tick label/.append style={font=\tiny}}
\pgfplotsset{every major tick/.append style={major tick length=2pt}}
\pgfplotsset{every minor tick/.append style={minor tick length=1pt}}
\pgfplotsset{every axis x label/.append style={at={(0.5,-0.22)}}}
\pgfplotsset{every axis y label/.append style={at={(-0.22,0.5)}}}
\begin{tikzpicture}
\tikzstyle{every node}=[font=\scriptsize]
\definecolor{color0}{rgb}{0.12156862745098,0.466666666666667,0.705882352941177}
\definecolor{color1}{rgb}{1,0.498039215686275,0.0549019607843137}
\definecolor{color2}{rgb}{0.172549019607843,0.627450980392157,0.172549019607843}
\definecolor{color3}{rgb}{0.83921568627451,0.152941176470588,0.156862745098039}
\definecolor{color4}{rgb}{0.580392156862745,0.403921568627451,0.741176470588235}

\begin{groupplot}[group style={group size= 4 by 1, horizontal sep=1.40cm, vertical sep=1.00cm}]



\nextgroupplot[title={DQN (ACC: 95.50\%)} ,
height=\figheight,
legend cell align={left},
legend columns=-1,
legend style={
  nodes={scale=0.9},
  fill opacity=0.8,
  draw opacity=1,
  text opacity=1,
  at={(1.20,1.40)},
  anchor=south west,
  draw=white!80!black
},
minor xtick={},
minor ytick={0.55,0.65,0.75,0.85,0.95,1.05},
tick align=outside,
tick pos=left,
width=\figwidth,
x grid style={white!69.0196078431373!black},
xmajorgrids,
xlabel={Task},
xmin=0.8, xmax=5.2,
xtick style={color=black},
xtick={1,2,3,4,5},
xticklabel style={font=\tiny, inner sep=1pt},
xticklabels={1,2,3,4,5},
y grid style={white!69.0196078431373!black},
ymajorgrids,
yminorgrids,
ylabel={Accuracy},
ymin=0.491, ymax=1.01,
ytick style={color=black},
ytick={0.5,0.6,0.7,0.8,0.9,1.0},
yticklabel style={font=\tiny, inner sep=0pt},
yticklabels={50, 60, 70, 80, 90, 100}
]
\addplot [color0, mark=*, mark size=1.5, mark options={solid}]
table {%
1 0.994017958641052
2 0.912761688232422
3 0.868394792079926
4 0.750747740268707
5 0.837986052036285
};\addlegendentry{Task 1}
\addplot [ color1, mark=*, mark size=1.5, mark options={solid}]
table {%
2 0.99297297000885
3 0.918378353118896
4 0.967027008533478
5 0.969729721546173
};\addlegendentry{Task 2}
\addplot [ color2, mark=*, mark size=1.5, mark options={solid}]
table {%
3 0.999067604541779
4 0.987412571907043
5 0.981818199157715
};\addlegendentry{Task 3}
\addplot [ color3, mark=*, mark size=1.5, mark options={solid}]
table {%
4 0.996513962745667
5 0.994521915912628
};\addlegendentry{Task 4}
\addplot [color4, mark=*, mark size=1.5, mark options={solid}]
table {%
5 0.990959346294403
};\addlegendentry{Task 5}



\input{Chapter4/figures/policy_illustrations/mnist/data_seed10/dqn_seed2/bubble_rs_dataset_seed_10_dqn_seed_2}




\nextgroupplot[title={Heur-LD (ACC: 90.77\%)} ,
height=\figheight,
legend cell align={left},
legend style={
  nodes={scale=0.9},
  fill opacity=0.8,
  draw opacity=1,
  text opacity=1,
  at={(3.72,0.28)},
  anchor=south west,
  draw=white!80!black
},
minor xtick={},
minor ytick={0.55,0.65,0.75,0.85,0.95,1.05},
tick align=outside,
tick pos=left,
width=\figwidth,
x grid style={white!69.0196078431373!black},
xmajorgrids,
xlabel={Task},
xmin=0.8, xmax=5.2,
xtick style={color=black},
xtick={1,2,3,4,5},
xticklabels={1,2,3,4,5},
xticklabel style={font=\tiny, inner sep=1pt},
y grid style={white!69.0196078431373!black},
ymajorgrids,
yminorgrids,
ylabel={Accuracy},
ymin=0.491, ymax=1.01,
ytick style={color=black},
ytick={0.5,0.6,0.7,0.8,0.9,1.0},
yticklabels={50, 60, 70, 80, 90, 100},
yticklabel style={font=\tiny, inner sep=0pt}
]
\addplot [color0, mark=*, mark size=1.5, mark options={solid}]
table {%
1 0.994017946161516
2 0.777168494516451
3 0.809571286141575
4 0.567298105682951
5 0.708374875373878
};
%\addlegendentry{Task 1}
\addplot [color1, mark=*, mark size=1.5, mark options={solid}]
table {%
2 0.994054054054054
3 0.970810810810811
4 0.792972972972973
5 0.855675675675676
};
%\addlegendentry{Task 2}
\addplot [color2, mark=*, mark size=1.5, mark options={solid}]
table {%
3 0.998601398601399
4 0.990675990675991
5 0.995804195804196
};
%\addlegendentry{Task 3}
\addplot [color3, mark=*, mark size=1.5, mark options={solid}]
table {%
4 0.99601593625498
5 0.989043824701195
};
%\addlegendentry{Task 4}
\addplot [color4, mark=*, mark size=1.5, mark options={solid}]
table {%
5 0.989452536413862
};
%\addlegendentry{Task 5}
%\addlegendentry{Task 5}




\input{Chapter4/figures/policy_illustrations/mnist/data_seed10/heuristic_local_drop/bubble_plot}

\end{groupplot}

\end{tikzpicture}
		%\vspace{-2mm}
		\caption{Split CIFAR-10}
		\label{fig:rewards_cifar10_2envs}
	\end{subfigure}
	\vspace{-2mm}
	\caption{Performance progress measured in ACC (\%) for all methods in the 2 test environments with (a) {\bf Split MNIST} and (b) {\bf Split CIFAR-10} in the New Task Orders experiment. We plot the performance progress for the RL algorithms for 100 evaluation steps equidistantly distributed over the training episodes. The performance for the Random, ETS, and Heuristic scheduling baselines are plotted as straight lines.  }
	\label{fig:rewards_new_task_orders_2envs}
	\vspace{-2mm}
\end{figure}


\vspace{-3mm}
\paragraph{DQN and A2C Architectures.}
The input layer has size $T-1$ where each unit is inputting the task performances since the states are represented by the validation accuracies $s_t = [A_{t, 1}^{(val)}, ..., A_{t, t}^{(val)}, 0, ..., 0]$. The current task can therefore be determined by the number of non-zero state inputs. The output layer has 35 units representing the possible actions at $T=5$ in the discrete action space (see Appendix \ref{paperD:app:action_space}). We use action masking on the output units to prevent the network from selection invalid actions for constructing the replay memory at the current task. The DQN is a 2-layer MLP with 512 hidden units and ReLU activations. For A2C, we use separate networks for parameterizing the policy and the value function, where both networks are 2-layer MLPs with 64 hidden units of Tanh activations. 

\vspace{-3mm}
\paragraph{Baselines.} We compare our proposed method to the following scheduling baselines: 
\begin{itemize}[topsep=0pt]%,leftmargin=*, ]
	\item {\bf Random.} Random policy that randomly selects task proportions from the action space on how to structure the replay memory at every task. 
	\item {\bf Equal Task Schedule (ETS).} Policy that selects equal task proportion such that the replay memory aims to fill the memory with an equal number of samples from every seen task. 
	\item {\bf Heuristic Global Drop (Heur-GD).} Heuristic policy that replays tasks with validation accuracy below a certain threshold proportional to the best achieved validation accuracy on the task.
	\item {\bf Heuristic Local Drop (Heur-LD).} Heuristic policy that replays tasks with validation accuracy below a threshold proportional to the previous achieved validation accuracy on the task. 
	\item {\bf Heuristic Accuracy Threshold (Heur-AT).} Heuristic policy that replays tasks with validation accuracy below a fixed threshold. 
\end{itemize}
The heuristics are based on the intuition that forgotten tasks should be replayed. They fill the replay memory with $M/k$ samples per task where $k$ is the number of selected tasks. If $k=0$, then replay is skipped at the current task. 

\vspace{-3mm}
\paragraph{Evaluation Protocol.} We use the average test accuracy over all tasks after learning the final task, i.e., $\ACC = \frac{1}{T} \sum_{i=1}^{T} A_{T, i}^{test}$ where $A_{T, i}^{test}$ is the test accuracy of task $i$ after learning task $T$. 
To assess generalization capability, we use a ranking method based on the $\ACC$ between the methods in every test environment for comparison (see Appendix \ref{paperD:app:ranking_method} for more details). 


\begin{comment}
\begin{table}[t]
	\footnotesize%\small
	\centering
	\caption{Average ranking (lower is better) for experiments on generalizing policies to environments with new task orders or a new dataset. %'S' stands for 'Split'%We train the DQN on 30 training Split MNIST environments and evaluate on 10 test environments. 
		We average the results over 10 test environments. %as well as 3 seeds for DQN and A2C.
	}
	\label{tab:average_ranking_rl_experiment}
	%\resizebox{\textwidth}{!}{
		\begin{tabular}{l c c c c c}
			\toprule %\toprule
			& \multicolumn{3}{c}{ {\bf New Task Order}} & \multicolumn{2}{c}{ {\bf New Dataset}}\\
			\cmidrule(lr){2-4}  \cmidrule(lr){5-6}
			{\bf Method} & S-MNIST & S-FashionMNIST & S-CIFAR-10 & S-notMNIST & S-FashionMNIST  \\ 
			\midrule
			Random         & 4.23   & 3.60 & 5.03 & 3.87 & 3.95\\ 
			ETS            & 3.80 & 4.57 & 5.37 & 4.27 & 3.63 \\
			\midrule 
			Heur-GD        & 4.48 & 4.25 & 3.98 & 4.58 & {\bf 2.77} \\
			Heur-LD        & 4.65 & 3.65 & 3.75 & 4.97 & 5.10 \\
			Heur-AT        & 4.33 & 3.87 & 3.43 & 4.28 & 3.72 \\
			\midrule
			DQN (Ours)     & {\bf 3.23} & {\bf 3.55} & 3.68 & 3.20 & 4.37 \\
			A2C (Ours)     & 3.27 & 4.52 & {\bf 2.75} & {\bf 2.83} & 4.47 \\
			\bottomrule %\bottomrule
		\end{tabular}
		%}
	\vspace{-2mm}
\end{table}
\end{comment}


\begin{table}[t]
	\centering
	\caption{Average ranking (lower is better) across methods in the policy generalization experiments. The best and second-best ranks are colored in \textcolor{forestgreen}{green} and \textcolor{orange}{orange} respectively.}
	\vspace{-3mm}
	\resizebox{0.98\textwidth}{!}{
	\begin{tabular}{lcccccc}
\toprule	
	& \multicolumn{4}{c}{\textbf{New Task Order}}        & \multicolumn{2}{c}{\textbf{New Dataset}} \\
	\cmidrule(lr){2-5} \cmidrule(lr){6-7}
	\textbf{Method} & S-MNIST & S-FashionMNIST & S-notMNIST & S-CIFAR-10 & S-FashionMNIST        & S-notMNIST       \\
	\midrule
	Random          & 4.58    & \textcolor{orange}{3.9}            & 4.14       & 5.57       & 4.53                  & 4.62             \\
	ETS             & 4.5     & 5.24           & 5.0          & 6.2        & \textcolor{orange}{4.14}                  & 4.76             \\
	\midrule
	Heur-GD         & 5.07    & 4.91           & \textcolor{forestgreen}{3.74}       & 4.63       & \textcolor{forestgreen}{3.26}                  & 5.31             \\
	Heur-LD         & 5.27    & 4.17           & 4.38       & 4.21       & 5.68                  & 5.68             \\
	Heur-AT         & 4.92    & 4.6            & 6.28       & 3.89       & 4.21                  & 4.97             \\
	\midrule 
	DQN (Ours)      & \textcolor{orange}{4.0}       & 4.26           & 3.97       & 4.47       & 4.97                  & 4.08             \\
	A2C (Ours)      & \textcolor{forestgreen}{3.56}    & 5.23           & \textcolor{orange}{3.84}       & \textcolor{forestgreen}{3.21}       & 4.76                  & \textcolor{orange}{3.38}             \\
	SAC (Ours)      & 4.1     & \textcolor{forestgreen}{3.69}           & 4.65       & \textcolor{orange}{3.82}       & 4.45                  & \textcolor{forestgreen}{3.2} \\
	\bottomrule             
\end{tabular}
	}
	\label{tab:average_ranking_rl_experiment}
	\vspace{-3mm}
\end{table}


\subsection{Policy Generalization to New Continual Learning Scenarios}\label{paperD:sec:results_on_policy_generalization}

In this section, we show that the policies learned with our RL-based framework using DQN and A2C are capable of generalizing across CL environments with new task orders and datasets unseen during training. 

\vspace{-3mm}
\begin{wrapfigure}{r}{0.5\textwidth}
	%\centering
	\setlength{\figwidth}{0.26\textwidth}
	\setlength{\figheight}{.14\textheight}
	\vspace{-2mm}
	\begin{subfigure}[b]{0.48\textwidth}
		\centering
		
%\pgfplotsset{every axis title/.append style={at={(0.5,0.82)}}} %{at={(0.5,0.84)}}}
%\pgfplotsset{every tick label/.append style={font=\tiny}}
%\pgfplotsset{every major tick/.append style={major tick length=2pt}}
%\pgfplotsset{every minor tick/.append style={minor tick length=1pt}}
%\pgfplotsset{every axis x label/.append style={at={(0.5,-0.27)}}}
%\pgfplotsset{every axis y label/.append style={at={(-0.17,0.5)}}}
%\pgfplotsset{scaled x ticks=false} % switching off scientific notation for x ticks

\begin{tikzpicture}
\tikzstyle{every node}=[font=\scriptsize]
\definecolor{color0}{rgb}{0.12156862745098,0.466666666666667,0.705882352941177}
\definecolor{color1}{rgb}{1,0.498039215686275,0.0549019607843137}
\definecolor{color2}{rgb}{0.172549019607843,0.627450980392157,0.172549019607843}
\definecolor{color3}{rgb}{0.83921568627451,0.152941176470588,0.156862745098039}
\definecolor{color4}{rgb}{0.580392156862745,0.403921568627451,0.741176470588235}
\definecolor{color5}{rgb}{0.549019607843137,0.337254901960784,0.294117647058824}


\begin{groupplot}[group style={group size= 2 by 1, horizontal sep=1.4cm, vertical sep=1.2cm}]

% This file was created with tikzplotlib v0.9.17.
%\begin{tikzpicture}

%\definecolor{color0}{rgb}{0.12156862745098,0.466666666666667,0.705882352941177}
%\definecolor{color1}{rgb}{1,0.498039215686275,0.0549019607843137}
%\definecolor{color2}{rgb}{0.172549019607843,0.627450980392157,0.172549019607843}
%\definecolor{color3}{rgb}{0.83921568627451,0.152941176470588,0.156862745098039}
%\definecolor{color4}{rgb}{0.580392156862745,0.403921568627451,0.741176470588235}

\nextgroupplot[ %begin{axis}[
height=\figheight,
legend cell align={left},
legend columns=-1,
legend style={
  nodes={scale=0.87},
  fill opacity=0.8,
  draw opacity=1,
  text opacity=1,
  at={(-0.43,1.36)},
  anchor=south west,
  draw=white!80!black
},
minor xtick={},
minor ytick={},
tick align=outside,
tick pos=left,
title={},
width=\figwidth,
x grid style={white!69.0196078431373!black},
xlabel={Task},
xmajorgrids,
xmin=0.8, xmax=5.2,
xtick style={color=black},
xtick={1,2,3,4,5},
xticklabels={1,2,3,4,5},
y grid style={white!69.0196078431373!black},
ylabel={Accuracy (\%)},
ymajorgrids,
ymin=0.7, ymax=1.01, %ymin=0.5, ymax=1.01,
ytick style={color=black},
ytick={0.5,0.6,0.7,0.8,0.9,1,1.1},
yticklabels={50,60,70,80,90,100,110}
]
\addplot [line width=1.5pt, color0, mark=*, mark size=1.5, mark options={solid}]
table {%
1 0.973999977111816
2 0.922500014305115
3 0.909500002861023
4 0.846000015735626
5 0.882499992847443
};
\addlegendentry{T1}
\addplot [line width=1.5pt, color1, mark=*, mark size=1.5, mark options={solid}]
table {%
2 0.966000020503998
3 0.828000009059906
4 0.755500018596649
5 0.904500007629395
};
\addlegendentry{T2}
\addplot [line width=1.5pt, color2, mark=*, mark size=1.5, mark options={solid}]
table {%
3 0.986999988555908
4 0.920000016689301
5 0.886500000953674
};
\addlegendentry{T3}
\addplot [line width=1.5pt, color3, mark=*, mark size=1.5, mark options={solid}]
table {%
4 0.94650000333786
5 0.840499997138977
};
\addlegendentry{T4}
\addplot [line width=1.5pt, color4, mark=*, mark size=1.5, mark options={solid}]
table {%
5 0.97350001335144
};
\addlegendentry{T5}
%\end{axis}

%\end{tikzpicture}


\input{PaperD/figures/policy_cifar10/bubble_rs_dataset_seed_19_a2c_seed_2}

\end{groupplot}

\end{tikzpicture}

		\vspace{-4mm}
		\caption{A2C - ACC: 89.75\%}
		\label{fig:policy_cifar10_a2c}
	\end{subfigure}
	\\[1mm]
	\begin{subfigure}[b]{0.48\textwidth}
		\centering
		\input{PaperD/figures/policy_cifar10/groupplot_ets}
		\vspace{-4mm}
		\caption{ETS - ACC: 88.32\%}
		\label{fig:policy_cifar10_ets}
	\end{subfigure}
	%\vspace{-3mm}
	\caption{Task accuracies and replay schedules for A2C and ETS for a Split CIFAR-10 environment.% The replay schedules are visualized as bubble plots and results are taken from 1 seed. %  Task order in environment is [[7, 9], [0, 4], [2, 1], [6, 5], [8, 3]].
	}
	\vspace{-3mm}
	\label{fig:policy_cifar10}
\end{wrapfigure}
\paragraph{Generalization to New Task Orders.} 
We show that the learned replay scheduling policies can generalize to CL environments with previously unseen task orders. We generate training and test environments with unique task orders for three datasets, namely, Split MNIST, FashionMNIST, and CIFAR-10. Table \ref{tab:average_ranking_rl_experiment} shows the average ranking for the DQN, A2C, and the baselines when being applied in 10 test environments. 
Our learned policies obtain the best average ranking across the datasets, where the DQN performs best for Split MNIST and FashionMNIST while A2C outperforms all methods on Split CIFAR-10. Figure \ref{fig:rewards_new_task_orders_2envs}(a) and (b) shows the performance progress for two test environments with Split MNIST and Split CIFAR-10 respectively (see Figure \ref{fig:policy_rewards_mnist_paperD} and \ref{fig:policy_rewards_cifar10_paperD} in Appendix \ref{paperD:app:additional_experimental_results} for all test environments). We observe that DQN exhibits noisier progress of the achieved ACC in these test environments than A2C. 
We provide further insights in the benefits of learning the replay scheduling policy by visualizing the replay schedule and task accuracy progress from A2C in one Split CIFAR-10 test environment in Figure \ref{fig:policy_cifar10}. Additionally, we show the replay schedule and task accuracies from the ETS baseline in the same environment. The replay schedules are visualized with bubble plots showing the selected task proportion to use for composing the replay memories at each task. Each circle color corresponds to a historical task and its size represents the proportion of replay samples at the current task. The sum of points in all circles at each column is fixed at all current tasks. In Figure \ref{fig:policy_cifar10_a2c}, we observe that A2C decides to replay task 2 more than task 1 as the performance on task 2 decreases, which results in a slightly better ACC metric achieved by A2C than ETS. These results show that the learned policy can flexibly consider replaying forgotten tasks to enhance the CL performance.
%or use other advanced RL methods which may generalize better. 

\begin{figure}[t]
	\centering
	\setlength{\figwidth}{0.26\textwidth}
	\setlength{\figheight}{.14\textheight}
	\begin{subfigure}[t]{0.48\textwidth}
		\centering
		
\pgfplotsset{every axis title/.append style={at={(0.5,0.82)}}}
\pgfplotsset{every tick label/.append style={font=\tiny}}
\pgfplotsset{every major tick/.append style={major tick length=2pt}}
\pgfplotsset{every minor tick/.append style={minor tick length=1pt}}
\pgfplotsset{every axis x label/.append style={at={(0.5,-0.22)}}}
\pgfplotsset{every axis y label/.append style={at={(-0.22,0.5)}}}
\begin{tikzpicture}
\tikzstyle{every node}=[font=\scriptsize]
\definecolor{color0}{rgb}{0.12156862745098,0.466666666666667,0.705882352941177}
\definecolor{color1}{rgb}{1,0.498039215686275,0.0549019607843137}
\definecolor{color2}{rgb}{0.172549019607843,0.627450980392157,0.172549019607843}
\definecolor{color3}{rgb}{0.83921568627451,0.152941176470588,0.156862745098039}
\definecolor{color4}{rgb}{0.580392156862745,0.403921568627451,0.741176470588235}

\begin{groupplot}[group style={group size= 4 by 1, horizontal sep=1.40cm, vertical sep=1.00cm}]



\nextgroupplot[title={DQN (ACC: 95.50\%)} ,
height=\figheight,
legend cell align={left},
legend columns=-1,
legend style={
  nodes={scale=0.9},
  fill opacity=0.8,
  draw opacity=1,
  text opacity=1,
  at={(1.20,1.40)},
  anchor=south west,
  draw=white!80!black
},
minor xtick={},
minor ytick={0.55,0.65,0.75,0.85,0.95,1.05},
tick align=outside,
tick pos=left,
width=\figwidth,
x grid style={white!69.0196078431373!black},
xmajorgrids,
xlabel={Task},
xmin=0.8, xmax=5.2,
xtick style={color=black},
xtick={1,2,3,4,5},
xticklabel style={font=\tiny, inner sep=1pt},
xticklabels={1,2,3,4,5},
y grid style={white!69.0196078431373!black},
ymajorgrids,
yminorgrids,
ylabel={Accuracy},
ymin=0.491, ymax=1.01,
ytick style={color=black},
ytick={0.5,0.6,0.7,0.8,0.9,1.0},
yticklabel style={font=\tiny, inner sep=0pt},
yticklabels={50, 60, 70, 80, 90, 100}
]
\addplot [color0, mark=*, mark size=1.5, mark options={solid}]
table {%
1 0.994017958641052
2 0.912761688232422
3 0.868394792079926
4 0.750747740268707
5 0.837986052036285
};\addlegendentry{Task 1}
\addplot [ color1, mark=*, mark size=1.5, mark options={solid}]
table {%
2 0.99297297000885
3 0.918378353118896
4 0.967027008533478
5 0.969729721546173
};\addlegendentry{Task 2}
\addplot [ color2, mark=*, mark size=1.5, mark options={solid}]
table {%
3 0.999067604541779
4 0.987412571907043
5 0.981818199157715
};\addlegendentry{Task 3}
\addplot [ color3, mark=*, mark size=1.5, mark options={solid}]
table {%
4 0.996513962745667
5 0.994521915912628
};\addlegendentry{Task 4}
\addplot [color4, mark=*, mark size=1.5, mark options={solid}]
table {%
5 0.990959346294403
};\addlegendentry{Task 5}



\input{Chapter4/figures/policy_illustrations/mnist/data_seed10/dqn_seed2/bubble_rs_dataset_seed_10_dqn_seed_2}




\nextgroupplot[title={Heur-LD (ACC: 90.77\%)} ,
height=\figheight,
legend cell align={left},
legend style={
  nodes={scale=0.9},
  fill opacity=0.8,
  draw opacity=1,
  text opacity=1,
  at={(3.72,0.28)},
  anchor=south west,
  draw=white!80!black
},
minor xtick={},
minor ytick={0.55,0.65,0.75,0.85,0.95,1.05},
tick align=outside,
tick pos=left,
width=\figwidth,
x grid style={white!69.0196078431373!black},
xmajorgrids,
xlabel={Task},
xmin=0.8, xmax=5.2,
xtick style={color=black},
xtick={1,2,3,4,5},
xticklabels={1,2,3,4,5},
xticklabel style={font=\tiny, inner sep=1pt},
y grid style={white!69.0196078431373!black},
ymajorgrids,
yminorgrids,
ylabel={Accuracy},
ymin=0.491, ymax=1.01,
ytick style={color=black},
ytick={0.5,0.6,0.7,0.8,0.9,1.0},
yticklabels={50, 60, 70, 80, 90, 100},
yticklabel style={font=\tiny, inner sep=0pt}
]
\addplot [color0, mark=*, mark size=1.5, mark options={solid}]
table {%
1 0.994017946161516
2 0.777168494516451
3 0.809571286141575
4 0.567298105682951
5 0.708374875373878
};
%\addlegendentry{Task 1}
\addplot [color1, mark=*, mark size=1.5, mark options={solid}]
table {%
2 0.994054054054054
3 0.970810810810811
4 0.792972972972973
5 0.855675675675676
};
%\addlegendentry{Task 2}
\addplot [color2, mark=*, mark size=1.5, mark options={solid}]
table {%
3 0.998601398601399
4 0.990675990675991
5 0.995804195804196
};
%\addlegendentry{Task 3}
\addplot [color3, mark=*, mark size=1.5, mark options={solid}]
table {%
4 0.99601593625498
5 0.989043824701195
};
%\addlegendentry{Task 4}
\addplot [color4, mark=*, mark size=1.5, mark options={solid}]
table {%
5 0.989452536413862
};
%\addlegendentry{Task 5}
%\addlegendentry{Task 5}




\input{Chapter4/figures/policy_illustrations/mnist/data_seed10/heuristic_local_drop/bubble_plot}

\end{groupplot}

\end{tikzpicture}
		\vspace{-4mm} % hack to get captions aligned..
		%\vspace{-2mm}
		\caption{Split FashionMNIST}
		\label{fig:rewards_fashionmnist_2envs}
	\end{subfigure}%
	\begin{subfigure}[t]{0.48\textwidth}
		\centering
		



\pgfplotsset{every axis title/.append style={at={(0.5,0.85)}}}
\pgfplotsset{every tick label/.append style={font=\scriptsize}}
\pgfplotsset{every major tick/.append style={major tick length=2pt}}
\pgfplotsset{every minor tick/.append style={minor tick length=1pt}}
\pgfplotsset{every axis x label/.append style={at={(0.5,-0.22)}}}
\pgfplotsset{every axis y label/.append style={at={(-0.30,0.5)}}}
\begin{tikzpicture}
\tikzstyle{every node}=[font=\scriptsize]
\definecolor{color0}{rgb}{0.12156862745098,0.466666666666667,0.705882352941177}
\definecolor{color1}{rgb}{1,0.498039215686275,0.0549019607843137}
\definecolor{color2}{rgb}{0.172549019607843,0.627450980392157,0.172549019607843}
\definecolor{color3}{rgb}{0.83921568627451,0.152941176470588,0.156862745098039}
\definecolor{color4}{rgb}{0.580392156862745,0.403921568627451,0.741176470588235}
\definecolor{color5}{rgb}{0.549019607843137,0.337254901960784,0.294117647058824}
\definecolor{color6}{rgb}{0.890196078431372,0.466666666666667,0.76078431372549}

\begin{groupplot}[group style={group size= 2 by 1, horizontal sep=1.00cm, vertical sep=1.20cm}]


\nextgroupplot[title=Seed 1,
height=\figheight,
legend cell align={left},
legend columns=-1,
legend style={
  fill opacity=0.8,
  draw opacity=1,
  text opacity=1,
  at={(0.10,1.29)},
  anchor=south west,
  draw=white!80!black
},
minor xtick={25, 75},
minor ytick={},
tick align=outside,
tick pos=left,
%title={Labels: [[2, 9], [6, 4], [0, 3], [1, 7], [8, 5]]},
width=\figwidth,
x grid style={white!69.0196078431373!black},
xlabel={Eval. Steps},
xminorgrids,
xmajorgrids,
xmin=-3.95, xmax=104.95,
xtick style={color=black},
xtick={-25,0,50,100,125},
xticklabels={-25,0,50,100,125},
y grid style={white!69.0196078431373!black},
%ylabel={ACC},
ymajorgrids,
ymin=0.854905004700025, ymax=0.962595010399818,
ytick style={color=black},
ytick={0.84,0.86,0.88,0.9,0.92,0.94,0.96,0.98},
yticklabels={84,86,88,90,92,94,96,98}
]
\addplot [semithick, color0]
table {%
1 0.924066670735677
2 0.911366661389669
3 0.938500006993612
4 0.941500008106232
5 0.945533339182536
6 0.943866674105326
7 0.947966674963633
8 0.95060000816981
9 0.951466675599416
10 0.947033341725667
11 0.947033341725667
12 0.951000010967255
13 0.951333343982697
14 0.951000010967255
15 0.951000010967255
16 0.951000010967255
17 0.950800009568532
18 0.955333344141642
19 0.951000010967255
20 0.951666676998138
21 0.945600012938182
22 0.952300012111664
23 0.949600013097127
24 0.9550000111262
25 0.951266678174337
26 0.957700010140737
27 0.930266670385997
28 0.945333341757456
29 0.918533333142598
30 0.94900000890096
31 0.926099995772044
32 0.919266664981842
33 0.942000007629395
34 0.905199992656708
35 0.93100000222524
36 0.920233333110809
37 0.919999996821086
38 0.93100000222524
39 0.919999996821086
40 0.931233338514964
41 0.916199998060862
42 0.931466674804687
43 0.931233338514964
44 0.920233333110809
45 0.931233338514964
46 0.920233333110809
47 0.930466667811076
48 0.942466680208842
49 0.931233338514964
50 0.942700016498566
51 0.931466674804687
52 0.931466666857402
53 0.942700016498566
54 0.942466680208842
55 0.930466667811076
56 0.942700016498566
57 0.942466680208842
58 0.94270000855128
59 0.930466667811076
60 0.920233333110809
61 0.930466667811076
62 0.920233325163523
63 0.931466674804687
64 0.931466674804687
65 0.931466674804687
66 0.941700009504954
67 0.941700009504954
68 0.920233333110809
69 0.941700009504954
70 0.931466674804687
71 0.942700016498566
72 0.942700016498566
73 0.930466667811076
74 0.941700009504954
75 0.930100007851919
76 0.941700009504954
77 0.930466667811076
78 0.930466667811076
79 0.930466667811076
80 0.930100007851919
81 0.930466667811076
82 0.941700009504954
83 0.930466667811076
84 0.940700002511342
85 0.941700009504954
86 0.941700009504954
87 0.929466660817464
88 0.939699995517731
89 0.930466667811076
90 0.940700002511342
91 0.929466660817464
92 0.939699995517731
93 0.930466667811076
94 0.919233326117198
95 0.929466660817464
96 0.929466660817464
97 0.920233333110809
98 0.920233333110809
99 0.941700009504954
100 0.919233326117198
};
%\addlegendentry{a2c, gae, envs=10, lr=0.0003, act=tanh}
\addplot [semithick, color1]
table {%
1 0.900799989700317
2 0.912233340740204
3 0.91263333161672
4 0.928866664568583
5 0.908866663773855
6 0.90889999071757
7 0.927533332506816
8 0.891566665967305
9 0.899466673533122
10 0.914566667874654
11 0.937033327420553
12 0.910133334000905
13 0.914766669273376
14 0.907233341534933
15 0.934766670068105
16 0.927966666221618
17 0.921366663773855
18 0.93866666952769
19 0.940600005785624
20 0.939800004164378
21 0.89859999815623
22 0.936033336321513
23 0.93260000149409
24 0.941900006930033
25 0.944600009918213
26 0.859800004959106
27 0.941400003433227
28 0.924700001875559
29 0.941266667842865
30 0.921700004736583
31 0.888233335812887
32 0.931266665458679
33 0.904833336671193
34 0.930966671307882
35 0.895066666603088
36 0.925733335812887
37 0.894133337338765
38 0.911666671435038
39 0.931633329391479
40 0.922200004259745
41 0.9121333360672
42 0.938533333937327
43 0.909999998410543
44 0.90293333530426
45 0.9064333319664
46 0.916133332252502
47 0.922466667493184
48 0.922466667493184
49 0.922466667493184
50 0.916199998060862
51 0.933999999364217
52 0.937000000476837
53 0.893700003623962
54 0.920366668701172
55 0.940266664822896
56 0.906733334064484
57 0.932999996344248
58 0.943666660785675
59 0.923433331648509
60 0.940099994341532
61 0.929866659641266
62 0.941633331775665
63 0.94883333047231
64 0.89816666841507
65 0.925533334414164
66 0.905799996852875
67 0.896833332379659
68 0.890966665744782
69 0.927433335781097
70 0.943099999427795
71 0.941200006008148
72 0.922533329327901
73 0.94256667693456
74 0.913033334414164
75 0.942700012524923
76 0.94699999888738
77 0.938700000445048
78 0.936299999554952
79 0.914599998792013
80 0.915766664346059
81 0.908499999841054
82 0.927266673247019
83 0.927900008360545
84 0.928466665744781
85 0.920866676171621
86 0.928833325703939
87 0.928266668319702
88 0.921433333555857
89 0.933300006389618
90 0.932666671276093
91 0.928233337402344
92 0.928266668319702
93 0.928233337402344
94 0.891733336448669
95 0.913099996248881
96 0.913099996248881
97 0.927066663901011
98 0.927466662724813
99 0.929166666666667
100 0.929166666666667
};
%\addlegendentry{DQN, n train envs=30}
\addplot [semithick, color2]
table {%
1 0.938733333333333
100 0.938733333333333
};
%\addlegendentry{Random}
\addplot [semithick, color3]
table {%
1 0.941
100 0.941
};
%\addlegendentry{ETS}
\addplot [semithick, color4]
table {%
1 0.9501
100 0.9501
};
%\addlegendentry{Heur-GD}
\addplot [semithick, color5, dash pattern=on 1pt off 3pt on 3pt off 3pt]
table {%
1 0.9003
100 0.9003
};
%\addlegendentry{Heur-LD}
\addplot [semithick, color6, dashed]
table {%
1 0.9409
100 0.9409
};
%\addlegendentry{Heur-AT}



\nextgroupplot[title=Seed 19,
height=\figheight,
legend cell align={left},
legend style={
  fill opacity=0.8,
  draw opacity=1,
  text opacity=1,
  at={(0.5,0.09)},
  anchor=south,
  draw=white!80!black
},
minor xtick={25, 75},
minor ytick={},
tick align=outside,
tick pos=left,
%title={Labels: [[1, 7], [9, 6], [8, 4], [3, 0], [2, 5]]},
width=\figwidth,
x grid style={white!69.0196078431373!black},
xlabel={Eval. Steps},
xminorgrids,
xmajorgrids,
xmin=-3.95, xmax=104.95,
xtick style={color=black},
xtick={-25,0,50,100,125},
xticklabels={-25,0,50,100,125},
y grid style={white!69.0196078431373!black},
%ylabel={ACC},
ymajorgrids,
ymin=0.731755009293556, ymax=0.822944995760918,
ytick style={color=black},
ytick={0.72,0.74,0.76,0.78,0.8,0.82,0.84},
yticklabels={72,74,76,78,80,82,84}
]
\addplot [semithick, color0]
table {%
	1 0.770119993686676
	2 0.768799998760223
	3 0.773239998817444
	4 0.778700003623962
	5 0.763700001239777
	6 0.765620002746582
	7 0.773039996623993
	8 0.780080006122589
	9 0.778359997272491
	10 0.781360001564026
	11 0.790859997272491
	12 0.786640000343323
	13 0.780300004482269
	14 0.786000001430512
	15 0.78067999124527
	16 0.778220002651214
	17 0.792040002346039
	18 0.788360004425049
	19 0.793059999942779
	20 0.780739998817444
	21 0.791559996604919
	22 0.789240002632141
	23 0.790059998035431
	24 0.779840002059937
	25 0.78672000169754
	26 0.777060000896454
	27 0.796420001983643
	28 0.79194000005722
	29 0.77565999507904
	30 0.795520000457764
	31 0.794500002861023
	32 0.782959997653961
	33 0.78746000289917
	34 0.782259993553162
	35 0.779920008182526
	36 0.783540003299713
	37 0.799879999160767
	38 0.791479997634888
	39 0.796820001602173
	40 0.800940001010895
	41 0.799299998283386
	42 0.795419998168945
	43 0.800999999046326
	44 0.78731999874115
	45 0.800119996070862
	46 0.782660000324249
	47 0.800239996910095
	48 0.811479997634888
	49 0.803199999332428
	50 0.796839997768402
	51 0.793839998245239
	52 0.791319997310638
	53 0.798840003013611
	54 0.79923999786377
	55 0.789359996318817
	56 0.794979996681213
	57 0.789359996318817
	58 0.795139997005463
	59 0.796240003108978
	60 0.796199998855591
	61 0.790579998493195
	62 0.790579998493195
	63 0.79231999874115
	64 0.792379996776581
	65 0.790579998493195
	66 0.795320000648499
	67 0.787279994487762
	68 0.785639996528625
	69 0.793399999141693
	70 0.791739997863769
	71 0.785639996528625
	72 0.783559994697571
	73 0.784979999065399
	74 0.785719997882843
	75 0.792219998836517
	76 0.783559997081757
	77 0.796579995155335
	78 0.795459997653961
	79 0.786000001430511
	80 0.79173999786377
	81 0.787559993267059
	82 0.801079993247986
	83 0.794759993553162
	84 0.792419996261597
	85 0.787999997138977
	86 0.79191999912262
	87 0.79293999671936
	88 0.780859999656677
	89 0.782959997653961
	90 0.790160000324249
	91 0.786640002727509
	92 0.780199999809265
	93 0.787580006122589
	94 0.788820004463196
	95 0.774800002574921
	96 0.775460002422333
	97 0.788640003204346
	98 0.785080001354218
	99 0.795279996395111
	100 0.799079999923706
};
%\addlegendentry{DQN}
\addplot [semithick, color1]
table {%
	1 0.758300001621246
	2 0.735900008678436
	3 0.775179996490479
	4 0.8
	5 0.807999997138977
	6 0.810799999237061
	7 0.804479999542236
	8 0.810399990081787
	9 0.795520000457764
	10 0.8
	11 0.804479999542236
	12 0.797199997901916
	13 0.792000002861023
	14 0.801679997444153
	15 0.797199997901916
	16 0.81079999923706
	17 0.802800002098084
	18 0.797199997901917
	19 0.805199995040893
	20 0.8
	21 0.807999997138977
	22 0.80239999294281
	23 0.801679997444153
	24 0.81319999217987
	25 0.792000002861023
	26 0.810399990081787
	27 0.807599987983704
	28 0.806879992485046
	29 0.807599987983704
	30 0.813199992179871
	31 0.813199992179871
	32 0.810399990081787
	33 0.794399995803833
	34 0.797119994163513
	35 0.813199992179871
	36 0.807999997138977
	37 0.807599987983704
	38 0.815999994277954
	39 0.796239995956421
	40 0.799599990844727
	41 0.807599987983704
	42 0.80239999294281
	43 0.804239993095398
	44 0.807599987983704
	45 0.801279988288879
	46 0.801439990997314
	47 0.804239993095398
	48 0.801279988288879
	49 0.801439990997314
	50 0.804119989871979
	51 0.806919991970062
	52 0.810399990081787
	53 0.802639994621277
	54 0.80479998588562
	55 0.801439990997314
	56 0.801439990997314
	57 0.807599987983704
	58 0.802319989204407
	59 0.799599990844727
	60 0.804119989871979
	61 0.80479998588562
	62 0.804119989871979
	63 0.80479998588562
	64 0.799839992523193
	65 0.803439993858337
	66 0.801439990997314
	67 0.80479998588562
	68 0.80479998588562
	69 0.80479998588562
	70 0.80479998588562
	71 0.80479998588562
	72 0.80479998588562
	73 0.80479998588562
	74 0.80479998588562
	75 0.80479998588562
	76 0.801439990997314
	77 0.801639993190765
	78 0.80479998588562
	79 0.802319989204407
	80 0.80479998588562
	81 0.80479998588562
	82 0.80479998588562
	83 0.80479998588562
	84 0.80479998588562
	85 0.80479998588562
	86 0.80479998588562
	87 0.80479998588562
	88 0.802319989204407
	89 0.80479998588562
	90 0.80479998588562
	91 0.80479998588562
	92 0.80479998588562
	93 0.80479998588562
	94 0.80479998588562
	95 0.80479998588562
	96 0.80479998588562
	97 0.80479998588562
	98 0.802319989204407
	99 0.80479998588562
	100 0.80479998588562
};
%\addlegendentry{A2C}
\addplot [semithick, color2]
table {%
	1 0.79190000295639
	2 0.79190000295639
	3 0.79190000295639
	4 0.79190000295639
	5 0.79190000295639
	6 0.79190000295639
	7 0.779560005664825
	8 0.777020008563995
	9 0.789179999828339
	10 0.792739999294281
	11 0.792400002479553
	12 0.788620002269745
	13 0.792400002479553
	14 0.792400002479553
	15 0.794879999160767
	16 0.791360001564026
	17 0.79735999584198
	18 0.797980000972748
	19 0.800459997653961
	20 0.797980000972748
	21 0.79735999584198
	22 0.79735999584198
	23 0.79735999584198
	24 0.802319989204407
	25 0.799839992523193
	26 0.799839992523193
	27 0.799839992523193
	28 0.799839992523193
	29 0.800459997653961
	30 0.79735999584198
	31 0.801080002784729
	32 0.797980000972748
	33 0.794639999866486
	34 0.797119996547699
	35 0.797119996547699
	36 0.79581999540329
	37 0.796059994697571
	38 0.796059994697571
	39 0.799399995803833
	40 0.796059994697571
	41 0.796059994697571
	42 0.799399995803833
	43 0.796059994697571
	44 0.796059994697571
	45 0.796059994697571
	46 0.793579998016357
	47 0.793579998016357
	48 0.793579998016357
	49 0.793579998016357
	50 0.796059994697571
	51 0.793339998722076
	52 0.789999997615814
	53 0.79581999540329
	54 0.792380003929138
	55 0.789040002822876
	56 0.792620003223419
	57 0.789280002117157
	58 0.789280002117157
	59 0.79175999879837
	60 0.79175999879837
	61 0.79175999879837
	62 0.789280002117157
	63 0.789040002822876
	64 0.795099999904633
	65 0.7957200050354
	66 0.795099999904633
	67 0.789280002117157
	68 0.7957200050354
	69 0.792980005741119
	70 0.792980005741119
	71 0.792980005741119
	72 0.793419997692108
	73 0.795460002422333
	74 0.792980005741119
	75 0.795460002422333
	76 0.795460002422333
	77 0.792820003032684
	78 0.792980005741119
	79 0.801900005340576
	80 0.798800003528595
	81 0.792980005741119
	82 0.799420008659363
	83 0.792980005741119
	84 0.799420008659363
	85 0.792980005741119
	86 0.792980005741119
	87 0.795460002422333
	88 0.799420008659363
	89 0.792980005741119
	90 0.792980005741119
	91 0.801660006046295
	92 0.792740006446838
	93 0.792980005741119
	94 0.799420008659363
	95 0.792740006446838
	96 0.790700001716614
	97 0.801900005340576
	98 0.798200001716614
	99 0.792980005741119
	100 0.800380005836487
};
%\addlegendentry{SAC}
\addplot [semithick, color3]
table {%
	1 0.74888
	100 0.74888
};
%\addlegendentry{Random}
\addplot [semithick, color4]
table {%
	1 0.7767
	100 0.7767
};
%\addlegendentry{ETS}
\addplot [semithick, color5]
table {%
	1 0.7787
	100 0.7787
};
%\addlegendentry{Heur-GD}
\addplot [semithick, color6, dash pattern=on 1pt off 3pt on 3pt off 3pt]
table {%
	1 0.775
	100 0.775
};
%\addlegendentry{Heur-LD}
\addplot [semithick, white!49.8039215686275!black, dashed]
table {%
	1 0.7787
	100 0.7787
};
%\addlegendentry{Heur-AT}



\end{groupplot}

\end{tikzpicture}
		\caption{Split notMNIST}
		\label{fig:rewards_notmnist_2envs}
	\end{subfigure}
	\vspace{-1mm}
	\caption{Performance progress measured in ACC (\%) for all methods in the 2 test environments with (a) {\bf Split FashionMNIST} and (b) {\bf Split notMNIST} in the New Dataset experiment. We plot the performance progress for the RL algorithms for 100 evaluation steps equidistantly distributed over the training episodes. The performance for the Random, ETS, and Heuristic scheduling baselines are plotted as straight lines.  }
	\label{fig:rewards_new_dataset_2envs}
	\vspace{-2mm}
\end{figure}

\vspace{-3mm}
\paragraph{Generalization to New Datasets.}
We show that the learned replay scheduling policy is capable of generalizing to CL environments with new datasets unseen in the training environments. We perform two sets of experiments:
\begin{enumerate}[topsep=1pt,noitemsep]
	\item Train with environments generated with Split MNIST and FashionMNIST, and test on environments generated with Split notMNIST.
	\item Train with environments generated with Split MNIST and notMNIST, and test on environments generated with Split FashionMNIST.
\end{enumerate}
%, 1) train with environments generated with Split MNIST and FashionMNIST and test on environments generated with Split notMNIST, and 2) train with environments generated with Split MNIST and notMNIST and test on environments generated with Split FashionMNIST.
%We perform experiments where the policies are applied in test environments containing either Split notMNIST or FashionMNIST datasets. To foster generalization, we enhance the diversity of the training environments by generating the environments with two types of datasets. For instance, when Split notMNIST is the target dataset, we train on environments with Split MNIST and FashionMNIST. Similarly, we use Split notMNIST for training when the test environments contain Split FashionMNIST. 
Table \ref{tab:average_ranking_rl_experiment} shows the average ranking for DQN, A2C, and the baselines when generalization to test environments with new datasets. We observe that both A2C and DQN successfully generalize to Split notMNIST environments outperforming all baselines. 
However, the learned policies have difficulties to generalize to Split FashionMNIST environments, which could be due to 
high variations in the dynamics between training and test environments.
To gain insights, we show the performance progress for two test environments with Split notMNIST and Split FashionMNIST datasets in Figure \ref{fig:rewards_new_dataset_2envs}(a) and (b) respectively (see Figure \ref{fig:policy_rewards_notmnist_new_dataset_paperD} and \ref{fig:policy_rewards_fashionmnist_new_dataset_paperD} in Appendix \ref{paperD:app:additional_experimental_results} for all test environments). 
In Figure \ref{fig:rewards_new_dataset_2envs}(b), we observe that both A2C and DQN has difficulties in converging to efficient replay scheduling policies, where A2C even collapses to a suboptimal replay schedule in second test environment (Seed 9).  
A possible reason could be that the policy encounters state transition dynamics in the test environments which has not been experienced during training, which makes generalization difficult for both DQN and A2C.
%The policy may exhibit state transition dynamics which has not been experienced during training, which makes generalization difficult for both DQN and A2C.
Potentially, the performance could be improved by generating more training environments for the agent to exhibit more variations in the CL scenarios or by using other advanced RL methods which may generalize better~\citeD{D:igl2019generalization}.





